

%% AP Physics MC Questions Archive
%%----------------------------------------


%% Oscillating Pendulum
%%----------------------------------------
\element{ap}{
\begin{question}{oscillating-pendulum-q01}
    If the length of a simple pendulum is doubled but the mass remains constant,
        its period is multiplied by a factor of:
    \begin{multicols}{3}
    \begin{choices}
        \wrongchoice{$\dfrac{1}{2}$}
        \wrongchoice{$\dfrac{1}{\sqrt{2}}$}
        \wrongchoice{$1$}
      \correctchoice{$\sqrt{2}$}
        \wrongchoice{$2$}
    \end{choices}
    \end{multicols}
\end{question}
}

\element{ap}{
\begin{question}{oscillating-pendulum-q02}
    A mass oscillating on a spring has a period of $T$.
    What is the ratio of the maximum velocity of the mass to the maximum displacement?
    \begin{multicols}{3}
    \begin{choices}
        \wrongchoice{$T$}
        \wrongchoice{$\dfrac{T}{2\pi}$}
        \wrongchoice{$1$}
      \correctchoice{$\dfrac{2\pi}{T}$}
        \wrongchoice{$\dfrac{1}{T}$}
    \end{choices}
    \end{multicols}
\end{question}
}

\element{ap}{
\begin{question}{oscillating-pendulum-q03}
    A simple pendulum has a period of oscillation of approximately \SI{2.0}{\second}.
    When the length of the pendulum is doubled,
        the period of oscillation is most nearly:
    \begin{multicols}{3}
    \begin{choices}
        \wrongchoice{\SI{0.5}{\second}}
        \wrongchoice{\SI{1.0}{\second}}
        \wrongchoice{\SI{1.4}{\second}}
        \wrongchoice{\SI{2.0}{\second}}
      \correctchoice{\SI{2.8}{\second}}
    \end{choices}
    \end{multicols}
\end{question}
}

\element{ap}{
\begin{question}{oscillating-pendulum-q04}
    The length of a simple pendulum with a period of \SI{6}{\second} on earth is most nearly
    \begin{multicols}{3}
    \begin{choices}
        \wrongchoice{\SI{4.5}{\meter}}
      \correctchoice{\SI{8.9}{\meter}}
        \wrongchoice{\SI{18}{\meter}}
        \wrongchoice{\SI{36}{\meter}}
        \wrongchoice{\SI{90}{\meter}}
    \end{choices}
    \end{multicols}
\end{question}
}

\element{ap}{
\begin{question}{oscillating-pendulum-q05}
    A simple pendulum has a period of \SI{3.0}{\second} on earth.
    On the moon, where the acceleration due to gravity is approximately one-sixth of its value on earth,
        its period would be most nearly:
    \begin{multicols}{3}
    \begin{choices}
        \wrongchoice{\SI{0.50}{\second}}
        \wrongchoice{\SI{1.2}{\second}}
        \wrongchoice{\SI{3.0}{\second}}
      \correctchoice{\SI{7.3}{\second}}
        \wrongchoice{\SI{18}{\second}}
    \end{choices}
    \end{multicols}
\end{question}
}

\element{ap}{
\begin{question}{oscillating-pendulum-q06}
    A simple pendulum of mass $M$ swings with a period of \SI{20}{\second} on Earth.
    This pendulum is brought to a planet with an acceleration due to gravity 4 times that of the earth.
    What is the period of the pendulum on this planet?
    \begin{multicols}{3}
    \begin{choices}
        \wrongchoice{\SI{20}{\second}}
      \correctchoice{\SI{10}{\second}}
        \wrongchoice{\SI{15}{\second}}
        \wrongchoice{\SI{200}{\second}}
        \wrongchoice{\SI{2}{\second}}
    \end{choices}
    \end{multicols}
\end{question}
}

\element{ap}{
\begin{question}{oscillating-pendulum-q07}
    A pendulum swings in simple harmonic motion with period $T$ and angular displacement $\theta$.
    If the same pendulum is instead raised to an initial displacement of $3\theta$,
        the new period will be:
    \begin{multicols}{3}
    \begin{choices}
        \wrongchoice{$\dfrac{T}{9}$}
        \wrongchoice{$\dfrac{T}{3}$}
      \correctchoice{$T$}
        \wrongchoice{$3T$}
        \wrongchoice{$9T$}
    \end{choices}
    \end{multicols}
\end{question}
}

\element{ap}{
\begin{question}{oscillating-pendulum-q08}
    A simple pendulum in simple harmonic motion on the Earth's surface has a period of \SI{1}{\second}.
    If the same pendulum is undergoing simple harmonic motion on the Moon,
        with an acceleration due to gravity \num{1/6} that of the Earth,
        under which of the following modifications would the pendulum have the same period?
    \begin{choices}
        \wrongchoice{Increase the length of the pendulum by a factor of $6$.}
      \correctchoice{Decrease the length of the pendulum by a factor of $6$.}
        \wrongchoice{Increase the length of the pendulum by a factor of $\sqrt{6}$.}
        \wrongchoice{Decrease the length of the pendulum by a factor of $\sqrt{6}$. }
        \wrongchoice{No change would be necessary.}
    \end{choices}
\end{question}
}

\element{ap}{
\begin{question}{oscillating-pendulum-q09}
    A simple harmonic oscillator has a frequency of \SI{2}{\hertz} and an amplitude of \SI{0.04}{\meter}.
    What is the period of the oscillations?
    \begin{multicols}{3}
    \begin{choices}
      \correctchoice{\SI{0.5}{\second}}
        \wrongchoice{\SI{0.2}{\second}}
        \wrongchoice{\SI{2}{\second}}
        \wrongchoice{\SI{5}{\second}}
        \wrongchoice{\SI{10}{\second}}
    \end{choices}
    \end{multicols}
\end{question}
}

\element{ap}{
\begin{question}{oscillating-pendulum-q10}
    A simple harmonic oscillator has a period of \SI{5}{\second} and an amplitude of \SI{1.2}{\meter}.
    What is the frequency of the oscillations?
    \begin{multicols}{3}
    \begin{choices}
        \wrongchoice{\SI{0.1}{\hertz}}
      \correctchoice{\SI{0.2}{\hertz}}
        \wrongchoice{\SI{0.5}{\hertz}}
        \wrongchoice{\SI{1}{\hertz}}
        \wrongchoice{\SI{5}{\hertz}}
    \end{choices}
    \end{multicols}
\end{question}
}

\element{ap}{
\begin{question}{oscillating-pendulum-q11}
    %% Base your answers to questions 11 through 13 on the following situation.
    A simple pendulum undergoes harmonic motion as it oscillates through small angles.
    The maximum angular displacement of the pendulum is $\theta_{\text{max}}$.
    The displacement of the pendulum is $\theta$.
    %% start question
    At which values of $\theta$ is the speed of the pendulum maximized?
    \begin{choices}
        \wrongchoice{$\theta=\dfrac{\theta_{\text{max}}}{4}$ and $\theta=-\dfrac{\theta_{\text{max}}}{4}$}
        \wrongchoice{$\theta=\dfrac{\theta_{\text{max}}}{2}$ and $\theta=-\dfrac{\theta_{\text{max}}}{2}$}
        \wrongchoice{$\theta=\theta_{\text{max}}$ and $ = –  \text{max} .$}
        \wrongchoice{$\theta=\theta_{\text{max}}$, $\theta=-\theta_{\text{max}}$, and $\theta=\text{zero}0$.}
      \correctchoice{$\theta=\text{zero}$ only.}
    \end{choices}
\end{question}
}

\element{ap}{
\begin{question}{oscillating-pendulum-q12}
    %% Base your answers to questions 11 through 13 on the following situation.
    A simple pendulum undergoes harmonic motion as it oscillates through small angles.
    The maximum angular displacement of the pendulum is $\theta_{\text{max}}$.
    The displacement of the pendulum is $\theta$.
    %% start question
    At which values of $\theta$ is the potential energy of the pendulum maximized?
    \begin{choices}
        \wrongchoice{$\theta=\dfrac{\theta_{\text{max}}}{4}$ and $\theta=\dfrac{\theta_{\text{max}}}{4}$}
        \wrongchoice{$\theta=\dfrac{\theta_{\text{max}}}{2}$ and $\theta=\dfrac{\theta_{\text{max}}}{2}$}
      \correctchoice{$\theta=\theta_{\text{max}}$ and $\theta=-\theta_{\text{max}}$}
        \wrongchoice{$\theta=\theta_{\text{max}}$, $\theta=-\theta_{\text{max}}$ and $\theta=\text{zero}$}
        \wrongchoice{$\theta=\text{zero}$ only.}
    \end{choices}
\end{question}
}

\element{ap}{
\begin{question}{oscillating-pendulum-q13}
    %% Base your answers to questions 11 through 13 on the following situation.
    A simple pendulum undergoes harmonic motion as it oscillates through small angles.
    The maximum angular displacement of the pendulum is $\theta_{\text{max}}$.
    The displacement of the pendulum is $\theta$.
    %% start question
    At which values of $\theta$ is the restoring force of the pendulum maximized?
    \begin{choices}
        \wrongchoice{$\theta=\dfrac{\theta_{\text{max}}}{4}$ and $\theta=\dfrac{\theta_{\text{max}}}{4}$}
        \wrongchoice{$\theta=\dfrac{\theta_{\text{max}}}{2}$ and $\theta=\dfrac{\theta_{\text{max}}}{2}$}
      \correctchoice{$\theta=\theta_{\text{max}}$ and $\theta=-\theta_{\text{max}}$}
        \wrongchoice{$\theta=\theta_{\text{max}}$, $\theta=-\theta_{\text{max}}$ and $\theta=\text{zero}$}
        \wrongchoice{$\theta=\text{zero}$ only.}
    \end{choices}
\end{question}
}

\element{ap}{
\begin{questionmult}{oscillating-pendulum-q14}
    A pendulum is released from a height $h$.
    Halfway through its period,
        it strikes an object of mass $m$ which connects with the pendulum bob.
    Which of the follow describe the pendulum-object system after to the collision.
    \begin{choices}
        \wrongchoice{The kinetic energy of the system increases.}
      \correctchoice{The period of the pendulum remains constant.}
      \correctchoice{The maximum height the pendulum reaches decreases.}
        %\wrongchoice{I only}
        %\wrongchoice{II only}
        %\wrongchoice{III only}
        %\correctchoice{II and III only}
        %\wrongchoice{I, II, and III only}
    \end{choices}
\end{questionmult}
}

\element{ap}{
\begin{question}{oscillating-pendulum-q15}
    A pendulum's period is initially $t$.
    Changes are made to the system so that the new period is $2t$.
    What may have been changed?
    \begin{choices}
        \wrongchoice{The length of the pendulum was decreased by a factor of 2.}
        \wrongchoice{The length of the pendulum was increased by a factor of 2.}
        \wrongchoice{The pendulum was moved to a planet with 2 times the mass of Earth.}
      \correctchoice{The length of the pendulum was increased by a factor of 4.}
        \wrongchoice{The pendulum was moved to a planet with 4 times the mass of earth.}
    \end{choices}
\end{question}
}

\element{ap}{
\begin{question}{oscillating-pendulum-q16}
    A pendulum swinging with a maximum amplitude of $\dfrac{\pi}{6}$ has a period of $T$.
    What must happen for the period to remain the same if the amplitude of motion is doubled?
    \begin{choices}
        \wrongchoice{The length must be increased by a factor of 2.}
        \wrongchoice{The length must be increased by a factor of 4.}
        \wrongchoice{The acceleration due to gravity must be increased by a factor of 2.}
        \wrongchoice{The acceleration due to gravity must be decreased by a factor of 2.}
      \correctchoice{The length and acceleration due to gravity must remain the same.}
    \end{choices}
\end{question}
}

\element{ap}{
\begin{question}{oscillating-pendulum-q17}
    A pendulum swinging with a maximum amplitude of \num{\pi/6} has a period of $T$.
    If the maximum amplitude is increased to \num{\pi/3},
        what is the new period of the pendulum?
    \begin{multicols}{3}
    \begin{choices}
        \wrongchoice{$\dfrac{T}{4}$}
        \wrongchoice{$\dfrac{T}{2}$}
      \correctchoice{$T$}
        \wrongchoice{$2T$}
        \wrongchoice{$4T$}
    \end{choices}
    \end{multicols}
\end{question}
}

\element{ap}{
\begin{question}{oscillating-pendulum-q18}
    Which of the following does \emph{not} exhibit periodic motion?
    \begin{choices}
        \wrongchoice{a simple pendulum}
        \wrongchoice{a spring-mass system}
        \wrongchoice{an electron moving perpendicularly to a constant, external, magnetic field}
        \wrongchoice{a standing wave}
      \correctchoice{a projectile}
    \end{choices}
\end{question}
}

\element{ap}{
\begin{questionmult}{oscillating-pendulum-q19}
    Which of the following can be related to the period of a simple pendulum?
    \begin{choices}
        \wrongchoice{The rotational inertia of the bob}
        \wrongchoice{The angular momentum of the bob at its lowest point}
        \wrongchoice{The maximum angular displacement of the bob}
        %\wrongchoice{I only}
        %\wrongchoice{III only}
        %\wrongchoice{I and III only}
        %\wrongchoice{I, II and III}
        %\correctchoice{None of the above}
    \end{choices}
\end{questionmult}
}

\element{ap}{
\begin{question}{oscillating-pendulum-q20}
    An ideal pendulum hangs stationary at its equilibrium position.
    Which of the following is true of the string supporting the mass?
    \begin{choices}
        \wrongchoice{It is doing work.}
        \wrongchoice{It is exerting a torque on the mass.}
      \correctchoice{The tension in the string is constant.}
        \wrongchoice{There is no tension in the string.}
        \wrongchoice{The force provided by the string is greater than the weight of the mass.}
    \end{choices}
\end{question}
}

\element{ap}{
\begin{question}{oscillating-pendulum-q21}
    Which of the following is true for a simple pendulum?
    \begin{choices}
        \wrongchoice{The kinetic and potential energies are equal at all times.}
        \wrongchoice{The kinetic and potential energies are both constant.}
        \wrongchoice{The maximum kinetic energy is equal to the minimum kinetic energy.}
      \correctchoice{The total energy is constant.}
        \wrongchoice{The total energy is equal to the sum of the maximum kinetic and potential energies.}
    \end{choices}
\end{question}
}

\element{ap}{
\begin{questionmult}{oscillating-pendulum-q22}
    The energy contained in a pendulum depends on its:
    \begin{choices}
        \wrongchoice{period}
      \correctchoice{amplitude}
      \correctchoice{mass}
        %\wrongchoice{II only}
        %\wrongchoice{II only}
        %\wrongchoice{I and III only}
        %\correctchoice{II and III only}
        %\wrongchoice{I, II, and III}
    \end{choices}
\end{questionmult}
}

\element{ap}{
\begin{question}{oscillating-pendulum-q23}
    Which one of the following statements is true concerning an object executing simple harmonic motion?
    \begin{choices}
        \wrongchoice{Its velocity is never zero.}
        \wrongchoice{Its acceleration is never zero.}
        \wrongchoice{Its velocity and acceleration are simultaneously zero.}
      \correctchoice{Its velocity is zero when its acceleration is a maximum.}
        \wrongchoice{Its maximum acceleration is equal to its maximum velocity.}
    \end{choices}
\end{question}
}

\element{ap}{
\begin{question}{oscillating-pendulum-q24}
    A \SI{5}{\kilo\gram} ball hangs from a \SI{10}{\meter} string.
    The ball is swung horizontally outward \ang{90} from its equilibrium position.
    Assuming the system behaves as a simple pendulum,
        find the maximum speed of the ball during its swing.
    \begin{multicols}{2}
    \begin{choices}
        \wrongchoice{\SI{50}{\meter\per\second}}
      \correctchoice{\SI{14}{\meter\per\second}}
        \wrongchoice{\SI{10}{\meter\per\second}}
        \wrongchoice{\SI{5}{\meter\per\second}}
        \wrongchoice{\SI{2}{\meter\per\second}}
    \end{choices}
    \end{multicols}
\end{question}
}

\element{ap}{
\begin{question}{oscillating-pendulum-q25}
    A pendulum of length $L$ swings in simple harmonic motion with period $T$ and angular displacement $\theta$ about the equilibrium point $\theta=0$.
    What is the maximum velocity of the pendulum?
    \begin{multicols}{2}
    \begin{choices}
        \wrongchoice{$\sqrt{gL\left(\sin\theta-\cos\theta\right)}$}
      \correctchoice{$\sqrt{2gL\left(1-\sin\theta\right)}$}
        \wrongchoice{$\sqrt{2gL\left(1-\cos\theta\right)}$}
        \wrongchoice{$\sqrt{2gL\sin\theta}$}
        \wrongchoice{$\sqrt{2gL\cos\theta}$}
    \end{choices}
    \end{multicols}
\end{question}
}


\endinput


