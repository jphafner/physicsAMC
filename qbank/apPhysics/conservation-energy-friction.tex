

%% AP Style MC Question Archive
%%----------------------------------------


%% Conservation of Energy with Friction
%%----------------------------------------
\element{ap}{
\begin{question}{energy-friction-q01}
    A \SI{3}{\kilo\gram} object slides \SI{90}{\meter} down a frictionless inclined plane dropping \SI{45}{\meter}.
    It then slides along a horizontal surface with a coefficient of kinetic friction of \num{1.0} until it stops.
    How far from the base of the inclined plane does it stop?
    \begin{multicols}{3}
    \begin{choices}
        \wrongchoice{\SI{400}{\meter}}
        \wrongchoice{\SI{450}{\meter}}
        \wrongchoice{\SI{500}{\meter}}
        \wrongchoice{\SI{525}{\meter}}
      \correctchoice{\SI{4 500}{\meter}}
    \end{choices}
    \end{multicols}
\end{question}
}

\element{ap}{
\begin{question}{energy-friction-q02}
    A horizontal force $F$ is used to push a \SI{3.0}{\kilo\gram} block,
        initially at rest, across a floor,
        with a constant acceleration of \SI{2.0}{\meter\per\second\squared}.
    If the frictional force between the block and the floor is \SI{4}{\newton},
        how much work is done by force $F$ on the block to move it \SI{15}{\meter}?
    \begin{multicols}{3}
    \begin{choices}
        \wrongchoice{\SI{60}{\joule}}
        \wrongchoice{\SI{90}{\joule}}
        \wrongchoice{\SI{120}{\joule}}
      \correctchoice{\SI{150}{\joule}}
        \wrongchoice{\SI{210}{\joule}}
    \end{choices}
    \end{multicols}
\end{question}
}

\element{ap}{
\begin{question}{energy-friction-q02}
    A \SI{4.0}{\newton} force is used to accelerate a \SI{2.0}{\kilo\gram} block from rest to a velocity of \SI{6.0}{\meter\per\second}.
    If the force was applied over a distance of \SI{15}{\meter},
     the coefficient of kinetic friction between the block and the surface is most nearly
    \begin{multicols}{3}
    \begin{choices}
        \wrongchoice{\num{0.03}}
        \wrongchoice{\num{0.04}}
        \wrongchoice{\num{0.06}}
      \correctchoice{\num{0.08}}
        \wrongchoice{\num{0.12}}
    \end{choices}
    \end{multicols}
\end{question}
}

%\newcommand{\apEnergyFrictionQFour}{
%\begin{tikzpicture}
%    %% NOTE: TODO: draw tikz
%\end{tikzpicture}
%}
%
%\element{ap}{
%\begin{question}{energy-friction-q04}
%    %Base your answers to questions 4 and 5 on the following information.
%    A roller coaster car starts at Point A and travels along a frictionless slope traveling to Point E which is at a slightly lower height.
%    \begin{center}
%        \apEnergyFrictionQFour
%    \end{center}
%    %% Start question
%    If it can be assumed that the car has the same speed at points $A$ and $E$,
%        which of the following can be a valid explanation about the roller coaster car?
%    \begin{choices}
%        \wrongchoice{The Gravitational Potential Energy is lower at Point $E$ than at Point $A$.}
%        \wrongchoice{The Gravitational Potential Energy is higher at Point $E$ than at Point $A$.}
%      \correctchoice{Friction exists between the wheels of the roller coaster car and the track.}
%        \wrongchoice{The roller coaster car loses Kinetic Energy between $A$ and $C$.}
%        \wrongchoice{The roller coaster car gains Kinetic Energy between $A$ and $C$.}
%    \end{choices}
%\end{question}
%}
%
%\element{ap}{
%\begin{question}{energy-friction-q05}
%    %Base your answers to questions 4 and 5 on the following information.
%    A roller coaster car starts at Point $A$ and travels along a frictionless slope traveling to Point $E$ which is at a slightly lower height.
%    \begin{center}
%        \apEnergyFrictionQFour
%    \end{center}
%    %% Start question
%    If it can be assumed that the car has the same speed at points $A$ and $E$,
%         which of the following statements is true?
%    \begin{choices}
%        \wrongchoice{The net work done in this system is zero.}
%      \correctchoice{The net work done in this system is positive.}
%        \wrongchoice{The net work done in this system is negative.}
%        \wrongchoice{The net work done in this system is zero for the roller coaster, but positive for the slope.}
%        \wrongchoice{The net work done in this system is positive for the roller coaster, but zero for the slope.}
%    \end{choices}
%\end{question}
%}

\element{ap}{
\begin{question}{energy-friction-q06}
    An object is dropped off a cliff of height $h$ and is subjected to an average force of air resistance of $F$.
    If the object has a mass of $m$,
        the kinetic energy it gains during the fall will be equal to
    \begin{multicols}{2}
    \begin{choices}
        \wrongchoice{$mgh$}
        \wrongchoice{$mgh + F$}
        \wrongchoice{$mgh - F$}
        \wrongchoice{$mgh + Fh$}
      \correctchoice{$mgh - Fh$}
    \end{choices}
    \end{multicols}
\end{question}
}

\element{ap}{
\begin{question}{energy-friction-q07}
    How much energy is required to push a \SI{50}{\kilo\gram} mass \SI{15}{\meter} across a surface with a coefficient of friction of \num{0.3} at a constant velocity?
    \begin{multicols}{2}
    \begin{choices}
        \wrongchoice{\SI{43.3}{\joule}}
        \wrongchoice{\SI{650}{\joule}}
      \correctchoice{\SI{2250}{\joule}}
        \wrongchoice{\SI{6350}{\joule}}
        \wrongchoice{\SI{9750}{\joule}}
    \end{choices}
    \end{multicols}
\end{question}
}


\endinput


