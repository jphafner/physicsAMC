

%% AP Physics MC Questions Archive
%%----------------------------------------


%% One Dimensional Motion with Calculus
%%----------------------------------------
\element{ap}{
\begin{question}{1d-motion-calculus-q01}
    The position of a vehicle moving on a straight track along the $x$-axis is given by the equation $x(t) = 3t^3-2t^2+t$,
        where $x$ is in meters and $t$ is in seconds.
    What is its velocity at time $t=\SI{3}{\second}$?
    \begin{multicols}{3}
    \begin{choices}
        \wrongchoice{\SI{47}{\meter\per\second}}
        \wrongchoice{\SI{50}{\meter\per\second}}
        \wrongchoice{\SI{60}{\meter\per\second}}
        \wrongchoice{\SI{66}{\meter\per\second}}
      \correctchoice{\SI{70}{\meter\per\second}}
    \end{choices}
    \end{multicols}
\end{question}
}

\element{ap}{
\begin{question}{1d-motion-calculus-q02}
    The velocity of a car traveling on a straight track along the $y$-axis is given by the equation $v(t) = -12t^2+6t+2$,
        where $v$ is in meters per second and $t$ is in seconds.
    The vehicle's initial position is $y=\SI{-1}{\meter}$.
    At which one of the following times is the car at the origin?
    \begin{multicols}{3}
    \begin{choices}
        \wrongchoice{zero}
      \correctchoice{\SI{1}{\second}}
        \wrongchoice{\SI{2}{\second}}
        \wrongchoice{\SI{3}{\second}}
        \wrongchoice{\SI{4}{\second}}
    \end{choices}
    \end{multicols}
\end{question}
}

\element{ap}{
\begin{question}{1d-motion-calculus-q03}
    The position of a vehicle moving on a straight track along the $x$-axis is given by the equation $x(t) = t^2+3t+5$,
        where $x$ is in meters and $t$ is in seconds.
    What is its acceleration at time $t=\SI{5}{\second}$?
    \begin{multicols}{3}
    \begin{choices}
        \wrongchoice{zero}
      \correctchoice{\SI{2}{\meter\per\second\squared}}
        \wrongchoice{\SI{5}{\meter\per\second\squared}}
        \wrongchoice{\SI{10}{\meter\per\second\squared}}
        \wrongchoice{\SI{13}{\meter\per\second\squared}}
    \end{choices}
    \end{multicols}
\end{question}
}

\element{ap}{
\begin{question}{1d-motion-calculus-q04}
    The equation $v(t) = 3t^2-4t+2$,
        where $v$ is in meters per second and $t$ is in seconds;
        gives the velocity of a vehicle moving along a straight track.
    The vehicle's initial position is \SI{3}{\meter}.
    What is the vehicle's position at $t=\SI{4}{\second}$?
    \begin{multicols}{3}
    \begin{choices}
        \wrongchoice{\SI{40}{\meter}}
        \wrongchoice{\SI{41}{\meter}}
        \wrongchoice{\SI{42}{\meter}}
      \correctchoice{\SI{43}{\meter}}
        \wrongchoice{\SI{44}{\meter}}
    \end{choices}
    \end{multicols}
\end{question}
}

\element{ap}{
\begin{question}{1d-motion-calculus-q05}
    The acceleration of a car traveling on a straight track along the $y$-axis is given by the equation $a=\SI{5}{\meter\per\second\squared}$,
        where $a$ is in meters per second squared.
    If at $t=\SI{0}{\second}$ the car's velocity is \SI{3}{\meter\per\second},
        what is its velocity at $t=\SI{2}{\second}$?
    \begin{multicols}{3}
    \begin{choices}
        \wrongchoice{\SI{3}{\meter\per\second}}
        \wrongchoice{\SI{5}{\meter\per\second}}
        \wrongchoice{\SI{10}{\meter\per\second}}
      \correctchoice{\SI{13}{\meter\per\second}}
        \wrongchoice{\SI{15}{\meter\per\second}}
    \end{choices}
    \end{multicols}
\end{question}
}

\element{ap}{
\begin{question}{1d-motion-calculus-q06}
    The acceleration of a car traveling on a straight track along the $x$-axis is given by the equation $a(t)=2t+1$,
        where $a$ is in meters per second squared and $t$ is in seconds.
    If $x(0)=0$ and $v(0)=0$,
        what is the car's displacement at $t=3$?
    \begin{multicols}{3}
    \begin{choices}
        \wrongchoice{\SI{1}{\meter}}
        \wrongchoice{\SI{7}{\meter}}
        \wrongchoice{\SI{9}{\meter}}
        \wrongchoice{\SI{12}{\meter}}
      \correctchoice{\SI{13.5}{\meter}}
    \end{choices}
    \end{multicols}
\end{question}
}

\element{ap}{
\begin{question}{1d-motion-calculus-q07}
    The velocity of a car traveling along the $y$-axis is given by the equation $v(t)=2t^2-8t+9$,
        where $v$ is in meters per second and $t$ is in seconds.
    At what time is the car's instantaneous acceleration equal to zero?
    \begin{multicols}{3}
    \begin{choices}
        \wrongchoice{zero}
        \wrongchoice{\SI{1}{\second}}
      \correctchoice{\SI{2}{\second}}
        \wrongchoice{\SI{3}{\second}}
        \wrongchoice{\SI{4}{\second}}
    \end{choices}
    \end{multicols}
\end{question}
}

\element{ap}{
\begin{question}{1d-motion-calculus-q08}
    The equation of the position of an object moving along the $x$-axis is given by $x(t)=1.5t^3-4.5t^2+0.5t$,
        where $x$ is in meters and $t$ is in seconds.
    What is the object's displacement when its instantaneous acceleration is equal to zero?
    \begin{multicols}{3}
    \begin{choices}
        \wrongchoice{zero}
      \correctchoice{\SI{-2.5}{\meter}}
        \wrongchoice{\SI{3}{\meter}}
        \wrongchoice{\SI{-12}{\meter}}
        \wrongchoice{\SI{12}{\meter}}
    \end{choices}
    \end{multicols}
\end{question}
}

\element{ap}{
\begin{question}{1d-motion-calculus-q09}
    An object moves along the $x$-axis with a velocity $v(t) = 3t^2-2t-3$,
        where $v$ is in meters per second and $t$ is in seconds.
    What is the total distance traveled during the time interval $\SI{2}{\second}<t<\SI{5}{\second}$?
    \begin{multicols}{3}
    \begin{choices}
        \wrongchoice{\SI{80}{\meter}}
        \wrongchoice{\SI{83}{\meter}}
        \wrongchoice{\SI{85}{\meter}}
      \correctchoice{\SI{87}{\meter}}
        \wrongchoice{\SI{90}{\meter}}
    \end{choices}
    \end{multicols}
\end{question}
}

\element{ap}{
\begin{question}{1d-motion-calculus-q10}
    The velocity of a particle moving along the $x$-axis is given by the equation $v(t) = 5+3t^2$,
        where $v$ is in meters per second and $t$ is in seconds.
    What is the average velocity during the interval $t=\SI{0}{\second}$ to $t=\SI{3}{\second}$?
    \begin{multicols}{3}
    \begin{choices}
        \wrongchoice{\SI{12}{\meter\per\second}}
      \correctchoice{\SI{14}{\meter\per\second}}
        \wrongchoice{\SI{28}{\meter\per\second}}
        \wrongchoice{\SI{31}{\meter\per\second}}
        \wrongchoice{\SI{42}{\meter\per\second}}
    \end{choices}
    \end{multicols}
\end{question}
}

\element{ap}{
\begin{question}{1d-motion-calculus-q11}
    The velocity of a particle moving along the $x$-axis is given by the equation $v(t)=1+5t+2t^2$,
        where $v$ is in meter per second and $t$ is in seconds.
    What is the average acceleration during the interval $t=\SI{0}{\second}$ to $t=\SI{2}{\second}$?
    \begin{multicols}{3}
    \begin{choices}
      \correctchoice{\SI{9}{\meter\per\second\squared}}
        \wrongchoice{\SI{10}{\meter\per\second\squared}}
        \wrongchoice{\SI{13}{\meter\per\second\squared}}
        \wrongchoice{\SI{18}{\meter\per\second\squared}}
        \wrongchoice{\SI{19}{\meter\per\second\squared}}
    \end{choices}
    \end{multicols}
\end{question}
}

\element{ap}{
\begin{question}{1d-motion-calculus-q12}
    The position of a particle moving along the $x$-axis is given by the equation $x(t)=1+2t^2+3t^3$,
        where $x$ is in meters and $t$ is in seconds.
    What is the average acceleration during the interval $t=\SI{0}{\second}$ to $t=\SI{1}{\second}$?
    \begin{multicols}{3}
    \begin{choices}
        \wrongchoice{\SI{6}{\meter\per\second\squared}}
        \wrongchoice{\SI{9}{\meter\per\second\squared}}
        \wrongchoice{\SI{13}{\meter\per\second\squared}}
      \correctchoice{\SI{18}{\meter\per\second\squared}}
        \wrongchoice{\SI{22}{\meter\per\second\squared}}
    \end{choices}
    \end{multicols}
\end{question}
}

\element{ap}{
\begin{question}{1d-motion-calculus-q13}
    The position of a particle moving along the $x$-axis is given by the equation $x(t)=2+6t^2$,
        where $x$ is in meters and $t$ is in seconds.
    What is the average velocity during the interval $t=\SI{0}{\second}$ to $t=\SI{0.5}{\second}$?
    \begin{multicols}{3}
    \begin{choices}
        \wrongchoice{\SI{2}{\meter\per\second}}
      \correctchoice{\SI{3}{\meter\per\second}}
        \wrongchoice{\SI{4}{\meter\per\second}}
        \wrongchoice{\SI{5}{\meter\per\second}}
        \wrongchoice{\SI{6}{\meter\per\second}}
    \end{choices}
    \end{multicols}
\end{question}
}

\element{ap}{
\begin{question}{1d-motion-calculus-q14}
    An object's motion is given by the equation $x(t)=2+4t^3$.
    What is the equation for the object's velocity?
    \begin{multicols}{2}
    \begin{choices}
        \wrongchoice{$v(t) = 2t + 12t^2$}
        \wrongchoice{$v(t) = 4t^2$}
        \wrongchoice{$v(t) = 2 t^{-1} + 4t^2$}
        \wrongchoice{$v(t) = 2t + t^4$}
      \correctchoice{$v(t) = 12t^2$}
    \end{choices}
    \end{multicols}
\end{question}
}

\element{ap}{
\begin{question}{1d-motion-calculus-q15}
    An object's motion is given by the equation $x(t) = 4t+4t^3$.
    What is the equation for the object's acceleration?
    \begin{multicols}{2}
    \begin{choices}
      \correctchoice{$a(t) = 24t$}
        \wrongchoice{$a(t) = 24t + 4$}
        \wrongchoice{$a(t) = t^4 + 2t^2$}
        \wrongchoice{$a(t) = 4t^2 + 4$}
        \wrongchoice{$a(t) = 12t^2 + 4$}
    \end{choices}
    \end{multicols}
\end{question}
}

\element{ap}{
\begin{question}{1d-motion-calculus-q16}
    A truck moving along a straight road at \SI{30}{\meter\per\second} applies its breaks such that its velocity is given by the equation $v(t)=30-2t$,
        where $v$ is in meter per second and $t$ is in seconds.
    What is the truck's acceleration at $t=\SI{1}{\second}$?
    \begin{multicols}{2}
    \begin{choices}
        \wrongchoice{\SI{-30}{\meter\per\second\squared}}
      \correctchoice{\SI{-2}{\meter\per\second\squared}}
        \wrongchoice{\SI{2}{\meter\per\second\squared}}
        \wrongchoice{\SI{15}{\meter\per\second\squared}}
        \wrongchoice{\SI{30}{\meter\per\second\squared}}
    \end{choices}
    \end{multicols}
\end{question}
}

\element{ap}{
\begin{question}{1d-motion-calculus-q17}
    The equation for the position $x$ of a particle whose acceleration is given by the equation $a(t) = 6t-3$ and starts at rest from the origin is:
    \begin{multicols}{2}
    \begin{choices}
        \wrongchoice{$x(t) = 3t^2 - 3t$}
        \wrongchoice{$x(t) = 6t^2 - 3t$}
      \correctchoice{$x(t) = t^3 - \dfrac{3t^2}{2}$}
        \wrongchoice{$x(t) = 6t^3 - 3t$}
        \wrongchoice{$x(t) = 3t^3 - t$}
    \end{choices}
    \end{multicols}
\end{question}
}

\element{ap}{
\begin{question}{1d-motion-calculus-q18}
    What is the change in velocity in the interval $t=\SI{0}{\second}$ to $t=\SI{2}{\second}$ of an object whose acceleration is given by $a(t) = \cos\left(2t\right)$?
    \begin{multicols}{3}
    \begin{choices}
      \correctchoice{\SI{0}{\meter\per\second}}
        \wrongchoice{\SI{0.5}{\meter\per\second}}
        \wrongchoice{\SI{1}{\meter\per\second}}
        \wrongchoice{\SI{-0.5}{\meter\per\second}}
        \wrongchoice{\SI{-1}{\meter\per\second}}
    \end{choices}
    \end{multicols}
\end{question}
}

\element{ap}{
\begin{question}{1d-motion-calculus-q19}
    The graph of the acceleration of an object is shown below.
    \begin{center}
    \begin{tikzpicture}
        \begin{axis}[
            axis y line=left,
            axis x line=bottom,
            axis line style={->},
            xlabel={time},
            xtick=\empty,
            ylabel={acceleration},
            ytick=\empty,
            xmin=0,xmax=11,
            ymin=0,ymax=11,
            width=0.8\columnwidth,
            height=0.5\columnwidth,
        ]
        \addplot[line width=1pt,domain=0:10]{x};
        \end{axis}
    \end{tikzpicture}
    \end{center}
    The highest order term in the equation for the position of this object has what sign and what exponent of $t$?
    \begin{multicols}{3}
    \begin{choices}
        \wrongchoice{$+1$}
        \wrongchoice{$-1$}
        \wrongchoice{$+2$}
        \wrongchoice{$-2$}
      \correctchoice{$+3$}
    \end{choices}
    \end{multicols}
\end{question}
}


\endinput


