

%% AP Physics MC Questions Archive
%%----------------------------------------


%% B-force on a wire
%%----------------------------------------
\element{ap}{
\begin{question}{b-force-wire-q01}
    The magnetic field inside a solenoid of length $l$ is $B$.
    A second solenoid has twice as many turns as the first one and is the same length.
    Both solenoids have the same current passing through them.
    What is the magnetic field inside the second solenoid?
    \begin{multicols}{3}
    \begin{choices}
        \wrongchoice{$\dfrac{B}{4}$}
        \wrongchoice{$\dfrac{B}{2}$}
        \wrongchoice{$B$}
      \correctchoice{$2B$}
        \wrongchoice{$4B$}
    \end{choices}
    \end{multicols}
\end{question}
}



1.
%% NOTE:
There is a clockwise current in a circular loop of wire in an
external magnetic field directed out of the page as shown in
the diagram above. The effect of the magnetic field is to
make the loop

A) expand.
B) contract.
C) rotate in the plane of the page.
D) rotate perpendicularly to the page.
E) move into the page.

2. Two long parallel wires, fixed a distance d apart in space,
each carry a current I. The force of attraction between them
is F. Which other arrangement of currents in long parallel
wires would produce the same force F?
A) a current of 3I and a distance of 3d
B) a current of 3I and a distance of 9d
C) a current of 3I and a distance of 6d
D) a current of 6I and a distance of 3d
E) a current of 9I and a distance of 3d


3. Two particles, with equal charge and equal masses and
velocities v 1 and v 2 travel in circular paths in a magnetic
field with radii R 1 and R 2 respectively. Which of the
following must be true?
A) The radii must be equal but the velocities might not be.
B) The velocities must be equal but the radii might not be.
C) Both the radii and the velocities must be equal.
D) R 1 v 1 = R 2 v 2
E) R 1 v 2 = R 2 v 1


4. Two long, straight, parallel wires 0.24 m apart are carrying
the same current I in the same direction. The force per unit
length felt by one wire from the other is 2 N/m. Find the
value of the current I.

A) 1.55 × 10 –4 A
B) 2.34 × 10 –4 A
C) 1.55 × 10 3 A
D) 2.34 × 10 3 A
E) 4.55 × 10 3 A

5. Two long, straight, parallel wires are placed a distance d
apart. A current of I runs through each, in opposite
directions. The force per unit length on each wire is
A) repulsive, magnitude (μ 0 /2p)I 2 /d
B) repulsive, magnitude (μ 0 /2p)I/d
C) repulsive, magnitude (μ 0 /2p)I 2 /d
D) attractive, magnitude (μ 0 /2p)I/d
E) attractive, magnitude (μ 0 /2p)I 2 /d 2

6. Two long, straight, parallel wires are a distance d apart.
Wire A carries a current of I, Wire B carries a current 2I.
The ratio of the force on Wire A to the force on Wire B is
A) 1:4
B) 1:2
C) 1:1
D) 2:1
E) 4:1


7. A wire in the plane of the page carries a current I directed
toward the bottom of the page. If the wire is located in a
uniform magnetic field B directed out of the page, the force
on the wire resulting from the magnetic field is
A) directed to the left
B) directed to the right
C) directed into the page
D) directed out of the page
E) zero

8. A wire in the plane of the page carries a current I directed
toward the bottom of the page. If the wire is located in a
uniform magnetic field B directed toward the top of the
page, the force on the wire resulting from the magnetic field
is
A) mg – IBl
B) mg + IBl
C) mg – IB/2
D) g + lB
E) (BI – l/2) + mg

Base your answers to questions 9 and 10 on the diagram below.
%% NOTE: tikz

9. In the picture above, a segment of length l of a current
carrying wire is suspended by a string in a uniform
magnetic field going out of the page. What is the tension T
on the string?

A) directed to the left
B) directed to the right
C) directed into the page
D) directed out of the page
E) zero

10. If the direction of the magnetic field is reversed, a new
tension T is established. What is the change in
tension, with regards to the original T, after the field
reversal?
A) + 2IBl
B) – 2IBl
C) mg + 2IBl
D) mg + IBl
E) mg – 2IBl






11.
%% NOTE:
A long straight wire of length 20 m with a mass per unit
length of 0.25 kg/m is lying across the ground
perpendicular to a uniform magnetic field of 4.5 T out of
the page as shown in the picture above. How much, and in
which direction, must current flow to reduce the normal
force on the wire to 0 N?
A) 0.11 A from left to right
B) 0.11 A from right to left
C) 0.55 A from right to left
D) 1.1 A from right to left
E) 1.1 A from left to right


12. The force acting on long current carrying wire in a
magnetic field is affected by all of the following EXCEPT

A) the length of strength of the magnetic field.
B) angle between the wire and the direction of the magnetic field.  the new wire.
C) the voltage across the wire.
D) the current in the wire.
E) the direction of current flow.


13. If two current carrying wires exert a force of 50 N on each
other, what force will they feel if the distance between them
is halved?
A) 12.5 N
B) 25 N
C) 50 N
D) 100 N
E) 200 N


14. If two current carrying wires exert a force of 10 N on each
other, what force will they feel if the distance between them
is doubled?
A) 2.5 N
B) 5 N
C) 10 N
D) 20 N
E) 40 N


15. Base your answer to the following question on Base your
answer to the following questions on the diagram below.
%% NOTE:
If wire is placed through point P parallel to the first wire,
which of the following best describes the force on the new
wire due to the original one?

A) It is non-zero.
B) It depends on the sign and magnitude of the charge on
C) It depends on the magnitude and direction of the current in the new wire.
D) It is parallel to the magnetic field.
E) It is directed in the same plane as the wires.



16. Two long parallel wires are fixed at a distance d apart and
each carry a current of I. The force of attraction between
them is F. If the distance between the wires is doubled and
the current in each of the wires is doubled, what is the new
force of attraction between the wires?
A) F/4
B) F/2
C) F
D) 2F
E) 4F


17. Two long parallel wires carry unequal currents in opposite
directions. One of the currents is much greater than the
other. Compared to the force felt by the wire with the
smaller current the force felt by the wire with the greater
current is

A) smaller and in the same direction
B) smaller and in the opposite direction
C) equal and in the same direction
D) equal and in the opposite direction
E) greater and in the same direction


18. Two long parallel wires each carry a current I and are a
distance 2d apart. If a third wire carrying no current is
placed between the two wires at a distance d from each of
the original wires, what is the force felt by the new wire?
A) 0
B) ƒ 0 I 2 /2pd
C) ƒ 0 I 2 /pd
D) 2ƒ 0 I 2 /pd
E) 4ƒ 0 I 2 /pd


19. What is the magnitude of the force per unit length on a long
wire with charge density Ø from a wire that carries a current
of 3 A and is a distance of 3 m away?
A) ƒ 0 Ø/2?
B) ƒ 0 Ø/6?
C) 3ƒ 0 Ø/2?
D) ƒ 0 Ø/3
E) 0




\endinput


