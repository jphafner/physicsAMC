

%% AP Physics MC Questions Archive
%%----------------------------------------


%% Third Law
%%----------------------------------------
\element{ap}{
\begin{question}{third-law-q01}
    A car drives along a curved track.
    The frictional force exerted by the track on the car is:
    \begin{choices}
        \wrongchoice{greater than the frictional force exerted by the car on the track}
        \wrongchoice{directed radially outward}
      \correctchoice{opposite in direction to the frictional force exerted by the car on the track}
        \wrongchoice{zero if the car's speed is constant}
        \wrongchoice{dependent on the radius of the track}
    \end{choices}
\end{question}
}

\element{ap}{
\begin{question}{third-law-q02}
    A force of magnitude $F$ pushes two blocks on a frictionless surface.
    \begin{center}
    \begin{tikzpicture}
        %% floor
        \node[anchor=north,fill,pattern=north east lines,minimum width=6cm, minimum height=0.05cm] at (0,0) {};
        \draw (-3,0) -- (3,0);
        %% blocks
        \node[draw,minimum height=2cm,minimum width=1cm,anchor=south] (A) at (-1,0) {\SI{2}{\kilo\gram}};
        \node[draw,minimum height=1cm,minimum width=2cm,anchor=south] (B) at (+0.5,0) {\SI{4}{\kilo\gram}};
        %% force
        \draw[very thick,<-] (A.west) -- ++(180:1cm) node[pos=0.5,anchor=south] {$F$};
    \end{tikzpicture}
    \end{center}
    What is the force the \SI{4}{\kilo\gram} block exerts on the \SI{2}{\kilo\gram} block?
    \begin{multicols}{3}
    \begin{choices}
        \wrongchoice{$4F$}
        \wrongchoice{$2F$}
        \wrongchoice{$\dfrac{4F}{3}$}
      \correctchoice{$\dfrac{2F}{3}$}
        \wrongchoice{$\dfrac{F}{3}$}
    \end{choices}
    \end{multicols}
\end{question}
}

\element{ap}{
\begin{question}{third-law-q03}
    There are two forces acting upon an object at rest on a horizontal floor:
        the pull of gravity and the normal force from the floor.
    These two forces:
    \begin{choices}
        \wrongchoice{have different magnitudes and the same direction}
        \wrongchoice{have different magnitudes and opposite directions}
        \wrongchoice{have the same magnitude and the same direction}
      \correctchoice{have the same magnitude and opposite directions}
        \wrongchoice{have the same magnitude and are perpendicular}
    \end{choices}
\end{question}
}

\element{ap}{
\begin{question}{third-law-q04}
    Two skaters, one of mass \SI{75}{\kilo\gram},
        the other of mass \SI{50}{\kilo\gram},
        stand next to each other on ice (negligible friction).
    If the heavier skater pushes the lighter with a force $F$,
        the ratio of the force felt by the lighter to the force felt by the heavier is:
    \begin{multicols}{3}
    \begin{choices}
        \wrongchoice{$1:3$}
        \wrongchoice{$2:3$}
      \correctchoice{$1:1$}
        \wrongchoice{$3:2$}
        \wrongchoice{$3:1$}
    \end{choices}
    \end{multicols}
\end{question}
}

\element{ap}{
\begin{question}{third-law-q05}
    Two skaters, one of mass \SI{100}{\kilo\gram},
        the other of mass \SI{50}{\kilo\gram} are on a frozen pond (negligible friction).
    If the heavier person pushed the lighter one with a force $F$,
        the ratio of the magnitude of the acceleration of the lighter skater to that of the heavier is:
    \begin{multicols}{3}
    \begin{choices}
        \wrongchoice{$1:4$}
        \wrongchoice{$1:2$}
        \wrongchoice{$1:1$}
      \correctchoice{$2:1$}
        \wrongchoice{$4:1$}
    \end{choices}
    \end{multicols}
\end{question}
}

\element{ap}{
\begin{question}{third-law-q06}
    A rocket engine acquires motion by ejecting hot gases in the opposite direction.
    This is an example of the law of:
    \begin{choices}
        \wrongchoice{conservation of heat}
        \wrongchoice{conservation of energy}
      \correctchoice{conservation of linear momentum}
        \wrongchoice{conservation of mass}
        \wrongchoice{conservation of angular momentum}
    \end{choices}
\end{question}
}

\element{ap}{
\begin{question}{third-law-q07}
    %% Base your answers to questions 7 and 8 on the information below.
    A cannon of mass \SI{55}{\kilo\gram} fires a projectile,
        exerting a force of \SI{150}{\newton} on the object.
    The cannon rests on a surface with a coefficient of static friction of \num{0.4}.
    %% Start Questions
    What happens to the cannon after it fires?
    \begin{choices}
        \wrongchoice{The cannon moves back \SI{15}{\meter} before coming to rest.}
        \wrongchoice{The cannon moves back \SI{30}{\meter} before coming to rest.}
        \wrongchoice{The cannon moves back \SI{0.15}{\meter} before coming to rest}
        \wrongchoice{The cannon moves back \SI{0.3}{\meter} before coming to rest.}
      \correctchoice{The cannon does not move.}
    \end{choices}
\end{question}
}

\element{ap}{
\begin{question}{third-law-q08}
    %% Base your answers to questions 7 and 8 on the information below.
    A cannon of mass \SI{55}{\kilo\gram} fires a projectile,
        exerting a force of \SI{150}{\newton} on the object.
    The cannon rests on a surface with a coefficient of static friction of \num{0.4}.
    %% Start Question
    What is the minimum amount of force the cannon must exert on the object for it to overcome the force of friction?
    \begin{multicols}{3}
    \begin{choices}
        \wrongchoice{\SI{150}{\newton}}
      \correctchoice{\SI{220}{\newton}}
        \wrongchoice{\SI{280}{\newton}}
        \wrongchoice{\SI{350}{\newton}}
        \wrongchoice{\SI{475}{\newton}}
    \end{choices}
    \end{multicols}
\end{question}
}

\element{ap}{
\begin{question}{third-law-q09}
    %% Base your answers to questions 9 and 10 on the information below.
    A child of mass \SI{60}{\kilo\gram} is standing on a frictionless surface.
    The child throws a ball of mass \SI{5}{\kilo\gram} with a force of \SI{30}{\newton}.
    %% Start question
    What is the magnitude of the child's acceleration over the surface?
    \begin{multicols}{3}
    \begin{choices}
        \wrongchoice{\SI{0.25}{\meter\per\second\squared}}
      \correctchoice{\SI{0.5}{\meter\per\second\squared}}
        \wrongchoice{\SI{0.75}{\meter\per\second\squared}}
        \wrongchoice{\SI{1}{\meter\per\second\squared}}
        \wrongchoice{\SI{1.25}{\meter\per\second\squared}}
    \end{choices}
    \end{multicols}
\end{question}
}

\element{ap}{
\begin{question}{third-law-q10}
    %% Base your answers to questions 9 and 10 on the information below.
    A child of mass \SI{60}{\kilo\gram} is standing on a frictionless surface.
    The child throws a ball of mass \SI{5}{\kilo\gram} with a force of \SI{30}{\newton}.
    %% Start question
    What is ratio of the force felt by the child to the force felt by the ball?
    \begin{multicols}{3}
    \begin{choices}
        \wrongchoice{$12:1$}
        \wrongchoice{$6:1$}
      \correctchoice{$1:1$}
        \wrongchoice{$1:6$}
        \wrongchoice{$1:12$}
    \end{choices}
    \end{multicols}
\end{question}
}

\element{ap}{
\begin{question}{third-law-q11}
    According to Newton's Third Law,
    \begin{choices}
        \wrongchoice{for every action there is a weaker, and opposite, reaction}
        \wrongchoice{for every action there is an equal, but delayed, reaction}
        \wrongchoice{for every action there is an equal, and similar, reaction}
      \correctchoice{for every action there is an equal, but opposite, reaction}
        \wrongchoice{for every action there is a weaker, but opposite, reaction}
    \end{choices}
\end{question}
}

\element{ap}{
\begin{question}{third-law-q12}
    Block $A$ of mass $2M$ rests upon Block $B$ of mass $M$.
    What is the ratio of the normal force exerted by Block $B$ on Block $A$ to the force of gravity exerted on the Block $A$--Block $B$ system?
    \begin{multicols}{3}
    \begin{choices}
        \wrongchoice{$5:2$}
        \wrongchoice{$3:2$}
        \wrongchoice{$1:1$}
      \correctchoice{$2:3$}
        \wrongchoice{$2:5$}
    \end{choices}
    \end{multicols}
\end{question}
}

\element{ap}{
\begin{question}{third-law-q13}
    A woman of mass $M$ standing on an ice rink (friction negligible) fires a bullet of mass $m$ from a gun parallel to the surface of the rink.
    The bullet accelerates at a rate of $a$,
        leaving the gun at velocity $v$ and hitting the surface of the rink after time $t$.
    What is the magnitude of the woman's acceleration?
    \begin{multicols}{3}
    \begin{choices}
        \wrongchoice{$\dfrac{vt}{a}$}
        \wrongchoice{$\dfrac{ta}{m}$}
        \wrongchoice{$\dfrac{vtm}{aM}$}
        \wrongchoice{$\dfrac{2vm}{M}$}
      \correctchoice{$\dfrac{ma}{M}$}
    \end{choices}
    \end{multicols}
\end{question}
}

\element{ap}{
\begin{question}{third-law-q14}
    The pain experienced when kicking a large boulder can be explained by:
    \begin{choices}
        \wrongchoice{Newton's First Law of Motion}
        \wrongchoice{Newton's Second Law of Motion}
      \correctchoice{Newton's Third Law of Motion}
        \wrongchoice{Faraday's Law}
        \wrongchoice{Lenz's Law}
    \end{choices}
\end{question}
}

\element{ap}{
\begin{question}{third-law-q15}
    The momentum of an object is directly proportional to its:
    \begin{multicols}{2}
    \begin{choices}
        \wrongchoice{kinetic energy}
        \wrongchoice{potential energy}
      \correctchoice{velocity}
        \wrongchoice{velocity squared}
        \wrongchoice{power}
    \end{choices}
    \end{multicols}
\end{question}
}

\element{ap}{
\begin{question}{third-law-q16}
    A physics teacher pushes against the wall with a force of \SI{100}{\newton}.
    What is the magnitude of the force exerted on the physics teacher by the wall.
    \begin{choices}
        \wrongchoice{\SI{0}{\newton}}
      \correctchoice{\SI{100}{\newton}}
        \wrongchoice{Depends on the mass of the teacher}
        \wrongchoice{Depends on the mass of the wall}
        \wrongchoice{Depends on both the mass of the teacher and the mass of the wall}
    \end{choices}
\end{question}
}

\element{ap}{
\begin{question}{third-law-q17}
    How can Newton's Third Law be explained with an object experiencing fluid friction?
    \begin{choices}
        \wrongchoice{The fluid exerts a force on the object that is an isolated force.}
      \correctchoice{The fluid exerts a force on the object and the object exerts an equal force on the molecules of the fluid.}
        \wrongchoice{The force does work to create heat energy.}
        \wrongchoice{The object decelerates as the sum of all forces is negative.}
        \wrongchoice{The object's motion is slowed because there is a force exerted on it in a direction opposite to its motion.}
    \end{choices}
\end{question}
}

\element{ap}{
\begin{question}{third-law-q18}
    A golfer swings a club and hits a golf ball.
    Which of the following best describes the collision between the club and the ball?
    \begin{choices}
        \wrongchoice{The club exerts a force on the ball that causes it to accelerate for a short period of time.}
        \wrongchoice{The club exerts a force on the ball that cause it to accelerate through the air.}
        \wrongchoice{The club and the ball exert equal and opposite forces on each other, but only ball accelerates due to this force.}
      \correctchoice{The club and the ball exert equal and opposite forces on each other and both objects experience an acceleration.}
        \wrongchoice{The momentum of the club is the same before and after it hits the ball.}
    \end{choices}
\end{question}
}

\element{ap}{
\begin{questionmult}{third-law-q19}
    A golfer swings a golf club and hits a ball.
    Which of the following best explains why the effects of the collision on the club are difficult to observe?
    \begin{choices}
      \correctchoice{The club's mass is large compared to that of the ball.}
      \correctchoice{The club's initial momentum is large compared to the impulse imparted to it by the collision.}
        \wrongchoice{The collision causes only the ball to accelerate, not the club.}
        %\wrongchoice{I only}
        %\wrongchoice{III only}
        %\correctchoice{I and II only}
        %\wrongchoice{II and III only}
        %\wrongchoice{I, II, and III}
    \end{choices}
\end{questionmult}
}


\endinput


