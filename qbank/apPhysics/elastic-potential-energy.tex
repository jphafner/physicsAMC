

%% AP Style MC Question Archive
%%----------------------------------------


%% Elastic Potential energy
%%----------------------------------------
\element{ap}{
\begin{question}{elastic-energy-q01}
    Five identical masses of mass $M$ are suspended by a spring stretched a distance of $L$.
    If three of the masses are removed,
        what is the potential energy stored in the spring?
    \begin{multicols}{3}
    \begin{choices}
        \wrongchoice{$\dfrac{4}{25} MgL$}
        \wrongchoice{$\dfrac{2}{5} MgL^2$}
      \correctchoice{$\dfrac{5}{2} MgL$}
        \wrongchoice{$\dfrac{4}{25} MgL^2$}
        \wrongchoice{$5 MgL$}
    \end{choices}
    \end{multicols}
\end{question}
}

\element{ap}{
\begin{question}{elastic-energy-q02}
    A \SI{1.0}{\kilo\gram} object is suspended from a spring with constant $k=\SI{16}{\newton\per\meter}$.
    The mass is pulled \SI{0.25}{\meter} downward from its equilibrium position and allowed to oscillate.
    What is the maximum kinetic energy of the object?
    \begin{multicols}{3}
    \begin{choices}
        \wrongchoice{\SI{0.25}{\joule}}
      \correctchoice{\SI{0.50}{\joule}}
        \wrongchoice{\SI{1.0}{\joule}}
        \wrongchoice{\SI{2.0}{\joule}}
        \wrongchoice{\SI{4.0}{\joule}}
    \end{choices}
    \end{multicols}
\end{question}
}

\element{ap}{
\begin{question}{elastic-energy-q03}
    Which graph below best represents the relationship between the potential energy stored in a spring ($PE$) and the change in the length of the spring from its equilibrium position ($X$)?
    \begin{multicols}{2}
    \begin{choices}
        \AMCboxDimensions{down=-1.5em}
        \wrongchoice{
            \begin{tikzpicture}
                \begin{axis}[
                    axis y line=left,
                    axis x line=middle,
                    axis line style={->},
                    xlabel={$X$},
                    x label style={
                        at={(current axis.right of origin)},
                        anchor=west,
                    },
                    xtick=\empty,
                    ylabel={$PE$},
                    y label style={
                        at={(current axis.above origin)},
                        anchor=south,
                        rotate=270,
                    },
                    ytick=\empty,
                    xmin=0,xmax=11,
                    ymin=0,ymax=11,
                    width=0.95\columnwidth,
                    very thin,
                ]
                \addplot[line width=1pt,domain=0:10]{8};
                \end{axis}
            \end{tikzpicture}
        }
        %% ANS is B
        \correctchoice{
            \begin{tikzpicture}
                \begin{axis}[
                    axis y line=left,
                    axis x line=middle,
                    axis line style={->},
                    xlabel={$X$},
                    x label style={
                        at={(current axis.right of origin)},
                        anchor=west,
                    },
                    xtick=\empty,
                    ylabel={$PE$},
                    y label style={
                        at={(current axis.above origin)},
                        anchor=south,
                        rotate=270,
                    },
                    ytick=\empty,
                    xmin=0,xmax=11,
                    ymin=0,ymax=11,
                    width=0.95\columnwidth,
                    very thin,
                ]
                \addplot[line width=1pt,domain=0:10]{0.1*x*x};
                \end{axis}
            \end{tikzpicture}
        }
        \wrongchoice{
            \begin{tikzpicture}
                \begin{axis}[
                    axis y line=left,
                    axis x line=middle,
                    axis line style={->},
                    xlabel={$X$},
                    x label style={
                        at={(current axis.right of origin)},
                        anchor=west,
                    },
                    xtick=\empty,
                    ylabel={$PE$},
                    y label style={
                        at={(current axis.above origin)},
                        anchor=south,
                        rotate=270,
                    },
                    ytick=\empty,
                    xmin=0,xmax=11,
                    ymin=0,ymax=11,
                    width=0.95\columnwidth,
                    very thin,
                ]
                \addplot[line width=1pt,domain=0:10]{x};
                \end{axis}
            \end{tikzpicture}
        }
        \wrongchoice{
            \begin{tikzpicture}
                \begin{axis}[
                    axis y line=left,
                    axis x line=middle,
                    axis line style={->},
                    xlabel={$X$},
                    x label style={
                        at={(current axis.right of origin)},
                        anchor=west,
                    },
                    xtick=\empty,
                    ylabel={$PE$},
                    y label style={
                        at={(current axis.above origin)},
                        anchor=south,
                        rotate=270,
                    },
                    ytick=\empty,
                    xmin=0,xmax=11,
                    ymin=0,ymax=11,
                    width=0.95\columnwidth,
                    very thin,
                ]
                \addplot[line width=1pt,domain=0:10]{10/x};
                \end{axis}
            \end{tikzpicture}
        }
        %\wrongchoice{
        %    \begin{tikzpicture}
        %        \draw[white] (-1.5,2) rectangle (1.5,2);
        %        \node[anchor=center] at (0,0) {None of the provided};
        %    \end{tikzpicture}
        %}
    \end{choices}
    \end{multicols}
\end{question}
}

\element{ap}{
\begin{question}{elastic-energy-q04}
    Which graph below best represents the relationship between the potential energy stored in a spring ($PE$) and the force required to keep the spring in equilibrium ($F$)?
    \begin{multicols}{2}
    \begin{choices}
        \AMCboxDimensions{down=-1.5em}
        \wrongchoice{
            \begin{tikzpicture}
                \begin{axis}[
                    axis y line=left,
                    axis x line=middle,
                    axis line style={->},
                    xlabel={$F$},
                    x label style={
                        at={(current axis.right of origin)},
                        anchor=west,
                    },
                    xtick=\empty,
                    ylabel={$PE$},
                    y label style={
                        at={(current axis.above origin)},
                        anchor=south,
                        rotate=270,
                    },
                    ytick=\empty,
                    xmin=0,xmax=11,
                    ymin=0,ymax=11,
                    width=0.95\columnwidth,
                    very thin,
                ]
                \addplot[line width=1pt,domain=0:10]{8};
                \end{axis}
            \end{tikzpicture}
        }
        %% ANS is B
        \correctchoice{
            \begin{tikzpicture}
                \begin{axis}[
                    axis y line=left,
                    axis x line=middle,
                    axis line style={->},
                    xlabel={$F$},
                    x label style={
                        at={(current axis.right of origin)},
                        anchor=west,
                    },
                    xtick=\empty,
                    ylabel={$PE$},
                    y label style={
                        at={(current axis.above origin)},
                        anchor=south,
                        rotate=270,
                    },
                    ytick=\empty,
                    xmin=0,xmax=11,
                    ymin=0,ymax=11,
                    width=0.95\columnwidth,
                    very thin,
                ]
                \addplot[line width=1pt,domain=0:10]{0.1*x*x};
                \end{axis}
            \end{tikzpicture}
        }
        \wrongchoice{
            \begin{tikzpicture}
                \begin{axis}[
                    axis y line=left,
                    axis x line=middle,
                    axis line style={->},
                    xlabel={$F$},
                    x label style={
                        at={(current axis.right of origin)},
                        anchor=west,
                    },
                    xtick=\empty,
                    ylabel={$PE$},
                    y label style={
                        at={(current axis.above origin)},
                        anchor=south,
                        rotate=270,
                    },
                    ytick=\empty,
                    xmin=0,xmax=11,
                    ymin=0,ymax=11,
                    width=0.95\columnwidth,
                    very thin,
                ]
                \addplot[line width=1pt,domain=0:10]{x};
                \end{axis}
            \end{tikzpicture}
        }
        \wrongchoice{
            \begin{tikzpicture}
                \begin{axis}[
                    axis y line=left,
                    axis x line=middle,
                    axis line style={->},
                    xlabel={$F$},
                    x label style={
                        at={(current axis.right of origin)},
                        anchor=west,
                    },
                    xtick=\empty,
                    ylabel={$PE$},
                    y label style={
                        at={(current axis.above origin)},
                        anchor=south,
                        rotate=270,
                    },
                    ytick=\empty,
                    xmin=0,xmax=11,
                    ymin=0,ymax=11,
                    width=0.95\columnwidth,
                    very thin,
                ]
                \addplot[line width=1pt,domain=0:10]{10/x};
                \end{axis}
            \end{tikzpicture}
        }
        %\wrongchoice{
        %    \begin{tikzpicture}
        %        \draw[white] (-1.5,2) rectangle (1.5,2);
        %        \node[anchor=center] at (0,0) {None of the provided};
        %    \end{tikzpicture}
        %}
    \end{choices}
    \end{multicols}
\end{question}
}

\element{ap}{
\begin{question}{elastic-energy-q05}
    A block of mass \SI{5.0}{\kilo\gram} is hung from a vertical spring stretching it \SI{0.40}{\meter}.
    The amount of energy stored in the spring is most nearly
    \begin{multicols}{3}
    \begin{choices}
        \wrongchoice{\SI{0.40}{\joule}}
        \wrongchoice{\SI{0.80}{\joule}}
        \wrongchoice{\SI{8.0}{\joule}}
        \wrongchoice{\SI{10}{\joule}}
      \correctchoice{\SI{20}{\joule}}
    \end{choices}
    \end{multicols}
\end{question}
}

\element{ap}{
\begin{question}{elastic-energy-q06}
    %% changed to reflect gravity
    Which of the following is true of the energy of a spring-mass system in a graviational field?
    \begin{choices}
        \wrongchoice{The total energy is greatest when the velocity is greatest.}
        \wrongchoice{The total energy is greatest when the displacement is greatest.}
        \wrongchoice{The potential energy is greatest when the displacement is least.}
        \wrongchoice{The kinetic energy is greatest when the potential energy is least.}
      \correctchoice{The total energy varies based on the position of the mass.}
    \end{choices}
\end{question}
}

\element{ap}{
\begin{question}{elastic-energy-q07}
    A spring of constant \SI{50}{\newton\per\meter} is used to launch a mass across a rough surface.
    The spring is compressed \SI{0.05}{\meter} and released.
    How much work is done by friction in order to bring the mass to rest?
    \begin{multicols}{3}
    \begin{choices}
      \correctchoice{\SI{0.063}{\joule}}
        \wrongchoice{\SI{0.125}{\joule}}
        \wrongchoice{\SI{1.25}{\joule}}
        \wrongchoice{\SI{2.5}{\joule}}
        \wrongchoice{\SI{100}{\joule}}
    \end{choices}
    \end{multicols}
\end{question}
}

\element{ap}{
\begin{question}{elastic-energy-q08}
    A \SI{3}{\kilo\gram} object is dropped from a height of \SI{4}{\meter} onto a spring.
    The maximum compression of the spring is \SI{0.5}{\meter}.
    What is the spring constant?
    \begin{multicols}{2}
    \begin{choices}
        \wrongchoice{\SI{120}{\newton\per\meter}}
        \wrongchoice{\SI{240}{\newton\per\meter}}
        \wrongchoice{\SI{480}{\newton\per\meter}}
        \wrongchoice{\SI{600}{\newton\per\meter}}
      \correctchoice{\SI{960}{\newton\per\meter}}
    \end{choices}
    \end{multicols}
\end{question}
}


\endinput


