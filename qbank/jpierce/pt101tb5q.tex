

%% Physics 101 Sample Test Questions by Dr. James Pierce
%%------------------------------------------------------------


%% JP's Physics 101 Test Bank 5
%%--------------------------------------------------


%% Topic: Capacitor
\element{jpierce}{
\begin{question}{pt101tb5-Q01}
    A device that stores separated electrical charges is called:
    \begin{multicols}{2}
    \begin{choices}
        \wrongchoice{a resistor}
        \wrongchoice{a conductor}
        \wrongchoice{a wire}
      \correctchoice{a capacitor}
        \wrongchoice{an insulator}
    \end{choices}
    \end{multicols}
\end{question}
}

\element{jpierce}{
\begin{question}{pt101tb5-Q02}
    When an electronic flash on a camera flashes,
        the burst of current used to create the flash of light is produced by the rapid discharge of:
    \begin{multicols}{2}
    \begin{choices}
        \wrongchoice{a battery.}
        \wrongchoice{a coiled spring.}
      \correctchoice{a capacitor.}
        \wrongchoice{a resistor.}
        \wrongchoice{an inductor.}
    \end{choices}
    \end{multicols}
\end{question}
}

\element{jpierce}{
\begin{question}{pt101tb5-Q03}
    A capacitor is a device used to:
    \begin{choices}
        \wrongchoice{measure the volume of a glass container.}
        \wrongchoice{convert electricity into heat.}
        \wrongchoice{force current through a wire.}
      \correctchoice{store separated electrical charges.}
        \wrongchoice{convert electricity into light.}
    \end{choices}
\end{question}
}

%% Topic: Charge
\element{jpierce}{
\begin{question}{pt101tb5-Q04}
    The two types of electrical charge are called
    \begin{choices}
        \wrongchoice{up and down.}
        \wrongchoice{black and white.}
        \wrongchoice{left and right.}
        \wrongchoice{red and green.}
      \correctchoice{positive and negative.}
    \end{choices}
\end{question}
}

\element{jpierce}{
\begin{question}{pt101tb5-Q05}
    A negatively charged object
    \begin{choices}
        \wrongchoice{has a deficiency of electrons.}
        \wrongchoice{has a deficiency of neutrons.}
        \wrongchoice{has an excess of neutrons.}
      \correctchoice{has an excess of electrons.}
        \wrongchoice{has an excess of protons.}
    \end{choices}
\end{question}
}

\element{jpierce}{
\begin{question}{pt101tb5-Q06}
    A positively charged object
    \begin{choices}
        \wrongchoice{has a deficiency of protons.}
      \correctchoice{has a deficiency of electrons.}
        \wrongchoice{has an excess of neutrons.}
        \wrongchoice{has a deficiency of neutrons.}
        \wrongchoice{has an excess of electrons.}
    \end{choices}
\end{question}
}

\element{jpierce}{
\begin{question}{pt101tb5-Q07}
    An object with no electrical charge
    \begin{choices}
        \wrongchoice{has an excess of neutrons.}
      \correctchoice{has an equal number of protons and electrons.}
        \wrongchoice{has an equal number of protons and neutrons.}
        \wrongchoice{has a deficiency of neutrons.}
        \wrongchoice{has an equal number of neutrons and electrons.}
    \end{choices}
\end{question}
}

\element{jpierce}{
\begin{question}{pt101tb5-Q08}
    The unit of electrical charge is the
    \begin{multicols}{2}
    \begin{choices}
        \wrongchoice{volt (\si{\volt}).}
      \correctchoice{coulomb (\si{\coulomb}).}
        \wrongchoice{watt (\si{\watt}).}
        \wrongchoice{newton (\si{\newton}).}
        \wrongchoice{joule (\si{\joule}).}
    \end{choices}
    \end{multicols}
\end{question}
}

%% Topic: Charging
\element{jpierce}{
\begin{question}{pt101tb5-Q09}
    An object can acquire an electrical charge by
    \begin{choices}
        \wrongchoice{radiation, convection, or friction.}
        \wrongchoice{induction, convection, or radiation.}
        \wrongchoice{contact, induction, or convection.}
      \correctchoice{friction, contact, or induction.}
        \wrongchoice{contact, friction, or convection.}
    \end{choices}
\end{question}
}

\element{jpierce}{
\begin{question}{pt101tb5-Q10}
    When you shuffle your feet as you walk across a carpet in the winter,
        you may become electrically charged through the process of:
    \begin{multicols}{2}
    \begin{choices}
        \wrongchoice{conduction.}
        \wrongchoice{convection.}
        \wrongchoice{contact.}
        \wrongchoice{induction.}
      \correctchoice{friction.}
    \end{choices}
    \end{multicols}
\end{question}
}

\element{jpierce}{
\begin{question}{pt101tb5-Q11}
    In the process of \rule[-0.1pt]{4em}{0.1pt}, an object becomes charged
        by rubbing it against another object, resulting in the transfer
        of electrons from one material to the other.
    \begin{multicols}{2}
    \begin{choices}
        \wrongchoice{induction}
        \wrongchoice{convection}
        \wrongchoice{contact}
      \correctchoice{friction}
        \wrongchoice{conduction}
    \end{choices}
    \end{multicols}
\end{question}
}

\element{jpierce}{
\begin{question}{pt101tb5-Q12}
    In the process of \rule[-0.1pt]{4em}{0.1pt}, an object becomes charged
        by touching it to another charged object, allowing transfer of
        electrons from one object to the other.
    \begin{multicols}{2}
    \begin{choices}
      \correctchoice{contact}
        \wrongchoice{convection}
        \wrongchoice{induction}
        \wrongchoice{friction}
        \wrongchoice{conduction}
    \end{choices}
    \end{multicols}
\end{question}
}

\element{jpierce}{
\begin{question}{pt101tb5-Q13}
    In the process of \rule[-0.1pt]{4em}{0.1pt}, an object becomes charged 
        by bringing a charged object nearby; separation of charges on the
        first object occurs and removal of one of these groups leaves the object charged.
    \begin{multicols}{2}
    \begin{choices}
        \wrongchoice{conduction}
        \wrongchoice{contact}
        \wrongchoice{convection}
      \correctchoice{induction}
        \wrongchoice{friction}
    \end{choices}
    \end{multicols}
\end{question}
}

%% Topic: Conductors/Insulators
\element{jpierce}{
\begin{question}{pt101tb5-Q14}
    A material will be a \rule[-0.1pt]{4em}{0.1pt} if it contains
        charges that are free to move around within the matter.
    \begin{multicols}{2}
    \begin{choices}
        \wrongchoice{solid}
        \wrongchoice{poor conductor}
      \correctchoice{good conductor}
        \wrongchoice{good insulator}
        \wrongchoice{fluid}
    \end{choices}
    \end{multicols}
\end{question}
}

\element{jpierce}{
\begin{question}{pt101tb5-Q15}
    A material will be a \rule[-0.1pt]{4em}{0.1pt} if it contains
        charges that are free to move around within the matter.
    \begin{multicols}{2}
    \begin{choices}
        \wrongchoice{poor reflector}
        \wrongchoice{fluid}
        \wrongchoice{good insulator}
        \wrongchoice{poor conductor}
      \correctchoice{poor insulator}
    \end{choices}
    \end{multicols}
\end{question}
}

\element{jpierce}{
\begin{question}{pt101tb5-Q16}
    A material will be a \rule[-0.1pt]{4em}{0.1pt} if it does not
        contain charges that are free to move around within the matter.
    \begin{multicols}{2}
    \begin{choices}
      \correctchoice{poor conductor}
        \wrongchoice{poor insulator}
        \wrongchoice{solid}
        \wrongchoice{fluid}
        \wrongchoice{good conductor}
    \end{choices}
    \end{multicols}
\end{question}
}

\element{jpierce}{
\begin{question}{pt101tb5-Q17}
    A material will be a \rule[-0.1pt]{4em}{0.1pt} if it does not
        contain charges that are free to move around within the matter.
    \begin{choices}
      \correctchoice{good insulator}
        \wrongchoice{solid}
        \wrongchoice{good conductor}
        \wrongchoice{poor insulator}
        \wrongchoice{fluid}
    \end{choices}
\end{question}
}

\element{jpierce}{
\begin{question}{pt101tb5-Q18}
    Good electrical conductors are usually
    \begin{choices}
      \correctchoice{good thermal conductors.}
        \wrongchoice{good thermal insulators.}
        \wrongchoice{poor thermal conductors.}
        \wrongchoice{good electical insulators.}
        \wrongchoice{transparent to light.}
    \end{choices}
\end{question}
}

\element{jpierce}{
\begin{question}{pt101tb5-Q19}
    Good electrical insulators are usually
    \begin{choices}
        \wrongchoice{opaque to light.}
      \correctchoice{poor thermal conductors.}
        \wrongchoice{poor thermal insulators.}
        \wrongchoice{good thermal conductors.}
        \wrongchoice{good electical conductors.}
    \end{choices}
\end{question}
}

%% Topic: Electric Fields
\element{jpierce}{
\begin{question}{pt101tb5-Q20}
    The lines of force for a point charge
    \begin{choices}
        \wrongchoice{connect points of equal charge.}
      \correctchoice{extend radially outward from it.}
        \wrongchoice{form concentric circles about it.}
        \wrongchoice{indicate the direction of motion of the point charge.}
        \wrongchoice{connect points of equal electric potential.}
    \end{choices}
\end{question}
}

\element{jpierce}{
\begin{question}{pt101tb5-Q21}
    The electric field inside a charged conductor (assuming no flow of current)
    \begin{choices}
        \wrongchoice{increases linearly from the center to the outside.}
        \wrongchoice{is strongest at the center.}
        \wrongchoice{decreases linearly from the center to the outside.}
        \wrongchoice{depends on the material composing the conductor.}
      \correctchoice{is zero.}
    \end{choices}
\end{question}
}

\element{jpierce}{
\begin{question}{pt101tb5-Q22}
    The electrostatic charges on a solid spherical conductor arrange themselves such that
    \begin{choices}
        \wrongchoice{they are slightly more concentrated toward the center of the sphere.}
        \wrongchoice{they are evenly spaced throughout the volume of the sphere.}
        \wrongchoice{they lie entirely at the center of the sphere.}
      \correctchoice{they lie entirely on the outside of the sphere.}
        \wrongchoice{they are slightly more concentrated toward the outside of the sphere.}
    \end{choices}
\end{question}
}

\element{jpierce}{
\begin{question}{pt101tb5-Q23}
    We are much more aware of gravitational forces than electrical forces in our lives because
    \begin{choices}
      \correctchoice{most of the matter we encounter has no net charge and feels no electrical force.}
        \wrongchoice{gravitational fields exist but electrical fields cannot.}
        \wrongchoice{gravity is a purely attractive force.}
        \wrongchoice{gravitational forces are inherently much stronger than electrical forces.}
        \wrongchoice{the charges we encounter are so far apart that the electrical forces between them are very weak.}
    \end{choices}
\end{question}
}

\element{jpierce}{
\begin{question}{pt101tb5-Q24}
    The unit used to measure electric force is the:
    \begin{multicols}{2}
    \begin{choices}
        \wrongchoice{volt (\si{\volt}).}
        \wrongchoice{joule (\si{\joule}).}
        \wrongchoice{watt (\si{\watt}).}
        \wrongchoice{coulomb (\si{\coulomb}).}
      \correctchoice{newton (\si{\newton}).}
    \end{choices}
    \end{multicols}
\end{question}
}

\element{jpierce}{
\begin{question}{pt101tb5-Q25}
    According to Coulomb's law, the electrostatic force between two charges
        is proportional to \rule[-0.1pt]{4em}{0.1pt} and
            inversely proportional to \rule[-0.1pt]{4em}{0.1pt}.
    \begin{choices}
      \correctchoice{the product of the charges; the square of the distance between the charges.}
        \wrongchoice{the difference between the charges; the cube of the distance between the charges.}
        \wrongchoice{the sum of the charges; the square of the distance between the charges.}
        \wrongchoice{the product of the charges; the distance between the charges.}
        \wrongchoice{the sum of the charges; the distance between the charges.}
    \end{choices}
\end{question}
}

%% Topic: Electric Forces
\element{jpierce}{
\begin{question}{pt101tb5-Q26}
    According to Coulomb's law, if the distance between two charges is doubled,
        the force each charge exerts on the other will
        be \rule[-0.1pt]{4em}{0.1pt} its previous value.
    \begin{multicols}{2}
    \begin{choices}
        \wrongchoice{four times}
        \wrongchoice{the same as}
      \correctchoice{one fourth of}
        \wrongchoice{double}
        \wrongchoice{one half of}
    \end{choices}
    \end{multicols}
\end{question}
}

\element{jpierce}{
\begin{question}{pt101tb5-Q27}
    According to Coulomb's law, if the distance between two charges is cut in half,
        the force each charge exerts on the other will
        be \rule[-0.1pt]{4em}{0.1pt} its previous value.
    \begin{multicols}{2}
    \begin{choices}
        \wrongchoice{one half of}
        \wrongchoice{one fourth of}
        \wrongchoice{the same as}
        \wrongchoice{double}
      \correctchoice{four times}
    \end{choices}
    \end{multicols}
\end{question}
}

\element{jpierce}{
\begin{question}{pt101tb5-Q28}
    According to Coulomb's law, if the electrostatic charge on each of two
        small spheres is doubled, the force each sphere exerts on the other
        will be \rule[-0.1pt]{4em}{0.1pt} its previous value.
    \begin{multicols}{2}
    \begin{choices}
        \wrongchoice{one half of}
      \correctchoice{four times}
        \wrongchoice{twice}
        \wrongchoice{the same as}
        \wrongchoice{one fourth of}
    \end{choices}
    \end{multicols}
\end{question}
}

\element{jpierce}{
\begin{question}{pt101tb5-Q29}
    According to Coulomb's law, if the distance between two charges is tripled, 
        the force each charge exerts on the other
        will be \rule[-0.1pt]{4em}{0.1pt} its previous value.
    \begin{multicols}{2}
    \begin{choices}
        \wrongchoice{three times}
      \correctchoice{one ninth of}
        \wrongchoice{the same as}
        \wrongchoice{nine times}
        \wrongchoice{one third of}
    \end{choices}
    \end{multicols}
\end{question}
}

\element{jpierce}{
\begin{question}{pt101tb5-Q30}
    According to Coulomb's law, if the electrostatic charge on each of two
        small spheres is tripled, the force each sphere exerts on the other
        will be \rule[-0.1pt]{4em}{0.1pt} its previous value.
    \begin{multicols}{2}
    \begin{choices}
        \wrongchoice{one third of}
      \correctchoice{nine times}
        \wrongchoice{one ninth of}
        \wrongchoice{the same as}
        \wrongchoice{three times}
    \end{choices}
    \end{multicols}
\end{question}
}

\element{jpierce}{
\begin{question}{pt101tb5-Q31}
    According to Coulomb's law, if the distance between two charges is quadrupled,
        the force each charge exerts on the other
        will be \rule[-0.1pt]{4em}{0.1pt} its previous value.
    \begin{multicols}{2}
    \begin{choices}
      \correctchoice{one sixteenth of}
        \wrongchoice{four times}
        \wrongchoice{sixteen times}
        \wrongchoice{the same as}
        \wrongchoice{one fourth of}
    \end{choices}
    \end{multicols}
\end{question}
}

%% Topic: Potential
\element{jpierce}{
\begin{question}{pt101tb5-Q32}
    The unit of electric potential is the:
    \begin{multicols}{2}
    \begin{choices}
      \correctchoice{volt (\si{\volt}).}
        \wrongchoice{watt (\si{\watt}).}
        \wrongchoice{joule (\si{\joule}).}
        \wrongchoice{coulomb (\si{\coulomb}).}
        \wrongchoice{newton (\si{\newton}).}
    \end{choices}
    \end{multicols}
\end{question}
}

\element{jpierce}{
\begin{question}{pt101tb5-Q33}
    The volt is the unit of electric:
    \begin{multicols}{2}
    \begin{choices}
        \wrongchoice{resistance.}
        \wrongchoice{power.}
        \wrongchoice{current.}
      \correctchoice{potential.}
        \wrongchoice{charge.}
    \end{choices}
    \end{multicols}
\end{question}
}

%% Topic: Current
\element{jpierce}{
\begin{question}{pt101tb5-Q34}
    The unit of electric current is the:
    \begin{multicols}{2}
    \begin{choices}
        \wrongchoice{watt (\si{\watt}).}
        \wrongchoice{joule (\si{\joule}).}
        \wrongchoice{volt (\si{\volt}).}
      \correctchoice{ampere (\si{\ampere}).}
        \wrongchoice{coulomb (\si{\coulomb}).}
    \end{choices}
    \end{multicols}
\end{question}
}

\element{jpierce}{
\begin{question}{pt101tb5-Q35}
    The flow of electric charge is called:
    \begin{multicols}{2}
    \begin{choices}
        \wrongchoice{voltage.}
        \wrongchoice{resistance.}
        \wrongchoice{power.}
        \wrongchoice{potential.}
      \correctchoice{current.}
    \end{choices}
    \end{multicols}
\end{question}
}

\element{jpierce}{
\begin{question}{pt101tb5-Q36}
    If the charges in an electrical circuit always flow in the same direction,
        the current is called
    \begin{choices}
        \wrongchoice{a constant current.}
      \correctchoice{a direct current.}
        \wrongchoice{a basic current.}
        \wrongchoice{an oscillating current.}
        \wrongchoice{an alternating current.}
    \end{choices}
\end{question}
}

\element{jpierce}{
\begin{question}{pt101tb5-Q37}
    If the charges in an electrical circuit periodically switch their direction of flow,
        the current is called
    \begin{choices}
        \wrongchoice{a constant current.}
        \wrongchoice{a direct current.}
        \wrongchoice{a basic current.}
      \correctchoice{an alternating current.}
        \wrongchoice{an oscillating current.}
    \end{choices}
\end{question}
}

\element{jpierce}{
\begin{question}{pt101tb5-Q38}
    The charges in \rule[-0.1pt]{4em}{0.1pt} circuit periodically switch their direction of flow,
        while the charges in \rule[-0.1pt]{4em}{0.1pt} circuit always flow in the same direction.
    \begin{multicols}{2}
    \begin{choices}
        \wrongchoice{a DA; a DC}
        \wrongchoice{an AC; an AD}
        \wrongchoice{an AD; an AC}
        \wrongchoice{a DC; an AC}
      \correctchoice{an AC; a DC}
    \end{choices}
    \end{multicols}
\end{question}
}

\element{jpierce}{
\begin{question}{pt101tb5-Q39}
    The charges in \rule[-0.1pt]{4em}{0.1pt} circuit always flow in the
        same direction while the charges in \rule[-0.1pt]{4em}{0.1pt} circuit
        periodically switch their direction of flow.
    \begin{multicols}{2}
    \begin{choices}
        \wrongchoice{an AC; a DC}
        \wrongchoice{an AC; an AD}
      \correctchoice{a DC; an AC}
        \wrongchoice{a DC; a DA}
        \wrongchoice{an AD; an AC}
    \end{choices}
    \end{multicols}
\end{question}
}

\element{jpierce}{
\begin{question}{pt101tb5-Q40}
    If a charge of \SI{12}{\coulomb} flows through a wire every \SI{3}{\second},
        the current in the wire is:
    \begin{multicols}{3}
    \begin{choices}
        \wrongchoice{\SI{12}{\ampere}}
      \correctchoice{\SI{4}{\ampere}}
        \wrongchoice{\SI{9}{\ampere}}
        \wrongchoice{\SI{3}{\ampere}}
        \wrongchoice{\SI{36}{\ampere}}
    \end{choices}
    \end{multicols}
\end{question}
}

\element{jpierce}{
\begin{question}{pt101tb5-Q41}
    If a charge of \SI{12}{\coulomb} flows through a wire every \SI{4}{\second},
        the current in the wire is:
    \begin{multicols}{3}
    \begin{choices}
      \correctchoice{\SI{3}{\ampere}}
        \wrongchoice{\SI{4}{\ampere}}
        \wrongchoice{\SI{8}{\ampere}}
        \wrongchoice{\SI{12}{\ampere}}
        \wrongchoice{\SI{48}{\ampere}}
    \end{choices}
    \end{multicols}
\end{question}
}

\element{jpierce}{
\begin{question}{pt101tb5-Q42}
    If the current in a wire is \SI{6}{\ampere},
        how much charge will flow through it in \SI{3}{\second}?
    \begin{multicols}{3}
    \begin{choices}
        \wrongchoice{\SI{2}{\coulomb}}
        \wrongchoice{\SI{3}{\coulomb}}
        \wrongchoice{\SI{9}{\coulomb}}
      \correctchoice{\SI{18}{\coulomb}}
        \wrongchoice{\SI{6}{\coulomb}}
    \end{choices}
    \end{multicols}
\end{question}
}

\element{jpierce}{
\begin{question}{pt101tb5-Q43}
    If the current in a wire is \SI{6}{\ampere},
        how much charge will flow through it in \SI{2}{\second}?
    \begin{multicols}{3}
    \begin{choices}
        \wrongchoice{\SI{2}{\coulomb}}
        \wrongchoice{\SI{8}{\coulomb}}
        \wrongchoice{\SI{6}{\coulomb}}
      \correctchoice{\SI{12}{\coulomb}}
        \wrongchoice{\SI{3}{\coulomb}}
    \end{choices}
    \end{multicols}
\end{question}
}

%% Topic: Ohm's Law
\element{jpierce}{
\begin{question}{pt101tb5-Q44}
    According to Ohm's law,
        the current in a circuit is equal to:
    \begin{choices}
        \wrongchoice{the difference between voltage and resistance.}
        \wrongchoice{the product of resistance and voltage.}
        \wrongchoice{the ratio of resistance to voltage.}
        \wrongchoice{the sum of resistance and voltage.}
      \correctchoice{the ratio of voltage to resistance.}
    \end{choices}
\end{question}
}

\element{jpierce}{
\begin{question}{pt101tb5-Q45}
    According to Ohm's law,
        the voltage drop across a light bulb in a circuit is equal
        to \rule[-0.1pt]{4em}{0.1pt} in the bulb.
    \begin{choices}
        \wrongchoice{the sum of resistance and current}
        \wrongchoice{the difference between current and resistance}
        \wrongchoice{the ratio of resistance to current}
      \correctchoice{the product of resistance and current}
        \wrongchoice{the ratio of current to resistance}
    \end{choices}
\end{question}
}

\element{jpierce}{
\begin{question}{pt101tb5-Q46}
    According to Ohm's law,
        the resistance of a light bulb in a circuit is
        equal to \rule[-0.1pt]{4em}{0.1pt} in the bulb.
    \begin{choices}
        \wrongchoice{the sum of voltage and current}
        \wrongchoice{the difference between voltage and current}
      \correctchoice{the ratio of voltage to current}
        \wrongchoice{the ratio of current to voltage}
        \wrongchoice{the product of voltage and current}
    \end{choices}
\end{question}
}

\element{jpierce}{
\begin{question}{pt101tb5-Q47}
    If a voltage of \SI{10}{\volt} produces a current of \SI{2}{\ampere}
        in an electrical device, the resistance must be:
    \begin{multicols}{3}
    \begin{choices}
        \wrongchoice{\SI{2}{\ohm}.}
      \correctchoice{\SI{5}{\ohm}.}
        \wrongchoice{\SI{10}{\ohm}.}
        \wrongchoice{\SI{20}{\ohm}.}
        \wrongchoice{\SI{8}{\ohm}.}
    \end{choices}
    \end{multicols}
\end{question}
}

\element{jpierce}{
\begin{question}{pt101tb5-Q48}
    If a voltage of \SI{110}{\volt} produces a current of \SI{2}{\ampere}
        in an electrical device, the resistance must be:
    \begin{multicols}{3}
    \begin{choices}
      \correctchoice{\SI{55}{\ohm}.}
        \wrongchoice{\SI{110}{\ohm}.}
        \wrongchoice{\SI{2}{\ohm}.}
        \wrongchoice{\SI{220}{\ohm}.}
        \wrongchoice{\SI{108}{\ohm}.}
    \end{choices}
    \end{multicols}
\end{question}
}

\element{jpierce}{
\begin{question}{pt101tb5-Q49}
    If a \SI{100}{\ohm} resistor has a \SI{5}{\ampere} current flowing through it,
        the voltage drop across the resistor is:
    \begin{multicols}{3}
    \begin{choices}
      \correctchoice{\SI{500}{\volt}.}
        \wrongchoice{\SI{100}{\volt}.}
        \wrongchoice{\SI{5}{\volt}.}
        \wrongchoice{\SI{20}{\volt}.}
        \wrongchoice{\SI{50}{\volt}.}
    \end{choices}
    \end{multicols}
\end{question}
}

\element{jpierce}{
\begin{question}{pt101tb5-Q50}
    If a \SI{10}{\ohm} resistor has a \SI{5}{\ampere} current flowing through it,
        the voltage drop across the resistor is:
    \begin{multicols}{3}
    \begin{choices}
        \wrongchoice{\SI{5}{\volt}.}
        \wrongchoice{\SI{10}{\volt}.}
        \wrongchoice{\SI{15}{\volt}.}
        \wrongchoice{\SI{2}{\volt}.}
      \correctchoice{\SI{50}{\volt}.}
    \end{choices}
    \end{multicols}
\end{question}
}

\element{jpierce}{
\begin{question}{pt101tb5-Q51}
    A voltage of \SI{100}{\volt} should produce a current
        of \rule[-0.1pt]{4em}{0.1pt} in a \SI{20}{\ohm} resistor.
    \begin{multicols}{3}
    \begin{choices}
        \wrongchoice{\SI{80}{\ampere}}
      \correctchoice{\SI{5}{\ampere}}
        \wrongchoice{\SI{60}{\ampere}}
        \wrongchoice{\SI{2000}{\ampere}}
        \wrongchoice{\SI{120}{\ampere}}
    \end{choices}
    \end{multicols}
\end{question}
}

\element{jpierce}{
\begin{question}{pt101tb5-Q52}
    A voltage of \SI{60}{\volt} should produce a current
        of \rule[-0.1pt]{4em}{0.1pt} in a \SI{6}{\ohm} resistor.
    \begin{multicols}{3}
    \begin{choices}
        \wrongchoice{\SI{360}{\ampere}}
        \wrongchoice{\SI{66}{\ampere}}
        \wrongchoice{\SI{36}{\ampere}}
        \wrongchoice{\SI{6}{\ampere}}
      \correctchoice{\SI{10}{\ampere}}
    \end{choices}
    \end{multicols}
\end{question}
}

\element{jpierce}{
\begin{question}{pt101tb5-Q53}
    Electric power is measured in:
    \begin{multicols}{2}
    \begin{choices}
      \correctchoice{watts (\si{\watt}).}
        \wrongchoice{coulombs (\si{\coulomb}).}
        \wrongchoice{ohms (\si{\ohm}).}
        \wrongchoice{volts (\si{\volt}).}
        \wrongchoice{amperes (\si{\ampere}).}
    \end{choices}
    \end{multicols}
\end{question}
}

\element{jpierce}{
\begin{question}{pt101tb5-Q54}
    The power consumed by an electric light bulb is:
    \begin{choices}
        \wrongchoice{the ratio of the current passing through it to the voltage across it.}
        \wrongchoice{the ratio of the voltage across it to the current passing through it.}
        \wrongchoice{the difference between the voltage across it and the current passing through it.}
      \correctchoice{the product of the voltage across it and the current passing through it.}
        \wrongchoice{the sum of the voltage across it and the current passing through it.}
    \end{choices}
\end{question}
}

%% Topic: Power
\element{jpierce}{
\begin{question}{pt101tb5-Q55}
    If \SI{120}{\volt} are used to light a \SI{60}{\watt} light bulb,
        the current in the bulb will be:
    \begin{multicols}{3}
    \begin{choices}
        \wrongchoice{\SI{60}{\ampere}.}
        \wrongchoice{\SI{2}{\ampere}.}
        \wrongchoice{\SI{180}{\ampere}.}
        \wrongchoice{\SI{120}{\ampere}.}
      \correctchoice{\SI{0.5}{\ampere}.}
    \end{choices}
    \end{multicols}
\end{question}
}

\element{jpierce}{
\begin{question}{pt101tb5-Q56}
    If \SI{120}{\volt} are used to light a \SI{30}{\watt} light bulb,
        the current in the bulb will be:
    \begin{multicols}{3}
    \begin{choices}
        \wrongchoice{\SI{4}{\ampere}.}
      \correctchoice{\SI{0.25}{\ampere}.}
        \wrongchoice{\SI{150}{\ampere}.}
        \wrongchoice{\SI{30}{\ampere}.}
        \wrongchoice{\SI{120}{\ampere}.}
    \end{choices}
    \end{multicols}
\end{question}
}

\element{jpierce}{
\begin{question}{pt101tb5-Q57}
    If \SI{120}{\volt} are used to light a \SI{100}{\watt} light bulb,
        the current in the bulb will be:
    \begin{multicols}{3}
    \begin{choices}
      \correctchoice{\SI{0.83}{\ampere}.}
        \wrongchoice{\SI{100}{\ampere}.}
        \wrongchoice{\SI{220}{\ampere}.}
        \wrongchoice{\SI{20}{\ampere}.}
        \wrongchoice{\SI{1.2}{\ampere}.}
    \end{choices}
    \end{multicols}
\end{question}
}

\element{jpierce}{
\begin{question}{pt101tb5-Q58}
    If \SI{120}{\volt} are used to light a \SI{150}{\watt} light bulb,
        the current in the bulb will be:
    \begin{multicols}{3}
    \begin{choices}
      \correctchoice{\SI{1.25}{\ampere}.}
        \wrongchoice{\SI{120}{\ampere}.}
        \wrongchoice{\SI{270}{\ampere}.}
        \wrongchoice{\SI{30}{\ampere}.}
        \wrongchoice{\SI{150}{\ampere}.}
    \end{choices}
    \end{multicols}
\end{question}
}

\element{jpierce}{
\begin{question}{pt101tb5-Q59}
    A light bulb that draws a current of \SI{1}{\ampere} when plugged
        into a \SI{120}{\volt} outlet would consume \rule[-0.1pt]{4em}{0.1pt} of power.
    \begin{multicols}{3}
    \begin{choices}
      \correctchoice{\SI{120}{\watt}}
        \wrongchoice{\SI{1200}{\watt}}
        \wrongchoice{\SI{1.2}{\watt}}
        \wrongchoice{\SI{12}{\watt}}
        \wrongchoice{\SI{1}{\watt}}
    \end{choices}
    \end{multicols}
\end{question}
}

\element{jpierce}{
\begin{question}{pt101tb5-Q60}
    A light bulb that draws a current of \SI{0.5}{\ampere} when plugged
        into a \SI{120}{\volt} outlet would consume \rule[-0.1pt]{4em}{0.1pt} of power.
    \begin{multicols}{3}
    \begin{choices}
        \wrongchoice{\SI{120}{\watt}}
        \wrongchoice{\SI{90}{\watt}}
        \wrongchoice{\SI{30}{\watt}}
        \wrongchoice{\SI{240}{\watt}}
      \correctchoice{\SI{60}{\watt}}
    \end{choices}
    \end{multicols}
\end{question}
}

%% topic: Resistance
\element{jpierce}{
\begin{question}{pt101tb5-Q61}
    Electrical resistance is measured in:
    \begin{multicols}{2}
    \begin{choices}
      \correctchoice{ohms (\si{\ohm}).}
        \wrongchoice{joules (\si{\joule}).}
        \wrongchoice{coulombs (\si{\coulomb}).}
        \wrongchoice{amperes (\si{\ampere}).}
        \wrongchoice{volts (\si{\volt}).}
    \end{choices}
    \end{multicols}
\end{question}
}

\element{jpierce}{
\begin{question}{pt101tb5-Q62}
    Wires that are \rule[-0.1pt]{4em}{0.1pt} have higher resistance.
    \begin{choices}
        \wrongchoice{longer and thicker}
      \correctchoice{longer and thinner}
        \wrongchoice{shorter and thinner}
        \wrongchoice{curved}
        \wrongchoice{shorter and thicker}
    \end{choices}
\end{question}
}

\element{jpierce}{
\begin{question}{pt101tb5-Q63}
    Wires that are \rule[-0.1pt]{4em}{0.1pt} have lower resistance.
    \begin{choices}
        \wrongchoice{longer and thicker}
        \wrongchoice{straight}
        \wrongchoice{shorter and thinner}
        \wrongchoice{longer and thinner}
      \correctchoice{shorter and thicker}
    \end{choices}
\end{question}
}

\element{jpierce}{
\begin{question}{pt101tb5-Q64}
    The resistance of the dry human body is about:
    \begin{multicols}{2}
    \begin{choices}
        \wrongchoice{\SI{100}{\ohm}.}
        \wrongchoice{\SI{1}{\ohm}.}
      \correctchoice{\SI{100 000}{\ohm}.}
        \wrongchoice{\SI{1000}{\ohm}.}
        \wrongchoice{\SI{10}{\ohm}.}
    \end{choices}
    \end{multicols}
\end{question}
}

\element{jpierce}{
\begin{question}{pt101tb5-Q65}
    The resistance of the dry human body is about:
    \begin{multicols}{2}
    \begin{choices}
      \correctchoice{\SI{100 000}{\ohm}.}
        \wrongchoice{\SI{0.00001}{\ohm}.}
        \wrongchoice{\SI{10}{\ohm}.}
        \wrongchoice{\SI{1000}{\ohm}.}
        \wrongchoice{\SI{0.01}{\ohm}.}
    \end{choices}
    \end{multicols}
\end{question}
}

\element{jpierce}{
\begin{question}{pt101tb5-Q66}
    The resistance of the dry human body is about \num{100 000}:
    \begin{multicols}{2}
    \begin{choices}
        \wrongchoice{coulombs (\si{\coulomb}).}
        \wrongchoice{amps (\si{\ampere}).}
        \wrongchoice{watts (\si{\watt}).}
        \wrongchoice{joules (\si{\joule}).}
      \correctchoice{ohms (\si{\ohm}).}
    \end{choices}
    \end{multicols}
\end{question}
}

\element{jpierce}{
\begin{question}{pt101tb5-Q67}
    If light bulb $A$ has twice the resistance of light bulb $B$ and the
        same current passes through each bulb, the voltage across
        bulb $A$ will be \rule[-0.1pt]{4em}{0.1pt} the voltage across bulb $B$.
    \begin{multicols}{2}
    \begin{choices}
      \correctchoice{two times}
        \wrongchoice{one fourth of}
        \wrongchoice{four times}
        \wrongchoice{equal to}
        \wrongchoice{one half of}
    \end{choices}
    \end{multicols}
\end{question}
}

\element{jpierce}{
\begin{question}{pt101tb5-Q68}
    If light bulb $A$ has four times the resistance of light bulb $B$ and the
        same current passes through each bulb, the voltage across
        bulb $A$ will be \rule[-0.1pt]{4em}{0.1pt} the voltage across bulb $B$.
    \begin{multicols}{2}
    \begin{choices}
        \wrongchoice{one half of}
      \correctchoice{four times}
        \wrongchoice{two times}
        \wrongchoice{equal to}
        \wrongchoice{one fourth of}
    \end{choices}
    \end{multicols}
\end{question}
}

\element{jpierce}{
\begin{question}{pt101tb5-Q69}
    If light bulb $A$ has twice the resistance of light bulb $B$ and the
        same voltage is applied across each bulb, the current in
        bulb $A$ will be \rule[-0.1pt]{4em}{0.1pt} the current in bulb $B$.
    \begin{multicols}{2}
    \begin{choices}
      \correctchoice{one half of}
        \wrongchoice{two times}
        \wrongchoice{equal to}
        \wrongchoice{four times}
        \wrongchoice{one fourth of}
    \end{choices}
    \end{multicols}
\end{question}
}

\element{jpierce}{
\begin{question}{pt101tb5-Q70}
    If light bulb $A$ has four times the resistance of light bulb $B$ and the
        same voltage is applied across each bulb, the current in
        bulb $A$ will be \rule[-0.1pt]{4em}{0.1pt} the current in bulb $B$.
    \begin{multicols}{2}
    \begin{choices}
      \correctchoice{one fourth of}
        \wrongchoice{two times}
        \wrongchoice{one half of}
        \wrongchoice{four times}
        \wrongchoice{equal to}
    \end{choices}
    \end{multicols}
\end{question}
}

\element{jpierce}{
\begin{question}{pt101tb5-Q71}
    If three light bulbs of different wattage are connected in parallel to a battery,
    \begin{choices}
      \correctchoice{the voltage drop across each bulb will be the same.}
        \wrongchoice{the light output of each bulb will be the same.}
        \wrongchoice{the current in each light bulb will be the same.}
        \wrongchoice{the resistance in each light bulb will be the same.}
        \wrongchoice{the power consumed by each light bulb will be the same.}
    \end{choices}
\end{question}
}

%% Topic: Series/Parallel
\element{jpierce}{
\begin{question}{pt101tb5-Q72}
    If three light bulbs of different wattage are connected in series to a battery,
    \begin{choices}
        \wrongchoice{the voltage drop across each bulb will be the same.}
        \wrongchoice{the power consumed by each light bulb will be the same.}
        \wrongchoice{the resistance in each light bulb will be the same.}
        \wrongchoice{the light output of each bulb will be the same.}
      \correctchoice{the current in each light bulb will be the same.}
    \end{choices}
\end{question}
}

\element{jpierce}{
\begin{question}{pt101tb5-Q73}
    If identical light bulbs $A$, $B$, and $C$ are connected such that $A$ and $B$
        are in parallel with each other and the AB combination is in series with $C$
        and the power supply, how will the brightness of the bulbs compare when
        the power is turned on?
    \begin{choices}
      \correctchoice{$A$ and $B$ will be the same, and $C$ will be brighter.}
        \wrongchoice{All three bulbs will be equally bright.}
        \wrongchoice{All three will be different, with $A$ faintest and $C$ brightest.}
        \wrongchoice{All three will be different, with $A$ brightest and $C$ faintest.}
        \wrongchoice{$A$ and $B$ will be the same, and $C$ will be fainter.}
    \end{choices}
\end{question}
}

\element{jpierce}{
\begin{question}{pt101tb5-Q74}
    Three light bulbs are connected as shown and the terminals (dots)
        are connected to a power supply. 
    \begin{center}
    \begin{circuitikz}
        %% NOTE:
    \end{circuitikz}
    \end{center}
    Which of these statements best describes the resulting circuit?
    \begin{choices}
        \wrongchoice{$L_1$, $L_2$, and $L_3$ are connected in series.}
        \wrongchoice{$L_1$ and $L_2$ are connected in series and together are in parallel with $L_3$.}
        \wrongchoice{$L_1$, $L_2$, and $L_3$ are all connected in parallel.}
        \wrongchoice{$L_1$ and $L_2$ are connected in parallel and together are in parallel with $L_3$.}
      \correctchoice{$L_1$ and $L_2$ are connected in parallel and together are in series with $L_3$.}
    \end{choices}
\end{question}
}

\element{jpierce}{
\begin{question}{pt101tb5-Q75}
    Three light bulbs are connected as shown and the terminals (dots)
        are connected to a power supply.
    \begin{center}
    \begin{circuitikz}
        %% NOTE:
    \end{circuitikz}
    \end{center}
    Which of these statements best describes the resulting circuit?
    \begin{choices}
        \wrongchoice{$L_1$, $L_2$, and $L_3$ are all connected in parallel.}
        \wrongchoice{$L_1$ and $L_2$ are connected in parallel and together are in series with $L_3$.}
        \wrongchoice{$L_1$, $L_2$, and $L_3$ are connected in series.}
        \wrongchoice{$L_2$ and $L_3$ are connected in series and together are in parallel with $L_1$.}
      \correctchoice{$L_2$ and $L_3$ are connected in parallel and together are in series with $L_1$.}
    \end{choices}
\end{question}
}

\element{jpierce}{
\begin{question}{pt101tb5-Q76}
    Three light bulbs are connected as shown and the terminals (dots) are
        connected to a power supply. 
    \begin{center}
    \begin{circuitikz}
        %% NOTE:
    \end{circuitikz}
    \end{center}
    Which of these statements best describes the resulting circuit?
    \begin{choices}
        \wrongchoice{$L_1$ and $L_2$ are connected in series and together are in parallel with $L_3$.}
      \correctchoice{$L_2$ and $L_3$ are connected in series and together are in parallel with $L_1$.}
        \wrongchoice{$L_1$, $L_2$, and $L_3$ are all connected in parallel.}
        \wrongchoice{$L_2$ and $L_3$ are connected in parallel and together are in series with $L_1$.}
        \wrongchoice{$L_1$ and $L_2$ are connected in parallel and together are in series with $L_3$.}
    \end{choices}
\end{question}
}

\element{jpierce}{
\begin{question}{pt101tb5-Q77}
    If identical light bulbs $A$, $B$, and $C$ are connected such
        that $B$ and $C$ are in parallel with each other and the $BC$
        combination is in series with A and the power supply,
        how will the brightness of the bulbs compare when the power is turned on?
    \begin{choices}
        \wrongchoice{All three bulbs will be equally bright.}
      \correctchoice{$B$ and $C$ will be the same, and $A$ will be brighter.}
        \wrongchoice{All three will be different, with $A$ brightest and $B$ faintest.}
        \wrongchoice{All three will be different, with $A$ faintest and $B$ brightest.}
        \wrongchoice{$B$ and $C$ will be the same, and $A$ will be fainter.}
    \end{choices}
\end{question}
}

%% Topic: EM Spectrum
\element{jpierce}{
\begin{question}{pt101tb5-Q78}
    The electromagnetic spectrum includes all of the following except:
    \begin{multicols}{2}
    \begin{choices}
        \wrongchoice{radio waves.}
        \wrongchoice{visible light.}
        \wrongchoice{microwaves.}
      \correctchoice{sound waves.}
        \wrongchoice{x-rays.}
    \end{choices}
    \end{multicols}
\end{question}
}

\element{jpierce}{
\begin{question}{pt101tb5-Q79}
    The electromagnetic spectrum includes all of the following except:
    \begin{multicols}{2}
    \begin{choices}
      \correctchoice{shock waves.}
        \wrongchoice{radio waves.}
        \wrongchoice{microwaves.}
        \wrongchoice{visible light.}
        \wrongchoice{x-rays.}
    \end{choices}
    \end{multicols}
\end{question}
}

\element{jpierce}{
\begin{question}{pt101tb5-Q80}
    In order of increasing wavelength, the electromagnetic spectrum includes
    \begin{choices}
      \correctchoice{ultraviolet, visible light, and infrared.}
        \wrongchoice{visible light, radio, and microwaves.}
        \wrongchoice{x-rays, microwaves, and visible light.}
        \wrongchoice{gamma rays, visible light, and ultraviolet.}
        \wrongchoice{infrared, visible light, and ultraviolet.}
    \end{choices}
\end{question}
}

\element{jpierce}{
\begin{question}{pt101tb5-Q81}
    In order of decreasing wavelength, the electromagnetic spectrum includes
    \begin{choices}
      \correctchoice{infrared, visible light, and ultraviolet.}
        \wrongchoice{gamma rays, visible light, and ultraviolet.}
        \wrongchoice{x-rays, microwaves, and visible light.}
        \wrongchoice{ultraviolet, visible light, and infrared.}
        \wrongchoice{visible light, radio, and microwaves.}
    \end{choices}
\end{question}
}

\element{jpierce}{
\begin{question}{pt101tb5-Q82}
    Of all the electromagnetic waves, those with highest energy
        are \rule[-0.1pt]{4em}{0.1pt} and those with
        lowest energy are \rule[-0.1pt]{4em}{0.1pt}.
    \begin{choices}
        \wrongchoice{visible light; infrared}
      \correctchoice{gamma rays; radio waves}
        \wrongchoice{radio waves; x-rays}
        \wrongchoice{x-rays; microwaves}
        \wrongchoice{microwaves; ultraviolet}
    \end{choices}
\end{question}
}

\element{jpierce}{
\begin{question}{pt101tb5-Q83}
    Of all the electromagnetic waves, those with lowest energy
        are \rule[-0.1pt]{4em}{0.1pt} and those with
        highest energy are \rule[-0.1pt]{4em}{0.1pt}.
    \begin{choices}
      \correctchoice{radio waves; gamma rays}
        \wrongchoice{infrared; ultraviolet}
        \wrongchoice{visible; x-rays}
        \wrongchoice{gamma rays; radio waves}
        \wrongchoice{ultraviolet; infrared}
    \end{choices}
\end{question}
}

\element{jpierce}{
\begin{question}{pt101tb5-Q84}
    The color of visible light that has the longest wavelength is:
    \begin{multicols}{3}
    \begin{choices}
      \correctchoice{red}
        \wrongchoice{violet}
        \wrongchoice{green}
        \wrongchoice{blue}
        \wrongchoice{orange}
    \end{choices}
    \end{multicols}
\end{question}
}

\element{jpierce}{
\begin{question}{pt101tb5-Q85}
    The color of visible light that has the shortest wavelength is:
    \begin{multicols}{3}
    \begin{choices}
        \wrongchoice{green}
        \wrongchoice{red}
        \wrongchoice{blue}
        \wrongchoice{orange}
      \correctchoice{violet}
    \end{choices}
    \end{multicols}
\end{question}
}

%% Topic: EM Wavespeed
\element{jpierce}{
\begin{question}{pt101tb5-Q86}
    Which of the following is true concerning electromagnetic waves in a vacuum?
    \begin{choices}
        \wrongchoice{Those with long wavelengths travel more rapidly than those with short wavelengths.}
        \wrongchoice{Those with low frequencies travel more rapidly than those with high frequencies.}
        \wrongchoice{Those with short wavelengths travel more rapidly than those with long wavelengths.}
      \correctchoice{They all travel at the same speed, independent of wavelength or frequency.}
        \wrongchoice{Those with high frequencies travel more rapidly than those with low frequencies.}
    \end{choices}
\end{question}
}

\element{jpierce}{
\begin{question}{pt101tb5-Q87}
    Which of the following is true concerning light waves in a vacuum?
    \begin{choices}
        \wrongchoice{Those with high frequencies travel more rapidly than those with low frequencies.}
      \correctchoice{They all travel at the same speed, independent of wavelength or frequency.}
        \wrongchoice{Those with long wavelengths travel more rapidly than those with short wavelengths.}
        \wrongchoice{Those with short wavelengths travel more rapidly than those with long wavelengths.}
        \wrongchoice{Those with low frequencies travel more rapidly than those with high frequencies.}
    \end{choices}
\end{question}
}

\element{jpierce}{
\begin{question}{pt101tb5-Q88}
    The speed of light in a vacuum
    \begin{choices}
      \correctchoice{is the same for all the different colors of light.}
        \wrongchoice{is chosen to be equal to the speed of yellow light, which moves faster than any other color.}
        \wrongchoice{is higher for green light than for violet light.}
        \wrongchoice{is higher for blue light than for red light.}
        \wrongchoice{is found by averaging the different speeds of all the different colors of light.}
    \end{choices}
\end{question}
}

\element{jpierce}{
\begin{question}{pt101tb5-Q89}
    Compared to visible light traveling in a vacuum,
        infrared rays would have
    \begin{choices}
        \wrongchoice{the same speed and shorter wavelengths.}
      \correctchoice{the same speed and longer wavelengths.}
        \wrongchoice{the same speed and the same wavelengths.}
        \wrongchoice{a lower speed and shorter wavelengths.}
        \wrongchoice{a lower speed and longer wavelengths.}
    \end{choices}
\end{question}
}

\element{jpierce}{
\begin{question}{pt101tb5-Q90}
    Compared to visible light traveling in a vacuum,
        ultraviolet rays would have
    \begin{choices}
        \wrongchoice{the same speed and longer wavelengths.}
        \wrongchoice{a lower speed and longer wavelengths.}
      \correctchoice{the same speed and shorter wavelengths.}
        \wrongchoice{a lower speed and shorter wavelengths.}
        \wrongchoice{the same speed and the same wavelengths.}
    \end{choices}
\end{question}
}

%% Topic: Radiation Transfer
\element{jpierce}{
\begin{question}{pt101tb5-Q91}
    A material is said to be opaque if
    \begin{choices}
        \wrongchoice{light can pass freely through it in a straight line.}
        \wrongchoice{it absorbs light and then re-emits it.}
        \wrongchoice{it cannot absorb any light.}
      \correctchoice{it absorbs light and redistributes the energy as thermal energy.}
        \wrongchoice{it cannot vibrate at a resonant frequency to match the frequency of the light.}
    \end{choices}
\end{question}
}

\element{jpierce}{
\begin{question}{pt101tb5-Q92}
    A material is said to be transparent if
    \begin{choices}
        \wrongchoice{it cannot emit any light.}
        \wrongchoice{it reflects light.}
        \wrongchoice{it absorbs light and redistributes the energy as thermal energy.}
        \wrongchoice{it can vibrate at a resonant frequency to match the frequency of the light.}
      \correctchoice{light can pass freely through it in a straight line.}
    \end{choices}
\end{question}
}

\element{jpierce}{
\begin{question}{pt101tb5-Q93}
    Glass is:
    \begin{choices}
        \wrongchoice{opaque to visible light and infrared but transparent to ultraviolet.}
        \wrongchoice{opaque to visible light and ultraviolet but transparent to infrared.}
        \wrongchoice{opaque to ultraviolet but transparent to visible light and infrared.}
      \correctchoice{opaque to ultraviolet and infrared but transparent to visible light.}
        \wrongchoice{opaque to visible light but transparent to ultraviolet and infrared.}
    \end{choices}
\end{question}
}

\element{jpierce}{
\begin{question}{pt101tb5-Q94}
    Glass is:
    \begin{choices}
      \correctchoice{transparent to visible light but opaque to ultraviolet and infrared.}
        \wrongchoice{transparent to ultraviolet but opaque to visible light and infrared.}
        \wrongchoice{transparent to ultraviolet and infrared but opaque to visible light.}
        \wrongchoice{transparent to visible light and ultraviolet but opaque to infrared.}
        \wrongchoice{transparent to visible light and infrared but opaque to ultraviolet.}
    \end{choices}
\end{question}
}

%% Topic: Image
\element{jpierce}{
\begin{question}{pt101tb5-Q95}
    Magnification of an object by a lens can be calculated as the ratio
        of the image distance to the object distance. 
    What are the units of magnification?
    \begin{choices}
        \wrongchoice{centimeters (\si{\centi\meter})}
      \correctchoice{none}
        \wrongchoice{inverse centimeters squared (\si{\per\centi\meter\squared})}
        \wrongchoice{inverse centimeters (\si{\per\centi\meter})}
        \wrongchoice{centimeters squared (\si{\centi\meter\squared})}
    \end{choices}
\end{question}
}

\element{jpierce}{
\begin{question}{pt101tb5-Q96}
    Which of the following is true?
    \begin{choices}
        \wrongchoice{The image seen in a plane mirror is a real image.}
        \wrongchoice{Only virtual images can be projected on a screen.}
        \wrongchoice{A light ray passing through the center of a converging lens will be bent to pass through the focus.}
        \wrongchoice{A virtual image is always upside down.}
      \correctchoice{A real image is formed where the rays from an object meet after passing through a lens.}
    \end{choices}
\end{question}
}

\element{jpierce}{
\begin{question}{pt101tb5-Q97}
    Which of the following is true?
    \begin{choices}
      \correctchoice{Only real images can be projected on a screen.}
        \wrongchoice{A virtual image is formed where the rays from an object meet after passing through a lens.}
        \wrongchoice{The image seen in a plane mirror is a real image.}
        \wrongchoice{A virtual image is always upside down.}
        \wrongchoice{A light ray passing through the center of a converging lens will be bent to pass through the focus.}
    \end{choices}
\end{question}
}

\element{jpierce}{
\begin{question}{pt101tb5-Q98}
    Which of the following is true?
    \begin{choices}
        \wrongchoice{A light ray passing through the center of a converging lens will be bent to pass through the focus.}
        \wrongchoice{A virtual image is always upside down.}
        \wrongchoice{A virtual image is formed where the rays from an object meet after passing through a lens.}
        \wrongchoice{Only virtual images can be projected on a screen.}
      \correctchoice{The image seen in a plane mirror is a virtual image.}
    \end{choices}
\end{question}
}

\element{jpierce}{
\begin{question}{pt101tb5-Q99}
    Which of the following is true?
    \begin{choices}
        \wrongchoice{A virtual image is formed where the rays from an object meet after passing through a lens.}
        \wrongchoice{A virtual image is always upside down.}
        \wrongchoice{The image seen in a plane mirror is a real image.}
        \wrongchoice{Only virtual images can be projected on a screen.}
      \correctchoice{A light ray passing through the center of a converging lens will not be bent.}
    \end{choices}
\end{question}
}

\element{jpierce}{
\begin{question}{pt101tb5-Q100}
    A converging lens is used to produce a real image of an object,
        with the image and object being equal in size. 
    As the object is moved farther from the lens,
        what will happen to the image?
    \begin{choices}
        \wrongchoice{The image will turn from upright to inverted.}
        \wrongchoice{The image will move farther from the lens.}
        \wrongchoice{The image will turn from inverted to upright.}
      \correctchoice{The image will become smaller.}
        \wrongchoice{The image will become larger.}
    \end{choices}
\end{question}
}

%% Topic: Lenses
\element{jpierce}{
\begin{question}{pt101tb5-Q101}
    A converging lens is:
    \begin{choices}
        \wrongchoice{uniform in thickness across its surface.}
      \correctchoice{thicker in the middle than at the edges.}
        \wrongchoice{thicker at the edges and in the middle but thinner in between.}
        \wrongchoice{thicker at the edges than in the middle}
        \wrongchoice{thinner at the edges and in the middle but thicker in between.}
    \end{choices}
\end{question}
}

\element{jpierce}{
\begin{question}{pt101tb5-Q102}
    A diverging lens is:
    \begin{choices}
      \correctchoice{thicker at the edges than in the middle}
        \wrongchoice{thicker in the middle than at the edges.}
        \wrongchoice{thinner at the edges and in the middle but thicker in between.}
        \wrongchoice{uniform in thickness across its surface.}
        \wrongchoice{thicker at the edges and in the middle but thinner in between.}
    \end{choices}
\end{question}
}

\element{jpierce}{
\begin{question}{pt101tb5-Q103}
    A converging lens is:
    \begin{choices}
        \wrongchoice{thinner in the middle than at the edges.}
        \wrongchoice{uniform in thickness across its surface.}
        \wrongchoice{thicker at the edges and in the middle but thinner in between.}
      \correctchoice{thinner at the edges than in the middle}
        \wrongchoice{thinner at the edges and in the middle but thicker in between.}
    \end{choices}
\end{question}
}

\element{jpierce}{
\begin{question}{pt101tb5-Q104}
    A diverging lens is:
    \begin{choices}
        \wrongchoice{thinner at the edges and in the middle but thicker in between.}
        \wrongchoice{uniform in thickness across its surface.}
        \wrongchoice{thicker at the edges and in the middle but thinner in between.}
      \correctchoice{thinner in the middle than at the edges.}
        \wrongchoice{thinner at the edges than in the middle}
    \end{choices}
\end{question}
}

\element{jpierce}{
\begin{question}{pt101tb5-Q105}
    A diverging lens produces:
    \begin{choices}
        \wrongchoice{a diminished, real image.}
      \correctchoice{a diminished, virtual image.}
        \wrongchoice{an enlarged, real image.}
        \wrongchoice{an enlarged, virtual image.}
        \wrongchoice{a lifesized imaginary image.}
    \end{choices}
\end{question}
}

\element{jpierce}{
\begin{question}{pt101tb5-Q106}
    When a converging lens is used as a magnifying glass,
        the image produced is:
    \begin{choices}
        \wrongchoice{real and inverted.}
      \correctchoice{virtual and upright.}
        \wrongchoice{virtual and inverted.}
        \wrongchoice{real and upright.}
        \wrongchoice{none of the above---a converging lens cannot be used as a magnifying glass.}
    \end{choices}
\end{question}
}

\element{jpierce}{
\begin{question}{pt101tb5-Q107}
    When you use a converging lens as a magnifying glass,
        the object must be placed:
    \begin{choices}
        \wrongchoice{behind you.}
        \wrongchoice{between you and the lens.}
        \wrongchoice{beyond the focal point.}
        \wrongchoice{at the focal point.}
      \correctchoice{inside the focal point.}
    \end{choices}
\end{question}
}

\element{jpierce}{
\begin{question}{pt101tb5-Q108}
    An object placed outside the focal point of a converging
        lens will form an image that is always
    \begin{choices}
        \wrongchoice{imaginary and inside out.}
        \wrongchoice{real and upright.}
        \wrongchoice{virtual and inverted.}
      \correctchoice{real and inverted.}
        \wrongchoice{virtual and upright.}
    \end{choices}
\end{question}
}

\element{jpierce}{
\begin{question}{pt101tb5-Q109}
    Which of the following is true? 
    When used alone,
    \begin{choices}
        \wrongchoice{diverging lenses can form only real images.}
        \wrongchoice{converging lenses can form only inverted images.}
        \wrongchoice{converging lenses can form only virtual images.}
      \correctchoice{diverging lenses can form only virtual images.}
        \wrongchoice{converging lenses can form only real images.}
    \end{choices}
\end{question}
}

\element{jpierce}{
\begin{question}{pt101tb5-Q110}
    Which of the following is true? 
    When used alone,
    \begin{choices}
        \wrongchoice{converging lenses can form only virtual images.}
        \wrongchoice{converging lenses can form only upright images.}
        \wrongchoice{converging lenses can form only real images.}
        \wrongchoice{diverging lenses can form only inverted images.}
      \correctchoice{diverging lenses can form only upright images.}
    \end{choices}
\end{question}
}

\element{jpierce}{
\begin{question}{pt101tb5-Q111}
    Which of the following is not true? 
    When used alone,
    \begin{choices}
        \wrongchoice{converging lenses can form virtual images.}
        \wrongchoice{converging lenses can form inverted images.}
        \wrongchoice{diverging lenses can form virtual images.}
        \wrongchoice{converging lenses can form real images.}
      \correctchoice{diverging lenses can form real images.}
    \end{choices}
\end{question}
}

%% Topic: Rainbow
\element{jpierce}{
\begin{question}{pt101tb5-Q112}
    Dispersion of light in spherical drops of water produces the phenomenon we know as
    \begin{choices}
        \wrongchoice{diffuse reflection.}
      \correctchoice{a rainbow.}
        \wrongchoice{total internal reflection.}
        \wrongchoice{a mirage.}
        \wrongchoice{a virtual image.}
    \end{choices}
\end{question}
}

\element{jpierce}{
\begin{question}{pt101tb5-Q113}
    The colors seen in a rainbow
    \begin{choices}
      \correctchoice{are produced when sunlight is refracted by raindrops.}
        \wrongchoice{are the colors of the different molecules that make up water.}
        \wrongchoice{are produced by raindrops of different shapes.}
        \wrongchoice{are produced by raindrops of different colors.}
        \wrongchoice{are the colors of the different atoms that make up water.}
    \end{choices}
\end{question}
}

\element{jpierce}{
\begin{question}{pt101tb5-Q114}
    The order of colors seen in a primary rainbow is:
    \begin{choices}
        \wrongchoice{green (outside) to red (inside).}
        \wrongchoice{violet (outside) to yellow (inside).}
        \wrongchoice{completely random and therefore unpredictable.}
        \wrongchoice{blue (outside) to orange (inside).}
      \correctchoice{red (outside) to violet (inside).}
    \end{choices}
\end{question}
}

\element{jpierce}{
\begin{question}{pt101tb5-Q115}
    The order of colors seen in a secondary rainbow is:
    \begin{choices}
        \wrongchoice{completely random and therefore unpredictable.}
        \wrongchoice{green (outside) to red (inside).}
        \wrongchoice{red (outside) to violet (inside).}
      \correctchoice{violet (outside) to yellow (inside).}
        \wrongchoice{blue (outside) to orange (inside).}
    \end{choices}
\end{question}
}

\element{jpierce}{
\begin{question}{pt101tb5-Q116}
    When you view a rainbow,
        the Sun will normally be
    \begin{choices}
        \wrongchoice{in front of you.}
        \wrongchoice{anywhere in the sky---there is no special position required.}
      \correctchoice{behind you.}
        \wrongchoice{below the horizon.}
        \wrongchoice{straight overhead.}
    \end{choices}
\end{question}
}

%% Topic: Reflection
\element{jpierce}{
\begin{question}{pt101tb5-Q117}
    The law of reflection says:
    \begin{choices}
        \wrongchoice{the angle of reflection equals the angle of refraction.}
      \correctchoice{the angle of reflection equals the angle of incidence.}
        \wrongchoice{all reflected rays are parallel to the incident ray.}
        \wrongchoice{all reflected rays are perpendicular to the incident ray.}
        \wrongchoice{all reflected rays are parallel to each other.}
    \end{choices}
\end{question}
}

\element{jpierce}{
\begin{question}{pt101tb5-Q118}
    A plane mirror produces an image that is:
    \begin{choices}
        \wrongchoice{virtual and enlarged.}
      \correctchoice{virtual and upright.}
        \wrongchoice{virtual and inverted.}
        \wrongchoice{real and inverted.}
        \wrongchoice{real and upright.}
    \end{choices}
\end{question}
}

\element{jpierce}{
\begin{question}{pt101tb5-Q119}
    The image you see in the mirror on your bathroom wall is:
    \begin{choices}
      \correctchoice{virtual and upright.}
        \wrongchoice{virtual and inverted.}
        \wrongchoice{real and inverted.}
        \wrongchoice{real and upright.}
        \wrongchoice{virtual and enlarged.}
    \end{choices}
\end{question}
}

\element{jpierce}{
\begin{question}{pt101tb5-Q120}
    Diffuse reflection occurs when:
    \begin{choices}
        \wrongchoice{the reflecting surface is extremely smooth compared to the wavelength of incident light.}
        \wrongchoice{incident light rays interfere with each other.}
        \wrongchoice{reflected light rays interfere with each other.}
      \correctchoice{the reflecting surface is rough compared to the wavelength of incident light.}
        \wrongchoice{reflected light rays interfere with incident light rays.}
    \end{choices}
\end{question}
}

%% Topic: Refraction
\element{jpierce}{
\begin{question}{pt101tb5-Q121}
    Refraction refers to:
    \begin{choices}
      \correctchoice{the bending of light rays as they pass from one medium into another.}
        \wrongchoice{the reversal of left and right as seen in a mirror image.}
        \wrongchoice{the interference between two light rays of slightly different frequencies.}
        \wrongchoice{the apparent reduction in the size of an object as seen through a lens.}
        \wrongchoice{the bouncing of light rays off the boundary between two different media.}
    \end{choices}
\end{question}
}

\element{jpierce}{
\begin{question}{pt101tb5-Q122}
    Fermat's principle of least time:
    \begin{choices}
      \correctchoice{applies to both reflection and refraction.}
        \wrongchoice{applies to reflection but not to refraction.}
        \wrongchoice{applies to neither reflection nor refraction.}
        \wrongchoice{applies to refraction but not to reflection.}
        \wrongchoice{is no longer believed to be true.}
    \end{choices}
\end{question}
}

\element{jpierce}{
\begin{question}{pt101tb5-Q123}
    Which of these phenomena is not attributed to refraction?
    \begin{choices}
        \wrongchoice{An object submerged in water appears to be closer to the surface than it really is.}
      \correctchoice{Objects appear to us to be different colors.}
        \wrongchoice{An object submerged in water appears to be magnified.}
        \wrongchoice{The setting sun appears to be just above the horizon moments after it has actually set.}
        \wrongchoice{The highway ahead appears to be wet, but it is only a mirage.}
    \end{choices}
\end{question}
}

\element{jpierce}{
\begin{question}{pt101tb5-Q124}
    A light ray passing from air into water at an angle of \ang{30} from the normal in air:
    \begin{choices}
        \wrongchoice{would make an angle greater than \ang{30} from the normal in water.}
        \wrongchoice{would be completely absorbed by the water surface.}
        \wrongchoice{would be completely reflected by the water surface.}
        \wrongchoice{would make an angle of \ang{30} from the normal in water.}
      \correctchoice{would make an angle less than \ang{30} from the normal in water.}
    \end{choices}
\end{question}
}

\element{jpierce}{
\begin{question}{pt101tb5-Q125}
    A light ray passing from air into water at an angle of \ang{20} from the normal in air:
    \begin{choices}
        \wrongchoice{would make an angle greater than \ang{20} from the normal in water.}
        \wrongchoice{would be completely absorbed by the water surface.}
        \wrongchoice{would make an angle of \ang{20} from the normal in water.}
        \wrongchoice{would be completely reflected by the water surface.}
      \correctchoice{would make an angle less than \ang{20} from the normal in water.}
    \end{choices}
\end{question}
}

\element{jpierce}{
\begin{question}{pt101tb5-Q126}
    A light ray passing from air into water at an angle of \ang{15} from the normal in air:
    \begin{choices}
        \wrongchoice{would be completely reflected by the water surface.}
      \correctchoice{would make an angle less than \ang{15} from the normal in water.}
        \wrongchoice{would make an angle greater than \ang{15} from the normal in water.}
        \wrongchoice{would make an angle of \ang{15} from the normal in water.}
        \wrongchoice{would be completely absorbed by the water surface.}
    \end{choices}
\end{question}
}

\element{jpierce}{
\begin{question}{pt101tb5-Q127}
    A light ray passing from water into air at an angle of \ang{20} from the normal in water:
    \begin{choices}
        \wrongchoice{would be completely absorbed by the water surface.}
        \wrongchoice{would make an angle less than \ang{20} from the normal in air.}
        \wrongchoice{would make an angle of \ang{20} from the normal in air.}
        \wrongchoice{would be completely reflected by the water surface.}
      \correctchoice{would make an angle greater than \ang{20} from the normal in air.}
    \end{choices}
\end{question}
}

\element{jpierce}{
\begin{question}{pt101tb5-Q128}
    A light ray passing from water into air at an angle of \ang{30} from the normal in water:
    \begin{choices}
        \wrongchoice{would be completely reflected by the water surface.}
        \wrongchoice{would make an angle of \ang{30} from the normal in air.}
      \correctchoice{would make an angle greater than \ang{30} from the normal in air.}
        \wrongchoice{would make an angle less than \ang{30} from the normal in air.}
        \wrongchoice{would be completely absorbed by the water surface.}
    \end{choices}
\end{question}
}

\element{jpierce}{
\begin{question}{pt101tb5-Q129}
    A light ray passing from water into air at an angle of \ang{15} from the normal in water:
    \begin{choices}
        \wrongchoice{would make an angle of \ang{15} from the normal in air.}
        \wrongchoice{would make an angle less than \ang{15} from the normal in air.}
      \correctchoice{would make an angle greater than \ang{15} from the normal in air.}
        \wrongchoice{would be completely reflected by the water surface.}
        \wrongchoice{would be completely absorbed by the water surface.}
    \end{choices}
\end{question}
}

%% Topic: Total Internal Reflection
\element{jpierce}{
\begin{question}{pt101tb5-Q130}
    Total internal reflection occurs when:
    \begin{choices}
        \wrongchoice{the angle of incidence is less than the critical angle.}
        \wrongchoice{the angle of refraction exceeds the critical angle.}
        \wrongchoice{the angle of reflection is less than the critical angle.}
      \correctchoice{the angle of incidence exceeds the critical angle.}
        \wrongchoice{the angle of reflection is equal to the angle of incidence.}
    \end{choices}
\end{question}
}

\element{jpierce}{
\begin{question}{pt101tb5-Q131}
    A practical application of total internal reflection is found in:
    \begin{choices}
        \wrongchoice{rainbows.}
        \wrongchoice{diverging lenses.}
      \correctchoice{fiber optics.}
        \wrongchoice{converging lenses.}
        \wrongchoice{bathroom mirrors.}
    \end{choices}
\end{question}
}

\element{jpierce}{
\begin{question}{pt101tb5-Q132}
    The critical angle is:
    \begin{choices}
        \wrongchoice{the angular radius of the arc of a rainbow.}
        \wrongchoice{the angle of incidence for which the angle of the refracted beam is the same.}
        \wrongchoice{the angle of incidence for which the angle of the reflected beam is the same.}
        \wrongchoice{the angle of incidence for which the angle of the refracted beam is \ang{0}.}
      \correctchoice{the angle of incidence for which the angle of the refracted beam is \ang{90}.}
    \end{choices}
\end{question}
}

\element{jpierce}{
\begin{question}{pt101tb5-Q133}
    The critical angle for the water-air boundary is \ang{48}. 
    This means that a beam of light incident on the boundary
        from the water side at an angle from the normal of \ang{50}:
    \begin{choices}
        \wrongchoice{will emerge in the air at an angle of \ang{82} from the normal.}
      \correctchoice{will undergo total internal reflection.}
        \wrongchoice{will emerge in the air at an angle of \ang{50} from the normal.}
        \wrongchoice{will emerge in the air at an angle of \ang{2} from the normal.}
        \wrongchoice{will emerge in the air at an angle of \ang{98} from the normal.}
    \end{choices}
\end{question}
}

%% Topic: Graph
\element{jpierce}{
\begin{question}{pt101tb5-Q134}
    If $T$ is directly proportional to $L$, then a plot of $T$ vs $L$ should be:
    \begin{choices}
        \wrongchoice{a circle.}
        \wrongchoice{a curve that is concave upward.}
        \wrongchoice{a curve that is concave downward.}
        \wrongchoice{a parabola.}
      \correctchoice{a straight line passing through the origin.}
    \end{choices}
\end{question}
}

%% Topic: Meter
\element{jpierce}{
\begin{question}{pt101tb5-Q135}
    One centimeter (\SI{1}{\centi\meter}) equals:
    \begin{multicols}{3}
    \begin{choices}
        \wrongchoice{\SI{0.001}{\meter}}
        \wrongchoice{\SI{0.1}{\meter}}
      \correctchoice{\SI{0.01}{\meter}}
        \wrongchoice{\SI{10}{\meter}}
        \wrongchoice{\SI{100}{\meter}}
    \end{choices}
    \end{multicols}
\end{question}
}

\element{jpierce}{
\begin{question}{pt101tb5-Q136}
    One meter (\SI{1}{\meter}) equals:
    \begin{multicols}{3}
    \begin{choices}
        \wrongchoice{\SI{0.001}{\centi\meter}}
      \correctchoice{\SI{100}{\centi\meter}}
        \wrongchoice{\SI{0.1}{\centi\meter}}
        \wrongchoice{\SI{10}{\centi\meter}}
        \wrongchoice{\SI{0.01}{\centi\meter}}
    \end{choices}
    \end{multicols}
\end{question}
}

\element{jpierce}{
\begin{question}{pt101tb5-Q137}
    How long is a meter stick?
    \begin{multicols}{3}
    \begin{choices}
        \wrongchoice{\SI{10}{\centi\meter}}
        \wrongchoice{\SI{100}{\milli\meter}}
      \correctchoice{\SI{100}{\centi\meter}}
        \wrongchoice{\SI{1000}{\centi\meter}}
        \wrongchoice{\SI{36}{\inch}}
    \end{choices}
    \end{multicols}
\end{question}
}

\element{jpierce}{
\begin{question}{pt101tb5-Q138}
    How long is a meter stick?
    \begin{multicols}{3}
    \begin{choices}
        \wrongchoice{\SI{100}{\milli\meter}}
        \wrongchoice{\SI{10}{\milli\meter}}
      \correctchoice{\SI{1000}{\milli\meter}}
        \wrongchoice{\SI{36}{\inch}}
        \wrongchoice{\SI{10}{\centi\meter}}
    \end{choices}
    \end{multicols}
\end{question}
}

\element{jpierce}{
\begin{question}{pt101tb5-Q139}
    One meter (\SI{1}{\meter}) is equal to:
    \begin{multicols}{3}
    \begin{choices}
        \wrongchoice{\SI{0.01}{\milli\meter}}
        \wrongchoice{\SI{10}{\milli\meter}}
        \wrongchoice{\SI{100}{\milli\meter}}
      \correctchoice{\SI{1000}{\milli\meter}}
        \wrongchoice{\SI{0.001}{\milli\meter}}
    \end{choices}
    \end{multicols}
\end{question}
}

\element{jpierce}{
\begin{question}{pt101tb5-Q140}
    One centimeter (\SI{1}{\centi\meter}) is equal to:
    \begin{multicols}{3}
    \begin{choices}
      \correctchoice{\SI{10}{\milli\meter}}
        \wrongchoice{\SI{0.1}{\milli\meter}}
        \wrongchoice{\SI{1000}{\milli\meter}}
        \wrongchoice{\SI{100}{\milli\meter}}
        \wrongchoice{\SI{0.01}{\milli\meter}}
    \end{choices}
    \end{multicols}
\end{question}
}

\element{jpierce}{
\begin{question}{pt101tb5-Q141}
    One millimeter (\SI{1}{\milli\meter}) is equal to:
    \begin{multicols}{3}
    \begin{choices}
        \wrongchoice{\SI{100}{\centi\meter}}
      \correctchoice{\SI{0.1}{\centi\meter}}
        \wrongchoice{\SI{0.01}{\centi\meter}}
        \wrongchoice{\SI{10}{\centi\meter}}
        \wrongchoice{\SI{0.001}{\centi\meter}}
    \end{choices}
    \end{multicols}
\end{question}
}

\element{jpierce}{
\begin{question}{pt101tb5-Q142}
    One millimeter (\SI{1}{\milli\meter}) is equal to:
    \begin{multicols}{3}
    \begin{choices}
        \wrongchoice{\SI{1000}{\meter}}
      \correctchoice{\SI{0.001}{\meter}}
        \wrongchoice{\SI{100}{\meter}}
        \wrongchoice{\SI{10}{\meter}}
        \wrongchoice{\SI{0.01}{\meter}}
    \end{choices}
    \end{multicols}
\end{question}
}

%% Topic: Percent
\element{jpierce}{
\begin{question}{pt101tb5-Q143}
    In the laboratory, the speed of sound is measured to be \SI{344}{\meter\per\second},
        different from the actual value of \SI{343}{\meter\per\second}. 
    What is the percent error in the measurement?
    \begin{multicols}{3}
    \begin{choices}
        \wrongchoice{\SI{0.1}{\percent}}
      \correctchoice{\SI{0.3}{\percent}}
        \wrongchoice{\SI{10}{\percent}}
        \wrongchoice{\SI{1}{\percent}}
        \wrongchoice{\SI{3}{\percent}}
    \end{choices}
    \end{multicols}
\end{question}
}


\endinput


