

%% Physics 101 Sample Test Questions by Dr. James Pierce
%%------------------------------------------------------------


%% Exam #3 ---- (atoms, solids, liquids, gases, etc.)
%%------------------------------------------------------------
\element{cpo-mc}{
\begin{question}{exam03-Q48}
    Protons have \rule[-0.1pt]{4em}{0.1pt} charge,
        neutrons have \rule[-0.1pt]{4em}{0.1pt} charge,
        and electrons have \rule[-0.1pt]{4em}{0.1pt} charge.
    \begin{choices}
        \wrongchoice{negative; positive; no}
      \correctchoice{positive; no; negative}
        \wrongchoice{positive; negative; no}
        \wrongchoice{negative; no; positive}
        \wrongchoice{no; negative; positive}
    \end{choices}
\end{question}
}

\element{cpo-mc}{
\begin{question}{exam03-Q49}
    Which of the following is true?
    \begin{choices}
        \wrongchoice{Some atoms do not belong to any particular element.}
        \wrongchoice{Some atoms belong to more than one element.}
        \wrongchoice{All atoms are identical.}
      \correctchoice{The number of protons in an atom determines which element it is.}
        \wrongchoice{The number of neutrons in an atom determines which element it is.}
    \end{choices}
\end{question}
}

\element{cpo-mc}{
\begin{question}{exam03-Q50}
    The mass of one hydrogen atom is approximately
    \begin{choices}
      \correctchoice{one atomic mass unit.}
        \wrongchoice{two atomic mass units.}
        \wrongchoice{\num{12} atomic mass units.}
        \wrongchoice{\num{16} atomic mass units.}
        \wrongchoice{\num{1/2} atomic mass unit.}
    \end{choices}
\end{question}
}

\element{cpo-mc}{
\begin{question}{exam03-Q51}
    An element with an atomic number of 92 and an atomic mass number of 238 would have
    \begin{choices}
      \correctchoice{92 protons, 146 neutrons, and 92 electrons.}
        \wrongchoice{92 protons, 146 neutrons, and 238 electrons.}
        \wrongchoice{92 protons, 238 neutrons, and 146 electrons.}
        \wrongchoice{146 protons, 92 neutrons, and 92 electrons.}
        \wrongchoice{146 protons, 92 neutrons, and 146 electrons.}
    \end{choices}
\end{question}
}

\element{cpo-mc}{
\begin{question}{exam03-Q52}
    Brownian motion is the
    \begin{choices}
      \correctchoice{random motion of microscopic particles being bombarded by even smaller atoms and molecules.}
        \wrongchoice{random motion of atoms and molecules being bombarded by larger microscopic particles.}
        \wrongchoice{vibration of atoms and molecules in a solid.}
        \wrongchoice{movement of electrons circulating within the atom.}
        \wrongchoice{very gradual flow of solid materials such as glass over long periods of time.}
    \end{choices}
\end{question}
}

\element{cpo-mc}{
\begin{question}{exam03-Q53}
    Chemical combinations of elements are called
    \begin{choices}
        \wrongchoice{mixtures.}
        \wrongchoice{groups.}
        \wrongchoice{shells.}
        \wrongchoice{nuclei.}
      \correctchoice{compounds.}
    \end{choices}
\end{question}
}

\element{cpo-mc}{
\begin{question}{exam03-Q54}
    Which of the following is a list of elements?
    \begin{choices}
        \wrongchoice{hydrogen, nitrogen, air}
        \wrongchoice{hydrogen, oxygen, water}
      \correctchoice{hydrogen, oxygen, nitrogen}
        \wrongchoice{air, nitrogen, oxygen}
        \wrongchoice{water, nitrogen, oxygen}
    \end{choices}
\end{question}
}

\element{cpo-mc}{
\begin{question}{exam03-Q55}
    Where on the periodic table would we find an element with
        one more proton and one more electron than silver?
    \begin{choices}
        \wrongchoice{Just above silver.}
        \wrongchoice{Just to the left of silver.}
        \wrongchoice{Just below silver.}
      \correctchoice{Just to the right of silver.}
        \wrongchoice{None of these---there is no such element.}
    \end{choices}
\end{question}
}

\element{cpo-mc}{
\begin{question}{exam03-Q56}
    Density is
    \begin{choices}
        \wrongchoice{mass times volume.}
      \correctchoice{mass divided by volume.}
        \wrongchoice{mass plus volume.}
        \wrongchoice{volume divided by mass.}
        \wrongchoice{mass minus volume.}
    \end{choices}
\end{question}
}

\element{cpo-mc}{
\begin{question}{exam03-Q57}
    \SI{1000}{\centi\meter\cubed} of water should have a mass of approximately:
    \begin{multicols}{2}
    \begin{choices}
        \wrongchoice{\SI{100}{\gram}}
        \wrongchoice{\SI{10}{\grams}}
        \wrongchoice{\SI{1}{\gram}}
      \correctchoice{\SI{1}{\kilo\gram}}
        \wrongchoice{\SI{1000}{\kilo\grams}}
    \end{choices}
    \end{multicols}
\end{question}
}

\element{cpo-mc}{
\begin{question}{exam03-Q58}
    A material is said to be \rule[-0.1pt]{4em}{0.1pt} if it changes shape when
        a deforming force acts on it and returns to its original shape when
        the deforming force is removed.
    \begin{multicols}{2}
    \begin{choices}
      \correctchoice{elastic}
        \wrongchoice{inelastic}
        \wrongchoice{plastic}
        \wrongchoice{stretchy}
        \wrongchoice{rigid}
    \end{choices}
    \end{multicols}
\end{question}
}

\element{cpo-mc}{
\begin{question}{exam03-Q59}
    Hooke's Law relates the
    \begin{choices}
      \correctchoice{distance a spring stretches to the force applied to the spring.}
        \wrongchoice{distance a spring stretches to the mass of the spring.}
        \wrongchoice{distance a spring stretches to the density of the spring.}
        \wrongchoice{density of a spring to the force applied to the spring.}
        \wrongchoice{density of a spring to the mass of the spring.}
    \end{choices}
\end{question}
}

\element{cpo-mc}{
\begin{question}{exam03-Q60}
    When the length of each edge of a cube is doubled,
        the cube's surface area increases by a factor of:
    \begin{multicols}{2}
    \begin{choices}
        \wrongchoice{\num{2}}
      \correctchoice{\num{4}}
        \wrongchoice{\num{6}}
        \wrongchoice{\num{8}}
        \wrongchoice{\num{16}}
    \end{choices}
    \end{multicols}
\end{question}
}

\element{cpo-mc}{
\begin{question}{exam03-Q61}
    When the length of each edge of a cube is tripled, the cube's volume
    \begin{choices}
        \wrongchoice{increases by a factor of \num{3}.}
        \wrongchoice{decreases by a factor of \num{1/3}.}
        \wrongchoice{increases by a factor of \num{9}.}
        \wrongchoice{decreases by a factor of \num{1/9}.}
      \correctchoice{increases by a factor of \num{27}.}
    \end{choices}
\end{question}
}

\element{cpo-mc}{
\begin{question}{exam03-Q62}
    The weight of a dome produces
    \begin{choices}
        \wrongchoice{tension forces parallel to the curve of the dome.}
      \correctchoice{compression forces parallel to the curve of the dome.}
        \wrongchoice{compression forces perpendicular to the curve of the dome.}
        \wrongchoice{tension forces acting vertically.}
        \wrongchoice{tension forces acting horizontally.}
    \end{choices}
\end{question}
}

\element{cpo-mc}{
\begin{question}{exam03-Q63}
    The buoyant force
    \begin{choices}
        \wrongchoice{is the force of gravity acting on a submerged object.}
        \wrongchoice{is the difference between a submerged object's weight and the weight of an equal mass of water.}
      \correctchoice{is the net upward force of the surrounding liquid acting on a submerged object.}
        \wrongchoice{is the net downward force of a submerged object acting on the surrounding liquid.}
        \wrongchoice{depends on the density of the submerged object.}
    \end{choices}
\end{question}
}

\element{cpo-mc}{
\begin{question}{exam03-Q64}
    The buoyant force on a block of wood floating in water
    \begin{choices}
        \wrongchoice{is equal to the weight of a volume of water with the same volume as the wood.}
      \correctchoice{is equal to the weight of the wood.}
        \wrongchoice{is greater than the weight of the wood.}
        \wrongchoice{is less than the weight of the wood.}
        \wrongchoice{cannot be calculated because the block is not completely submerged.}
    \end{choices}
\end{question}
}

\element{cpo-mc}{
\begin{question}{exam03-Q65}
    An object with a mass of \SI{1}{\kilo\gram} displaces \SI{700}{\milli\liter} of water. 
    Which of the following is true?
    \begin{choices}
      \correctchoice{The weight of this object is \SI{10}{\newton}.} 
        \wrongchoice{The weight of this object is \SI{7}{\newton}.} 
        \wrongchoice{The weight of this object is \SI{3}{\newton}.} 
        \wrongchoice{The buoyant force on this object is \SI{3}{\newton}.} 
        \wrongchoice{The buoyant force on this object is \SI{10}{\newton}.} 
    \end{choices}
\end{question}
}

\element{cpo-mc}{
\begin{question}{exam03-Q66}
    An object with a mass of \SI{1}{\kilo\gram} displaces \SI{0.6}{\kilo\gram} of water. 
    Which of the following is true?
    \begin{choices}
        \wrongchoice{The buoyant force on this object is \SI{10}{\newton}.}
      \correctchoice{The buoyant force on this object is \SI{6}{\newton}.}
        \wrongchoice{The buoyant force on this object is \SI{4}{\newton}.}
        \wrongchoice{The density of this object is less than that of water.}
        \wrongchoice{This object will not sink in water.}
    \end{choices}
\end{question}
}

\element{cpo-mc}{
\begin{question}{exam03-Q67}
    The water pressure in a lake behind a dam depends on
    \begin{choices}
        \wrongchoice{the volume of lake water behind the dam.}
        \wrongchoice{the surface area of the lake.}
        \wrongchoice{the distance from the dam at which the pressure is measured.}
      \correctchoice{the depth below the surface at which the pressure is measured.}
        \wrongchoice{the number of fish in the lake.}
    \end{choices}
\end{question}
}

\element{cpo-mc}{
\begin{question}{exam03-Q68}
    When air is removed from a metal can by a vacuum pump,
        the can buckles inwards and is crushed.
    This occurs because
    \begin{choices}
        \wrongchoice{the air pressure on the inside of the can is greater than the air pressure on the outside of the can.}
      \correctchoice{the air pressure on the outside of the can is greater than the air pressure on the inside of the can.}
        \wrongchoice{the loss of air molecules from inside the can weakens the metal.}
        \wrongchoice{the opposite sides of the empty can strongly attract each other.}
        \wrongchoice{of Bernoulli's principle.}
    \end{choices}
\end{question}
}

\element{cpo-mc}{
\begin{question}{exam03-Q69}
    Bernoulli's principle explains why
    \begin{choices}
        \wrongchoice{a hot air balloon rises.}
        \wrongchoice{liquid rises in a drinking straw.}
      \correctchoice{airplanes fly.}
        \wrongchoice{dead fish float.}
        \wrongchoice{submarines can remain submerged.}
    \end{choices}
\end{question}
}

\element{cpo-mc}{
\begin{question}{exam03-Q70}
    In order to decrease the pressure in an automobile tire,
        one normally
    \begin{choices}
        \wrongchoice{decreases the temperature of the tire.}
        \wrongchoice{increases the volume of the tire.}
        \wrongchoice{increases the density of air in the tire.}
      \correctchoice{decreases the number of air molecules in the tire.}
        \wrongchoice{decreases the surface area of the tire.}
    \end{choices}
\end{question}
}

\element{cpo-mc}{
\begin{question}{exam03-Q71}
    According to Boyle's Law,
        if the volume occupied by a certain gas is doubled,
    \begin{choices}
        \wrongchoice{the pressure of the gas will be doubled.}
        \wrongchoice{the pressure of the gas will be quadrupled.}
        \wrongchoice{the pressure of the gas will remain constant.}
      \correctchoice{the pressure of the gas will be halved.}
        \wrongchoice{the number of atoms in the gas will be halved.}
    \end{choices}
\end{question}
}

\element{cpo-mc}{
\begin{question}{exam03-Q72}
    Archimedes' Principle states that an object surrounded by air
        is buoyed up by a force equal to the
    \begin{choices}
      \correctchoice{weight of the air it displaces.}
        \wrongchoice{weight of the object.}
        \wrongchoice{total pressure on the object.}
        \wrongchoice{difference between the weight of the object and the weight of the air it displaces.}
        \wrongchoice{weight of Archimedes.}
    \end{choices}
\end{question}
}

\endinput


