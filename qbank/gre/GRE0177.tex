

%% http://ctan.mirrorcatalogs.com/macros/latex/contrib/physics/physics.pdf
%%--------------------------------------------------------------------------------


%% GRE Physics 0177 Practice Exam
%%----------------------------------------

%% Page 12
\element{gre}{
\begin{question}{GRE0177-Q01}
    Which of the following best illustrates the acceleration
        of a pendulum bob at points $a$ through $e$?
    \begin{choices}
        \AMCboxDimensions{down=-0.4cm}
        \wrongchoice{
            \begin{tikzpicture}
            \end{tikzpicture}
        }
        %\wrongchoice{\includegraphics[keepaspectratio,scale=0.95]{GRE0177-Q01-A}}
        %\wrongchoice{\includegraphics[keepaspectratio,scale=0.95]{GRE0177-Q01-B}}
        %\correctchoice{\includegraphics[keepaspectratio,scale=0.95]{GRE0177-Q01-C}}
        %\wrongchoice{\includegraphics[keepaspectratio,scale=0.95]{GRE0177-Q01-D}}
        %\wrongchoice{\includegraphics[keepaspectratio,scale=0.95]{GRE0177-Q01-E}}
    \end{choices}
\end{question}
}

%% F = m \omega^2 r = m \mu g
\element{gre}{
\begin{question}{GRE0177-Q02}
    The coefficient of static friction between a small coin and the surface of a turntable is \num{0.30}.
    The turntable rotates at \num{33.3} revolutions per minute.
    What is the maximum distance from the center of the turntable at which the coin will not slide?
    \begin{multicols}{2}
    \begin{choices}
        \wrongchoice{\SI{0.024}{\meter}}
        \wrongchoice{\SI{0.048}{\meter}}
        \wrongchoice{\SI{0.121}{\meter}}
      \correctchoice{\SI{0.242}{\meter}}
        \wrongchoice{\SI{0.484}{\meter}}
    \end{choices}
    \end{multicols}
\end{question}
}

%% F = v^2/t = GMm/r^2
%% V = 2 \pi r / T
\element{gre}{
\begin{question}{GRE0177-Q03}
    A satellite of mass $m$ orbits a planet of mass $M$ in a circular orbit of radius $R$.
    The time required for one revolution is:
    \begin{choices}
        \wrongchoice{independent of $M$}
        \wrongchoice{proportional to  $\sqrt{M}$}
        \wrongchoice{linear in $R$}
      \correctchoice{proportional to $R^{3/2}$}
        \wrongchoice{proportional to $R^2$}
    \end{choices}
\end{question}
}

%% KE_i = mv_0^2, KE_f = 2/3 m v_0^2
\element{gre}{
\begin{question}{GRE0177-Q04}
    In a nonrelativity, one-dimensional collision,
        a particle of mass $2m$ collides with a particle of mass $m$ at rest.
    If the particles stick together after the collision,
        what fraction of the initial kinetic energy is lost in the collision?
    \begin{multicols}{3}
    \begin{choices}
        \wrongchoice{\num{0}}
        \wrongchoice{\num{1/4}}
      \correctchoice{\num{1/3}}
        \wrongchoice{\num{1/2}}
        \wrongchoice{\num{2/3}}
    \end{choices}
    \end{multicols}
\end{question}
}

%% Page 14

%% There are six degrees of freedom
%% Three axial and three rotational
\element{gre}{
\begin{question}{GRE0177-Q05}
    A three-dimensional harmonic oscillator is in thermal equilibrium with a temperature reservoir at temperature $T$.
    The average total energy of the oscillator is:
    \begin{multicols}{3}
    \begin{choices}
        \wrongchoice{$\dfrac{kT}{2}$}
        \wrongchoice{$kT$}
        \wrongchoice{$\dfrac{3kT}{2}$}
      \correctchoice{$3kT$}
        \wrongchoice{$6kT$}
    \end{choices}
    \end{multicols}
\end{question}
}

%% in general -> dW = P dV
%% adiabatic  -> PV^\gamma = constant
%% monatomic  -> \gamma = 5/2
%% isothermal -> P = nRT / V
\element{gre}{
\begin{question}{GRE0177-Q06}
    An ideal monatomic gas expands quasi-statically to twice its volume.
    If the process is isothermal,
        the work done by the gas is $W_i$.
    If the process is adiabatic,
        the work done by the gas is $W_a$.
    Which of the following is true?
    \begin{multicols}{2}
    \begin{choices}
        \wrongchoice{$W_i = W_a$}
        \wrongchoice{$0 = W_i < W_a$}
        \wrongchoice{$0 < W_i < W_a$}
        \wrongchoice{$0 = W_a < W_i$}
      \correctchoice{$0 < W_a < W_i$}
    \end{choices}
    \end{multicols}
\end{question}
}

\element{gre}{
\begin{question}{GRE0177-Q07}
    Two long, identical bar magnets are placed under a horizontal piece of paper,
        as shown in the figure below.
    \begin{center}
    \begin{tikzpicture}
        %% NOTE: TODO:
    \end{tikzpicture}
    %\includegraphics[keepaspectratio,scale=0.95]{GREphysics-GRE0177-Q07}
    \end{center}
    The paper is covered with iron filings.
    When the two north poles are small distance apart and touching the paper,
        the iron filings move into a patter that shows the magnetic field line.
    Which of the following best illustrates the patter that results?
    \begin{multicols}{2}
    \begin{choices}
        \AMCboxDimensions{down=-0.4cm}
        \wrongchoice{
            \begin{tikzpicture}
            \end{tikzpicture}
        }
        %\wrongchoice{\includegraphics[keepaspectratio,scale=0.95]{GRE0177-Q07-A}}
        %\correctchoice{\includegraphics[keepaspectratio,scale=0.95]{GRE0177-Q07-B}}
        %\wrongchoice{\includegraphics[keepaspectratio,scale=0.95]{GRE0177-Q07-C}}
        %\wrongchoice{\includegraphics[keepaspectratio,scale=0.95]{GRE0177-Q07-D}}
        %\wrongchoice{\includegraphics[keepaspectratio,scale=0.95]{GRE0177-Q07-E}}
    \end{choices}
    \end{multicols}
\end{question}
}

%% Page 16
\element{gre}{
\begin{question}{GRE0177-Q08}
    A positive charge $Q$ is located at a distance $L$,
        above an infinite grounded conducting plane,
        as shown in the figure above.
    What is the total charge induced on the plane?
    \begin{multicols}{3}
    \begin{choices}
        \wrongchoice{$2Q$}
        \wrongchoice{$Q$}
        \wrongchoice{$0$}
      \correctchoice{$-Q$}
        \wrongchoice{$-2Q$}
    \end{choices}
    \end{multicols}
\end{question}
}

\element{gre}{
\begin{question}{GRE0177-Q09}
    Five positive charges of magnitude $q$ are arranged symmetrically
        around the circumference of a circle of radius $r$.
    What is the magnitude of the electric fiedl at the center of the circle?
    ($k=\tfrac{1}{4\pi \epsilon_0}$)
    \begin{multicols}{2}
    \begin{choices}
      \correctchoice{$0$}
        \wrongchoice{$\dfrac{kq}{r^2}$}
        \wrongchoice{$\dfrac{5kq}{r^2}$}
        \wrongchoice{$\dfrac{kq}{r^2}\cos{\dfrac{2\pi}{5}}$}
        \wrongchoice{$\dfrac{5kq}{r^2}\cos{\dfrac{2\pi}{5}}$}
    \end{choices}
    \end{multicols}
\end{question}
}

%% 1/C_eq = 1/C_1 + 1/C_2
%% dW = V dq = q/C dq
%% W = 1/2 C V^2
\element{gre}{
\begin{question}{GRE0177-Q10}
    A \SI{3}{\micro\farad} capacitor is connected in series
        with a \SI{6}{\micro\farad} capacitor.
    When a \SI{300}{\volt} potential difference is applied across this combination,
        the total energy stored in the two capacitors is
    \begin{multicols}{2}
    \begin{choices}
      \correctchoice{\SI{0.09}{\joule}}
        \wrongchoice{\SI{0.18}{\joule}}
        \wrongchoice{\SI{0.27}{\joule}}
        \wrongchoice{\SI{0.41}{\joule}}
        \wrongchoice{\SI{0.81}{\joule}}
    \end{choices}
    \end{multicols}
\end{question}
}

%% Use first image as second object
\element{gre}{
\begin{question}{GRE0177-Q11}
    An object is located \SI{40}{\centi\meter} from the first two thin converging lenses of focal lengths
        \SI{20}{\centi\meter} and \SI{10}{\centi\meter},
        respecively, as shown in the figure below.
    \begin{center}
    \begin{tikzpicture}
        %% NOTE: TODO:
    \end{tikzpicture}
    %\includegraphics[keepaspectratio,width=0.95\linewidth]{GREphysics-GRE0177-Q11}
    \end{center}
    The lenses are separated by \SI{30}{\centi\meter}.
    The final image formed by the two-lens system is located:
    \begin{choices}
      \correctchoice{\SI{5.0}{\centi\meter} to the right of the second lens}
        \wrongchoice{\SI{13.3}{\centi\meter} to the right of the second lens}
        \wrongchoice{infinitely far to the right of the second lens}
        \wrongchoice{\SI{13.3}{\centi\meter} to the left of the second lens}
        \wrongchoice{\SI{100}{\centi\meter} to the left of the second lens}
    \end{choices}
\end{question}
}

%% only one option is virtual
\element{gre}{
\begin{question}{GRE0177-Q12}
    A spherical, concave mirror is shown in the figure above.
    The focal point $F$ and the location of the object $O$ are indicated.
    At what point will the image be located?
    \begin{multicols}{3}
    \begin{choices}
        \wrongchoice{I}
        \wrongchoice{II}
        \wrongchoice{III}
        \wrongchoice{IV}
      \correctchoice{V}
    \end{choices}
    \end{multicols}
\end{question}
}

%% Page 18
\element{gre}{
\begin{question}{GRE0177-Q13}
    Two stars are separated by an angle of \SI{3e-5}{\radian}.
    What is the diameter of the smallest telescope that can
        resolve the two stars using visible light
        ($\lambda = \SI{600}{\nano\meter}$)?
    [Ignore any effects due to Earth's atmosphere.]
    \begin{multicols}{2}
    \begin{choices}
        \wrongchoice{\SI{1}{\milli\meter}}
      \correctchoice{\SI{2.5}{\centi\meter}}
        \wrongchoice{\SI{10}{\centi\meter}}
        \wrongchoice{\SI{2.5}{\meter}}
        \wrongchoice{\SI{10}{\meter}}
    \end{choices}
    \end{multicols}
\end{question}
}

%% A_detector = \pi D^2 / 4
%% A_sphere   = 4 \pi R^2
%% ansser = A_detector / A_sphere
\element{gre}{
\begin{question}{GRE0177-Q14}
    An \SI{8}{\centi\meter} diameter by \SI{8}{\centi\meter} long
        NaI(TI) detector detects gamma rays of a specific energy
        from a point source of radioactivity.
    What the source is placed just next to the detector at the
        center of the circular face, \SI{50}{\percent} of all
        emitted gamma rays at that energy are detected.
    If the detector is moved to \SI{1}{\meter} away,
        the fraction of detected gamma rays drops to
    \begin{multicols}{2}
    \begin{choices}
        \wrongchoice{\num{e-4}}
        \wrongchoice{\num{2e-4}}
      \correctchoice{\num{4e-4}}
        \wrongchoice{\num{8\pi e-4}}
        \wrongchoice{\num{16\pi e-4}}
    \end{choices}
    \end{multicols}
\end{question}
}

%% Testing discrimination of precision and accuracy
\element{gre}{
\begin{question}{GRE0177-Q15}
    Five classes of students measure the height of a building.
    Each class uses a different method and each measures the height
        many difference times.
    The data for each class are plotted below.
    Which class made the most precise measurement?
    \begin{choices}
        \AMCboxDimensions{down=-0.4cm}
        \wrongchoice{
            \begin{tikzpicture}
                %% NOTE: TODO:
            \end{tikzpicture}
        }
        %\correctchoice{\includegraphics[keepaspectratio]{GRE0177-Q15-A}}
        %\wrongchoice{\includegraphics[keepaspectratio]{GRE0177-Q15-B}}
        %\wrongchoice{\includegraphics[keepaspectratio]{GRE0177-Q15-C}}
        %\wrongchoice{\includegraphics[keepaspectratio]{GRE0177-Q15-D}}
        %\wrongchoice{\includegraphics[keepaspectratio]{GRE0177-Q15-E}}
    \end{choices}
\end{question}
}

%% Page 20

%% Assume poisson distribution
%% Noise to data ratio is \sqrt{N}
\element{gre}{
\begin{question}{GRE0177-Q16}
    A student makes \num{10} \num{1}-second measurements
        of the disintigration of a sample of a long-lived
        radioactive isotope and obtains the following values.
    \begin{equation*}
        3, 0, 2, 1, 2, 4, 0, 1, 2, 5
    \end{equation*}
    How long should the student count to establish the rate to
        an uncertainty of \SI{1}{\percent}
    \begin{multicols}{2}
    \begin{choices}
        \wrongchoice{\SI{80}{\second}}
        \wrongchoice{\SI{160}{\second}}
        \wrongchoice{\SI{2 000}{\second}}
      \correctchoice{\SI{5 000}{\second}}
        \wrongchoice{\SI{6 400}{\second}}
    \end{choices}
    \end{multicols}
\end{question}
}

\element{gre}{
\begin{question}{GRE0177-Q17}
    The ground state electron configuration for phosphorus,
        which has \num{15} electrons, is
    \begin{multicols}{2}
    \begin{choices}
        \wrongchoice{$1s^2 2s^2 2p^6 3s^1 3p^4$}
        \wrongchoice{$1s^2 2s^2 2p^6 3s^2 3p^3$}
      \correctchoice{$1s^2 2s^2 2p^6 3s^2 3d^3$}
        \wrongchoice{$1s^2 2s^2 2p^6 3s^1 3d^4$}
        \wrongchoice{$1s^2 2s^2 2p^6 3p^2 3d^3$}
    \end{choices}
    \end{multicols}
\end{question}
}

\element{gre}{
\begin{question}{GRE0177-Q18}
    The energy required to remove both electrons
        from the helium atom in its ground state
        is \SI{79.0}{\eV}.
    How much energy is required to ionize helium
        (i.e. to remove one electron)?
    \begin{multicols}{2}
    \begin{choices}
      \correctchoice{\SI{24.6}{\eV}}
        \wrongchoice{\SI{39.5}{\eV}}
        \wrongchoice{\SI{51.8}{\eV}}
        \wrongchoice{\SI{54.4}{\eV}}
        \wrongchoice{\SI{65.4}{\eV}}
    \end{choices}
    \end{multicols}
\end{question}
}

\element{gre}{
\begin{question}{GRE0177-Q19}
    The primary source of the Sun's energy is a series of thermonuclear reactions in which the energy
        produced is $c^2$ times the mass difference between
    \begin{choices}
        \wrongchoice{two hydrogen atoms and one helium atom}
      \correctchoice{four hydrogen atoms and one helium atom}
        \wrongchoice{six hydrogen atoms and two helium atom}
        \wrongchoice{three hydrogen atoms and one carbon atom}
        \wrongchoice{two hydrogen atoms plus two heliumn atom and one carbon atom}
    \end{choices}
\end{question}
}

\element{gre}{
\begin{question}{GRE0177-Q20}
    In the production of x-rays, the term ``bremsstrahlung'' refers
        to which of the following?
    \begin{choices}
        \wrongchoice{The cut-off wavelength, $\lambda_{min}$, of the
            x-ray tube} 
        \wrongchoice{The discrete x-ray lines emitted when an
            electron in an outer orbit fills a vacancy in
            an inner orbit of the atoms in the target
            metal of the x-ray tube}
        \wrongchoice{The discrete x-ray lines absorbed when an
            electron in an inner orbit fills a vacancy in
            an outer orbit of the atoms in the target
            metal of the x-ray tube}
        \wrongchoice{The smooth, continuous x-ray spectra
            produced by high-energy blackbody
            radiation from the x-ray tube}
      \correctchoice{The smooth, continuous x-ray spectra
            produced by rapidly decelerating electrons
            in the target metal of the x-ray tube}
    \end{choices}
\end{question}
}

%% E \prop 1/n^2
%% \lambda = c / f
\element{gre}{
\begin{question}{GRE0177-Q21}
    In the hydrogen spectrum, the ratio of the wavelength for
        Lyman-$\alpha$ radiation ($n=2$ to $n=1$) to 
        Balmer-$\alpha$ radiation ($n=3$ to $n=2$) is
    \begin{multicols}{3}
    \begin{choices}
        \wrongchoice{\num{5/48}}
      \correctchoice{\num{5/27}}
        \wrongchoice{\num{1/3}}
        \wrongchoice{\num{3}}
        \wrongchoice{\num{27/5}}
    \end{choices}
    \end{multicols}
\end{question}
}

%% Page 22
\element{gre}{
\begin{question}{GRE0177-Q22}
    An astronomer observes a very small moon orbiting a planet
        and measure the moon's minimum and maximum distances
        from the planet's center and the moon's maximum
        orbital speed.
    Which of the following \emph{cannot} be calculated from these measurements?
    \begin{choices}
      \correctchoice{Mass of the moon}
        \wrongchoice{Mass of the planet}
        \wrongchoice{Minimum speed of the moon}
        \wrongchoice{Period of the orbit}
        \wrongchoice{Semimajor axis of the orbit}
    \end{choices}
\end{question}
}

\element{gre}{
\begin{question}{GRE0177-Q23}
    A particel is constrained to move in a circle with a
        \SI{10}{\meter} radius.
    At one instance, the particle's speed is \SI{10}{\meter\per\second}
        and is increasing at a rate of \SI{10}{\meter\per\second\squared}.
    The angle between the particle's velocity and acceleration
        vector is
    \begin{multicols}{3}
    \begin{choices}
        \wrongchoice{\ang{0}}
        \wrongchoice{\ang{30}}
      \correctchoice{\ang{45}}
        \wrongchoice{\ang{60}}
        \wrongchoice{\ang{90}}
    \end{choices}
    \end{multicols}
\end{question}
}

\element{gre}{
\begin{question}{GRE0177-Q24}
    A stone is thrown at an angle of \ang{45} above the horizontal $x$-axis in the $+x$-direction.
    If air resistance is ignored, which of the velocity versus time graphs shown above best represents $v_x$ verses $t$ and $v_y$ verse $t$, respecively?
    %\includegraphics[keepaspectratio]{GRE0177-Q24-I}
    %\includegraphics[keepaspectratio]{GRE0177-Q24-II}
    %\includegraphics[keepaspectratio]{GRE0177-Q24-III}
    %\includegraphics[keepaspectratio]{GRE0177-Q24-IV}
    %\includegraphics[keepaspectratio]{GRE0177-Q24-V}
    \begin{center}
    \begin{tikzpicture}
        %% NOTE: TODO: pgfplots x5
    \end{tikzpicture}
    \end{center}
    \begin{center}
    \begin{tabu}{cX[c]X[c]}
        \toprule
        \makebox[1.5em][c]{\textnumero}
            & $v_x$ vs. $t$
            & $v_y$ vs. $t$ \\
        \bottomrule
    \end{tabu}
    \end{center}
    \begin{choices}
        \wrongchoice{\begin{tabu}{X[c]X[c]} I  &  IV \\ \end{tabu}}
        \wrongchoice{\begin{tabu}{X[c]X[c]} II &   I \\ \end{tabu}}
        \wrongchoice{\begin{tabu}{X[c]X[c]} II & III \\ \end{tabu}}
      \correctchoice{\begin{tabu}{X[c]X[c]} II &   V \\ \end{tabu}}
        \wrongchoice{\begin{tabu}{X[c]X[c]} IV &   V \\ \end{tabu}}
    \end{choices}
\end{question}
}


%% Page 26
\element{gre}{
\begin{question}{GRE0177-Q25}
    Seven pennies are arranged in a hexagonal, planar
        pattern so as to touch each neighbor, as shown in
        the figure below.
    \begin{center}
    \begin{tikzpicture}
            %% NOTE: TODO:
    \end{tikzpicture}
        %\includegraphics[keepaspectratio,width=0.95\linewidth]{GREphysics-GRE0177-Q25}
    \end{center}
    Each penny is a uniform disk of mass $m$ and radius $r$.
    What is the moment of inertia of the system of seven
        about an axis that passes through the center of
        the central penny and is normal to the plane of
        the pennies?
    %% Parallel axis rule -> I = I_cm + MR^2
    %% moment of disk -> I = 1/2 m r^2
    \begin{multicols}{3}
    \begin{choices}
        \wrongchoice{$\dfrac{7 m r^2}{2}$}
        \wrongchoice{$\dfrac{13 m r^2}{2}$}
        \wrongchoice{$\dfrac{29 m r^2}{2}$}
        \wrongchoice{$\dfrac{49 m r^2}{2}$}
      \correctchoice{$\dfrac{55 m r^2}{2}$}
    \end{choices}
    \end{multicols}
\end{question}
}

\element{gre}{
\begin{question}{GRE0177-Q26}
    A thin uniform rod of mass $M$ and length $L$ is
        positioned vertically above an anchored frictionless
        pivot point, as shown below, and then allowed to
        fall to the ground.
    \begin{center}
    \begin{tikzpicture}
            %% NOTE: TODO:
    \end{tikzpicture}
        %\includegraphics[keepaspectratio]{GREphysics-GRE0177-Q26}
    \end{center}
    With what speed does the free end of the rod strike the ground?
    %% Think conservation of energy
    %% E = mgh + 1/2 mv^2 + 1/2 I \omega^2
    \begin{multicols}{2}
    \begin{choices}
        \wrongchoice{$\sqrt{\frac{1}{3}gL}$}
        \wrongchoice{$\sqrt{gL}$}
      \correctchoice{$\sqrt{3gL}$}
        \wrongchoice{$\sqrt{12 gL}$}
        \wrongchoice{$12 \sqrt{gL}$}
    \end{choices}
    \end{multicols}
\end{question}
}

\element{gre}{
\begin{question}{GRE0177-Q27}
    The eigenvalues of a Hermitian operator are always
    \begin{multicols}{2}
    \begin{choices}
      \correctchoice{real}
        \wrongchoice{imaginary}
        \wrongchoice{degenerate}
        \wrongchoice{linear}
        \wrongchoice{positive}
    \end{choices}
    \end{multicols}
\end{question}
}

\element{gre}{
\begin{question}{GRE0177-Q28}
    The states $\ket{1}$, $\ket{2}$, and $\ket{3}$ are orthonormal.
    \begin{align*}
        \ket{\lambda_1} &= 5 \ket{1} - 3 \ket{2} + 2 \ket{3} \\
        \ket{\lambda_2} &=   \ket{1} - 5 \ket{2} + x \ket{3}
    \end{align*}
    For what values of $x$ are the states $\ket{\lambda_1}$
        and $\ket{\lambda_2}$ given above orthogonal?
    \begin{multicols}{3}
    \begin{choices}
        \wrongchoice{10}
        \wrongchoice{5}
        \wrongchoice{0}
        \wrongchoice{-5}
      \correctchoice{-10}
    \end{choices}
    \end{multicols}
\end{question}
}

\element{gre}{
\begin{question}{GRE0177-Q29}
    The state
    $\Psi = \frac{1}{\sqrt{6}} \Psi_{-1} + \frac{1}{\sqrt{2}} \Psi_{1} + \frac{1}{\sqrt{3}} \Psi_{2}$
    is a linear combination of three orthonormal eigenstates of the
        operator $\hat{O}$ corresponding to eigenstates $-1$, $1$, and $2$.
    What is the expectation value of $\hat{O}$ for this state?
    %% <\phi|O|\phi> = -1/6 + 1/2 + 2/3
    \begin{multicols}{2}
    \begin{choices}
        \wrongchoice{$\dfrac{2}{3}$}
        \wrongchoice{$\sqrt{\dfrac{7}{6}}$}
      \correctchoice{$1$}
        \wrongchoice{$\dfrac{4}{3}$}
        \wrongchoice{$\dfrac{\sqrt{3} + 2\sqrt{2} - 1}{\sqrt{6}}$}
    \end{choices}
    \end{multicols}
\end{question}
}

%% Page 28
\element{gre}{
\begin{question}{GRE0177-Q30}
    Which of the following functions could represent the radial
        wave function for an electron in an atom?
    ($r$ is the distance of the electron from the nucleus: $A$ and $b$ are constants.)
    \begin{enumerate}
        \item[I]    $A e^{-br}$
        \item[II]   $A \sin{br}$
        \item[III]  $\frac{A}{r}$
    \end{enumerate}
    %% Radial function must be exponentially decreasing
    \begin{multicols}{2}
    \begin{choices}
      \correctchoice{I only}
        \wrongchoice{II only}
        \wrongchoice{I and II only}
        \wrongchoice{I and III only}
        \wrongchoice{I, II, and III}
    \end{choices}
    \end{multicols}
\end{question}
}

%% Page 5
\element{gre}{
\begin{question}{GRE0177-Q31}
    Positronium is an atom formed by an electron and a positron (antielectron).
    It is similar to the hydrogen atom, with the positron replacing the proton.
    If a positronium atom makes a transition from the state with
        $n=3$ to a state with $n=1$, the energy of the photon emitted in this
        transition is closest to
    %% Energy level of positronium is half those of Hydrogen
    %% 6.8eV instead of 13.6 eV.
    \begin{multicols}{2}
    \begin{choices}
      \correctchoice{\SI{6.0}{\eV}}
        \wrongchoice{\SI{6.8}{\eV}}
        \wrongchoice{\SI{12.2}{\eV}}
        \wrongchoice{\SI{13.6}{\eV}}
        \wrongchoice{\SI{24.2}{\eV}}
    \end{choices}
    \end{multicols}
\end{question}
}

\element{gre}{
\begin{question}{GRE0177-Q32}
    If the total energy of a particle of mass $m$ is equal to twice its
        rest energy, then the magnitude of the particle's relativisitic momentum is
    %% E^2 = m^2 + p^2 = 4
    \begin{multicols}{3}
    \begin{choices}
        \wrongchoice{$\frac{mc}{2}$}
        \wrongchoice{$\frac{mc}{\sqrt{2}}$}
        \wrongchoice{$mc$}
      \correctchoice{$\sqrt{3}mc$}
        \wrongchoice{$2mc$}
    \end{choices}
    \end{multicols}
\end{question}
}

\element{gre}{
\begin{question}{GRE0177-Q33}
    If a charged pion that decays in \SI{1e-8}{\second} in its own rest
        frame is to travel \SI{30}{\meter} in the laboratory before decaying,
        the pion's speed must be most nearly
    %% proper time of 1e-8: t = \gamma t_0
    %% lab frame of 30:  L = 30 = vt 
    \begin{multicols}{2}
    \begin{choices}
        \wrongchoice{\SI{0.44e8}{\meter\per\second}}
        \wrongchoice{\SI{2.84e8}{\meter\per\second}}
        \wrongchoice{\SI{2.90e8}{\meter\per\second}}
      \correctchoice{\SI{2.98e8}{\meter\per\second}}
        \wrongchoice{\SI{3.00e8}{\meter\per\second}}
    \end{choices}
    \end{multicols}
\end{question}
}

\element{gre}{
\begin{question}{GRE0177-Q34}
    In an inertial reference frame $S$, two events occur on the $x$-axis
        separated in time by $\delta{}t$ and in space by $\delta{}x$.
    In another inertial reference frame $S'$, moving in the $x$-direction
        relative to $S$, the two events could occur at the same time under
        which, if any, of the following conditions?
    \begin{choices}
        \wrongchoice{For any values of $\delta{}x$ and $\delta{}t$}
        \wrongchoice{Only if $\left| \frac{\delta{}x}{\delta{}t} \right| < c$}
      \correctchoice{Only if $\left| \frac{\delta{}x}{\delta{}t} \right| > c$}
        \wrongchoice{Only if $\left| \frac{\delta{}x}{\delta{}t} \right| = c$}
        \wrongchoice{Under no condition}
    \end{choices}
\end{question}
}

\element{gre}{
\begin{question}{GRE0177-Q35}
    If the absolute temperature of a blackbody is increased by a factor of $3$,
        the energy radiated per unit area does which of the following?
    \begin{choices}
        \wrongchoice{Decreases by a factor of $81$}
        \wrongchoice{Decreases by a factor of $9$}
        \wrongchoice{Increases by a factor of $9$}
        \wrongchoice{Increases by a factor of $27$}
      \correctchoice{Increases by a factor of $81$}
    \end{choices}
\end{question}
}

%% Page 30
\element{gre}{
\begin{question}{GRE0177-Q36}
    Consider a quasi-static adiabatic expansion of an
        ideal gas from an initial state $i$ to a final
        state $f$.
    Which of the following statements is \emph{not} true?
    \begin{choices}
        \wrongchoice{No heat flows into or out of the gas}
        \wrongchoice{The entropy of state $i$ equals the entropy of state $f$.}
        \wrongchoice{The change of internal energy of the gas is $-\int{}P\,\mathrm{d}V$}
        \wrongchoice{The mechanical work done by the gas is $\int{}P\,\mathrm{d}V$}
      \correctchoice{The temperature of the gas remains constant.}
    \end{choices}
\end{question}
}

\element{gre}{
\begin{question}{GRE0177-Q37}
    \begin{center}
    \begin{tikzpicture}
            %% NOTE: TODO:
    \end{tikzpicture}
        %\includegraphics[keepaspectratio]{GREphysics-GRE0177-Q37}
    \end{center}
    A constant amount of an ideal gas undergoes the cyclic process $ABCA$
        in the $PV$ diagram shown above.
    The path $BC$ is isothermal.
    The work done by the gas during one complete cycle,
        beginning and ending at $A$, is most nearly
    \begin{multicols}{3}
    \begin{choices}
        \wrongchoice{\SI{600}{\kilo\joule}}
        \wrongchoice{\SI{300}{\kilo\joule}}
        \wrongchoice{zero}
        \wrongchoice{\SI{-300}{\kilo\joule}}
        \wrongchoice{\SI{-600}{\kilo\joule}}
    \end{choices}
    \end{multicols}
\end{question}
}

\element{gre}{
\begin{question}{GRE0177-Q38}
    \begin{center}
    \begin{tikzpicture}
            %% NOTE: TODO:
    \end{tikzpicture}
        %\includegraphics[keepaspectratio]{GREphysics-GRE0177-Q37}
    \end{center}
    An $AC$ circuit consists of the elements shown above,
        with $R=\SI{10000}{\ohm}$, $L=\SI{25}{\milli\henry}$,
        and C an adjustable capacitance.
    The $AC$ voltage generator supplies a signal with an
        amplitude of \SI{40}{\volt} and angular frequency
        of \SI{1000}{\radian\per\second}.
    For what value of $C$ is the ampitude of the current maximized?
    \begin{multicols}{3}
    \begin{choices}
        \wrongchoice{\SI{4}{\nano\farad}}
        \wrongchoice{\SI{40}{\nano\farad}}
        \wrongchoice{\SI{4}{\micro\farad}}
        \wrongchoice{\SI{40}{\micro\farad}}
        \wrongchoice{\SI{400}{\micro\farad}}
    \end{choices}
    \end{multicols}
\end{question}
}

\element{gre}{
\begin{question}{GRE0177-Q39}
    Which two of the following circuits are high-pass filters?
    \begin{multicols}{2}
        %% Enumerate with numbers and change roman to arabic
        %\includegraphics[width=linewidth,keepaspectratio]{GRE0177-Q39-I}
        %\includegraphics[width=linewidth,keepaspectratio]{GRE0177-Q39-II}
        %\includegraphics[width=linewidth,keepaspectratio]{GRE0177-Q39-III}
        %\includegraphics[width=linewidth,keepaspectratio]{GRE0177-Q39-IV}
    \end{multicols}
    \begin{multicols}{2}
    \begin{choices}
        \wrongchoice{I and II}
        \wrongchoice{I and III}
        \wrongchoice{I and IV}
        \wrongchoice{II and III}
        \wrongchoice{II and IV}
    \end{choices}
    \end{multicols}
\end{question}
}

%% TODO: Start writing here
\element{gre}{
\begin{question}{GRE0177-Q40}
    \begin{center}
    \begin{circuitikz}
        %% NOTE: TODO: draw circuit
    \end{circuitikz}
    \end{center}
    In the circuit shown above, the switch $S$ is closed at $t=0$.
    Which of the following best rerepsents the voltage across the inductor,
        as seen on an oscilloscope?
    \begin{multicols}{2}
    \begin{choices}
        %% ANS is D
        \AMCboxDimensions{down=-0.4cm}
        \ctikzset{bipoles/length=0.75cm}
        \wrongchoice{
            \begin{tikzpicture}
                %% NOTE: TODO: pgfplots
            \end{tikzpicture}
        }
    \end{choices}
    \end{multicols}
\end{question}
}

%% page 34
\element{gre}{
\begin{question}{GRE0177-Q41}
    Maxwell's equations can be written in the form shown below.
    If magnetic charges exists and if it is conserved,
        which of these equations will have to be changed?
    \begin{itemize}
        \item[I.] $\nabla\cdot\mathbf{E}=\rho/\epsilon_0$
        \item[II.] $\nabla\cdot\mathbf{B}=0$
        \item[III.] $\nabla\times\mathbf{E}=-\dfrac{\partial\mathbf{B}}{\partial t}$
        \item[IV.] $\nabla\times\mathbf{B}=\mu_0\mathbf{J} + \mu_0\epsilon_0\dfrac{\partial\mathbf{E}}{\partial t}$
    \end{itemize}
    %% NOTE: questionmult
    \begin{multicols}{2}
    \begin{choices}
        %% ANS is E
        \wrongchoice{I only}
        \wrongchoice{II only}
        \wrongchoice{III only}
        \wrongchoice{I and IV}
      \correctchoice{II and III}
    \end{choices}
    \end{multicols}
\end{question}
}

\element{gre}{
\begin{question}{GRE0177-Q42}
    \begin{center}
    \begin{tikzpicture}
        %% NOTE: TODO:
    \end{tikzpicture}
    Three wire loops and an observer are positioned as shown in the figure above.
    From the observer's point of view, a current $I$ flows counterclockwise in the middle loop,
        which is moving towards the observer with a velocity $v$.
    Loops $A$ and $B$ are stationary.
    The same observer would notice that:
    \begin{choices}
        %% ANS is C
        \wrongchoice{clockwise currents are induced in loops $A$ and $B$}
        \wrongchoice{counterclockwise currents are induced in loops $A$ and $B$}
      \correctchoice{a clockwise current is induced in loop $A$, but a counterclockwise current is induced in loop $B$}
        \wrongchoice{a counterclockwise current is induced in loop $A$, but a clockwise current is induced in loop $B$}
        \wrongchoice{a counterclockwise current is induced in loop $A$, but no current is induced in loop $B$}
    \end{choices}
\end{question}
}

%% page 36
\element{gre}{
\begin{question}{GRE0177-Q43}
    The components of the orbital angular momentum operator $\mathbf{L} = \left(L_x,L_y,L_z\right)$ satisfy the following commutation relations.
    \begin{align*}
        \left[ L_x, L_y \right] &= i\hbar L_z \\
        \left[ L_y, L_x \right] &= i\hbar L_x \\
        \left[ L_z, L_x \right] &= i\hbar L_y \\
    \end{align*}
    What is the value of the commutator $\left[ L_x L_y, L_z \right]$?
    \begin{multicols}{2}
    \begin{choices}
        %% ANS is D
        \wrongchoice{$2i\hbar L_x L_y$}
        \wrongchoice{$i\hbar \left(L_x^2 + L_y^2\right)$}
        \wrongchoice{$-i\hbar \left(L_x^2 + L_y^2\right)$}
        \wrongchoice{$i\hbar \left(L_x^2 - L_y^2\right)$}
        \wrongchoice{$-i\hbar \left(L_x^2 - L_y^2\right)$}
    \end{choices}
    \end{multicols}
\end{question}
}

\element{gre}{
\begin{question}{GRE0177-Q44}
    The energy eigenstates for a particle of mass $m$ in a box of length $L$ have wave functions $\phi_n(x) = \sqrt{2/L}\sin\left(n\pi x/L\right)$ and energies $E_n=n^2 \pi^2 \hbar^2/2mL^2$, where $n=1,2,3,\ldots$
    At time $t=0$, the particle is in a state described as follows.
    \begin{equation*}
        \Psi \left(t=0\right) = \frac{1}{\sqrt{14}} \left[\phi_1 + 2\phi_2 + 3\phi_3 \right]
    \end{equation*}
    Which of the following is a possible result of a measurement of energy for the state $\Psi$?
    \begin{multicols}{3}
    \begin{choices}
        %% ANS is D
        \wrongchoice{$2E_1$}
        \wrongchoice{$5E_1$}
        \wrongchoice{$7E_1$}
        \wrongchoice{$9E_1$}
        \wrongchoice{$14E_1$}
    \end{choices}
    \end{multicols}
\end{question}
}

\element{gre}{
\begin{question}{GRE0177-Q45}
    Let $\ket{n}$ represent the normalized $n^{th}$ energy eigenstate of the one-dimensional harmonic oscillator,
        $H \ket{n} = \hbar\omega\left(n+\dfrac{1}{2}\right)\ket{n}$.
    If $\ket{\Psi}$ is a normalized ensemble state that can be expanded as a linear combination
    \begin{equation*}
        \ket{\Psi} = \frac{1}{\sqrt{14}}\ket{1} - \frac{2}{\sqrt{14}}\ket{2} + \frac{3}{\sqrt{14}}\ket{3}
    \end{equation*}
    of the eigenstates, what is the expectation value of the energy operator in this ensemble state?
    \begin{multicols}{3}
    \begin{choices}
        %% ANS is B
        \wrongchoice{$\dfrac{102}{14} \hbar\omega$}
        \wrongchoice{$\dfrac{43}{14} \hbar\omega$}
        \wrongchoice{$\dfrac{23}{14} \hbar\omega$}
        \wrongchoice{$\dfrac{17}{14} \hbar\omega$}
        \wrongchoice{$\dfrac{7}{14} \hbar\omega$}
    \end{choices}
    \end{multicols}
\end{question}
}

\element{gre}{
\begin{question}{GRE0177-Q46}
    A free particle with initial kinetic energy $E$ and de Broglie wavelength $lambda$ enters a region in which it has a potential energy $V$.
    What is the particle's new de Broglie wavelength?
    \begin{multicols}{2}
    \begin{choices}
        %% ANS is E
        \wrongchoice{$\lambda \left(1+E/V\right)$}
        \wrongchoice{$\lambda \left(1-E/V\right)$}
        \wrongchoice{$\lambda \left(1-E/V\right)^{-1}$}
        \wrongchoice{$\lambda \left(1+E/V\right)^{1/2}$}
        \wrongchoice{$\lambda \left(1-E/V\right)^{1/2}$}
    \end{choices}
    \end{multicols}
\end{question}
}

\element{gre}{
\begin{question}{GRE0177-Q47}
    A sealed and thermally insulated container of total volume $V$ is divided into two equal volumes by an impermeable wall.
    The left half of the container is initially occupied by $n$ moles of an ideal gas at temperature $T$.
    Which of the following gives the change in entropy of the system when the wall is suddenly removed and the gas expands to fill the entire volume?
    \begin{multicols}{2}
    \begin{choices}
        %% ANS is B
        \wrongchoice{$2nR \ln 2$}
        \wrongchoice{$nR \ln 2$}
        \wrongchoice{$\dfrac{1}{2}nR \ln 2$}
        \wrongchoice{$-nR \ln 2$}
        \wrongchoice{$-2nR \ln 2$}
    \end{choices}
    \end{multicols}
\end{question}
}

\element{gre}{
\begin{question}{GRE0177-Q48}
    A gaseous mixture of \ce{O2} (molecular mass 32 u) and \ce{N2} (molecular mass 28 u) is maintained at constant temperature.
    What is the ratio $\dfrac{v_{rms} (\ce{N2})}{v_{rms}(\ce{)2}}$ of the root-mean-square speeds of the molecules?
    \begin{multicols}{2}
    \begin{choices}
        %% ANS is C
        \wrongchoice{$\dfrac{7}{8}$}
        \wrongchoice{$\sqrt{\dfrac{7}{8}}$}
        \wrongchoice{$\sqrt{\dfrac{8}{7}}$}
        \wrongchoice{$\left(\dfrac{8}{7}\right)^2$}
        \wrongchoice{$\ln\left(\dfrac{8}{7}\right)$}
    \end{choices}
    \end{multicols}
\end{question}
}

\element{gre}{
\begin{question}{GRE0177-Q49}
    In a Maxwell-Boltzmann system with two states of energies $\epsilon$ and $2\epsilon$,
        respectively, and a degeneracy of 2 for each state,
        the partition function is:
    \begin{multicols}{2}
    \begin{choices}
        %% ANS is E
        \wrongchoice{$\mathrm{e}^{-\epsilon /kT}$}
        \wrongchoice{$2\mathrm{e}^{-2\epsilon /kT}$}
        \wrongchoice{$2\mathrm{e}^{-3\epsilon /kT}$}
        \wrongchoice{$\mathrm{e}^{-\epsilon /kT} + \mathrm{e}^{-2\epsilon /kT}$}
        \wrongchoice{$2\left[\mathrm{e}^{-\epsilon /kT} + \mathrm{e}^{-2\epsilon /kT}\right]$}
    \end{choices}
    \end{multicols}
\end{question}
}

\element{gre}{
\begin{question}{GRE0177-Q50}
    At \SI{20}{\degreeCelsius}, a pipe open at both ends resonantes at a frequency of \SI{440}{\hertz}.
    At what frequency does the same pipe resonante on a particularly cold day when the speed of such is 3 percent lower than it would be at \SI{20}{\degreeCelsius}?
    \begin{multicols}{2}
    \begin{choices}
        %% ANS is B
        \wrongchoice{\SI{414}{\hertz}}
        \wrongchoice{\SI{427}{\hertz}}
        \wrongchoice{\SI{433}{\hertz}}
        \wrongchoice{\SI{440}{\hertz}}
        \wrongchoice{\SI{453}{\hertz}}
    \end{choices}
    \end{multicols}
\end{question}
}

\element{gre}{
\begin{question}{GRE0177-Q51}
    Unpolarized light of intensity $I_0$ is incident on a series of three polarizing filters.
    The axis of the second filter is oriented at \ang{45} to that of the first filter,
        while the axis of the third filter is oriented at \ang{90} to that of the first filter.
    What is the intensity of the light transmitted through the third filter?
    \begin{multicols}{2}
    \begin{choices}
        %% ANS is B
        \wrongchoice{zero}
        \wrongchoice{$\dfrac{I_0}{8}$}
        \wrongchoice{$\dfrac{I_0}{4}$}
        \wrongchoice{$\dfrac{I_0}{2}$}
        \wrongchoice{$\dfrac{I_0}{\sqrt{2}}$}
    \end{choices}
    \end{multicols}
\end{question}
}

\element{gre}{
\begin{question}{GRE0177-Q52}
    \begin{center}
    \begin{tikzpicture}
        %% NOTE: TODO:
    \end{tikzpicture}
    \end{center}
    The conventional unit cell of a body-centered cubic Bravais lattice is shown in the figure above.
    The conventional cell has volume $a^3$.
    What is the volume of the primitive unit cell?
    \begin{multicols}{2}
    \begin{choices}
        %% ANS is  C
        \wrongchoice{$\dfrac{a^3}{8}$}
        \wrongchoice{$\dfrac{a^3}{4}$}
        \wrongchoice{$\dfrac{a^3}{2}$}
        \wrongchoice{$a^3$}
        \wrongchoice{$2a^3$}
    \end{choices}
    \end{multicols}
\end{question}
}


%% page 40
\element{gre}{
\begin{question}{GRE0177-Q53}
    Which of the following best represents the temperature dependence of the resistivity of an undoped semiconductor?
    \begin{multicols}{2}
    \begin{choices}
        %% ANS is B
        \AMCboxDimensions{down=-0.4cm}
        \wrongchoice{
            \begin{tikzpicture} 
                %% NOTE: TODO: pgfplots
            \end{tikzpicture} 
        }
    \end{choices}
    \end{multicols}
\end{question}
}

\element{gre}{
\begin{question}{GRE0177-Q54}
    \begin{center}
    \begin{tikzpicture}
        %% NOTE: TODO: pgfplots
    \end{tikzpicture}
    \end{center}
    The figure above shows a plot of the time-dependent force $F_x(t)$ acting on a particle in motion along the $x$-axis.
    What is the total impulse delivered to the particle?
    \begin{multicols}{3}
    \begin{choices}
        %% ANS is C
        \wrongchoice{zero}
        \wrongchoice{\SI{1}{\kilo\gram\meter\per\second}}
        \wrongchoice{\SI{2}{\kilo\gram\meter\per\second}}
        \wrongchoice{\SI{3}{\kilo\gram\meter\per\second}}
        \wrongchoice{\SI{4}{\kilo\gram\meter\per\second}}
    \end{choices}
    \end{multicols}
\end{question}
}

\element{gre}{
\begin{question}{GRE0177-Q55}
    \begin{center}
    \begin{tikzpicture}
        %% NOTE: TODO: pgfplots
    \end{tikzpicture}
    \end{center}
    A particle of mass $m$ is moving along the $x$-axis with speed $v$ when it collides with a particle of mass $2m$ initially at rest.
    After the collision, the first particle has come to rest,
        and the second particle has split into two equal-mass pieces that move at equal angles $\theta>0$ with the $x$-axis,
        as shown in the figure above.
    Which of the following statements correctly describes the speeds of the two pieces?
    \begin{choices}
        %% ANS is E
        \wrongchoice{Each piece moves with speed $v$}
        \wrongchoice{One of the pieces moves with speed $v$, the other moves with speed less than $v$}
        \wrongchoice{Each piece moves with speed $v/2$}
        \wrongchoice{One of the pieces moves with speed $v/2$, the other moves with speed greater than $v/2$}
        \wrongchoice{Each piece moves with speed greater than $v/2$}
    \end{choices}
\end{question}
}

%% page 42
\element{gre}{
\begin{question}{GRE0177-Q56}
    A balloon is to be filled with helium and used to suspend a mass of \SI{300}{\kilo\gram} in air.
    If the mass of the balloon is neglected, which of the following gives the approximate volume of helium required?
    (The density of air is \SI{1.29}{\kilo\gram\per\meter\cubed} and the density of helium is \SI{0.18}{\kilo\gram\per\meter\cubed}.)
    \begin{multicols}{2}
    \begin{choices}
        %% ANS is: D
        \wrongchoice{\SI{50}{\meter\cubed}}
        \wrongchoice{\SI{95}{\meter\cubed}}
        \wrongchoice{\SI{135}{\meter\cubed}}
        \wrongchoice{\SI{270}{\meter\cubed}}
        \wrongchoice{\SI{540}{\meter\cubed}}
    \end{choices}
    \end{multicols}
\end{question}
}

\element{gre}{
\begin{question}{GRE0177-Q57}
    \begin{center}
    \begin{tikzpicture}
        %% NOTE: TODO: pgfplots
    \end{tikzpicture}
    \end{center}
    A stream of water of density $\rho$,
        cross-section area $A$, and speed $v$ strikes a wall that is perpendicular to the direction of the stream,
        as show in the figure above.
    The water then flows sideways across the wall.
    The force exerted by the stream on the wall is:
    \begin{multicols}{2}
    \begin{choices}
        %% ANS is A
        \wrongchoice{$\rho v^2 A$}
        \wrongchoice{$\dfrac{\rho v^2 A}{2}$}
        \wrongchoice{$\rho g h A$}
        \wrongchoice{$\dfrac{v^2 A}{\rho}$}
        \wrongchoice{$\dfrac{v^2 A}{2\rho}$}
    \end{choices}
    \end{multicols}
\end{question}
}

\element{gre}{
\begin{question}{GRE0177-Q58}
    A proton moves in the $+z$-direction after being accelerated from rest through a potential difference $V$.
    The proton then passes through a region with a uniform electric field $E$ in the $+x$-direction and a uniform magnetic field $B$ in the $+y$-direction,
        but the proton's trajectory is not affected.
    If the experiment were repeated using a potential difference of $2V$,
        the proton would then be:
    \begin{choices}
        %% ANS is B
        \wrongchoice{deflected in the $+x$-direction}
        \wrongchoice{deflected in the $-x$-direction}
        \wrongchoice{deflected in the $+y$-direction}
        \wrongchoice{deflected in the $-y$-direction}
        \wrongchoice{undeflected}
    \end{choices}
\end{question}
}

\element{gre}{
\begin{question}{GRE0177-Q59}
    For an inductor and capacitor connected in series,
        the equation describing the motion of the charge is
    \begin{equation*}
        L \frac{\dd^2 Q}{\dd t^2} + \frac{1}{C} Q = 0\,,
    \end{equation*}
    where $L$ is inductance, $C$ the capacitance, and $Q$ is the charge.
    An analogous equation can be written for a simple harmonic oscillator with position $x$,
        mass $m$ and spring constant $k$.
    Which of the following correctly lists the mechanical analogs of $L$, $C$, and $Q$?
    \begin{center}
    \begin{tabu}{cX[c]X[c]X[c]}
        \toprule
        \makebox[1.5em][c]{\textnumero}
            & $L$ & $C$ & $Q$ \\
        \bottomrule
    \end{tabu}
    \end{center}
    \begin{choices}
        %% ANS is B
        \wrongchoice{\begin{tabu}{X[c]X[c]X[c]} $m$   & $k$   & $x$   \\ \end{tabu}}
        \wrongchoice{\begin{tabu}{X[c]X[c]X[c]} $m$   & $1/k$ & $x$   \\ \end{tabu}}
        \wrongchoice{\begin{tabu}{X[c]X[c]X[c]} $k$   & $x$   & $m$   \\ \end{tabu}}
        \wrongchoice{\begin{tabu}{X[c]X[c]X[c]} $1/k$ & $1/m$ & $x$   \\ \end{tabu}}
        \wrongchoice{\begin{tabu}{X[c]X[c]X[c]} $x$   & $1/k$ & $1/m$ \\ \end{tabu}}
    \end{choices}
\end{question}
}

%% page 44
\element{gre}{
\begin{question}{GRE0177-Q60}
    \begin{center}
    \begin{tikzpicture}
        %% NOTE: TODO: circle
    \end{tikzpicture}
    \end{center}
    An infinite, uniformly charged sheet with surface charge density $\sigma$ cuts through a spherical Gaussian surface of radius $R$ at a distance $x$ from its center,
        as shown in the figure above.
    The electric flux $\Phi$ through the Gaussian surface is:
    \begin{multicols}{2}
    \begin{choices}
        %% ANS is D
        \wrongchoice{$\dfrac{\pi R^2 \sigma}{\epsilon_0}$}
        \wrongchoice{$\dfrac{2\pi R^2 \sigma}{\epsilon_0}$}
        \wrongchoice{$\dfrac{\pi\left(R-x\right)^2 \sigma}{\epsilon_0}$}
        \wrongchoice{$\dfrac{\pi\left(R^2-x^2\right)\sigma}{\epsilon_0}$}
        \wrongchoice{$\dfrac{2\pi\left(R^2-x^2\right)\sigma}{\epsilon_0}$}
    \end{choices}
    \end{multicols}
\end{question}
}

\element{gre}{
\begin{question}{GRE0177-Q61}
    \begin{center}
    \begin{tikzpicture}
        %% NOTE: TODO: circle
    \end{tikzpicture}
    \end{center}
    An electromagnetic plane wave, propagating in vacuum, has an electric field given by $E=E_0\cos\left(kx-\omega t\right)$ and is normally incident on a perfect conductor at $x=0$,
        as shown in the figure above.
    Immediately to the left of the conductor,
        the total electric field $E$ and the total magnetic field $B$ are given by which of the following?
    \begin{center}
    \begin{tabu}{cX[c]X[c]}
        \toprule
        \makebox[1.5em][c]{\textnumero}
            & $E$
            & $B$ \\
        \bottomrule
    \end{tabu}
    \end{center}
    \begin{choices}
        \wrongchoice{\begin{tabu}{X[c]X[c]} zero                            & zero \\ \end{tabu}}
        \wrongchoice{\begin{tabu}{X[c]X[c]} $2E_0\cos\left(\omega t\right)$ & $0$ \\ \end{tabu}}
        \wrongchoice{\begin{tabu}{X[c]X[c]} zero                            & $\dfrac{2E_0}{c}\cos\left(\omega t\right)$ \\ \end{tabu}}
        \wrongchoice{\begin{tabu}{X[c]X[c]} $2E_0\cos\left(\omega t\right)$ & $\dfrac{2E_0}{c}\cos\left(\omega t\right)$ \\ \end{tabu}}
        \wrongchoice{\begin{tabu}{X[c]X[c]} $2E_0\cos\left(\omega t\right)$ & $\dfrac{2E_0}{c}\sin\left(\omega t\right)$ \\ \end{tabu}}
    \end{choices}
\end{question}
}

\element{gre}{
\begin{question}{GRE0177-Q62}
    A nonrelativistic particle with a charge twice that of an electron moves through a uniform magnetic field.
    The field has a strength of $\pi/4$ tesla and is perpendicular to the velocity of the particle.
    What is the particle's mass if it has a cylotron frequency of \num{1 600} hertz?
    \begin{multicols}{2}
    \begin{choices}
        \wrongchoice{\SI{2.5e-23}{\kilo\gram}}
        \wrongchoice{\SI{1.2e-22}{\kilo\gram}}
        \wrongchoice{\SI{3.3e-22}{\kilo\gram}}
        \wrongchoice{\SI{5.0e-21}{\kilo\gram}}
        \wrongchoice{\SI{7.5e-21}{\kilo\gram}}
    \end{choices}
    \end{multicols}
\end{question}
}

%% page 46
\element{gre}{
\begin{question}{GRE0177-Q63}
    \begin{center}
    \begin{tikzpicture}
        %% NOTE: TODO: pgfplots
    \end{tikzpicture}
    \begin{multicols}{2}
    The distribution of relative intensity $I\left((\lambda\right)$ of blackbody radiation from a solid object \emph{versus} the wavelength $\lambda$ is shown in the figure above.
    If the Wien displacement law constant is \SI{2.9e-3}{\meter\kelvin},
        what is the approximate temperature of the object?
    \begin{multicols}{2}
    \begin{choices}
        \wrongchoice{\SI{10}{\kelvin}}
        \wrongchoice{\SI{50}{\kelvin}}
        \wrongchoice{\SI{250}{\kelvin}}
        \wrongchoice{\SI{1500}{\kelvin}}
        \wrongchoice{\SI{6250}{\kelvin}}
    \end{choices}
    \end{multicols}
\end{question}
}

\element{gre}{
\begin{question}{GRE0177-Q64}
    Electromagnetic radiation provides a means to probe aspects of the physical universe.
    Which of the following statements regarding radiation spectra is \emph{not} correct?
    \begin{choices}
        \wrongchoice{Lines in the infrared, visible, and ultraviolet regions of the spectrum reveal primarily the nuclear structure of the sample.}
        \wrongchoice{The wavelengths identified in an absorptionspectrum of an element are among those in its emission spectrum.}
        \wrongchoice{Absorption spectra can be used to determine which elements are present in distant stars.}
        \wrongchoice{Spectral analysis can be used to identify the composition of galactic dust.}
        \wrongchoice{Band spectra are due to molecules.}
    \end{choices}
\end{question}
}

\element{gre}{
\begin{question}{GRE0177-Q65}
    \begin{equation*}
        C = 3kN_A \left(\frac{hv}{kT}\right)^2 \frac{\mathrm{e}^{hv/kT}}{\left(\mathrm{e}^{hv/kT}-1\right)^2}
    \end{equation*}
    Einstein's formula for the molar heat capacity $C$ of solids is given above.
    At high temperatures, $C$ approaches which of the following?
    \begin{multicols}{2}
    \begin{choices}
        \wrongchoice{zero}
        \wrongchoice{$3kN_A\left(\dfrac{hv}{kT}\right)$}
        \wrongchoice{$3kN_Ahv$}
        \wrongchoice{$3kN_A$}
        \wrongchoice{$N_Ahv$}
    \end{choices}
    \end{multicols}
\end{question}
}

\element{gre}{
\begin{question}{GRE0177-Q66}
    A sample of radioactive nuclei of a certain elements can decay only by $\gamma$-emission and $\beta$-emission.
    If the half-life for $\gamma$-emission is 24 minutes and that for $\beta$-emission is 36 minutes,
        the half-life for the sample is:
    \begin{multicols}{2}
    \begin{choices}
        \wrongchoice{30 minutes}
        \wrongchoice{24 minutes}
        \wrongchoice{20.8 minutes}
        \wrongchoice{14.4 minutes}
        \wrongchoice{6 minutes}
    \end{choices}
    \end{multicols}
\end{question}
}

\element{gre}{
\begin{question}{GRE0177-Q67}
    The \ce{^{238}U} nucleus has a binding energy of about \SI{7.5}{\mega\eV} per nucleon.
    If the nucleus were to fission into two equal fragments,
        each would have a kinetic energy of just over \SI{100}{\mega\eV}.
    From this, it can be concluded that:
    \begin{choices}
        \wrongchoice{\ce{^{238}U} cannot fission spontaneously}
        \wrongchoice{\ce{^{238}U} has a large neutron excess}
        \wrongchoice{nuclei near $A=120$ have masses greater than half that of \ce{^{238}U}}
        \wrongchoice{nuclei near $A=120$ must be bound by about \SI{6.7}{\mega\eV} per nucleon}
        \wrongchoice{nuclei near $A=120$ must be bound by about \SI{8.5}{\mega\eV} per nucleon}
    \end{choices}
\end{question}
}

%% page 48
\element{gre}{
\begin{question}{GRE0177-Q68}
    When \ce{^{7}_{4}Be} transforms into \ce{^{7}_{3}Li}, it does so by:
    \begin{choices}
        \wrongchoice{emitting an alpha particle only}
        \wrongchoice{emitting an electron only}
        \wrongchoice{emitting a neutron only}
        \wrongchoice{emitting a positron only}
        \wrongchoice{electron capture by the nucleus with emission of a neutrino}
    \end{choices}
\end{question}
}

\element{gre}{
\begin{question}{GRE0177-Q69}
    Blue light of wavelength 480 nanometers is most strongly reflected off a thin film of oil on a glass slide when viewed near normal incidence.
    Assuming that the index of refration of the oil is 1.2 and that of the glass is 1.6,
        what is teh minimum thickness of the oil film (other than zero)?
    \begin{multicols}{2}
    \begin{choices}
        \wrongchoice{\SI{150}{\nano\meter}}
        \wrongchoice{\SI{200}{\nano\meter}}
        \wrongchoice{\SI{300}{\nano\meter}}
        \wrongchoice{\SI{400}{\nano\meter}}
        \wrongchoice{\SI{480}{\nano\meter}}
    \end{choices}
    \end{multicols}
\end{question}
}

\element{gre}{
\begin{question}{GRE0177-Q70}
    Light from a laser falls on a pair of very narrow slits separated by 0.5 micrometer, and bright fringes separated by 1.0 millimeter are observed on a distant screen.
    If the frequency of the laser light is doubled,
        what will be the separation of the bright fringes?
    \begin{multicols}{2}
    \begin{choices}
        \wrongchoice{\SI{0.25}{\milli\meter}}
        \wrongchoice{\SI{0.5}{\milli\meter}}
        \wrongchoice{\SI{1.0}{\milli\meter}}
        \wrongchoice{\SI{2.0}{\milli\meter}}
        \wrongchoice{\SI{2.5}{\milli\meter}}
    \end{choices}
    \end{multicols}
\end{question}
}

\element{gre}{
\begin{question}{GRE0177-Q71}
    The ultraviolet Lyman alpha line of hydrogen with wavelength 121.5 nanometers is emitted by an astronomical object.
    An observer on Earth measures the wavelength of the light received from the object to be 607.5 nanometers.
    The observer can conclude that the object is moving with a radial velocity of:
    \begin{choices}
        \wrongchoice{\SI{2.4e8}{\meter\per\second} toward Earth}
        \wrongchoice{\SI{2.8e8}{\meter\per\second} toward Earth}
        \wrongchoice{\SI{2.4e8}{\meter\per\second} away from Earth}
        \wrongchoice{\SI{2.8e8}{\meter\per\second} away from Earth}
        \wrongchoice{\SI{12e8}{\meter\per\second} from Earth}
    \end{choices}
\end{question}
}

\element{gre}{
\begin{question}{GRE0177-Q72}
    \begin{center}
    \begin{tikzpicture}
        %% NOTE: TODO: tikzpicture
    \end{tikzpicture}
    \end{center}
    Two identical blocks are connected by a spring.
    The combination is suspended, at rest, from a string attached to the ceiling,
        as shown in the figure above.
    The string breaks suddenly.
    Immediately after the string breaks,
        what is the downward acceleration of the upper block?
    \begin{multicols}{2}
    \begin{choices}
        \wrongchoice{zero}
        \wrongchoice{$\dfrac{g}{2}$}
        \wrongchoice{$g$}
        \wrongchoice{$\sqrt{2}g$}
        \wrongchoice{$2g$}
    \end{choices}
    \end{multicols}
\end{question}
}

\element{gre}{
\begin{question}{GRE0177-Q73}
    \begin{center}
    \begin{tikzpicture}
        %% NOTE: TODO: tikzpicture
    \end{tikzpicture}
    \end{center}
    For the system consisting of the two blocks shown in the figure above,
        the minimum horizontal force $F$ is applied so that block $B$ does not fall under the influence of gravity.
    The masses of $A$ and $B$ are \SI{16.0}{\kilo\gram} and \SI{4.00}{\kilo\gram} respectively.
    The horizontal surface is frictionless and the coefficient of friction between the two blocks is 0.50.
    The magnitude of $F$ is most nearly:
    \begin{multicols}{2}
    \begin{choices}
        \wrongchoice{\SI{50}{\newton}}
        \wrongchoice{\SI{100}{\newton}}
        \wrongchoice{\SI{200}{\newton}}
        \wrongchoice{\SI{400}{\newton}}
        \wrongchoice{\SI{1 600}{\newton}}
    \end{choices}
    \end{multicols}
\end{question}
}

%% page 50
\element{gre}{
\begin{question}{GRE0177-Q74}
    The Lagrangia for a mechanical system is
    \begin{equation*}
        L = a \dot{q}^2 + b q^4\,,
    \end{equation*}
    where $q$ is a generalized coordinate and $a$ and $b$ are constants.
    The equation of motion for this system is:
    \begin{multicols}{2}
    \begin{choices}
        \wrongchoice{$\dot{q} = \sqrt{\dfrac{b}{a}} q^2$}
        \wrongchoice{$\dot{q} = \dfrac{2b}{a} q^3$}
        \wrongchoice{$\ddot{q} = -\dfrac{b}{a} q^2$}
        \wrongchoice{$\ddot{q} = +\dfrac{2b}{a} q^2$}
        \wrongchoice{$\ddot{q} = \dfrac{b}{a} q^2$}
    \end{choices}
    \end{multicols}
\end{question}
}

\element{gre}{
\begin{question}{GRE0177-Q75}
    \begin{equation*}
        %% NOTE: TODO: fill matrix with vecotrs
    \end{equation*}
    The matrix shown above transforms teh components of a vector in one coordinate frame $S$ to the components of the same vector in a second coordinate frame $S\prime$.
    This matrix represents a rotation of the reference frame $S$ by:
    \begin{choices}
        \wrongchoice{\ang{30} clockwise about the $x$-axis}
        \wrongchoice{\ang{30} counterclockwise about the $z$-axis}
        \wrongchoice{\ang{45} clockwise about the $z$-axis}
        \wrongchoice{\ang{60} clockwise about the $y$-axis}
        \wrongchoice{\ang{60} counterclockwise about the $z$-axis}
    \end{choices}
\end{question}
}

\element{gre}{
\begin{question}{GRE0177-Q76}
    The mean kinetic energy of the conduction electrons in metals is ordinarily much higher than $kT$ because:
    \begin{choices}
        \wrongchoice{electrons have many more degrees of freedome than atoms do}
        \wrongchoice{the electrons and the lattice are not in thermal equilibrium}
        \wrongchoice{the electrons form a degenerate Fermi gas}
        \wrongchoice{electrons in metals are highly relativistic}
        \wrongchoice{electrons interact strongly with phonons}
    \end{choices}
\end{question}
}

\element{gre}{
\begin{question}{GRE0177-Q77}
    An ensemble of systems in thermal equilibrium with a resevoir for which $kT=\SI{0.025}{\eV}$.
    State $A$ has an energy that is \SI{0.1}{\eV} above that of state $B$.
    If it is assumed the systems obey Maxwell-Boltmann statistics and that the degeneracies of the two states are the same,
        the the ratio of the number of systems in state $A$ to the number in state $B$ is:
    \begin{multicols}{2}
    \begin{choices}
        \wrongchoice{$\mathrm{e}^{+4}$}
        \wrongchoice{$\mathrm{e}^{+0.25}$}
        \wrongchoice{$1$}
        \wrongchoice{$\mathrm{e}^{-0.25}$}
        \wrongchoice{$\mathrm{e}^{-4}$}
    \end{choices}
    \end{multicols}
\end{question}
}

\element{gre}{
\begin{question}{GRE0177-Q78}
    The muon decays with a characteristic lifetime of about \num{e-6} second into an electron,
        a muon neutrino, and an electron antineutrino.
    The muon is forbidden from decaying into an electron and just a single neutrino by the law of conservation of:
    \begin{choices}
        \wrongchoice{charge}
        \wrongchoice{mass}
        \wrongchoice{energy and momentum}
        \wrongchoice{baryon number}
        \wrongchoice{lepton number}
    \end{choices}
\end{question}
}

%% NOTE: double check clight on Q79 and Q80
\element{gre}{
\begin{question}{GRE0177-Q79}
    A particle leaving a cyclotron has a total relativistic energy of \SI{10}{\giga\eV} and a relativistic momentum of \SI{8}{\giga\eV\per\clight}.
    What is the rest mass of this particle?
    \begin{multicols}{2}
    \begin{choices}
        \wrongchoice{\SI{0.25}{\gige\eV\per\clight}}
        \wrongchoice{\SI{1.20}{\gige\eV\per\clight}}
        \wrongchoice{\SI{2.00}{\gige\eV\per\clight}}
        \wrongchoice{\SI{6.00}{\gige\eV\per\clight}}
        \wrongchoice{\SI{16.0}{\gige\eV\per\clight}}
    \end{choices}
    \end{multicols}
\end{question}
}

%% page 52
\element{gre}{
\begin{question}{GRE0177-Q80}
    A tube of water is traveling at $1/2 c$ relative to the lab frame when a beam of light traveling in the same direction in the water relative to the lab frame?
    (The index of refraction of water is $4/3$.)
    \begin{multicols}{2}
    \begin{choices}
        \wrongchoice{\SI{1/2}{\clight}}
        \wrongchoice{\SI{2/3}{\clight}}
        \wrongchoice{\SI{5/6}{\clight}}
        \wrongchoice{\SI{10/11}{\clight}}
        \wrongchoice{$c$}
    \end{choices}
    \end{multicols}
\end{question}
}

\element{gre}{
\begin{question}{GRE0177-Q81}
    Which of the following is the orbital angular momentum eigenfunction $Y_{l}^{m}\left(\theta,\phi\right)$ in a state for which the operators $\mathbf{L}^2$ and $L_z$ have eigenvalues $6\hbar^2$ and $-\hbar$, respectively?
    \begin{multicols}{2}
    \begin{choices}
        \wrongchoice{$Y_{2}^{1}\left(\theta,\phi\right)$}
        \wrongchoice{$Y_{2}^{-1}\left(\theta,\phi\right)$}
        \wrongchoice{$\dfrac{1}{\sqrt{2}}\left[Y_{2}^{1}\left(\theta,\phi\right)+Y_{2}^{-1}\left(\theta,\phi\right)\right]$}
        \wrongchoice{$Y_{3}^{2}\left(\theta,\phi\right)$}
        \wrongchoice{$Y_{3}^{-1}\left(\theta,\phi\right)$}
    \end{choices}
    \end{multicols}
\end{question}
}

\element{gre}{
\begin{question}{GRE0177-Q82}
    Let $\ket{\alpha}$ represent the state of an electron with spin up and $\ket{\beta}$ the state of an electron with spin down.
    Valid spin eigenfunctoins for a triplet state $\left(^{3}S\right)$ of a two-electron atom include which of the following?
    \begin{itemize}
        \item[I.] $\ket{\alpha}_1 \ket{\alpha}_2$
        \item[I.] $\dfrac{1}{sqrt{2}}\left(\ket{\alpha}_{1}\ket{\beta}_{2}-\ket{\alpha}_{2}\ket{\beta}_{1}\right)$
        \item[I.] $\dfrac{1}{sqrt{2}}\left(\ket{\alpha}_{1}\ket{\beta}_{2}+\ket{\alpha}_{2}\ket{\beta}_{1}\right)$
    \end{itemize}
    \begin{choices}
        \wrongchoice{I only}
        \wrongchoice{II only}
        \wrongchoice{III only}
        \wrongchoice{I and III}
        \wrongchoice{II and III}
    \end{choices}
\end{question}
}

\element{gre}{
\begin{question}{GRE0177-Q83}
    The state of a spin-$\frac{1}{2}$ particle can be represented using the eigenstates $\ket{\uparrow}$ and $\ket{\downarrow}$ of the $S_z$ operator.
    \begin{align*}
        S_z \ket{\upparrow}  &= \frac{1}{2} h \ket{\uparrow} \\
        S_z \ket{\downarrow} &= -\frac{1}{2} h \ket{\downarrow}
    \end{align*}
    Given the Pauli matrix $\sigma_x = 
        \begin{bmatrix}
            0 & 1 \\
            1 & 0 \\
        \end{bmatrix}$,
        which of the following is an eigenstate of $S_x$ with eigenvalue $-\dfrac{1}{2}\hbar$?
    \begin{multicols}{2}
    \begin{choices}
        \wrongchoice{$\ket{\downarrow}$}
        \wrongchoice{$\dfrac{1}{\sqrt{2}}\left(\ket{\uparrow}+\ket{\downarrow}\right)$}
        \wrongchoice{$\dfrac{1}{\sqrt{2}}\left(\ket{\uparrow}-\ket{\downarrow}\right)$}
        \wrongchoice{$\dfrac{1}{\sqrt{2}}\left(\ket{\uparrow}+i\ket{\downarrow}\right)$}
        \wrongchoice{$\dfrac{1}{\sqrt{2}}\left(\ket{\uparrow}-i\ket{\downarrow}\right)$}
    \end{choices}
    \end{multicols}
\end{question}
}

\element{gre}{
\begin{question}{GRE0177-Q84}
    \begin{center}
    \begin{tikzpicture}
        %% NOTE: TODO:
    \end{tikzpicture}
    \end{center}
    An energy-level diagram of the $n=1$ and $n=2$ levels of atomic hydrogen (including the effects of spin-orbit couplnig and relativity) is shown in the figure above.
    Three transitions are labeled $A$, $B$, and $C$.
    Which of the transitions will be possible electric-dipole transitions?
    \begin{multicols}{2}
    \begin{choices}
        \wrongchoice{$B$ only}
        \wrongchoice{$C$ only}
        \wrongchoice{$A$ and $C$ only}
        \wrongchoice{$B$ and $C$ only}
        \wrongchoice{$A$, $B$ and $C$}
    \end{choices}
    \end{multicols}
\end{question}
}

%% page 54
\element{gre}{
\begin{question}{GRE0177-Q85}
    One end of a Nichrome wire of length $2L$ and cross-sectional area $A$ is attached to an end of another Nichrome wire of length $L$ and cross-sectional area $2A$.
    If the free end of the longer wire is at an electric potential of 8.0 volts,
        and the free end of the shorter wire is at an electric potential of 1.0 volt,
        the potential at the junction of the two wires is most nearly equal to:
    \begin{multicols}{2}
    \begin{choices}
        \wrongchoice{\SI{2.4}{\volt}}
        \wrongchoice{\SI{3.3}{\volt}}
        \wrongchoice{\SI{4.5}{\volt}}
        \wrongchoice{\SI{5.7}{\volt}}
        \wrongchoice{\SI{6.6}{\volt}}
    \end{choices}
    \end{multicols}
\end{question}
}

\element{gre}{
\begin{question}{GRE0177-Q86}
    \begin{center}
    \begin{tikzpicture}
        %% NOTE: TODO:
    \end{tikzpicture}
    \end{center}
    A coil of 15 turns, each of radius 1 centimeter,
    is rotating at a constant angular velocity $\omega=\SI{300}{\radian\per\second}$ in a uniform magnetic field of 0.5 tesla,
        as shown in the figure above.
    Assume at time $t=0$ that the normal $\mathbf{\hat{n}}$ to the coil plane is along the $y$-direction and that the self-inductance of the coil can be neglected.
    If the coil resistance is 9 ohms,
        what will be the magnitude of the induced current in milliamperes?
    \begin{multicols}{2}
    \begin{choices}
        \wrongchoice{$22\pi\sin\omega t$}
        \wrongchoice{$50\pi\sin\omega t$}
        \wrongchoice{$0.08\pi\cos\omega t$}
        \wrongchoice{$1.7\pi\cos\omega t$}
        \wrongchoice{$25\pi\cos\omega t$}
    \end{choices}
    \end{multicols}
\end{question}
}

%% page 56
\element{gre}{
\begin{question}{GRE0177-Q87}
    \begin{center}
    \begin{tikzpicture}
        %% NOTE: TODO:
    \end{tikzpicture}
    \end{center}
    Two spherical, nonconducting, and very thin shells of uniformly distributed positive charge $Q$ and radius $d$ are located a distance $10d$ from each other.
    A positive point charge $q$ is placed inside one of the shells at a distance $d/2$ from the center,
        on the line connecting the centers of the two shells,
        as shown in the figure above.
    What is the net force on the charge $q$?
    \begin{multicols}{2}
    \begin{choices}
        \wrongchoice{$\dfrac{qQ}{361\pi\epsilon_0 d^2}$ to the left}
        \wrongchoice{$\dfrac{qQ}{361\pi\epsilon_0 d^2}$ to the right}
        \wrongchoice{$\dfrac{qQ}{441\pi\epsilon_0 d^2}$ to the left}
        \wrongchoice{$\dfrac{qQ}{441\pi\epsilon_0 d^2}$ to the right}
        \wrongchoice{$\dfrac{360qQ}{361\pi\epsilon_0 d^2}$ to the left}
    \end{choices}
    \end{multicols}
\end{question}
}

%% page 58
\element{gre}{
\begin{question}{GRE0177-Q88}
    \begin{center}
    \begin{tikzpicture}
        %% NOTE: TODO:
    \end{tikzpicture}
    \end{center}
    A segment of wire is bent into an arc of radius $R$ and subtended angle $\theta$,
        as shown in the figure above.
    Point $P$ is at the center of the circular segment.
    The wire carries current $I$.
    What is the magnitude of the field at $P$?
    \begin{multicols}{2}
    \begin{choices}
        \wrongchoice{zero}
        \wrongchoice{$\dfrac{\mu_0 I\theta}{\left(2\pi\right)^{2}R}$}
        \wrongchoice{$\dfrac{\mu_0 I\theta}{4\pi R}$}
        \wrongchoice{$\dfrac{\mu_0 I\theta}{4\pi R^2}$}
        \wrongchoice{$\dfrac{\mu_0 I}{2\theta R^2}$}
    \end{choices}
    \end{multicols}
\end{question}
}

\element{gre}{
\begin{question}{GRE0177-Q89}
    \begin{center}
    \begin{tikzpicture}
        %% NOTE: TODO:
    \end{tikzpicture}
    \end{center}
    A child is standing on the edge of a merry-go-round that has the shape of a solid disk,
        as shown in the figure above.
    The mass of the child is 40 kilograms.
    The merry-go-round has a mass of 200 kilograms and a radius of 2.5 meters,
        and it is rotating with an angular velocity of $\omega=\SI{2.0}{\radian\per\second}$.
    The child then walks slowly toward the center of the merry-go-round.
    What will be the final angular velocity of the merry-go-round when the child reaches the center?
    (The size of the child can be neglected.)
    \begin{multicols}{2}
    \begin{choices}
        \wrongchoice{\SI{2.0}{\radian\per\second}}
        \wrongchoice{\SI{2.2}{\radian\per\second}}
        \wrongchoice{\SI{2.4}{\radian\per\second}}
        \wrongchoice{\SI{2.6}{\radian\per\second}}
        \wrongchoice{\SI{2.8}{\radian\per\second}}
    \end{choices}
    \end{multicols}
\end{question}
}

%% page 60
\element{gre}{
\begin{question}{GRE0177-Q90}
    \begin{center}
    \begin{tikzpicture}
        %% NOTE: TODO:
    \end{tikzpicture}
    \end{center}
    Two identical springs with spring constant $k$ are connected to identical masses of mass $M$,
        as shown in the figures above.
    The ratio of the period for the springs connected in parallel (Figure 1) to the period for the springs connected in series (Figure 2) is:
    \begin{multicols}{3}
    \begin{choices}
        \wrongchoice{$\dfrac{1}{2}$}
        \wrongchoice{$\dfrac{1}{\sqrt{2}}$}
        \wrongchoice{$1$}
        \wrongchoice{$\sqrt{2}$}
        \wrongchoice{$2$}
    \end{choices}
    \end{multicols}
\end{question}
}

%% page 62
\element{gre}{
\begin{question}{GRE0177-Q91}
    \begin{center}
    \begin{tikzpicture}
        %% NOTE: TODO:
    \end{tikzpicture}
    \end{center}
    The cylinder shown above, with mass $M$ and radius $R$,
        has a radially dependent density.
    The cylinder starts from rest and rolls without slipping down an inclined plane of height $H$.
    At the bottom of the plane its translational speed is $\sqrt{\dfrac{8gH}{7}}$.
    Which of the following is the rotational inertia of the clinder?
    \begin{multicols}{3}
    \begin{choices}
        \wrongchoice{$\dfrac{1}{2}MR^2$}
        \wrongchoice{$\dfrac{3}{4}MR^2$}
        \wrongchoice{$\dfrac{7}{8}MR^2$}
        \wrongchoice{$MR^2$}
        \wrongchoice{$\dfrac{7}{4}MR^2$}
    \end{choices}
    \end{multicols}
\end{question}
}

\element{gre}{
\begin{question}{GRE0177-Q92}
    \begin{center}
    \begin{tikzpicture}
        %% NOTE: TODO:
    \end{tikzpicture}
    \end{center}
    Two small equal masses $m$ are connected by an ideal massless spring that has equilibrium length $l_0$ and force constant $k$, as shown in the figure above.
    The system is free to move without friction in the plane of the page.
    If $p_1$ and $p_2$ represent the magnitudes of the momenta of the two masses,
        a Hamiltonian for this system is:
    \begin{choices}
        \wrongchoice{$\dfrac{1}{2}\{\dfrac{p_1^2}{m}+\dfrac{p_2^2}{m} - 2k\left(l-l_0\right)\}$}
        \wrongchoice{$\dfrac{1}{2}\{\dfrac{p_1^2}{m}+\dfrac{p_2^2}{m} + 2k\left(l-l_0\right)^2\}$}
        \wrongchoice{$\dfrac{1}{2}\{\dfrac{p_1^2}{m}+\dfrac{p_2^2}{m} - k\left(l-l_0\right)\}$}
        \wrongchoice{$\dfrac{1}{2}\{\dfrac{p_1^2}{m}+\dfrac{p_2^2}{m} - k\left(l-l_0\right)^2\}$}
        \wrongchoice{$\dfrac{1}{2}\{\dfrac{p_1^2}{m}+\dfrac{p_2^2}{m} + k\left(l-l_0\right)^2\}$}
    \end{choices}
\end{question}
}

%% page 64
\element{gre}{
\begin{question}{GRE0177-Q93}
    The solution to the Schr\"{o}dinger equation for the ground state of hydrogen is:
    \begin{equation*}
        \Psi_0 = \dfrac{1}{\sqrt{\pi a_0^3} \mathrm{e}^{-r/a_0}\, ,
    \end{equation*}
    where $a_0$ is the Bohr radius and $r$ is the distance from the origin.
    Which of the following is the most probably value for $r$?
    \begin{multicols}{3}
    \begin{choices}
        \wrongchoice{zero}
        \wrongchoice{$\dfrac{a_0}{2}$}
        \wrongchoice{$a_0$}
        \wrongchoice{$2a_0$}
        \wrongchoice{$\infty$}
    \end{choices}
    \end{multicols}
\end{question}
}

\element{gre}{
\begin{question}{GRE0177-Q94}
    The raising and lowering operator for the quantum harmonic oscillator satisfy
    \begin{equation*}
        a^{\dagger}\ket{n} = \sqrt{n+1}\ket{n+1},\quad a\ket{n} = \sqrt{n}\ket{n-1}
    \end{equation*}
    for energy eigenstates $\ket{n}$ with energy $E_n$.
    Which of the following gives the first-order shift in the $n=2$ energy level due to the perturbation
    \begin{equation*}
        \Delta H = V \left(a + a^{\dagger}\right)^2
    \end{equation*}
    where $V$ is a constant?
    \begin{multicols}{3}
    \begin{choices}
        \wrongchoice{zero}
        \wrongchoice{$V$}
        \wrongchoice{$\sqrt{2} V$}
        \wrongchoice{$2\sqrt{2} V$}
        \wrongchoice{$5 V$}
    \end{choices}
    \end{multicols}
\end{question}
}

\element{gre}{
\begin{question}{GRE0177-Q95}
    \begin{center}
    \begin{tikzpicture}
        %% NOTE: TODO:
    \end{tikzpicture}
    \end{center}
    An infinite slab of insulating material with dielectric constant $K$ and permitivity $\epsilon=K\epsilon_0$ is placed in a uniform electric field of magnitude $E_0$.
    The field is perpendicular to the surface of the material,
        as shown in the figure above.
    The magnitude of the electric field inside the material is:
    \begin{multicols}{3}
    \begin{choices}
        \wrongchoice{$\dfrac{E_0}{K}$}
        \wrongchoice{$\dfrac{E_0}{K\epsilon_0}$}
        \wrongchoice{$E_0$}
        \wrongchoice{$K\epsilon_0 E_0$}
        \wrongchoice{$K E_0$}
    \end{choices}
    \end{multicols}
\end{question}
}

\element{gre}{
\begin{question}{GRE0177-Q96}
    A uniformly charged sphere of total charge $Q$ expands and contracts between radii $R_1$ and $R_2$ at a frequency $f$.
    The total power radiated by the sphere is:
    \begin{choices}
        \wrongchoice{proportional to $Q$}
        \wrongchoice{proportional to $f^2$}
        \wrongchoice{proportional to $f^4$}
        \wrongchoice{proportional to $\left(\dfrac{R_2}{R_2}\right)$}
        \wrongchoice{zero}
    \end{choices}
\end{question}
}

%% page 66
\element{gre}{
\begin{question}{GRE0177-Q97}
    \begin{center}
    \begin{tikzpicture}
        %% NOTE: TODO:
    \end{tikzpicture}
    \end{center}
    A beam of light has a small wavelength spread $\delta\lambda$ about a central wavelength $\lambda$.
    The beam travels in vacuum until it enters a glass plate at an angle $\theta$ relative to the normal to the plate,
        as shown in the figure above.
    The index of refraction of the glass is given by $n\left(\lambda\right)$.
    The angular spread $\delta\theta\prime$ of the refracted beam is given by:
    \begin{choices}
        \wrongchoice{$\delta\theta\prime = \left|\dfrac{1}{n}\delta\lambda\right|$}
        \wrongchoice{$\delta\theta\prime = \left|\dfrac{dn(\lambda)}{d\lambda}\delta\lambda\right|$}
        \wrongchoice{$\delta\theta\prime = \left|\dfrac{1}{\lambda}\dfrac{d\lambda}{dn}\delta\lambda\right|$}
        \wrongchoice{$\delta\theta\prime = \left|\dfrac{\sin\theta}{\sin\theta\prime}\dfrac{\delta\lambda}{\lambda}\delta\lambda\right|$}
        \wrongchoice{$\delta\theta\prime = \left|\dfrac{\tan\theta\prime}{n}\dfrac{dn(\lambda)}{d\lambda}\delta\lambda\right|$}
    \end{choices}
\end{question}
}

\element{gre}{
\begin{question}{GRE0177-Q98}
    Support that a system in quantum state $i$ has energy $E_i$.
    In thermal equilibrium, the expression
    \begin{equation*}
        \dfrac{ }{ }
    \end{equation*}
    represents which of the following?
    \begin{choices}
        \wrongchoice{The average energy of the system}
        \wrongchoice{The partition function}
        \wrongchoice{Unity}
        \wrongchoice{The probability to find the system with energy $E_i$}
        \wrongchoice{The entropy of the system}
    \end{choices}
\end{question}
}

\element{gre}{
\begin{question}{GRE0177-Q99}
    A photon strikes an electron of mass $m$ that is inititially at rest,
        creating an electron-positron pair.
    The photon is destroyed and the positron and two electrons move off at equal speeds along the initial direction of the photon.
    The energy of the photon was:
    \begin{multicols}{2}
    \begin{choices}
        \wrongchoice{$mc^2$}
        \wrongchoice{$2mc^2$}
        \wrongchoice{$3mc^2$}
        \wrongchoice{$4mc^2$}
        \wrongchoice{$5mc^2$}
    \end{choices}
    \end{multicols}
\end{question}
}

%% page 68
\element{gre}{
\begin{question}{GRE0177-Q100}
    \begin{center}
    \begin{tikzpicture}
        %% NOTE: TODO:
    \end{tikzpicture}
    \end{center}
    A Michelson interferometer is configured as a wavemeter,
        as shown in the figure above, so that a ratio of fringe counts may be used to compare the wavelengths of two lasers with high precision.
    When the mirror in the right arm of the interferometer is translated through a distance $d$, \num{100 000} interference fringes pass across the detector for green light and \num{85 865} fringes pass across the detector for red $\left(\lambda=\SI{632.82}{\nano\meter}\right)$ light.
    The wavelength of the green light is:
    \begin{multicols}{2}
    \begin{choices}
        \wrongchoice{\SI{500.33}{\nano\meter}}
        \wrongchoice{\SI{543.37}{\nano\meter}}
        \wrongchoice{\SI{590.19}{\nano\meter}}
        \wrongchoice{\SI{736.99}{\nano\meter}}
        \wrongchoice{\SI{858.65}{\nano\meter}}
    \end{choices}
    \end{multicols}
\end{question}
}


\endinput


