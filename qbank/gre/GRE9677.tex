

%% http://ctan.mirrorcatalogs.com/macros/latex/contrib/physics/physics.pdf
%%--------------------------------------------------------------------------------

%\newcommand{\dd}{\mathop{}\,\mathrm{d}}

%% GRE Physics 9677 Practice Exam
%%----------------------------------------

%% Page 12
\element{gre}{
\begin{question}{GRE9677-Q01}
    \begin{center}
    \begin{tikzpicture}
        %% NOTE: TODO: figure 1 and 2
    \end{tikzpicture}
    \end{center}
    The capacitor shown shown in Figure 1 above is charged by connecting switch $S$ to contact $a$.
    If switch $S$ is thrown to contact $b$ at time $t=0$,
        which of the curves in Figure 2 above represents the magnitude of the current through the resistor $R$ as a function of time?
    \begin{multicols}{3}
    \begin{choices}[o]
        \wrongchoice{$A$}
        \wrongchoice{$B$}
        \wrongchoice{$C$}
        \wrongchoice{$D$}
        \wrongchoice{$E$}
    \end{choices}
    \end{multicols}
\end{question}
}

\element{gre}{
\begin{question}{GRE9677-Q02}
    \begin{center}
    \begin{tikzpicture}
        %% NOTE: TODO: figure 1 and 2
    \end{tikzpicture}
    \end{center}
    The circuit shown above is in a uniform magnetic field that is into the page and is decreasing in magnitude at the rate of 150 tesla/second.
    The ammeter reads:
    \begin{multicols}{3}
    \begin{choices}
        \wrongchoice{\SI{0.15}{\ampere}}
        \wrongchoice{\SI{0.35}{\ampere}}
        \wrongchoice{\SI{0.50}{\ampere}}
        \wrongchoice{\SI{0.65}{\ampere}}
        \wrongchoice{\SI{0.80}{\ampere}}
    \end{choices}
    \end{multicols}
\end{question}
}

%% page 14
%% questions 3-4
\element{gre}{
\begin{question}{GRE9677-Q03}
    Questions 3-4 refer to a thin, nonconducting ring of radius $R$, as shown below,
        which has a charge $Q$ uniformly spread out on it.
    \begin{center}
        %% \newcommand
    \end{center}
    %% start question
    The electric potential at a point $P$,
        which is located on the axis of symmetry a distance $x$ from the center of the ring, is given by:
    \begin{multicols}{2}
    \begin{choices}
        \wrongchoice{$\dfrac{Q}{4\pi\epsilon_0 x}$}
        \wrongchoice{$\dfrac{Q}{4\pi\epsilon_0\sqrt{R^2+x^2}}$}
        \wrongchoice{$\dfrac{Qx}{4\pi\epsilon_0\left(R^2+x^2\right)}$}
        \wrongchoice{$\dfrac{Qx}{4\pi\epsilon_0\left(R^2+x^2\right)^{3/2}}$}
        \wrongchoice{$\dfrac{QR}{4\pi\epsilon_0\left(R^2+x^2\right)}$}
    \end{choices}
    \end{multicols}
\end{question}
}

\element{gre}{
\begin{question}{GRE9677-Q04}
    Questions 3-4 refer to a thin, nonconducting ring of radius $R$, as shown below,
        which has a charge $Q$ uniformly spread out on it.
    \begin{center}
        %% \newcommand
    \end{center}
    %% start question
    A small particle of mass $m$ and charge $-q$ is placed at point $P$ and released.
    If $R>>x$, the particle will undergo oscillations along the axis of symmetry with an angular frequency that is equal to:
    \begin{multicols}{2}
    \begin{choices}
        \wrongchoice{$\sqrt{\dfrac{qQ}{4\pi\epsilon_0 mR^3}}$}
        \wrongchoice{$\sqrt{\dfrac{qQx}{4\pi\epsilon_0 mR^4}}$}
        \wrongchoice{$\dfrac{qQx}{4\pi\epsilon_0 mR^3}$}
        \wrongchoice{$\dfrac{qQx}{4\pi\epsilon_0 mR^4}$}
        \wrongchoice{$\sqrt{\dfrac{qQx}{4\pi\epsilon_0 m}}\dfrac{1}{R^2+x^2}$}
    \end{choices}
    \end{multicols}
\end{question}
}

%% page 16
\element{gre}{
\begin{question}{GRE9677-Q05}
    \begin{center}
    \begin{tikzpicture}
        %% NOTE: TODO: figure 1 and 2
    \end{tikzpicture}
    \end{center}
    A car travels with constant speed on a circular road on level ground.
    In the diagram above, $\mathbf{F}_{air}$ is the force of air resistance on the car.
    Which of the other forces shown best represents the horizontal force of the road on the car's tires?
    \begin{multicols}{3}
    \begin{choices}[o]
        \wrongchoice{$\mathbf{F}_A$}
        \wrongchoice{$\mathbf{F}_B$}
        \wrongchoice{$\mathbf{F}_C$}
        \wrongchoice{$\mathbf{F}_D$}
        \wrongchoice{$\mathbf{F}_E$}
    \end{choices}
    \end{multicols}
\end{question}
}

\element{gre}{
\begin{question}{GRE9677-Q06}
    \begin{center}
    \begin{tikzpicture}
        %% NOTE: TODO: figure 1 and 2
    \end{tikzpicture}
    \end{center}
    A block of mass $m$ sliding down an incline at constant speed is initially at a height $h$ above the ground,
        as shown in the figure above.
    The coefficient of kinetic friction between the mass and the incline is $\mu$.
    If teh mass continues to slide down the incline at a constant speed,
        how much energy is dissipated by friction by the time the mass reaches the bottom of the incline?
    \begin{multicols}{2}
    \begin{choices}
        \wrongchoice{$\dfrac{mgh}{\mu}$}
        \wrongchoice{$mgh$}
        \wrongchoice{$\dfrac{\mu{}mgh}{\sin\theta}$}
        \wrongchoice{$mgh\sin\theta$}
        \wrongchoice{zero}
    \end{choices}
    \end{multicols}
\end{question}
}

\element{gre}{
\begin{question}{GRE9677-Q07}
    \begin{center}
    \begin{tikzpicture}
        %% NOTE: TODO: figure 1 and 2
    \end{tikzpicture}
    \end{center}
    As shown above, a ball of mass $m$, suspended on the end of a wire,
        is released from height $h$ and collides elastically,
        when it is at its lowest point,
        with a block of mass $2m$ at rest on a frictionless surface.
    After the collision, the ball rises to a final height equal to:
    \begin{multicols}{3}
    \begin{choices}
        \wrongchoice{$\dfrac{1}{9} h$}
        \wrongchoice{$\dfrac{1}{8} h$}
        \wrongchoice{$\dfrac{1}{3} h$}
        \wrongchoice{$\dfrac{1}{2} h$}
        \wrongchoice{$\dfrac{2}{3} h$}
    \end{choices}
    \end{multicols}
\end{question}
}

\element{gre}{
\begin{question}{GRE9677-Q08}
    A particle of mass $m$ undergoes harmonic oscillation with period $T_0$.
    A force $f$ proportional to the speed $v$ of the particle, $f=-bv$,
        is introduced.
    If the particle continues to oscillate,
        the period with $f$ acting is:
    \begin{choices}
        \wrongchoice{larger than $T_0$}
        \wrongchoice{smaller than $T_0$}
        \wrongchoice{independent of $b$}
        \wrongchoice{dependent linearly on $b$}
        \wrongchoice{constant changing}
    \end{choices}
\end{question}
}

\element{gre}{
\begin{question}{GRE9677-Q09}
    In the spectrum of hydrogen,
        what is the ratio of the longest wavelength in the Lyman series ($n_f=1$) to the longest wavelength in the Balmer series ($n_f=2$)?
    \begin{multicols}{3}
    \begin{choices}
        \wrongchoice{$\dfrac{5}{27}$}
        \wrongchoice{$\dfrac{1}{3}$}
        \wrongchoice{$\dfrac{4}{9}$}
        \wrongchoice{$\dfrac{3}{2}$}
        \wrongchoice{$3$}
    \end{choices}
    \end{multicols}
\end{question}
}

%% page 18
\element{gre}{
\begin{question}{GRE9677-Q10}
    Internal conversion is the process whereby an excited nucleus transfers its energy directly to one of the most bound atomic electrons,
        causing the electron to be ejected from the atom and leaving the atom in an excited state.
    The most probably process after an internal conversion electron is ejected from an atom with a high atomic number is that the:
    \begin{choices}
        \wrongchoice{atom returns to its ground state through inelastic collisions with other atoms}
        \wrongchoice{atom emits one or several x-rays}
        \wrongchoice{atom emits a $\gamma$-ray}
        \wrongchoice{atom emits an electron}
        \wrongchoice{atom emits a positron}
    \end{choices}
\end{question}
}

\element{gre}{
\begin{question}{GRE9677-Q11}
    A beam of neutral atoms in their ground state is moving into the plane of this page and passes through a region of a strong inhomogenous magnetic field that is directed upward in the plane of the page.
    After the beam passes through this field,
        a detector would find that it has been:
    \begin{choices}
        \wrongchoice{deflected upward}
        \wrongchoice{deflected to the right}
        \wrongchoice{undeflected}
        \wrongchoice{split vertically into two beams}
        \wrongchoice{split horizontally into three beams}
    \end{choices}
\end{question}
}

\element{gre}{
\begin{question}{GRE9677-Q12}
    The ground state of positronium is most nearly equal to:
    \begin{multicols}{3}
    \begin{choices}
        \wrongchoice{\SI{-27.2}{\eV}}
        \wrongchoice{\SI{-13.6}{\eV}}
        \wrongchoice{\SI{-6.8}{\eV}}
        \wrongchoice{\SI{-3.4}{\eV}}
        \wrongchoice{\SI{13.6}{\eV}}
    \end{choices}
    \end{multicols}
\end{question}
}

\element{gre}{
\begin{question}{GRE9677-Q13}
    a 100 watt electric heating element is placed in a pan containing one liter of water.
    Although the heating element is on for a long time,
        the water, though close to boiling,
        does not boil.
    When the heating element is removed,
        approxomately how long will it take the water to cool by \SI{1}{\degreeCelsius}?
    (Assume that the specific heat for water is \SI{4.2}{\kilo\joule\per\kilo\gram\per\degreeCelsius}.)
    \begin{multicols}{3}
    \begin{choices}
        \wrongchoice{\SI{20}{\second}}
        \wrongchoice{\SI{40}{\second}}
        \wrongchoice{\SI{60}{\second}}
        \wrongchoice{\SI{130}{\second}}
        \wrongchoice{\SI{200}{\second}}
    \end{choices}
    \end{multicols}
\end{question}
}

\element{gre}{
\begin{question}{GRE9677-Q14}
    Two identical 1.0 kilogram blocks of copper metal,
        one initially at a temperature $T_1=\SI{0}{\degreeCelsius}$ and the other initially at a temperature $T_2=\SI{100}{\degreeCelsius}$,
        are enclosed in a perfectly insulating container.
    The two blocks are initially separated.
    When the blocks are placed in contact,
        they come to equilibrium at a final temperature $T_f$.
    The amount of heat exchanged between the two blocks in the process is equal to which of the following?
    (The specifif heat of copper metal is equal to \SI{0.1}{\kilo\calorie\per\kilo\gram\per\kelvin}.)
    \begin{multicols}{3}
    \begin{choices}
        \wrongchoice{\SI{50}{\kilo\calorie}}
        \wrongchoice{\SI{25}{\kilo\calorie}}
        \wrongchoice{\SI{10}{\kilo\calorie}}
        \wrongchoice{\SI{5}{\kilo\calorie}}
        \wrongchoice{\SI{1}{\kilo\calorie}}
    \end{choices}
    \end{multicols}
\end{question}
}

%% page 20
\element{gre}{
\begin{question}{GRE9677-Q15}
    \begin{center}
    \begin{tikzpicture}
        %% NOTE: TODO: pgfplots
    \end{tikzpicture}
    \end{center}
    Suppose one mole of an ideal gas undergoes the reversible cycle $ABCA$ shown in the $P$-$V$ diagram above,
        where $AB$ is an isotherm.
    The molar heat capacities are $C_p$ at constant pressure and $C_v$ at constant volume.
    The net heat added to the gas during the cycle is equal to:
    \begin{choices}
        \wrongchoice{$\dfrac{RT_hV_2}{V_1}$}
        \wrongchoice{$-C_p\left(T_h-T_c\right)$}
        \wrongchoice{$C_c\left(T_h-T_c\right)$}
        \wrongchoice{$RT_h\ln\dfrac{V_2}{V_1} - C_p\left(T_h-T_c\right)$}
        \wrongchoice{$RT_h\ln\dfrac{V_2}{V_1} - R\left(T_h-T_c\right)$}
    \end{choices}
\end{question}
}

\element{gre}{
\begin{question}{GRE9677-Q16}
    The mean free path for the molecules of a gas is approximately given by $\dfrac{1}{\eta\sigma}$,
        where $\eta$ is the number density and $\sigma$ is the collision cross section.
    The mean free path for air molecules at room conditions is approximately:
    \begin{multicols}{3}
    \begin{choices}
        \wrongchoice{\SI{e-4}{\meter}}
        \wrongchoice{\SI{e-7}{\meter}}
        \wrongchoice{\SI{e-10}{\meter}}
        \wrongchoice{\SI{e-13}{\meter}}
        \wrongchoice{\SI{e-16}{\meter}}
    \end{choices}
    \end{multicols}
\end{question}
}

\element{gre}{
\begin{question}{GRE9677-Q17}
    \begin{center}
    \begin{tikzpicture}
        %% NOTE: TODO: pgfplots
    \end{tikzpicture}
    \end{center}
    The wave function for a particle constrained to move in one dimension is shown in the graph above ($\Psi=0$ for $x\leq 0$ and $x\geq 5$).
    What is the probability that the particle would be found between $x=2$ and $x=4$?
    \begin{multicols}{3}
    \begin{choices}
        \wrongchoice{$\dfrac{17}{64}$}
        \wrongchoice{$\dfrac{25}{64}$}
        \wrongchoice{$\dfrac{5}{8}$}
        \wrongchoice{$\sqrt{\dfrac{5}{8}}$}
        \wrongchoice{$\dfrac{13}{16}$}
    \end{choices}
    \end{multicols}
\end{question}
}

%% page 22
\element{gre}{
\begin{question}{GRE9677-Q18}
    \begin{center}
    \begin{tikzpicture}
        %% NOTE: TODO: energy diagram
    \end{tikzpicture}
    \end{center}
    Consider a potential of the form
    \begin{equation*}
        V(x) = 0, X\leq a \\
    \end{equation*}
    as shown in the figure above.
    Which of the following wave functions is possible for a particle incident from the left with energy $E<V_0$?
    \begin{multicols}{2}
    \begin{choices}
        \AMCboxDimensions{down=-0.4cm}
        \wrongchoice{
            \begin{tikzpicture}
            \end{tikzpicture}
        }
    \end{choices}
    \end{multicols}
\end{question}
}

\element{gre}{
\begin{question}{GRE9677-Q19}
    When alpha particles are directed onto atoms in a thin metal foil,
        some make very close collisions with the nuclei of the atoms and are scattered at large angles.
    If an alpha particle with an initial kinetic energy of \SI{5}{\mega\eV} happens to be scattered through an angle of \ang{180},
        which of the following must have been its distance of closest approach to the scattering nucleus?
    (Assume that the metal foil is made of silver, with $Z=50$)
    \begin{multicols}{2}
    \begin{choices}
        \wrongchoice{$1.22\times{}50^{1/3}\,\mathrm{fm}$}
        \wrongchoice{\SI{2.9e-14}{\meter}}
        \wrongchoice{\SI{1.0e-12}{\meter}}
        \wrongchoice{\SI{3.0e-8}{\meter}}
        \wrongchoice{\SI{1.7e-7}{\meter}}
    \end{choices}
    \end{multicols}
\end{question}
}

\element{gre}{
\begin{question}{GRE9677-Q20}
    A helium atom, mass $4u$, travels with nonrelativistic speed $v$ normal to the surface of a certain material,
        makes an elastic collision with an (essentially free) surface atom,
        and leaves in the opposite direction with speed $0.6v$.
    The atom on the surface must be an atom of:
    \begin{choices}
        \wrongchoice{hydrogen, mass \SI{1}{\atomicmassunit}}
        \wrongchoice{helium, mass \SI{4}{\atomicmassunit}}
        \wrongchoice{carbon, mass \SI{12}{\atomicmassunit}}
        \wrongchoice{oxygen, mass \SI{16}{\atomicmassunit}}
        \wrongchoice{silicon, mass \SI{28}{\atomicmassunit}}
    \end{choices}
\end{question}
}

\element{gre}{
\begin{question}{GRE9677-Q21}
    The period of a physical pendulum is $2\pi\sqrt{I/mgd}$,
        where $I$ is the moment of inertia about the pivot point and $d$ is the distance from the pivot to the center of mass.
    A circular hoop hangs from a nail on a barn wall.
    The mass of the hoop is 3 kilograms and its radius is 20 centimeters.
    If it is displaced slightly by a passing breeze,
        what is the period of the resulting oscillations?
    \begin{multicols}{3}
    \begin{choices}
        \wrongchoice{\SI{0.63}{\second}}
        \wrongchoice{\SI{1.0}{\second}}
        \wrongchoice{\SI{1.3}{\second}}
        \wrongchoice{\SI{1.8}{\second}}
        \wrongchoice{\SI{2.1}{\second}}
    \end{choices}
    \end{multicols}
\end{question}
}

\element{gre}{
\begin{question}{GRE9677-Q22}
    The curvature of Mars is such that its surface drops a vertical distance of 2.0 meters for every 3600 meters tangent to the surface.
    In addition,
        the gravitational acceleration near its surface is 0.4 times that near the surface of Earth.
    What is the speed a golf ball would need to orbit Mars near the surface,
        ignoring the effects of air resistance?
    \begin{multicols}{2}
    \begin{choices}
        \wrongchoice{\SI{0.9}{\kilo\meter\per\second}}
        \wrongchoice{\SI{1.8}{\kilo\meter\per\second}}
        \wrongchoice{\SI{3.6}{\kilo\meter\per\second}}
        \wrongchoice{\SI{4.5}{\kilo\meter\per\second}}
        \wrongchoice{\SI{5.4}{\kilo\meter\per\second}}
    \end{choices}
    \end{multicols}
\end{question}
}

\element{gre}{
\begin{question}{GRE9677-Q23}
    Suppose that the gravitational force law between two massive objects were $\mathbf{F}_{12}=\mathbf{\hat{r}}_{12}gm_1m_2/r_{12}^{2+\epsilon}$,
        where $\epsilon$ is a small positive number.
    Which of the following statments would be \emph{false}?
    \begin{choices}
        \wrongchoice{The total mechanical energy of the planet-Sun system would be conserved.}
        \wrongchoice{The angular momentum of a single planet moving about the Sun would be conserved.}
        \wrongchoice{The periods of planets in circular orbits would be proportional to the $(3+\epsilon)/2$ power of their respective orbital radii.}
        \wrongchoice{A single planet could move in a stationary noncircular elliptical orbit about the Sun.}
        \wrongchoice{A single planet could move in a stationary circular orbit about the Sun.}
    \end{choices}
\end{question}
}

\element{gre}{
\begin{question}{GRE9677-Q24}
    Two identical conducting spheres, $A$ and $B$, carry equal charge.
    They are initially separated by a distance much larger than their diameters,
        and the force between them is $F$.
    A third identical conducting sphere, $C$, is uncharged.
    Sphere $C$ is first touched to $A$, then to $B$, and then removed.
    As a result, the force between $A$ and $B$ is equal to:
    \begin{multicols}{3}
    \begin{choices}
        \wrongchoice{zero}
        \wrongchoice{$\dfrac{F}{16}$}
        \wrongchoice{$\dfrac{F}{4}$}
        \wrongchoice{$\dfrac{3F}{8}$}
        \wrongchoice{$\dfrac{F}{2}$}
    \end{choices}
    \end{multicols}
\end{question}
}

\element{gre}{
\begin{question}{GRE9677-Q25}
    \begin{center}
    \begin{circuitikz}
        %% NOTE: TODO: circuit
    \end{circuitikz}
    \end{center}
    Two real capacitors of equal capacitance ($C_1=C_2$) are shown in the figure above.
    Initially, while the switch $S$ is open,
        one of the capacitors is uncharged and the other carries charge $Q_0$.
    The energy stored in the charged capacitor is $U_0$.
    Sometime after the switch is closed,
        the capacitors $C_1$ and $C_2$ carry charges $Q_1$ and $Q_2$,
        respectively; the voltages across capacitors are $V_1$ and $V_2$;
        and the energies stored in the capacitors are $U_1$ and $U_2$.
    Which of the following statements is \emph{incorrect}?
    \begin{multicols}{2}
    \begin{choices}
        \wrongchoice{$Q_0=\dfrac{1}{2}\left(Q_1+Q_2\right)$}
        \wrongchoice{$Q_1=Q_2$}
        \wrongchoice{$V_1=V_2$}
        \wrongchoice{$U_1=U_2$}
        \wrongchoice{$U_0=U_1+U_2$}
    \end{choices}
    \end{multicols}
\end{question}
}

\element{gre}{
\begin{question}{GRE9677-Q26}
    A series $RLC$ circuit is used in a radio to tune to an FM station broadcasting at \SI{103.7}{\mega\hertz}.
    The resistance in the circuit is 10 ohms and the inductance is 2.0 microhenries.
    What is the best estimate of the capacitance that should be used?
    \begin{multicols}{2}
    \begin{choices}
        \wrongchoice{\SI{200}{\pico\farad}}
        \wrongchoice{\SI{50}{\pico\farad}}
        \wrongchoice{\SI{1}{\pico\farad}}
        \wrongchoice{\SI{0.2}{\pico\farad}}
        \wrongchoice{\SI{0.02}{\pico\farad}}
    \end{choices}
    \end{multicols}
\end{question}
}

%% page 26
\element{gre}{
\begin{question}{GRE9677-Q27}
    In laboratory experiments, graphs are employed to determine how one measured variable depends on another.
    These graphs generally fall into three categories:
        linear, semilog (logorithmic \emph{versus} linear), and log-log. 
    Which type of graph listed in the third column below would \emph{not} be the best for plotting data to test the relationship given in the first and second columns?
    \begin{center}
    \begin{tabu}{cX[c]X[2c]X[c]}
        \toprule
        \makebox[1.5em][c]{\textnumero}
            & Relation
            & Variables Plotted
            & Type of Graph \\
        \bottomrule
    \end{tabu}
    \end{center}
    \begin{choices}
        \wrongchoice{\begin{tabu}{X[c]X[2c]X[c]} $\dfrac{\dd{}N}{dt}\approx\mathrm{e}^{-2t}$ & Activity vs. time for a radioactive isotope & Semilog \\ \end{tabu}}
        \wrongchoice{\begin{tabu}{X[c]X[2c]X[c]} $\dfrac{V_0}{d}$                            & Stopping potential vs. frequency for the photoelectric effect & Linear \\ \end{tabu}}
        \wrongchoice{\begin{tabu}{X[c]X[2c]X[c]} $\dfrac{V_0}{d}$                            & Distance vs. time for an object undergoing constant acceleration & Log-log  \\ \end{tabu}}
        \wrongchoice{\begin{tabu}{X[c]X[2c]X[c]} $\dfrac{V_0}{\kappa d}$                     & Gain vs. frequency for a low-pass filter & Linear \\ \end{tabu}}
        \wrongchoice{\begin{tabu}{X[c]X[2c]X[c]} $\dfrac{V_0}{\kappa d}$                     & Power radiated vs. temperature for blackbody radiation & Log-log \\ \end{tabu}}
    \end{choices}
\end{question}
}

\element{gre}{
\begin{question}{GRE9677-Q28}
    \begin{center}
    \begin{tikzpicture}
        %% NOTE: TODO: pgfplots
    \end{tikzpicture}
    \end{center}
    The figure above represents the trace on the screen of a cathode ray oscilloscope.
    The screen is graduated in centimeters.
    The spot on the screen moves horizontally with a constant speed of 0.5 centimeter/millisecond,
        and the vertical scale is 2 volts/centimeter.
    The signal is a superposition of two oscillations.
    Which of the following ard most nearly the observed amplitude and frequency of these two oscillations?
    \begin{center}
    \begin{tabu}{cX[c]X[c]}
        \toprule
        \makebox[1.5em][c]{\textnumero}
            & Oscillation 1
            & Oscillation 2 \\
        \bottomrule
    \end{tabu}
    \end{center}
    \begin{choices}
        \wrongchoice{\begin{tabu}{X[r]X[r]X[r]X[r]} \SI{5}{\volt} & \SI{250}{\volt} & \SI{2.5}{\volt} & \SI{1000}{\hertz} \\ \end{tabu}}
        \wrongchoice{\begin{tabu}{X[r]X[r]X[r]X[r]} \SI{1.5}{\volt} & \SI{250}{\volt} & \SI{3}{\volt} & \SI{1500}{\hertz} \\ \end{tabu}}
        \wrongchoice{\begin{tabu}{X[r]X[r]X[r]X[r]} \SI{5}{\volt} & \SI{6}{\volt} & \SI{2}{\volt} & \SI{2}{\hertz} \\ \end{tabu}}
        \wrongchoice{\begin{tabu}{X[r]X[r]X[r]X[r]} \SI{2.5}{\volt} & \SI{83}{\volt} & \SI{1.25}{\volt} & \SI{500}{\hertz} \\ \end{tabu}}
        \wrongchoice{\begin{tabu}{X[r]X[r]X[r]X[r]} \SI{6.14}{\volt} & \SI{98}{\volt} & \SI{1.35}{\volt} & \SI{275}{\hertz} \\ \end{tabu}}
    \end{choices}
\end{question}
}

%% page 28
\element{gre}{
\begin{question}{GRE9677-Q29}
    The characteristic distance at which quantum gravitational effects are significant,
        the Planck length, can be determined from a suitable combination of the physical constants $G$, $\hbar$, and $c$.
    Which of the following correctly gives the Planck length?
    \begin{multicols}{3}
    \begin{choices}
        \wrongchoice{$G\hbar c$}
        \wrongchoice{$G\hbar^2 c^3$}
        \wrongchoice{$G^2 \hbar c$}
        \wrongchoice{$\sqrt{G} \hbar^2 c$}
        \wrongchoice{$\sqrt{\dfrac{G\hbar}{c^3}}$}
    \end{choices}
    \end{multicols}
\end{question}
}

\element{gre}{
\begin{question}{GRE9677-Q30}
    \begin{center}
    \begin{tikzpicture}
        %% NOTE: TODO: figure 1 and 2
    \end{tikzpicture}
    \end{center}
    An open-ended U-tube of uniform cross-sectional area contains water (density 1.0 gram/centimeter$^3$) standing initially 20 centimeters from the bottom in each arm.
    An immiscible liquid of density 4.0 grams/centimeter$^3$ is added to one arm until a layer 5 centimeters high forms,
        as shown in the figure above.
    What is the ratio $h_2/h_1$ of the heights of the liquid in the two arms?
    \begin{multicols}{3}
    \begin{choices}
        \wrongchoice{$\dfrac{3}{1}$}
        \wrongchoice{$\dfrac{5}{2}$}
        \wrongchoice{$\dfrac{2}{1}$}
        \wrongchoice{$\dfrac{3}{2}$}
        \wrongchoice{$\dfrac{1}{1}$}
    \end{choices}
    \end{multicols}
\end{question}
}

\element{gre}{
\begin{question}{GRE9677-Q31}
    A sphere of mass $m$ is released from rest in a stationary viscous medium.
    In addition to the gravitational force of magnitude $mg$,
        the sphere experiences a retarding force of magnitude $bv$, where $v$ is the speed of the sphere and $b$ is a constant.
    Assume that the buoyant force is neglible.
    Which of the following statements about the sphere is correct?
    \begin{choices}
        \wrongchoice{Its kinetic energy decreases due to the retarding force.}
        \wrongchoice{Its kinetic energy increases to a maximum, then decreases to zero due to the retarding force.}
        \wrongchoice{Its speed increases to a maximum, then decreases back to a final terminal speed.}
        \wrongchoice{Its speed increases monotonically, approaching a terminal speed that depends on $b$ but not on $m$.}
        \wrongchoice{Its speed increases monotonically, approaching a terminal speed that depends on both $b$ and $m$.}
    \end{choices}
\end{question}
}

%% page 30
\element{gre}{
\begin{question}{GRE9677-Q32}
    Three equal masses $m$ are rigidly connected to each other by massless rods of length $l$ forming an equilateral triangle,
        as shown above.
    The assembly is to given an angular velocity $\omega$ about an axis perpendicular to the triangle.
    For fixed $\omega$, the ratio of the kinetic energy of the assembly for an axis through $B$ compared with that for an axis through $A$ is equal to:
    \begin{multicols}{3}
    \begin{choices}
        \wrongchoice{3}
        \wrongchoice{2}
        \wrongchoice{1}
        \wrongchoice{1/2}
        \wrongchoice{1/3}
    \end{choices}
    \end{multicols}
\end{question}
}

\element{gre}{
\begin{question}{GRE9677-Q33}
    A diatomic molecule is initially in the state $\Psi\left(\Omega,\Phi\right)=\left(5Y_1^1+3Y_5^1+2Y_5^{-1}\right)/\left(38\right)^{1/2}$,
        where $Y_l^m$ is a spherical harmonic.
    If measurements are made of the total angular momentum quantum number $l$ and of the azimuthal angular momentum quantum number $m$,
        what is the probability of obtaining the result $l=5$?
    \begin{multicols}{3}
    \begin{choices}
        \wrongchoice{$\dfrac{36}{144}$}
        \wrongchoice{$\dfrac{9}{38}$}
        \wrongchoice{$\dfrac{13}{38}$}
        \wrongchoice{$\dfrac{5}{\left(38\right)^{1/2}}$}
        \wrongchoice{$\dfrac{34}{38}$}
    \end{choices}
    \end{multicols}
\end{question}
}

\element{gre}{
\begin{question}{GRE9677-Q34}
    When the beta-decay of \ce{^{60}Co} nuclei is observed at low temperatures in a magnetic field that aligns the spins of the nuclei,
        it is found that the electrons are emitted preferentially in a direction opposite to the \ce{^{60}Co} spin direction.
    Which of the following invariances is violated by this decay?
    \begin{choices}
        \wrongchoice{Gauge invariance}
        \wrongchoice{Time invariance}
        \wrongchoice{Translation invariance}
        \wrongchoice{Reflection invariance}
        \wrongchoice{Rotation invariance}
    \end{choices}
\end{question}
}

\element{gre}{
\begin{question}{GRE9677-Q35}
    The wave function for identical fermions is antisymmetric under particle interchange.
    Which of the following is a consequence of this property?
    \begin{choices}
        \wrongchoice{Pauli exclusion principle}
        \wrongchoice{Bohr correspondence principle}
        \wrongchoice{Heisenberg uncertainty principle}
        \wrongchoice{Bose-Einstein condensation}
        \wrongchoice{Fermi's golden rule}
    \end{choices}
\end{question}
}

\element{gre}{
\begin{question}{GRE9677-Q36}
    A lump of clay whose rest mass is 4 kilograms is traveling at three-fifths the speed of light when it collides head-on with an identical lump going the opposite direction at the same speed.
    If the two lumps stick together and no energy is radiated away,
        what is the mass of the composite lump?
    \begin{multicols}{3}
    \begin{choices}
        \wrongchoice{\SI{4}{\kilo\gram}}
        \wrongchoice{\SI{6.4}{\kilo\gram}}
        \wrongchoice{\SI{8}{\kilo\gram}}
        \wrongchoice{\SI{10}{\kilo\gram}}
        \wrongchoice{\SI{13.3}{\kilo\gram}}
    \end{choices}
    \end{multicols}
\end{question}
}

\element{gre}{
\begin{question}{GRE9677-Q37}
    An atom moving at speed $0.3c$ emits an electron along the same direction with speed $0.6c$ in the internal rest frame of the atom.
    The speed of the electron in the lab frame is equal to:
    \begin{multicols}{3}
    \begin{choices}
        \wrongchoice{$0.25c$}
        \wrongchoice{$0.51c$}
        \wrongchoice{$0.66c$}
        \wrongchoice{$0.76c$}
        \wrongchoice{$0.90c$}
    \end{choices}
    \end{multicols}
\end{question}
}

\element{gre}{
\begin{question}{GRE9677-Q38}
    What is the speed of a particle having a momentum of \SI{5}{\mega\eV\per\clight} and a total relativistic energy of \SI{10}{\mega\eV}?
    \begin{multicols}{3}
    \begin{choices}
        \wrongchoice{$c$}
        \wrongchoice{$0.75c$}
        \wrongchoice{$\dfrac{1}{\sqrt{3}} c$}
        \wrongchoice{$\dfrac{1}{2} c$}
        \wrongchoice{$\dfrac{1}{4} c$}
    \end{choices}
    \end{multicols}
\end{question}
}

%% page 32
\element{gre}{
\begin{question}{GRE9677-Q39}
    Which of the following atoms has the lowest ionization potential?
    \begin{multicols}{3}
    \begin{choices}
        \wrongchoice{\ce{^{2}_{4}He}}
        \wrongchoice{\ce{^{7}_{14}N}}
        \wrongchoice{\ce{^{8}_{16}O}}
        \wrongchoice{\ce{^{18}_{40}Ar}}
        \wrongchoice{\ce{^{55}_{133}Cs}}
    \end{choices}
    \end{multicols}
\end{question}
}

\element{gre}{
\begin{question}{GRE9677-Q40}
    If a singly ionized helium atom in an $n=4$ state emits a photon of wavelength 470 nanometers,
        which of the following gives the approximate final energy level, $E_f$, of the atom,
        and the $n$ value, $n_f$, of this final state?
    \begin{center}
    \begin{tabu}{cX[c]X[c]}
        \toprule
        \makebox[1.5em][c]{\textnumero}
            & $E_f\left(\si{\eV}\right)$
            & $n_f$ \\
        \bottomrule
    \end{tabu}
    \end{center}
    \begin{choices}
        \wrongchoice{\begin{tabu}{X[c]X[c]} -6.0 & 3\\ \end{tabu}}
        \wrongchoice{\begin{tabu}{X[c]X[c]} -6.0 & 2\\ \end{tabu}}
        \wrongchoice{\begin{tabu}{X[c]X[c]}  -14 & 2\\ \end{tabu}}
        \wrongchoice{\begin{tabu}{X[c]X[c]}  -14 & 1\\ \end{tabu}}
        \wrongchoice{\begin{tabu}{X[c]X[c]}  -52 & 1\\ \end{tabu}}
    \end{choices}
\end{question}
}

\element{gre}{
\begin{question}{GRE9677-Q41}
    A $3p$ electron is found in the $^{3}P_{3/2}$ energy level of a hydrogen atom.
    Which of the following is true about the electron in this state?
    \begin{choices}
        \wrongchoice{It is allowed to make an electric dipole transition to the $^{2}S_{1/2}$ level.}
        \wrongchoice{It is allowed to make an electric dipole transition to the $^{2}P_{1/2}$ level.}
        \wrongchoice{It has quantum numbers $l=3$, $j=3/2$, $s=1/2$.}
        \wrongchoice{It has quantum numbers $n=3$, $j=l$, $s=3/2$.}
        \wrongchoice{It has exactly the same energy as it would in the $^{3}D_{3/2}$ level.}
    \end{choices}
\end{question}
}

\element{gre}{
\begin{question}{GRE9677-Q42}
    Light of wavelength 500 nanometers is incident on sodium,
        with work function 2.28 electron volts.
    What is teh maximum kinetic energy of the ejected photoelectrons?
    \begin{multicols}{2}
    \begin{choices}
        \wrongchoice{\SI{0.03}{\eV}}
        \wrongchoice{\SI{0.2}{\eV}}
        \wrongchoice{\SI{0.6}{\eV}}
        \wrongchoice{\SI{1.3}{\eV}}
        \wrongchoice{\SI{2.0}{\eV}}
    \end{choices}
    \end{multicols}
\end{question}
}

\element{gre}{
\begin{question}{GRE9677-Q43}
    The line integral of $\mathbf{u} = y\mathbf{i}-x\mathbf{j}+z\mathbf{k}$ around a circle of radius $R$ in the $xy$-plane with center at the origin is equal to:
    \begin{multicols}{3}
    \begin{choices}
        \wrongchoice{zero}
        \wrongchoice{$2\pi{}R$}
        \wrongchoice{$2\pi{}R^2$}
        \wrongchoice{$\dfrac{\pi{}R^2}{4}$}
        \wrongchoice{$3R^3$}
    \end{choices}
    \end{multicols}
\end{question}
}

\element{gre}{
\begin{question}{GRE9677-Q44}
    A particle of unit mass undergoes one-dimensional motion such that its velocity varies according to
    \begin{equation*}
        v(x) = \beta x^{-n} \, ,
    \end{equation*}
    where $\beta$ and $n$ are constants and $x$ is the position of the particle.
    What is the acceleration of the particle as a function of $x$?
    \begin{multicols}{2}
    \begin{choices}
        \wrongchoice{$-n\beta^2 x^{-2n-1}$}
        \wrongchoice{$-n\beta^2 x^{-n-1}$}
        \wrongchoice{$-n\beta^2 x^{-n}$}
        \wrongchoice{$-\beta x^{-n+1}$}
        \wrongchoice{$-\beta x^{-2n+1}$}
    \end{choices}
    \end{multicols}
\end{question}
}

%% page 34
\element{gre}{
\begin{question}{GRE9677-Q45}
    The circuits below consists of two-element combinations of capacitors, diodes, and resistors.
    $V_{\text{in}}$ represents an ac-voltage with variable frequency.
    It is desired to build a circuit for which $V_{\text{out}}\approx{}V_{\text{in}}$ at high frequencies and $V_{\text{out}}\approx{}0$ at low frequencies.
    Which of the following circuits will perform this task?
    \begin{multicols}{2}
    \begin{choices}
        \AMCboxDimensions{down=-0.75cm}
        \ctikzset{bipoles/length=0.75cm}
        \wrongchoice{
            \begin{circuitikz}
            \end{circuitikz}
        }
    \end{choices}
    \end{multicols}
\end{question}
}

\element{gre}{
\begin{question}{GRE9677-Q46}
    \begin{center}
    \begin{tikzpicture}
        %% NOTE: TODO: figure 1 and 2
    \end{tikzpicture}
    \end{center}
    A circular wire loop of radius $R$ rotates with an angular speed $\omega$ in a uniform magnetic field $\mathbf{B}$,
        as shown in the figure above.
    If the emf $\epsilon$ induced in the loop is $\epsilon\sin\omega{}t$,
        then the angular speed of the loop is:
    \begin{multicols}{2}
    \begin{choices}
        \wrongchoice{$\dfrac{\epsilon_0{}R}{B}$}
        \wrongchoice{$\dfrac{2\pi\epsilon_0{}}{R}$}
        \wrongchoice{$\dfrac{\epsilon_0}{B\pi{}R^2}$}
        \wrongchoice{$\dfrac{\epsilon_0^2}{BR^2}$}
        \wrongchoice{$\tan^{-1}\left(\dfrac{\epsilon_0}{Bc}\right)$}
    \end{choices}
    \end{multicols}
\end{question}
}

\element{gre}{
\begin{question}{GRE9677-Q47}
    \begin{center}
    \begin{tikzpicture}
        %% NOTE: TODO: figure 1 and 2
    \end{tikzpicture}
    \end{center}
    A wire is being wound around a rotating wooden cylinder of radius $R$.
    One end of the wire is connected to the axis of the cylinder,
        as shown in the figure above.
    The cylinder is placed in a uniform magnetic field of magnitude $B$ parallel to its axis and rotates at $N$ revolutions per second.
    What is the potential difference between the open ends of the wire?
    \begin{multicols}{2}
    \begin{choices}
        \wrongchoice{zero}
        \wrongchoice{$2\pi{}NBR$}
        \wrongchoice{$\pi{}NBR^2$}
        \wrongchoice{$\dfrac{BR^2}{N}$}
        \wrongchoice{$\pi{}NBR^3$}
    \end{choices}
    \end{multicols}
\end{question}
}

%% page 36
\element{gre}{
\begin{question}{GRE9677-Q48}
    The half-life of a $\pi^+$ meson at rest is \num{2.5e-8} second.
    A beam of $\pi^+$ mesons is generated at a point 15 meters from a detector.
    Only $\dfrac{1}{2}$ of the $\pi^+$ mesons live to reach the detector.
    The speed of the $\pi^+$ mesons is:
    \begin{multicols}{3}
    \begin{choices}
        \wrongchoice{$\dfrac{1}{2} c$}
        \wrongchoice{$\sqrt{\dfrac{2}{5}} c$}
        \wrongchoice{$\dfrac{2}{\sqrt{5}} c$}
        \wrongchoice{$c$}
        \wrongchoice{$2c$}
    \end{choices}
    \end{multicols}
\end{question}
}

\element{gre}{
\begin{question}{GRE9677-Q49}
    The infinite $xy$-plane is a nonconducting surface,
        with surface charge density $\sigma$, as measured by an observer at rest on the surface.
    A second observer moves with velocity $v\mathbf{\hat{x}}$ relative to the surface,
        at height $h$ above it.
    Which of the following expressions gives the electric field measured by this second observer?
    \begin{choices}
        \wrongchoice{$\dfrac{\sigma}{2\epsilon_0} \mathbf{\hat{z}}$}
        \wrongchoice{$\dfrac{\sigma}{2\epsilon_0}\sqrt{1-\dfrac{v^2}{c^2}} \mathbf{\hat{z}}$}
        \wrongchoice{$\dfrac{\sigma}{2\epsilon_0\sqrt{1-\dfrac{v^2}{c^2}}} \mathbf{\hat{z}}$}
        \wrongchoice{$\dfrac{\sigma}{2\epsilon_0} \left(\sqrt{1-\dfrac{v^2}{c^2}}\mathbf{\hat{z}}+\dfrac{v}{c}\mathbf{\hat{x}}\right)$}
        \wrongchoice{$\dfrac{\sigma}{2\epsilon_0} \left(\sqrt{1-\dfrac{v^2}{c^2}}\mathbf{\hat{z}}-\dfrac{v}{c}\mathbf{\hat{y}}\right)$}
    \end{choices}
\end{question}
}

\element{gre}{
\begin{question}{GRE9677-Q50}
    In inertial frame $S$, two events occur at the same instant in time and $3c$ minutes apart inspace.
    In inertial frame $S\prime$, the same events occur at $5c$ minutes apart.
    What is the time interval between the events in $S\prime$?
    \begin{multicols}{2}
    \begin{choices}
        \wrongchoice{\SI{0}{\minute}}
        \wrongchoice{\SI{2}{\minute}}
        \wrongchoice{\SI{4}{\minute}}
        \wrongchoice{\SI{8}{\minute}}
        \wrongchoice{\SI{16}{\minute}}
    \end{choices}
    \end{multicols}
\end{question}
}

\element{gre}{
\begin{question}{GRE9677-Q51}
    The solution to the Schr\"{o}dinger equation for a particle bound in a one-dimensional,
        infinitely deep potential well,
        indexed by quantum number $n$,
        indicates that in the middle of the well the probabilty density vanishes for:
    \begin{choices}
        \wrongchoice{the ground state ($n=1$) only}
        \wrongchoice{states of even $n$ ($n=2,4,\ldots$)}
        \wrongchoice{states of odd $n$ ($n=1,3,\ldots$)}
        \wrongchoice{all states ($n=1,2,3,\ldots$)}
        \wrongchoice{all states except the ground state}
    \end{choices}
\end{question}
}

\element{gre}{
\begin{question}{GRE9677-Q52}
    At a given instant of time, a rigid rotator is in the state $\Psi\left(\theta,\phi\right)=\sqrt{\dfrac{3}{4\pi}}\sin\theta\sin\phi$,
        where $\theta$ is the polar angle relative to the $z$-axis and $\phi$ is the azimuthal angle.
    Measurement will find which of the following possible values of the $z$-component of the angular momentum, $L_z$?
    \begin{multicols}{2}
    \begin{choices}
        \wrongchoice{zero}
        \wrongchoice{$\dfrac{\hbar}{2}$, $-\dfrac{\hbar}{2}$}
        \wrongchoice{$\hbar$, $-\hbar$}
        \wrongchoice{$2\hbar$, $-2\hbar$}
        \wrongchoice{$\hbar$, 0, $-2\hbar$}
    \end{choices}
    \end{multicols}
\end{question}
}

\element{gre}{
\begin{question}{GRE9677-Q53}
    Positronium is the bound state of an electron and a positron.
    Consider only the states of zero orbital angular momentum $(l=0)$.
    The most probably decay product of any such state of positronium with spin zero (singlet) is:
    \begin{multicols}{2}
    \begin{choices}[o]
        \wrongchoice{0 photon}
        \wrongchoice{1 photon}
        \wrongchoice{2 photon}
        \wrongchoice{3 photon}
        \wrongchoice{4 photon}
    \end{choices}
    \end{multicols}
\end{question}
}

%% page 38
%% question 54-55
\element{gre}{
\begin{question}{GRE9677-Q54}
    Questions 54--55 concern a plane electromagnetic wave that is a superposition of two independent orthogonal plane waves and can be written as the real part of $\mathbf{E}=\mathbf{\hat{x}}E_1\exp{i\left[kz-\omega{}t\right]}+\mathbf{\hat{y}}E_2\exp{i\left[kz-\omega{}t+\pi\right]}$,
        where $k$, $\omega$, $E_1$, and $E_2$ are real.
    %% start question
    If $E_2=E_1$, the tip of the electric field vector will describe a trajectory that,
        as viewed along the $z$-axis from positive $z$ and looking toward the origin, is a:
    \begin{choices}
        \wrongchoice{line at \ang{45} to the $+z$-axis}
        \wrongchoice{line at \ang{135} to the $+z$-axis}
        \wrongchoice{clockwise circle}
        \wrongchoice{counterclockwise circle}
        \wrongchoice{random path}
    \end{choices}
\end{question}
}

\element{gre}{
\begin{question}{GRE9677-Q55}
    Questions 54--55 concern a plane electromagnetic wave that is a superposition of two independent orthogonal plane waves and can be written as the real part of $\mathbf{E}=\mathbf{\hat{x}}E_1\exp{i\left[kz-\omega{}t\right]}+\mathbf{\hat{y}}E_2\exp{i\left[kz-\omega{}t+\pi\right]}$,
        where $k$, $\omega$, $E_1$, and $E_2$ are real.
    %% start question
    If the plane wave is split and recombined on a screen after the two portions,
        which are polarized in the $x$- and $y$-directions,
        have traveled an optical path difference of $2\pi/k$,
        the observed average intensity will be proportional to:
    \begin{multicols}{2}
    \begin{choices}
        \wrongchoice{$E_1^2 + E_2^2$}
        \wrongchoice{$E_1^2 - E_2^2$}
        \wrongchoice{$\left(E_1 + E_2\right)^2$}
        \wrongchoice{$\left(E_1 - E_2\right)^2$}
        \wrongchoice{zero}
    \end{choices}
    \end{multicols}
\end{question}
}

\element{gre}{
\begin{question}{GRE9677-Q56}
    A light source is at the bottom of a pool of water
        (the index of refraction of water is 1.33).
    At what minimum angle of incidence will a ray be totally reflected at the surface?
    \begin{multicols}{3}
    \begin{choices}
        \wrongchoice{\ang{0}}
        \wrongchoice{\ang{25}}
        \wrongchoice{\ang{50}}
        \wrongchoice{\ang{75}}
        \wrongchoice{\ang{90}}
    \end{choices}
    \end{multicols}
\end{question}
}

\element{gre}{
\begin{question}{GRE9677-Q57}
    Consider a single-slit diffraction pattern for a slit of width $d$.
    It is observed that for light of wavelength 400 nanometers,
        the angle between the first minimum and the central maximum is \num{4e-3} radians.
    The value of $d$ is:
    \begin{multicols}{2}
    \begin{choices}
        \wrongchoice{\SI{1e-5}{\meter}}
        \wrongchoice{\SI{5e-5}{\meter}}
        \wrongchoice{\SI{1e-4}{\meter}}
        \wrongchoice{\SI{2e-4}{\meter}}
        \wrongchoice{\SI{1e-3}{\meter}}
    \end{choices}
    \end{multicols}
\end{question}
}

\element{gre}{
\begin{question}{GRE9677-Q58}
    A collimated laser beam emerging from a commercial HeNe laser has a diamter of about 1 millimeter.
    In order to convert this beam into a well collimated beam of diameter 10 millimeters,
        to convex lenses are to be used.
    The first lens is of focal length 1.5 centimeters and is to be mounted at the output of the laser.
    What is the focal length, $f$,
        of the second lens and how far from the first lens should it be placed?
    \begin{center}
        \begin{tabu}{cX[c]X[c]}
        \toprule
        \makebox[1.5em][c]{\textnumero}
            %% NOTE: $f/\si{\centi\meter}
            & $f$
            & Distance \\
        \bottomrule
    \end{tabu}
    \end{center}
    \begin{choices}
        \wrongchoice{\begin{tabu}{X[c]X[c]} \SI{4.5}{\centi\meter} & \SI{6.0}{\centi\meter} \\ \end{tabu}}
        \wrongchoice{\begin{tabu}{X[c]X[c]} \SI{10}{\centi\meter} & \SI{10}{\centi\meter} \\ \end{tabu}}
        \wrongchoice{\begin{tabu}{X[c]X[c]} \SI{10}{\centi\meter} & \SI{11.5}{\centi\meter} \\ \end{tabu}}
        \wrongchoice{\begin{tabu}{X[c]X[c]} \SI{15}{\centi\meter} & \SI{15}{\centi\meter} \\ \end{tabu}}
        \wrongchoice{\begin{tabu}{X[c]X[c]} \SI{15}{\centi\meter} & \SI{16.5}{\centi\meter} \\ \end{tabu}}
    \end{choices}
\end{question}
}

\element{gre}{
\begin{question}{GRE9677-Q59}
    The approximate number of photons in a femtosecond (\SI{e-15}{\second}) pulse of 600 nanometers wavelength light from a 10 kilowatt peak-power dye laser is:
    \begin{multicols}{3}
    \begin{choices}
        \wrongchoice{\num{e3}}
        \wrongchoice{\num{e7}}
        \wrongchoice{\num{e11}}
        \wrongchoice{\num{e15}}
        \wrongchoice{\num{e18}}
    \end{choices}
    \end{multicols}
\end{question}
}

\element{gre}{
\begin{question}{GRE9677-Q60}
    The Lyman alpha spectral line of hydrogen $\left(\lambda=\SI{122}{\nano\meter}\right)$ differs by \num{1.8e-12} meter in spectra taken at opposite ends of the Sun's equator.
    What is the speed of a particle on the equator due to the Sun's rotation,
        in kilometers per second?
    \begin{multicols}{2}
    \begin{choices}
        \wrongchoice{\SI{0.22}{\kilo\meter\per\second}}
        \wrongchoice{\SI{2.2}{\kilo\meter\per\second}}
        \wrongchoice{\SI{22}{\kilo\meter\per\second}}
        \wrongchoice{\SI{220}{\kilo\meter\per\second}}
        \wrongchoice{\SI{2200}{\kilo\meter\per\second}}
    \end{choices}
    \end{multicols}
\end{question}
}

%% page 40
\element{gre}{
\begin{question}{GRE9677-Q61}
    A sphere of radius $S$ carries charge density proportional to the square of the distance from the center:
        $\rho=Ar^2$, where $A$ is a positive constant.
    At a distance of $R/2$ from the center,
        the magnitude of the electric field is:
    \begin{multicols}{3}
    \begin{choices}
        \wrongchoice{$\dfrac{A}{4\pi\epsilon_0}$}
        \wrongchoice{$\dfrac{AR^3}{40\pi\epsilon_0}$}
        \wrongchoice{$\dfrac{AR^3}{24\pi\epsilon_0}$}
        \wrongchoice{$\dfrac{AR^3}{5\pi\epsilon_0}$}
        \wrongchoice{$\dfrac{AR^3}{3\pi\epsilon_0}$}
    \end{choices}
    \end{multicols}
\end{question}
}

\element{gre}{
\begin{question}{GRE9677-Q62}
    Two capacitors of capacitances 1.0 microfarad and 2.0 microfarads are each charged by being connected across a 5.0 volt battery.
    They are disconnected from the battery and then connected to each other with resistive wires so that plates of opposite charge are connected together.
    What will be the magnitude of the final voltage across the 2.0 microfarad capacitor?
    \begin{multicols}{3}
    \begin{choices}
        \wrongchoice{\SI{0}{\volt}}
        \wrongchoice{\SI{0.6}{\volt}}
        \wrongchoice{\SI{1.7}{\volt}}
        \wrongchoice{\SI{3.3}{\volt}}
        \wrongchoice{\SI{5.0}{\volt}}
    \end{choices}
    \end{multicols}
\end{question}
}

\element{gre}{
\begin{question}{GRE9677-Q63}
    According to the Standard Model of elementary particles,
        which of the following is \emph{not} a composite object?
    \begin{multicols}{2}
    \begin{choices}
        \wrongchoice{Muon}
        \wrongchoice{Pi-meson}
        \wrongchoice{Neutron}
        \wrongchoice{Deuteron}
        \wrongchoice{Alpha particle}
    \end{choices}
    \end{multicols}
\end{question}
}

\element{gre}{
\begin{question}{GRE9677-Q64}
    The binding energy of a heavy nucleus is about 7 million electron volts per nucleon,
        whereas the binding energy of a medium-weight nucleus is about 8 million electron volts per nucleon.
    Therefore, the total kinetic energy liberated when a heavy nucleus undergoes symmetric fission is most nearly:
    \begin{multicols}{2}
    \begin{choices}
        \wrongchoice{\SI{1876}{\mega\eV}}
        \wrongchoice{\SI{938}{\mega\eV}}
        \wrongchoice{\SI{200}{\mega\eV}}
        \wrongchoice{\SI{8}{\mega\eV}}
        \wrongchoice{\SI{7}{\mega\eV}}
    \end{choices}
    \end{multicols}
\end{question}
}

\element{gre}{
\begin{question}{GRE9677-Q65}
    A man of mass $m$ on an initially stationary boat gets off the boat by leaping to the left in an exactly horizontal direction.
    Immediately after the leap,
        the boat, of mass $M$,
        is observed to be moving to the right at speed $v$.
    How much work did the man do during leap (both on his own body and on the boat)?
    \begin{multicols}{2}
    \begin{choices}
        \wrongchoice{$\dfrac{1}{2}Mv^2$}
        \wrongchoice{$\dfrac{1}{2}mv^2$}
        \wrongchoice{$\dfrac{1}{2}\left(M+m\right)v^2$}
        \wrongchoice{$\dfrac{1}{2}\left(M+\dfrac{M^2}{m}\right)v^2$}
        \wrongchoice{$\dfrac{1}{2}\left(\dfrac{Mn}{M+n}\right)v^2$}
    \end{choices}
    \end{multicols}
\end{question}
}

\element{gre}{
\begin{question}{GRE9677-Q66}
    When it is about the same distance from the Sun as is Jupiter,
        a spacecraft on a mission to the outer planets has a speed that is 1.5 times the speed of Jupiter in its orbit.
    Which of the following describes the orbit of the spacecraft about the Sun?
    \begin{multicols}{2}
    \begin{choices}
        \wrongchoice{Spiral}
        \wrongchoice{Circle}
        \wrongchoice{Ellipse}
        \wrongchoice{Parabola}
        \wrongchoice{Hyperbola}
    \end{choices}
    \end{multicols}
\end{question}
}

%% page 42
\element{gre}{
\begin{question}{GRE9677-Q67}
    A black hole is an object whose gravitational field is so strong that even light cannot escape.
    To what approximate radius would Earth (mass = \SI{5.98e24}{\kilo\gram}) have to be compressed in order to become a black hole?
    \begin{multicols}{3}
    \begin{choices}
        \wrongchoice{\SI{1}{\nano\meter}}
        \wrongchoice{\SI{1}{\micro\meter}}
        \wrongchoice{\SI{1}{\centi\meter}}
        \wrongchoice{\SI{100}{\meter}}
        \wrongchoice{\SI{10}{\kilo\meter}}
    \end{choices}
    \end{multicols}
\end{question}
}

\element{gre}{
\begin{question}{GRE9677-Q68}
    \begin{center}
    \begin{tikzpicture}
        %% NOTE: TODO: figure 1 and 2
    \end{tikzpicture}
    \end{center}
    A bead is constrained to slide on a frictionless rod that is fixed at an angle $\theta$ with a vertical axis and is rotating with angular frequency $\omega$ about the axis,
        as shown above.
    Taking the distance $s$ along the rod as the variable,
        the Lagrangian for the bead is equal to:
    \begin{choices}
        \wrongchoice{$\dfrac{1}{2}m\dot{s}^2 - mgs\cos\theta$}
        \wrongchoice{$\dfrac{1}{2}m\dot{s}^2 + \dfrac{1}{2}m\left(\omega{}s\right)^2 - mgs$}
        \wrongchoice{$\dfrac{1}{2}m\dot{s}^2 + \dfrac{1}{2}m\left(\omega{}s\cos\theta\right)^2 + mgs\cos\theta$}
        \wrongchoice{$\dfrac{1}{2}m\left(\dot{s}\sin\theta\right)^2 - -mgs\cos\theta$}
        \wrongchoice{$\dfrac{1}{2}m\dot{s}^2 + \dfrac{1}{2}m\left(\omega{}s\sin\theta\right)^2 - mgs\cos\theta$}
    \end{choices}
\end{question}
}

\element{gre}{
\begin{question}{GRE9677-Q69}
    \begin{center}
    \begin{tikzpicture}
        %% NOTE: TODO: figure 1 and 2
    \end{tikzpicture}
    \end{center}
    Two long conductors are arranged as shown above to form overlapping cylinders,
        each of radius $r$, whose centers are separated by a distance $d$.
    Current of density $J$ flows into the plane of the page along the shaded part of one conductor and an equal current portion of the other,
        as shown.
    what are the magnitude and the direction of the magnitude field at point $A$?
    \begin{choices}
        \wrongchoice{$\left(\dfrac{\mu_0}{2\pi}\right)\pi{}dJ$, in the $+y$-direction}
        \wrongchoice{$\left(\dfrac{\mu_0}{2\pi}\right)\dfrac{d^2J}{r}$, in the $+y$-direction}
        \wrongchoice{$\left(\dfrac{\mu_0}{2\pi}\right)\dfrac{4d^2J}{r}$, in the $-y$-direction}
        \wrongchoice{$\left(\dfrac{\mu_0}{2\pi}\right)\dfrac{Jr^2}{d}$, in the $-y$-direction}
        \wrongchoice{There is no magnitude field at $A$}
    \end{choices}
\end{question}
}

\element{gre}{
\begin{question}{GRE9677-Q70}
    A charged particle, $A$, moving at a speed much less than $c$,
        decelerates uniformly.
    A second particle, $B$, has one-half the mass,
        twice the charge, three times the velocity,
        and four times the acceleration of particle $A$.
    According to classical electrodynamics,
        the ratio of $P_B/P_A$ of the powers radiated is:
    \begin{multicols}{3}
    \begin{choices}
        \wrongchoice{16}
        \wrongchoice{32}
        \wrongchoice{48}
        \wrongchoice{64}
        \wrongchoice{72}
    \end{choices}
    \end{multicols}
\end{question}
}

%% page 44
\element{gre}{
\begin{question}{GRE9677-Q71}
    \begin{center}
    \begin{tikzpicture}
        %% NOTE: TODO: figure 1 and 2
    \end{tikzpicture}
    \end{center}
    The figure above shows the trajectory of a particle that is deflected as it moves through the uniform electric field between parallel plates. 
    There is potential difference $V$ and distance $d$ between the plates,
        and they have length $L$. 
    The particle (mass $m$,
        charge $q$) has nonerelativistic speed $v$ before it enters the field,
        and its direction at this time is perpendicular to the field. 
    For small deflections,
        which of the following expressions is the best approximation to the deflection angle $\theta$
    \begin{choices}
        \wrongchoice{$\mathrm{arctan}\left(\dfrac{L}{d}\dfrac{Vq}{mv^2}\right)$}
        \wrongchoice{$\mathrm{arctan}\left(\dfrac{L}{d}\left(\dfrac{Vq}{mv^2}\right)^2\right)$}
        \wrongchoice{$\mathrm{arctan}\left(\left(\dfrac{L}{d}\right)\dfrac{Vq}{mv^2}\right)^2$}
        \wrongchoice{$\mathrm{arctan}\left(\dfrac{L}{d}\sqrt{\dfrac{2Vq}{mv^2}}\right)$}
        \wrongchoice{$\mathrm{arctan}\left(\sqrt{\dfrac{L}{d}}\dfrac{2Vq}{mv^2}\right)$}
    \end{choices}
\end{question}
}

\element{gre}{
\begin{question}{GRE9677-Q72}
    In a voltage amplifier,
        which of the following is \emph{not} usually a result of introducing negative feedback?
    \begin{choices}
        \wrongchoice{Increased amplification}
        \wrongchoice{Increased bandwidth}
        \wrongchoice{Increased stability}
        \wrongchoice{Decreased distortion}
        \wrongchoice{Decreased voltage gain}
    \end{choices}
\end{question}
}

\element{gre}{
\begin{question}{GRE9677-Q73}
    The adiabatic expansion of an ideal gas is described by the equation $PV^{\gamma}=C$,
        $\gamma$ and $C$ are constants.
    The work done by the gas in expanding adiabatically from the state $\left(V_i,P_i\right)$ to $\left(V_f,P_f\right)$ is equal to:
    \begin{choices}
        \wrongchoice{$P_f V_f$}
        \wrongchoice{$\dfrac{P_i+P_f}{2}\left(V_f-V_i\right)$}
        \wrongchoice{$\dfrac{P_fV_f-P_iV_i}{1-\gamma}$}
        \wrongchoice{$\dfrac{P_i\left(V_f^{1+\gamma}-V_i^{1+\gamma}\right)}{1+\gamma}$}
        \wrongchoice{$\dfrac{P_f\left(V_f^{1-\gamma}-V_i^{1-\gamma}\right)}{1+\gamma}$}
    \end{choices}
\end{question}
}

\element{gre}{
\begin{question}{GRE9677-Q74}
    A body of mass $m$ with specific heat $C$ at temperature \SI{500}{\kelvin} is brought into contact with an identical body at temperature \SI{100}{\kelvin},
        and the two are isolated from their surroundings.
    The change in entropy of the system is equal to:
    \begin{multicols}{2}
    \begin{choices}
        \wrongchoice{$\left(\dfrac{4}{3}\right)mC$}
        \wrongchoice{$mC\ln\left(\dfrac{9}{5}\right)$}
        \wrongchoice{$mC\ln\left(3\right)$}
        \wrongchoice{$-mC\ln\left(\dfrac{5}{3}\right)$}
        \wrongchoice{zero}
    \end{choices}
    \end{multicols}
\end{question}
}

%% page 46
\element{gre}{
\begin{question}{GRE9677-Q75}
    \begin{center}
    \begin{tikzpicture}
        %% NOTE: TODO: figure 1 and 2
    \end{tikzpicture}
    \end{center}
    Window $A$ is a pane of glass 4 millimeters thick, as shown above.
    Window $B$ is a sandwich consisting of two extremely thin laters of glass separated by an air gap 2 millimeters thick,
        as shown above.
    If the thermal conductivities of glass and air 0.8 watt/meter\si{\degreeCelsius} and 0.025 watt/meter\si{\degreeCelsius},
        respectiviely,
        then the ratio of the heat flow through window $A$ to the heat flow through window $B$ is:
    \begin{multicols}{3}
    \begin{choices}
        \wrongchoice{2}
        \wrongchoice{4}
        \wrongchoice{8}
        \wrongchoice{16}
        \wrongchoice{32}
    \end{choices}
    \end{multicols}
\end{question}
}

\element{gre}{
\begin{question}{GRE9677-Q76}
    A gaussian wave packet travels through free space.
    Which of the following statements about the wave packet are correct for all such wave packets?
    \begin{itemize}
        \item[I.] The average momentum of the wave packet is zero.
        \item[II.] The width of the wave packet increases with time, as $i\to\infty$.
        \item[III.] The amplitude of the wave packet remains constant with time.
        \item[IV.] The narrower the wave packet is in momentum space, the wider it is in coordinate space.
    \end{itemize}
    \begin{choices}
        \wrongchoice{I and III only}
        \wrongchoice{II and IV only}
        \wrongchoice{I, II, and IV only}
        \wrongchoice{II, III, and IV only}
        \wrongchoice{I, II, III, and IV}
    \end{choices}
\end{question}
}

%% page 48
\element{gre}{
\begin{question}{GRE9677-Q77}
    Two ions, 1 and 2, at fixed separation, with spin angular momentum operators $\mathbf{S}_1$ and $\mathbf{S}_2$,
        have the interaction Hamiltonian $H=-j\mathbf{S}_1\cdot\mathbf{S}_2$,
        where $J>0$.
    The values of $\mathbf{S}_1^2$ and $\mathbf{S}_2^2$ are fixed at $S_1\left(S_1+1\right)$ and $S_2\left(S_2+1\right)$, respectively.
    Which of the following is the energy of the ground state of the system?
    \begin{choices}
        \wrongchoice{zero}
        \wrongchoice{$-J S_1 S_2$}
        \wrongchoice{$-J\left[S_1\left(S_1+1\right) - S_2\left(S_2+1\right)\right]$}
        \wrongchoice{$-\dfrac{J}{2}\left[\left(S_1+S_2\right)\left(S_1+S_2+1\right) - S_1\left(S_1+1\right) - S_2\left(S_2+1\right)\right]$}
        \wrongchoice{$-\dfrac{J}{2}\left[\dfrac{S_1\left(S_1+1\right)+S_2\left(S_2+1\right)}{\left(S_1+S_2\right)\left(S_1+S_2+1\right)}\right]$}
    \end{choices}
\end{question}
}

\element{gre}{
\begin{question}{GRE9677-Q78}
    In an $n$-type semiconductor,
        which of the following is true of impurity atoms?
    \begin{choices}
        \wrongchoice{They accept electrons from the filled valence band into empty energy levels just above the valence band.}
        \wrongchoice{They accept electrons from the filled valence band into empty energy levels just below the valence band.}
        \wrongchoice{They accept electrons from the conduction band into empty energy levels just below the valence band.}
        \wrongchoice{They donate electrons to the filled valence band from the donor level just above the valence band.}
        \wrongchoice{They donate electrons to the filled valence band from the donor level just below the valence band.}
    \end{choices}
\end{question}
}

\element{gre}{
\begin{question}{GRE9677-Q79}
    For an ideal diatomic gas in thermal equilibrium,
        the ratio of the molar heat capacity at constant volume at very high temperatures to that at very low temperatures is equal to:
    \begin{multicols}{3}
    \begin{choices}
        \wrongchoice{$1$}
        \wrongchoice{$\dfrac{5}{3}$}
        \wrongchoice{$2$}
        \wrongchoice{$\dfrac{7}{3}$}
        \wrongchoice{$3$}
    \end{choices}
    \end{multicols}
\end{question}
}

\element{gre}{
\begin{question}{GRE9677-Q80}
    A string consists of two parts attached at $x=0$.
    The right part of the string $\left(x>0\right)$ has mass $\mu_r$ per unit length and the left part of the string $\left(x<0\right)$ has mass $\mu_l$ per unit length.
    The string tension is $T$.
    If a wave of unit amplitude travels along the left part of the string,
        as shown in the figure above,
        what is the amplitude of teh wave that is transmitted to the right part of the string?
    \begin{multicols}{2}
    \begin{choices}
        \wrongchoice{one}
        \wrongchoice{$\dfrac{2}{1+\sqrt{\mu_l/\mu_r}}$}
        \wrongchoice{$\dfrac{2\sqrt{\mu_l/\mu_r}}{1+\sqrt{\mu_l/\mu_r}}$}
        \wrongchoice{$\dfrac{\sqrt{\mu_l/\mu_r}-1}{\sqrt{\mu_l/\mu_r}+1}$}
        \wrongchoice{zero}
    \end{choices}
    \end{multicols}
\end{question}
}

\element{gre}{
\begin{question}{GRE9677-Q81}
    A piano tuner who wishes to tune the note D\textsubscript{2} corresponding to a frequency of 73.416 hertz has tuned A\textsubscript{4} to a frequency of 440.000 hertz.
    Which harmonic of D\textsubscript{2} (counting the fundamental as the first harmonic) will give the lowest number of beats per second,
        and approximately how many beats will this be when the two notes are tuned properly?
    \begin{center}
    \begin{tabu}{cX[c]X[c]}
        \toprule
        \makebox[1.5em][c]{\textnumero}
            & Harmonic
            & Number of beats \\
        \bottomrule
    \end{tabu}
    \end{center}
    \begin{choices}
        \wrongchoice{\begin{tabu}{X[c]X[c]} 6 & 5 \\ \end{tabu}}
        \wrongchoice{\begin{tabu}{X[c]X[c]} 6 & 0.5 \\ \end{tabu}}
        \wrongchoice{\begin{tabu}{X[c]X[c]} 5 & 0.1 \\ \end{tabu}}
        \wrongchoice{\begin{tabu}{X[c]X[c]} 3 & 0.372 \\ \end{tabu}}
        \wrongchoice{\begin{tabu}{X[c]X[c]} 2 & 4.5 \\ \end{tabu}}
    \end{choices}
\end{question}
}

\element{gre}{
\begin{question}{GRE9677-Q82}
    Consider two horizontal glass plates with a thin film of air between them.
    For what values of the thickness of the file of air will the film,
        as seen by reflected light, appear bright if it is illuminated normally from above by blue light of wavelength 488 nanometers?
    \begin{choices}
        \wrongchoice{0, \SI{122}{\nano\meter}, \SI{244}{\nano\meter}}
        \wrongchoice{0, \SI{122}{\nano\meter}, \SI{366}{\nano\meter}}
        \wrongchoice{0, \SI{244}{\nano\meter}, \SI{488}{\nano\meter}}
        \wrongchoice{\SI{122}{\nano\meter}, \SI{244}{\nano\meter}, \SI{366}{\nano\meter}}
        \wrongchoice{\SI{122}{\nano\meter}, \SI{366}{\nano\meter}, \SI{610}{\nano\meter}}
    \end{choices}
\end{question}
}

%% page 50
\element{gre}{
\begin{question}{GRE9677-Q83}
    \begin{center}
    \begin{tikzpicture}
        %% NOTE: TODO: figure 1 and 2
    \end{tikzpicture}
    \end{center}
    Consider a particle moving without friction on a rippled surface,
        as shown above. 
    Gravity acts down in the negative $h$ direction. The elevation $h(x)$ of the surface is given by $h(x) = d\cos\left(kx\right)$. 
    If the particle starts at $x=0$ with a speed $v$ in the $x$ direction,
        for what values of $v$ will the particle stay on the surface at all times?
    \begin{multicols}{2}
    \begin{choices}
        \wrongchoice{$v\leq\sqrt{gd}$}
        \wrongchoice{$v\leq\sqrt{\dfrac{g}{k}}$}
        \wrongchoice{$v\leq\sqrt{gkd^2}$}
        \wrongchoice{$v\leq\sqrt{\dfrac{g}{k^2 d}}$}
        \wrongchoice{$v>0$}
    \end{choices}
    \end{multicols}
\end{question}
}

%% page 52
\element{gre}{
\begin{question}{GRE9677-Q84}
    \begin{center}
    \begin{tikzpicture}
        %% NOTE: TODO: figure 1 and 2
    \end{tikzpicture}
    \end{center}
    Two pendulums are attached to a massless spring, as shown above.
    The arms of the pendulums are of identical lengths $l$,
        but the pendulum balls have unequal masses $m_1$ and $m_2$.
    The initial distance between the masses is the equilibrium length of the spring,
        which has spring constant $K$.
    What is the highest normal mode frequency of this system?
    \begin{multicols}{2}
    \begin{choices}
        \wrongchoice{$\sqrt{\dfrac{g}{l}}$}
        \wrongchoice{$\sqrt{\dfrac{K}{m_1+m_2}}$}
        \wrongchoice{$\sqrt{\dfrac{K}{m_1}+\dfrac{K}{m_2}}$}
        \wrongchoice{$\sqrt{\dfrac{g}{l}+\dfrac{K}{m_1}+\dfrac{K}{m_2}}$}
        \wrongchoice{$\sqrt{\dfrac{2g}{l}+\dfrac{K}{m_1+m_2}}$}
    \end{choices}
    \end{multicols}
\end{question}
}

%% page 54
\element{gre}{
\begin{question}{GRE9677-Q85}
    \begin{center}
    \begin{tikzpicture}
        %% NOTE: TODO: figure 1 and 2
    \end{tikzpicture}
    \end{center}
    Small-amplitude standing waves of wavelength $\lambda$ occur on a string with tension $T$,
        mass per unit length $\mu$, and length $L$. 
    One end of the string is fixed and the other end is attached to a ring of mass $M$ that slides on a frictionless rod,
        as shown in the figure above. 
    When gravity is neglected,
        which of the following conditions correctly determines the wavelength? 
    (You might want to consider the limiting cases $M\to{}0$ and $M\to\infty$.)
    \begin{choices}
        \wrongchoice{$\dfrac{\mu}{M}=\dfrac{2\pi}{\lambda}\mathrm{cot}\dfrac{2\pi{}L}{\lambda}$}
        \wrongchoice{$\dfrac{\mu}{M}=\dfrac{2\pi}{\lambda}\mathrm{tan}\dfrac{2\pi{}L}{\lambda}$}
        \wrongchoice{$\dfrac{\mu}{M}=\dfrac{2\pi}{\lambda}\mathrm{sin}\dfrac{2\pi{}L}{\lambda}$}
        \wrongchoice{$\lambda=\dfrac{2L}{n}$, $n=1,2,3\ldots$}
        \wrongchoice{$\lambda=\dfrac{2L}{n+\frac{1}{2}}$, $n=1,2,3\ldots$}
    \end{choices}
\end{question}
}

%% page 56
\element{gre}{
\begin{question}{GRE9677-Q86}
    A positvely charged particle is moving in the $xy$-plane in a region where there is non-zero uniform magnetic field $B$ in the $+z$-direction and a non-zero uniform electric field $E$ in the $+y$-direction.
    Which of the following is a possible trajectory for the particle?
    \begin{multicols}{2}
    \begin{choices}
        \AMCboxDimensions{down=-0.75cm}
        \wrongchoice{
            \begin{tikzpicture}
                %% NOTE: TODO: figure 1 and 2
            \end{tikzpicture}
        }
    \end{choices}
    \end{multicols}
\end{question}
}

\element{gre}{
\begin{question}{GRE9677-Q87}
    \begin{center}
    \begin{tikzpicture}
        %% NOTE: TODO: figure 1 and 2
    \end{tikzpicture}
    \end{center}
    Two small pith balls, each carrying a charge $q$, are attached to the ends of a light rod of length $d$,
        which is suspended from the ceiling by a thin torsion-free fiber,
        as shown in the figure above.
    There is a uniform magnetic field $\mathbf{B}$,
        pointing straight down,
        in the cylindrical region of radius $R$ around the fiber.
    The system is initially at rest.
    If the magnetic field is turned off,
        which of the following describes what happens to the system?
    \begin{choices}
        \wrongchoice{It rotates with angular momentum $QBR^2$}
        \wrongchoice{It rotates with angular momentum $\dfrac{1}{4}QBd^2$}
        \wrongchoice{It rotates with angular momentum $\dfrac{1}{4}qBRd$}
        \wrongchoice{It does not rotate because to do so would violate conservation of angular momentum.}
        \wrongchoice{It does not move because magnetic forces do no work.}
    \end{choices}
\end{question}
}

%% page 58
\element{gre}{
\begin{question}{GRE9677-Q88}
    \begin{center}
    \begin{tikzpicture}
        %% NOTE: TODO: figure 1 and 2
    \end{tikzpicture}
    \end{center}
    A coaxial cable has the cross section shown in the figure above.
    The shaded region is insulated.
    The regions in which $r<a$ and $b<r<c$ are conducting.
    A uniform dc current density of total current $I$ flows along the inner part of the cable $(\left(r<a\right)$ and returns along the outer part of the cable $\left(b<r<c\right)$ in the directions shown.
    The radial dependence of the magnitude of the magnetic field,
        $H$, is shown by which of the following?
    \begin{multicols}{2}
    \begin{choices}
        \AMCboxDimensions{down=-0.4cm}
        \wrongchoice{
            \begin{tikzpicture}
                %% NOTE: TODO: pgfplots
            \end{tikzpicture}
        }
    \end{choices}
    \end{multicols}
\end{question}
}

%% page 60
\element{gre}{
\begin{question}{GRE9677-Q89}
    \begin{center}
    \begin{tikzpicture}
        %% NOTE: TODO: figure 1 and 2
    \end{tikzpicture}
    \end{center}
    A particle with charge $q$ and momentum $p$ is moving in the horizontal plane under the action of a uniform vertical magnetic field of magnitude $B$.
    Measurements are made of the particle's trajectory to determine the ``sagitta'' $s$ and half-chord length $l$,
        as shown in the figure above.
    Which of the following expressions gives the particle's momentum in terms of $q$, $B$, $s$, and $l$?
    (Assume $s\ll{}l$.)
    \begin{multicols}{2}
    \begin{choices}
        \wrongchoice{$\dfrac{qBs^2}{2l}$}
        \wrongchoice{$\dfrac{qBs^2}{l}$}
        \wrongchoice{$\dfrac{qBl}{s}$}
        \wrongchoice{$\dfrac{qBl^2}{2s}$}
        \wrongchoice{$\dfrac{qBl^2}{8s}$}
    \end{choices}
    \end{multicols}
\end{question}
}

%\element{gre}{
%\begin{question}{GRE9677-Q90}
%    THIS ITEM WAS NOT SCORED
%    \begin{choices}
%        \wrongchoice{}
%    \end{choices}
%\end{question}
%}

%% page 62
\element{gre}{
\begin{question}{GRE9677-Q91}
    \begin{center}
    \begin{tikzpicture}
        %% NOTE: TODO: figure 1 and 2
    \end{tikzpicture}
    \end{center}
    An experimenter needs to heat a small sample to \SI{900}{\kelvin},
        but the only available oven has a maximum temperature of \SI{600}{\kelvin}.
    Could the experimenter heat the sample to \SI{900}{\kelvin} by using a large lens to concentrate the radiation from the oven onto the sample,
        as shown above?
    \begin{choices}
        \wrongchoice{Yes, if the volume of the oven is at least $3/2$ the volume of the sample.}
        \wrongchoice{Yes, if the area of the front of the oven is at least $3/2$ the area of the front of the sample.}
        \wrongchoice{Yes, if the sample is placed at the focal point of the lens.}
        \wrongchoice{No, because it would violate conservation of energy.}
        \wrongchoice{No, because it would violate the second law of thermodynamics.}
    \end{choices}
\end{question}
}

\element{gre}{
\begin{question}{GRE9677-Q92}
    A particle of mass $m$ moves in a one-dimensional potential $V(x)=-ax^2+bX^4$, where $a$ and $b$ are positive constants.
    The angular frequency of small oscillations about the minima of the potential is equal to:
    \begin{multicols}{2}
    \begin{choices}
        \wrongchoice{$\pi\left(\dfrac{a}{2b}\right)^{1/2}$}
        \wrongchoice{$\pi\left(\dfrac{a}{m}\right)^{1/2}$}
        \wrongchoice{$\pi\left(\dfrac{a}{mb}\right)^{1/2}$}
        \wrongchoice{$2\left(\dfrac{a}{m}\right)^{1/2}$}
        \wrongchoice{$\left(\dfrac{a}{2m}\right)^{1/2}$}
    \end{choices}
    \end{multicols}
\end{question}
}

%% page 64
\element{gre}{
\begin{question}{GRE9677-Q93}
    \begin{center}
    \begin{tikzpicture}
        %% NOTE: TODO: figure 1 and 2
    \end{tikzpicture}
    \end{center}
    A particle of mass $m$ moves in the potential shown above. 
    The period of the motion when the particle has energy $E$ is:
    \begin{multicols}{2}
    \begin{choices}
        \wrongchoice{$\sqrt{\dfrac{k}{m}}$}
        \wrongchoice{$2\pi\sqrt{\dfrac{m}{k}}$}
        \wrongchoice{$2\sqrt{\dfrac{2E}{mg^2}}$}
        \wrongchoice{$\pi\sqrt{\dfrac{m}{k}} + 2\sqrt{\dfrac{2E}{mg^2}}$}
        \wrongchoice{$2\pi\sqrt{\dfrac{m}{k}} + 4\sqrt{\dfrac{2E}{mg^2}}$}
    \end{choices}
    \end{multicols}
\end{question}
}

%% page 66
\element{gre}{
\begin{question}{GRE9677-Q94}
    A system consists of $N$ weakly interacting subsystems,
        each with two internal quantum states with energies $0$ and $\epsilon$.
    The internal energy for this system at absolute temperature $T$ is equal to:
    \begin{multicols}{2}
    \begin{choices}
        \wrongchoice{$N\epsilon$}
        \wrongchoice{$\dfrac{3}{2}NKT$}
        \wrongchoice{$N\epsilon\mathrm{e}^{-\epsilon/KT}$}
        \wrongchoice{$\dfrac{N\epsilon}{\mathrm{e}^{-\epsilon/KT}+1}$}
        \wrongchoice{$\dfrac{N\epsilon}{1+\mathrm{e}^{-\epsilon/KT}}$}
    \end{choices}
    \end{multicols}
\end{question}
}

\element{gre}{
\begin{question}{GRE9677-Q95}
    Which of the following curves is characteristic of the specific heat $C_v$ of a metal such as lead, tin, or aluminum in the temperature region where it becomes superconducting?
    \begin{multicols}{2}
    \begin{choices}
        \AMCboxDimensions{down=-0.4cm}
        \wrongchoice{
            \begin{tikzpicture}
                %% NOTE: TODO: pgfplots
            \end{tikzpicture}
        }
    \end{choices}
    \end{multicols}
\end{question}
}

%% page 68
\element{gre}{
\begin{question}{GRE9677-Q96}
    Which of the following reasons explains why a photon cannot decay to an electron and a positron $\left(\gamma\to{}e^{+}+e^{-}\right)$ in free space?
    \begin{choices}
        \wrongchoice{Linear momentum and energy are not both conserved.}
        \wrongchoice{Linear momentum and angular momentum are not both conserved.}
        \wrongchoice{Angular momentum and partity are not both conserved.}
        \wrongchoice{Parity and strangeness are not both conserved.}
        \wrongchoice{Charge and lepton number are not both conserved.}
    \end{choices}
\end{question}
}

\element{gre}{
\begin{question}{GRE9677-Q97}
    A particle of mass $m$ has the wave function $\Psi\left(x,t\right)=\mathrm{E}^{i\omega{}t}\left[\right]$,
        where $\alpha$ and $\beta$ are complex constants and $\omega$ and $k$ are real constants.
    The probability current density is equal to which of the following?
    (Note: $\alpha^{\ast}$ denotes the complex conjugates of $\alpha$ and $|\alpha|^2 = \alpha^{\ast}\alpha$.)
    \begin{choices}
        \wrongchoice{zero}
        \wrongchoice{$\dfrac{\hbar{}k}{m}$}
        \wrongchoice{$\dfrac{\hbar{}k}{2m}\left(|\alpha|^2 + |\beta|^2\right)$}
        \wrongchoice{$\dfrac{\hbar{}k}{2}\left(|\alpha|^2 - |\beta|^2\right)$}
        \wrongchoice{$\dfrac{\hbar{}k}{2mi}\left(\alpha^{\ast}\beta - \beta^{\ast}\alpha\right)$}
    \end{choices}
\end{question}
}

\element{gre}{
\begin{question}{GRE9677-Q98}
    A particle of mass $m$ is acted on by a harmonic force with potential energy function $V(x)=m\omega^2 x^2/2$
        (a one-dimensional simple harmonic oscillator.)
    If there is a wall at $x=0$ so that $V=\infty$ for $x<0$,
        then the energy are equal to:
    \begin{choices}
        \wrongchoice{$0$, $\hbar\omega$, $2\hbar\omega$, $\ldots$}
        \wrongchoice{$0$, $\dfrac{\hbar\omega}{2}$, $\hbar\omega$, $\ldots$}
        \wrongchoice{$\dfrac{\hbar\omega}{2}$, $\dfrac{3\hbar\omega}{2}$, $\dfrac{5\hbar\omega}{2}$, $\ldots$}
        \wrongchoice{$\dfrac{3\hbar\omega}{2}$, $\dfrac{7\hbar\omega}{2}$, $\dfrac{11\hbar\omega}{2}$, $\ldots$}
        \wrongchoice{$0$, $\dfrac{3\hbar\omega}{2}$, $\dfrac{5\hbar\omega}{2}$, $\ldots$}
    \end{choices}
\end{question}
}

%% page 70
\element{gre}{
\begin{question}{GRE9677-Q99}
    The electronic energy levels of atoms of a certain gas are given by $E_n=E_1 n^2$,
        where $n=1,2,3,\ldots$
    Assume that transitions are allowed between all levels.
    If one wanted to construct a laser from this gas by pumping the $n=1\to{}n=3$ transition, which energy level or levels would have to e metastable?
    \begin{choices}
        \wrongchoice{$n=1$ only}
        \wrongchoice{$n=2$ only}
        \wrongchoice{$n=1$ and $n=3$ only}
        \wrongchoice{$n=1$, $n=2$,  and $n=3$}
        \wrongchoice{None}
    \end{choices}
\end{question}
}

\element{gre}{
\begin{question}{GRE9677-Q100}
    The operator $\hat{a}=\sqrt{\dfrac{m\omega_0}{2\hbar}}\left(\hat{x}+i\dfrac{\hat{p}}{m\omega_0}\right)$, when operating on a harmonic energy eigenstate $\Psi_n$ with energy $E_n$,
        produced another energy eigenstate whose energy is $E_n=\hbar\omega_0$.
    Which of the following is true?
    \begin{itemize}
        \item[I.] $\hat{a}$ commutes with the Hamiltonian.
        \item[II.] $\hat{a}$ is a Hermitian operator and therefore an observable.
        \item[III.] The adjoint operator $\hat{a}^{\dagger}\neq\hat{a}$ commutes with the Hamiltonian.
    \end{itemize}
    \begin{choices}
        \wrongchoice{I only}
        \wrongchoice{II only}
        \wrongchoice{III only}
        \wrongchoice{I and II only}
        \wrongchoice{I and III only}
    \end{choices}
\end{question}
}


\endinput


