
%% British Columbia Provincial Physics Exams
%%--------------------------------------------------


%% Section January 2000
%%----------------------
\element{questionBank}{
\begin{question}{Jan2000-Q01}
    Which of the following situations involves the use of kinematics?
    \begin{choices}
        \wrongchoice{Solving a back emf problem}
      \correctchoice{Solving a projectile motion problem}
        \wrongchoice{Determining the internal resistance of a cell}
        \wrongchoice{Determining the sum of two momentum vectors}
    \end{choices}
\end{question}
}

\element{questionBank}{
\begin{question}{Jan2000-Q02}
    Consider the diagram below.
    \begin{center}
    \begin{tikzpicture}
        %% NOTE: TODO: draw tikz
    \end{tikzpicture}
    \end{center}
    What are the components, $F_x$ and $F_y$, of the \SI{125}{newton} force?
    \begin{choices}
        \wrongchoice{$F_x=\SI{-62.6}{\newton}$ and $F_y=\SI{72.2}{\newton}$}
        \wrongchoice{$F_x=\SI{-72.2}{\newton}$ and $F_y=\SI{62.5}{\newton}$}
        \wrongchoice{$F_x=\SI{-62.6}{\newton}$ and $F_y=\SI{108}{\newton}$}
        \wrongchoice{$F_x=\SI{-108}{\newton}$ and $F_y=\SI{62.5}{\newton}$}
    \end{choices}
\end{question}
}

\element{questionBank}{
\begin{question}{Jan2000-Q03}
    A projectile is launched at \ang{35.0} above the horizontal
        with an initial velocity of \SI{120}{\meter\per\second}.
    What is the projectile's speed \SI{3.00}{\second} later?
    \begin{multicols}{2}
    \begin{choices}
        \wrongchoice{\SI{68.8}{\meter\per\second}}
        \wrongchoice{\SI{98.3}{\meter\per\second}}
        \wrongchoice{\SI{106}{\meter\per\second}}
        \wrongchoice{\SI{120}{\meter\per\second}}
    \end{choices}
    \end{multicols}
\end{question}
}

\element{questionBank}{
\begin{question}{Jan2000-Q04}
    A block of mass $m$ remains at rest on an incline as shown in the diagram.
    \begin{center}
    \begin{tikzpicture}
        %% NOTE: TODO: draw tikz
    \end{tikzpicture}
    \end{center}
    The force acting up the ramp on this block is
    \begin{choices}
        \wrongchoice{$0$.}
        \wrongchoice{$mg$.}
        \wrongchoice{less than $mg$.}
        \wrongchoice{more than $mg$.}
    \end{choices}
\end{question}
}

\element{questionBank}{
\begin{question}{Jan2000-Q05}
    What is the minimum work done when a \SI{65}{\kilo\gram} student
        climbs an \SI{8.0}{\meter} high stairway in \SI{12}{\second}?
    \begin{multicols}{2}
    \begin{choices}
        \wrongchoice{\SI{420}{\joule}}
        \wrongchoice{\SI{520}{\joule}}
        \wrongchoice{\SI{5 100}{\joule}}
        \wrongchoice{\SI{6 200}{\joule}}
    \end{choices}
    \end{multicols}
\end{question}
}

\element{questionBank}{
\begin{question}{Jan2000-Q06}
    Which of the following is equal to impulse?
    \begin{multicols}{2}
    \begin{choices}
        \wrongchoice{Energy}
        \wrongchoice{Momentum}
        \wrongchoice{Change in energy}
        \wrongchoice{Change in momentum}
    \end{choices}
    \end{multicols}
\end{question}
}

\element{questionBank}{
\begin{question}{Jan2000-Q07}
    A \SI{1.50e3}{\kilo\gram} car travelling at \SI{11.0}{\meter\per\second}
        collides with a wall as shown.
    \begin{center}
    \begin{tikzpicture}
        %% NOTE: TODO: draw tikz
    \end{tikzpicture}
    \end{center}
    The car rebounds off the wall with a speed of \SI{1.3}{\meter\per\second}. 
    If the collision lasts for \SI{1.7}{\second},
        what force does the wall apply to the car during the collision?
    \begin{multicols}{2}
    \begin{choices}
        \wrongchoice{\SI{8.6e3}{\newton}}
        \wrongchoice{\SI{1.1e4}{\newton}}
        \wrongchoice{\SI{1.5e4}{\newton}}
        \wrongchoice{\SI{1.8e4}{\newton}}
    \end{choices}
    \end{multicols}
\end{question}
}

\element{questionBank}{
\begin{question}{Jan2000-Q08}
    A \SI{1 500}{\kilo\gram} car travelling at \SI{25}{\meter\per\second}
        collides with a \SI{2 500}{\kilo\gram} van stopped at a traffic light. 
    As a result of the collision the two vehicles become entangled. 
    With what initial speed will the entangled mass move off, 
        and is the collision elastic or inelastic?
    \begin{choices}
        \wrongchoice{speed: \SI{9.4}{\meter\per\second}, type: Elastic}
        \wrongchoice{speed: \SI{9.4}{\meter\per\second}, type: Inelastic}
        \wrongchoice{speed: \SI{15}{\meter\per\second}, type: Elastic}
        \wrongchoice{speed: \SI{15}{\meter\per\second}, type: Inelastic}
    \end{choices}
\end{question}
}

\element{questionBank}{
\begin{question}{Jan2000-Q09}
    Three objects travel as shown.
    \begin{center}
    \begin{tikzpicture}
        %% NOTE: TODO: draw tikz
    \end{tikzpicture}
    \end{center}
    What is the magnitude of the momentum of object $R$
        so that the combined masses remain stationary after they collide?
    \begin{multicols}{2}
    \begin{choices}
        \wrongchoice{\SI{19}{\kilo\gram\meter\per\second}}
        \wrongchoice{\SI{30}{\kilo\gram\meter\per\second}}
        \wrongchoice{\SI{36}{\kilo\gram\meter\per\second}}
        \wrongchoice{\SI{48}{\kilo\gram\meter\per\second}}
    \end{choices}
    \end{multicols}
\end{question}
}

\element{questionBank}{
\begin{question}{Jan2000-Q10}
    A force $F$ is applied to a uniform horizontal beam as shown in the diagram below.
    \begin{center}
    \begin{tikzpicture}
        %% NOTE: TODO: draw tikz
    \end{tikzpicture}
    \end{center}
    Which of the following is a correct expression for the
        torque on the beam about pivot point $P$ due to this force?
    \begin{multicols}{2}
    \begin{choices}
        \wrongchoice{$F\sin\theta\cdot d$}
        \wrongchoice{$F\sin\theta\cdot dl$}
        \wrongchoice{$F\cos\theta\cdot d$}
        \wrongchoice{$F\cos\theta\cdot dl$}
    \end{choices}
    \end{multicols}
\end{question}
}

\element{questionBank}{
\begin{question}{Jan2000-Q11}
    What is the magnitude of the sum of the two forces shown in the diagram below?
    \begin{center}
    \begin{tikzpicture}
        %% NOTE: TODO: draw tikz
    \end{tikzpicture}
    \end{center}
    \begin{multicols}{2}
    \begin{choices}
        \wrongchoice{\SI{46}{\newton}}
        \wrongchoice{\SI{102}{\newton}}
        \wrongchoice{\SI{137}{\newton}}
        \wrongchoice{\SI{142}{\newton}}
    \end{choices}
    \end{multicols}
\end{question}
}



