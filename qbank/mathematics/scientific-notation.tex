

%% Scientific Notation
%%--------------------------------------------------

%% significant figures in 
%% 1. addition
%% 2. multiplication

%% Solving for a single variable in a first degree equation. 

%% coefficient and significand or mantissa
%% E notation

%% Significance arithmetic is a set of rules (sometimes called significant figure rules)
%%      for approximating the propagation of uncertainty in scientific or statistical calculations.
%% These rules can be used to find the appropriate number of significant figures
%%      to use to represent the result of a calculation

%% Propagation of uncertainty


%% Equations
%%--------------------
\element{SI}{
\begin{questionmult}{SI-Q01}
\luaexec{
    %% Question
    local Q = [[
        If the period of a harmonic motion of a mass on a spring
            is related to the mass  pendulum is related to the length 
        Which of the following are \string\emph{defined}
            units in the International System of Units?
    ]]
    %% Random Permutations
    local tab1 = {}
    for i=1,22 do
        tab1[i] = i
    end
    local tab2 = {}
    for i=1,7 do
        tab2[i] = i
    end
    tab1 = permute(tab1,22,22)
    tab2 = permute(tab2,7,7)
    %% Random correct vs wrong
    local n1 = math.random(1,3)
    local n2 = 4 - n1
    %% Print Question
    tex.print( Q )
    %% Print MC Options
    tex.print( BeginMulticols )
        tex.print( BeginChoices )
            for i=1,n2 do
                tex.print( string.format(CorrectChoice,defined[tab2[i]]) )
            end
            for i=1,n1 do
                tex.print( string.format(WrongChoice,derived[tab1[i]]) )
            end
        tex.print( EndChoices )
    tex.print( EndMulticols )
}
\end{questionmult}
}


%% Derived Units
%%--------------------
\element{SI}{
\begin{questionmult}{SI-Q02}
\luaexec{
    %% Question
    local Q = [[
        Which of the following are \string\emph{derived}
            units in the International System of Units?
    ]]
    %% Random Permutations
    local tab1 = {}
    for i=1,22 do
        tab1[i] = i
    end
    local tab2 = {}
    for i=1,7 do
        tab2[i] = i
    end
    tab1 = permute(tab1,22,22)
    tab2 = permute(tab2,7,7)
    %% Random correct vs wrong
    local n1 = math.random(1,3)
    local n2 = 4 - n1
    %% Print Question
    tex.print( Q )
    %% Print MC Options
    tex.print( BeginMulticols )
        tex.print( BeginChoices )
            for i=1,n1 do
                tex.print( string.format(CorrectChoice,derived[tab1[i]]) )
            end
            for i=1,n2 do
                tex.print( string.format(WrongChoice,defined[tab2[i]]) )
            end
        tex.print( EndChoices )
    tex.print( EndMulticols )
}
\end{questionmult}
}

