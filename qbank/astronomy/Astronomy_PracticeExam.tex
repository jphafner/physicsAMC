
b) in the Galactic halo
c) in the Galactic center
d) in the intergalactic medium, far from galaxies
e) all of the above

43) Globular clusters are composed primarily of
a) old stars b) young stars c) hot stars d) massive stars e) blue stars

44) Young stars are located primarily in
a) the spiral arms b) the halo c) the globular clusters
d) the nucleus e) all of the above

45) Which type of stars contain the lowest percentage of heavy elements?
a) young stars b) old stars c) massive stars d) hot stars e) binary stars

46) Approximately what fraction of stars are members of binary star systems?
a) very few, a few percent b) about half, 50 percent
c) almost all stars are binaries d) we have no idea
e) there are only 5 known binary star systems

47) Binary stars are our primary source of information about
a) stellar colors b) stellar mass c) stellar chemical composition
d) stellar temperatures e) both a) and d)

48) In 1920 a famous debate was held between Shapley and Curtis.
The debate concerned
a) the nature of spiral nebulae
b) the nature of the redshifts observed in distant galaxies
c) the nature of binary stars
d) the nature of planetary nebulae
e) the nature of supernovae

49) Using Cepheid variable stars as distance indicators,
Hubble discovered that the Andromeda nebula was
a) a separate galaxy outside the Milky Way
b) a relatively nearby planetary nebulae
c) a distant elliptical galaxy
d) a reflection nebula
e) a new constellation

50) The Milky Way can be best described as
a) an elliptical galaxy b) an irregular galaxy c) a supernova remnant
d) a planetary nebula e) none of the above

51) Which type of galaxy contains the largest percentage of gas and dust?
a) elliptical b) spiral c) irregular d) interacting e) barred spiral

52) Which type of galaxy contains a significant population of young stars?
a) elliptical b) spiral c) irregular d) all of the above do
e) both b) and c), but not a)

53) Roughly how many stars are there in the Milky Way?
a) thousands, but less than a million
b) millions, but less than a billion
c) more than a billion
d) only a few hundred known
e) The Milky Way is not a galaxy, it is a planetary nebula

54) Roughly how many galaxies are there in the visible universe?
a) thousands, but less than a million
b) millions, but less than a billion
c) more than a billion
d) only a few hundred known
e) there are precisely 12765 known galaxies

55) Galaxies tend to
a) be found in clusters
b) be moving away from the Milky Way
c) be isolated from one another
d) be found orbiting our galaxy
e) both a) and b)

56) In the late 1920's Edwin Hubble made a profound discovery.
a) He found that the Milky Way was rotating.
b) He found that the galaxies were made of gas and dust.
c) He found that quasars were very distant.
d) He discovered the Cosmic Microwave Background radiation.
e) He found that the Universe was expanding.

57) The Hubble Space Telescope is particularly useful because
a) It is above the Earth's absorbing and distorting atmosphere.
b) It is the world's biggest telescope
c) It is perfectly stationary, allowing precise measurements
d) all of the above
e) none of the above

58) What is the difference between mass and weight?
a) mass is a measure of the total volume of an object
b) weight is mass squared
c) mass and weight are equivalent
d) weight is a force that depends on the strength of the local gravity
e) none of the above

59) Why do we think that when we observe quasars we are seeing the Universe
as it was billions of years ago?
a) Because quasars are very old
b) Because quasars are so distant that it has taken the light billions
of years to reach us.
c) Because quasars form and die very quickly
d) Because quasars are moving toward us so rapidly that the light travels
much slower than normal
e) none of the above

60) What is the visible "surface" of the sun called?
a) the photosphere b) the corona c) the envelope d) the magnetosphere
e) the stratosphere

61) What is the source of energy for stars?
a) chemical burning b) nuclear fission c) nuclear fusion
d) gravitational contraction e) all of the above are sources

62) Where was the carbon in your body manufactured?
a) the Big Bang b) a planetary nebula c) in the core of the sun
d) in Detroit MI e) none of the above

63) What elements were thought to be produced in the Big Bang?
a) Hydrogen only
b) Hydrogen, Helium, and very little else
c) All elements we see today, but in smaller concentrations
d) Helium only
e) no elements were created in the Big Bang, all were made in stars

64) What discoveries pretty much ruled out the steady state model of the
Universe?
a) the discovery of the spiral nebulae
b) the discovery that the Universe was expanding
c) the discovery of quasars and the cosmic microwave radiation.
d) the discovery that our galaxy was an elliptical galaxy
e) it is not ruled out, but is the accepted model of the universe.

65) Why don't nuclear reactions occur near the Sun's surface?
a) it is too hot b) it isn't hot enough c) they do!
d) there is no hydrogen e) there is too much turbulence

66) What is meant by the so-called greenhouse effect?
a) the heating of the atmosphere by radiation from carbon dioxide
gas that has been excited by sunlight.
b) the heating of the atmosphere by UV light from the sun
c) the heating of the atmosphere by infrared radiation from the
planet that is absorbed by certain gasses in the atmosphere
d) the heating of the atmosphere by cosmic rays
e) the heating of the atmosphere by the burning of fossil fuels.

67) What is the approximate phase of the moon today?
a) between new and first quarter
b) between 1st quarter and full
c) between full and third quarter
d) between third quarter and new
e) none of the above

68) The correct sequence of the average distances of the planets in our solar
system, ordered from the outermost, inward to the sun is
a) Pluto, Saturn, Neptune, Uranus, Jupiter, Mars, Earth, Venus, Mercury
b) Mercury, Venus, Earth, Mars, Jupiter, Saturn, Uranus, Neptune, Pluto
c) Mars, Saturn, Venus, Mercury, Earth, Uranus, Neptune, Pluto, Jupiter
d) Pluto, Neptune, Uranus, Saturn, Jupiter, Mars, Earth, Venus, Mercury
e) none of the above

69) Hydrostatic equilibrium in a star is a balance between
a) hydrogen and helium
b) water, hydrogen, and oxygen
c) water and electric (static) charge
d) gravity and magnetic pressure
e) none of the above

70) A white dwarf is a star having roughly the mass of the sun
and the size of the
a) sun b) city of San Diego c) earth c) state of Kansas
e) none of the above

71) If the sun were suddenly to collapse into a black hole,
the gravitational force on earth would
a) double
b) become so strong that the earth would be "sucked" into the sun
c) decrease because black holes cause gravity at large distances
to disappear
d) remain the same
e) none of the above

72) The most likely object to be found at the center of a planetary nebula is
a) a white dwarf star d) a main sequence star
b) a neutron star or black hole e) none of the above
c) a red giant star

73) A match-box sized chunk of neutron star stuff on Earth would
a) weigh approximately as much as the sun
b) weigh as much as a cube of sugar
c) be as dense as water
d) weigh approximately as much as Mount Everest
e) none of the above

74) Population II stars differ from population I stars in that
a) they are larger on the average
b) they lack heavy elements in their chemical composition
c) they are not found in globular star clusters
d) they are not found in spiral galaxies
e) they do not show absorption lines in their spectra

75) Dust which exists in interstellar space causes
a) light from distant stars to appear "bluer" than normal
b) a shift in the spectrum lines of nebulae associated with the dust
c) strong radio signals at a wavelength of 21 cm to be emitted
d) stellar formation to cease and terminate within the region
e) none of the above

76) On April 9, 1986 the phase of the moon was first quarter.
When did the moon rise on that day?
a) sunset b) midnight c) noon d) sunrise
e) the Moon was down all day

77) Suppose you are at the beach and observe a crescent moon about to set
on the western horizon. It is dark and you cannot read your watch.
The time is
a) between sunset and midnight b) just before dawn
c) exactly midnight d) between midnight and sunrise
e) impossible to estimate without more information.

78) Two reflecting telescopes have primary mirrors 1 m and 2 m in diameter.
How many times more light is gathered by the larger telescope in any
given interval of time?
a) 4 times as much b) twice as much c) the same d) 16 times as much
e) none of the above

79) Sirius, the apparently brightest star in the sky is 8 light-years away.
How bright would it appear if it were twice as close, that is at a distance
of 4 light-years?
a) one-half as bright d) eight times brighter
b) two times as bright e) sixteen times brighter
c) four times brighter

80) The distance from the Earth to Proxima Centauri, the closest star, is about
1 parsec. The distance from Venus to Proxima Centauri is about:
a) 1 Astronomical Unit c) 2 parsecs e) 2 seconds of arc
b) one-half parsec d) 1 parsec

81) The Doppler effect
a) can tell you the velocity of a body but not whether it is approaching
or receding
b) can tell you whether a body is approaching or receding but not its
velocity
c) works for optical radiation but not for radio waves
d) could turn out to be wrong in the unlikely event that General Relativity
is incorrect
e) works for electromagnetic radiation of any wavelength

82) Who was the astronomer who developed three laws of planetary motion
which modified the Copernican ideas?
a) Galileo b) Ptolemy c) Brahe d) Kepler e) none of the above

83) A good candidate for a black hole binary is one where:
a) the visible star is a white dwarf, and the invisible star is
of about equal mass
b) neither star is visible
c) the unseen star seems to be quite massive
d) the visible star has been mostly eaten away
e) both stars are visible and very massive

84) The usual way we determine the orbital inclination of a spectroscopic
binary star system is from:
a) eclipses b) resolving the two stars directly with a telescope
c) radial velocity variations d) a determination of the temperatures
e) none of the above

85) Prior to the 1960's, one Cosmology that was debated as a possibility was
the Steady State model. It proposed that the Universe was infinite,
ageless, and unchanging on the large scale. In order to satisfy the
observed expansion of the Universe, the Steady State model required:
a) the continuous creation of matter from nothing, which in turn drove the
expansion and replenished the extragalactic environment
b) rotation of the Universe as well as expansion
c) blue-shifted spectra of galaxies at very large distances
d) slowing of the speed of light with distance to mimic extragalactic
redshifts
e) antigravity as well as antimatter

86) One possible energy source for quasars may be:
a) exploding stars b) electron collisions
c) rapid quasar rotation d) supermassive black holes
e) none of the above

87) The gravitational force is different from the electromagnetic force in that
a) the gravitational force is always attractive, but the electromagnetic
force can be either attractive or repulsive.
b) the gravitational force only depends on the mass of the attracting body.
c) the gravitational force gets weaker as the electrical force gets stronger.
d) gravity can be neutralized or "shielded"
e) gravity is much stronger in the domain of atoms and molecules.
e) all of the above

88) Is it possible to see a full moon at noon?
a) yes b) no c) in winter d) in summer e) if its a "blue" moon

89) During our summer (June - Sept), the night time hours in San Francisco
(approx 400 miles north of here) are
a) shorter than in San Diego
b) longer than in San Diego
c) the same as in San Diego
d) sometimes shorter, sometimes longer, depending on the sunspot cycle
e) sometimes shorter, sometimes longer, depending on the lunar phase.

90) Which of the following ideas is the most difficult to understand or explain
if quasars are assumed to be very distant?
a) the processes which result in a large amount of energy being generated
in a small region of space
b) the large red shift in the emission lines
c) the lifetimes of the quasars
d) the chemical composition of the quasars
e) none of the above

91) Which fundamental property of a star is most important in determining
its further evolution
a) chemical composition b) apparent brightness
c) mass d) age e) none of the above

92) X-rays and radio waves are similar in that
a) their wavelengths are comparable
b) they are both electromagnetic radiation, but X-rays travel faster
c) their photons have similar energies
d) all of the above
e) none of the above

93) What is the difference between a comet and a meteor?
a) a comet is made primarily of rocky material, while a meteor
is made of ice and gasses
b) a meteor is a chunk of rock, while a comet is made of ice
c) a meteor is a short-lived streak of light in the sky, while
a comet is a a ball of ices in orbit about the sun
d) they are two names for the same thing
e) none of the above

94) The question posed by Olber's paradox is
a) Why does the moon follow me when I drive in my car at night?
b) Why is it dark at night?
c) Why is the milky way pancake-shaped?
d) Why is the more massive star the least evolved?
e) none of the above

95) The cosmic microwave background radiation was discoved by
a) The space telescope
b) AT&T scientists testing a satellite communications system
c) radio engineers shortly after World War II
d) Cal Tech astronomers back in the 50's
e) none of the above

96) A star-like object, blue in color, displaying spectral emission lines
that are redshifted to an unusually large degree are known as
a) pulsars b) quasars c) black holes d) laptars e) none of the above

97) What are the most important factors in determining whether a given planet
will reatin a significant atmosphere?
a) its magnetic field
b) its mass
c) its color
d) its distance from the sun
e) both b) and d)

98) Why isn't the earth covered with craters like the moon?
a) The moon is older
b) The moon is younger
c) erosion erased them
d) There are more volcanos on the moon
e) both c) and d)

99) What causes a comet to have a tail?
a) The gravitational pull of the sun
b) The gravitational pull of the earth
c) the interaction of the comet with the solar wind
d) both a) and b)
e) none of the above

100) Astronomy is
a) the study of the effect constellations have on the motion of the Earth.
b) an essentially hypothetical science.
c) a primarily observational physical science.
d) a primarily experimental life science.
e) a scientific term for the study of astrology.
b) in the Galactic halo
c) in the Galactic center
d) in the intergalactic medium, far from galaxies
e) all of the above

43) Globular clusters are composed primarily of
a) old stars b) young stars c) hot stars d) massive stars e) blue stars

44) Young stars are located primarily in
a) the spiral arms b) the halo c) the globular clusters
d) the nucleus e) all of the above

45) Which type of stars contain the lowest percentage of heavy elements?
a) young stars b) old stars c) massive stars d) hot stars e) binary stars

46) Approximately what fraction of stars are members of binary star systems?
a) very few, a few percent b) about half, 50 percent
c) almost all stars are binaries d) we have no idea
e) there are only 5 known binary star systems

47) Binary stars are our primary source of information about
a) stellar colors b) stellar mass c) stellar chemical composition
d) stellar temperatures e) both a) and d)

48) In 1920 a famous debate was held between Shapley and Curtis.
The debate concerned
a) the nature of spiral nebulae
b) the nature of the redshifts observed in distant galaxies
c) the nature of binary stars
d) the nature of planetary nebulae
e) the nature of supernovae

49) Using Cepheid variable stars as distance indicators,
Hubble discovered that the Andromeda nebula was
a) a separate galaxy outside the Milky Way
b) a relatively nearby planetary nebulae
c) a distant elliptical galaxy
d) a reflection nebula
e) a new constellation

50) The Milky Way can be best described as
a) an elliptical galaxy b) an irregular galaxy c) a supernova remnant
d) a planetary nebula e) none of the above

51) Which type of galaxy contains the largest percentage of gas and dust?
a) elliptical b) spiral c) irregular d) interacting e) barred spiral

52) Which type of galaxy contains a significant population of young stars?
a) elliptical b) spiral c) irregular d) all of the above do
e) both b) and c), but not a)

53) Roughly how many stars are there in the Milky Way?
a) thousands, but less than a million
b) millions, but less than a billion
c) more than a billion
d) only a few hundred known
e) The Milky Way is not a galaxy, it is a planetary nebula

54) Roughly how many galaxies are there in the visible universe?
a) thousands, but less than a million
b) millions, but less than a billion
c) more than a billion
d) only a few hundred known
e) there are precisely 12765 known galaxies

55) Galaxies tend to
a) be found in clusters
b) be moving away from the Milky Way
c) be isolated from one another
d) be found orbiting our galaxy
e) both a) and b)

56) In the late 1920's Edwin Hubble made a profound discovery.
a) He found that the Milky Way was rotating.
b) He found that the galaxies were made of gas and dust.
c) He found that quasars were very distant.
d) He discovered the Cosmic Microwave Background radiation.
e) He found that the Universe was expanding.

57) The Hubble Space Telescope is particularly useful because
a) It is above the Earth's absorbing and distorting atmosphere.
b) It is the world's biggest telescope
c) It is perfectly stationary, allowing precise measurements
d) all of the above
e) none of the above

58) What is the difference between mass and weight?
a) mass is a measure of the total volume of an object
b) weight is mass squared
c) mass and weight are equivalent
d) weight is a force that depends on the strength of the local gravity
e) none of the above

59) Why do we think that when we observe quasars we are seeing the Universe
as it was billions of years ago?
a) Because quasars are very old
b) Because quasars are so distant that it has taken the light billions
of years to reach us.
c) Because quasars form and die very quickly
d) Because quasars are moving toward us so rapidly that the light travels
much slower than normal
e) none of the above

60) What is the visible "surface" of the sun called?
a) the photosphere b) the corona c) the envelope d) the magnetosphere
e) the stratosphere

61) What is the source of energy for stars?
a) chemical burning b) nuclear fission c) nuclear fusion
d) gravitational contraction e) all of the above are sources

62) Where was the carbon in your body manufactured?
a) the Big Bang b) a planetary nebula c) in the core of the sun
d) in Detroit MI e) none of the above

63) What elements were thought to be produced in the Big Bang?
a) Hydrogen only
b) Hydrogen, Helium, and very little else
c) All elements we see today, but in smaller concentrations
d) Helium only
e) no elements were created in the Big Bang, all were made in stars

64) What discoveries pretty much ruled out the steady state model of the
Universe?
a) the discovery of the spiral nebulae
b) the discovery that the Universe was expanding
c) the discovery of quasars and the cosmic microwave radiation.
d) the discovery that our galaxy was an elliptical galaxy
e) it is not ruled out, but is the accepted model of the universe.

65) Why don't nuclear reactions occur near the Sun's surface?
a) it is too hot b) it isn't hot enough c) they do!
d) there is no hydrogen e) there is too much turbulence

66) What is meant by the so-called greenhouse effect?
a) the heating of the atmosphere by radiation from carbon dioxide
gas that has been excited by sunlight.
b) the heating of the atmosphere by UV light from the sun
c) the heating of the atmosphere by infrared radiation from the
planet that is absorbed by certain gasses in the atmosphere
d) the heating of the atmosphere by cosmic rays
e) the heating of the atmosphere by the burning of fossil fuels.

67) What is the approximate phase of the moon today?
a) between new and first quarter
b) between 1st quarter and full
c) between full and third quarter
d) between third quarter and new
e) none of the above

68) The correct sequence of the average distances of the planets in our solar
system, ordered from the outermost, inward to the sun is
a) Pluto, Saturn, Neptune, Uranus, Jupiter, Mars, Earth, Venus, Mercury
b) Mercury, Venus, Earth, Mars, Jupiter, Saturn, Uranus, Neptune, Pluto
c) Mars, Saturn, Venus, Mercury, Earth, Uranus, Neptune, Pluto, Jupiter
d) Pluto, Neptune, Uranus, Saturn, Jupiter, Mars, Earth, Venus, Mercury
e) none of the above

69) Hydrostatic equilibrium in a star is a balance between
a) hydrogen and helium
b) water, hydrogen, and oxygen
c) water and electric (static) charge
d) gravity and magnetic pressure
e) none of the above

70) A white dwarf is a star having roughly the mass of the sun
and the size of the
a) sun b) city of San Diego c) earth c) state of Kansas
e) none of the above

71) If the sun were suddenly to collapse into a black hole,
the gravitational force on earth would
a) double
b) become so strong that the earth would be "sucked" into the sun
c) decrease because black holes cause gravity at large distances
to disappear
d) remain the same
e) none of the above

72) The most likely object to be found at the center of a planetary nebula is
a) a white dwarf star d) a main sequence star
b) a neutron star or black hole e) none of the above
c) a red giant star

73) A match-box sized chunk of neutron star stuff on Earth would
a) weigh approximately as much as the sun
b) weigh as much as a cube of sugar
c) be as dense as water
d) weigh approximately as much as Mount Everest
e) none of the above

74) Population II stars differ from population I stars in that
a) they are larger on the average
b) they lack heavy elements in their chemical composition
c) they are not found in globular star clusters
d) they are not found in spiral galaxies
e) they do not show absorption lines in their spectra

75) Dust which exists in interstellar space causes
a) light from distant stars to appear "bluer" than normal
b) a shift in the spectrum lines of nebulae associated with the dust
c) strong radio signals at a wavelength of 21 cm to be emitted
d) stellar formation to cease and terminate within the region
e) none of the above

76) On April 9, 1986 the phase of the moon was first quarter.
When did the moon rise on that day?
a) sunset b) midnight c) noon d) sunrise
e) the Moon was down all day

77) Suppose you are at the beach and observe a crescent moon about to set
on the western horizon. It is dark and you cannot read your watch.
The time is
a) between sunset and midnight b) just before dawn
c) exactly midnight d) between midnight and sunrise
e) impossible to estimate without more information.

78) Two reflecting telescopes have primary mirrors 1 m and 2 m in diameter.
How many times more light is gathered by the larger telescope in any
given interval of time?
a) 4 times as much b) twice as much c) the same d) 16 times as much
e) none of the above

79) Sirius, the apparently brightest star in the sky is 8 light-years away.
How bright would it appear if it were twice as close, that is at a distance
of 4 light-years?
a) one-half as bright d) eight times brighter
b) two times as bright e) sixteen times brighter
c) four times brighter

80) The distance from the Earth to Proxima Centauri, the closest star, is about
1 parsec. The distance from Venus to Proxima Centauri is about:
a) 1 Astronomical Unit c) 2 parsecs e) 2 seconds of arc
b) one-half parsec d) 1 parsec

81) The Doppler effect
a) can tell you the velocity of a body but not whether it is approaching
or receding
b) can tell you whether a body is approaching or receding but not its
velocity
c) works for optical radiation but not for radio waves
d) could turn out to be wrong in the unlikely event that General Relativity
is incorrect
e) works for electromagnetic radiation of any wavelength

82) Who was the astronomer who developed three laws of planetary motion
which modified the Copernican ideas?
a) Galileo b) Ptolemy c) Brahe d) Kepler e) none of the above

83) A good candidate for a black hole binary is one where:
a) the visible star is a white dwarf, and the invisible star is
of about equal mass
b) neither star is visible
c) the unseen star seems to be quite massive
d) the visible star has been mostly eaten away
e) both stars are visible and very massive

84) The usual way we determine the orbital inclination of a spectroscopic
binary star system is from:
a) eclipses b) resolving the two stars directly with a telescope
c) radial velocity variations d) a determination of the temperatures
e) none of the above

85) Prior to the 1960's, one Cosmology that was debated as a possibility was
the Steady State model. It proposed that the Universe was infinite,
ageless, and unchanging on the large scale. In order to satisfy the
observed expansion of the Universe, the Steady State model required:
a) the continuous creation of matter from nothing, which in turn drove the
expansion and replenished the extragalactic environment
b) rotation of the Universe as well as expansion
c) blue-shifted spectra of galaxies at very large distances
d) slowing of the speed of light with distance to mimic extragalactic
redshifts
e) antigravity as well as antimatter

86) One possible energy source for quasars may be:
a) exploding stars b) electron collisions
c) rapid quasar rotation d) supermassive black holes
e) none of the above

87) The gravitational force is different from the electromagnetic force in that
a) the gravitational force is always attractive, but the electromagnetic
force can be either attractive or repulsive.
b) the gravitational force only depends on the mass of the attracting body.
c) the gravitational force gets weaker as the electrical force gets stronger.
d) gravity can be neutralized or "shielded"
e) gravity is much stronger in the domain of atoms and molecules.
e) all of the above

88) Is it possible to see a full moon at noon?
a) yes b) no c) in winter d) in summer e) if its a "blue" moon

89) During our summer (June - Sept), the night time hours in San Francisco
(approx 400 miles north of here) are
a) shorter than in San Diego
b) longer than in San Diego
c) the same as in San Diego
d) sometimes shorter, sometimes longer, depending on the sunspot cycle
e) sometimes shorter, sometimes longer, depending on the lunar phase.

90) Which of the following ideas is the most difficult to understand or explain
if quasars are assumed to be very distant?
a) the processes which result in a large amount of energy being generated
in a small region of space
b) the large red shift in the emission lines
c) the lifetimes of the quasars
d) the chemical composition of the quasars
e) none of the above

91) Which fundamental property of a star is most important in determining
its further evolution
a) chemical composition b) apparent brightness
c) mass d) age e) none of the above

92) X-rays and radio waves are similar in that
a) their wavelengths are comparable
b) they are both electromagnetic radiation, but X-rays travel faster
c) their photons have similar energies
d) all of the above
e) none of the above

93) What is the difference between a comet and a meteor?
a) a comet is made primarily of rocky material, while a meteor
is made of ice and gasses
b) a meteor is a chunk of rock, while a comet is made of ice
c) a meteor is a short-lived streak of light in the sky, while
a comet is a a ball of ices in orbit about the sun
d) they are two names for the same thing
e) none of the above

94) The question posed by Olber's paradox is
a) Why does the moon follow me when I drive in my car at night?
b) Why is it dark at night?
c) Why is the milky way pancake-shaped?
d) Why is the more massive star the least evolved?
e) none of the above

95) The cosmic microwave background radiation was discoved by
a) The space telescope
b) AT&T scientists testing a satellite communications system
c) radio engineers shortly after World War II
d) Cal Tech astronomers back in the 50's
e) none of the above

96) A star-like object, blue in color, displaying spectral emission lines
that are redshifted to an unusually large degree are known as
a) pulsars b) quasars c) black holes d) laptars e) none of the above

97) What are the most important factors in determining whether a given planet
will reatin a significant atmosphere?
a) its magnetic field
b) its mass
c) its color
d) its distance from the sun
e) both b) and d)

98) Why isn't the earth covered with craters like the moon?
a) The moon is older
b) The moon is younger
c) erosion erased them
d) There are more volcanos on the moon
e) both c) and d)

99) What causes a comet to have a tail?
a) The gravitational pull of the sun
b) The gravitational pull of the earth
c) the interaction of the comet with the solar wind
d) both a) and b)
e) none of the above

100) Astronomy is
a) the study of the effect constellations have on the motion of the Earth.
b) an essentially hypothetical science.
c) a primarily observational physical science.
d) a primarily experimental life science.
e) a scientific term for the study of astrology.
b) in the Galactic halo
c) in the Galactic center
d) in the intergalactic medium, far from galaxies
e) all of the above

43) Globular clusters are composed primarily of
a) old stars b) young stars c) hot stars d) massive stars e) blue stars

44) Young stars are located primarily in
a) the spiral arms b) the halo c) the globular clusters
d) the nucleus e) all of the above

45) Which type of stars contain the lowest percentage of heavy elements?
a) young stars b) old stars c) massive stars d) hot stars e) binary stars

46) Approximately what fraction of stars are members of binary star systems?
a) very few, a few percent b) about half, 50 percent
c) almost all stars are binaries d) we have no idea
e) there are only 5 known binary star systems

47) Binary stars are our primary source of information about
a) stellar colors b) stellar mass c) stellar chemical composition
d) stellar temperatures e) both a) and d)

48) In 1920 a famous debate was held between Shapley and Curtis.
The debate concerned
a) the nature of spiral nebulae
b) the nature of the redshifts observed in distant galaxies
c) the nature of binary stars
d) the nature of planetary nebulae
e) the nature of supernovae

49) Using Cepheid variable stars as distance indicators,
Hubble discovered that the Andromeda nebula was
a) a separate galaxy outside the Milky Way
b) a relatively nearby planetary nebulae
c) a distant elliptical galaxy
d) a reflection nebula
e) a new constellation

50) The Milky Way can be best described as
a) an elliptical galaxy b) an irregular galaxy c) a supernova remnant
d) a planetary nebula e) none of the above

51) Which type of galaxy contains the largest percentage of gas and dust?
a) elliptical b) spiral c) irregular d) interacting e) barred spiral

52) Which type of galaxy contains a significant population of young stars?
a) elliptical b) spiral c) irregular d) all of the above do
e) both b) and c), but not a)

53) Roughly how many stars are there in the Milky Way?
a) thousands, but less than a million
b) millions, but less than a billion
c) more than a billion
d) only a few hundred known
e) The Milky Way is not a galaxy, it is a planetary nebula

54) Roughly how many galaxies are there in the visible universe?
a) thousands, but less than a million
b) millions, but less than a billion
c) more than a billion
d) only a few hundred known
e) there are precisely 12765 known galaxies

55) Galaxies tend to
a) be found in clusters
b) be moving away from the Milky Way
c) be isolated from one another
d) be found orbiting our galaxy
e) both a) and b)

56) In the late 1920's Edwin Hubble made a profound discovery.
a) He found that the Milky Way was rotating.
b) He found that the galaxies were made of gas and dust.
c) He found that quasars were very distant.
d) He discovered the Cosmic Microwave Background radiation.
e) He found that the Universe was expanding.

57) The Hubble Space Telescope is particularly useful because
a) It is above the Earth's absorbing and distorting atmosphere.
b) It is the world's biggest telescope
c) It is perfectly stationary, allowing precise measurements
d) all of the above
e) none of the above

58) What is the difference between mass and weight?
a) mass is a measure of the total volume of an object
b) weight is mass squared
c) mass and weight are equivalent
d) weight is a force that depends on the strength of the local gravity
e) none of the above

59) Why do we think that when we observe quasars we are seeing the Universe
as it was billions of years ago?
a) Because quasars are very old
b) Because quasars are so distant that it has taken the light billions
of years to reach us.
c) Because quasars form and die very quickly
d) Because quasars are moving toward us so rapidly that the light travels
much slower than normal
e) none of the above

60) What is the visible "surface" of the sun called?
a) the photosphere b) the corona c) the envelope d) the magnetosphere
e) the stratosphere

61) What is the source of energy for stars?
a) chemical burning b) nuclear fission c) nuclear fusion
d) gravitational contraction e) all of the above are sources

62) Where was the carbon in your body manufactured?
a) the Big Bang b) a planetary nebula c) in the core of the sun
d) in Detroit MI e) none of the above

63) What elements were thought to be produced in the Big Bang?
a) Hydrogen only
b) Hydrogen, Helium, and very little else
c) All elements we see today, but in smaller concentrations
d) Helium only
e) no elements were created in the Big Bang, all were made in stars

64) What discoveries pretty much ruled out the steady state model of the
Universe?
a) the discovery of the spiral nebulae
b) the discovery that the Universe was expanding
c) the discovery of quasars and the cosmic microwave radiation.
d) the discovery that our galaxy was an elliptical galaxy
e) it is not ruled out, but is the accepted model of the universe.

65) Why don't nuclear reactions occur near the Sun's surface?
a) it is too hot b) it isn't hot enough c) they do!
d) there is no hydrogen e) there is too much turbulence

66) What is meant by the so-called greenhouse effect?
a) the heating of the atmosphere by radiation from carbon dioxide
gas that has been excited by sunlight.
b) the heating of the atmosphere by UV light from the sun
c) the heating of the atmosphere by infrared radiation from the
planet that is absorbed by certain gasses in the atmosphere
d) the heating of the atmosphere by cosmic rays
e) the heating of the atmosphere by the burning of fossil fuels.

67) What is the approximate phase of the moon today?
a) between new and first quarter
b) between 1st quarter and full
c) between full and third quarter
d) between third quarter and new
e) none of the above

68) The correct sequence of the average distances of the planets in our solar
system, ordered from the outermost, inward to the sun is
a) Pluto, Saturn, Neptune, Uranus, Jupiter, Mars, Earth, Venus, Mercury
b) Mercury, Venus, Earth, Mars, Jupiter, Saturn, Uranus, Neptune, Pluto
c) Mars, Saturn, Venus, Mercury, Earth, Uranus, Neptune, Pluto, Jupiter
d) Pluto, Neptune, Uranus, Saturn, Jupiter, Mars, Earth, Venus, Mercury
e) none of the above

69) Hydrostatic equilibrium in a star is a balance between
a) hydrogen and helium
b) water, hydrogen, and oxygen
c) water and electric (static) charge
d) gravity and magnetic pressure
e) none of the above

70) A white dwarf is a star having roughly the mass of the sun
and the size of the
a) sun b) city of San Diego c) earth c) state of Kansas
e) none of the above

71) If the sun were suddenly to collapse into a black hole,
the gravitational force on earth would
a) double
b) become so strong that the earth would be "sucked" into the sun
c) decrease because black holes cause gravity at large distances
to disappear
d) remain the same
e) none of the above

72) The most likely object to be found at the center of a planetary nebula is
a) a white dwarf star d) a main sequence star
b) a neutron star or black hole e) none of the above
c) a red giant star

73) A match-box sized chunk of neutron star stuff on Earth would
a) weigh approximately as much as the sun
b) weigh as much as a cube of sugar
c) be as dense as water
d) weigh approximately as much as Mount Everest
e) none of the above

74) Population II stars differ from population I stars in that
a) they are larger on the average
b) they lack heavy elements in their chemical composition
c) they are not found in globular star clusters
d) they are not found in spiral galaxies
e) they do not show absorption lines in their spectra

75) Dust which exists in interstellar space causes
a) light from distant stars to appear "bluer" than normal
b) a shift in the spectrum lines of nebulae associated with the dust
c) strong radio signals at a wavelength of 21 cm to be emitted
d) stellar formation to cease and terminate within the region
e) none of the above

76) On April 9, 1986 the phase of the moon was first quarter.
When did the moon rise on that day?
a) sunset b) midnight c) noon d) sunrise
e) the Moon was down all day

77) Suppose you are at the beach and observe a crescent moon about to set
on the western horizon. It is dark and you cannot read your watch.
The time is
a) between sunset and midnight b) just before dawn
c) exactly midnight d) between midnight and sunrise
e) impossible to estimate without more information.

78) Two reflecting telescopes have primary mirrors 1 m and 2 m in diameter.
How many times more light is gathered by the larger telescope in any
given interval of time?
a) 4 times as much b) twice as much c) the same d) 16 times as much
e) none of the above

79) Sirius, the apparently brightest star in the sky is 8 light-years away.
How bright would it appear if it were twice as close, that is at a distance
of 4 light-years?
a) one-half as bright d) eight times brighter
b) two times as bright e) sixteen times brighter
c) four times brighter

80) The distance from the Earth to Proxima Centauri, the closest star, is about
1 parsec. The distance from Venus to Proxima Centauri is about:
a) 1 Astronomical Unit c) 2 parsecs e) 2 seconds of arc
b) one-half parsec d) 1 parsec

81) The Doppler effect
a) can tell you the velocity of a body but not whether it is approaching
or receding
b) can tell you whether a body is approaching or receding but not its
velocity
c) works for optical radiation but not for radio waves
d) could turn out to be wrong in the unlikely event that General Relativity
is incorrect
e) works for electromagnetic radiation of any wavelength

82) Who was the astronomer who developed three laws of planetary motion
which modified the Copernican ideas?
a) Galileo b) Ptolemy c) Brahe d) Kepler e) none of the above

83) A good candidate for a black hole binary is one where:
a) the visible star is a white dwarf, and the invisible star is
of about equal mass
b) neither star is visible
c) the unseen star seems to be quite massive
d) the visible star has been mostly eaten away
e) both stars are visible and very massive

84) The usual way we determine the orbital inclination of a spectroscopic
binary star system is from:
a) eclipses b) resolving the two stars directly with a telescope
c) radial velocity variations d) a determination of the temperatures
e) none of the above

85) Prior to the 1960's, one Cosmology that was debated as a possibility was
the Steady State model. It proposed that the Universe was infinite,
ageless, and unchanging on the large scale. In order to satisfy the
observed expansion of the Universe, the Steady State model required:
a) the continuous creation of matter from nothing, which in turn drove the
expansion and replenished the extragalactic environment
b) rotation of the Universe as well as expansion
c) blue-shifted spectra of galaxies at very large distances
d) slowing of the speed of light with distance to mimic extragalactic
redshifts
e) antigravity as well as antimatter

86) One possible energy source for quasars may be:
a) exploding stars b) electron collisions
c) rapid quasar rotation d) supermassive black holes
e) none of the above

87) The gravitational force is different from the electromagnetic force in that
a) the gravitational force is always attractive, but the electromagnetic
force can be either attractive or repulsive.
b) the gravitational force only depends on the mass of the attracting body.
c) the gravitational force gets weaker as the electrical force gets stronger.
d) gravity can be neutralized or "shielded"
e) gravity is much stronger in the domain of atoms and molecules.
e) all of the above

88) Is it possible to see a full moon at noon?
a) yes b) no c) in winter d) in summer e) if its a "blue" moon

89) During our summer (June - Sept), the night time hours in San Francisco
(approx 400 miles north of here) are
a) shorter than in San Diego
b) longer than in San Diego
c) the same as in San Diego
d) sometimes shorter, sometimes longer, depending on the sunspot cycle
e) sometimes shorter, sometimes longer, depending on the lunar phase.

90) Which of the following ideas is the most difficult to understand or explain
if quasars are assumed to be very distant?
a) the processes which result in a large amount of energy being generated
in a small region of space
b) the large red shift in the emission lines
c) the lifetimes of the quasars
d) the chemical composition of the quasars
e) none of the above

91) Which fundamental property of a star is most important in determining
its further evolution
a) chemical composition b) apparent brightness
c) mass d) age e) none of the above

92) X-rays and radio waves are similar in that
a) their wavelengths are comparable
b) they are both electromagnetic radiation, but X-rays travel faster
c) their photons have similar energies
d) all of the above
e) none of the above

93) What is the difference between a comet and a meteor?
a) a comet is made primarily of rocky material, while a meteor
is made of ice and gasses
b) a meteor is a chunk of rock, while a comet is made of ice
c) a meteor is a short-lived streak of light in the sky, while
a comet is a a ball of ices in orbit about the sun
d) they are two names for the same thing
e) none of the above

94) The question posed by Olber's paradox is
a) Why does the moon follow me when I drive in my car at night?
b) Why is it dark at night?
c) Why is the milky way pancake-shaped?
d) Why is the more massive star the least evolved?
e) none of the above

95) The cosmic microwave background radiation was discoved by
a) The space telescope
b) AT&T scientists testing a satellite communications system
c) radio engineers shortly after World War II
d) Cal Tech astronomers back in the 50's
e) none of the above

96) A star-like object, blue in color, displaying spectral emission lines
that are redshifted to an unusually large degree are known as
a) pulsars b) quasars c) black holes d) laptars e) none of the above

97) What are the most important factors in determining whether a given planet
will reatin a significant atmosphere?
a) its magnetic field
b) its mass
c) its color
d) its distance from the sun
e) both b) and d)

98) Why isn't the earth covered with craters like the moon?
a) The moon is older
b) The moon is younger
c) erosion erased them
d) There are more volcanos on the moon
e) both c) and d)

99) What causes a comet to have a tail?
a) The gravitational pull of the sun
b) The gravitational pull of the earth
c) the interaction of the comet with the solar wind
d) both a) and b)
e) none of the above

100) Astronomy is
a) the study of the effect constellations have on the motion of the Earth.
b) an essentially hypothetical science.
c) a primarily observational physical science.
d) a primarily experimental life science.
e) a scientific term for the study of astrology.
b) in the Galactic halo
c) in the Galactic center
d) in the intergalactic medium, far from galaxies
e) all of the above

43) Globular clusters are composed primarily of
a) old stars b) young stars c) hot stars d) massive stars e) blue stars

44) Young stars are located primarily in
a) the spiral arms b) the halo c) the globular clusters
d) the nucleus e) all of the above

45) Which type of stars contain the lowest percentage of heavy elements?
a) young stars b) old stars c) massive stars d) hot stars e) binary stars

46) Approximately what fraction of stars are members of binary star systems?
a) very few, a few percent b) about half, 50 percent
c) almost all stars are binaries d) we have no idea
e) there are only 5 known binary star systems

47) Binary stars are our primary source of information about
a) stellar colors b) stellar mass c) stellar chemical composition
d) stellar temperatures e) both a) and d)

48) In 1920 a famous debate was held between Shapley and Curtis.
The debate concerned
a) the nature of spiral nebulae
b) the nature of the redshifts observed in distant galaxies
c) the nature of binary stars
d) the nature of planetary nebulae
e) the nature of supernovae

49) Using Cepheid variable stars as distance indicators,
Hubble discovered that the Andromeda nebula was
a) a separate galaxy outside the Milky Way
b) a relatively nearby planetary nebulae
c) a distant elliptical galaxy
d) a reflection nebula
e) a new constellation

50) The Milky Way can be best described as
a) an elliptical galaxy b) an irregular galaxy c) a supernova remnant
d) a planetary nebula e) none of the above

51) Which type of galaxy contains the largest percentage of gas and dust?
a) elliptical b) spiral c) irregular d) interacting e) barred spiral

52) Which type of galaxy contains a significant population of young stars?
a) elliptical b) spiral c) irregular d) all of the above do
e) both b) and c), but not a)

53) Roughly how many stars are there in the Milky Way?
a) thousands, but less than a million
b) millions, but less than a billion
c) more than a billion
d) only a few hundred known
e) The Milky Way is not a galaxy, it is a planetary nebula

54) Roughly how many galaxies are there in the visible universe?
a) thousands, but less than a million
b) millions, but less than a billion
c) more than a billion
d) only a few hundred known
e) there are precisely 12765 known galaxies

55) Galaxies tend to
a) be found in clusters
b) be moving away from the Milky Way
c) be isolated from one another
d) be found orbiting our galaxy
e) both a) and b)

56) In the late 1920's Edwin Hubble made a profound discovery.
a) He found that the Milky Way was rotating.
b) He found that the galaxies were made of gas and dust.
c) He found that quasars were very distant.
d) He discovered the Cosmic Microwave Background radiation.
e) He found that the Universe was expanding.

57) The Hubble Space Telescope is particularly useful because
a) It is above the Earth's absorbing and distorting atmosphere.
b) It is the world's biggest telescope
c) It is perfectly stationary, allowing precise measurements
d) all of the above
e) none of the above

58) What is the difference between mass and weight?
a) mass is a measure of the total volume of an object
b) weight is mass squared
c) mass and weight are equivalent
d) weight is a force that depends on the strength of the local gravity
e) none of the above

59) Why do we think that when we observe quasars we are seeing the Universe
as it was billions of years ago?
a) Because quasars are very old
b) Because quasars are so distant that it has taken the light billions
of years to reach us.
c) Because quasars form and die very quickly
d) Because quasars are moving toward us so rapidly that the light travels
much slower than normal
e) none of the above

60) What is the visible "surface" of the sun called?
a) the photosphere b) the corona c) the envelope d) the magnetosphere
e) the stratosphere

61) What is the source of energy for stars?
a) chemical burning b) nuclear fission c) nuclear fusion
d) gravitational contraction e) all of the above are sources

62) Where was the carbon in your body manufactured?
a) the Big Bang b) a planetary nebula c) in the core of the sun
d) in Detroit MI e) none of the above

63) What elements were thought to be produced in the Big Bang?
a) Hydrogen only
b) Hydrogen, Helium, and very little else
c) All elements we see today, but in smaller concentrations
d) Helium only
e) no elements were created in the Big Bang, all were made in stars

64) What discoveries pretty much ruled out the steady state model of the
Universe?
a) the discovery of the spiral nebulae
b) the discovery that the Universe was expanding
c) the discovery of quasars and the cosmic microwave radiation.
d) the discovery that our galaxy was an elliptical galaxy
e) it is not ruled out, but is the accepted model of the universe.

65) Why don't nuclear reactions occur near the Sun's surface?
a) it is too hot b) it isn't hot enough c) they do!
d) there is no hydrogen e) there is too much turbulence

66) What is meant by the so-called greenhouse effect?
a) the heating of the atmosphere by radiation from carbon dioxide
gas that has been excited by sunlight.
b) the heating of the atmosphere by UV light from the sun
c) the heating of the atmosphere by infrared radiation from the
planet that is absorbed by certain gasses in the atmosphere
d) the heating of the atmosphere by cosmic rays
e) the heating of the atmosphere by the burning of fossil fuels.

67) What is the approximate phase of the moon today?
a) between new and first quarter
b) between 1st quarter and full
c) between full and third quarter
d) between third quarter and new
e) none of the above

68) The correct sequence of the average distances of the planets in our solar
system, ordered from the outermost, inward to the sun is
a) Pluto, Saturn, Neptune, Uranus, Jupiter, Mars, Earth, Venus, Mercury
b) Mercury, Venus, Earth, Mars, Jupiter, Saturn, Uranus, Neptune, Pluto
c) Mars, Saturn, Venus, Mercury, Earth, Uranus, Neptune, Pluto, Jupiter
d) Pluto, Neptune, Uranus, Saturn, Jupiter, Mars, Earth, Venus, Mercury
e) none of the above

69) Hydrostatic equilibrium in a star is a balance between
a) hydrogen and helium
b) water, hydrogen, and oxygen
c) water and electric (static) charge
d) gravity and magnetic pressure
e) none of the above

70) A white dwarf is a star having roughly the mass of the sun
and the size of the
a) sun b) city of San Diego c) earth c) state of Kansas
e) none of the above

71) If the sun were suddenly to collapse into a black hole,
the gravitational force on earth would
a) double
b) become so strong that the earth would be "sucked" into the sun
c) decrease because black holes cause gravity at large distances
to disappear
d) remain the same
e) none of the above

72) The most likely object to be found at the center of a planetary nebula is
a) a white dwarf star d) a main sequence star
b) a neutron star or black hole e) none of the above
c) a red giant star

73) A match-box sized chunk of neutron star stuff on Earth would
a) weigh approximately as much as the sun
b) weigh as much as a cube of sugar
c) be as dense as water
d) weigh approximately as much as Mount Everest
e) none of the above

74) Population II stars differ from population I stars in that
a) they are larger on the average
b) they lack heavy elements in their chemical composition
c) they are not found in globular star clusters
d) they are not found in spiral galaxies
e) they do not show absorption lines in their spectra

75) Dust which exists in interstellar space causes
a) light from distant stars to appear "bluer" than normal
b) a shift in the spectrum lines of nebulae associated with the dust
c) strong radio signals at a wavelength of 21 cm to be emitted
d) stellar formation to cease and terminate within the region
e) none of the above

76) On April 9, 1986 the phase of the moon was first quarter.
When did the moon rise on that day?
a) sunset b) midnight c) noon d) sunrise
e) the Moon was down all day

77) Suppose you are at the beach and observe a crescent moon about to set
on the western horizon. It is dark and you cannot read your watch.
The time is
a) between sunset and midnight b) just before dawn
c) exactly midnight d) between midnight and sunrise
e) impossible to estimate without more information.

78) Two reflecting telescopes have primary mirrors 1 m and 2 m in diameter.
How many times more light is gathered by the larger telescope in any
given interval of time?
a) 4 times as much b) twice as much c) the same d) 16 times as much
e) none of the above

79) Sirius, the apparently brightest star in the sky is 8 light-years away.
How bright would it appear if it were twice as close, that is at a distance
of 4 light-years?
a) one-half as bright d) eight times brighter
b) two times as bright e) sixteen times brighter
c) four times brighter

80) The distance from the Earth to Proxima Centauri, the closest star, is about
1 parsec. The distance from Venus to Proxima Centauri is about:
a) 1 Astronomical Unit c) 2 parsecs e) 2 seconds of arc
b) one-half parsec d) 1 parsec

81) The Doppler effect
a) can tell you the velocity of a body but not whether it is approaching
or receding
b) can tell you whether a body is approaching or receding but not its
velocity
c) works for optical radiation but not for radio waves
d) could turn out to be wrong in the unlikely event that General Relativity
is incorrect
e) works for electromagnetic radiation of any wavelength

82) Who was the astronomer who developed three laws of planetary motion
which modified the Copernican ideas?
a) Galileo b) Ptolemy c) Brahe d) Kepler e) none of the above

83) A good candidate for a black hole binary is one where:
a) the visible star is a white dwarf, and the invisible star is
of about equal mass
b) neither star is visible
c) the unseen star seems to be quite massive
d) the visible star has been mostly eaten away
e) both stars are visible and very massive

84) The usual way we determine the orbital inclination of a spectroscopic
binary star system is from:
a) eclipses b) resolving the two stars directly with a telescope
c) radial velocity variations d) a determination of the temperatures
e) none of the above

85) Prior to the 1960's, one Cosmology that was debated as a possibility was
the Steady State model. It proposed that the Universe was infinite,
ageless, and unchanging on the large scale. In order to satisfy the
observed expansion of the Universe, the Steady State model required:
a) the continuous creation of matter from nothing, which in turn drove the
expansion and replenished the extragalactic environment
b) rotation of the Universe as well as expansion
c) blue-shifted spectra of galaxies at very large distances
d) slowing of the speed of light with distance to mimic extragalactic
redshifts
e) antigravity as well as antimatter

86) One possible energy source for quasars may be:
a) exploding stars b) electron collisions
c) rapid quasar rotation d) supermassive black holes
e) none of the above

87) The gravitational force is different from the electromagnetic force in that
a) the gravitational force is always attractive, but the electromagnetic
force can be either attractive or repulsive.
b) the gravitational force only depends on the mass of the attracting body.
c) the gravitational force gets weaker as the electrical force gets stronger.
d) gravity can be neutralized or "shielded"
e) gravity is much stronger in the domain of atoms and molecules.
e) all of the above

88) Is it possible to see a full moon at noon?
a) yes b) no c) in winter d) in summer e) if its a "blue" moon

89) During our summer (June - Sept), the night time hours in San Francisco
(approx 400 miles north of here) are
a) shorter than in San Diego
b) longer than in San Diego
c) the same as in San Diego
d) sometimes shorter, sometimes longer, depending on the sunspot cycle
e) sometimes shorter, sometimes longer, depending on the lunar phase.

90) Which of the following ideas is the most difficult to understand or explain
if quasars are assumed to be very distant?
a) the processes which result in a large amount of energy being generated
in a small region of space
b) the large red shift in the emission lines
c) the lifetimes of the quasars
d) the chemical composition of the quasars
e) none of the above

91) Which fundamental property of a star is most important in determining
its further evolution
a) chemical composition b) apparent brightness
c) mass d) age e) none of the above

92) X-rays and radio waves are similar in that
a) their wavelengths are comparable
b) they are both electromagnetic radiation, but X-rays travel faster
c) their photons have similar energies
d) all of the above
e) none of the above

93) What is the difference between a comet and a meteor?
a) a comet is made primarily of rocky material, while a meteor
is made of ice and gasses
b) a meteor is a chunk of rock, while a comet is made of ice
c) a meteor is a short-lived streak of light in the sky, while
a comet is a a ball of ices in orbit about the sun
d) they are two names for the same thing
e) none of the above

94) The question posed by Olber's paradox is
a) Why does the moon follow me when I drive in my car at night?
b) Why is it dark at night?
c) Why is the milky way pancake-shaped?
d) Why is the more massive star the least evolved?
e) none of the above

95) The cosmic microwave background radiation was discoved by
a) The space telescope
b) AT&T scientists testing a satellite communications system
c) radio engineers shortly after World War II
d) Cal Tech astronomers back in the 50's
e) none of the above

96) A star-like object, blue in color, displaying spectral emission lines
that are redshifted to an unusually large degree are known as
a) pulsars b) quasars c) black holes d) laptars e) none of the above

97) What are the most important factors in determining whether a given planet
will reatin a significant atmosphere?
a) its magnetic field
b) its mass
c) its color
d) its distance from the sun
e) both b) and d)

98) Why isn't the earth covered with craters like the moon?
a) The moon is older
b) The moon is younger
c) erosion erased them
d) There are more volcanos on the moon
e) both c) and d)

99) What causes a comet to have a tail?
a) The gravitational pull of the sun
b) The gravitational pull of the earth
c) the interaction of the comet with the solar wind
d) both a) and b)
e) none of the above

100) Astronomy is
a) the study of the effect constellations have on the motion of the Earth.
b) an essentially hypothetical science.
c) a primarily observational physical science.
d) a primarily experimental life science.
e) a scientific term for the study of astrology.
b) in the Galactic halo
c) in the Galactic center
d) in the intergalactic medium, far from galaxies
e) all of the above

43) Globular clusters are composed primarily of
a) old stars b) young stars c) hot stars d) massive stars e) blue stars

44) Young stars are located primarily in
a) the spiral arms b) the halo c) the globular clusters
d) the nucleus e) all of the above

45) Which type of stars contain the lowest percentage of heavy elements?
a) young stars b) old stars c) massive stars d) hot stars e) binary stars

46) Approximately what fraction of stars are members of binary star systems?
a) very few, a few percent b) about half, 50 percent
c) almost all stars are binaries d) we have no idea
e) there are only 5 known binary star systems

47) Binary stars are our primary source of information about
a) stellar colors b) stellar mass c) stellar chemical composition
d) stellar temperatures e) both a) and d)

48) In 1920 a famous debate was held between Shapley and Curtis.
The debate concerned
a) the nature of spiral nebulae
b) the nature of the redshifts observed in distant galaxies
c) the nature of binary stars
d) the nature of planetary nebulae
e) the nature of supernovae

49) Using Cepheid variable stars as distance indicators,
Hubble discovered that the Andromeda nebula was
a) a separate galaxy outside the Milky Way
b) a relatively nearby planetary nebulae
c) a distant elliptical galaxy
d) a reflection nebula
e) a new constellation

50) The Milky Way can be best described as
a) an elliptical galaxy b) an irregular galaxy c) a supernova remnant
d) a planetary nebula e) none of the above

51) Which type of galaxy contains the largest percentage of gas and dust?
a) elliptical b) spiral c) irregular d) interacting e) barred spiral

52) Which type of galaxy contains a significant population of young stars?
a) elliptical b) spiral c) irregular d) all of the above do
e) both b) and c), but not a)

53) Roughly how many stars are there in the Milky Way?
a) thousands, but less than a million
b) millions, but less than a billion
c) more than a billion
d) only a few hundred known
e) The Milky Way is not a galaxy, it is a planetary nebula

54) Roughly how many galaxies are there in the visible universe?
a) thousands, but less than a million
b) millions, but less than a billion
c) more than a billion
d) only a few hundred known
e) there are precisely 12765 known galaxies

55) Galaxies tend to
a) be found in clusters
b) be moving away from the Milky Way
c) be isolated from one another
d) be found orbiting our galaxy
e) both a) and b)

56) In the late 1920's Edwin Hubble made a profound discovery.
a) He found that the Milky Way was rotating.
b) He found that the galaxies were made of gas and dust.
c) He found that quasars were very distant.
d) He discovered the Cosmic Microwave Background radiation.
e) He found that the Universe was expanding.

57) The Hubble Space Telescope is particularly useful because
a) It is above the Earth's absorbing and distorting atmosphere.
b) It is the world's biggest telescope
c) It is perfectly stationary, allowing precise measurements
d) all of the above
e) none of the above

58) What is the difference between mass and weight?
a) mass is a measure of the total volume of an object
b) weight is mass squared
c) mass and weight are equivalent
d) weight is a force that depends on the strength of the local gravity
e) none of the above

59) Why do we think that when we observe quasars we are seeing the Universe
as it was billions of years ago?
a) Because quasars are very old
b) Because quasars are so distant that it has taken the light billions
of years to reach us.
c) Because quasars form and die very quickly
d) Because quasars are moving toward us so rapidly that the light travels
much slower than normal
e) none of the above

60) What is the visible "surface" of the sun called?
a) the photosphere b) the corona c) the envelope d) the magnetosphere
e) the stratosphere

61) What is the source of energy for stars?
a) chemical burning b) nuclear fission c) nuclear fusion
d) gravitational contraction e) all of the above are sources

62) Where was the carbon in your body manufactured?
a) the Big Bang b) a planetary nebula c) in the core of the sun
d) in Detroit MI e) none of the above

63) What elements were thought to be produced in the Big Bang?
a) Hydrogen only
b) Hydrogen, Helium, and very little else
c) All elements we see today, but in smaller concentrations
d) Helium only
e) no elements were created in the Big Bang, all were made in stars

64) What discoveries pretty much ruled out the steady state model of the
Universe?
a) the discovery of the spiral nebulae
b) the discovery that the Universe was expanding
c) the discovery of quasars and the cosmic microwave radiation.
d) the discovery that our galaxy was an elliptical galaxy
e) it is not ruled out, but is the accepted model of the universe.

65) Why don't nuclear reactions occur near the Sun's surface?
a) it is too hot b) it isn't hot enough c) they do!
d) there is no hydrogen e) there is too much turbulence

66) What is meant by the so-called greenhouse effect?
a) the heating of the atmosphere by radiation from carbon dioxide
gas that has been excited by sunlight.
b) the heating of the atmosphere by UV light from the sun
c) the heating of the atmosphere by infrared radiation from the
planet that is absorbed by certain gasses in the atmosphere
d) the heating of the atmosphere by cosmic rays
e) the heating of the atmosphere by the burning of fossil fuels.

67) What is the approximate phase of the moon today?
a) between new and first quarter
b) between 1st quarter and full
c) between full and third quarter
d) between third quarter and new
e) none of the above

68) The correct sequence of the average distances of the planets in our solar
system, ordered from the outermost, inward to the sun is
a) Pluto, Saturn, Neptune, Uranus, Jupiter, Mars, Earth, Venus, Mercury
b) Mercury, Venus, Earth, Mars, Jupiter, Saturn, Uranus, Neptune, Pluto
c) Mars, Saturn, Venus, Mercury, Earth, Uranus, Neptune, Pluto, Jupiter
d) Pluto, Neptune, Uranus, Saturn, Jupiter, Mars, Earth, Venus, Mercury
e) none of the above

69) Hydrostatic equilibrium in a star is a balance between
a) hydrogen and helium
b) water, hydrogen, and oxygen
c) water and electric (static) charge
d) gravity and magnetic pressure
e) none of the above

70) A white dwarf is a star having roughly the mass of the sun
and the size of the
a) sun b) city of San Diego c) earth c) state of Kansas
e) none of the above

71) If the sun were suddenly to collapse into a black hole,
the gravitational force on earth would
a) double
b) become so strong that the earth would be "sucked" into the sun
c) decrease because black holes cause gravity at large distances
to disappear
d) remain the same
e) none of the above

72) The most likely object to be found at the center of a planetary nebula is
a) a white dwarf star d) a main sequence star
b) a neutron star or black hole e) none of the above
c) a red giant star

73) A match-box sized chunk of neutron star stuff on Earth would
a) weigh approximately as much as the sun
b) weigh as much as a cube of sugar
c) be as dense as water
d) weigh approximately as much as Mount Everest
e) none of the above

74) Population II stars differ from population I stars in that
a) they are larger on the average
b) they lack heavy elements in their chemical composition
c) they are not found in globular star clusters
d) they are not found in spiral galaxies
e) they do not show absorption lines in their spectra

75) Dust which exists in interstellar space causes
a) light from distant stars to appear "bluer" than normal
b) a shift in the spectrum lines of nebulae associated with the dust
c) strong radio signals at a wavelength of 21 cm to be emitted
d) stellar formation to cease and terminate within the region
e) none of the above

76) On April 9, 1986 the phase of the moon was first quarter.
When did the moon rise on that day?
a) sunset b) midnight c) noon d) sunrise
e) the Moon was down all day

77) Suppose you are at the beach and observe a crescent moon about to set
on the western horizon. It is dark and you cannot read your watch.
The time is
a) between sunset and midnight b) just before dawn
c) exactly midnight d) between midnight and sunrise
e) impossible to estimate without more information.

78) Two reflecting telescopes have primary mirrors 1 m and 2 m in diameter.
How many times more light is gathered by the larger telescope in any
given interval of time?
a) 4 times as much b) twice as much c) the same d) 16 times as much
e) none of the above

79) Sirius, the apparently brightest star in the sky is 8 light-years away.
How bright would it appear if it were twice as close, that is at a distance
of 4 light-years?
a) one-half as bright d) eight times brighter
b) two times as bright e) sixteen times brighter
c) four times brighter

80) The distance from the Earth to Proxima Centauri, the closest star, is about
1 parsec. The distance from Venus to Proxima Centauri is about:
a) 1 Astronomical Unit c) 2 parsecs e) 2 seconds of arc
b) one-half parsec d) 1 parsec

81) The Doppler effect
a) can tell you the velocity of a body but not whether it is approaching
or receding
b) can tell you whether a body is approaching or receding but not its
velocity
c) works for optical radiation but not for radio waves
d) could turn out to be wrong in the unlikely event that General Relativity
is incorrect
e) works for electromagnetic radiation of any wavelength

82) Who was the astronomer who developed three laws of planetary motion
which modified the Copernican ideas?
a) Galileo b) Ptolemy c) Brahe d) Kepler e) none of the above

83) A good candidate for a black hole binary is one where:
a) the visible star is a white dwarf, and the invisible star is
of about equal mass
b) neither star is visible
c) the unseen star seems to be quite massive
d) the visible star has been mostly eaten away
e) both stars are visible and very massive

84) The usual way we determine the orbital inclination of a spectroscopic
binary star system is from:
a) eclipses b) resolving the two stars directly with a telescope
c) radial velocity variations d) a determination of the temperatures
e) none of the above

85) Prior to the 1960's, one Cosmology that was debated as a possibility was
the Steady State model. It proposed that the Universe was infinite,
ageless, and unchanging on the large scale. In order to satisfy the
observed expansion of the Universe, the Steady State model required:
a) the continuous creation of matter from nothing, which in turn drove the
expansion and replenished the extragalactic environment
b) rotation of the Universe as well as expansion
c) blue-shifted spectra of galaxies at very large distances
d) slowing of the speed of light with distance to mimic extragalactic
redshifts
e) antigravity as well as antimatter

86) One possible energy source for quasars may be:
a) exploding stars b) electron collisions
c) rapid quasar rotation d) supermassive black holes
e) none of the above

87) The gravitational force is different from the electromagnetic force in that
a) the gravitational force is always attractive, but the electromagnetic
force can be either attractive or repulsive.
b) the gravitational force only depends on the mass of the attracting body.
c) the gravitational force gets weaker as the electrical force gets stronger.
d) gravity can be neutralized or "shielded"
e) gravity is much stronger in the domain of atoms and molecules.
e) all of the above

88) Is it possible to see a full moon at noon?
a) yes b) no c) in winter d) in summer e) if its a "blue" moon

89) During our summer (June - Sept), the night time hours in San Francisco
(approx 400 miles north of here) are
a) shorter than in San Diego
b) longer than in San Diego
c) the same as in San Diego
d) sometimes shorter, sometimes longer, depending on the sunspot cycle
e) sometimes shorter, sometimes longer, depending on the lunar phase.

90) Which of the following ideas is the most difficult to understand or explain
if quasars are assumed to be very distant?
a) the processes which result in a large amount of energy being generated
in a small region of space
b) the large red shift in the emission lines
c) the lifetimes of the quasars
d) the chemical composition of the quasars
e) none of the above

91) Which fundamental property of a star is most important in determining
its further evolution
a) chemical composition b) apparent brightness
c) mass d) age e) none of the above

92) X-rays and radio waves are similar in that
a) their wavelengths are comparable
b) they are both electromagnetic radiation, but X-rays travel faster
c) their photons have similar energies
d) all of the above
e) none of the above

93) What is the difference between a comet and a meteor?
a) a comet is made primarily of rocky material, while a meteor
is made of ice and gasses
b) a meteor is a chunk of rock, while a comet is made of ice
c) a meteor is a short-lived streak of light in the sky, while
a comet is a a ball of ices in orbit about the sun
d) they are two names for the same thing
e) none of the above

94) The question posed by Olber's paradox is
a) Why does the moon follow me when I drive in my car at night?
b) Why is it dark at night?
c) Why is the milky way pancake-shaped?
d) Why is the more massive star the least evolved?
e) none of the above

95) The cosmic microwave background radiation was discoved by
a) The space telescope
b) AT&T scientists testing a satellite communications system
c) radio engineers shortly after World War II
d) Cal Tech astronomers back in the 50's
e) none of the above

96) A star-like object, blue in color, displaying spectral emission lines
that are redshifted to an unusually large degree are known as
a) pulsars b) quasars c) black holes d) laptars e) none of the above

97) What are the most important factors in determining whether a given planet
will reatin a significant atmosphere?
a) its magnetic field
b) its mass
c) its color
d) its distance from the sun
e) both b) and d)

98) Why isn't the earth covered with craters like the moon?
a) The moon is older
b) The moon is younger
c) erosion erased them
d) There are more volcanos on the moon
e) both c) and d)

99) What causes a comet to have a tail?
a) The gravitational pull of the sun
b) The gravitational pull of the earth
c) the interaction of the comet with the solar wind
d) both a) and b)
e) none of the above

100) Astronomy is
a) the study of the effect constellations have on the motion of the Earth.
b) an essentially hypothetical science.
c) a primarily observational physical science.
d) a primarily experimental life science.
e) a scientific term for the study of astrology.
b) in the Galactic halo
c) in the Galactic center
d) in the intergalactic medium, far from galaxies
e) all of the above

43) Globular clusters are composed primarily of
a) old stars b) young stars c) hot stars d) massive stars e) blue stars

44) Young stars are located primarily in
a) the spiral arms b) the halo c) the globular clusters
d) the nucleus e) all of the above

45) Which type of stars contain the lowest percentage of heavy elements?
a) young stars b) old stars c) massive stars d) hot stars e) binary stars

46) Approximately what fraction of stars are members of binary star systems?
a) very few, a few percent b) about half, 50 percent
c) almost all stars are binaries d) we have no idea
e) there are only 5 known binary star systems

47) Binary stars are our primary source of information about
a) stellar colors b) stellar mass c) stellar chemical composition
d) stellar temperatures e) both a) and d)

48) In 1920 a famous debate was held between Shapley and Curtis.
The debate concerned
a) the nature of spiral nebulae
b) the nature of the redshifts observed in distant galaxies
c) the nature of binary stars
d) the nature of planetary nebulae
e) the nature of supernovae

49) Using Cepheid variable stars as distance indicators,
Hubble discovered that the Andromeda nebula was
a) a separate galaxy outside the Milky Way
b) a relatively nearby planetary nebulae
c) a distant elliptical galaxy
d) a reflection nebula
e) a new constellation

50) The Milky Way can be best described as
a) an elliptical galaxy b) an irregular galaxy c) a supernova remnant
d) a planetary nebula e) none of the above

51) Which type of galaxy contains the largest percentage of gas and dust?
a) elliptical b) spiral c) irregular d) interacting e) barred spiral

52) Which type of galaxy contains a significant population of young stars?
a) elliptical b) spiral c) irregular d) all of the above do
e) both b) and c), but not a)

53) Roughly how many stars are there in the Milky Way?
a) thousands, but less than a million
b) millions, but less than a billion
c) more than a billion
d) only a few hundred known
e) The Milky Way is not a galaxy, it is a planetary nebula

54) Roughly how many galaxies are there in the visible universe?
a) thousands, but less than a million
b) millions, but less than a billion
c) more than a billion
d) only a few hundred known
e) there are precisely 12765 known galaxies

55) Galaxies tend to
a) be found in clusters
b) be moving away from the Milky Way
c) be isolated from one another
d) be found orbiting our galaxy
e) both a) and b)

56) In the late 1920's Edwin Hubble made a profound discovery.
a) He found that the Milky Way was rotating.
b) He found that the galaxies were made of gas and dust.
c) He found that quasars were very distant.
d) He discovered the Cosmic Microwave Background radiation.
e) He found that the Universe was expanding.

57) The Hubble Space Telescope is particularly useful because
a) It is above the Earth's absorbing and distorting atmosphere.
b) It is the world's biggest telescope
c) It is perfectly stationary, allowing precise measurements
d) all of the above
e) none of the above

58) What is the difference between mass and weight?
a) mass is a measure of the total volume of an object
b) weight is mass squared
c) mass and weight are equivalent
d) weight is a force that depends on the strength of the local gravity
e) none of the above

59) Why do we think that when we observe quasars we are seeing the Universe
as it was billions of years ago?
a) Because quasars are very old
b) Because quasars are so distant that it has taken the light billions
of years to reach us.
c) Because quasars form and die very quickly
d) Because quasars are moving toward us so rapidly that the light travels
much slower than normal
e) none of the above

60) What is the visible "surface" of the sun called?
a) the photosphere b) the corona c) the envelope d) the magnetosphere
e) the stratosphere

61) What is the source of energy for stars?
a) chemical burning b) nuclear fission c) nuclear fusion
d) gravitational contraction e) all of the above are sources

62) Where was the carbon in your body manufactured?
a) the Big Bang b) a planetary nebula c) in the core of the sun
d) in Detroit MI e) none of the above

63) What elements were thought to be produced in the Big Bang?
a) Hydrogen only
b) Hydrogen, Helium, and very little else
c) All elements we see today, but in smaller concentrations
d) Helium only
e) no elements were created in the Big Bang, all were made in stars

64) What discoveries pretty much ruled out the steady state model of the
Universe?
a) the discovery of the spiral nebulae
b) the discovery that the Universe was expanding
c) the discovery of quasars and the cosmic microwave radiation.
d) the discovery that our galaxy was an elliptical galaxy
e) it is not ruled out, but is the accepted model of the universe.

65) Why don't nuclear reactions occur near the Sun's surface?
a) it is too hot b) it isn't hot enough c) they do!
d) there is no hydrogen e) there is too much turbulence

66) What is meant by the so-called greenhouse effect?
a) the heating of the atmosphere by radiation from carbon dioxide
gas that has been excited by sunlight.
b) the heating of the atmosphere by UV light from the sun
c) the heating of the atmosphere by infrared radiation from the
planet that is absorbed by certain gasses in the atmosphere
d) the heating of the atmosphere by cosmic rays
e) the heating of the atmosphere by the burning of fossil fuels.

67) What is the approximate phase of the moon today?
a) between new and first quarter
b) between 1st quarter and full
c) between full and third quarter
d) between third quarter and new
e) none of the above

68) The correct sequence of the average distances of the planets in our solar
system, ordered from the outermost, inward to the sun is
a) Pluto, Saturn, Neptune, Uranus, Jupiter, Mars, Earth, Venus, Mercury
b) Mercury, Venus, Earth, Mars, Jupiter, Saturn, Uranus, Neptune, Pluto
c) Mars, Saturn, Venus, Mercury, Earth, Uranus, Neptune, Pluto, Jupiter
d) Pluto, Neptune, Uranus, Saturn, Jupiter, Mars, Earth, Venus, Mercury
e) none of the above

69) Hydrostatic equilibrium in a star is a balance between
a) hydrogen and helium
b) water, hydrogen, and oxygen
c) water and electric (static) charge
d) gravity and magnetic pressure
e) none of the above

70) A white dwarf is a star having roughly the mass of the sun
and the size of the
a) sun b) city of San Diego c) earth c) state of Kansas
e) none of the above

71) If the sun were suddenly to collapse into a black hole,
the gravitational force on earth would
a) double
b) become so strong that the earth would be "sucked" into the sun
c) decrease because black holes cause gravity at large distances
to disappear
d) remain the same
e) none of the above

72) The most likely object to be found at the center of a planetary nebula is
a) a white dwarf star d) a main sequence star
b) a neutron star or black hole e) none of the above
c) a red giant star

73) A match-box sized chunk of neutron star stuff on Earth would
a) weigh approximately as much as the sun
b) weigh as much as a cube of sugar
c) be as dense as water
d) weigh approximately as much as Mount Everest
e) none of the above

74) Population II stars differ from population I stars in that
a) they are larger on the average
b) they lack heavy elements in their chemical composition
c) they are not found in globular star clusters
d) they are not found in spiral galaxies
e) they do not show absorption lines in their spectra

75) Dust which exists in interstellar space causes
a) light from distant stars to appear "bluer" than normal
b) a shift in the spectrum lines of nebulae associated with the dust
c) strong radio signals at a wavelength of 21 cm to be emitted
d) stellar formation to cease and terminate within the region
e) none of the above

76) On April 9, 1986 the phase of the moon was first quarter.
When did the moon rise on that day?
a) sunset b) midnight c) noon d) sunrise
e) the Moon was down all day

77) Suppose you are at the beach and observe a crescent moon about to set
on the western horizon. It is dark and you cannot read your watch.
The time is
a) between sunset and midnight b) just before dawn
c) exactly midnight d) between midnight and sunrise
e) impossible to estimate without more information.

78) Two reflecting telescopes have primary mirrors 1 m and 2 m in diameter.
How many times more light is gathered by the larger telescope in any
given interval of time?
a) 4 times as much b) twice as much c) the same d) 16 times as much
e) none of the above

79) Sirius, the apparently brightest star in the sky is 8 light-years away.
How bright would it appear if it were twice as close, that is at a distance
of 4 light-years?
a) one-half as bright d) eight times brighter
b) two times as bright e) sixteen times brighter
c) four times brighter

80) The distance from the Earth to Proxima Centauri, the closest star, is about
1 parsec. The distance from Venus to Proxima Centauri is about:
a) 1 Astronomical Unit c) 2 parsecs e) 2 seconds of arc
b) one-half parsec d) 1 parsec

81) The Doppler effect
a) can tell you the velocity of a body but not whether it is approaching
or receding
b) can tell you whether a body is approaching or receding but not its
velocity
c) works for optical radiation but not for radio waves
d) could turn out to be wrong in the unlikely event that General Relativity
is incorrect
e) works for electromagnetic radiation of any wavelength

82) Who was the astronomer who developed three laws of planetary motion
which modified the Copernican ideas?
a) Galileo b) Ptolemy c) Brahe d) Kepler e) none of the above

83) A good candidate for a black hole binary is one where:
a) the visible star is a white dwarf, and the invisible star is
of about equal mass
b) neither star is visible
c) the unseen star seems to be quite massive
d) the visible star has been mostly eaten away
e) both stars are visible and very massive

84) The usual way we determine the orbital inclination of a spectroscopic
binary star system is from:
a) eclipses b) resolving the two stars directly with a telescope
c) radial velocity variations d) a determination of the temperatures
e) none of the above

85) Prior to the 1960's, one Cosmology that was debated as a possibility was
the Steady State model. It proposed that the Universe was infinite,
ageless, and unchanging on the large scale. In order to satisfy the
observed expansion of the Universe, the Steady State model required:
a) the continuous creation of matter from nothing, which in turn drove the
expansion and replenished the extragalactic environment
b) rotation of the Universe as well as expansion
c) blue-shifted spectra of galaxies at very large distances
d) slowing of the speed of light with distance to mimic extragalactic
redshifts
e) antigravity as well as antimatter

86) One possible energy source for quasars may be:
a) exploding stars b) electron collisions
c) rapid quasar rotation d) supermassive black holes
e) none of the above

87) The gravitational force is different from the electromagnetic force in that
a) the gravitational force is always attractive, but the electromagnetic
force can be either attractive or repulsive.
b) the gravitational force only depends on the mass of the attracting body.
c) the gravitational force gets weaker as the electrical force gets stronger.
d) gravity can be neutralized or "shielded"
e) gravity is much stronger in the domain of atoms and molecules.
e) all of the above

88) Is it possible to see a full moon at noon?
a) yes b) no c) in winter d) in summer e) if its a "blue" moon

89) During our summer (June - Sept), the night time hours in San Francisco
(approx 400 miles north of here) are
a) shorter than in San Diego
b) longer than in San Diego
c) the same as in San Diego
d) sometimes shorter, sometimes longer, depending on the sunspot cycle
e) sometimes shorter, sometimes longer, depending on the lunar phase.

90) Which of the following ideas is the most difficult to understand or explain
if quasars are assumed to be very distant?
a) the processes which result in a large amount of energy being generated
in a small region of space
b) the large red shift in the emission lines
c) the lifetimes of the quasars
d) the chemical composition of the quasars
e) none of the above

91) Which fundamental property of a star is most important in determining
its further evolution
a) chemical composition b) apparent brightness
c) mass d) age e) none of the above

92) X-rays and radio waves are similar in that
a) their wavelengths are comparable
b) they are both electromagnetic radiation, but X-rays travel faster
c) their photons have similar energies
d) all of the above
e) none of the above

93) What is the difference between a comet and a meteor?
a) a comet is made primarily of rocky material, while a meteor
is made of ice and gasses
b) a meteor is a chunk of rock, while a comet is made of ice
c) a meteor is a short-lived streak of light in the sky, while
a comet is a a ball of ices in orbit about the sun
d) they are two names for the same thing
e) none of the above

94) The question posed by Olber's paradox is
a) Why does the moon follow me when I drive in my car at night?
b) Why is it dark at night?
c) Why is the milky way pancake-shaped?
d) Why is the more massive star the least evolved?
e) none of the above

95) The cosmic microwave background radiation was discoved by
a) The space telescope
b) AT&T scientists testing a satellite communications system
c) radio engineers shortly after World War II
d) Cal Tech astronomers back in the 50's
e) none of the above

96) A star-like object, blue in color, displaying spectral emission lines
that are redshifted to an unusually large degree are known as
a) pulsars b) quasars c) black holes d) laptars e) none of the above

97) What are the most important factors in determining whether a given planet
will reatin a significant atmosphere?
a) its magnetic field
b) its mass
c) its color
d) its distance from the sun
e) both b) and d)

98) Why isn't the earth covered with craters like the moon?
a) The moon is older
b) The moon is younger
c) erosion erased them
d) There are more volcanos on the moon
e) both c) and d)

99) What causes a comet to have a tail?
a) The gravitational pull of the sun
b) The gravitational pull of the earth
c) the interaction of the comet with the solar wind
d) both a) and b)
e) none of the above

100) Astronomy is
a) the study of the effect constellations have on the motion of the Earth.
b) an essentially hypothetical science.
c) a primarily observational physical science.
d) a primarily experimental life science.
e) a scientific term for the study of astrology.
b) in the Galactic halo
c) in the Galactic center
d) in the intergalactic medium, far from galaxies
e) all of the above

43) Globular clusters are composed primarily of
a) old stars b) young stars c) hot stars d) massive stars e) blue stars

44) Young stars are located primarily in
a) the spiral arms b) the halo c) the globular clusters
d) the nucleus e) all of the above

45) Which type of stars contain the lowest percentage of heavy elements?
a) young stars b) old stars c) massive stars d) hot stars e) binary stars

46) Approximately what fraction of stars are members of binary star systems?
a) very few, a few percent b) about half, 50 percent
c) almost all stars are binaries d) we have no idea
e) there are only 5 known binary star systems

47) Binary stars are our primary source of information about
a) stellar colors b) stellar mass c) stellar chemical composition
d) stellar temperatures e) both a) and d)

48) In 1920 a famous debate was held between Shapley and Curtis.
The debate concerned
a) the nature of spiral nebulae
b) the nature of the redshifts observed in distant galaxies
c) the nature of binary stars
d) the nature of planetary nebulae
e) the nature of supernovae

49) Using Cepheid variable stars as distance indicators,
Hubble discovered that the Andromeda nebula was
a) a separate galaxy outside the Milky Way
b) a relatively nearby planetary nebulae
c) a distant elliptical galaxy
d) a reflection nebula
e) a new constellation

50) The Milky Way can be best described as
a) an elliptical galaxy b) an irregular galaxy c) a supernova remnant
d) a planetary nebula e) none of the above

51) Which type of galaxy contains the largest percentage of gas and dust?
a) elliptical b) spiral c) irregular d) interacting e) barred spiral

52) Which type of galaxy contains a significant population of young stars?
a) elliptical b) spiral c) irregular d) all of the above do
e) both b) and c), but not a)

53) Roughly how many stars are there in the Milky Way?
a) thousands, but less than a million
b) millions, but less than a billion
c) more than a billion
d) only a few hundred known
e) The Milky Way is not a galaxy, it is a planetary nebula

54) Roughly how many galaxies are there in the visible universe?
a) thousands, but less than a million
b) millions, but less than a billion
c) more than a billion
d) only a few hundred known
e) there are precisely 12765 known galaxies

55) Galaxies tend to
a) be found in clusters
b) be moving away from the Milky Way
c) be isolated from one another
d) be found orbiting our galaxy
e) both a) and b)

56) In the late 1920's Edwin Hubble made a profound discovery.
a) He found that the Milky Way was rotating.
b) He found that the galaxies were made of gas and dust.
c) He found that quasars were very distant.
d) He discovered the Cosmic Microwave Background radiation.
e) He found that the Universe was expanding.

57) The Hubble Space Telescope is particularly useful because
a) It is above the Earth's absorbing and distorting atmosphere.
b) It is the world's biggest telescope
c) It is perfectly stationary, allowing precise measurements
d) all of the above
e) none of the above

58) What is the difference between mass and weight?
a) mass is a measure of the total volume of an object
b) weight is mass squared
c) mass and weight are equivalent
d) weight is a force that depends on the strength of the local gravity
e) none of the above

59) Why do we think that when we observe quasars we are seeing the Universe
as it was billions of years ago?
a) Because quasars are very old
b) Because quasars are so distant that it has taken the light billions
of years to reach us.
c) Because quasars form and die very quickly
d) Because quasars are moving toward us so rapidly that the light travels
much slower than normal
e) none of the above

60) What is the visible "surface" of the sun called?
a) the photosphere b) the corona c) the envelope d) the magnetosphere
e) the stratosphere

61) What is the source of energy for stars?
a) chemical burning b) nuclear fission c) nuclear fusion
d) gravitational contraction e) all of the above are sources

62) Where was the carbon in your body manufactured?
a) the Big Bang b) a planetary nebula c) in the core of the sun
d) in Detroit MI e) none of the above

63) What elements were thought to be produced in the Big Bang?
a) Hydrogen only
b) Hydrogen, Helium, and very little else
c) All elements we see today, but in smaller concentrations
d) Helium only
e) no elements were created in the Big Bang, all were made in stars

64) What discoveries pretty much ruled out the steady state model of the
Universe?
a) the discovery of the spiral nebulae
b) the discovery that the Universe was expanding
c) the discovery of quasars and the cosmic microwave radiation.
d) the discovery that our galaxy was an elliptical galaxy
e) it is not ruled out, but is the accepted model of the universe.

65) Why don't nuclear reactions occur near the Sun's surface?
a) it is too hot b) it isn't hot enough c) they do!
d) there is no hydrogen e) there is too much turbulence

66) What is meant by the so-called greenhouse effect?
a) the heating of the atmosphere by radiation from carbon dioxide
gas that has been excited by sunlight.
b) the heating of the atmosphere by UV light from the sun
c) the heating of the atmosphere by infrared radiation from the
planet that is absorbed by certain gasses in the atmosphere
d) the heating of the atmosphere by cosmic rays
e) the heating of the atmosphere by the burning of fossil fuels.

67) What is the approximate phase of the moon today?
a) between new and first quarter
b) between 1st quarter and full
c) between full and third quarter
d) between third quarter and new
e) none of the above

68) The correct sequence of the average distances of the planets in our solar
system, ordered from the outermost, inward to the sun is
a) Pluto, Saturn, Neptune, Uranus, Jupiter, Mars, Earth, Venus, Mercury
b) Mercury, Venus, Earth, Mars, Jupiter, Saturn, Uranus, Neptune, Pluto
c) Mars, Saturn, Venus, Mercury, Earth, Uranus, Neptune, Pluto, Jupiter
d) Pluto, Neptune, Uranus, Saturn, Jupiter, Mars, Earth, Venus, Mercury
e) none of the above

69) Hydrostatic equilibrium in a star is a balance between
a) hydrogen and helium
b) water, hydrogen, and oxygen
c) water and electric (static) charge
d) gravity and magnetic pressure
e) none of the above

70) A white dwarf is a star having roughly the mass of the sun
and the size of the
a) sun b) city of San Diego c) earth c) state of Kansas
e) none of the above

71) If the sun were suddenly to collapse into a black hole,
the gravitational force on earth would
a) double
b) become so strong that the earth would be "sucked" into the sun
c) decrease because black holes cause gravity at large distances
to disappear
d) remain the same
e) none of the above

72) The most likely object to be found at the center of a planetary nebula is
a) a white dwarf star d) a main sequence star
b) a neutron star or black hole e) none of the above
c) a red giant star

73) A match-box sized chunk of neutron star stuff on Earth would
a) weigh approximately as much as the sun
b) weigh as much as a cube of sugar
c) be as dense as water
d) weigh approximately as much as Mount Everest
e) none of the above

74) Population II stars differ from population I stars in that
a) they are larger on the average
b) they lack heavy elements in their chemical composition
c) they are not found in globular star clusters
d) they are not found in spiral galaxies
e) they do not show absorption lines in their spectra

75) Dust which exists in interstellar space causes
a) light from distant stars to appear "bluer" than normal
b) a shift in the spectrum lines of nebulae associated with the dust
c) strong radio signals at a wavelength of 21 cm to be emitted
d) stellar formation to cease and terminate within the region
e) none of the above

76) On April 9, 1986 the phase of the moon was first quarter.
When did the moon rise on that day?
a) sunset b) midnight c) noon d) sunrise
e) the Moon was down all day

77) Suppose you are at the beach and observe a crescent moon about to set
on the western horizon. It is dark and you cannot read your watch.
The time is
a) between sunset and midnight b) just before dawn
c) exactly midnight d) between midnight and sunrise
e) impossible to estimate without more information.

78) Two reflecting telescopes have primary mirrors 1 m and 2 m in diameter.
How many times more light is gathered by the larger telescope in any
given interval of time?
a) 4 times as much b) twice as much c) the same d) 16 times as much
e) none of the above

79) Sirius, the apparently brightest star in the sky is 8 light-years away.
How bright would it appear if it were twice as close, that is at a distance
of 4 light-years?
a) one-half as bright d) eight times brighter
b) two times as bright e) sixteen times brighter
c) four times brighter

80) The distance from the Earth to Proxima Centauri, the closest star, is about
1 parsec. The distance from Venus to Proxima Centauri is about:
a) 1 Astronomical Unit c) 2 parsecs e) 2 seconds of arc
b) one-half parsec d) 1 parsec

81) The Doppler effect
a) can tell you the velocity of a body but not whether it is approaching
or receding
b) can tell you whether a body is approaching or receding but not its
velocity
c) works for optical radiation but not for radio waves
d) could turn out to be wrong in the unlikely event that General Relativity
is incorrect
e) works for electromagnetic radiation of any wavelength

82) Who was the astronomer who developed three laws of planetary motion
which modified the Copernican ideas?
a) Galileo b) Ptolemy c) Brahe d) Kepler e) none of the above

83) A good candidate for a black hole binary is one where:
a) the visible star is a white dwarf, and the invisible star is
of about equal mass
b) neither star is visible
c) the unseen star seems to be quite massive
d) the visible star has been mostly eaten away
e) both stars are visible and very massive

84) The usual way we determine the orbital inclination of a spectroscopic
binary star system is from:
a) eclipses b) resolving the two stars directly with a telescope
c) radial velocity variations d) a determination of the temperatures
e) none of the above

85) Prior to the 1960's, one Cosmology that was debated as a possibility was
the Steady State model. It proposed that the Universe was infinite,
ageless, and unchanging on the large scale. In order to satisfy the
observed expansion of the Universe, the Steady State model required:
a) the continuous creation of matter from nothing, which in turn drove the
expansion and replenished the extragalactic environment
b) rotation of the Universe as well as expansion
c) blue-shifted spectra of galaxies at very large distances
d) slowing of the speed of light with distance to mimic extragalactic
redshifts
e) antigravity as well as antimatter

86) One possible energy source for quasars may be:
a) exploding stars b) electron collisions
c) rapid quasar rotation d) supermassive black holes
e) none of the above

87) The gravitational force is different from the electromagnetic force in that
a) the gravitational force is always attractive, but the electromagnetic
force can be either attractive or repulsive.
b) the gravitational force only depends on the mass of the attracting body.
c) the gravitational force gets weaker as the electrical force gets stronger.
d) gravity can be neutralized or "shielded"
e) gravity is much stronger in the domain of atoms and molecules.
e) all of the above

88) Is it possible to see a full moon at noon?
a) yes b) no c) in winter d) in summer e) if its a "blue" moon

89) During our summer (June - Sept), the night time hours in San Francisco
(approx 400 miles north of here) are
a) shorter than in San Diego
b) longer than in San Diego
c) the same as in San Diego
d) sometimes shorter, sometimes longer, depending on the sunspot cycle
e) sometimes shorter, sometimes longer, depending on the lunar phase.

90) Which of the following ideas is the most difficult to understand or explain
if quasars are assumed to be very distant?
a) the processes which result in a large amount of energy being generated
in a small region of space
b) the large red shift in the emission lines
c) the lifetimes of the quasars
d) the chemical composition of the quasars
e) none of the above

91) Which fundamental property of a star is most important in determining
its further evolution
a) chemical composition b) apparent brightness
c) mass d) age e) none of the above

92) X-rays and radio waves are similar in that
a) their wavelengths are comparable
b) they are both electromagnetic radiation, but X-rays travel faster
c) their photons have similar energies
d) all of the above
e) none of the above

93) What is the difference between a comet and a meteor?
a) a comet is made primarily of rocky material, while a meteor
is made of ice and gasses
b) a meteor is a chunk of rock, while a comet is made of ice
c) a meteor is a short-lived streak of light in the sky, while
a comet is a a ball of ices in orbit about the sun
d) they are two names for the same thing
e) none of the above

94) The question posed by Olber's paradox is
a) Why does the moon follow me when I drive in my car at night?
b) Why is it dark at night?
c) Why is the milky way pancake-shaped?
d) Why is the more massive star the least evolved?
e) none of the above

95) The cosmic microwave background radiation was discoved by
a) The space telescope
b) AT&T scientists testing a satellite communications system
c) radio engineers shortly after World War II
d) Cal Tech astronomers back in the 50's
e) none of the above

96) A star-like object, blue in color, displaying spectral emission lines
that are redshifted to an unusually large degree are known as
a) pulsars b) quasars c) black holes d) laptars e) none of the above

97) What are the most important factors in determining whether a given planet
will reatin a significant atmosphere?
a) its magnetic field
b) its mass
c) its color
d) its distance from the sun
e) both b) and d)

98) Why isn't the earth covered with craters like the moon?
a) The moon is older
b) The moon is younger
c) erosion erased them
d) There are more volcanos on the moon
e) both c) and d)

99) What causes a comet to have a tail?
a) The gravitational pull of the sun
b) The gravitational pull of the earth
c) the interaction of the comet with the solar wind
d) both a) and b)
e) none of the above

100) Astronomy is
a) the study of the effect constellations have on the motion of the Earth.
b) an essentially hypothetical science.
c) a primarily observational physical science.
d) a primarily experimental life science.
e) a scientific term for the study of astrology.
b) in the Galactic halo
c) in the Galactic center
d) in the intergalactic medium, far from galaxies
e) all of the above

43) Globular clusters are composed primarily of
a) old stars b) young stars c) hot stars d) massive stars e) blue stars

44) Young stars are located primarily in
a) the spiral arms b) the halo c) the globular clusters
d) the nucleus e) all of the above

45) Which type of stars contain the lowest percentage of heavy elements?
a) young stars b) old stars c) massive stars d) hot stars e) binary stars

46) Approximately what fraction of stars are members of binary star systems?
a) very few, a few percent b) about half, 50 percent
c) almost all stars are binaries d) we have no idea
e) there are only 5 known binary star systems

47) Binary stars are our primary source of information about
a) stellar colors b) stellar mass c) stellar chemical composition
d) stellar temperatures e) both a) and d)

48) In 1920 a famous debate was held between Shapley and Curtis.
The debate concerned
a) the nature of spiral nebulae
b) the nature of the redshifts observed in distant galaxies
c) the nature of binary stars
d) the nature of planetary nebulae
e) the nature of supernovae

49) Using Cepheid variable stars as distance indicators,
Hubble discovered that the Andromeda nebula was
a) a separate galaxy outside the Milky Way
b) a relatively nearby planetary nebulae
c) a distant elliptical galaxy
d) a reflection nebula
e) a new constellation

50) The Milky Way can be best described as
a) an elliptical galaxy b) an irregular galaxy c) a supernova remnant
d) a planetary nebula e) none of the above

51) Which type of galaxy contains the largest percentage of gas and dust?
a) elliptical b) spiral c) irregular d) interacting e) barred spiral

52) Which type of galaxy contains a significant population of young stars?
a) elliptical b) spiral c) irregular d) all of the above do
e) both b) and c), but not a)

53) Roughly how many stars are there in the Milky Way?
a) thousands, but less than a million
b) millions, but less than a billion
c) more than a billion
d) only a few hundred known
e) The Milky Way is not a galaxy, it is a planetary nebula

54) Roughly how many galaxies are there in the visible universe?
a) thousands, but less than a million
b) millions, but less than a billion
c) more than a billion
d) only a few hundred known
e) there are precisely 12765 known galaxies

55) Galaxies tend to
a) be found in clusters
b) be moving away from the Milky Way
c) be isolated from one another
d) be found orbiting our galaxy
e) both a) and b)

56) In the late 1920's Edwin Hubble made a profound discovery.
a) He found that the Milky Way was rotating.
b) He found that the galaxies were made of gas and dust.
c) He found that quasars were very distant.
d) He discovered the Cosmic Microwave Background radiation.
e) He found that the Universe was expanding.

57) The Hubble Space Telescope is particularly useful because
a) It is above the Earth's absorbing and distorting atmosphere.
b) It is the world's biggest telescope
c) It is perfectly stationary, allowing precise measurements
d) all of the above
e) none of the above

58) What is the difference between mass and weight?
a) mass is a measure of the total volume of an object
b) weight is mass squared
c) mass and weight are equivalent
d) weight is a force that depends on the strength of the local gravity
e) none of the above

59) Why do we think that when we observe quasars we are seeing the Universe
as it was billions of years ago?
a) Because quasars are very old
b) Because quasars are so distant that it has taken the light billions
of years to reach us.
c) Because quasars form and die very quickly
d) Because quasars are moving toward us so rapidly that the light travels
much slower than normal
e) none of the above

60) What is the visible "surface" of the sun called?
a) the photosphere b) the corona c) the envelope d) the magnetosphere
e) the stratosphere

61) What is the source of energy for stars?
a) chemical burning b) nuclear fission c) nuclear fusion
d) gravitational contraction e) all of the above are sources

62) Where was the carbon in your body manufactured?
a) the Big Bang b) a planetary nebula c) in the core of the sun
d) in Detroit MI e) none of the above

63) What elements were thought to be produced in the Big Bang?
a) Hydrogen only
b) Hydrogen, Helium, and very little else
c) All elements we see today, but in smaller concentrations
d) Helium only
e) no elements were created in the Big Bang, all were made in stars

64) What discoveries pretty much ruled out the steady state model of the
Universe?
a) the discovery of the spiral nebulae
b) the discovery that the Universe was expanding
c) the discovery of quasars and the cosmic microwave radiation.
d) the discovery that our galaxy was an elliptical galaxy
e) it is not ruled out, but is the accepted model of the universe.

65) Why don't nuclear reactions occur near the Sun's surface?
a) it is too hot b) it isn't hot enough c) they do!
d) there is no hydrogen e) there is too much turbulence

66) What is meant by the so-called greenhouse effect?
a) the heating of the atmosphere by radiation from carbon dioxide
gas that has been excited by sunlight.
b) the heating of the atmosphere by UV light from the sun
c) the heating of the atmosphere by infrared radiation from the
planet that is absorbed by certain gasses in the atmosphere
d) the heating of the atmosphere by cosmic rays
e) the heating of the atmosphere by the burning of fossil fuels.

67) What is the approximate phase of the moon today?
a) between new and first quarter
b) between 1st quarter and full
c) between full and third quarter
d) between third quarter and new
e) none of the above

68) The correct sequence of the average distances of the planets in our solar
system, ordered from the outermost, inward to the sun is
a) Pluto, Saturn, Neptune, Uranus, Jupiter, Mars, Earth, Venus, Mercury
b) Mercury, Venus, Earth, Mars, Jupiter, Saturn, Uranus, Neptune, Pluto
c) Mars, Saturn, Venus, Mercury, Earth, Uranus, Neptune, Pluto, Jupiter
d) Pluto, Neptune, Uranus, Saturn, Jupiter, Mars, Earth, Venus, Mercury
e) none of the above

69) Hydrostatic equilibrium in a star is a balance between
a) hydrogen and helium
b) water, hydrogen, and oxygen
c) water and electric (static) charge
d) gravity and magnetic pressure
e) none of the above

70) A white dwarf is a star having roughly the mass of the sun
and the size of the
a) sun b) city of San Diego c) earth c) state of Kansas
e) none of the above

71) If the sun were suddenly to collapse into a black hole,
the gravitational force on earth would
a) double
b) become so strong that the earth would be "sucked" into the sun
c) decrease because black holes cause gravity at large distances
to disappear
d) remain the same
e) none of the above

72) The most likely object to be found at the center of a planetary nebula is
a) a white dwarf star d) a main sequence star
b) a neutron star or black hole e) none of the above
c) a red giant star

73) A match-box sized chunk of neutron star stuff on Earth would
a) weigh approximately as much as the sun
b) weigh as much as a cube of sugar
c) be as dense as water
d) weigh approximately as much as Mount Everest
e) none of the above

74) Population II stars differ from population I stars in that
a) they are larger on the average
b) they lack heavy elements in their chemical composition
c) they are not found in globular star clusters
d) they are not found in spiral galaxies
e) they do not show absorption lines in their spectra

75) Dust which exists in interstellar space causes
a) light from distant stars to appear "bluer" than normal
b) a shift in the spectrum lines of nebulae associated with the dust
c) strong radio signals at a wavelength of 21 cm to be emitted
d) stellar formation to cease and terminate within the region
e) none of the above

76) On April 9, 1986 the phase of the moon was first quarter.
When did the moon rise on that day?
a) sunset b) midnight c) noon d) sunrise
e) the Moon was down all day

77) Suppose you are at the beach and observe a crescent moon about to set
on the western horizon. It is dark and you cannot read your watch.
The time is
a) between sunset and midnight b) just before dawn
c) exactly midnight d) between midnight and sunrise
e) impossible to estimate without more information.

78) Two reflecting telescopes have primary mirrors 1 m and 2 m in diameter.
How many times more light is gathered by the larger telescope in any
given interval of time?
a) 4 times as much b) twice as much c) the same d) 16 times as much
e) none of the above

79) Sirius, the apparently brightest star in the sky is 8 light-years away.
How bright would it appear if it were twice as close, that is at a distance
of 4 light-years?
a) one-half as bright d) eight times brighter
b) two times as bright e) sixteen times brighter
c) four times brighter

80) The distance from the Earth to Proxima Centauri, the closest star, is about
1 parsec. The distance from Venus to Proxima Centauri is about:
a) 1 Astronomical Unit c) 2 parsecs e) 2 seconds of arc
b) one-half parsec d) 1 parsec

81) The Doppler effect
a) can tell you the velocity of a body but not whether it is approaching
or receding
b) can tell you whether a body is approaching or receding but not its
velocity
c) works for optical radiation but not for radio waves
d) could turn out to be wrong in the unlikely event that General Relativity
is incorrect
e) works for electromagnetic radiation of any wavelength

82) Who was the astronomer who developed three laws of planetary motion
which modified the Copernican ideas?
a) Galileo b) Ptolemy c) Brahe d) Kepler e) none of the above

83) A good candidate for a black hole binary is one where:
a) the visible star is a white dwarf, and the invisible star is
of about equal mass
b) neither star is visible
c) the unseen star seems to be quite massive
d) the visible star has been mostly eaten away
e) both stars are visible and very massive

84) The usual way we determine the orbital inclination of a spectroscopic
binary star system is from:
a) eclipses b) resolving the two stars directly with a telescope
c) radial velocity variations d) a determination of the temperatures
e) none of the above

85) Prior to the 1960's, one Cosmology that was debated as a possibility was
the Steady State model. It proposed that the Universe was infinite,
ageless, and unchanging on the large scale. In order to satisfy the
observed expansion of the Universe, the Steady State model required:
a) the continuous creation of matter from nothing, which in turn drove the
expansion and replenished the extragalactic environment
b) rotation of the Universe as well as expansion
c) blue-shifted spectra of galaxies at very large distances
d) slowing of the speed of light with distance to mimic extragalactic
redshifts
e) antigravity as well as antimatter

86) One possible energy source for quasars may be:
a) exploding stars b) electron collisions
c) rapid quasar rotation d) supermassive black holes
e) none of the above

87) The gravitational force is different from the electromagnetic force in that
a) the gravitational force is always attractive, but the electromagnetic
force can be either attractive or repulsive.
b) the gravitational force only depends on the mass of the attracting body.
c) the gravitational force gets weaker as the electrical force gets stronger.
d) gravity can be neutralized or "shielded"
e) gravity is much stronger in the domain of atoms and molecules.
e) all of the above

88) Is it possible to see a full moon at noon?
a) yes b) no c) in winter d) in summer e) if its a "blue" moon

89) During our summer (June - Sept), the night time hours in San Francisco
(approx 400 miles north of here) are
a) shorter than in San Diego
b) longer than in San Diego
c) the same as in San Diego
d) sometimes shorter, sometimes longer, depending on the sunspot cycle
e) sometimes shorter, sometimes longer, depending on the lunar phase.

90) Which of the following ideas is the most difficult to understand or explain
if quasars are assumed to be very distant?
a) the processes which result in a large amount of energy being generated
in a small region of space
b) the large red shift in the emission lines
c) the lifetimes of the quasars
d) the chemical composition of the quasars
e) none of the above

91) Which fundamental property of a star is most important in determining
its further evolution
a) chemical composition b) apparent brightness
c) mass d) age e) none of the above

92) X-rays and radio waves are similar in that
a) their wavelengths are comparable
b) they are both electromagnetic radiation, but X-rays travel faster
c) their photons have similar energies
d) all of the above
e) none of the above

93) What is the difference between a comet and a meteor?
a) a comet is made primarily of rocky material, while a meteor
is made of ice and gasses
b) a meteor is a chunk of rock, while a comet is made of ice
c) a meteor is a short-lived streak of light in the sky, while
a comet is a a ball of ices in orbit about the sun
d) they are two names for the same thing
e) none of the above

94) The question posed by Olber's paradox is
a) Why does the moon follow me when I drive in my car at night?
b) Why is it dark at night?
c) Why is the milky way pancake-shaped?
d) Why is the more massive star the least evolved?
e) none of the above

95) The cosmic microwave background radiation was discoved by
a) The space telescope
b) AT&T scientists testing a satellite communications system
c) radio engineers shortly after World War II
d) Cal Tech astronomers back in the 50's
e) none of the above

96) A star-like object, blue in color, displaying spectral emission lines
that are redshifted to an unusually large degree are known as
a) pulsars b) quasars c) black holes d) laptars e) none of the above

97) What are the most important factors in determining whether a given planet
will reatin a significant atmosphere?
a) its magnetic field
b) its mass
c) its color
d) its distance from the sun
e) both b) and d)

98) Why isn't the earth covered with craters like the moon?
a) The moon is older
b) The moon is younger
c) erosion erased them
d) There are more volcanos on the moon
e) both c) and d)

99) What causes a comet to have a tail?
a) The gravitational pull of the sun
b) The gravitational pull of the earth
c) the interaction of the comet with the solar wind
d) both a) and b)
e) none of the above

100) Astronomy is
a) the study of the effect constellations have on the motion of the Earth.
b) an essentially hypothetical science.
c) a primarily observational physical science.
d) a primarily experimental life science.
e) a scientific term for the study of astrology.
