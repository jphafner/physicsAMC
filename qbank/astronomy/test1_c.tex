
%% University of Northern Iowa
%%  Morgans Astronomy Exams
%%--------------------------------------------------

%% this section contains 15 problems

\element{morgans}{
\begin{question}{test1C-Q01}
    The most readily observed motion of a celestial object is produced by:
    \begin{choices}
        \wrongchoice{the motion of the planets across the sky.}
      \correctchoice{the rotation of the Earth.}
        \wrongchoice{the revolution of the Earth.}
        \wrongchoice{the motion of the Sun around the galaxy.}
    \end{choices}
\end{question}
}

\element{morgans}{
\begin{question}{test1C-Q02}
    If Sirius, the brightest star in the sky,
        rises at 10:00 p.m. one night,
        the following night it will rise at:
    \begin{choices}
        \wrongchoice{9:30.}
      \correctchoice{9:56.}
        \wrongchoice{10:00.}
        \wrongchoice{10:04.}
        \wrongchoice{10:30.}
    \end{choices}
\end{question}
}

\element{morgans}{
\begin{question}{test1C-Q03}
    For the Incas in their city of Machu Picchu at a latitude of \ang{-13},
        at the winter solstice (December 21) the Sun will be at what altitude (height above the ground)?
    \begin{choices}
        \wrongchoice{13 degrees}
        \wrongchoice{23 1/2 degrees}
        \wrongchoice{24 1/2 degrees}
      \correctchoice{79 1/2 degrees}
        \wrongchoice{77 degrees}
    \end{choices}
\end{question}
}

\element{morgans}{
\begin{question}{test1C-Q04}
    If a solar eclipse occurred 2 weeks ago,
        what would be the phase of the Moon today?
    \begin{choices}
        \wrongchoice{first quarter}
      \correctchoice{full}
        \wrongchoice{third quarter}
        \wrongchoice{new}
        \wrongchoice{waxing crescent}
    \end{choices}
\end{question}
}

\element{morgans}{
\begin{question}{test1C-Q05}
    When Venus' angle from the Sun is as large as possible and it is seen in the morning sky,
        it is at the location known as:
    \begin{choices}
        \wrongchoice{Inferior conjunction.}
        \wrongchoice{Superior conjunction.}
      \correctchoice{Greatest western elongation.}
        \wrongchoice{Greatest eastern elongation.}
    \end{choices}
\end{question}
}

\element{morgans}{
\begin{question}{test1C-Q06}
    Which one of the following statements about the Copernican model is \emph{false}? 
    \begin{choices}
      \correctchoice{it was more accurate than the Ptolemaic system in predicting planetary motions}
        \wrongchoice{relative planetary distances could be deduced from it}
        \wrongchoice{relative planetary speeds could be determined from it}
        \wrongchoice{retrograde motion could be easily explained by it}
        \wrongchoice{none: all of the above statements are true}
    \end{choices}
\end{question}
}

\element{morgans}{
\begin{question}{test1C-Q07}
    From his analysis of the motion of Mars,
        Kepler was able to conclude that:
    \begin{choices}
        \wrongchoice{the planets move in circular orbits with the Sun at the center.}
        \wrongchoice{the planets move with hyperbolic orbits.}
        \wrongchoice{the planets move with uniform speed.}
      \correctchoice{the planets move with varying orbital speed.}
        \wrongchoice{Mars is always at the same distance from the Earth.}
    \end{choices}
\end{question}
}

\element{morgans}{
\begin{question}{test1C-Q08}
    If a comet orbits the Sun with a period of 8 years,
        what is its average distance from the Sun?
    \begin{choices}
        \wrongchoice{8 A. U.}
      \correctchoice{4 A. U.}
        \wrongchoice{2 A. U.}
        \wrongchoice{1 A. U.}
        \wrongchoice{10 A. U.}
    \end{choices}
\end{question}
}

\element{morgans}{
\begin{question}{test1C-Q09}
    Take three identical bricks; strap two of them together. 
    Which statement is true? 
    \begin{choices}
        \wrongchoice{the combined bricks, when dropped, will fall twice as fast as the single brick}
        \wrongchoice{the combined bricks, when dropped, will fall four times as fast as the single brick due to the inverse square law of gravity}
      \correctchoice{the gravitational force between the Earth and the combined bricks is twice as strong as the gravitational force between the Earth and the single brick}
        \wrongchoice{the gravitational force between the Earth and the combined bricks is the same as the gravitational force between the Earth and the single brick}
    \end{choices}
\end{question}
}

\element{morgans}{
\begin{question}{test1C-Q10}
    Which of the following statements about the Earth's orbit is \emph{false}?
    \begin{choices}
        \wrongchoice{The orbit is elliptical}
        \wrongchoice{The average distance between the Earth and the Sun is 1 A. U.}
      \correctchoice{The orbital velocity is constant}
        \wrongchoice{the period of the orbit is 1 year}
    \end{choices}
\end{question}
}

\element{morgans}{
\begin{question}{test1C-Q11}
    Which of the following lists of types of light is in the correct order from shortest to longest wavelength? 
    (NOTE: Not all types of light are included in each list.)
    \begin{choices}
        \wrongchoice{radio, microwave, red, green, blue, gamma ray}
        \wrongchoice{X ray, ultraviolet, infrared, red, radio}
      \correctchoice{gamma ray, ultraviolet, yellow, red, microwave, radio}
        \wrongchoice{microwave, X ray, green, red, blue, infrared}
    \end{choices}
\end{question}
}

\element{morgans}{
\begin{question}{test1C-Q12}
    Spectral lines unique to each type of atom are caused by:
    \begin{choices}
        \wrongchoice{each atom having a unique set of protons.}
      \correctchoice{the unique sets of electron orbits.}
        \wrongchoice{the neutron-electron interaction being unique for each atom.}
        \wrongchoice{each type of photon emitted by the atom being unique.}
        \wrongchoice{none of the above; spectral lines are not unique to each type of atom.}
    \end{choices}
\end{question}
}

\element{morgans}{
\begin{question}{test1C-Q13}
    The greater the \rule[-0.1pt]{4em}{0.1pt} the greater the light gathering power. 
    (Pick the \emph{one} most correct choice.)
    \begin{choices}
        \wrongchoice{objective focal length}
        \wrongchoice{resolution}
      \correctchoice{objective area}
        \wrongchoice{magnification}
    \end{choices}
\end{question}
}

\element{morgans}{
\begin{question}{test1C-Q14}
    Which of the following types of light would you only be able to study through the use of a satellite telescope?
    \begin{choices}
      \correctchoice{X ray}
        \wrongchoice{visible}
        \wrongchoice{infrared}
        \wrongchoice{radio}
    \end{choices}
\end{question}
}

\element{morgans}{
\begin{question}{test1C-Q15}
    Which of the following does \emph{not} affect the quality of an observing site?
    \begin{choices}
        \wrongchoice{latitude}
      \correctchoice{longitude}
        \wrongchoice{altitude}
        \wrongchoice{distance from populated areas}
    \end{choices}
\end{question}
}


\begin{comment}
    Fill In
    Place the most appropriate word or words in the blank. You may have to click on the blank to activate it before you start typing in your answer.
    The location straight over your head is known as your .

    The celestial coordinate corresponding to longitude on the Earth is called .

    The time interval for the revolution of the Moon around the Earth with respect to the stars is called .

    was the ancient scientist who wrote book Almagest which summarized his work as well as the astronomical knowledge of earlier cultures.

    A measure of the amount of material in an object is its .

    is the fundamental force which has influence over the largest possible distances.

    The is the distribution of electromagnetic energy with wavelength.

    The type of electromagnetic wave which is adjacent to visible light but at a longer wavelength is radiation.

    A charged atomic particle is called a(n) .

    A is a type of telescope which uses lenses to gather and focus light
\end{comment}


\endinput


