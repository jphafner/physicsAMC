
%% university of northern iowa
%%  morgans astronomy exams
%%--------------------------------------------------

%% this section contains 20 problems

\element{morgans}{
\begin{question}{test4C-q01}
    Which of the following types of objects do \emph{not} generally lie near the plane of the ecliptic?
    \begin{choices}
      \correctchoice{long period comets}
        \wrongchoice{short period comets}
        \wrongchoice{planets}
        \wrongchoice{asteroids}
        \wrongchoice{interplanetary dust grains}
    \end{choices}
\end{question}
}

\element{morgans}{
\begin{question}{test4C-q02}
    Planetismals are:
    \begin{choices}
      \correctchoice{small objects that went into the formation of the planets.}
        \wrongchoice{another name for moons.}
        \wrongchoice{another name for asteroids.}
        \wrongchoice{produced by impacts of meteorites on planets.}
    \end{choices}
\end{question}
}

\element{morgans}{
\begin{question}{test4C-q03}
    How were the Jovian planets able to retain their original atmospheres?
    \begin{choices}
      \correctchoice{they have high escape velocities and low temperatures}
        \wrongchoice{the atmospheres are composed mainly of heavy molecules such as CO2}
        \wrongchoice{they have strong magnetic fields}
        \wrongchoice{the question is incorrect---they \emph{did} lose all of their original atmosphere}
    \end{choices}
\end{question}
}

\element{morgans}{
\begin{question}{test4C-q04}
    How was the carbon dioxide removed from the Earth's early atmosphere?
    \begin{choices}
        \wrongchoice{it was blown off by volcanic eruptions}
        \wrongchoice{it was absorbed by the plant life}
      \correctchoice{it was absorbed by the rocks}
        \wrongchoice{it was blown off by meteoric impacts}
    \end{choices}
\end{question}
}

\element{morgans}{
\begin{question}{test4C-q05}
    Which theory is currently viewed as the best one to explain the origin of the Moon?
    \begin{choices}
      \correctchoice{impact theory---the Moon formed from debris produced by an impact on the Earth.}
        \wrongchoice{capture theory---the Moon formed elsewhere in the solar system and was later captured by the Earth's gravity.}
        \wrongchoice{co-formation---the Moon and Earth formed as two separate object out of the same material.}
    \end{choices}
\end{question}
}

\element{morgans}{
\begin{question}{test4C-q06}
    From the discovery of Mercury's magnetic field astronomers conclude:
    \begin{choices}
        \wrongchoice{Mercury is uniform throughout its interior.}
        \wrongchoice{Mercury has a very small solid core.}
      \correctchoice{Mercury has a very large core.}
        \wrongchoice{nothing about the interior of Mercury.}
    \end{choices}
\end{question}
}

\element{morgans}{
\begin{question}{test4C-q07}
    Why is there no evidence for small impact craters on Venus?
    \begin{choices}
        \wrongchoice{no objects have ever reached the surface, due to Venus' strong magnetic field}
      \correctchoice{the thick atmosphere burns up most small meteors}
        \wrongchoice{the impact craters have been eroded by water and wind}
        \wrongchoice{the question is incorrect, there are many small impact craters on Venus}
    \end{choices}
\end{question}
}

\element{morgans}{
\begin{question}{test4C-q08}
    The surface of Venus is best characterized as:
    \begin{choices}
      \correctchoice{rolling plains, covering most of the planet.}
        \wrongchoice{rugged, highland-type regions covering about \SI{75}{\percent} of the surface.}
        \wrongchoice{mare (lowlands) covering more than half of the surface.}
        \wrongchoice{rolling plains covering about \SI{10}{\percent} of the surface.}
    \end{choices}
\end{question}
}

\element{morgans}{
\begin{question}{test4C-q09}
    How many satellites is Mars known to have?
    \begin{choices}
        \wrongchoice{0}
        \wrongchoice{1}
      \correctchoice{2}
        \wrongchoice{3}
        \wrongchoice{4}
    \end{choices}
\end{question}
}

\element{morgans}{
\begin{question}{test4C-q10}
    The Martian satellites probably formed:
    \begin{choices}
        \wrongchoice{out of the same material from which Mars formed.}
      \correctchoice{in the asteroid belt, and were later captured.}
        \wrongchoice{about a billion years after Mars was formed.}
        \wrongchoice{about a billion years before Mars was formed.}
    \end{choices}
\end{question}
}

\element{morgans}{
\begin{question}{test4C-q11}
    Which planet has scarps on its surface?
    \begin{choices}
      \correctchoice{Mercury}
        \wrongchoice{Venus}
        \wrongchoice{Earth}
        \wrongchoice{Mars}
    \end{choices}
\end{question}
}

\element{morgans}{
\begin{question}{test4C-q12}
    Near the core of Jupiter, hydrogen is:
    \begin{choices}
        \wrongchoice{a low temperature gas.}
        \wrongchoice{a high temperature gas.}
        \wrongchoice{a solid.}
        \wrongchoice{a liquid.}
      \correctchoice{a liquid metal.}
    \end{choices}
\end{question}
}

\element{morgans}{
\begin{question}{test4C-q13}
    The source of Jupiter's excess energy is thought to be:
    \begin{choices}
        \wrongchoice{lightning bolts in the atmosphere.}
      \correctchoice{internal heat left over from its formation.}
        \wrongchoice{produced by tides between the planet and the Sun.}
        \wrongchoice{absorption of energy from beyond the solar system with subsequent re-emission.}
    \end{choices}
\end{question}
}

\element{morgans}{
\begin{question}{test4C-q14}
    Volcanos on Io result from:
    \begin{choices}
         \wrongchoice{its hot atmosphere.}
         \wrongchoice{plate tectonics.}
         \wrongchoice{high velocity meteoroid impacts.}
       \correctchoice{tidal friction from Jupiter.}
         \wrongchoice{nothing: volcanoes are on Europa, not Io.}
    \end{choices}
\end{question}
}

\element{morgans}{
\begin{question}{test4C-q15}
    Seasons on Uranus,
        which has a period of revolution of 80 years, are:
    \begin{choices}
      \correctchoice{approximately 20 Earth years long.}
        \wrongchoice{approximately 40 Earth years long.}
        \wrongchoice{approximately 80 Earth years long.}
        \wrongchoice{non-existent.}
        \wrongchoice{approximately three Earth months long.}
    \end{choices}
\end{question}
}

\element{morgans}{
\begin{question}{test4C-q16}
    Uranus has how many known satellites?
    \begin{choices}
        \wrongchoice{none}
        \wrongchoice{7}
        \wrongchoice{10}
        \wrongchoice{17}
      \correctchoice{More than 20.}
    \end{choices}
\end{question}
}

\element{morgans}{
\begin{question}{test4C-q17}
    Pluto has how many known satellites?
    \begin{choices}
        \wrongchoice{none}
      \correctchoice{one}
        \wrongchoice{two}
        \wrongchoice{three}
        \wrongchoice{five}
    \end{choices}
\end{question}
}

\element{morgans}{
\begin{question}{test4C-q18}
    The orbits of long-period comets are:
    \begin{choices}
        \wrongchoice{circular.}
        \wrongchoice{in the ecliptic.}
      \correctchoice{randomly oriented with respect to the ecliptic.}
        \wrongchoice{similar to Pluto's orbit.}
        \wrongchoice{the same as the orbits of short-period comets.}
    \end{choices}
\end{question}
}

\element{morgans}{
\begin{question}{test4C-q19}
    Comet gas tails always extend from the comet in the direction:
    \begin{choices}
        \wrongchoice{to the Sun.}
      \correctchoice{opposite the Sun.}
        \wrongchoice{opposite the comet's motion.}
        \wrongchoice{of the comet's motion.}
        \wrongchoice{to Jupiter.}
    \end{choices}
\end{question}
}

\element{morgans}{
\begin{question}{test4C-q20}
    Which of the following characterizes Jovian planets? 
    \begin{choices}
        \wrongchoice{small size}
        \wrongchoice{high density}
      \correctchoice{primarily hydrogen and helium}
        \wrongchoice{slow rotation}
    \end{choices}
\end{question}
}

\begin{comment}
    Fill In
    Place the most appropriate word or words in the blank. You may have to click on the blank to activate it before you start typing in your answer.
    is the chemical element which makes up most of the atmosphere of the Jovian planets.
    waves move faster and can travel through both liquids and solids.
    Amongst the terrestrial planets, has the strongest magnetic field.
    is the name of the process in which one tectonic plate is submerged below another along a line where two plates collide.
    is the name of the extensive, smooth lowland regions on the Moon.
    The is the reason Venus' surface temperature is high.
    Near the core of Jupiter is a layer composed of hydrogen in a state called .
    The satellite of Neptune which has an atmosphere is .
    Objects located primarily between the orbits of Mars and Jupiter are called .
    The curved tail of a comet is called the tail.
\end{comment}


\endinput


