

%% Astronomy II Practice Exam 4
%%--------------------------------------------------

%% this section contains 35 problems

\element{astr}{
\begin{question}{ASTRI-4-Q01}
    What is the maximum number of planets readily visible to the naked eye on a given night?
    \begin{multicols}{2}
    \begin{choices}
        \wrongchoice{\num{1}}
        \wrongchoice{\num{2}}
        \wrongchoice{\num{3}}
        \wrongchoice{\num{4}}
      \correctchoice{\num{5}}
    \end{choices}
    \end{multicols}
\end{question}
}

\element{astr}{
\begin{question}{ASTRI-4-Q02}
    Planets:
    \begin{multicols}{2}
    \begin{choices}
        \wrongchoice{move rapidly across the sky relative to the stars.}
        \wrongchoice{are stationary relative to the stars.}
        \wrongchoice{all move at the same rate relative to the stars.}
      \correctchoice{move slowly relative to the stars.}
        \wrongchoice{can appear anywhere in the sky.}
    \end{choices}
    \end{multicols}
\end{question}
}

\element{astr}{
\begin{question}{ASTRI-4-Q03}
    The Moon:
    \begin{choices}
        \wrongchoice{may appear anywhere in the sky.}
      \correctchoice{always appears within a few degrees of the zodiac.}
        \wrongchoice{always appears within a few degrees of the celestial equator.}
        \wrongchoice{generally appears opposite the Sun.}
    \end{choices}
\end{question}
}

\element{astr}{
\begin{question}{ASTRI-4-Q04}
    The most readily observed motion of a celestial object is produced by:
    \begin{choices}
        \wrongchoice{the motion of the planets across the sky.}
      \correctchoice{the rotation of the Earth.}
        \wrongchoice{the revolution of the Earth.}
        \wrongchoice{the motion of the Sun around the galaxy.}
    \end{choices}
\end{question}
}

\element{astr}{
\begin{question}{ASTRI-4-Q05}
    An astronomical unit is the:
    \begin{choices}
        \wrongchoice{distance from the Earth to the Moon.}
      \correctchoice{distance from the Earth to the Sun.}
        \wrongchoice{distance from the Earth to the nearest star.}
        \wrongchoice{distance light travels in one year.}
        \wrongchoice{circumference of the Earth.}
    \end{choices}
\end{question}
}

\element{astr}{
\begin{question}{ASTRI-4-Q06}
    Which of the following statements about planets is \emph{false}?
    \begin{choices}
      \correctchoice{none are visible to observers on the Earth}
        \wrongchoice{they move relative to the stars}
        \wrongchoice{they are found along the zodiac}
        \wrongchoice{they do not twinkle as stars do}
    \end{choices}
\end{question}
}

\element{astr}{
\begin{question}{ASTRI-4-Q07}
    Where must an observer be located on the Earth to view the entire sky over the course of a year?
    \begin{choices}
        \wrongchoice{the north pole}
        \wrongchoice{the south pole}
      \correctchoice{the equator}
        \wrongchoice{anywhere on the Earth}
    \end{choices}
\end{question}
}

\element{astr}{
\begin{question}{ASTRI-4-Q08}
    Diurnal motions are caused by:
    \begin{choices}
        \wrongchoice{the rapid rotations of heavenly bodies.}
        \wrongchoice{the motion of the Moon about the Earth.}
        \wrongchoice{the motion of the Sun about the Earth.}
      \correctchoice{the motion of the Earth on its rotation axis.}
        \wrongchoice{the precession of the Earth's axis.}
    \end{choices}
\end{question}
}

\element{astr}{
\begin{question}{ASTRI-4-Q09}
    Suppose you are on a strange planet. 
    Since you have had an astronomy class at the university,
        you are aware of the daily motion of stars about a fixed point in the sky. 
    Furthermore, you notice that this fixed point is 30 degrees above the horizon. 
    You then deduce that your latitude on this planet is:
    \begin{multicols}{3}
    \begin{choices}
        \wrongchoice{\ang{0}}
        \wrongchoice{\ang{15}}
      \correctchoice{\ang{30}}
        \wrongchoice{\ang{45}}
        \wrongchoice{\ang{60}}
    \end{choices}
    \end{multicols}
\end{question}
}

\element{astr}{
\begin{question}{ASTRI-4-Q10}
    Precession is:
    \begin{choices}
        \wrongchoice{the accuracy with which numbers are given in astronomy.}
      \correctchoice{the slow motion of the Earth's rotation axis on the celestial sphere.}
        \wrongchoice{the apparent backward motion of planets on the celestial sphere.}
        \wrongchoice{the daily eastward motion of the Sun around the celestial sphere.}
    \end{choices}
\end{question}
}

\element{astr}{
\begin{question}{ASTRI-4-Q11}
    Planet $X$ has its rotation axis perpendicular to its orbital plane. 
    Its seasons would be:
    \begin{choices}
        \wrongchoice{shorter than those on Earth.}
        \wrongchoice{longer than those on Earth.}
        \wrongchoice{the same as those on Earth.}
      \correctchoice{constant.}
    \end{choices}
\end{question}
}

\element{astr}{
\begin{question}{ASTRI-4-Q12}
    If a solar eclipse occured 2 weeks ago,
        what would be the phase of the Moon today?
    \begin{choices}
        \wrongchoice{first quarter}
      \correctchoice{full}
        \wrongchoice{third quarter}
        \wrongchoice{new}
        \wrongchoice{waxing crescent}
    \end{choices}
\end{question}
}

\element{astr}{
\begin{question}{ASTRI-4-Q13}
    Eclipses do not occur each month because:
    \begin{choices}
        \wrongchoice{the Moon is always in the ecliptic.}
        \wrongchoice{the Moon is never in the ecliptic.}
        \wrongchoice{the Earth's axis is tilted to the ecliptic.}
        \wrongchoice{the Moon's orbit is in the ecliptic.}
      \correctchoice{the Moon's orbit is not in the ecliptic.}
    \end{choices}
\end{question}
}

\element{astr}{
\begin{question}{ASTRI-4-Q14}
    In order for a solar eclipse to occur,
        the Moon must be:
    \begin{choices}
      \correctchoice{near new Moon.}
        \wrongchoice{near first or last quarter.}
        \wrongchoice{high in the sky.}
        \wrongchoice{near full Moon.}
        \wrongchoice{in a retrograde loop.}
    \end{choices}
\end{question}
}

\element{astr}{
\begin{question}{ASTRI-4-Q15}
    The length of the tropical year is:
    \begin{choices}
        \wrongchoice{equal to the length of the Earth's sidereal period.}
        \wrongchoice{equal to the length of the Earth's synodic period.}
      \correctchoice{the time interval from one vernal equinox to the next.}
        \wrongchoice{twelve lunar months.}
    \end{choices}
\end{question}
}

\element{astr}{
\begin{question}{ASTRI-4-Q16}
    If the Moon did \emph{not} rotate we would observe
    \begin{choices}
        \wrongchoice{the same as we now observe.}
        \wrongchoice{only the lunar back side.}
        \wrongchoice{the lunar north polar region.}
      \correctchoice{both the front and back side of the Moon.}
    \end{choices}
\end{question}
}

\element{astr}{
\begin{question}{ASTRI-4-Q17}
    Aristotle:
    \begin{choices}
        \wrongchoice{was the first great observational astronomer.}
      \correctchoice{stated physical laws and then attempted to use them to explain how the universe works.}
        \wrongchoice{discovered the first four elements in the periodic table of elements.}
        \wrongchoice{taught Plato the basic laws of nature.}
    \end{choices}
\end{question}
}

\element{astr}{
\begin{question}{ASTRI-4-Q18}
    Ptolemy:
    \begin{choices}
        \wrongchoice{invented calculus and used it to predict the positions of the planets at any given time.}
      \correctchoice{wrote books summarizing the astronomical knowledge of earlier cultures.}
        \wrongchoice{was the first of the great Greek astronomers.}
        \wrongchoice{was the first to detect stellar parallax.}
    \end{choices}
\end{question}
}

\element{astr}{
\begin{question}{ASTRI-4-Q19}
    The Almagest was written by:
    \begin{choices}
        \wrongchoice{Plato}
        \wrongchoice{Aristotle}
        \wrongchoice{Hipparchus}
      \correctchoice{Ptolemy}
        \wrongchoice{Pythagorus}
    \end{choices}
\end{question}
}

\element{astr}{
\begin{question}{ASTRI-4-Q20}
    The reason Copernicus became a ``heliocentrist'' was that:
    \begin{choices}
        \wrongchoice{the evidence was overwhelmingly strong.}
        \wrongchoice{the evidence was weak but gaining strength.}
      \correctchoice{it was philosophically pleasing to him.}
        \wrongchoice{the laws of physics as then understood indicated a heliocentric universe.}
    \end{choices}
\end{question}
}

\element{astr}{
\begin{question}{ASTRI-4-Q21}
    Which one of the following statements about the Copernican model is \emph{false}?
    \begin{choices}
      \correctchoice{it was more accurate than the Ptolemaic system in predicting planetary motions}
        \wrongchoice{relative planetary distances could be deduced from it}
        \wrongchoice{relative planetary speeds could be determined from it}
        \wrongchoice{retrograde motion could be easily explained by it}
        \wrongchoice{none: all of the above statements are true}
    \end{choices}
\end{question}
}

\element{astr}{
\begin{question}{ASTRI-4-Q22}
    Tycho Brahe's principal contribution to astronomy was:
    \begin{choices}
        \wrongchoice{his noble blood.}
        \wrongchoice{his suggested model for the solar system (which had a fixed Earth with the Sun revolving about it but the rest of the planets revolving about the Sun).}
      \correctchoice{the accuracy of his observations and the completeness of his records.}
        \wrongchoice{his choice of Galileo as an assistant.}
    \end{choices}
\end{question}
}

\element{astr}{
\begin{question}{ASTRI-4-Q23}
    Which of the following is a statement of Kepler's first law?
    \begin{choices}
        \wrongchoice{planets move in perfect circles with the Sun at the center}
        \wrongchoice{planets move along an elliptical path with the Sun at the center}
      \correctchoice{planets move along an elliptical path with the Sun at one of the foci}
        \wrongchoice{planets move along an elliptical path with the Earth at one of the foci}
    \end{choices}
\end{question}
}

\element{astr}{
\begin{question}{ASTRI-4-Q24}
    In simple language, Kepler's second law means that:
    \begin{choices}
      \correctchoice{a planet moves more rapidly when near the Sun than when farther away.}
        \wrongchoice{planets close to the Sun have shorter periods than those farther away.}
        \wrongchoice{the Sun is at the center of planetary orbits.}
        \wrongchoice{slowly moving planets are close to the Sun.}
    \end{choices}
\end{question}
}

\element{astr}{
\begin{question}{ASTRI-4-Q25}
    In non-mathematical terms,
        Kepler's third law says that:
    \begin{choices}
        \wrongchoice{a planet moves more rapidly when near the Sun than when farther away.}
      \correctchoice{planets close to the Sun have shorter periods than those farther away.}
        \wrongchoice{the Sun is at the center of planetary orbits.}
        \wrongchoice{slowly moving planets are close to the Sun.}
    \end{choices}
\end{question}
}

\element{astr}{
\begin{question}{ASTRI-4-Q26}
    Who is often considered to be the first truly modern scientist?
    \begin{choices}
        \wrongchoice{Brahe}
        \wrongchoice{Kepler}
        \wrongchoice{Copernicus}
        \wrongchoice{Aristotle}
      \correctchoice{Galileo}
    \end{choices}
\end{question}
}

\element{astr}{
\begin{question}{ASTRI-4-Q27}
    Which of the following did Galileo \emph{not} observe?
    \begin{choices}
        \wrongchoice{sunspots}
      \correctchoice{the moons of Mars}
        \wrongchoice{the phases of Venus}
        \wrongchoice{the craters on the Moon}
    \end{choices}
\end{question}
}

\element{astr}{
\begin{question}{ASTRI-4-Q28}
    Which of the following has the greatest mass?
    \begin{choices}
        %% questionmulti with all correct?
        \wrongchoice{\SI{100}{\pound} of goose feathers}
        \wrongchoice{\SI{100}{\pound} of lead}
        \wrongchoice{a \SI{100}{\pound} person}
      \correctchoice{all other options are the same mass}
    \end{choices}
\end{question}
}

\element{astr}{
\begin{question}{ASTRI-4-Q29}
    Acceleration is defined as:
    \begin{choices}
      \correctchoice{the rate of change of velocity.}
        \wrongchoice{the rate of change of position.}
        \wrongchoice{the rate of change of distance.}
        \wrongchoice{how fast an object moves.}
        \wrongchoice{how fast an object changes position.}
    \end{choices}
\end{question}
}

\element{astr}{
\begin{question}{ASTRI-4-Q30}
    Which of the following would \emph{not} occur if the Earth's mass were doubled? 
    (NOTE: the radius remains the same)
    \begin{choices}
      \correctchoice{your mass would double}
        \wrongchoice{your weight would double}
        \wrongchoice{the surface gravity would double}
        \wrongchoice{the escape velocity would increase}
    \end{choices}
\end{question}
}

\element{astr}{
\begin{question}{ASTRI-4-Q31}
    While on the Moon, the Apollo astronauts demonstrated Galileo's experiment at the Leaning Tower of Pisa by dropping a feather and a hammer. 
    They reached the ground at the same time because:
    \begin{choices}
        \wrongchoice{the force of gravity is larger on the feathers than on the hammer.}
        \wrongchoice{the force of gravity has no effect on either object.}
      \correctchoice{the acceleration of each object is the same.}
        \wrongchoice{the astronauts showed Galileo's experiment to be false.}
    \end{choices}
\end{question}
}

\element{astr}{
\begin{question}{ASTRI-4-Q32}
    Take three identical bricks; strap two of them together. 
    Which statement is true?
    \begin{choices}
        \wrongchoice{the combined bricks, when dropped, will fall twice as fast as the single brick}
        \wrongchoice{the combined bricks, when dropped, will fall four times as fast as the single brick due to the inverse square law of gravity}
      \correctchoice{the gravitational force between the Earth and the combined bricks is twice as strong as the gravitational force between the Earth and the single brick}
        \wrongchoice{the gravitational force between the Earth and the combined bricks is the same as the gravitational force between the Earth and the single brick}
    \end{choices}
\end{question}
}

\element{astr}{
\begin{question}{ASTRI-4-Q33}
    If in a violent moment you kick a wall, your foot will hurt.
    This is best explained by:
    \begin{choices}
        \wrongchoice{Newton's first law of motion.}
        \wrongchoice{Newton's second law of motion.}
      \correctchoice{Newton's third law of motion.}
        \wrongchoice{the universal law of gravity.}
    \end{choices}
\end{question}
}

\element{astr}{
\begin{question}{ASTRI-4-Q34}
    Which of the following statements about the Earth's orbit is \emph{false}?
    \begin{choices}
        \wrongchoice{The orbit is elliptical}
        \wrongchoice{The average distance to between the Earth and the Sun is 1 A. U.}
      \correctchoice{The orbital velocity is constant}
        \wrongchoice{the period of the orbit is 1 year}
    \end{choices}
\end{question}
}

\element{astr}{
\begin{question}{ASTRI-4-Q35}
    The escape velocity from a planet's surface depends upon:
    \begin{choices}
        \wrongchoice{your mass and the planet's mass.}
        \wrongchoice{your mass and the planet's radius.}
      \correctchoice{the planet's mass and radius.}
        \wrongchoice{your mass and the planets orbital period.}
        \wrongchoice{the planet's mass and orbital period.}
    \end{choices}
\end{question}
}

% Answer Key
%  1. e       
%  2. d
%  3. b
%  4. b       
%  5. b
%  6. a
%  7. c
%  8. d
%  9. c
% 10. b
% 11. d
% 12. b
% 13. e
% 14. a
% 15. c
% 16. d
% 17. b
% 18. b
% 19. d
% 20. c
% 21. a
% 22. c
% 23. c
% 24. a
% 25. b
% 26. e
% 27. b
% 28. d
% 29. a
% 30. a
% 31. c
% 32. c
% 33. c
% 34. c
% 35. c

 
\endinput 


