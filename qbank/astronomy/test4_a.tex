
%% university of northern iowa
%%  morgans astronomy exams
%%--------------------------------------------------

%% this section contains 20 problems

\element{morgans}{
\begin{question}{test4A-q01}
    Which of the following statements about the solar system is \emph{false}?
    \begin{choices}
        \wrongchoice{all planets orbit the Sun in the same direction}
        \wrongchoice{all planets are found within 20 degrees of the ecliptic plane}
      \correctchoice{all planets orbit in eliptical paths}
        \wrongchoice{all planets orbit in a manner similar to a rigid phonograph record}
    \end{choices}
\end{question}
}

\element{morgans}{
\begin{question}{test4A-q02}
    Terrestrial planets start with different chemical compositions than Jovian planets because:
    \begin{choices}
        \wrongchoice{they formed from radically different material.}
      \correctchoice{of high temperatures in the inner solar system.}
        \wrongchoice{terrestrial planets are smaller than Jovian planets.}
        \wrongchoice{Jovian planets have stronger magnetic fields.}
        \wrongchoice{the many satellites of the Jovian planets modified the planet's atmosphere.}
    \end{choices}
\end{question}
}

\element{morgans}{
\begin{question}{test4A-q03}
    Why is it believed that Mars' atmosphere was thicker in the past?
    \begin{choices}
        \wrongchoice{we have detected that in rock samples}
        \wrongchoice{we have seen it in old Earth based spectroscopic studies}
      \correctchoice{it is needed to account for the presence of major water erosion features on the surface}
        \wrongchoice{its atmosphere is identical to that of Venus which has liquid water on its surface}
    \end{choices}
\end{question}
}

\element{morgans}{
\begin{question}{test4A-q04}
    A magnetic dynamo is caused by a combination of fluid,
        electrically conductin core and:
    \begin{choices}
      \correctchoice{rotation.}
        \wrongchoice{cosmic rays from space.}
        \wrongchoice{lunar tidal action.}
        \wrongchoice{the solar gravitational pull on the Earth's equatorial bulge.}
    \end{choices}
\end{question}
}

\element{morgans}{
\begin{question}{test4A-q05}
    Compared with the side of the Moon facing the Earth,
        the lunar backside has:
    \begin{choices}
        \wrongchoice{more mare.}
        \wrongchoice{substantially fewer craters.}
        \wrongchoice{substantially fewer mountain ranges.}
      \correctchoice{a thicker crust.}
        \wrongchoice{one active volcano, while the near side has none.}
    \end{choices}
\end{question}
}

\element{morgans}{
\begin{question}{test4A-q06}
    Mercury's rotation period was determined from Earth-based observations by:
    \begin{choices}
        \wrongchoice{visual observations of the motion of surface features across the planet.}
      \correctchoice{Doppler shift observations of radar signals.}
        \wrongchoice{Doppler shift observations of ultraviolet light.}
        \wrongchoice{theoretical calculations of the spin-orbit coupling.}
        \wrongchoice{using Kepler's third law with its satellite.}
    \end{choices}
\end{question}
}

\element{morgans}{
\begin{question}{test4A-q07}
    Which of the following probes did NOT observe or explore Venus?
    \begin{choices}
        \wrongchoice{Magellan}
        \wrongchoice{Venera}
        \wrongchoice{Mariner 10}
      \correctchoice{Viking}
    \end{choices}
\end{question}
}

\element{morgans}{
\begin{question}{test4A-q08}
    The pressure of Venus' atmosphere at its surface is:
    \begin{choices}
        \wrongchoice{0.1 that of the Earth's atmospheric pressure.}
        \wrongchoice{the same as the Earth's atmospheric pressure.}
        \wrongchoice{10 times the Earth's atmospheric pressure.}
      \correctchoice{100 times the Earth's atmospheric pressure.}
        \wrongchoice{too large to measure.}
    \end{choices}
\end{question}
}

\element{morgans}{
\begin{question}{test4A-q09}
    In terms of composition,
        the atmosphere of Mars is most similar to that of:
    \begin{choices}
        \wrongchoice{Mercury.}
      \correctchoice{Venus.}
        \wrongchoice{Earth.}
        \wrongchoice{Moon.}
    \end{choices}
\end{question}
}

\element{morgans}{
\begin{question}{test4A-q10}
    Which one of the following statements is correct?
    \begin{choices}
        \wrongchoice{there has never been any running water on Mars}
        \wrongchoice{space probes have found small amounts of running water in the bottom of some deep canyons}
      \correctchoice{water ice probably exists below the surface}
        \wrongchoice{the polar caps consist entirely of water ice}
    \end{choices}
\end{question}
}

\element{morgans}{
\begin{question}{test4A-q11}
    Which planet has the largest known volcano?
    \begin{choices}
        \wrongchoice{Mercury}
        \wrongchoice{Venus}
        \wrongchoice{Earth}
      \correctchoice{Mars}
    \end{choices}
\end{question}
}

\element{morgans}{
\begin{question}{test4A-q12}
    Pluto was discovered by:
    \begin{choices}
        \wrongchoice{David Tholen.}
        \wrongchoice{William Herschel.}
        \wrongchoice{Galileo.}
        \wrongchoice{Hipparchus.}
      \correctchoice{Clyde W. Tombaugh.}
    \end{choices}
\end{question}
}

\element{morgans}{
\begin{question}{test4A-q13}
    The visible structure in Saturn's atmosphere is more muted than Jupiter's visible structure because Saturn:
    \begin{choices}
        \wrongchoice{does not have a volcanically active satellite like Io.}
        \wrongchoice{has a weaker magnetic field.}
        \wrongchoice{has a thicker water cloud layer.}
      \correctchoice{has a thick layer of haze high in the atmosphere.}
        \wrongchoice{has a vastly different chemical composition.}
    \end{choices}
\end{question}
}

\element{morgans}{
\begin{question}{test4A-q14}
    Which of the Galilean satellites shows the oldest surface?
    \begin{choices}
        \wrongchoice{Io}
        \wrongchoice{Ganymede}
        \wrongchoice{Callisto}
        \wrongchoice{Europa}
        \wrongchoice{the surfaces are all the same age---the age of the solar system}
    \end{choices}
\end{question}
}

\element{morgans}{
\begin{question}{test4A-q15}
    Which of of the following planets has retrograde rotation? 
    \begin{choices}
        \wrongchoice{Jupiter and Neptune}
        \wrongchoice{Jupiter and Saturn}
      \correctchoice{Uranus and Pluto}
        \wrongchoice{Uranus and Neptune}
    \end{choices}
\end{question}
}

\element{morgans}{
\begin{question}{test4A-q16}
    Rings around Uranus were discovered by means of:
    \begin{choices}
        \wrongchoice{photography from a high altitude observatory on Earth.}
        \wrongchoice{photography from spacecraft.}
      \correctchoice{variations of a star's brightness as it passed behind the ring.}
        \wrongchoice{theoretical prediction.}
        \wrongchoice{none of the above, since Uranus has no known rings.}
    \end{choices}
\end{question}
}

\element{morgans}{
\begin{question}{test4A-q17}
    Which of the following statements about Pluto is true?
    \begin{choices}
        \wrongchoice{it is currently the closest planet to the Sun}
        \wrongchoice{its orbit is in the plane of the Earth's orbit}
        \wrongchoice{the orbit is as round as the Earth's orbit}
      \correctchoice{it is the smallest planet}
    \end{choices}
\end{question}
}

\element{morgans}{
\begin{question}{test4A-q18}
    If all the material in the asteroid belt were in a single body,
        the mass of this body would be roughly that of:
    \begin{choices}
        \wrongchoice{the Moon.}
        \wrongchoice{the Earth.}
        \wrongchoice{Jupiter.}
        \wrongchoice{none of the provided, it is much less than any planet.}
    \end{choices}
\end{question}
}

\element{morgans}{
\begin{question}{test4A-q19}
    The nucleus of Halley's comet has over time:
    \begin{choices}
        \wrongchoice{increased in size.}
        \wrongchoice{decreased in size.}
        \wrongchoice{remained the same.}
    \end{choices}
\end{question}
}

\element{morgans}{
\begin{question}{test4A-q20}
    Which of the following statements is true?
    \begin{choices}
        \wrongchoice{all planetary orbits lie within the ecliptic}
        \wrongchoice{all planetary orbits are circles to within the accuracy of our measurements}
        \wrongchoice{as viewed from the north pole, all planets orbit counterclockwise}
        \wrongchoice{the rotation axis of each planet is perpendicular to its orbital plane}
        \wrongchoice{planets always rotate in the same direction in which they revolve}
    \end{choices}
\end{question}
}

\begin{comment}
    Fill In
    Place the most appropriate word or words in the blank. You may have to click on the blank to activate it before you start typing in your answer.
     
     
    The direction of motion of all planets around the Sun is said to be .

    Segments of the Earth's crust which move about over the mantle are known as .

    is the name applied to the distance from a planet at which tidal forces keep particles from coalescing, and marks the boundary for the formation of rings.

    The Hawaii and Icelandic volcanos produce rock.

    Consider a rock composed of elements A and B. Element A, with a half-life of 1 billion years, decays into element B. In addition, when the rock was formed, element B was not present. Now, however, only 1/16 of the rock is composed of element A and the rest is composed of element B. The rock is years old.

    The conclusion that Mercury has a large core comes from observations of its .

    Mars has known satellites.

    Pluto has known satellites.

    Pluto's atmosphere consists of .

    The standard model for the structure of a comet's nucleus is known as the model.
\end{comment}

\endinput



