
%% university of northern iowa
%%  morgans astronomy exams
%%--------------------------------------------------

%% this section contains 20 problems

\element{morgans}{
\begin{question}{test4E-q01}
    The atmospheres of the Jovian planets consist mostly of:
    \begin{choices}
        \wrongchoice{oxygen.}
        \wrongchoice{carbon dioxide.}
        \wrongchoice{water vapor.}
      \correctchoice{hydrogen.}
        \wrongchoice{nitrogen.}
    \end{choices}
\end{question}
}

\element{morgans}{
\begin{question}{test4E-q02}
    The production of a planetary magnetic field requires:
    \begin{choices}
        \wrongchoice{high planetary mass.}
        \wrongchoice{at least one satellite.}
        \wrongchoice{an atmosphere.}
      \correctchoice{rotation.}
        \wrongchoice{a ring around the planet.}
    \end{choices}
\end{question}
}

\element{morgans}{
\begin{question}{test4E-q03}
    S-waves travel through:
    \begin{choices}
        \wrongchoice{liquids, gases and solids.}
        \wrongchoice{liquids and solids only.}
        \wrongchoice{liquids only.}
      \correctchoice{solids only.}
    \end{choices}
\end{question}
}

\element{morgans}{
\begin{question}{test4E-q04}
    The nitrogen in the Earth's atmosphere came originally from:
    \begin{choices}
        \wrongchoice{the initial atmosphere of the Earth.}
        \wrongchoice{the solar wind.}
        \wrongchoice{the break-up of ammonia.}
      \correctchoice{outgassing from volcanos.}
        \wrongchoice{cosmic rays.}
    \end{choices}
\end{question}
}

\element{morgans}{
\begin{question}{test4E-q05}
    Which one of the following statements describes the formation of the lunar mare?
    \begin{choices}
        \wrongchoice{melting and solidification followed by impact}
        \wrongchoice{volcanism}
      \correctchoice{impact followed by volcanism}
        \wrongchoice{olcanism followed by impact}
    \end{choices}
\end{question}
}

\element{morgans}{
\begin{question}{test4E-q06}
    Jumbled terrain is found:
    \begin{choices}
      \correctchoice{opposite the Caloris basin.}
        \wrongchoice{within large impact basins.}
        \wrongchoice{surrounding scarps.}
        \wrongchoice{only within the Caloris basin.}
    \end{choices}
\end{question}
}

\element{morgans}{
\begin{question}{test4E-q07}
    The temperature on the surface of Venus is closest to \rule[-0.1pt]{4em}{0.1pt} degrees F.
    \begin{choices}
        \wrongchoice{100}
        \wrongchoice{500}
      \correctchoice{1,000}
        \wrongchoice{10,000}
        \wrongchoice{1,000,000}
    \end{choices}
\end{question}
}

\element{morgans}{
\begin{question}{test4E-q08}
    Which one of the following is important in producing the greenhouse effect?
    \begin{choices}
        \wrongchoice{slow rotation}
      \correctchoice{carbon dioxide}
        \wrongchoice{strong magnetic field}
        \wrongchoice{rapid rotation}
        \wrongchoice{oxygen}
    \end{choices}
\end{question}
}

\element{morgans}{
\begin{question}{test4E-q09}
    The yearly variation in the telescopic appearance of Mars is caused by:
    \begin{choices}
        \wrongchoice{variations in atmospheric transparency of the Earth's atmosphere.}
        \wrongchoice{cloud formation in the Martian stratosphere.}
      \correctchoice{blowing dust covering and uncovering rock beneath the dust.}
        \wrongchoice{vegetation growing in the spring and summer, and the subsequent dying in the fall and winter.}
        \wrongchoice{the changing atmospheric transparency caused by the injection of volcanic ash into the atmosphere and its subsequent settling to the surface.}
    \end{choices}
\end{question}
}

\element{morgans}{
\begin{question}{test4E-q10}
    Which planet has the highest surface temperature?
    \begin{choices}
        \wrongchoice{Mercury}
      \correctchoice{Venus}
        \wrongchoice{Earth}
        \wrongchoice{Mars}
    \end{choices}
\end{question}
}

\element{morgans}{
\begin{question}{test4E-q11}
    Jupiter's diameter is approximately \rule[-0.1pt]{4em}{0.1pt} times that of the Earth.
    \begin{choices}
        \wrongchoice{3}
        \wrongchoice{7}
      \correctchoice{11}
        \wrongchoice{25}
        \wrongchoice{100}
    \end{choices}
\end{question}
}

\element{morgans}{
\begin{question}{test4E-q12}
    Which of the following is the correct structure of Jupiter from the core to the surface?
    \begin{choices}
        \wrongchoice{solid rocky core, liquid metallic helium, liquid metallic hydrogen, liquid hydrogen, visible atmosphere}
        \wrongchoice{solid rocky core, liquid metallic hydrogen, liquid hydrogen, visible atmosphere}
        \wrongchoice{solid hydrogen core, liquid iron outer core, liquid hydrogen, visible atmosphere}
        \wrongchoice{solid rocky core, liquid hydrogen, liquid helium, visible atmosphere}
    \end{choices}
\end{question}
}

\element{morgans}{
\begin{question}{test4E-q13}
    How many moons is Saturn known to have?
    \begin{choices}
        \wrongchoice{5}
        \wrongchoice{8}
        \wrongchoice{22}
      \correctchoice{more than 28}
        \wrongchoice{none}
    \end{choices}
\end{question}
}

\element{morgans}{
\begin{question}{test4E-q14}
    Which Saturnian satellite is known to have an atmosphere?
    \begin{choices}
        \wrongchoice{Tethys}
      \correctchoice{Titan}
        \wrongchoice{Triton}
        \wrongchoice{Dione}
        \wrongchoice{Rhea}
    \end{choices}
\end{question}
}

\element{morgans}{
\begin{question}{test4E-q15}
    The Voyager 2 spacecraft observed Uranus' clouds which:
    \begin{choices}
        \wrongchoice{rotate clockwise.}
        \wrongchoice{rotate counter-clockwise.}
        \wrongchoice{do not rotate at all.}
      \correctchoice{did not exist! Uranus was virtually cloudless when Voyager 2 went by.}
    \end{choices}
\end{question}
}

\element{morgans}{
\begin{question}{test4E-q16}
    Which of the following describes the magnetic fields of Uranus and Neptune?
    \begin{choices}
      \correctchoice{off center and tilted with respect to the rotation axis}
        \wrongchoice{off center and aligned with the rotation axis}
        \wrongchoice{through the center and tilted with respect to the rotation axis}
        \wrongchoice{through the center and aligned with the rotation axis}
    \end{choices}
\end{question}
}

\element{morgans}{
\begin{question}{test4E-q17}
    The first asteroid to be discovered,
        and the largest, is named:
    \begin{choices}
        \wrongchoice{Astro-1.}
        \wrongchoice{Juno.}
      \correctchoice{Ceres.}
        \wrongchoice{Tycho.}
    \end{choices}
\end{question}
}

\element{morgans}{
\begin{question}{test4E-q18}
    The Oort cloud is located:
    \begin{choices}
        \wrongchoice{between the orbits of Mars and Jupiter.}
        \wrongchoice{just beyond Pluto's orbit.}
      \correctchoice{between 1000 and 50,000 A. U. from the Sun}
        \wrongchoice{at nearly the distance to the nearest star.}
        \wrongchoice{somewhere, but we have no idea since it has never been seen!}
    \end{choices}
\end{question}
}

\element{morgans}{
\begin{question}{test4E-q19}
    A meteor is:
    \begin{choices}
        \wrongchoice{a rock or grain of sand passing through the Earth's atmosphere.}
      \correctchoice{the trail left by a rock or piece of sand as it passes through the Earth's atmosphere.}
        \wrongchoice{a rock often found in a museum.}
        \wrongchoice{a solar wind particle trapped in the Earth's magnetic field.}
    \end{choices}
\end{question}
}

\element{morgans}{
\begin{question}{test4E-q20}
    The contraction of an interstellar cloud to become a star is caused by:
    \begin{choices}
        \wrongchoice{magnetic forces.}
        \wrongchoice{electric forces.}
        \wrongchoice{nuclear forces.}
      \correctchoice{gravitational forces.}

    \end{choices}
\end{question}
}


\begin{comment}
    Fill In
    Place the most appropriate word or words in the blank. You may have to click on the blank to activate it before you start typing in your answer.
    The Magellan spacecraft has until recently been in orbit about the planet .

    The is the layer of a terrestrial planet directly underneath the crust.

    The trapping of the Sun's radiant energy within the Earth's atmosphere produces heat via the .

    To produce a magnetic field, a planet should have a rapid rotation rate and a(n) .

    types of elements condense (and also become gaseous at low temperatures).

    is the temperature of Venus' surface.

    The Galilean satellite having the largest number of craters is .

    Neptune's ring system is most similar to that of the planet .

    Small bodies orbiting the Sun in elliptical orbits making random angles with the ecliptic are called .

    A cometary nucleus has a diameter of roughly km.
\end{comment}


\endinput


