
%% Astronomy:
%%--------------------------------------------------

%% Chapter 04: Learning Objectives and Study Questions for
%%------------------------------------------------------------


% 1. Describe what types of bodies behave as blackbodies, and use Wien's law and the Stefan-Boltzmann law to calculate how such objects' colors and brightnesses (total EM outputs) respond to changes in their temperatures.
% 2. Distinguish between absorption and emission spectra, and explain their differences in terms of the temperatures (and, thus, initial electronic states) of the low-density gases that produce them.
% 3. Predict whether the "jump" of an electron between specified orbits around an atom's nucleus will result in the absorption or emission of a photon, and how the wavelength or energy of the photon will depend on the energies of the orbits.
% 4. Describe how the Doppler effect alters the apparent lengths of waves being emitted by a moving source, and use this information to determine the relative motion of celestial objects towards or away from Earth by comparing their spectra to a reference spectrum from a stationary source.

\element{astr}{
\begin{question}{ch04-q01}
    Different colors of light correspond to electromagnetic waves with different \rule[-0.1pt]{4em}{0.1pt}.
    \begin{choices}
        \wrongchoice{speeds}
    \end{choices}
\end{question}
}

1. According to Wien’s law, as a dense object (blackbody) becomes hotter, it emits its peak
radiation at a _____ wavelength.
A. brighter
B. dimmer
C. longer
D. shorter
E. faster
2. According to the Stefan-Boltzmann law, dense objects (blackbodies) emit _____ E-M
radiation as their temperatures rise.
A. more
B. less
C. longer λ
D. shorter λ
E. faster
3. A star with an absolute (K) temperature three times that of another will emit _____ times as
much E-M radiation.
A. 3
B. 9
C. 27
D. 81
E. indeterminate, cannot tell
4. Absorption spectra, which are seen as dark lines on continuous spectra, are produced by
_____.
A. blackbodies
B. hot low-density gases
C. cool low-density gases
D. gravity


E. dark matter
5. Emission spectra, which are seen as discrete bright lines on dark backgrounds, are produced
by _____.
A. blackbodies
B. hot low-density gases
C. cool low-density gases
D. gravity
E. dark matter
6. Continuous spectra, which include all the colors of the rainbow, are produced by:
A.
B.
C.
D.
E.
blackbodies
hot low-density gases
cool low-density gases
gravity
dark matter
7. The nucleus of an atom is composed of _____.
A. protons and electrons
B. protons and neutrons
C. neutrons and electrons
D. positrons and neutrons
E. positrons and electrons
8. Electrons orbit at relatively great distances from the nucleus, and those in the closest orbits
have _____ energies.
A. the highest
B. intermediate
C. the lowest
D. the most variable
E. the least measureable
9. Electrons can only jump to a higher energy level if they _____ a photon with energy equal to
the gap between the levels.
A. absorb
B. emit
C. radiate
D. collide with
E. lose
10. In a cool low-density gas, most electrons are in _____ energy levels.
A. a wide variety of
B. their highest possible
C. intermediate
D. their lowest possible
E. scattered uniformly among different


%% NOTE: diagram graph
11. Consider an atom in which two electrons jump to lower energy levels. The first jumps from
level 3 to 1 and the second from level 2 to 1. Which emits a
photon of shorter wavelength light in order to make its
jump?
A. first electron
B. second electron
C. both emit same wavelength photons
D. both absorb photons to make these jumps
E. indeterminate, cannot tell
12. When an object emitting light moves away from us, its
light _____.
A. is shifted to shorter wavelengths (blueshifted)
B. is shifted to longer wavelengths (redshifted)
C. becomes brighter
D. moves faster
E. moves more slowly
13. The larger the blueshift of an object, the _____ it is moving _____ us.
A. faster, away from
B. slower, away from
C. faster, towards
D. slower, towards
E. more steadily, towards
14. Consider the two absorption spectra shown below; they are for the same element. The top
one is from a stationary object and the bottom one is from a moving object. By comparing
them, you can tell that the second object is moving _____.
A. towards us
B. away from us
C. neither towards nor away (is stationary)
D. faster than light
E. you can’t really tell about its motion


\endinput



