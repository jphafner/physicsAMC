
%% University of Northern Iowa
%%  Morgans Astronomy Exams
%%--------------------------------------------------

%% this section contains 15 problems

\element{morgans}{
\begin{question}{test3D-Q01}
    Cepheid variable stars are important to astronomy primarily because they allow us to determine stellar:
    \begin{choices}
        \wrongchoice{distance.}
      \correctchoice{mass.}
        \wrongchoice{temperature.}
        \wrongchoice{pulsation period.}
        \wrongchoice{chemical composition.}
    \end{choices}
\end{question}
}

\element{morgans}{
\begin{question}{test3D-Q02}
    Early astronomers trying to determine the size and extent of the galaxy by counting stars in all directions,
        would not get the correct shape and size because:
    \begin{choices}
        \wrongchoice{they were unable to determine the temperatures of stars.}
      \correctchoice{they were unaware of the absorption (dimming) of starlight by interstellar dust.}
        \wrongchoice{accurate stellar masses were unavailable.}
        \wrongchoice{they were a bunch of idiots.}
    \end{choices}
\end{question}
}

\element{morgans}{
\begin{question}{test3D-Q03}
    The primary use of observations of 21-cm radiation is to determine:
    \begin{choices}
        \wrongchoice{distances to distant stars.}
        \wrongchoice{the chemical composition of the interstellar medium.}
      \correctchoice{the distribution of hydrogen in the Galaxy.}
        \wrongchoice{the distance of the Sun from the center of the Galaxy.}
    \end{choices}
\end{question}
}

\element{morgans}{
\begin{question}{test3D-Q04}
    Which of the following would NOT be useful in locating the galaxy's spiral structure?
    \begin{choices}
        \wrongchoice{O Stars}
        \wrongchoice{H II regions}
      \correctchoice{M Stars}
        \wrongchoice{Supernova remnants}
        \wrongchoice{Giant molecular clouds}
    \end{choices}
\end{question}
}

\element{morgans}{
\begin{question}{test3D-Q05}
    Where would you expect to find the highest concentration of heavy elements in the interstellar medium?
    \begin{choices}
        \wrongchoice{amongst stars in globular clusters}
        \wrongchoice{amongst stars in the galactic center}
      \correctchoice{near the spiral arms}
        \wrongchoice{in the regions far from the spiral arms}
    \end{choices}
\end{question}
}

\element{morgans}{
\begin{question}{test3D-Q06}
    Important differences between stars of Pop. I and Pop. II are that:
    \begin{choices}
        \wrongchoice{Pop. II stars are young and move slowly around the Galaxy.}
        \wrongchoice{Pop. II stars are metal poor and young.}
      \correctchoice{Pop. II stars are metal poor stars in the galactic halo.}
        \wrongchoice{Pop. II stars are slowly moving stars in the galactic disc.}
        \wrongchoice{Pop. II stars are young, metal rich stars in the galactic nucleus.}
    \end{choices}
\end{question}
}

\element{morgans}{
\begin{question}{test3D-Q07}
    A galaxy with a large nucleus and tight spiral arms would be classified by Hubble as:
    \begin{choices}
      \correctchoice{Sa.}
        \wrongchoice{Sb.}
        \wrongchoice{Sc.}
        \wrongchoice{elliptical.}
        \wrongchoice{irregular.}
    \end{choices}
\end{question}
}

\element{morgans}{
\begin{question}{test3D-Q08}
    Kepler's 3rd law as modified by Newton is generally used to find a galaxy's:
    \begin{choices}
        \wrongchoice{distance.}
      \correctchoice{mass.}
        \wrongchoice{velocity.}
        \wrongchoice{rotation speed.}
        \wrongchoice{radius.}
    \end{choices}
\end{question}
}

\element{morgans}{
\begin{question}{test3D-Q09}
    Galaxies in clusters:
    \begin{choices}
        \wrongchoice{tend to move randomly within the cluster, and eventually will leave the cluster.}
        \wrongchoice{do not orbit, but are fixed in space.}
      \correctchoice{orbit about the center of the cluster.}
        \wrongchoice{often merge in the center to form a giant spiral galaxy.}
    \end{choices}
\end{question}
}

\element{morgans}{
\begin{question}{test3D-Q10}
    The Local Group consists of about \rule[-0.1pt]{4em}{0.1pt} members.
    \begin{choices}
        %% NOTE: spell out!?
        \wrongchoice{2}
        \wrongchoice{10}
      \correctchoice{40}
        \wrongchoice{80}
        \wrongchoice{100}
    \end{choices}
\end{question}
}

\element{morgans}{
\begin{question}{test3D-Q11}
    The structure of the Universe is most like that of:
    \begin{choices}
        \wrongchoice{a solid brick, with no gaps or variations.}
        \wrongchoice{a pizza, with most material located in a disk.}
      \correctchoice{a gigantic Swiss cheese, with big gaps between material.}
        \wrongchoice{a sack full of marbles, with little gaps between material.}
    \end{choices}
\end{question}
}

\element{morgans}{
\begin{question}{test3D-Q12}
    Even though the Andromeda galaxy is predicted to have a velocity of 40 km/s away from us,
        its has a motion of 100 km/s towards us. 
    What produces this motion?
    \begin{choices}
        \wrongchoice{the Andromeda galaxy is misnamed - it is really a gas cloud located in our own galaxy}
      \correctchoice{random motions within the cluster are larger than the motions produced by the expansion of the universe at this relatively nearby distance}
        \wrongchoice{the Andromeda galaxy is so far away that it is beyond the limits of the effectiveness of Hubble constant}
        \wrongchoice{the information in the question is incorrect, the Andromeda galaxy is moving away from us}
    \end{choices}
\end{question}
}

\element{morgans}{
\begin{question}{test3D-Q13}
    The temperature of the background radiation is nearly:
    \begin{choices}
      \correctchoice{3 K.}
        \wrongchoice{10 K.}
        \wrongchoice{30 K.}
        \wrongchoice{100 K.}
        \wrongchoice{1,000 K.}
    \end{choices}
\end{question}
}

\element{morgans}{
\begin{question}{test3D-Q14}
    One of the space probes launched by NASA to investigate the microwave background radiation is called:
    \begin{choices}
        \wrongchoice{Mariner.}
        \wrongchoice{IUE.}
      \correctchoice{COBE.}
        \wrongchoice{Galileo.}
        \wrongchoice{Ulysses.}
    \end{choices}
\end{question}
}

\element{morgans}{
\begin{question}{test3D-Q15}
    If an object is observed to be emitting synchrotron radiation,
        we may conclude that:
    \begin{choices}
        \wrongchoice{it has a very high temperature.}
        \wrongchoice{it has a very low temperature.}
        \wrongchoice{the object is moving near the speed of light.}
      \correctchoice{the object has a strong magnetic field.}
        \wrongchoice{the object has no magnetic field.}
    \end{choices}
\end{question}
}

\element{morgans}{
\begin{question}{test3D-Q16}
    A value for a redshift of 4.85 for a quasar indicates:
    \begin{choices}
        \wrongchoice{that the quasar is traveling 485 times the speed of light away from us.}
        \wrongchoice{that the quasar is traveling 4.85 times the speed of light away from us.}
        \wrongchoice{that the quasar is traveling 4.85 times the speed of light towards us.}
      \correctchoice{that the quasar is traveling at relativistic velocities, close to, but not greater than the speed of light.}
    \end{choices}
\end{question}
}

\element{morgans}{
\begin{question}{test3D-Q17}
    Quasar absorption lines in what is called the ``Lyman alpha forest'' generally:
    \begin{choices}
        \wrongchoice{have the same redshift as the emission lines.}
        \wrongchoice{have greater redshifts than the emission lines.}
      \correctchoice{have smaller redshifts compared to the emission lines.}
        \wrongchoice{are formed by clouds in our galaxy.}
        \wrongchoice{have a blueshift equal to the emission line redshift.}
    \end{choices}
\end{question}
}

\element{morgans}{
\begin{question}{test3D-Q18}
    The critical density is roughly:
    \begin{choices}
        \wrongchoice{the density of air at sea level on Earth.}
      \correctchoice{about 3 protons per cubic meter.}
        \wrongchoice{the density of a white dwarf.}
        \wrongchoice{that of the interstellar medium---1 proton per cubic centimeter.}
        \wrongchoice{the density at the center of the Sun.}
    \end{choices}
\end{question}
}

\element{morgans}{
\begin{question}{test3D-Q19}
    When we observe distant galaxies we are observing:
    \begin{choices}
      \correctchoice{very young objects.}
        \wrongchoice{very old objects.}
        \wrongchoice{objects having the approximate age of the Milky Way.}
        \wrongchoice{distant galaxies; no statement may be made about age.}
    \end{choices}
\end{question}
}

\element{morgans}{
\begin{question}{test3D-Q20}
    If a galaxy is 100 Mpc away and has a velocity away from us of 5000 km/s then the value of the Hubble Constant based on these objects is:
    \begin{choices}
        \wrongchoice{1/2 km/s/Mpc}
        \wrongchoice{20 km/s/Mpc}
        \wrongchoice{10 km/s/Mpc}
      \correctchoice{50 km/s/Mpc}
    \end{choices}
\end{question}
}

\begin{comment}
    Fill In
    Place the most appropriate word or words in the blank. You may have to click on the blank to activate it before you start typing in your answer.
     
    The specific name given to massive, luminous stars having a relation between their period of light variation and their absolute magnitude is .

    mapped out the location and direction of the Galactic center using globular clusters.

    Globular clusters are located in the part of the galaxy known as the .

    The Magellanic Clouds are classified as galaxies.

    Two irregular galaxies, considered to be satellites of the Milky Way are the .

    The expansion of the universe was discovered by .

    is the space probe launched by NASA to investigate the microwave background radiation.

    Emission line objects thought to exist only in the very early universe are called .

    is a radio source galaxy with one very long jet projecting from it. Recent Hubble telescope observations indicate that there is a large black hole in its core.

    The problem as to why the night sky is dark rather than light is known as
\end{comment}


\endinput


