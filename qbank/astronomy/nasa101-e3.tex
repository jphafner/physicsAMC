
%% NASA Astronomy 101 Exam 3
%%--------------------------------------------------

\element{nasa}{
\begin{question}{exam03a2-q01}
    Which of the following sequences of spectral types represent the coolest to hottest stars?
    \begin{choices}
        \wrongchoice{MKGFABO }
        \wrongchoice{ABFGKMO}
        \wrongchoice{OMKGFBA}
        \wrongchoice{OBAFGKM}
        \wrongchoice{MFKGABO}
    \end{choices}
\end{question}
}

\element{nasa}{
\begin{question}{exam03a2-q02}
    Chili Pepper is a main sequence star that appears red when viewed from Earth,
        as a result which of the following is (always) true: 
    \begin{choices}
        \wrongchoice{It is hotter than an O spectral type main sequence star.  }
        \wrongchoice{It will appear dimmer than a white dwarf.}
        \wrongchoice{It will live longer than a B spectral type main sequence star.}
        \wrongchoice{It is the same size as a red giant star of the same temperature.}
        \wrongchoice{None of the above are correct.}
    \end{choices}
\end{question}
}

\element{nasa}{
\begin{question}{exam03a2-q03}
    Which of the following was the planet that formed at the nearest location to the Sun that was at a temperature below the freezing point of water?  
    \begin{multicols}{2}
    \begin{choices}
        \wrongchoice{Mercury}
        \wrongchoice{Mars}
        \wrongchoice{Jupiter}
        \wrongchoice{Saturn}
        \wrongchoice{Neptune}
    \end{choices}
    \end{multicols}
\end{question}
}

\element{nasa}{
\begin{question}{exam03a2-q04}
    Which one property of a star will determine the rest of the characteristics of that star’s life?
    \begin{multicols}{2}
    \begin{choices}
        \wrongchoice{luminosity}
        \wrongchoice{temperature}
        \wrongchoice{color}
        \wrongchoice{mass}
        \wrongchoice{spectral type}
    \end{choices}
    \end{multicols}
\end{question}
}

\element{nasa}{
\begin{question}{exam03a2-q05}
    %% COPIED 
    Listed below are several astronomical objects (1-6).  
    Which is the correct ranking for the diameter of these objects,
        from largest to smallest?
    \begin{enumerate}
        \item Neptune
        \item a one-solar-mass white dwarf
        \item a three-solar-mass neutron star
        \item Mercury
        \item M5 spectral type main sequence star
        \item a ten-solar-mass black hole 
    \end{enumerate}
    \begin{multicols}{2}
    \begin{choices}
        \wrongchoice{$6>3>2>5>1>4$}
        \wrongchoice{$5>1>4>2>3>6$}
        \wrongchoice{$3=6>2>5>1>4$}
        \wrongchoice{$1>4>6>3>2>5$}
        \wrongchoice{$5>1>2>4>3>6$}
    \end{choices}
    \end{multicols}
\end{question}
}

\element{nasa}{
\begin{question}{exam03a2-q06}
    The HR Diagram at right is provided to assist with answering the following two questions.
    \begin{center}
    \begin{tikzpicture}
        %% NOTE: luminiosity vs temperature (red giants, white dwarfs, main sequence)
    \end{tikzpicture}
    \end{center}
    Which is hotter,
        a main sequence star with an absolute magnitude of $M=8$ or a white dwarf with a luminosity 100 times smaller than the Sun?
    \begin{multicols}{2}
    \begin{choices}
        \wrongchoice{the main sequence star}
        \wrongchoice{the white dwarf}
        \wrongchoice{They have the same temperature.}
        \wrongchoice{There is insufficient information to determine this.}
    \end{choices}
    \end{multicols}
\end{question}
}

\element{nasa}{
\begin{question}{exam03a2-q07}
    The HR Diagram at right is provided to assist with answering the following two questions.
    \begin{center}
    \begin{tikzpicture}
        %% NOTE: luminiosity vs temperature (red giants, white dwarfs, main sequence)
    \end{tikzpicture}
    \end{center}
    Which statement is the most correct about the comparison between a spectral type G5 main sequence star and a spectral type O5 main sequence star?
    \begin{choices}
        \wrongchoice{The G5 star is cooler, dimmer, smaller, and will not live as long as the O5 main sequence star. }
        \wrongchoice{The G5 star is hotter, brighter, smaller, and will live longer than the O5 main sequence star.}
        \wrongchoice{The G5 star is smaller, hotter, brighter, and will not live as long as the O5 main sequence star.}
        \wrongchoice{The G5 star is dimmer, cooler, smaller, and will live longer than the O5 main sequence star.}
        \wrongchoice{The G5 star is hotter, dimmer, larger, and will live longer than the O5 main sequence star.}
    \end{choices}
\end{question}
}

\element{nasa}{
\begin{question}{exam03a2-q08}
    How does the Sun produce the energy that heats our planet?
    \begin{choices}
        \wrongchoice{The gases inside the Sun are burning and producing large amounts of energy.}
        \wrongchoice{Gas inside the Sun heats up when compressed, giving off large amounts of energy.}
        \wrongchoice{Heat trapped by magnetic fields in the Sun is released as energy.}
        \wrongchoice{Hydrogen is combined into helium, giving off large amounts of energy.}
        \wrongchoice{The core of the Sun has radioactive atoms that give off energy as they decay. }
    \end{choices}
\end{question}
}

\element{nasa}{
\begin{question}{exam03a2-q09}
    Star A has a mass of 4 solar masses and Star B has a mass of 8 solar masses. 
    How will the fusion rate of Star A compare to the fusion rate of Star B? 
    \begin{choices}
        \wrongchoice{Star A’s fusion rate will be more than two times slower than that of Star B.}
        \wrongchoice{Star A’s fusion rate will be two times slower than that of Star B.}
        \wrongchoice{Star A’s fusion rate will be the same as that of Star B.}
        \wrongchoice{Star A’s fusion rate will be two times faster than that of Star B.}
        \wrongchoice{Star A’s fusion rate will be more than two times faster than that of Star B.}
    \end{choices}
\end{question}
}

\element{nasa}{
\begin{question}{exam03a2-q10}
    Goofy Star has an absolute magnitude of +8.0 and belongs to spectral type K.  
    Daffy Star has an absolute magnitude of -2.0 and belongs to spectral type K. 
    Use this information to answer the following two questions.
    %% start question
    Which of the following is true about the color of the two stars?
    \begin{choices}
        \wrongchoice{Goofy Star would appear blue.}
        \wrongchoice{Daffy Star would appear blue.}
        \wrongchoice{They would both appear the same color.}
        \wrongchoice{There is not enough information to determine this.}
    \end{choices}
\end{question}
}

\element{nasa}{
\begin{question}{exam03a2-q11}
    Goofy Star has an absolute magnitude of +8.0 and belongs to spectral type K.  
    Daffy Star has an absolute magnitude of -2.0 and belongs to spectral type K. 
    Use this information to answer the following two questions.
    %% start question
    Which star has the largest surface area? 
    \begin{choices}
        \wrongchoice{Goofy Star }
        \wrongchoice{Daffy Star }
        \wrongchoice{They have the same surface area.}
        \wrongchoice{There is not enough information to determine this.}
    \end{choices}
\end{question}
}

\element{nasa}{
\begin{question}{exam03a2-q12}
    If Earth’s atmosphere were able to completely absorb visible light,
        which of the following would be true? 
    \begin{choices}
        \wrongchoice{The Earth’s surface temperature would be cooler than it is today.}
        \wrongchoice{The Earth’s surface temperature would be warmer than it is today.}
        \wrongchoice{The Earth’s surface temperature would be the same temperature as it is today.}
        \wrongchoice{There is not enough information to answer this question.}
    \end{choices}
\end{question}
}

\element{nasa}{
\begin{question}{exam03a2-q13}
    How many Moon orbital diameters would fit across a star five times larger than the Sun?
    \begin{multicols}{2}
    \begin{choices}
        \wrongchoice{\num{5}}
        \wrongchoice{\num{10}}
        \wrongchoice{\num{20}}
        \wrongchoice{\num{50}}
        \wrongchoice{\num{550}}
    \end{choices}
    \end{multicols}
\end{question}
}

\element{nasa}{
\begin{question}{exam03a2-q14}
    From which binary star system (made up of two main sequence stars)
        described below would an Earth observer detect the least amount of total light? 
    \begin{choices}
        \wrongchoice{When a star with an absolute magnitude of 8.0 is in front of a star with an absolute magnitude of -2.0.}
        \wrongchoice{When a star with an absolute magnitude of 2.0 is behind a star with an absolute magnitude of -2.0.}
        \wrongchoice{When a star with an absolute magnitude of 8.0 is behind a star with an absolute magnitude of 2.0.}
        \wrongchoice{When a star with an absolute magnitude of 8.0 is in front of a star with an absolute magnitude of 2.0.}
        \wrongchoice{When a star with an absolute magnitude of 8.0 is behind a star with an absolute magnitude of -2.0.}
    \end{choices}
\end{question}
}

\element{nasa}{
\begin{question}{exam03a2-q15}
    A star on the lower right-hand part of the main sequence will \rule[-0.1pt]{4em}{0.1pt} than a star on the upper left-hand part of the main sequence. 
    \begin{choices}
        \wrongchoice{have a shorter lifetime.}
        \wrongchoice{be hotter.}
        \wrongchoice{be larger.}
        \wrongchoice{be more luminous.}
        \wrongchoice{be less massive}
    \end{choices}
\end{question}
}

\element{nasa}{
\begin{question}{exam03a2-q16}
    Astronomers have discovered massive gas giant planets like Jupiter orbiting their companion stars at closer than 0.7 AU (about the distance of Venus’s orbit).
    Why don't astronomers believe that these gas giant planets originally formed at these locations? 
    \begin{choices}
        \wrongchoice{The planets' gravity would have been too large to form that close to the star.}
        \wrongchoice{The temperature was too high at this distance from the star for gas giant planets to form.}
        \wrongchoice{Their orbital periods are too long for them to be located that close to their companion stars.}
        \wrongchoice{A young star's solar wind would have blown the planets farther away.}
    \end{choices}
\end{question}
}

\element{nasa}{
\begin{question}{exam03a2-q17}
    Which of the following is a primary characteristic of greenhouse gases?
    \begin{choices}
        \wrongchoice{They absorb more molecules in the atmosphere than they give off.}
        \wrongchoice{They concentrate sunlight as it travels through the atmosphere.}
        \wrongchoice{They can completely trap forms of light in the atmosphere.}
        \wrongchoice{They absorb some forms of light but allow other forms of light to pass through.}
        \wrongchoice{They absorb more light than they give off.}
    \end{choices}
\end{question}
}

\element{nasa}{
\begin{question}{exam03a2-q18}
    Which of the following lists, in the correct order,
        a possible evolutionary path for a star?
    \begin{choices}
        \wrongchoice{Main Sequence Star, Red Giant, Planetary Nebula, White Dwarf}
        \wrongchoice{Main Sequence Star, Red Giant, Neutron Star, White Dwarf, Nothing}
        \wrongchoice{Main Sequence Star, Red Giant, Type I Supernova, Black Hole}
        \wrongchoice{Main Sequence Star, Red Giant, Type II Supernova, Planetary Nebula, Neutron Star}
        \wrongchoice{Main Sequence Star, Red Giant, Planetary Nebula, Black Hole}
    \end{choices}
\end{question}
}

\element{nasa}{
\begin{question}{exam03a2-q19}
    For a white dwarf to become a nova,
        it is necessary for it to:
    \begin{choices}
        \wrongchoice{have a companion }
        \wrongchoice{become a black hole. }
        \wrongchoice{have begun life as a high-mass star. }
        \wrongchoice{expand into a giant. }
    \end{choices}
\end{question}
}

\element{nasa}{
\begin{question}{exam03a2-q20}
    Which one of the planets listed below initially formed at the outermost location where the temperature was high enough for water to be a liquid?
    \begin{multicols}{2}
    \begin{choices}
        \wrongchoice{Earth}
        \wrongchoice{Mars}
        \wrongchoice{Jupiter}
        \wrongchoice{Saturn}
        \wrongchoice{Neptune}
    \end{choices}
    \end{multicols}
\end{question}
}

\element{nasa}{
\begin{question}{exam03a2-q21}
    Jupiter is five times farther from the Sun than Earth.  
    Approximately how many Suns would fit between the Sun and Jupiter?
    \begin{multicols}{3}
    \begin{choices}
         \wrongchoice{5}
         \wrongchoice{10}
         \wrongchoice{50}
         \wrongchoice{150}
         \wrongchoice{550}
    \end{choices}
    \end{multicols}
\end{question}
}

\element{nasa}{
\begin{question}{exam03a2-q22}
    The star Moto has an apparent magnitude of $-1.0$ and an absolute magnitude of $+3.0$.
    If it were moved 10 times farther from Earth as it is now,
        which one of the following would occur?
    \begin{choices}
        \wrongchoice{absolute magnitude number would decrease (gets smaller)}
        \wrongchoice{apparent magnitude number would decrease (gets smaller)}
        \wrongchoice{apparent magnitude number would stay the same}
        \wrongchoice{absolute magnitude number would increase (gets bigger)}
        \wrongchoice{apparent magnitude number would increase (gets bigger)}
    \end{choices}
\end{question}
}

\element{nasa}{
\begin{question}{exam03a2-q23}
    Which of the graphs (A--D) would correspond with a B spectral type main sequence star orbiting a K spectral type Red Giant star in a binary star system?
    If none of the graphs is correct,
        bubble in an ``e''.
    \begin{multicols}{2}
    \begin{choices}
        \wrongchoice{
            \begin{tikzpicture}
                %% NOTE: 
            \end{tikzpicture}
        }
    \end{choices}
    \end{multicols}
\end{question}
}

\element{nasa}{
\begin{question}{exam03a2-q24}
    Approximately how many Moons would fit between the Earth and the Moon?
    \begin{multicols}{3}
    \begin{choices}
        \wrongchoice{30}
        \wrongchoice{50}
        \wrongchoice{120}
        \wrongchoice{150}
        \wrongchoice{220}
    \end{choices}
    \end{multicols}
\end{question}
}

\element{nasa}{
\begin{question}{exam03a2-q25}
    Which of the following statements is always true of any two stars
        (including Red Giants and White Dwarfs) that have the same absolute magnitude? 
    \begin{choices}
        \wrongchoice{They have the same temperature.}
        \wrongchoice{They have the same luminosity.}
        \wrongchoice{They have the same spectral type.}
        \wrongchoice{They have the same surface area.}
        \wrongchoice{They have the same mass.}
    \end{choices}
\end{question}
}

\element{nasa}{
\begin{question}{exam03a2-q26}
    The sketches below illustrates how two main sequence stars might look at three different times.
    Use this set of sketches to answer the next two questions. 
    %% text slant or italics?
    Note: The sketch with the small circle shown with dashed lines illustrates the time when the smaller star was located behind the larger star.
    %% NOTE: easy tikzpicutre graphics
    %% start questions
    In which case shown would the amount of light we would observe from Earth be the least.  
    \begin{choices}
        \wrongchoice{at time A}
        \wrongchoice{at time B}
        \wrongchoice{at time C}
        \wrongchoice{At more than one of the times.}
        \wrongchoice{There is not enough information to determine this.}
    \end{choices}
\end{question}
}

\element{nasa}{
\begin{question}{exam03a2-q27}
    The sketches below illustrates how two main sequence stars might look at three different times.
    Use this set of sketches to answer the next two questions. 
    %% text slant or italics?
    Note: The sketch with the small circle shown with dashed lines illustrates the time when the smaller star was located behind the larger star.
    %% NOTE: easy tikzpicutre graphics
    %% start questions
    In which case shown would the amount of light we would observe from Earth be the greatest.  
    \begin{choices}
        \wrongchoice{at time A}
        \wrongchoice{at time B}
        \wrongchoice{at time C}
        \wrongchoice{At more than one of the times.}
        \wrongchoice{There is not enough information to determine this.}
    \end{choices}
\end{question}
}

\element{nasa}{
\begin{question}{exam03a2-q28}
    The star Aprilia is 3 parsecs away.  
    Its apparent magnitude is $+3.0$.
    What is most likely its absolute magnitude?
    \begin{multicols}{3}
    \begin{choices}
        \wrongchoice{$+6.0$}
        \wrongchoice{$+3.0$}
        \wrongchoice{$+0.30$}
        \wrongchoice{$-3.0$}
        \wrongchoice{$-6.0$}
    \end{choices}
    \end{multicols}
\end{question}
}

\element{nasa}{
\begin{question}{exam03a2-q29}
    Nismo star is a G spectral type star that is 1000 times more luminous than the Sun.  Nopi star has the same absolute magnitude as the Sun and belongs to spectral type G.  Which star has the greatest surface temperature? 
    \begin{choices}
        \wrongchoice{Nismo}
        \wrongchoice{Nopi}
        \wrongchoice{They have the same temperature.}
        \wrongchoice{There is insufficient information to determine this.}
    \end{choices}
\end{question}
}

\element{nasa}{
\begin{question}{exam03a2-q30}
    Cutter is a star with an apparent magnitude of +2.3 and an absolute magnitude of -1.1.  
    Casper is a star with an apparent magnitude of -3.7 and an absolute magnitude of -2.3.  
    Answer the following four questions using this information.
    %% start question
    Which of the stars described above gives off more light?
    \begin{choices}
        \wrongchoice{Casper}
        \wrongchoice{Cutter}
        \wrongchoice{They give off the same amount of light.}
        \wrongchoice{There is insufficient information to determine this.}
    \end{choices}
\end{question}
}

\element{nasa}{
\begin{question}{exam03a2-q31}
    Cutter is a star with an apparent magnitude of +2.3 and an absolute magnitude of -1.1.  
    Casper is a star with an apparent magnitude of -3.7 and an absolute magnitude of -2.3.  
    Answer the following four questions using this information.
    %% start question
    Which of these stars is closer to Earth?
    \begin{choices}
        \wrongchoice{Casper}
        \wrongchoice{Cutter}
        \wrongchoice{They are the same distance from Earth.}
        \wrongchoice{There is insufficient information to determine this.}
    \end{choices}
\end{question}
}

\element{nasa}{
\begin{question}{exam03a2-q32}
    Cutter is a star with an apparent magnitude of +2.3 and an absolute magnitude of -1.1.  
    Casper is a star with an apparent magnitude of -3.7 and an absolute magnitude of -2.3.  
    Answer the following four questions using this information.
    %% start question
    Which of these stars would look brighter in the night sky?
    \begin{choices}
        \wrongchoice{Casper}
        \wrongchoice{Cutter}
        \wrongchoice{They would appear equally bright form Earth.}
        \wrongchoice{There is insufficient information to determine this.}
    \end{choices}
\end{question}
}

\element{nasa}{
\begin{question}{exam03a2-q33}
    Cutter is a star with an apparent magnitude of +2.3 and an absolute magnitude of -1.1.  
    Casper is a star with an apparent magnitude of -3.7 and an absolute magnitude of -2.3.  
    Answer the following four questions using this information.
    %% start question
    If both of these stars are main sequence stars,
        which star is hotter?
    \begin{choices}
        \wrongchoice{Casper}
        \wrongchoice{Cutter}
        \wrongchoice{They would be the same temperature.}
        \wrongchoice{There is insufficient information to determine this.}
    \end{choices}
\end{question}
}

\element{nasa}{
\begin{question}{exam03a2-q34}
    Which of the following is most likely the energy output versus wavelength graph for the energy given off by the person sitting next to you taking this test? 
    \begin{multicols}{2}
    \begin{choices}
        \wrongchoice{
            \begin{tikzpicture}
                %% NOTE:
            \end{tikzpicture}
        }
    \end{choices}
    \end{multicols}
\end{question}
}

\element{nasa}{
\begin{question}{exam03a2-q35}
    Consider the information given below about three main sequence stars A, B, and C.
    \begin{itemize}
        \item Star A will be a main sequence star for 4.5 billion years.
        \item Star B has the same luminosity as the Sun.
        \item Star C has a spectral type of M5.
    \end{itemize}
    Which of the following is a true statement about these stars?
    \begin{choices}
        \wrongchoice{Star A has the greatest mass.}
        \wrongchoice{Star B has the greatest mass.}
        \wrongchoice{Star C has the greatest mass.}
        \wrongchoice{Stars A, B and C all have approximately the same mass.}
        \wrongchoice{There is insufficient information to determine this. }
    \end{choices}
\end{question}
}

\element{nasa}{
\begin{question}{exam03a2-q36}
    Which of the following is part of the Earth's natural greenhouse effect? 
    \begin{choices}
        \wrongchoice{Earth's atmosphere continually becomes thicker with greenhouse gases. }
        \wrongchoice{Infrared light becomes permanently trapped in our atmosphere by greenhouse gasses.}
        \wrongchoice{The ozone hole causes significant increases in surface temperature.}
        \wrongchoice{Earth’s surface and atmospheric gases absorb energy and then give off infrared light.}
        \wrongchoice{Heat is transferred in the atmosphere through the circulation of greenhouse gasses.}
    \end{choices}
\end{question}
}

\element{nasa}{
\begin{question}{exam03a2-q37}
    Black holes are formed by:
    \begin{choices}
        \wrongchoice{a lack of any light in a region of space.}
        \wrongchoice{supernovae from the most massive stars.}
        \wrongchoice{supernovae from white dwarfs stars.}
        \wrongchoice{collapsed dark nebulae.}
    \end{choices}
\end{question}
}

\element{nasa}{
\begin{question}{exam03a2-q38}
    If you were constructing a scale model of the solar system that used an Earth that was the size of a large beach ball (3 feet), which of the following sizes would most closely approximate the scaled size of the Sun in your model? 
    \begin{choices}
        \wrongchoice{10 feet (height of a basketball goal)}
        \wrongchoice{30 feet (length of a school bus)}
        \wrongchoice{100 feet (height of an 8 story building)}
        \wrongchoice{300 feet (length of a football field)}
    \end{choices}
\end{question}
}

\element{nasa}{
\begin{question}{exam03a2-q39}
    When would you receive the least amount of light from a binary star system consisting of a M5 Red Giant and an M5 main sequence star?
    \begin{choices}
        \wrongchoice{When the Red Giant is in front of the main sequence star.}
        \wrongchoice{When the main sequence star is in front of the Red Giant.}
        \wrongchoice{You would receive the same amount of light for both situations described in choices “a” and “b”.}
    \end{choices}
\end{question}
}

\element{nasa}{
\begin{question}{exam03a2-q40}
    Earth's surface is primarily heated by which two forms of energy? 
    \begin{choices}
        \wrongchoice{ultraviolet and visible }
        \wrongchoice{ultraviolet and infrared }
        \wrongchoice{x-ray and ultraviolet }
        \wrongchoice{visible and radio }
        \wrongchoice{visible and infrared}
    \end{choices}
\end{question}
}

\element{nasa}{
\begin{question}{exam03a2-q41}
    In a main sequence star,
        gravitational collapse is balanced by:
    \begin{choices}
        \wrongchoice{convection of stellar material from the core.}
        \wrongchoice{pressure caused by energy produced during nuclear fusion.}
        \wrongchoice{solid material at the stellar core.}
        \wrongchoice{pressure from coronal mass ejections from the core.}
        \wrongchoice{interior cooling of the star.}
    \end{choices}
\end{question}
}

\element{nasa}{
\begin{question}{exam03a2-q42}
    An isolated star with twice the mass of the Sun will eventually
    \begin{choices}
        \wrongchoice{become a steadily cooling white dwarf.}
        \wrongchoice{collapse into black hole.}
        \wrongchoice{form a neutron star.}
        \wrongchoice{explode as a Type Ia supernova, leaving no remnant.}
    \end{choices}
\end{question}
}

\element{nasa}{
\begin{question}{exam03a2-q43}
    If ten Jupiters can fit across the surface (width) of the Sun,
        then approximately how many Earth's could fit across Jupiter?
    \begin{multicols}{2}
    \begin{choices}
        \wrongchoice{2}
        \wrongchoice{5}
        \wrongchoice{10}
        \wrongchoice{50}
        \wrongchoice{100}
    \end{choices}
    \end{multicols}
\end{question}
}

\element{nasa}{
\begin{question}{exam03a2-q44}
    If the energy output of the Sun were to change so that it produced the same amount of visible light as it currently does ultraviolet light,
        which of the following would be true? 
    \begin{choices}
        \wrongchoice{The Earth's surface temperature would be the same temperature as it is today.}
        \wrongchoice{The Earth's surface temperature would be cooler than it is today.}
        \wrongchoice{The Earth's surface temperature would be warmer than it is today.}
    \end{choices}
\end{question}
}

\element{nasa}{
\begin{question}{exam03a2-q45}
    Which of the following is true of a binary star system consisting of a Red Giant and a White Dwarf?
    \begin{choices}
        \wrongchoice{You will receive more light when the dwarf is behind the giant than when the giant is behind the dwarf.}
        \wrongchoice{The time it takes for the dwarf to pass behind the giant is shorter than the time for the giant to pass behind the dwarf.}
        \wrongchoice{The force of gravity exerted on the dwarf by the giant is stronger than the force of gravity exerted on the giant by the dwarf.}
        \wrongchoice{The orbital period of the dwarf is shorter than the orbital period of the giant.}
        \wrongchoice{None of the above.}
    \end{choices}
\end{question}
}

\element{nasa}{
\begin{question}{exam03a2-q46}
    %% COPIED
    During the beginning of star formation,
        the force that dominates the collapse of a cloud of gas and dust is:
    \begin{choices}
        \wrongchoice{electrostatic.}
        \wrongchoice{gravity.}
        \wrongchoice{magnetism.}
        \wrongchoice{friction.}
    \end{choices}
\end{question}
}

\element{nasa}{
\begin{question}{exam03a2-q47}
    How many planets (not including Pluto) formed at locations in the early solar nebula at temperatures cooler than your body temperature?
    \begin{choices}
        \wrongchoice{Only one}
        \wrongchoice{Two planets}
        \wrongchoice{Three planets}
        \wrongchoice{Four planets}
        \wrongchoice{More than four planets}
    \end{choices}
\end{question}
}

\element{nasa}{
\begin{question}{exam03a2-q48}
    The mass of three clouds of gas and dust are provided below. 
    Imagine that each cloud will collapse to form one star. 
    Use this information to answer the next two questions. 
    \begin{description}
        \item[Cloud X:] 60 times the mass of the Sun
        \item[Cloud Y:]  7 times the mass of the Sun
        \item[Cloud Z:]  2 times the mass of the Sun
    \end{description}
    %% start questions
    Which is most likely to form into red main-sequence star?
    \begin{choices}
        \wrongchoice{Cloud X}
        \wrongchoice{Cloud Y}
        \wrongchoice{Cloud Z}
        \wrongchoice{More than one of the clouds could form into a red main-sequence star.}
        \wrongchoice{None of the clouds will form into a red main-sequence star.}
    \end{choices}
\end{question}
}

\element{nasa}{
\begin{question}{exam03a2-q49}
    The mass of three clouds of gas and dust are provided below. 
    Imagine that each cloud will collapse to form one star. 
    Use this information to answer the next two questions. 
    \begin{description}
        \item[Cloud X:] 60 times the mass of the Sun
        \item[Cloud Y:]  7 times the mass of the Sun
        \item[Cloud Z:]  2 times the mass of the Sun
    \end{description}
    %% start questions
    Which is most likely to become a white dwarf during the end of its life?
    \begin{choices}
        \wrongchoice{Cloud X}
        \wrongchoice{Cloud Y}
        \wrongchoice{Cloud Z}
        \wrongchoice{More than one of the clouds could form a star that will become a white dwarf during the end of its life. }
        \wrongchoice{None of the clouds could form a star that will become a white dwarf during the end of its life.}
    \end{choices}
\end{question}
}

\element{nasa}{
\begin{question}{exam03a2-q50}
    Listed below are several astronomical objects (1-6).  
    Which is the correct ranking for the mass of these objects,
        from largest to smallest?
    \begin{enumerate}
        \item Neptune
        \item a one solar mass white dwarf
        \item a typical neutron star
        \item Earth
        \item K5 spectral type main sequence star
        \item the typical black hole 
    \end{enumerate}
    \begin{choices}
        \wrongchoice{$3>6>2>5>1>4$}
        \wrongchoice{$5>6>2>3>1>4$}
        \wrongchoice{$6>3>2>5>1>4$}
        \wrongchoice{$2>5>3>6>4>1$}
        \wrongchoice{$6>3>2>5>4>1$}
    \end{choices}
\end{question}
}


\endinput


