
%% University of Northern Iowa
%%  Morgans Astronomy Exams
%%--------------------------------------------------

%% this section contains 15 problems

\element{morgans}{
\begin{question}{test2C-Q01}
    The two most abundant elements in the Sun,
        with the most abundant given first, are:
    \begin{choices}
        \wrongchoice{carbon and oxygen.}
        \wrongchoice{iron and hydrogen.}
        \wrongchoice{helium and nitrogen.}
        \wrongchoice{nitrogen and helium.}
      \correctchoice{hydrogen and helium.}
    \end{choices}
\end{question}
}

\element{morgans}{
\begin{question}{test2C-Q02}
    A heavier form of hydrogen is:
    \begin{choices}
        \wrongchoice{helium.}
      \correctchoice{deuterium.}
        \wrongchoice{lithium.}
        \wrongchoice{a neutrino.}
        \wrongchoice{a neutron.}
    \end{choices}
\end{question}
}

\element{morgans}{
\begin{question}{test2C-Q03}
    Which feature is closely associated with convection?
    \begin{choices}
        \wrongchoice{sunspots}
        \wrongchoice{prominences}
      \correctchoice{granules}
        \wrongchoice{solar oscillations}
    \end{choices}
\end{question}
}

\element{morgans}{
\begin{question}{test2C-Q04}
    Solar oscillations are studied by means of:
    \begin{choices}
        \wrongchoice{observing periodic changes in the Sun's angular size.}
      \correctchoice{Doppler shift observations.}
        \wrongchoice{observing strong color changes in the Sun.}
        \wrongchoice{solar eclipses.}
    \end{choices}
\end{question}
}

\element{morgans}{
\begin{question}{test2C-Q05}
    If a star has a parallax of 0.25 seconds of arc,
        what is its distance in parsecs?
    \begin{choices}
        \wrongchoice{0.25}
        \wrongchoice{0.625}
      \correctchoice{4}
        \wrongchoice{16}
        \wrongchoice{256}
    \end{choices}
\end{question}
}

\element{morgans}{
\begin{question}{test2C-Q06}
    What type of stars have a mixture of ionized and atomic metals visible in their spectra?
    \begin{choices}
        \wrongchoice{A}
        \wrongchoice{B}
      \correctchoice{G}
        \wrongchoice{M}
    \end{choices}
\end{question}
}

\element{morgans}{
\begin{question}{test2C-Q07}
    The terms absolute magnitude and \rule[-0.1pt]{4em}{0.1pt} may be used interchangeably for many purposes (although they are not equal to each other numerically or dimensionally).
    \begin{choices}
      \correctchoice{luminosity}
        \wrongchoice{apparent magnitude}
        \wrongchoice{bolometric magnitude}
        \wrongchoice{color index}
        \wrongchoice{intensity}
    \end{choices}
\end{question}
}

\element{morgans}{
\begin{question}{test2C-Q08}
    Stellar diameters may be determined from studies of \rule[-0.1pt]{4em}{0.1pt} stars.
    \begin{choices}
        \wrongchoice{visual binary}
        \wrongchoice{astrometric binary}
        \wrongchoice{spectroscopic binary}
      \correctchoice{eclipsing binary}
    \end{choices}
\end{question}
}

\element{morgans}{
\begin{question}{test2C-Q09}
    The central temperature of a star like the Sun is typically:
    \begin{choices}
        \wrongchoice{100 K.}
        \wrongchoice{10,000 K.}
      \correctchoice{10 million K.}
        \wrongchoice{10 billion K.}
        \wrongchoice{10 trillion K.}
    \end{choices}
\end{question}
}

\element{morgans}{
\begin{question}{test2C-Q10}
    Where does a star spend most of its life?
    \begin{choices}
        \wrongchoice{During the formation process.}
        \wrongchoice{As a red giant}
        \wrongchoice{As a supergiant.}
      \correctchoice{On the main sequence.}
        \wrongchoice{During the planetary nebula stage.}
    \end{choices}
\end{question}
}

\element{morgans}{
\begin{question}{test2C-Q11}
    A star with half of the mass of the Sun has a luminosity only 0.1 times that of the Sun. How does the star's main sequence life time compare with that of the Sun?
    \begin{choices}
        \wrongchoice{it will be about 10 times longer than the Sun's}
        \wrongchoice{it will be about 10 times shorter than the Sun's}
      \correctchoice{it will be about 5 times longer than the Sun's}
        \wrongchoice{it will be about 5 times shorter than the Sun's}
    \end{choices}
\end{question}
}

\element{morgans}{
\begin{question}{test2C-Q12}
    Young stellar groups may be expected to be associated with:
    \begin{choices}
        \wrongchoice{red dwarfs.}
        \wrongchoice{white dwarfs.}
      \correctchoice{dust.}
        \wrongchoice{planetary nebula.}
        \wrongchoice{solar-type stars.}
    \end{choices}
\end{question}
}

\element{morgans}{
\begin{question}{test2C-Q13}
    What causes the decrease in luminosity as a protostar evolves towards the main sequence?
    \begin{choices}
        \wrongchoice{the temperature decreases}
      \correctchoice{the radius decreases}
        \wrongchoice{the temperature increases}
        \wrongchoice{the radius increases}
    \end{choices}
\end{question}
}

\element{morgans}{
\begin{question}{test2C-Q14}
    Which of the following occurs during and immediately after the phase of the hydrogen burning shell?
    \begin{choices}
        \wrongchoice{the core shrinks until the star becomes a white dwarf}
        \wrongchoice{the helium flash occurs}
        \wrongchoice{the core temperature decreases while the envelope temperature increases}
        \wrongchoice{the star becomes a supernova}
      \correctchoice{the envelope expands and cools, and the star becomes a red giant}
    \end{choices}
\end{question}
}

\element{morgans}{
\begin{question}{test2C-Q15}
    After a Sun-like star enters its second red giant phase, its internal structure would consist of
    \begin{choices}
      \correctchoice{a carbon core surrounded by a helium burning shell, which is surrounded by a hydrogen burning shell.}
        \wrongchoice{a helium core surrounded by a hydrogen burning shell.}
        \wrongchoice{a iron core surrounded by many different layers of shell burning.}
        \wrongchoice{a oxygen core surrounded by a carbon burning shell, a helium burning shell, and a hydrogen burning shell.}
    \end{choices}
\end{question}
}

\element{morgans}{
\begin{question}{test2C-Q16}
    Why doesn't a helium flash occur in a large mass star?
    \begin{choices}
        \wrongchoice{the core is not hot enough}
        \wrongchoice{the core is degenerate}
      \correctchoice{the core is not degenerate}
        \wrongchoice{it does occur, the question is incorrect}
    \end{choices}
\end{question}
}

\element{morgans}{
\begin{question}{test2C-Q17}
    A neutron star's size is that of:
    \begin{choices}
        \wrongchoice{the Sun.}
        \wrongchoice{the Earth.}
        \wrongchoice{the Earth's orbit.}
        \wrongchoice{the State of Iowa.}
      \correctchoice{a typical city.}
    \end{choices}
\end{question}
}

\element{morgans}{
\begin{question}{test2C-Q18}
    White dwarfs are composed mostly of:
    \begin{choices}
        \wrongchoice{normal (perfect) gases.}
      \correctchoice{degenerate gases.}
        \wrongchoice{equal amounts of normal (perfect) and degenerate gases.}
        \wrongchoice{hot, solid material.}
    \end{choices}
\end{question}
}

\element{morgans}{
\begin{question}{test2C-Q19}
    The internal properties of a neutron star are most similar to those of a \rule[-0.1pt]{4em}{0.1pt} star.
    \begin{choices}
        \wrongchoice{main sequence}
        \wrongchoice{red giant}
        \wrongchoice{red dwarf}
      \correctchoice{white dwarf}
        \wrongchoice{solar-type}
        \wrongchoice{blue supergiant}
    \end{choices}
\end{question}
}

\element{morgans}{
\begin{question}{test2C-Q20}
    The accretion disk surrounding a neutron star is very hot due to compression caused by gravitational forces. This implies the object will emit strongly in which spectral region?
    \begin{choices}
      \correctchoice{X ray}
        \wrongchoice{ultraviolet}
        \wrongchoice{visual}
        \wrongchoice{infrared}
        \wrongchoice{radio}
    \end{choices}
\end{question}
}

\begin{comment}
    Fill In
    Place the most appropriate word or words in the blank. You may have to click on the blank to activate it before you start typing in your answer.
     
    The study of the oscillations of the Sun is known as .

    The very hot, extended outer atmosphere of the Sun is called the .

    The distance to a star having a parallax of 1 second of arc is called a(n) .

    An H-R diagram is a graph of . (list 2)

    The balance between gravity and outward pressure is called .

    The violent stage that an early solar-mass star goes through, where it produces H-H objects is known as .

    The maximum mass of a white dwarf star is solar masses.

    Outflow seen in young stars in just two opposite direction is known as .

    The star that became Supernova 1987A was a.

    An object that has collapsed to such a small radius that its gravitational force traps photons of light is known as a(n)
\end{comment}


\endinput


