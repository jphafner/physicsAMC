
%% Astronomy:
%%--------------------------------------------------

%% Chapter 05: Learning Objectives and Study Questions for
%%------------------------------------------------------------


% 1. Explain how the nebular model for the origin of the solar system accounts for the regular patterns in the motions and compositions of the Sun, planets, and smaller bodies.
% 2. Calculate the average density of a planet given its radius and mass, and classify it as a terrestrial, Jovian, or icy planet.
% 3. Briefly describe three techniques used to indirectly detect extrasolar planets.


\element{astr}{
\begin{question}{ch05-q01}
    Different colors of light correspond to electromagnetic waves with different \rule[-0.1pt]{4em}{0.1pt}.
    \begin{choices}
        \wrongchoice{speeds}
    \end{choices}
\end{question}
}

1. Although H, He, and a small amount of Li were formed during the Big Bang, all of the
heavier elements were formed _____.
A. by black holes
B. in stars
C. spontaneously
D. from dark matter
E. earlier
2. Radiometric dating of meteorites indicates the Solar System formed from the
collapse of a nebula about _____ years ago.
A. 6,000
B. 250 million
C. 2.5 billion
D. 4.6 billion
E. 13.7 billion
3. Gravitational collapse caused most of the mass of the solar nebula to be _____.
A. cast out of the system
B. spun into a disk around the protosun
C. concentrated in the protosun
D. concentrated into planetesimals
E. pulled into a central black hole
4. All of the planets in the Solar System are thought to have grown primarily by _____.
A. expulsion of matter from the young Sun
B. condensation of volatiles
C. gravitational instability
D. accretion of planetesimals
E. eating their Wheaties
5. Any successful model for the origin of the Solar System must account for the _____.
A. common orbital direction of the planets
B. systematic compositional differences among the planets
C. common orbital plane of the planets
D. systematic differences in mass among the planets
E. all of these features


6. The inner, terrestrial planets are _____ in mass and _____ in volatile elements and
compounds than the outer, giant planets.
A. larger, richer
B. larger, poorer
C. smaller, richer
D. smaller, poorer
E. similar, poorer
7. The planet Argos has a mass 5.616×10 24 kg and a volume of 1.040×10 21 m 3 . What is
the density of Argos?
A. 1.852×10 -4 kg/m 3
B. 54,000 kg/m 3
C. 5400 kg/m 3
D. 1852 kg/m 3
E. indeterminate, cannot tell from the data given
8. Based on its density, Argos is best classified as _____.
A. a terrestrial planet
B. a giant planet
C. a dwarf planet
D. a nebular planet
E. none of these
9. The presence of planets orbiting other stars has been inferred from _____.
A. cyclic shifts in a star’s radial velocity
B. cyclic dimming of a star transit
C. cyclic brightening of a star by microlensing
D. “wobbling” of a star as it moves across the sky
E. all of these observations
10. Most extrasolar planets detected to date are _____.
A. apparently habitable
B. more massive than Earth
C. closer to their star than 1 AU
D. farther from their star than 1 AU
E. orbiting stars in other galaxies



\endinput



