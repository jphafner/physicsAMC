
%% University of Northern Iowa
%%  Morgans Astronomy Exams
%%--------------------------------------------------

%% this section contains 15 problems

\element{morgans}{
\begin{question}{test1A-Q01}
    Planets:
    \begin{choices}
        \wrongchoice{move rapidly across the sky relative to the stars.}
        \wrongchoice{are stationary relative to the stars.}
        \wrongchoice{all move at the same rate relative to the stars.}
      \correctchoice{move slowly relative to the stars.}
        \wrongchoice{can appear anywhere in the sky.}
    \end{choices}
\end{question}
}


A nuclear reaction which occurs at sometime after hydrogen is exhausted in the core is the

a. CNO cycle.
b. triple alpha reaction.
c. helium-iron chain.
d. electron-positron cycle.
e. tritium-alpha cycle.

The last supernova observed in our galaxy occurred

a. about 40 years ago.
b. about 400 years ago.
c. about 4000 years ago.
d. last week.

What is the main activity occurring in the Orion nebula?

a. supernovae
b. planetary nebula are forming
c. star formation
d. helium flashes

Hydrogen burning for a Sun-like star lasts approximately

a. one million years.
b. ten million years.
c. one hundred million years.
d. one billion years.
e. ten billion years.

The ignition of helium in the degenerate core of a one solar mass star produces

a. a supernova.
b. a black hole.
c. the formation of new heavy elements.
d. a helium flash.
e. nothing much - the core expands, cools, and continues burning.

Stars more massive than the Sun obtain their energy while on the main sequence from

a. the proton-proton cycle.
b. the CNO cycle.
c. the triple-alpha reaction.
d. gravitational contraction.

What element is observed in the spectra of Type II supernova?

a. hydrogen
b. helium
c. iron
d. uranium

The observation of __________ from Supernova 1987A proves that our general understanding of how supernovae form is correct.

a. ultraviolet photons
b. X-ray photons
c. gamma ray photons
d. neutrons
e. neutrinos

As a white dwarf increases in mass (as a result of mass transfer from a companion, for example) 

a. its diameter becomes smaller.
b. its diameter becomes larger.
c. its evolution slows down.
d. its nuclear reactions produce more energy.
e. it becomes a black dwarf.

If a neutron star's magnetic field were precisely aligned with its rotation axis then 

a. an observer located perpendicular to the rotation axis would see it as a pulsar.
b. an observer aligned with the rotation axis would see it as a pulsar.
c. all observers would see it as a pulsar, regardless of their orientation to it.
d. it would not be observed to be a pulsar.

From the outsider's point of view, in watching a star collapse to form a black hole, the collapse would appear to take

a. only a fraction of a second.
b. a few hours.
c. forever.

Fill In
Place the most appropriate word or words in the blank. You may have to click on the blank to activate it before you start typing in your answer.
 
An antimatter electron is called a(n) .

The specific temperature of the visible surface of the Sun is K.

Knowledge of a star's parallax allows us to compute its .

The specific nuclear reaction providing energy in the Sun is the .

The thin layer of hot gas just outside the photosphere of the Sun and other cool stars is called the .

A rapid burst of nuclear reactions in the degenerate core of a moderate-mass star in the hydrogen shell-burning phase when the star is a red giant is called the .

The Sun is percent hydrogen.

A star with a main sequence mass of 10 solar masses will most likely end up as a(n) .

A rapidly rotating neutron star that emits periodic bursts of electromagnetic radiation, probably by the emission of beams of radiation from the magnetic poles, is called a(n) .

The region of infinite density and pressure at the center of a black hole is called the .

