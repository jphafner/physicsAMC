
%% University of Northern Iowa
%%  Morgans Astronomy Exams
%%--------------------------------------------------

%% this section contains 15 problems

\element{morgans}{
\begin{question}{test1E-Q01}
    An astronomical unit is the:
    \begin{choices}
        \wrongchoice{distance from the Earth to the Moon.}
      \correctchoice{distance from the Earth to the Sun.}
        \wrongchoice{distance from the Earth to the nearest star.}
        \wrongchoice{distance light travels in one year.}
        \wrongchoice{circumference of the Earth.}
    \end{choices}
\end{question}
}

\element{morgans}{
\begin{question}{test1E-Q02}
    Suppose you are on a strange planet. 
    Since you have had an astronomy class at the university,
        you are aware of the daily motion of stars about a fixed point in the sky. 
    Furthermore, you notice that this fixed point is 30 degrees above the horizon. 
    You then deduce that your latitude on this planet is:
    \begin{choices}
        \wrongchoice{0 degrees.}
        \wrongchoice{15 degrees.}
      \correctchoice{30 degrees.}
        \wrongchoice{45 degrees.}
        \wrongchoice{60 degrees.}
    \end{choices}
\end{question}
}

\element{morgans}{
\begin{question}{test1E-Q03}
    On what day would the Sun rise for someone standing on the Earth's south pole?
    \begin{choices}
        \wrongchoice{the vernal equinox (March 21)}
      \correctchoice{the autumnal equinox (Sept. 21)}
        \wrongchoice{the summer solstice (June 21)}
        \wrongchoice{the winter solstice (December 21)}
        \wrongchoice{it never rises}
    \end{choices}
\end{question}
}

\element{morgans}{
\begin{question}{test1E-Q04}
    In order for a solar eclipse to occur,
        the Moon must be:
    \begin{choices}
      \correctchoice{near new Moon.}
        \wrongchoice{near first or last quarter.}
        \wrongchoice{high in the sky.}
        \wrongchoice{near full Moon.}
        \wrongchoice{in a retrograde loop.}
    \end{choices}
\end{question}
}

\element{morgans}{
\begin{question}{test1E-Q05}
    If Mars were to rise at noon,
        what configuration would it be in? 
    \begin{choices}
        \wrongchoice{conjunction}
        \wrongchoice{opposition}
      \correctchoice{quadrature}
        \wrongchoice{greatest elongation}
    \end{choices}
\end{question}
}

\element{morgans}{
\begin{question}{test1E-Q06}
    The Copernican model of the solar system is similar to Aristotle's in that it:
    \begin{choices}
        \wrongchoice{used epicycles.}
        \wrongchoice{placed the Sun in the center.}
      \correctchoice{used perfect circular motions.}
        \wrongchoice{placed the Earth at the center.}
    \end{choices}
\end{question}
}

\element{morgans}{
\begin{question}{test1E-Q07}
    In non-mathematical terms,
        Kepler's third law says that:
    \begin{choices}
        \wrongchoice{a planet moves more rapidly when near the Sun than when farther away.}
      \correctchoice{planets close to the Sun have shorter periods than those farther away.}
        \wrongchoice{the Sun is at the center of planetary orbits.}
        \wrongchoice{slowly moving planets are close to the Sun.}
    \end{choices}
\end{question}
}

\element{morgans}{
\begin{question}{test1E-Q08}
    The Earth's rapid rotation causes it to bulge at the equator,
        making the equatorial diameter slightly larger than the polar diameter
        (6378 versus 6335 km). 
    If a person weighs 100 lbs at the equator,
        what could be the weight at the pole?
    (Neglect centripetal force.)
    (NOTE: the numbers are not really correct.)
    \begin{choices}
        \wrongchoice{43 lbs}
        \wrongchoice{99 lbs}
        \wrongchoice{100 lbs}
      \correctchoice{101 lbs}
        \wrongchoice{143 lbs}
    \end{choices}
\end{question}
}

\element{morgans}{
\begin{question}{test1E-Q09}
    If you weigh 3 times more than your dog and you both go sky diving,
        which of the following statements is true? 
    (NOTE: Ignore air and wind resistance)
    \begin{choices}
        \wrongchoice{your rate of acceleration towards the Earth would be greater than that of your dog's}
        \wrongchoice{your rate of acceleration towards the Earth would be less than that of your dog's}
      \correctchoice{your rate of acceleration towards the Earth would be equal to that of your dog's}
        \wrongchoice{the gravitational force of the Earth would repel both of you}
    \end{choices}
\end{question}
}

\element{morgans}{
\begin{question}{test1E-Q10}
    The fact that the escape velocity from the Moon is less than that from the Earth is due primarily to: 
    \begin{choices}
        \wrongchoice{the Moon's distance from the Earth.}
      \correctchoice{the smaller mass (81 times less than the Earth's mass).}
        \wrongchoice{the smaller radius (about one-fourth the Earth's radius).}
        \wrongchoice{the higher temperature of the Moon.}
        \wrongchoice{the Moon being alternately closer and farther from the Sun than the Earth as it orbits the Earth each month.}
    \end{choices}
\end{question}
}

\element{morgans}{
\begin{question}{test1E-Q11}
    The energy of a photon is:
    \begin{choices}
        \wrongchoice{proportional to the wavelength and inversely proportional to the frequency.}
        \wrongchoice{proportional to the wavelength and proportional to the frequency.}
        \wrongchoice{inversely proportional to the wavelength and inversely proportional to the frequency.}
      \correctchoice{inversely proportional to the wavelength and proportional to the frequency.}
    \end{choices}
\end{question}
}

\element{morgans}{
\begin{question}{test1E-Q12}
    A star has an absorption spectrum showing many lines corresponding to silicon. 
    Before it reaches an observer,
        the light from this star passes through a cool gas cloud containing a large amount of silicon. 
    What will the observer detect?
    \begin{choices}
      \correctchoice{an absorption spectrum with many silicon lines}
        \wrongchoice{an absorption and emission spectrum with lines corresponding to silicon}
        \wrongchoice{an emission spectrum of many silicon lines}
        \wrongchoice{a continuous spectrum}
    \end{choices}
\end{question}
}

\element{morgans}{
\begin{question}{test1E-Q13}
    If you wished to observe in optical light an object with at most a resolution of 0.01 arc seconds,
        which of the following telescopes would you use?
    \begin{choices}
        \wrongchoice{a 100 cm reflector at sea level}
        \wrongchoice{a 10 cm reflector at sea level}
        \wrongchoice{a 10 cm refractor at sea level}
      \correctchoice{a 10 cm telescope in orbit}
    \end{choices}
\end{question}
}

\element{morgans}{
\begin{question}{test1E-Q14}
    Which of the following types of light can only be observed from space?
    \begin{choices}
        \wrongchoice{visible}
        \wrongchoice{radio}
        \wrongchoice{infrared}
      \correctchoice{gamma ray}
    \end{choices}
\end{question}
}

\element{morgans}{
\begin{question}{test1E-Q15}
    What type of light is observed using the ROSAT satellite?
    \begin{choices}
      \correctchoice{X ray}
        \wrongchoice{ultraviolet}
        \wrongchoice{radio}
        \wrongchoice{microwave}
    \end{choices}
\end{question}
}

\begin{comment}
    Fill In
    Place the most appropriate word or words in the blank. You may have to click on the blank to activate it before you start typing in your answer.
     
    The motion of the Earth around the Sun is called .

    The slow motion of the Earth's rotation axis around the celestial sphere is called .

    When a superior planet is located 90 degrees away from the Sun it is at .

    In 1543, published a theory of a heliocentric solar system, though it included epicycles and perfect circular motions.

    The average distance between the Earth and the Sun is known as the .

    When a planet travels at the slowest rate during its orbit about the Sun, it is located at .

    The is the term for the rate at which electromagnetic waves pass a fixed point each second.

    A(n) spectrum is produced by a dense gas at a high temperature.

    The type of telescope which uses a lens as the main optical element is a(n) .

    is the term which describes the ability of a telescope to see fine detail in an image.
\end{comment}


\endinput


