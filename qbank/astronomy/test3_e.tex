
%% university of northern iowa
%%  morgans astronomy exams
%%--------------------------------------------------

%% this section contains 15 problems

\element{morgans}{
\begin{question}{test3E-q01}
    An important relation observed for Cepheid variables is that between:
    \begin{choices}
        \wrongchoice{mass and luminosity.}
        \wrongchoice{luminosity and temperature.}
      \correctchoice{period and luminosity.}
        \wrongchoice{period and radius.}
        \wrongchoice{mass and temperature.}
    \end{choices}
\end{question}
}

\element{morgans}{
\begin{question}{test3E-q02}
    The period of revolution of the Sun about the galactic center is closest to:
    \begin{choices}
        \wrongchoice{3 million years.}
        \wrongchoice{25 million years.}
      \correctchoice{250 million years.}
        \wrongchoice{2 billion years.}
        \wrongchoice{25 billion years.}
    \end{choices}
\end{question}
}

\element{morgans}{
\begin{question}{test3E-q03}
    The mass of the Milky Way Galaxy inward from some position in the disk is most easily determined from:
    \begin{choices}
        \wrongchoice{application of Newton's second law.}
      \correctchoice{Kepler's third law as modified by Newton.}
        \wrongchoice{21-cm radiation.}
        \wrongchoice{spectral analysis of Cepheid variables.}
        \wrongchoice{Doppler shift measurements of the galactic nucleus.}
    \end{choices}
\end{question}
}

\element{morgans}{
\begin{question}{test3E-q04}
    The Milky Way Galaxy's most massive component is:
    \begin{choices}
        \wrongchoice{the stars that are found in the disk.}
        \wrongchoice{the black hole in the nucleus.}
      \correctchoice{the dark matter in halo.}
        \wrongchoice{the gas clouds found near the spiral arms.}
    \end{choices}
\end{question}
}

\element{morgans}{
\begin{question}{test3E-q05}
    The density wave which is thought to travel throughout the Galaxy is most analogous to:
    \begin{choices}
        \wrongchoice{a water wave.}
      \correctchoice{a sound wave.}
        \wrongchoice{a light wave.}
    \end{choices}
\end{question}
}

\element{morgans}{
\begin{question}{test3E-q06}
    Most stars near the Sun are primarily \rule[-0.1pt]{4em}{0.1pt} stars.
    \begin{choices}
      \correctchoice{Population I}
        \wrongchoice{Population II}
        \wrongchoice{Population III}
        \wrongchoice{Population IV}
        \wrongchoice{Population V}
    \end{choices}
\end{question}
}

\element{morgans}{
\begin{question}{test3E-q07}
    The Magellanic Clouds are:
    \begin{choices}
        \wrongchoice{irregular galaxies.}
        \wrongchoice{spiral galaxies.}
        \wrongchoice{elliptical galaxies.}
        \wrongchoice{large clouds of gas and dust.}
        \wrongchoice{distant globular clusters.}
    \end{choices}
\end{question}
}

\element{morgans}{
\begin{question}{test3E-q08}
    What is the easiest way of measuring the rotation of a spiral galaxy?
    \begin{choices}
        \wrongchoice{pick out individual stars, and measure their Doppler shifts}
      \correctchoice{use 21-cm emission to measure the motion of the gas in the disk}
        \wrongchoice{look for supernova, and observe how they expand}
        \wrongchoice{determine the location and distribution of globular clusters within the galaxy}
    \end{choices}
\end{question}
}

\element{morgans}{
\begin{question}{test3E-q09}
    The Local Group is:
    \begin{choices}
        \wrongchoice{the nearest group of stars to the solar system.}
        \wrongchoice{the nearest open cluster.}
        \wrongchoice{the Andromeda Galaxy and its companions.}
      \correctchoice{the cluster of galaxies in which the Milky Way galaxy is located.}
    \end{choices}
\end{question}
}

\element{morgans}{
\begin{question}{test3E-q10}
    Which of the following statements concerning the Magellanic Clouds is true?
    \begin{choices}
        \wrongchoice{they are elliptical galaxies}
        \wrongchoice{they are in orbit about M31}
        \wrongchoice{they appear as fuzzy patches to the naked eye for observers in North America}
      \correctchoice{they are active sites of star formation}
        \wrongchoice{they are both similar to the Milky Way}
    \end{choices}
\end{question}
}

\element{morgans}{
\begin{question}{test3E-q11}
    The expansion of the universe was discovered by:
    \begin{choices}
        \wrongchoice{Shapley.}
        \wrongchoice{Curtis.}
        \wrongchoice{Russell.}
        \wrongchoice{Penzias and Wilson.}
      \correctchoice{Hubble.}
    \end{choices}
\end{question}
}

\element{morgans}{
\begin{question}{test3E-q12}
    The Hubble constant is related to the \rule[-0.1pt]{4em}{0.1pt} of the universe.
    \begin{choices}
        \wrongchoice{size}
        \wrongchoice{mass}
      \correctchoice{age}
        \wrongchoice{luminosity}
    \end{choices}
\end{question}
}

\element{morgans}{
\begin{question}{test3E-q13}
    Who discovered the cosmic microwave background radiation?
    \begin{choices}
        \wrongchoice{Curtis and Shapley}
        \wrongchoice{Hubble and Sandage}
        \wrongchoice{Gamow and Dicke}
      \correctchoice{Penzias and Wilson}
    \end{choices}
\end{question}
}

\element{morgans}{
\begin{question}{test3E-q14}
    The most recent results from NASA probes indicate that the microwave background radiation:
    \begin{choices}
        \wrongchoice{is smooth to an accuracy of a millionth of a degree.}
        \wrongchoice{is very rough, varying by about 2-3 degrees.}
      \correctchoice{has slight temperature variations on the order of fractions of degrees.}
        \wrongchoice{has temperature variations towards the galactic center.}
    \end{choices}
\end{question}
}

\element{morgans}{
\begin{question}{test3E-q15}
    Type I Seyfert galaxies are distinguished from Type II because they:
    \begin{choices}
        \wrongchoice{vary in brightness.}
      \correctchoice{have broad emission features.}
        \wrongchoice{have narrow emission features.}
        \wrongchoice{have no emission features.}
    \end{choices}
\end{question}
}

\element{morgans}{
\begin{question}{test3E-q16}
    The most distant quasars have velocities that are roughly:
    \begin{choices}
        \wrongchoice{\SI{50}{\percent} the speed of light.}
      \correctchoice{\SI{95}{\percent} the speed of light.}
        \wrongchoice{\SI{101}{\percent} the speed of light.}
        \wrongchoice{\SI{200}{\percent} the speed of light.}
        \wrongchoice{\SI{500}{\percent} the speed of light.}
    \end{choices}
\end{question}
}

\element{morgans}{
\begin{question}{test3E-q17}
    Since quasars are at great distances with the light taking a long time to reach us,
        the light we are observing comes from \rule[-0.1pt]{4em}{0.1pt}.
    \begin{choices}
        \wrongchoice{old objects.}
      \correctchoice{young objects.}
        \wrongchoice{medium age objects.}
        \wrongchoice{dying objects.}
        \wrongchoice{none of the provided; no statement concerning age may be made.}
    \end{choices}
\end{question}
}

\element{morgans}{
\begin{question}{test3E-q18}
    At the present time, observations of the characteristics of the universe indicate that the universe is:
    \begin{choices}
        \wrongchoice{spherical.}
        \wrongchoice{hyperbolic.}
      \correctchoice{flat.}
        \wrongchoice{static.}
    \end{choices}
\end{question}
}

\element{morgans}{
\begin{question}{test3E-q19}
    Except for hydrogen the most abundant element formed in the big bang was:
    \begin{choices}
        \wrongchoice{helium.}
        \wrongchoice{oxygen.}
        \wrongchoice{nitrogen.}
        \wrongchoice{carbon.}
        \wrongchoice{iron.}
    \end{choices}
\end{question}
}

\element{morgans}{
\begin{question}{test3E-q20}
    If a galaxy is moving away from you at a velocity of 4000 km/s and the Hubble Constant equals 50 km/s/Mpc,
        how far away is the galaxy based upon Hubble's law?
    \begin{choices}
        \wrongchoice{20 Mpc}
        \wrongchoice{1/2 Mpc}
        \wrongchoice{40 Mpc}
        \wrongchoice{80 Mpc}
        \wrongchoice{40,000 Mpc}
    \end{choices}
\end{question}
}


\begin{comment}
    Fill In
    Place the most appropriate word or words in the blank. You may have to click on the blank to activate it before you start typing in your answer.
     
    The diameter of the disk of our galaxy is approximately .

    Young stars belong to Population .

    Galaxies characterized by smooth spheroidal forms, few if any young stars, and little if any interstellar matter are classified as .

    The Andromeda Galaxy (M31) was shown to be far beyond the limits of our galaxy when stars were discovered to be in it (give the specific star type).

    The most elongated type of elliptical galaxy is known as a type .

    The greater the distance to a galaxy, the greater its , according to Hubble's law.

    The background radiation was accidentally discovered by  (two names).

    Radiation produced by charged particles moving rapidly in a magnetic field is called .

    The study of the origins of the universe is called .

    The curvature of the universe in which the expansion will eventually stop and reverse is known as a(n) .
\end{comment}


\endinput


