
%% Astronomy I Practice Exam 1
%%--------------------------------------------------

%% this section contains 16 problems

\element{astr}{
\begin{question}{ASTRI-1-Q01}
    Measurements made on the celestial sphere are made in units of:
    \begin{multicols}{2}
    \begin{choices}
        \wrongchoice{miles.}
        \wrongchoice{kilometers.}
      \correctchoice{degrees.}
        \wrongchoice{light years.}
    \end{choices}
    \end{multicols}
\end{question}
}

\element{astr}{
\begin{question}{ASTRI-1-Q02}
    An astronomical unit is the:
    \begin{choices}
        \wrongchoice{distance from the Earth to the Moon.}
      \correctchoice{distance from the Earth to the Sun.}
        \wrongchoice{distance from the Earth to the nearest star.}
        \wrongchoice{distance light travels in one year.}
        \wrongchoice{circumference of the Earth.}
    \end{choices}
\end{question}
}

\element{astr}{
\begin{question}{ASTRI-1-Q03}
    Which of the following types of electromagnetic radiation has a wavelength adjacent to but longer than visible light?
    \begin{choices}
        \wrongchoice{radio}
      \correctchoice{infrared}
        \wrongchoice{X-ray}
        \wrongchoice{ultraviolet}
        \wrongchoice{gamma ray}
    \end{choices}
\end{question}
}

\element{astr}{
\begin{question}{ASTRI-1-Q04}
    The energy of a photon is:
    \begin{choices}
        \wrongchoice{proportional to the wavelength and inversely proportional to the frequency.}
        \wrongchoice{proportional to the wavelength and proportional to the frequency.}
        \wrongchoice{inversely proportional to the wavelength and inversely proportional to the frequency.}
      \correctchoice{inversely proportional to the wavelength and proportional to the frequency.}
    \end{choices}
\end{question}
}

\element{astr}{
\begin{question}{ASTRI-1-Q05}
    Radio waves have:
    \begin{choices}
        \wrongchoice{high energy and long wavelength.}
      \correctchoice{low energy and long wavelength.}
        \wrongchoice{low energy and short wavelength.}
        \wrongchoice{high energy and short wavelength.}
    \end{choices}
\end{question}
}

\element{astr}{
\begin{question}{ASTRI-1-Q06}
    If Star $A$ is hotter than Star $B$,
        and Star $A$ is emitting most of its light at a wavelength corresponding to yellow light,
        which of the following statements is true?
    \begin{choices}
      \correctchoice{Star $B$ will emit most of its light at a wavelength longer than yellow}
        \wrongchoice{Star $B$ will emit most of its light at a wavelength shorter than yellow}
        \wrongchoice{Star $B$ will emit most of its light at the same wavelength as Star A}
        \wrongchoice{more information is required to answer this question}
    \end{choices}
\end{question}
}

\element{astr}{
\begin{question}{ASTRI-1-Q07}
    If two stars have the same surface area but one has 3 times the temperature of the other,
        how many times more energy is radiated by the more luminous star?
    \begin{multicols}{3}
    \begin{choices}
        \wrongchoice{\num{3}}
        \wrongchoice{\num{9}}
        \wrongchoice{\num{12}}
        \wrongchoice{\num{27}}
      \correctchoice{\num{81}}
    \end{choices}
    \end{multicols}
\end{question}
}

\element{astr}{
\begin{question}{ASTRI-1-Q08}
    Star $A$ is radiating two times as much energy as Star $B$,
        but it is two times the distance from us. 
    Which star will appear brighter, and by how much?
    \begin{choices}
        \wrongchoice{Star $A$ will be 2 times brighter}
      \correctchoice{Star $B$ will be 2 times brighter}
        \wrongchoice{Star $A$ will be 4 times brighter}
        \wrongchoice{Star $B$ will be 4 times brighter}
        \wrongchoice{they will both have the same observed brightness}
    \end{choices}
\end{question}
}

\element{astr}{
\begin{question}{ASTRI-1-Q09}
    Which one of the following types of spectrum always occurs with an absorption (or dark line) spectrum?
    \begin{choices}
        \wrongchoice{bright line}
        \wrongchoice{emission line}
      \correctchoice{continuous}
    \end{choices}
\end{question}
}

\element{astr}{
\begin{question}{ASTRI-1-Q10}
    Spectral lines unique to each type of atom are caused by:
    \begin{choices}
        \wrongchoice{each atom having a unique set of protons.}
      \correctchoice{the unique sets of electron orbits.}
        \wrongchoice{the neutron-electron interaction being unique for each atom.}
        \wrongchoice{each type of photon emitted by the atom being unique.}
        \wrongchoice{none of the above; spectral lines are not unique to each type of atom.}
    \end{choices}
\end{question}
}

\element{astr}{
\begin{question}{ASTRI-1-Q11}
    A star has an absorption spectrum showing many lines corresponding to silicon.
    Before it reaches an observer,
        the light from this star passes through a cool gas cloud containing a large amount of silicon. What will the observer detect?
    \begin{choices}
      \correctchoice{an absorption spectrum with many silicon lines}
        \wrongchoice{an absorption and emission spectrum with lines corresponding to silicon}
        \wrongchoice{an emission spectrum of many silicon lines}
        \wrongchoice{a continuous spectrum}
    \end{choices}
\end{question}
}

\element{astr}{
\begin{question}{ASTRI-1-Q12}
    Consider a cloud of (cool) gas between a star and an observer to be moving away from a source of continuous radiation (and towards the observer). 
    Suppose the atoms in the gas have two energy levels separated by an energy corresponding to 5000 Angstroms.
    The observer will see a spectrum with absorption at a wavelength
    \begin{choices}
      \correctchoice{less than 5000 Angstroms.}
        \wrongchoice{equal to 5000 Angstroms.}
        \wrongchoice{greater than 5000 Angstroms.}
        \wrongchoice{no absorption will take place.}
    \end{choices}
\end{question}
}

\element{astr}{
\begin{question}{ASTRI-1-Q13}
    If an electron moves from a lower energy level to the next higher energy level, then:
    \begin{choices}
      \correctchoice{the atom has become excited.}
        \wrongchoice{the atom has become ionized.}
        \wrongchoice{the atom's light will be blue shifted.}
        \wrongchoice{the atom's light will be red shifted.}
    \end{choices}
\end{question}
}

\element{astr}{
\begin{question}{ASTRI-1-Q14}
    The degree of ionization of an atom (i.e. the number of electrons lost) depends on:
    \begin{choices}
        \wrongchoice{the level of the ground state.}
      \correctchoice{the temperature of the gas.}
        \wrongchoice{the energy difference between the ground state and the first excited state.}
        \wrongchoice{the distance to the observer.}
    \end{choices}
\end{question}
}

\element{astr}{
\begin{question}{ASTRI-1-Q15}
    Which of the following types of light can only be observed from space?
    \begin{multicols}{2}
    \begin{choices}
        \wrongchoice{visible}
        \wrongchoice{radio}
        \wrongchoice{infrared}
      \correctchoice{gamma ray}
    \end{choices}
    \end{multicols}
\end{question}
}

\element{astr}{
\begin{question}{ASTRI-1-Q16}
    The determination of stellar parallax is important because it allows the direct determination of:
    \begin{multicols}{2}
    \begin{choices}
        \wrongchoice{mass.}
      \correctchoice{distance.}
        \wrongchoice{diameter.}
        \wrongchoice{velocity.}
    \end{choices}
    \end{multicols}
\end{question}
}

% Answer Key
% 1. c
% 2. b
% 3. b
% 4. d
% 5. b
% 6. a
% 7. e
% 8. b
% 9. c
% 10. b
% 11. a
% 12. a
% 13. a
% 14. b
% 15. d
% 16. b 


\endinput


