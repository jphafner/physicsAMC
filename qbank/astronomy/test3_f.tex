
%% university of northern iowa
%%  morgans astronomy exams
%%--------------------------------------------------

%% this section contains 15 problems

\element{morgans}{
\begin{question}{test3F-q01}
    Which of the following objects or techniques is most useful to determine the distance to the most distant objects in the Milky Way Galaxy?
    \begin{choices}
        \wrongchoice{solar-type stars}
      \correctchoice{Cepheid variables}
        \wrongchoice{RR Lyrae variables}
        \wrongchoice{main sequence fitting methods}
        \wrongchoice{white dwarf stars}
    \end{choices}
\end{question}
}

\element{morgans}{
\begin{question}{test3F-q02}
    Spiral arms appear to be prominent in spiral galaxies because:
    \begin{choices}
        \wrongchoice{all the stars are distributed in a spiral pattern.}
        \wrongchoice{cool stars are distributed in a spiral pattern while the hot stars are spread more uniformly.}
      \correctchoice{hot stars are distributed in a spiral pattern while the cool stars are spread more uniformly.}
        \wrongchoice{globular clusters are distributed in a spiral pattern around the Galaxy.}
    \end{choices}
\end{question}
}

\element{morgans}{
\begin{question}{test3F-q03}
    The mass of our Galaxy,
        in units of solar mass, is nearest to:
    \begin{choices}
        \wrongchoice{a few million.}
        \wrongchoice{a few billion.}
        \wrongchoice{tens of billions}
      \correctchoice{hundreds of billions}
        \wrongchoice{hundreds of trillions}
    \end{choices}
\end{question}
}

\element{morgans}{
\begin{question}{test3F-q04}
    The spiral arms are delineated (outlined) by:
    \begin{choices}
        \wrongchoice{white dwarfs.}
      \correctchoice{O and B stars.}
        \wrongchoice{solar-type stars.}
        \wrongchoice{globular clusters.}
    \end{choices}
\end{question}
}

\element{morgans}{
\begin{question}{test3F-q05}
    Which of the following statements is \emph{true}?
    \begin{choices}
        \wrongchoice{the material in the disk and the spiral arms move at the same rate around the Galaxy}
      \correctchoice{the material in the disk moves at a different rate than the spiral arms about the Galaxy}
        \wrongchoice{the parts of the spiral arms located closer to the center of the Galaxy move faster than those parts located near the edge of the Galaxy}
        \wrongchoice{the parts of the spiral arms located close to the edge of the Galaxy move faster than those parts located near the center of the Galaxy}
    \end{choices}
\end{question}
}

\element{morgans}{
\begin{question}{test3F-q06}
    In the famous 1920 astronomical debate,
        the primary spokesman for the view that the ``nebulae'' were close to our galaxy was:
    \begin{choices}
        \wrongchoice{Hubble.}
      \correctchoice{Shapley.}
        \wrongchoice{Curtis.}
        \wrongchoice{Wilson.}
        \wrongchoice{Russell.}
    \end{choices}
\end{question}
}

\element{morgans}{
\begin{question}{test3F-q07}
    S0 galaxies are best described as:
    \begin{choices}
        \wrongchoice{large type I irregulars.}
        \wrongchoice{large type II irregulars.}
      \correctchoice{spiral galaxies without arms, but with a disk.}
        \wrongchoice{completely spherical elliptical galaxies.}
    \end{choices}
\end{question}
}

\element{morgans}{
\begin{question}{test3F-q08}
    The fact that galactic masses determined from binary systems are always larger than the masses determined by measurements of internal motions within a galaxy is evidence for:
    \begin{choices}
        \wrongchoice{black holes in galactic nuclei.}
        \wrongchoice{substantial mass loss by stars during their evolution.}
        \wrongchoice{mass loss by galaxies during their evolution.}
      \correctchoice{the presence of dark matter}
    \end{choices}
\end{question}
}

\element{morgans}{
\begin{question}{test3F-q09}
    Study of the Local Group is hindered primarily by:
    \begin{choices}
        \wrongchoice{the great distances of its members.}
        \wrongchoice{the nearness of the members.}
        \wrongchoice{the large number of members.}
      \correctchoice{interstellar dust.}
        \wrongchoice{intergalactic dust.}
    \end{choices}
\end{question}
}

\element{morgans}{
\begin{question}{test3F-q10}
    Studies of the Andromeda Galaxy are most important for our knowledge of the \rule[-0.1pt]{4em}{0.1pt} of distant galaxies. 
    \begin{choices}
      \correctchoice{distance}
        \wrongchoice{masses}
        \wrongchoice{structure}
        \wrongchoice{galaxy types}
    \end{choices}
\end{question}
}

\element{morgans}{
\begin{question}{test3F-q11}
    Spectra of distant galaxies show:
    \begin{choices}
      \correctchoice{a large red shift.}
        \wrongchoice{a large blue shift.}
        \wrongchoice{no spectral shift.}
        \wrongchoice{a small red shift.}
        \wrongchoice{a small blue shift.}
    \end{choices}
\end{question}
}

\element{morgans}{
\begin{question}{test3F-q12}
    The intensity of the microwave background radiation is greatest in the direction of:
    \begin{choices}
        \wrongchoice{the galactic center.}
        \wrongchoice{M31.}
        \wrongchoice{rich galaxy clusters.}
        \wrongchoice{the Sun.}
      \correctchoice{no particular direction; it is the same all over.}
    \end{choices}
\end{question}
}

\element{morgans}{
\begin{question}{test3F-q13}
    When the radio appearance of a typical radio galaxy is compared with its optical counterpart,
        the radio object looks like:
    \begin{choices}
        \wrongchoice{a tadpole.}
        \wrongchoice{a jellyfish.}
      \correctchoice{a figure eight (8).}
        \wrongchoice{a spiral galaxy.}
    \end{choices}
\end{question}
}

\element{morgans}{
\begin{question}{test3F-q14}
    The optical spectrum of the nucleus of a Seyfert galaxy consists of:
    \begin{choices}
        \wrongchoice{a bright continuous spectrum.}
      \correctchoice{a strong emission line spectrum.}
        \wrongchoice{very dark absorption lines.}
        \wrongchoice{none of the above; the objects are bright in the radio but too faint optically to obtain a spectrum.}
    \end{choices}
\end{question}
}

\element{morgans}{
\begin{question}{test3F-q15}
    An active galaxy with a bright central core, variable brightness,
        but no features in its spectrum, is known as an:
    \begin{choices}
        \wrongchoice{quasar.}
        \wrongchoice{Seyfert galaxy.}
        \wrongchoice{BL Lac object.}
        \wrongchoice{RR Lyrae.}
    \end{choices}
\end{question}
}

\element{morgans}{
\begin{question}{test3F-q16}
    Which of the following is a solution to Olbers's paradox?
    \begin{choices}
        \wrongchoice{interstellar extinction makes the night sky dark}
        \wrongchoice{the universe is infinite in extent, and in age}
      \correctchoice{the universe is not infinite in age}
        \wrongchoice{the sky is not dark, but is full of X-ray radiation}
    \end{choices}
\end{question}
}

\element{morgans}{
\begin{question}{test3F-q17}
    What fraction of the matter in the universe appears to be in the form of dark matter?
    \begin{choices}
        \wrongchoice{less than 10}{\percent}}
      \correctchoice{around 30}{\percent}}
        \wrongchoice{around 50}{\percent}}
        \wrongchoice{around 80}{\percent}}
        \wrongchoice{much more than 80}{\percent}}
    \end{choices}
\end{question}
}

\element{morgans}{
\begin{question}{test3F-q18}
    Deuterium in the universe was produced:
    \begin{choices}
        \wrongchoice{by fusion in stellar interiors.}
        \wrongchoice{by fission in stellar interiors.}
      \correctchoice{in the big bang.}
        \wrongchoice{by high density radiation forming matter (according to $E=mc^2$).}
        \wrongchoice{in none of the above; deuterium has only been formed in laboratories on Earth and is not present elsewhere in the universe.}
    \end{choices}
\end{question}
}

\element{morgans}{
\begin{question}{test3F-q19}
    Current estimates of the age of the Universe give a value of around:
    \begin{choices}
        \wrongchoice{5 million years.}
        \wrongchoice{10 billion years.}
      \correctchoice{14 billion years.}
        \wrongchoice{20 billion years.}
    \end{choices}
\end{question}
}

\element{morgans}{
\begin{question}{test3F-q20}
    Currently, astronomers believe that:
    \begin{choices}
      \correctchoice{the universe will expand forever.}
        \wrongchoice{the universe will someday collapse.}
        \wrongchoice{the age of the universe is 15.87 billion years.}
        \wrongchoice{they don't know a lot about the universe.}
    \end{choices}
\end{question}
}

\begin{comment}
    Fill In
    Place the most appropriate word or words in the blank. You may have to click on the blank to activate it before you start typing in your answer.
     
    The distance of the Sun from the center of the Galaxy is .

    Metal poor stars belong to Population .

    A galaxy whose type is similar to that of our galaxy is classified as a .

    The mass of a galaxy can be obtained by application of .

    In terms of colors, an elliptical galaxy would appear than a spiral.

    The Hubble constant is inversely related to the of the universe.

    A galaxy whose luminosity is greatest in radio wavelengths is called a(n) .

    The radio emission in a typical radio galaxy is produced by .

    The study of the structure and evolution of the universe is called .

    The term for the density at which there is just enough matter in the universe to halt the collapse is the 
\end{comment}


\endinput


