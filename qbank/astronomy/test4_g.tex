
%% university of northern iowa
%%  morgans astronomy exams
%%--------------------------------------------------

%% this section contains 20 problems

\element{morgans}{
\begin{question}{test4G-q01}
    Which of the following is \emph{not} a planetary probe?
    \begin{choices}
        \wrongchoice{Magellan}
        \wrongchoice{Galileo}
        \wrongchoice{Viking}
        \wrongchoice{Voyager}
      \correctchoice{Pilgrim}
    \end{choices}
\end{question}
}

\element{morgans}{
\begin{question}{test4G-q02}
    Which planets show evidence of water erosion?
    \begin{choices}
        \wrongchoice{Earth and the Moon}
        \wrongchoice{Earth and Venus}
      \correctchoice{Earth and Mars}
        \wrongchoice{Venus and Mars}
    \end{choices}
\end{question}
}

\element{morgans}{
\begin{question}{test4G-q03}
    If the Earth's interior were entirely solid,
        which of the following statements is true?
    \begin{choices}
        \wrongchoice{only P waves would travel through the Earth}
        \wrongchoice{only S waves would travel through the Earth}
      \correctchoice{both P and S waves would travel through the Earth}
        \wrongchoice{neither P nor S waves would travel through the Earth}
    \end{choices}
\end{question}
}

\element{morgans}{
\begin{question}{test4G-q04}
    The Moon:
    \begin{choices}
      \correctchoice{rotates on its axis with the same period as its revolution about the Earth.}
        \wrongchoice{does not rotate.}
        \wrongchoice{was formed from material identical in chemical composition to the Earth's crust.}
        \wrongchoice{always points the same face toward the Sun.}
    \end{choices}
\end{question}
}

\element{morgans}{
\begin{question}{test4G-q05}
    Mercury is difficult to see from Earth primarily because:
    \begin{choices}
        \wrongchoice{it is such a small planet.}
        \wrongchoice{it is very faint.}
        \wrongchoice{it rotates slowly.}
      \correctchoice{it always appears near the Sun.}
        \wrongchoice{the orbit is highly elliptical.}
    \end{choices}
\end{question}
}

\element{morgans}{
\begin{question}{test4G-q06}
    Which of the following statements about Mercury's rotation is true? 
    \begin{choices}
        \wrongchoice{rotation period = revolution period}
        \wrongchoice{planet does not rotate}
      \correctchoice{planet rotates 3 times for every 2 revolutions}
        \wrongchoice{planet revolves 3 times for every 2 rotations}
    \end{choices}
\end{question}
}

\element{morgans}{
\begin{question}{test4G-q07}
    We are able to study the surface of Venus by bouncing \rule[-0.1pt]{4em}{0.1pt} off the surface.
    \begin{choices}
        \wrongchoice{X rays}
        \wrongchoice{ultraviolet radiation}
        \wrongchoice{visual radiation}
        \wrongchoice{infrared waves}
        \wrongchoice{radio waves}
    \end{choices}
\end{question}
}

\element{morgans}{
\begin{question}{test4G-q08}
    Venus' rotation is:
    \begin{choices}
        \wrongchoice{rapid and direct.}
        \wrongchoice{rapid and retrograde.}
      \correctchoice{slow and retrograde.}
        \wrongchoice{slow and direct.}
        \wrongchoice{zero (it does not rotate).}
    \end{choices}
\end{question}
}

\element{morgans}{
\begin{question}{test4G-q09}
    Which one of the following best describes the Valles Marineris?
    \begin{choices}
        \wrongchoice{200 km long}
      \correctchoice{4,000 km long}
        \wrongchoice{10,000 km long}
        \wrongchoice{a trench going completely around the planet}
    \end{choices}
\end{question}
}

\element{morgans}{
\begin{question}{test4G-q10}
    Which planet has a retrograde rotation?
    \begin{choices}
        \wrongchoice{Mercury}
      \correctchoice{Venus}
        \wrongchoice{Earth}
        \wrongchoice{Mars}
    \end{choices}
\end{question}
}

\element{morgans}{
\begin{question}{test4G-q11}
    Which of the following is \emph{not} one of the factors which distinguishes the Jovian planets from the terrestrial planets?
    \begin{choices}
        \wrongchoice{the Jovians are more massive}
        \wrongchoice{the Jovians are colder}
      \correctchoice{the Jovians have higher densities}
        \wrongchoice{the Jovians have stronger magnetic fields}
    \end{choices}
\end{question}
}

\element{morgans}{
\begin{question}{test4G-q12}
    The size of material comprising Saturn's rings is similar to:
    \begin{choices}
        \wrongchoice{the satellites of Saturn.}
        \wrongchoice{aircraft carriers}
      \correctchoice{specks of dust.}
        \wrongchoice{individual atoms.}
    \end{choices}
\end{question}
}

\element{morgans}{
\begin{question}{test4G-q13}
    Which pair of Jovian satellites are less dense than any of the other pairs?
    \begin{choices}
        \wrongchoice{Io and Callisto}
        \wrongchoice{Europa and Ganymede}
        \wrongchoice{Io and Europa}
        \wrongchoice{Io and Ganymede}
      \correctchoice{Ganymede and Callisto}
    \end{choices}
\end{question}
}

\element{morgans}{
\begin{question}{test4G-q14}
    Which of the following support the idea that Saturn's rings are \emph{not} solid? 
    \begin{choices}
        \wrongchoice{they disappear when seen edge-on}
      \correctchoice{stars generally can be seen through them}
        \wrongchoice{Doppler shift measurements show the rotation velocity to disagree with Kepler's third law}
        \wrongchoice{the rings are located outside the Roche limit}
    \end{choices}
\end{question}
}

\element{morgans}{
\begin{question}{test4G-q15}
    What atmospheric gas is responsible for the bluish color of Uranus and Neptune?
    \begin{choices}
        \wrongchoice{ammonia}
        \wrongchoice{water vapor}
        \wrongchoice{carbon dioxide}
      \correctchoice{methane}
        \wrongchoice{hydrogen}
    \end{choices}
\end{question}
}

\element{morgans}{
\begin{question}{test4G-q16}
    Which Neptunian satellite is known to have an atmosphere?
    \begin{choices}
      \correctchoice{Triton}
        \wrongchoice{Nereid}
        \wrongchoice{Iapetus}
        \wrongchoice{Charon}
        \wrongchoice{Titan}
    \end{choices}
\end{question}
}

\element{morgans}{
\begin{question}{test4G-q17}
    Which of the following can \emph{not} be determined from Earth based studies of asteroids?
    \begin{choices}
      \correctchoice{internal temperature}
        \wrongchoice{shape}
        \wrongchoice{size}
        \wrongchoice{surface composition}
        \wrongchoice{surface temperature}
    \end{choices}
\end{question}
}

\element{morgans}{
\begin{question}{test4G-q18}
    The Kuiper belt of comets is:
    \begin{choices}
        \wrongchoice{the modern name for the Oort cloud.}
      \correctchoice{located at the edge of our planetary system.}
        \wrongchoice{consists of long-period comets.}
        \wrongchoice{has a mass 10 times that of the Earth.}
    \end{choices}
\end{question}
}

\element{morgans}{
\begin{question}{test4G-q19}
    Meteorites are composed of:
    \begin{choices}
        \wrongchoice{hydrogen ices.}
        \wrongchoice{helium ices.}
      \correctchoice{heavy elements like rocks and metals.}
        \wrongchoice{unknown elements, unlike any found on the Earth.}
    \end{choices}
\end{question}
}

\element{morgans}{
\begin{question}{test4G-q20}
    Terrestrial planets formed from the coalescing of:
    \begin{choices}
        \wrongchoice{asteroids.}
        \wrongchoice{the solar nebula.}
      \correctchoice{planetesimals.}
        \wrongchoice{planetoids.}
        \wrongchoice{minor bodies.}
    \end{choices}
\end{question}
}


\begin{comment}
    Fill In
    Place the most appropriate word or words in the blank. You may have to click on the blank to activate it before you start typing in your answer.
    The pre-planetary bodies which coalesced to form the Earth are known as .

    The is the point of origin of an earthquake.

    percent of the Earth's atmosphere is made up of oxygen.

    seismic waves cannot travel through liquid materials.

    The type of surface on Mercury directly opposite Caloris Basin is known as .

    is the main atmospheric constituent of Mars.

    The largest apparent gap visible in Saturn's rings as observed from Earth is called the .

    Pluto was discovered by .

    The name of the group of cometary bodies hypothesized to be orbiting the Sun at a great distance is the .

    The main component of Titan's atmosphere is .
\end{comment}


\endinput


