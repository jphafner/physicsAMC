
%% University of Northern Iowa
%%  Morgans Astronomy Exams
%%--------------------------------------------------

%% this section contains 15 problems

\element{morgans}{
\begin{question}{test2B-Q01}
    How does a gamma-ray photon,
        produced in the core of the Sun,
        emerge at the surface as a visible light photon?
    \begin{choices}
      \correctchoice{it loses energy through absorptions and re-emissions}
        \wrongchoice{it gains energy through absorptions and re-emissions}
        \wrongchoice{it does not actually change, all photons from the surface of the Sun are gamma-ray photons}
        \wrongchoice{it is not a gamma-ray photon, since all photons produced in the core of the Sun are visible light photons}
    \end{choices}
\end{question}
}

\element{morgans}{
\begin{question}{test2B-Q02}
    The difference between ``normal'' hydrogen and ``heavy'' hydrogen (deuterium) is an additional:
    \begin{choices}
      \correctchoice{neutron.}
        \wrongchoice{proton.}
        \wrongchoice{electron.}
        \wrongchoice{neutrino.}
        \wrongchoice{nothing; they are identical.}
    \end{choices}
\end{question}
}

\element{morgans}{
\begin{question}{test2B-Q03}
    The continuous spectrum of the Sun comes from the:
    \begin{choices}
      \correctchoice{interior.}
        \wrongchoice{photosphere.}
        \wrongchoice{chromosphere.}
        \wrongchoice{corona.}
    \end{choices}
\end{question}
}

\element{morgans}{
\begin{question}{test2B-Q04}
    A complete period of the solar magnetic field cycle takes:
    \begin{choices}
        \wrongchoice{one year.}
        \wrongchoice{11 years.}
      \correctchoice{22 years.}
        \wrongchoice{33 years.}
        \wrongchoice{44 years.}
    \end{choices}
\end{question}
}

\element{morgans}{
\begin{question}{test2B-Q05}
    Star $A$ has a parallax of 0.5 seconds of arc,
        while Star $B$ has a parallax of 0.6 seconds of arc. 
    Which star is more luminous?
    \begin{choices}
        \wrongchoice{A}
        \wrongchoice{B}
        \wrongchoice{Both the same luminosity}
      \correctchoice{Not possible to determine luminosity from the information given}
    \end{choices}
\end{question}
}

\element{morgans}{
\begin{question}{test2B-Q06}
    If a stellar spectrum has lines of neutral helium but not ionized helium,
        the star is of spectral type:
    \begin{choices}
        \wrongchoice{O.}
      \correctchoice{B.}
        \wrongchoice{A.}
        \wrongchoice{G.}
        \wrongchoice{M.}
    \end{choices}
\end{question}
}

\element{morgans}{
\begin{question}{test2B-Q07}
    Stellar surface temperatures range from:
    \begin{choices}
        \wrongchoice{5,000-6,000 K.}
        \wrongchoice{4,000-10,000 K.}
        \wrongchoice{3,000-20,000 K.}
      \correctchoice{2,000-40,000 K.}
        \wrongchoice{100-100,000 K.}
    \end{choices}
\end{question}
}

\element{morgans}{
\begin{question}{test2B-Q08}
    Star $A$ and Star $B$ are both F3V type stars. 
    Yet Star $A$ has an apparent magnitude of +6,
        while Star $B$ has an apparent magnitude of +8. 
    Which star is closer?
    \begin{choices}
      \correctchoice{A}
        \wrongchoice{B}
        \wrongchoice{they are both at the same distance}
        \wrongchoice{it is not possible to determine distance from the information given}
    \end{choices}
\end{question}
}

\element{morgans}{
\begin{question}{test2B-Q09}
    The absorption lines in the spectrum of a star are formed in its:
    \begin{choices}
        \wrongchoice{core.}
      \correctchoice{photosphere.}
        \wrongchoice{chromosphere.}
        \wrongchoice{corona.}
    \end{choices}
\end{question}
}

\element{morgans}{
\begin{question}{test2B-Q10}
    For stars like the Sun energy is transported by \rule[-0.1pt]{4em}{0.1pt} in the core and by \rule[-0.1pt]{4em}{0.1pt} in the envelope.
    \begin{choices}
        \wrongchoice{radiation, radiation}
        \wrongchoice{convection, radiation}
        \wrongchoice{convection, convection}
      \correctchoice{radiation, convection}
    \end{choices}
\end{question}
}

\element{morgans}{
\begin{question}{test2B-Q11}
    A star with more extended outer layers will have \rule[-0.1pt]{4em}{0.1pt} surface gravity than a star with the same mass but a smaller radius.
    \begin{choices}
        \wrongchoice{the same}
        \wrongchoice{a larger}
      \correctchoice{a smaller}
    \end{choices}
\end{question}
}

\element{morgans}{
\begin{question}{test2B-Q12}
    The main sequence lifetime of a star is dependent upon its:
    \begin{choices}
      \correctchoice{mass.}
        \wrongchoice{composition.}
        \wrongchoice{temperature.}
        \wrongchoice{apparent magnitude.}
        \wrongchoice{velocity.}
    \end{choices}
\end{question}
}

\element{morgans}{
\begin{question}{test2B-Q13}
    During the formation of a star,
        the contraction stops when:
    \begin{choices}
        \wrongchoice{the star collapses into a black hole.}
        \wrongchoice{the star collapses into a white dwarf.}
      \correctchoice{hydrogen burning becomes the dominant energy source.}
        \wrongchoice{helium burning becomes the dominant energy source.}
        \wrongchoice{the star becomes a T Tauri star.}
    \end{choices}
\end{question}
}

\element{morgans}{
\begin{question}{test2B-Q14}
    When the hydrogen in the core of a star has been converted to helium,
        the core will next:
    \begin{choices}
      \correctchoice{contract.}
        \wrongchoice{expand.}
        \wrongchoice{burn helium.}
        \wrongchoice{decrease in temperature.}
        \wrongchoice{explode.}
    \end{choices}
\end{question}
}

\element{morgans}{
\begin{question}{test2B-Q15}
    A star will fuse helium for:
    \begin{choices}
        \wrongchoice{a longer time than it fused hydrogen.}
      \correctchoice{a shorter time than it fused hydrogen.}
        \wrongchoice{the same amout of time it took to fuse hydrogen.}
        \wrongchoice{only a fraction of a second.}
    \end{choices}
\end{question}
}

\element{morgans}{
\begin{question}{test2B-Q16}
    Each time a form of nuclear fuel is exhausted in the core of a star,
        the star:
    \begin{choices}
        \wrongchoice{returns to the main sequence.}
      \correctchoice{returns to the red giant branch.}
        \wrongchoice{returns to the white dwarf region.}
        \wrongchoice{explodes in a supernova explosion.}
        \wrongchoice{ejects a planetary nebula.}
    \end{choices}
\end{question}
}

\element{morgans}{
\begin{question}{test2B-Q17}
    A black hole is called that because:
    \begin{choices}
        \wrongchoice{its color is black.}
        \wrongchoice{its energy output depends on its temperature.}
      \correctchoice{photons can not be emitted from it.}
        \wrongchoice{they exist only in the dark reaches of space where there are no stars.}
    \end{choices}
\end{question}
}

\element{morgans}{
\begin{question}{test2B-Q18}
    A white dwarf will cool to become a black dwarf in several:
    \begin{choices}
        \wrongchoice{thousand years.}
        \wrongchoice{million years.}
      \correctchoice{billion years.}
    \end{choices}
\end{question}
}

\element{morgans}{
\begin{question}{test2B-Q19}
    The \rule[-0.1pt]{4em}{0.1pt} is an important supernova remnant. 
    \begin{choices}
        \wrongchoice{Orion nebula}
      \correctchoice{Crab nebula}
        \wrongchoice{Pleiades nebula}
        \wrongchoice{Horsehead nebula}
        \wrongchoice{Andromeda nebula}
    \end{choices}
\end{question}
}

\element{morgans}{
\begin{question}{test2B-Q20}
    Binary X-ray sources are known to be binary because:
    \begin{choices}
        \wrongchoice{the two stars are observed visually as visual binary stars.}
        \wrongchoice{they are astrometric binaries.}
      \correctchoice{eclipses are observed.}
        \wrongchoice{the name is a misnomer since no X-ray objects are known to be binary.}
    \end{choices}
\end{question}
}

\begin{comment}
    Fill In
    Place the most appropriate word or words in the blank. You may have to click on the blank to activate it before you start typing in your answer.
     
    The astronomical system of brightness measurements is called the system.

    A subatomic particle emitted in nuclear reactions; too few are observed to be emitted by the Sun: .

    A thin layer of hot gas just above the visible surface of the Sun is called the .

    The apparent magnitude a star would have if it were at a distance of 10 parsecs is called the .

    The specific temperature required for hydrogen burning in the Sun is .

    A loose cluster of stars found near a region of star formation is known as a(n) .

    The compact, degenerate remnant of a low-mass star is called a(n) .

    The main element observed in the spectra of Type II supernovae is .

    A is an object which has a gravitation force so strong that light can not escape from it.

    A very compact, dense stellar remnant having a mass between 1.5 and 3 solar masses is called a(n) 
\end{comment}

\endinput



