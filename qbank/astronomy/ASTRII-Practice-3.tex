SAMPLE EXAM 3: FOR STELLAR EVOLUTION
 
   1. The masses of stars on the main sequence __________ from the
      lower right to the upper left.
      a.  increase
      b.  decrease
      c.  are all the same
      d.  are randomly distributed
      
   2. The fundamental quantity which determines a star's central
      pressure and temperature is its
      a.  mass.
      b.  luminosity.
      c.  surface temperature.
      d.  chemical composition.
      e.  radius.
      
   3. The Russell-Vogt theorem (about chemical composition) states that the properties of a star
      at any given age depend completely upon the star's
      a.  luminosity and radius.
      b.  mass and luminosity.
      c.  mass, chemical composition, and radius.
      d.  radius, and chemical composition.
      e.  mass and chemical composition.
      
   4. The lifetime of a star is determined by its initial
      a.  temperature and luminosity.
      b.  temperature and radius.
      c.  temperature and mass.
      d.  mass and luminosity.
      e.  mass and radius.
      
   5. Stars evolve because of changes in
      a.  chemical composition of the core.
      b.  luminosity of the star.
      c.  mass of the star.
      d.  chemical composition of the surface.
      e.  surface temperature.
      
   6. The triple alpha reaction converts _________ into __________.
      a.  hydrogen, helium
      b.  helium, hydrogen
      c.  helium, carbon
      d.  carbon, nitrogen
      e.  hydrogen, nitrogen
      
   7. A nuclear reaction which occurs at sometime after hydrogen is
      exhausted in the core is the
      a.  CNO cycle.
      b.  triple alpha reaction.
      c.  helium-iron chain.
      d.  electron-positron cycle.
      e.  tritium-alpha cycle.
      
 
   8. The temperature of a star's core will __________ as the star
      fuses heavier elements.
      a.  increase
      b.  decrease
      c.  remain the same
      
   9. Which one of the following is NOT a class of star cluster?
      a.  galactic clusters
      b.  open clusters
      c.  closed clusters
      d.  globular clusters
      e.  stellar associations
      
  10. Which of the following is NOT an assumption made in the study
      of star clusters?
      a.  all the stars are at the same distance
      b.  all the stars formed at the same time
      c.  all the stars have the same chemical composition
      d.  all the stars have the same mass
      
  11. Which of the following can NOT be obtained from a
      color-magnitude diagram?
      a.  a cluster's age
      b.  a cluster's distance
      c.  the relative brightnesses of stars in a cluster
      d.  the cluster velocity
      
  12. ____________________________________________
NOTE: This requires an image
    
      |                                              |
            
      What would be the correct order of the clusters shown above,
      if they are arranged from youngest to oldest?
      a.  ABC
      b.  ACB
      c.  BAC
      d.  BCA
      e.  CAB
      f.  CBA
      
  13. The evolution of a star depends primarily on the star's
      a.  radius.
      b.  mass.
      c.  luminosity.
      d.  density.
      e.  temperature.
 
 
  14. Stars on the upper end of the main sequence next evolve into
      a.  red dwarfs.
      b.  lower main sequence stars.
      c.  solar-type stars.
      d.  white dwarfs.
      e.  red giants.
      
  15. The main sequence turn-off is useful in determining a
      cluster's
      a.  mass.
      b.  age.
      c.  distance.
      d.  apparent magnitude.
      e.  velocity.
      
  16. The H-R diagram for cluster A consists of stars on the main
      sequence, while the diagram for cluster B has stars both on
      the main sequence and just above the main sequence at lower
      temperatures. Which cluster is older?
      a.  cluster A
      b.  cluster B
      c.  both clusters are the same age
      d.  more information is required
      
  17. A T Tauri star is one which is
      a.  like the Sun.
      b.  variable and shedding mass.
      c.  old and shedding mass.
      d.  becoming a white dwarf.
      e.  a main sequence star.
      
  18. Protostars in dark, dusty regions may be studied in the
      __________ spectral region.
      a.  X-ray
      b.  ultraviolet
      c.  visual
      d.  infrared
      e.  gamma-ray
      
  19. During the formation of a star, the contraction stops when
      a.  the star collapses into a black hole.
      b.  the star collapses into a white dwarf.
      c.  hydrogen burning becomes the dominant energy source.
      d.  helium burning becomes the dominant energy source.
      e.  the star becomes a T Tauri star.
      
  20. Hydrogen burning for a Sun-like star lasts approximately
      a.  one million years.
      b.  ten million years.
      c.  one hundred million years.
      d.  one billion years.
      e.  ten billion years.
      
  21. What types of stars are found on the zero-age main sequence
      (ZAMS)?
      a.  red giants
      b.  protostar
      c.  newly formed stars
      d.  stars that have not yet started to burn hydrogen
      e.  white dwarfs
      
  22. When the hydrogen in the core of a star has been converted to
      helium, the core will next
      a.  contract.
      b.  expand.
      c.  burn helium.
      d.  decrease in temperature.
      e.  explode.
      
  23. Which of the following occurs during and immediately after the
      phase of the hydrogen burning shell?
      a.  the core shrinks until the star becomes a white dwarf
      b.  the helium flash occurs
      c.  the core temperature decreases while the envelope temperature increases
      d.  the star becomes a supernova
      e.  the envelope expands and cools, and the star becomes a red
          giant
      
  24. As a degenerate gas is heated, it will
      a.  expand.
      b.  contract.
      c.  neither expand nor contract.
      d.  oscillate.
      
  25. The ignition of helium in the degenerate core of a one solar
      mass star produces
      a.  a supernova.
      b.  a black hole.
      c.  the formation of new heavy elements.
      d.  a helium flash.
      e.  nothing much - the core expands, cools, and continues
          burning.
       
  26. In a degenerate electron gas the outward pressure which keeps
      the star from collapsing is
      a.  dependent upon temperature.
      b.  dependent upon mass.
      c.  independent of temperature.
      d.  independent of mass.
      
  27. Which of the following is NOT a characteristic of a red giant?
      a.  extended outer layers
      b.  degenerate core
      c.  cool core
      d.  cool surface temperature
      
 
  28. The horizontal branch is
      a.  a region on the H-R diagram where stars have roughly the same temperature.
      b.  a region on the H-R diagram where stars have roughly the same luminosity.
      c.  seen only in young clusters where protostars are evolving towards the main sequence.
      d.  the track that white dwarfs follow.
      
  29. After a Sun-like star enters its second red giant phase, its
      internal structure would consist of
      a.  a carbon core surrounded by a helium burning shell, which is surrounded by a hydrogen burning shell.
      b.  a helium core surrounded by a hydrogen burning shell.
      c.  a iron core surrounded by many different layers of shell burning.
      d.  a oxygen core surrounded by a carbon burning shell, a helium burning shell, and a hydrogen burning shell.
      
  30. A planetary nebula is
      a.  a collapsing gas cloud out of which planets will form.
      b.  an expanding gas cloud that was ejected by a star.
      c.  the cloud of gas blown off by a T Tauri type star.
      d.  a gas cloud which goes into the formation of red giants.
      
  31. The stellar remnant of a one solar mass star is a
      a.  white dwarf.
      b.  neutron star.
      c.  pulsar.
      d.  black hole.
      e.  main sequence star.
      
  32. High mass stars evolve more rapidly than low mass ones because
      the high mass stars
      a.  are larger
      b.  have higher core temperatures.
      c.  have higher core densities.
      d.  are made of more massive elements.
      e.  are in the lower right corner of the H-R diagram.
      
  33. In the most massive stars, the heaviest element which will be
      produced in the core will be
      a.  helium.
      b.  oxygen.
      c.  silicon.
      d.  iron.
      e.  uranium.
      
 
  34. Each time a form of nuclear fuel is exhausted in the core of a
      star, the star
      a.  returns to the main sequence.
      b.  returns to the red giant branch.
      c.  returns to the white dwarf region.
      d.  explodes in a supernova explosion.
      e.  ejects a planetary nebula.
      
  35. The most mass a white dwarf can have is about
      a.  1 Solar Mass.
      b.  1.4 Solar Mass.
      c.  3 Solar Masses.
      d.  10 Solar Masses.
      e.  there is no limit to the mass.
      
  36. What element is observed in the spectra of Type II supernova?
      a.  hydrogen
      b.  helium
      c.  iron
      d.  uranium
      
  37. A star which has a main sequence mass of 10 solar masses will
      most likely end up as
      a.  a T Tauri star.
      b.  a white dwarf.
      c.  a neutron star.
      d.  a black hole.
      
  38. If a neutron star has more mass than its mass limit, it will
      a.  expand catastrophically.
      b.  contract catastrophically.
      c.  begin a new phase of nuclear reactions.
      d.  nothing will happen.
      
  39. A black hole is called that because
      a.  its color is black.
      b.  its energy output depends on its temperature.
      c.  photons can not be emitted from it.
      d.  they exist only in the dark reaches of space where there
          are no stars.
      
  40. A neutron star's size is that of
      a.  the Sun.
      b.  the Earth.
      c.  the Earth's orbit.
      d.  the State of Iowa.
      e.  a typical city.
      
  41. The most massive stars are thought to end up as
      a.  black holes.
      b.  planetary nebulae.
      c.  neutron stars.
      d.  white dwarfs.
      
 
  42. Heavy elements which are mixed into the material from which
      new generations of stars may come primarily from
      a.  the big bang.
      b.  planetary nebulae.
      c.  supernovae.
      d.  super-neutron stars.
      e.  Wolf-Rayet stars.
      
 
  43. After the initial outburst of Supernova 1987A had diminished,
      the main source of luminosity was energy released from
      a.  radioactive decay.
      b.  the neutron star formed during the explosion.
      c.  the high temperature core which remains.
      d.  the expanding envelope.
      e.  the ejection of slowing moving neutrinos.
      
  44. What was the most energetic component of Supernova 1987A?
      a.  the visible light
      b.  the kinetic energy of the expanding gas cloud
      c.  the energy released in neutrinos
      d.  the X-ray energy
      
  45. Which of the following objects would have the highest
      gravitational redshift?
      a.  a white dwarf
      b.  the Sun
      c.  a neutron star
      d.  a black hole
      
  46. A white dwarf will cool to become a black dwarf in several
      a.  thousand years.
      b.  million years.
      c.  billion years.
      
  47. White dwarfs are composed mostly of
      a.  normal (perfect) gases.
      b.  degenerate gases.
      c.  equal amounts of normal (perfect) and degenerate gases.
      d.  hot, solid material.
      
  48. All novae are thought to involve a
      a.  white dwarf.
      b.  main sequence star.
      c.  supergiant.
      d.  neutron star.
      e.  black hole.
      
  49. Novae explosions are caused by
      a.  exploding white dwarfs.
      b.  interstellar matter falling onto the surface of a star, usually a white dwarf.
      c.  material falling into a black hole.
      d.  mass lost from a normal star falling onto a white dwarf companion.
      e.  strong flares on a stellar surface (large-scale ``solar'' flares).
      
  50. How can astronomers determine which type of supernovae they
      are observing?
      a.  Type I supernovae fade much more quickly than Type II
      b.  the brightnesses are different
      c.  the spectral features are different
      d.  by determining exactly which object produced the supernova
      
 
  51. The internal properties of a neutron star are most similar to
      those of a __________ star.
      a.  main sequence
      b.  red giant
      c.  red dwarf
      d.  white dwarf
      e.  solar-type
      f.  blue supergiant
      
  52. Stellar remnants with masses between 2 and 3 solar masses will
      be
      a.  white dwarfs.
      b.  neutron stars.
      c.  black holes.
      d.  planetary nebulae.
      
  53. Pulsars are known to be
      a.  pulsating white dwarfs.
      b.  pulsating neutron stars.
      c.  rotating white dwarfs.
      d.  rotating neutron stars.
      e.  rotating black holes.
      
  54. Binary X-ray sources are known to be binary because
      a.  the two stars are observed visually as visual binary
          stars.
      b.  they are astrometric binaries.
      c.  eclipses are observed.
      d.  the name is a misnomer since no X-ray objects are known to
          be binary.
      
  55. The accretion disk surrounding a neutron star is very hot due
      to compression caused by gravitational forces. This implies
      the object will emit strongly in which spectral region?
      a.  X ray
      b.  ultraviolet
      c.  visual
      d.  infrared
      e.  radio
      
  56. If the Sun were suddenly to be replaced by a solar-mass black
      hole the gravitational force on the Earth (1 A. U. away) would
      a.  double.
      b.  become so strong that the Earth would be "sucked" into the
          black hole.
      c.  decrease because black holes cause gravity at large distances to disappear.
      d.  remain the same.
      
  57. The Schwarzschild radius of a black hole is
      a.  the radius of the star when it is on the main sequence.
      b.  the distance from a black hole inside of which light cannot escape.
      c.  the theoretical size of the smallest possible white dwarf.
      d.  the size of a star when it begins hydrogen burning just prior to reaching the main sequence.
      e.  the size of the early protosun.
      
  58. From the outsider's point of view, in watching a star collapse
      to form a black hole, the collapse would appear to take
      a.  only a fraction of a second.
      b.  a few hours.
      c.  forever.
      
  59. Which of the following lists the stellar remnants in order of
      decreasing maximum mass?
      a.  neutron star, white dwarf, black hole
      b.  black hole, neutron star, white dwarf
      c.  white dwarf, black hole, neutron star
      d.  black hole, white dwarf, neutron star
      e.  they all have approximately the same mass
      
  60. A black hole is really
      a.  a star of temperature 0 K.
      b.  densely packed matter.
      c.  strongly curved space.
      d.  at the center of most stars and provides the star's
          energy.
      
  61. How is it possible to detect the presence of a black hole?
      a.  using a large optical telescope to see its surface
      b.  using a large radio telescope to see its surface
      c.  by its effect upon other objects around it
      d.  by watching material cross its event horizon


Answer Key for Sample Exam 3
 
01. a  
02. a  
03. e  
04. d  
05. a  
06. c  
07. b  
08. a  
09. C
10. d   
11. d   
12. e   
13. b   
14. e   
15. b    
16. a   
17. b        
18. d        
19. c        
20. e        
21. c        
22. a        
23. e        
24. c         
25. d         
26. c         
27. c         
28. b         
29. a         
30. b     
31. a         
32. b         
33. d         
34. b         
35. b         
36. a         
37. c         
38. b         
39. c         
40. e         
41. a         
42. c         
43. a    
44. c   
45. d   
46. c   
47. b   
48. a   
49. d   
50. c   
51. d   
52. b   
53. d   
54. c   
55. a   
56. d   
57. b   
58. c   
59. b
60. c   
61. c   

