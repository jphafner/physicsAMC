
%% Astronomy:
%%--------------------------------------------------

%% Chapter 01: Learning Objectives and Study Questions for
%%------------------------------------------------------------


% 1. Convert numbers from standard to scientific (powers of 10) notation and vice versa.
% 2. Perform basic computations (+, -, ×, ÷, ˆ) with numbers expressed in scientific notation on a calculator.
% 3. Perform conversions among common time and distance units.
% 4. Locate objects on a map of the night sky using declination and right ascension.
% 5. Correlate common time intervals (e.g., days, months, seasons, and years) with the corresponding cycles of motion in the Earth-Sun-Moon system.
% 6. Explain how the inclination of Earth's axis causes the seasons and how the strength of seasonality would differ if Earth's axis were inclined more or less than it is today.
% 7. Predict the position of the Moon in the sky given your local time and the Moon's current phase.
% 8. Predict if an eclipse is possible at a given time in the lunar cycle and, if so, which type (solar or lunar) it would be.


\element{astr}{
\begin{question}{ch01-q01}
    A \rule[-0.1pt]{4em}{0.1pt} is the largest of the following distance units.
    \begin{choices}
        \wrongchoice{meter}
        \wrongchoice{kilometer}
        \wrongchoice{astronomical unit}
        \wrongchoice{light year}
        \wrongchoice{parsec}
    \end{choices}
\end{question}
}

\element{astr}{
\begin{question}{ch01-q02}
    One astronomical unit equals \SI{150 000 000}{\kilo\meter}; or $1.5\times 10^x\,\si{\kilo\meter}$, where $x$ equals \rule[-0.1pt]{4em}{0.1pt}.
    \begin{multicols}{3}
    \begin{choices}
        \wrongchoice{3}
        \wrongchoice{6}
        \wrongchoice{7}
        \wrongchoice{8}
        \wrongchoice{9}
    \end{choices}
    \end{multicols}
\end{question}
}

\element{astr}{
\begin{question}{ch01-q03}
    The star Proxima Centauri is approximately 4.2 light years, or \rule[-0.1pt]{4em}{0.1pt} pc,
        away from Earth.
    \begin{multicols}{3}
    \begin{choices}
        \wrongchoice{0.77}
        \wrongchoice{1.0}
        \wrongchoice{1.3}
        \wrongchoice{7.5}
        \wrongchoice{13}
    \end{choices}
    \end{multicols}
\end{question}
}

\element{astr}{
\begin{question}{ch01-q04}
    Earth rotates on its axis once every:
    \begin{multicols}{2}
    \begin{choices}
        \wrongchoice{day}
        \wrongchoice{week}
        \wrongchoice{month}
        \wrongchoice{season}
        \wrongchoice{year}
    \end{choices}
    \end{multicols}
\end{question}
}

\element{astr}{
\begin{question}{ch01-q05}
    Earth revolves around the Sun once every:
    \begin{multicols}{2}
    \begin{choices}
        \wrongchoice{day}
        \wrongchoice{week}
        \wrongchoice{month}
        \wrongchoice{season}
        \wrongchoice{year}
    \end{choices}
    \end{multicols}
\end{question}
}

\element{astr}{
\begin{question}{ch01-q06}
    The Moon revolves around the Earth and rotates on its axis once every:
    \begin{multicols}{2}
    \begin{choices}
        \wrongchoice{day}
        \wrongchoice{week}
        \wrongchoice{month}
        \wrongchoice{season}
        \wrongchoice{year}
    \end{choices}
    \end{multicols}
\end{question}
}

\element{astr}{
\begin{question}{ch01-q07}
    The tilt of Earth’s axis with respect to its orbital plane causes the seasons,
        in part, by changing the:
    \begin{choices}
        \wrongchoice{distance between the Earth and Sun}
        \wrongchoice{angle at which the Sun’s rays strike Earth’s surface}
        \wrongchoice{orbital alignment of the Moon}
        \wrongchoice{speed at which Earth orbits the Sun}
        \wrongchoice{strengths of Earth’s tides}
    \end{choices}
\end{question}
}

8. Which season begins on the day when Earth’s North Pole is pointed most directly
towards the Sun?
A. spring
B. summer
C. fall
D. winter
E. indeterminate, cannot tell
9. Which season is just
beginning for people in
the southern hemisphere
when Earth is in the
position shown in the
adjacent sketch?
A. spring
B. summer
C. fall
D. winter
E. indeterminate, cannot tell
10. The Moon completes one cycle of its phases every _____.
A. day
B. week
C. month
D. season
E. year


11. A full moon will just begin to rise above the horizon at _____.
A. dawn, 6am
B. noon, 12pm
C. dusk, 6pm
D. midnight, 12am
E. indeterminate, cannot tell
12. What time will the Moon shown in the
accompanying sketch set?
A. dawn, 6am
B. noon, 12pm
C. dusk, 6pm
D. midnight, 12am
E. indeterminate, cannot tell
13. During a lunar eclipse the light of the _____ is dimmed or blacked out.
A. Moon
B. Sun
C. Earth
D. nearby planets
E. stars
14. Eclipses occur at times of new or full moons when _____.
A. Earth’s north pole is pointed directly towards the Sun
B. Earth’s north pole is pointed directly away from the Sun
C. Earth’s north pole is pointed neither towards or away from the Sun
D. the Sun lies in the plane of Earth’s orbit
E. the Moon lies in the plane of Earth’s orbit
15. If an eclipse is occurring at the time shown
in the sketch below, it will be a _____
eclipse.
A. lunar
B. solar
C. partial
D. total
E. penumbral






\endinput



