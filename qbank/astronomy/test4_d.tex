
%% university of northern iowa
%%  morgans astronomy exams
%%--------------------------------------------------

%% this section contains 20 problems

\element{morgans}{
\begin{question}{test4A-q01}
    Which of the following is NOT a characteristic of a terrestrial planet?
    \begin{choices}
        \wrongchoice{high density}
        \wrongchoice{hard surface}
        \wrongchoice{small}
      \correctchoice{far from the Sun}
    \end{choices}
\end{question}
}

\element{morgans}{
\begin{question}{test4A-q02}
    The driving force behind plate motions is currently thought to be:
    \begin{choices}
        \wrongchoice{tidal forces.}
        \wrongchoice{differential gravitational forces.}
        \wrongchoice{gravity.}
      \correctchoice{convection.}
        \wrongchoice{continental drift.}
    \end{choices}
\end{question}
}

\element{morgans}{
\begin{question}{test4A-q03}
    One effect produced by tides is:
    \begin{choices}
        \wrongchoice{differentiation.}
        \wrongchoice{the dynamo effect which produces planetary magnetic fields.}
      \correctchoice{heating of the planet's interior.}
        \wrongchoice{an increase in a planet's rotation speed.}
    \end{choices}
\end{question}
}

\element{morgans}{
\begin{question}{test4A-q04}
    The most common gas in the Earth's early atmosphere is thought to have been:
    \begin{choices}
        \wrongchoice{oxygen.}
      \correctchoice{hydrogen.}
        \wrongchoice{water.}
        \wrongchoice{carbon dioxide.}
        \wrongchoice{ozone.}
    \end{choices}
\end{question}
}

\element{morgans}{
\begin{question}{test4A-q05}
    The slowing of the Moon's rate of rotation will cause the Moon to:
    \begin{choices}
        \wrongchoice{get closer to the Earth until it reaches a stable orbit.}
      \correctchoice{get further from the Earth until it reaches a stable orbit.}
        \wrongchoice{get closer and eventually crash into the Earth.}
        \wrongchoice{be thrown away from the Earth.}
    \end{choices}
\end{question}
}

\element{morgans}{
\begin{question}{test4A-q06}
    Mercury has prominent and extensive scarps---another word for:
    \begin{choices}
      \correctchoice{cliffs.}
        \wrongchoice{valleys.}
        \wrongchoice{mountains.}
        \wrongchoice{volcanos.}
    \end{choices}
\end{question}
}

\element{morgans}{
\begin{question}{test4A-q07}
    Which of the following is \emph{not} a feature of Venus' surface?
    \begin{choices}
        \wrongchoice{volcanoes}
        \wrongchoice{large impact craters}
        \wrongchoice{lava flows}
      \correctchoice{river channels}
    \end{choices}
\end{question}
}

\element{morgans}{
\begin{question}{test4A-q08}
    If the average density of Venus is similar to that of Earth,
        and if the surface material on Venus is made up of very low density material,
        what may one conclude about the interior of Venus?
    \begin{choices}
        \wrongchoice{nothing}
        \wrongchoice{the interior is composed of low density material}
      \correctchoice{the interior has a large, dense core}
        \wrongchoice{the planet is undifferentiated}
        \wrongchoice{the planet has a hot, rapidly rotating core producing the planet's magnetic field}
    \end{choices}
\end{question}
}

\element{morgans}{
\begin{question}{test4A-q09}
    The major constituent of the Martian atmosphere is:
    \begin{choices}
        \wrongchoice{water.}
      \correctchoice{carbon dioxide.}
        \wrongchoice{nitrogen.}
        \wrongchoice{oxygen.}
        \wrongchoice{none of the above since they are about equal in abundance}
    \end{choices}
\end{question}
}

\element{morgans}{
\begin{question}{test4A-q10}
    Which of the following is the correct order of planets arranged in order of increasing atmospheric pressure?
    \begin{choices}
        \wrongchoice{Mercury, Venus, Earth, Mars}
        \wrongchoice{Mercury, Earth, Mars, Venus}
      \correctchoice{Mercury, Mars, Earth, Venus}
        \wrongchoice{Mars, Mercury, Earth, Venus}
    \end{choices}
\end{question}
}

\element{morgans}{
\begin{question}{test4A-q11}
    Which planet has the slowest rotation rate?
    \begin{choices}
        \wrongchoice{Mercury}
      \correctchoice{Venus}
        \wrongchoice{Earth}
        \wrongchoice{Mars}
    \end{choices}
\end{question}
}

\element{morgans}{
\begin{question}{test4A-q12}
    How do we know what the internal structures of the Jovian planets are like?
    \begin{choices}
        \wrongchoice{probes have been sent into the interiors of the planets and have returned data about the conditions}
        \wrongchoice{astronomers use the Earth's internal structure as a basis of comparison}
        \wrongchoice{astronomers examine the composition of the satellites of these planets}
      \correctchoice{astronomers use information about the physical characteristics as well as laws of physics to obtain theoretical models}
    \end{choices}
\end{question}
}

\element{morgans}{
\begin{question}{test4A-q13}
    The Galilean satellites consist of:
    \begin{choices}
        \wrongchoice{rock.}
        \wrongchoice{ice.}
      \correctchoice{a rock/ice mixture.}
        \wrongchoice{liquid nitrogen oceans.}
        \wrongchoice{gaseous materials.}
    \end{choices}
\end{question}
}

\element{morgans}{
\begin{question}{test4A-q14}
    The source of heat which produces Io's volcanic activity is:
    \begin{choices}
        \wrongchoice{the solar wind.}
      \correctchoice{the Jupiter-Io tidal force.}
        \wrongchoice{Jupiter's rapid rotation rate.}
        \wrongchoice{radioactivity.}
        \wrongchoice{the release of gravitational potential energy from the satellite's formation.}
    \end{choices}
\end{question}
}

\element{morgans}{
\begin{question}{test4A-q15}
    Which of the following molecules is found in the spectrum of Uranus and Neptune?
    \begin{choices}
        \wrongchoice{methane}
        \wrongchoice{carbon dioxide}
        \wrongchoice{water vapor}
        \wrongchoice{nitrogen}
        \wrongchoice{ozone}
    \end{choices}
\end{question}
}

\element{morgans}{
\begin{question}{test4A-q16}
    Which of the following describes the rings of Neptune?
    \begin{choices}
        \wrongchoice{very extensive, well formed circles and made up of bright particles}
      \correctchoice{asymmetric circles which are uneven and lumpy}
        \wrongchoice{smooth, symmetric circles made up of dark particles}
        \wrongchoice{Neptune does not have any rings}
    \end{choices}
\end{question}
}

\element{morgans}{
\begin{question}{test4A-q17}
    Which object is Pluto most similar to in structure and composition?
    \begin{choices}
        \wrongchoice{Io}
        \wrongchoice{Jupiter}
        \wrongchoice{Earth}
      \correctchoice{Triton}
        \wrongchoice{Uranus}
    \end{choices}
\end{question}
}

\element{morgans}{
\begin{question}{test4A-q18}
    What is the Oort Cloud?
    \begin{choices}
        \wrongchoice{another name for the Great Red Spot on Jupiter}
      \correctchoice{the name of the spherical cloud of comets surrounding the Sun}
        \wrongchoice{the name given to the Great Dark Spot on Neptune}
        \wrongchoice{a giant cloud structure observed in Venus's atmosphere}
        \wrongchoice{a bright region of the sky caused by the presence of a cloud of asteroids}
    \end{choices}
\end{question}
}

\element{morgans}{
\begin{question}{test4A-q19}
    The fact that volatile gasses are present in comets today tells us that comets have been:
    \begin{choices}
        \wrongchoice{subjected to high temperatures.}
        \wrongchoice{hot in the past, but are cool today.}
        \wrongchoice{cold in the past, but hot today.}
      \correctchoice{cold in the past and still cold today.}
    \end{choices}
\end{question}
}

\element{morgans}{
\begin{question}{test4A-q20}
    What does the presence of radioactive isotopes in the solar system tell us?
    \begin{choices}
      \correctchoice{a supernova must have occured soon before the formation of the solar system}
        \wrongchoice{the solar system formed in a catastrophic process}
        \wrongchoice{the solar nebula had a typical composition compared to other interstellar clouds}
        \wrongchoice{the Sun was unusually hot during its formation}
    \end{choices}
\end{question}
}

\begin{comment}
    Fill In
    Place the most appropriate word or words in the blank. You may have to click on the blank to activate it before you start typing in your answer.
     
    is the planet that has not yet been visited by a probe.

    The crust is relatively thin and made up mainly of basaltic material.

    The Earth's atmosphere is not original, but is a atmosphere.

    The mobile upper level of the Earth's mantle is the .

    Small rock fragments cemented together by the pressure of meteoritic impacts are known as .

    The gas which constitutes the primary gas in Venus' atmosphere is .

    The Jovian satellite having volcanos is .

    The color of Neptune is .

    Small bodies orbiting the Sun in circular orbits located in the ecliptic are called .

    The particle from space which strikes the Earth's surface is known as a(n)
\end{comment}


\endinput


