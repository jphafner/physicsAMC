
%% University of Northern Iowa
%%  Morgans Astronomy Exams
%%--------------------------------------------------

%% this section contains XX problems

\element{morgans}{
\begin{question}{test1B-Q01}
    The Moon:
    \begin{choices}
        \wrongchoice{may appear anywhere in the sky.}
      \correctchoice{always appears within a few degrees of the zodiac.}
        \wrongchoice{always appears within a few degrees of the celestial equator.}
        \wrongchoice{generally appears opposite the Sun.}
    \end{choices}
\end{question}
}

\element{morgans}{
\begin{question}{test1B-Q02}
    Where must an observer be located on the Earth to view the entire sky over the course of a year? 
    \begin{choices}
        \wrongchoice{the north pole}
        \wrongchoice{the south pole}
      \correctchoice{the equator}
        \wrongchoice{anywhere on the Earth}
    \end{choices}
\end{question}
}

\element{morgans}{
\begin{question}{test1B-Q03}
    Planet $X$ has its rotation axis perpendicular to its orbital plane. 
    Its seasons would be:
    \begin{choices}
        \wrongchoice{shorter than those on Earth}
        \wrongchoice{longer than those on Earth}
        \wrongchoice{the same as those on Earth}
      \correctchoice{constant}
    \end{choices}
\end{question}
}

\element{morgans}{
\begin{question}{test1B-Q04}
    If the Moon rises at 6:00 a.m., what phase is it? 
    \begin{choices}
      \correctchoice{new}
        \wrongchoice{first quarter}
        \wrongchoice{full}
        \wrongchoice{third quarter}
    \end{choices}
\end{question}
}

\element{morgans}{
\begin{question}{test1B-Q05}
    If the Moon did \emph{not} rotate we would observe:
    \begin{choices}
        \wrongchoice{the same as we now observe.}
        \wrongchoice{only the lunar back side.}
        \wrongchoice{the lunar north polar region.}
      \correctchoice{both the front and back side of the Moon.}
    \end{choices}
\end{question}
}

\element{morgans}{
\begin{question}{test1B-Q06}
    If you were to compare the accuracy of Copernicus' model to that of Ptolomy you'd find that:
    \begin{choices}
        \wrongchoice{Copernicus' model was more accurate.}
        \wrongchoice{Copernicus' model was less accurate.}
      \correctchoice{they were about the same in terms of accuracy}
        \wrongchoice{it is impossible to compare them since we know nothing of Ptolomy's model.}
    \end{choices}
\end{question}
}

\element{morgans}{
\begin{question}{test1B-Q07}
    Which of the following is a statement of Kepler's first law?
    \begin{choices}
        \wrongchoice{planets move in perfect circles with the Sun at the center}
        \wrongchoice{planets move along an elliptical path with the Sun at the center}
      \correctchoice{planets move along an elliptical path with the Sun at one of the foci}
        \wrongchoice{planets move along an elliptical path with the Earth at one of the foci}
    \end{choices}
\end{question}
}

\element{morgans}{
\begin{question}{test1B-Q08}
    The expression ``Period Squared is proportional to orbital distance cubed'' is a statement of:
    \begin{choices}
        \wrongchoice{Newton's first law.}
        \wrongchoice{Newton's second law.}
        \wrongchoice{Brahe's third law.}
        \wrongchoice{Galileo's second law.}
      \correctchoice{Kepler's third law.}
    \end{choices}
\end{question}
}

\element{morgans}{
\begin{question}{test1B-Q09}
    While on the Moon,
        the Apollo astronauts demonstrated Galileo's experiment at the Leaning Tower of Pisa by dropping a feather and a hammer.
    They reached the ground at the same time because:
    \begin{choices}
        \wrongchoice{the force of gravity is larger on the feathers than on the hammer.}
        \wrongchoice{the force of gravity has no effect on either object.}
      \correctchoice{the acceleration of each object is the same.}
        \wrongchoice{the astronauts showed Galileo's experiment to be false.}
    \end{choices}
\end{question}
}

\element{morgans}{
\begin{question}{test1B-Q10}
    Suppose you are an astronaut taking a space walk to fix your spacecraft with a hammer. 
    Your life-line breaks and the jets on your backpack are out of fuel. 
    How would you return to your craft safely (without the help of someone else!)? 
    (Assume your altitude is the same as that of the spacecraft.)
    \begin{choices}
        \wrongchoice{point your magnetic boots towards the ship and allow the magnetic force to pull you back}
        \wrongchoice{throw the hammer in disgust at the space ship}
      \correctchoice{throw the hammer away from the space ship}
        \wrongchoice{fling your arms around in circles}
        \wrongchoice{kiss your ship goodbye!}
    \end{choices}
\end{question}
}

\element{morgans}{
\begin{question}{test1B-Q11}
    Which of the following types of electromagnetic radiation has a wavelength adjacent to but longer than visible light?
    \begin{choices}
        \wrongchoice{radio}
      \correctchoice{infrared}
        \wrongchoice{X ray}
        \wrongchoice{ultraviolet}
        \wrongchoice{gamma ray}
    \end{choices}
\end{question}
}

\element{morgans}{
\begin{question}{test1B-Q12}
    Which of the following statements is true about hot, dense glowing bodies? 
    \begin{choices}
        \wrongchoice{they emit only radiation corresponding to the visible part of the spectrum}
      \correctchoice{they emit radiation at all wavelengths}
        \wrongchoice{the hotter they are, the redder they are}
        \wrongchoice{their spectrum depends uniquely on their chemical composition}
    \end{choices}
\end{question}
}

\element{morgans}{
\begin{question}{test1B-Q13}
    When the atoms in a gas become excited:
    \begin{choices}
        \wrongchoice{the gas is being ionized.}
      \correctchoice{the electrons are moving to higher energy levels.}
        \wrongchoice{the electrons are moving to lower energy levels.}
        \wrongchoice{the protons are moving to higher energy levels.}
    \end{choices}
\end{question}
}

\element{morgans}{
\begin{question}{test1B-Q14}
    Which one of the following is the primary reason astronomers desire to have telescopes above the Earth's atmosphere? 
    \begin{choices}
        \wrongchoice{the magnification will be improved}
      \correctchoice{observations may be made in wavelengths that can not be seen from the surface of the Earth.}
        \wrongchoice{observations may be made in the radio region of the spectrum}
        \wrongchoice{the resolving power will be better in space because it no longer depends on the telescope diameter as is the case for telescopes on the Earth's surface}
    \end{choices}
\end{question}
}

\element{morgans}{
\begin{question}{test1B-Q15}
    What type of light did the Uhuru telescope observe?
    \begin{choices}
        \wrongchoice{Radio}
        \wrongchoice{Infrared}
        \wrongchoice{Ultraviolet}
      \correctchoice{X-ray}
        \wrongchoice{Gamma-ray}
    \end{choices}
\end{question}
}

%% Fill In
%% Place the most appropriate word or words in the blank. 
%% You may have to click on the blank to activate it before you start typing in your answer.
\begin{comment}
    Q: The east-west path of the planets along the celestial sphere is known as the.

    The celestial coordinate corresponding to latitude on the Earth is called .

    The time interval between successive full Moons is called .

    The darkest part of the shadow during an eclipse is called .

    The term which describes the tendency of an object to resist a change in its motion is .

    The recoil of a gun is an example of .

    All electromagnetic waves are identical except for differences in their .

    The speed of light in a vacuum is .

    The process in which electrons are lost from an atom is .

    A 5 inch diameter telescope will collect times more light than a 1 inch diamter telescope.
\end{comment}

\endinput

