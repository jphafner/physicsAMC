
%% University of Northern Iowa
%%  Morgans Astronomy Exams
%%--------------------------------------------------

%% this section contains 15 problems

\element{morgans}{
\begin{question}{test2A-Q01}
    The temperature of the center of the Sun is roughly:
    \begin{choices}
        \wrongchoice{10,000 K.}
        \wrongchoice{100,000 K.}
        \wrongchoice{1,000,000 K.}
      \correctchoice{10,000,000 K.}
        \wrongchoice{100,000,000 K.}
    \end{choices}
\end{question}
}

\element{morgans}{
\begin{question}{test2A-Q02}
    The amount of energy released by nuclear reactions in a star does NOT depend on:
    \begin{choices}
        \wrongchoice{temperature of the core.}
        \wrongchoice{pressure of the core.}
        \wrongchoice{the difference of mass of the incoming and outgoing particles.}
        \wrongchoice{the speed of light.}
      \correctchoice{the temperature of the corona.}
    \end{choices}
\end{question}
}

\element{morgans}{
\begin{question}{test2A-Q03}
    The temperature of the solar corona is approximately:
    \begin{choices}
        \wrongchoice{5,000 K.}
        \wrongchoice{10,000 K.}
        \wrongchoice{50,000 K.}
        \wrongchoice{100,000 K.}
      \correctchoice{1,000,000 K.}
    \end{choices}
\end{question}
}

\element{morgans}{
\begin{question}{test2A-Q04}
    The most important clue to the origin of sunspots is their:
    \begin{choices}
        \wrongchoice{appearance.}
        \wrongchoice{lower temperature than the surroundings.}
        \wrongchoice{position in the atmosphere.}
      \correctchoice{strong magnetic field.}
    \end{choices}
\end{question}
}

\element{morgans}{
\begin{question}{test2A-Q05}
    Star $A$ has a parallax of 0.5 seconds of arc,
        while Star $B$ has a parallax of 0.6 seconds of arc. 
    Which star is closer?
    \begin{choices}
        \wrongchoice{$A$}
      \correctchoice{$B$}
        \wrongchoice{Both the same distance}
        \wrongchoice{Not possible to determine distance from the information given}
    \end{choices}
\end{question}
}

\element{morgans}{
\begin{question}{test2A-Q06}
    If a stellar spectrum shows strong lines produced by ionized helium,
        the star is of spectral type:
    \begin{choices}
      \correctchoice{O.}
        \wrongchoice{B.}
        \wrongchoice{A.}
        \wrongchoice{F.}
        \wrongchoice{M.}
    \end{choices}
\end{question}
}

\element{morgans}{
\begin{question}{test2A-Q07}
    The methods of temperature determination we have discussed give us the temperature of the:
    \begin{choices}
      \correctchoice{photosphere.}
        \wrongchoice{region just below the photosphere.}
        \wrongchoice{point half-way to the center.}
        \wrongchoice{center.}
        \wrongchoice{corona.}
    \end{choices}
\end{question}
}

\element{morgans}{
\begin{question}{test2A-Q08}
    Whenever a star's absolute magnitude can be determined,
        its \rule[-0.1pt]{4em}{0.1pt} can also be determined if we have also observed the star's apparent magnitude.
    \begin{choices}
      \correctchoice{distance}
        \wrongchoice{mass}
        \wrongchoice{radius}
        \wrongchoice{temperature}
        \wrongchoice{spectral class}
    \end{choices}
\end{question}
}

\element{morgans}{
\begin{question}{test2A-Q09}
    The mass-luminosity relation for main sequence stars says:
    \begin{choices}
      \correctchoice{high mass, high luminosity.}
        \wrongchoice{high mass, low luminosity.}
        \wrongchoice{luminosity is constant for all masses.}
        \wrongchoice{luminosity is independent of mass.}
    \end{choices}
\end{question}
}

\element{morgans}{
\begin{question}{test2A-Q10}
    Fusion reactions can occur only under conditions of:
    \begin{choices}
        \wrongchoice{high temperature and low density.}
      \correctchoice{high temperature and high density.}
        \wrongchoice{low temperature and high density.}
        \wrongchoice{low temperature and low density.}
    \end{choices}
\end{question}
}

\element{morgans}{
\begin{question}{test2A-Q11}
    The temperature of a star's core will \rule[-0.1pt]{4em}{0.1pt} as the star fuses heavier elements.
    \begin{choices}
      \correctchoice{increase}
        \wrongchoice{decrease}
        \wrongchoice{remain the same}
    \end{choices}
\end{question}
}

\element{morgans}{
\begin{question}{test2A-Q12}
    Stars within a star cluster differ from one another primarily in:
    \begin{choices}
        \wrongchoice{distance.}
        \wrongchoice{age.}
        \wrongchoice{chemical composition.}
      \correctchoice{mass.}
    \end{choices}
\end{question}
}

\element{morgans}{
\begin{question}{test2A-Q13}
    A one solar mass star will probably form from a cloud having a mass of:
    \begin{choices}
        \wrongchoice{one solar mass.}
        \wrongchoice{less than one solar mass.}
        \wrongchoice{greater than one solar mass.}
    \end{choices}
\end{question}
}

\element{morgans}{
\begin{question}{test2A-Q14}
    What types of stars are found on the zero-age main sequence (ZAMS)?
    \begin{choices}
        \wrongchoice{red giants}
        \wrongchoice{protostar}
        \wrongchoice{newly formed stars}
        \wrongchoice{stars that have not yet started to burn hydrogen}
        \wrongchoice{white dwarfs}
    \end{choices}
\end{question}
}

\element{morgans}{
\begin{question}{test2A-Q15}
    Which of the following is \emph{not} a characteristic of a red giant?
    \begin{choices}
        \wrongchoice{extended outer layers}
        \wrongchoice{degenerate core}
      \correctchoice{cool core}
        \wrongchoice{cool surface temperature}
    \end{choices}
\end{question}
}

\element{morgans}{
\begin{question}{test2A-Q16}
    CNO reaction produces:
    \begin{choices}
        \wrongchoice{hydrogen.}
      \correctchoice{helium.}
        \wrongchoice{carbon.}
        \wrongchoice{oxygen.}
    \end{choices}
\end{question}
}

\element{morgans}{
\begin{question}{test2A-Q17}
    If a neutron star has more mass than its mass limit, it will:
    \begin{choices}
        \wrongchoice{expand catastrophically.}
      \correctchoice{contract catastrophically.}
        \wrongchoice{begin a new phase of nuclear reactions.}
        \wrongchoice{nothing will happen.}
    \end{choices}
\end{question}
}

\element{morgans}{
\begin{question}{test2A-Q18}
    White dwarfs cannot be more massive than \rule[-0.1pt]{4em}{0.1pt} solar masses. 
    \begin{choices}
        \wrongchoice{0.4}
        \wrongchoice{1.0}
      \correctchoice{1.4}
        \wrongchoice{2.4}
        \wrongchoice{5.4}
    \end{choices}
\end{question}
}

\element{morgans}{
\begin{question}{test2A-Q19}
    In what spectral range are most supernova remnants easily observed in?
    \begin{choices}
        \wrongchoice{X ray}
        \wrongchoice{gamma-ray}
        \wrongchoice{visible}
      \correctchoice{radio}
    \end{choices}
\end{question}
}

\element{morgans}{
\begin{question}{test2A-Q20}
    Why does a pulsar's rotation rate increase after mass is added to it?
    \begin{choices}
      \correctchoice{its angular momentum has increased}
        \wrongchoice{its angular momentum has decreased}
        \wrongchoice{its magnetic field strength is increased}
        \wrongchoice{its magnetic field strength is decreased}
    \end{choices}
\end{question}
}

\begin{comment}
    Fill In
    Place the most appropriate word or words in the blank. You may have to click on the blank to activate it before you start typing in your answer.
     
    The word meaning the total energy emitted by the Sun each second is .

    The visible surface layer of the Sun from which continuous radiation escapes and absorption lines form is called the .

    The period of the Sun oscillations is minutes.

    A star having low temperature and high luminosity is called a(n) .

    The specific helium burning reaction that produces carbon is called the reaction.

    The fundamental quantity which determines a star's properties and fate is .

    A cloud of glowing ionized gas usually taking the form of a hollow sphere or shell, ejected by a star in the late stages of evolution, is called a(n) .

    Stars more massive than the Sun do not have a helium flash because the core does not become .

    The mass limit for a white dwarf is known as the limit.

    A sporadic source of intense X rays, probably consisting of a neutron star onto which new matter falls at irregular intervals is called a(n)
\end{comment}

\endinput



