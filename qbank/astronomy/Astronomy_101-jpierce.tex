%% Table of Contents:
% A:  Introduction; Star Names and Brightness
% B:  The Celestial Sphere and Rotation
% C:  Seasons and Revolution
% D:  The Moon, Tides, and Eclipses
% E:  Light and Telescopes
% F:  Atoms, Photons, and Spectra
% G:  Parallax and Magnitudes
% H:  Binary Stars and Stellar Masses
% I:  Stellar Properties and the HR Diagram
% J:  The Interstellar Medium, Star Formation, and Stellar Structure
% K:  Stellar Evolution and Stellar Endpoints
% L:  The Milky Way
% M:  Galaxies
% N:  Active Galaxies and Quasars
% O:  Cosmology
% P:  The Solar System
% Q:  Perspectives

%% Answers:
%% 1 D
%% 2 B
%% 3 B
%% 4 C
%% 5 B
%% 6 B
%% 7 A
%% 8 D
%% 9 D
%% 10 C
%% 11 D
%% 12 B
%% 13 B
%% 14 B
%% 15 B
%% 16 D
%% 17 A
%% 18 C
%% 19 B
%% 20 D
%% 21 A
%% 22 D
%% 23 B
%% 24 E
%% 25 D
%% 26 A
%% 27 C
%% 28 B
%% 29 B
%% 30 D
%% 31 B
%% 32 E
%% 33 E
%% 34 C
%% 35 C
%% 36 B
%% 37 C
%% 38 D
%% 39 D
%% 40 A
%% 41 A
%% 42 D
%% 43 D
%% 44 B
%% 45 E
%% 46 C
%% 47 C
%% 48 C
%% 49 A
%% 50 B
%% 51 B
%% 52 C
%% 53 D
%% 54 E
%% 55 A
%% 56 D
%% 57 D
%% 58 A
%% 59 D
%% 60 B
%% 61 C
%% 62 B
%% 63 A
%% 64 B
%% 65 B
%% 66 E
%% 67 B
%% 68 D
%% 69 A
%% 70 A
%% 71 B
%% 72 C
%% 73 D
%% 74 A
%% 75 E
%% 76 C
%% 77 D
%% 78 D
%% 79 E
%% 80 B
%% 81 B
%% 82 D
%% 83 C
%% 84 E
%% 85 B
%% 86 D
%% 87 E
%% 88 D
%% 89 A
%% 90 B
%% 91 C
%% 92 A
%% 93 C
%% 94 B
%% 95 B
%% 96 A
%% 97 D
%% 98 D
%% 99 D
%% 100 E
%% 101 B
%% 102 C
%% 103 E
%% 104 E
%% 105 E
%% 106 E
%% 107 D
%% 108 C
%% 109 E
%% 110 C
%% 111 C
%% 112 B
%% 113 C
%% 114 A
%% 115 C
%% 116 A
%% 117 D
%% 118 D
%% 119 D
%% 120 D
%% 121 A
%% 122 B
%% 123 A
%% 124 C
%% 125 A
%% 126 E
%% 127 A
%% 128 B
%% 129 A
%% 130 A
%% 131 A
%% 132 B
%% 133 B
%% 134 B
%% 135 D
%% 136 A
%% 137 D
%% 138 B
%% 139 D
%% 140 B
%% 141 E
%% 142 C
%% 143 D
%% 144 C
%% 145 D
%% 146 B
%% 147 B
%% 148 B
%% 149 D
%% 150 D
%% 151 B
%% 152 A
%% 153 B
%% 154 C
%% 155 C
%% 156 D
%% 157 C
%% 158 E
%% 159 E
%% 160 C
%% 161 D
%% 162 B
%% 163 E
%% 164 D
%% 165 C
%% 166 A
%% 167 E
%% 168 C
%% 169 B
%% 170 E
%% 171 D
%% 172 A
%% 173 A
%% 174 D
%% 175 D
%% 176 B
%% 177 E
%% 178 B
%% 179 B
%% 180 E
%% 181 A
%% 182 B
%% 183 B
%% 184 B
%% 185 C
%% 186 A
%% 187 D
%% 188 W
%% 189 S
%% 190 X
%% 191 D
%% 192 C
%% 193 A
%% 194 E
%% 195 C
%% 196 E
%% 197 E
%% 198 B
%% 199 D
%% 200 B


%% A:  Introduction; Star Names and Brightness
\element{jpierce}{
\begin{question}{A-Q01}
    Hipparchus rated stars on a scale of 1 to 6, according to their:
    \begin{multicols}{2}
    \begin{choices}
        \wrongchoice{color}
        \wrongchoice{distance}
        \wrongchoice{position in the sky}
        \wrongchoice{brightness}
        \wrongchoice{location within a constellation}
    \end{choices}
    \end{multicols}
\end{question}
}

\element{jpierce}{
\begin{question}{A-Q02}
    A 1st magnitude star is \rule[-0.1pt]{4em}{0.1pt} times as bright as a 6th magnitude star.
    \begin{multicols}{2}
    \begin{choices}
        \wrongchoice{\num{2.5}}
        \wrongchoice{\num{100}}
        \wrongchoice{\num{1/6}}
        \wrongchoice{\num{5}}
        \wrongchoice{\num{6}}
    \end{choices}
    \end{multicols}
\end{question}
}

\element{jpierce}{
\begin{question}{A-Q03}
    A \rule[-0.1pt]{4em}{0.1pt} magnitude star is 100 times as bright as an 8th magnitude star.
    \begin{multicols}{2}
    \begin{choices}
        \wrongchoice{1\textsuperscript{st}}
        \wrongchoice{3\textsuperscript{rd}}
        \wrongchoice{7\textsuperscript{th}}
        \wrongchoice{13\textsuperscript{th}}
        \wrongchoice{108\textsuperscript{th}}
    \end{choices}
    \end{multicols}
\end{question}
}

\element{jpierce}{
\begin{question}{A-Q04}
    The Bayer designation ``alpha Orionis'' refers to:
    \begin{choices}
        \wrongchoice{the constellation, alpha Orionis.}
        \wrongchoice{a particular star, Orionis, in the constellation alpha.}
        \wrongchoice{a particular star, alpha, in the constellation Orion.}
        \wrongchoice{any star in the constellation alpha Orionis.}
        \wrongchoice{the planet, alpha Orionis.}
    \end{choices}
\end{question}
}

\element{jpierce}{
\begin{question}{A-Q05}
    The Bayer designation for Betelgeuse is \rule[-0.1pt]{4em}{0.1pt}.
    \begin{multicols}{2}
    \begin{choices}
        \wrongchoice{HD 39801}
        \wrongchoice{alpha Orionis}
        \wrongchoice{HR 2061}
        \wrongchoice{58 Orionis}
        %% TODO: Check this
        \wrongchoice{BD +7¡ 1055}
    \end{choices}
    \end{multicols}
\end{question}
}

\element{jpierce}{
\begin{question}{A-Q06}
    The diameter of the \rule[-0.1pt]{4em}{0.1pt} is about 100 times the diameter of the \rule[-0.1pt]{4em}{0.1pt},
        which in turn is about 4 times the diameter of the \rule[-0.1pt]{4em}{0.1pt}.
    \begin{multicols}{2}
    \begin{choices}
        \wrongchoice{Sun; Moon; Earth}
        \wrongchoice{Sun; Earth; Moon}
        \wrongchoice{Earth; Sun; Moon}
        \wrongchoice{Earth; Moon; Sun}
        \wrongchoice{Moon; Earth; Sun}
    \end{choices}
    \end{multicols}
\end{question}
}

%% B:  The Celestial Sphere and Rotation
\element{jpierce}{
\begin{question}{B-Q07}
    A star seen at the zenith at Mankato must be \rule[-0.1pt]{4em}{0.1pt} the Celestial Equator.
    \begin{multicols}{2}
    \begin{choices}
        \wrongchoice{north of}
        \wrongchoice{south of}
        \wrongchoice{east of}
        \wrongchoice{west of}
        \wrongchoice{exactly on}
    \end{choices}
    \end{multicols}
\end{question}
}

\element{jpierce}{
\begin{question}{B-Q08}
    What portion of the celestial sphere is visible to a Minnesota observer at any particular time?
    \begin{choices}
        \wrongchoice{only the northern celestial hemisphere}
        \wrongchoice{only the southern celestial hemisphere}
        \wrongchoice{half of the northern celestial hemisphere and half of the southern celestial hemisphere}
        \wrongchoice{most of the northern celestial hemisphere and less than half of the southern celestial hemisphere}
        \wrongchoice{most of the southern celestial hemisphere and less than half of the northern celestial hemisphere}
    \end{choices}
\end{question}
}

\element{jpierce}{
\begin{question}{B-Q09}
    The North Celestial Pole will be on the horizon for an observer who is:
    \begin{choices}
        \wrongchoice{at the North Pole.}
        \wrongchoice{at the South Pole.}
        \wrongchoice{at Mankato.}
        \wrongchoice{at the Equator.}
        \wrongchoice{any place in the southern hemisphere.}
    \end{choices}
\end{question}
}

\element{jpierce}{
\begin{question}{B-Q10}
    As seen from the earth, which of the following is the correct orientation for directions in the sky?
    %% NOTE: image
    %%<img src="Astronomy%20101%20Sample%20Questions_files/dir10.jpg">
    \begin{multciols}{3}
    \begin{choices}
        \wrongchoice{Q}
        \wrongchoice{K}
        \wrongchoice{G}
        \wrongchoice{R}
        \wrongchoice{M}
    \end{choices}
    \end{multciols}
\end{question}
}

\element{jpierce}{
\begin{question}{B-Q11}
    Diurnal motion is related to:
    %% NOTE: questionmult
    \begin{choices}
        \wrongchoice{the rising and setting of the sun.}
        \wrongchoice{star trails.}
        \wrongchoice{the sun's movement along the ecliptic.}
        \wrongchoice{A and B}
        \wrongchoice{A, B and C}
    \end{choices}
\end{question}
}

\element{jpierce}{
\begin{question}{B-Q12}
    If the stars are moving in counter-clockwise circles about your zenith,
        where on earth are you standing?
    \begin{choices}
        \wrongchoice{at the Equator}
        \wrongchoice{at the North Pole}
        \wrongchoice{at the South Pole}
        \wrongchoice{at Mankato}
        \wrongchoice{There is no place on Earth where this is possible.}
    \end{choices}
\end{question}
}

\element{jpierce}{
\begin{question}{B-Q13}
    The stars that we see looking west from Mankato are moving \rule[-0.1pt]{4em}{0.1pt} due to diurnal motion.
    \begin{choices}
        \wrongchoice{straight down}
        \wrongchoice{down and to the right}
        \wrongchoice{down and to the left}
        \wrongchoice{independently in many different directions}
        \wrongchoice{upwards}
    \end{choices}
\end{question}
}

\element{jpierce}{
\begin{question}{B-Q14}
    The rotation of the earth defines the location of:
    \begin{choices}
        \wrongchoice{the nadir.}
        \wrongchoice{the equator.}
        \wrongchoice{the zenith.}
        \wrongchoice{the horizon.}
        \wrongchoice{Polaris.}
    \end{choices}
\end{question}
}

\element{jpierce}{
\begin{question}{B-Q15}
    Star trails form as the stars move from \rule[-0.1pt]{4em}{0.1pt} in our sky.
    \begin{choices}
        \wrongchoice{west to east}
        \wrongchoice{east to west}
        \wrongchoice{north to south}
        \wrongchoice{southeast to northwest}
        \wrongchoice{northwest to southeast}
    \end{choices}
\end{question}
}

\element{jpierce}{
\begin{question}{B-Q16}
    %% <img src="Astronomy\%20101%20Sample\%20Questions_files/st16.jpg">
    Sketches of four star trails pictures taken at Mankato are shown above.
    How should they be labeled to identify the four directions the camera was pointed
        (in order from left to right)?
    \begin{choices}
        \wrongchoice{W-E-S-N}
        \wrongchoice{N-S-E-W}
        \wrongchoice{E-W-N-S}
        \wrongchoice{W-E-N-S}
        \wrongchoice{E-W-S-N}
    \end{choices}
\end{question}
}

\element{jpierce}{
\begin{question}{B-Q17}
    In Minnesota, a star that rises directly east will
        pass \rule[-0.1pt]{4em}{0.1pt} the Zenith and
        set \rule[-0.1pt]{4em}{0.1pt}.
    \begin{choices}
        \wrongchoice{south of; directly west}
        \wrongchoice{through; directly west}
        \wrongchoice{north of; directly west}
        \wrongchoice{north of; south of west}
        \wrongchoice{south of; north of west}
    \end{choices}
\end{question}
}


%% C:  Seasons and Revolution
\element{jpierce}{
\begin{question}{C-Q18}
    The earth reaches aphelion in its elliptical orbit during the month of:
    \begin{choices}
        \wrongchoice{January.}
        \wrongchoice{March.}
        \wrongchoice{July.}
        \wrongchoice{October.}
        \wrongchoice{December.}
    \end{choices}
\end{question}
}

\element{jpierce}{
\begin{question}{C-Q19}
    In the Northern Hemisphere,
        the nights are shorter than the daylight hours during:
    \begin{choices}
        \wrongchoice{summer.}
        \wrongchoice{summer and spring.}
        \wrongchoice{summer and autumn.}
        \wrongchoice{summer and winter.}
        \wrongchoice{the whole year.}
    \end{choices}
\end{question}
}

\element{jpierce}{
\begin{question}{C-Q20}
    The sun can be seen at the zenith as far north as the \rule[-0.1pt]{4em}{0.1pt} and as far south as the \rule[-0.1pt]{4em}{0.1pt}.
    \begin{choices}
        \wrongchoice{North Pole; South Pole}
        \wrongchoice{Arctic Circle; Antarctic Circle}
        \wrongchoice{Tropic of Capricorn; Tropic of Cancer}
        \wrongchoice{Tropic of Cancer; Tropic of Capricorn}
        \wrongchoice{Vernal Equinox; Autumnal Equinox}
    \end{choices}
\end{question}
}

\element{jpierce}{
\begin{question}{C-Q21}
    As seen from a point above the North Pole,
        the earth rotates \rule[-0.1pt]{4em}{0.1pt} and revolves \rule[-0.1pt]{4em}{0.1pt}.
    \begin{choices}
        \wrongchoice{CCW; CCW}
        \wrongchoice{CCW; CW}
        \wrongchoice{CW; CCW}
        \wrongchoice{CW; CW}
        \wrongchoice{to the left; to the right}
    \end{choices}
\end{question}
}

\element{jpierce}{
\begin{question}{C-Q22}
    As seen from Minnesota,
        the sun sets south of west 
    \begin{choices}
        \wrongchoice{on only one day each year.}
        \wrongchoice{on only two days each year.}
        \wrongchoice{for about one quarter of the year.}
        \wrongchoice{for about one half of the year.}
        \wrongchoice{for the whole year.}
    \end{choices}
\end{question}
}

\element{jpierce}{
\begin{question}{C-Q23}
    The seasons on earth are caused primarily by
    \begin{choices}
        \wrongchoice{the variation in the earth's distance from the sun.}
        \wrongchoice{the tilt of the earth's rotational axis with respect to the ecliptic.}
        \wrongchoice{the inclination of the moon's orbit with respect to the ecliptic.}
        \wrongchoice{the phases of the moon.}
        \wrongchoice{variations in the earth's rotational rate.}
    \end{choices}
\end{question}
}

\element{jpierce}{
\begin{question}{C-Q24}
    The Autumnal Equinox is a point:
    \begin{choices}
        \wrongchoice{on the Celestial Equator.}
        \wrongchoice{on the Ecliptic.}
        \wrongchoice{in time.}
        \wrongchoice{A and B}
        \wrongchoice{A, B and C}
    \end{choices}
\end{question}
}

\element{jpierce}{
\begin{question}{C-Q25}
    In the northern hemisphere,
        the Winter Solstice marks the:
    \begin{choices}
        \wrongchoice{beginning of summer.}
        \wrongchoice{end of winter.}
        \wrongchoice{beginning of spring.}
        \wrongchoice{end of autumn.}
        \wrongchoice{middle of winter.}
    \end{choices}
\end{question}
}

\element{jpierce}{
\begin{question}{C-Q26}
    At Mankato Minnesota,
        the noon sun is:
    \begin{choices}
        \wrongchoice{highest in the sky during the month of June.}
        \wrongchoice{highest in the sky during the month of December.}
        \wrongchoice{highest in the sky during the month of March.}
        \wrongchoice{highest in the sky during the month of August.}
        \wrongchoice{always at the zenith.}
    \end{choices}
\end{question}
}

\element{jpierce}{
\begin{question}{C-Q27}
    In Minnesota, on May 21 the noonday sun is about as high as the noonday sun on \rule[-0.1pt]{4em}{0.1pt}.
    \begin{choices}
        \wrongchoice{January 21}
        \wrongchoice{March 21}
        \wrongchoice{July 21}
        \wrongchoice{September 21}
        \wrongchoice{November 21}
    \end{choices}
\end{question}
}

\element{jpierce}{
\begin{question}{C-Q28}
    During the month of September the sun moves \rule[-0.1pt]{4em}{0.1pt} across the celestial sphere.
    \begin{choices}
        \wrongchoice{northward and eastward}
        \wrongchoice{southward and eastward}
        \wrongchoice{southward and westward}
        \wrongchoice{northward and westward}
        \wrongchoice{northward and southward}
    \end{choices}
\end{question}
}

\element{jpierce}{
\begin{question}{C-Q29}
    The tilt of the earth's rotational axis is about \rule[-0.1pt]{4em}{0.1pt} degrees with respect to the orbital axis.
    \begin{multicols}{3}
    \begin{choices}
        \wrongchoice{zero}
        \wrongchoice{23.5}
        \wrongchoice{72}
        \wrongchoice{90}
        \wrongchoice{44}
    \end{choices}
    \end{multicols}
\end{question}
}

\element{jpierce}{
\begin{question}{C-Q30}
    Precession causes changes in \rule[-0.1pt]{4em}{0.1pt} relative to the stars.
    \begin{choices}[o]
        \wrongchoice{the position of the NCP}
        \wrongchoice{the position of the Vernal Equinox}
        \wrongchoice{the position of Polaris}
        \wrongchoice{A and B}
        \wrongchoice{A and C}
    \end{choices}
\end{question}
}

\element{jpierce}{
\begin{question}{C-Q31}
    Precession is caused by:
    \begin{choices}
        \wrongchoice{the apparent north and south motions of the sun in the sky.}
        \wrongchoice{the gravitational pull of the sun and moon on the spinning earth.}
        \wrongchoice{the gravitational pull of the stars on the revolving earth.}
        \wrongchoice{the changing distance between the earth and sun.}
        \wrongchoice{star trails.}
    \end{choices}
\end{question}
}

\element{jpierce}{
\begin{question}{C-Q32}
    The Zodiac is a band of constellations around the sky that contains the paths of:
    \begin{choices}
        \wrongchoice{the Sun and planets, but not the Moon.}
        \wrongchoice{the Sun and Moon, but not the planets.}
        \wrongchoice{the planets, but not the Sun and Moon.}
        \wrongchoice{the Moon and planets, but not the Sun}
        \wrongchoice{the Sun, the Moon, and the planets.}
    \end{choices}
\end{question}
}


%% D:  The Moon, Tides, and Eclipses
\element{jpierce}{
\begin{question}{D-Q33}
    The best time to observe earthshine is generally around:
    \begin{choices}
        \wrongchoice{noon.}
        \wrongchoice{midnight.}
        \wrongchoice{mid-morning.}
        \wrongchoice{mid-afternoon.}
        \wrongchoice{twilight.}
    \end{choices}
\end{question}
}

\element{jpierce}{
\begin{question}{D-Q34}
    The sidereal month is measured with respect to the \rule[-0.1pt]{4em}{0.1pt},
        and the synodic month is measured with respect to the \rule[-0.1pt]{4em}{0.1pt}.
    \begin{choices}
        \wrongchoice{sun; moon}
        \wrongchoice{moon; sun}
        \wrongchoice{stars; sun}
        \wrongchoice{stars; moon}
        \wrongchoice{sun; stars}
    \end{choices}
\end{question}
}

\element{jpierce}{
\begin{question}{D-Q35}
    Suppose the Sun, Moon,
        and the star Aldebaran all rose at the same time today as seen from Mankato, Minnesota.
    Which will rise first tomorrow?
    \begin{choices}
        \wrongchoice{the Sun}
        \wrongchoice{the Moon}
        \wrongchoice{Aldebaran}
        \wrongchoice{All three will rise together.}
        \wrongchoice{This is nonsense---the Moon never rises at the same time as the Sun.}
    \end{choices}
\end{question}
}

\element{jpierce}{
\begin{question}{D-Q36}
    The orbit of the moon:
    \begin{choices}
        \wrongchoice{is inclined \ang{5} to the celestial equator.}
        \wrongchoice{is inclined \ang{5} to the ecliptic.}
        \wrongchoice{is inclined \ang{23.5} to the celestial equator.}
        \wrongchoice{is parallel to the ecliptic.}
        \wrongchoice{is parallel to the celestial equator.}
    \end{choices}
\end{question}
}

\element{jpierce}{
\begin{question}{D-Q37}
    If the moon was full a week ago,
        it should be a \rule[-0.1pt]{4em}{0.1pt} moon now.
    \begin{choices}
        \wrongchoice{new}
        \wrongchoice{first quarter}
        \wrongchoice{last quarter}
        \wrongchoice{waxing gibbous}
        \wrongchoice{waxing crescent}
    \end{choices}
\end{question}
}

\element{jpierce}{
\begin{question}{D-Q38}
    Which of these pairs does not indicate opposite directions on the Celestial Sphere?
    \begin{choices}
        \wrongchoice{the Zenith and the Nadir}
        \wrongchoice{the North Celestial Pole and the South Celestial Pole}
        \wrongchoice{the Summer Solstice and the Winter Solstice}
        \wrongchoice{the Sun and the New Moon}
        \wrongchoice{the Vernal Equinox and the Autumnal Equinox}
    \end{choices}
\end{question}
}

\element{jpierce}{
\begin{question}{D-Q39}
    The moon phase immediately following the first quarter phase is:
    \begin{choices}
        \wrongchoice{waxing crescent.}
        \wrongchoice{waning crescent.}
        \wrongchoice{waning gibbous.}
        \wrongchoice{waxing gibbous.}
        \wrongchoice{totally unpredictable.}
    \end{choices}
\end{question}
}

\element{jpierce}{
\begin{question}{D-Q40}
    If the full moon rises in the east,
        it will set in the \rule[-0.1pt]{4em}{0.1pt}, and if it rises in the southeast,
        it will set in the \rule[-0.1pt]{4em}{0.1pt}.
    \begin{choices}
        \wrongchoice{west; southwest}
        \wrongchoice{west; northwest}
        \wrongchoice{northwest; west}
        \wrongchoice{southwest; west}
        \wrongchoice{none of these---the full moon always rises directly east and sets directly west.}
    \end{choices}
\end{question}
}

\element{jpierce}{
\begin{question}{D-Q41}
    At what time of day does the first quarter moon rise in Minnesota?
    \begin{choices}
        \wrongchoice{noon}
        \wrongchoice{midnight}
        \wrongchoice{sunrise}
        \wrongchoice{sunset}
        \wrongchoice{The first quarter moon can rise at any time of the day or night.}
    \end{choices}
\end{question}
}

\element{jpierce}{
\begin{question}{D-Q42}
    If the moon is seen setting at noon in Minnesota,
        what is its phase?
    \begin{choices}
        \wrongchoice{new}
        \wrongchoice{full}
        \wrongchoice{first quarter}
        \wrongchoice{last quarter}
        \wrongchoice{It could be any phase.}
    \end{choices}
\end{question}
}

\element{jpierce}{
\begin{question}{D-Q43}
    Which of these phases of the moon could be viewed at both midnight and sunset in Minnesota?
    \begin{choices}
        \wrongchoice{waxing crescent}
        \wrongchoice{waning crescent}
        \wrongchoice{waning gibbous}
        \wrongchoice{waxing gibbous}
        \wrongchoice{Any phase can be viewed at any time.}
    \end{choices}
\end{question}
}

\element{jpierce}{
\begin{question}{D-Q44}
    The fact that the moon always keeps the same face toward the earth means that 
    \begin{choices}
        \wrongchoice{the moon does not rotate.}
        \wrongchoice{the moon rotates once in each revolution around the earth.}
        \wrongchoice{the moon never sets.}
        \wrongchoice{the moon rotates backwards with respect to the earth.}
        \wrongchoice{the moon rotates at the same rate that the earth revolves.}
    \end{choices}
\end{question}
}

\element{jpierce}{
\begin{question}{D-Q45}
    On December 22, the sun is at the \rule[-0.1pt]{4em}{0.1pt};
        if a full moon occurs on this date, the moon will appear close to the \rule[-0.1pt]{4em}{0.1pt}.
    \begin{choices}
        \wrongchoice{vernal equinox; autumnal equinox}
        \wrongchoice{vernal equinox; vernal equinox}
        \wrongchoice{autumnal equinox; vernal equinox}
        \wrongchoice{summer solstice; winter solstice}
        \wrongchoice{winter solstice; summer solstice}
    \end{choices}
\end{question}
}

\element{jpierce}{
\begin{question}{D-Q46}
    Neap tides occur near \rule[-0.1pt]{4em}{0.1pt} moon and \rule[-0.1pt]{4em}{0.1pt} moon.
    \begin{choices}
        \wrongchoice{new; first quarter}
        \wrongchoice{full; last quarter}
        \wrongchoice{first quarter; last quarter}
        \wrongchoice{new; full}
        \wrongchoice{new; third quarter}
    \end{choices}
\end{question}
}

\element{jpierce}{
\begin{question}{D-Q47}
    At an average coastal location, the lowest water level occurs at \rule[-0.1pt]{4em}{0.1pt} tide,
        and the highest water level occurs at \rule[-0.1pt]{4em}{0.1pt} tide.
    \begin{choices}
        \wrongchoice{spring; neap}
        \wrongchoice{neap; neap}
        \wrongchoice{spring; spring}
        \wrongchoice{neap; spring}
        \wrongchoice{spring; autumn}
    \end{choices}
\end{question}
}

\element{jpierce}{
\begin{question}{D-Q48}
    The nodes in the moon's orbit are:
    \begin{choices}
        \wrongchoice{places where the moon is closer than usual.}
        \wrongchoice{places where the moon moves more slowly than usual.}
        \wrongchoice{places where the moon's orbit intersects the ecliptic.}
        \wrongchoice{places where the moon's orbit intersects the celestial equator.}
        \wrongchoice{easily visible in a small telescope.}
    \end{choices}
\end{question}
}

\element{jpierce}{
\begin{question}{D-Q49}
    A total solar eclipse:
    \begin{choices}
        \wrongchoice{occurs every few years.}
        \wrongchoice{can occur only at certain latitudes.}
        \wrongchoice{can last up to two hours at one point on earth.}
        \wrongchoice{is always followed by a total lunar eclipse about four weeks later.}
        \wrongchoice{can only occur when the moon is near apogee.}
    \end{choices}
\end{question}
}

\element{jpierce}{
\begin{question}{D-Q50}
    Which of these produces a total lunar eclipse?
    \begin{choices}
        \wrongchoice{The moon passes thru earth's penumbra.}
        \wrongchoice{The moon passes thru earth's umbra.}
        \wrongchoice{The earth passes thru the moon's umbra.}
        \wrongchoice{The earth passes thru the moon's penumbra.}
        \wrongchoice{The moon becomes a new moon.}
    \end{choices}
\end{question}
}

\element{jpierce}{
\begin{question}{D-Q51}
    A given total lunar eclipse can be seen from:
    \begin{choices}
        \wrongchoice{the entire earth.}
        \wrongchoice{half of the earth.}
        \wrongchoice{anywhere within the moon's penumbra.}
        \wrongchoice{anywhere within the moon's umbra.}
        \wrongchoice{points along the eclipse track.}
    \end{choices}
\end{question}
}

\element{jpierce}{
\begin{question}{D-Q52}
    An eclipse will be annular if:
    \begin{choices}
        \wrongchoice{the moon is in earth's penumbra.}
        \wrongchoice{the full moon is too far from earth.}
        \wrongchoice{the new moon is too far from earth.}
        \wrongchoice{the earth's umbra does not reach the moon.}
        \wrongchoice{the earth's penumbra does not reach the moon.}
    \end{choices}
\end{question}
}

\element{jpierce}{
\begin{question}{D-Q53}
    During a solar eclipse, the \rule[-0.1pt]{4em}{0.1pt} is in the \rule[-0.1pt]{4em}{0.1pt} shadow.
    \begin{choices}
        \wrongchoice{sun; earth's}
        \wrongchoice{sun; moon's}
        \wrongchoice{moon; earth's}
        \wrongchoice{earth; moon's}
        \wrongchoice{earth; sun's}
    \end{choices}
\end{question}
}


%% E:  Light and Telescopes
\element{jpierce}{
\begin{question}{E-Q54}
    Our atmosphere is transparent to \rule[-0.1pt]{4em}{0.1pt} but blocks most rays from the rest of the electromagnetic spectrum.
    \begin{choices}
        \wrongchoice{visible light and X-rays}
        \wrongchoice{X-rays and infrared rays}
        \wrongchoice{infrared rays and ultraviolet rays}
        \wrongchoice{ultraviolet rays and radio waves}
        \wrongchoice{radio waves and visible light}
    \end{choices}
\end{question}
}

\element{jpierce}{
\begin{question}{E-Q55}
    Of the following forms of electromagnetic radiation,
        that which consists of photons of the highest energy is:
    \begin{choices}
        \wrongchoice{X-rays.}
        \wrongchoice{visible light.}
        \wrongchoice{infrared.}
        \wrongchoice{radio waves.}
        \wrongchoice{ultraviolet.}
    \end{choices}
\end{question}
}

\element{jpierce}{
\begin{question}{E-Q56}
    A long-wavelength electromagnetic wave has:
    \begin{choices}
        \wrongchoice{high frequency and low energy.}
        \wrongchoice{high frequency and high energy.}
        \wrongchoice{low frequency and high energy.}
        \wrongchoice{low frequency and low energy.}
        \wrongchoice{high energy and a blue color.}
    \end{choices}
\end{question}
}

\element{jpierce}{
\begin{question}{E-Q57}
    The \rule[-0.1pt]{4em}{0.1pt} in a telescope acts as a light funnel and the \rule[-0.1pt]{4em}{0.1pt} acts as a magnifying glass.
    \begin{choices}
        \wrongchoice{eyepiece; objective}
        \wrongchoice{diagonal; eyepiece}
        \wrongchoice{objective; diagonal}
        \wrongchoice{objective; eyepiece}
        \wrongchoice{eyepiece; diagonal}
    \end{choices}
\end{question}
}

\element{jpierce}{
\begin{question}{E-Q58}
    Telescopes make stars:
    \begin{choices}
        \wrongchoice{appear brighter.}
        \wrongchoice{appear bigger.}
        \wrongchoice{appear smaller.}
        \wrongchoice{appear bluer.}
        \wrongchoice{appear redder.}
    \end{choices}
\end{question}
}

\element{jpierce}{
\begin{question}{E-Q59}
    A 6-inch telescope collects more light than a 3-inch telescope.  
    How much more?
    \begin{choices}
        \wrongchoice{\SI{50}{\percent} more}
        \wrongchoice{twice as much}
        \wrongchoice{three times as much}
        \wrongchoice{four times as much}
        \wrongchoice{six times as much}
    \end{choices}
\end{question}
}

\element{jpierce}{
\begin{question}{E-Q60}
    The magnifying power of an astronomical telescope:
    \begin{choices}
        \wrongchoice{is its most fundamental property.}
        \wrongchoice{can be modified simply by changing the eyepiece.}
        \wrongchoice{can be modified simply by changing the objective.}
        \wrongchoice{cannot be modified without returning the telescope to the factory.}
        \wrongchoice{is proportional to the square of the aperture.}
    \end{choices}
\end{question}
}

\element{jpierce}{
\begin{question}{E-Q61}
    Chromatic aberration is a problem with \rule[-0.1pt]{4em}{0.1pt} telescopes.
    \begin{choices}
        \wrongchoice{Newtonian}
        \wrongchoice{Cassegrain}
        \wrongchoice{Refracting}
        \wrongchoice{Radio}
        \wrongchoice{Reflecting}
    \end{choices}
\end{question}
}

\element{jpierce}{
\begin{question}{E-Q62}
    The \rule[-0.1pt]{4em}{0.1pt} telescope has a thin corrector plate and a spherical mirror.
    \begin{choices}
        \wrongchoice{Newtonian}
        \wrongchoice{Schmidt-Cassegrain}
        \wrongchoice{Refracting}
        \wrongchoice{Radio}
        \wrongchoice{Cassegrain}
    \end{choices}
\end{question}
}

\element{jpierce}{
\begin{question}{E-Q63}
    This is a sketch of a \rule[-0.1pt]{4em}{0.1pt} telescope.
        <img src="Astronomy%20101%20Sample%20Questions_files/tele63.jpg">
    \begin{choices}
        \wrongchoice{Newtonian}
        \wrongchoice{Radio}
        \wrongchoice{Refracting}
        \wrongchoice{Cassegrain}
        \wrongchoice{Schmidt-Cassegrain}
    \end{choices}
\end{question}
}


%% F:  Atoms, Photons, and Spectra
\element{jpierce}{
\begin{question}{E-Q64}
    Different isotopes of the same element have:
    \begin{choices}
        \wrongchoice{the same atomic weight and different atomic numbers.}
        \wrongchoice{the same atomic number and different atomic weights.}
        \wrongchoice{the same number of protons and different numbers of electrons.}
        \wrongchoice{the same number of neutrons and different numbers of electrons.}
        \wrongchoice{the same number of neutrons and different numbers of protons.}
    \end{choices}
\end{question}
}

\element{jpierce}{
\begin{question}{E-Q65}
    The electronic structure of an atom is determined by the \rule[-0.1pt]{4em}{0.1pt} the atom contains.
    \begin{choices}
        \wrongchoice{number of neutrons}
        \wrongchoice{number of protons and the number of electrons}
        \wrongchoice{number of protons}
        \wrongchoice{number of electrons}
        \wrongchoice{number of protons and the number of neutrons}
    \end{choices}
\end{question}
}

\element{jpierce}{
\begin{question}{E-Q66}
    H I refers to \rule[-0.1pt]{4em}{0.1pt} while H II refers to \rule[-0.1pt]{4em}{0.1pt}.
    \begin{choices}
        \wrongchoice{atomic hydrogen; molecular hydrogen}
        \wrongchoice{atomic helium; molecular helium}
        \wrongchoice{ionized hydrogen; neutral hydrogen}
        \wrongchoice{molecular hydrogen; atomic hydrogen}
        \wrongchoice{neutral hydrogen; ionized hydrogen}
    \end{choices}
\end{question}
}

\element{jpierce}{
\begin{question}{E-Q67}
    In order for a hydrogen atom to undergo Balmer absorption,
        the electron in it must first be:
    \begin{choices}
        \wrongchoice{in the ground state.}
        \wrongchoice{excited to the second level.}
        \wrongchoice{excited to the third level.}
        \wrongchoice{excited to the fourth level.}
        \wrongchoice{ionized.}
    \end{choices}
\end{question}
}

\element{jpierce}{
\begin{question}{E-Q68}
    The intensity of radiation emitted by a blackbody is solely dependent on the \rule[-0.1pt]{4em}{0.1pt} of the blackbody.
    \begin{choices}
        \wrongchoice{composition}
        \wrongchoice{shape}
        \wrongchoice{size}
        \wrongchoice{temperature}
        \wrongchoice{texture}
    \end{choices}
\end{question}
}

\element{jpierce}{
\begin{question}{E-Q69}
    As the temperature of a radiating body is increased,
        the light emitted by the body:
    \begin{choices}
        \wrongchoice{becomes bluer and more intense.}
        \wrongchoice{becomes redder and more intense.}
        \wrongchoice{becomes redder and less intense.}
        \wrongchoice{becomes bluer and less intense.}
        \wrongchoice{shifts toward longer wavelengths.}
    \end{choices}
\end{question}
}

\element{jpierce}{
\begin{question}{E-Q70}
    A hot opaque body will radiate \rule[-0.1pt]{4em}{0.1pt} spectrum.
    \begin{choices}
        \wrongchoice{a continuous}
        \wrongchoice{an absorption}
        \wrongchoice{an emission}
        \wrongchoice{an ionization}
        \wrongchoice{a recombination}
    \end{choices}
\end{question}
}

\element{jpierce}{
\begin{question}{E-Q71}
    If light from a hot, opaque body passes through a cooler,
        thin gas, the observed spectrum will show:
    \begin{choices}
        \wrongchoice{narrow, bright lines on a dark background.}
        \wrongchoice{narrow, dark lines on a bright background.}
        \wrongchoice{light at all wavelengths.}
        \wrongchoice{emission lines.}
        \wrongchoice{total darkness, since the gas absorbs all of the light of the source.}
    \end{choices}
\end{question}
}

\element{jpierce}{
\begin{question}{E-Q72}
    Compared to a blue photon, a red photon has \rule[-0.1pt]{4em}{0.1pt} wavelength and \rule[-0.1pt]{4em}{0.1pt} speed.
    \begin{choices}
        \wrongchoice{longer; faster}
        \wrongchoice{longer; slower}
        \wrongchoice{longer; the same}
        \wrongchoice{shorter; the same}
        \wrongchoice{shorter; faster}
    \end{choices}
\end{question}
}

\element{jpierce}{
\begin{question}{E-Q73}
    Stellar absorption lines give direct information about the \rule[-0.1pt]{4em}{0.1pt} of the star.
    \begin{choices}[o]
        \wrongchoice{temperature}
        \wrongchoice{composition}
        \wrongchoice{mass}
        \wrongchoice{A and B}
        \wrongchoice{B and C}
    \end{choices}
\end{question}
}

\element{jpierce}{
\begin{question}{E-Q74}
    The stellar spectral classification scheme ranges from \rule[-0.1pt]{4em}{0.1pt} type M stars to \rule[-0.1pt]{4em}{0.1pt} type O stars.
    \begin{choices}
        \wrongchoice{cool, red; hot, blue}
        \wrongchoice{hot, blue; cool, red}
        \wrongchoice{cool, blue; hot, red}
        \wrongchoice{hot, red; cool, blue}
        \wrongchoice{cool, green; hot, yellow}
    \end{choices}
\end{question}
}

\element{jpierce}{
\begin{question}{E-Q75}
    The stellar spectral classification sequence,
        in order of increasing temperature, is \rule[-0.1pt]{4em}{0.1pt}.
    \begin{choices}
        \wrongchoice{OBAFGKM}
        \wrongchoice{FKOGMAB}
        \wrongchoice{OBFAKGM}
        \wrongchoice{ABFGKMO}
        \wrongchoice{MKGFABO}
    \end{choices}
\end{question}
}

\element{jpierce}{
\begin{question}{E-Q76}
    When an atom emits light,
        an electron in the atom \rule[-0.1pt]{4em}{0.1pt} energy and jumps to a \rule[-0.1pt]{4em}{0.1pt} orbit,
        while a photon is \rule[-0.1pt]{4em}{0.1pt}.
    \begin{choices}
        \wrongchoice{gains; higher; destroyed}
        \wrongchoice{gains; higher; created}
        \wrongchoice{loses; lower; created}
        \wrongchoice{loses; lower; destroyed}
        \wrongchoice{gains; lower; destroyed}
    \end{choices}
\end{question}
}

\element{jpierce}{
\begin{question}{E-Q77}
    When an atom emits light,
        an electron in the atom \rule[-0.1pt]{4em}{0.1pt} energy and jumps to a \rule[-0.1pt]{4em}{0.1pt} orbit,
        while a photon is \rule[-0.1pt]{4em}{0.1pt}.
    \begin{choices}
        \wrongchoice{gains; higher; destroyed}
        \wrongchoice{gains; higher; created}
        \wrongchoice{loses; lower; created}
        \wrongchoice{loses; lower; destroyed}
        \wrongchoice{gains; lower; destroyed}
    \end{choices}
\end{question}
}


%% G:  Parallax and Magnitudes
\element{jpierce}{
\begin{question}{E-Q77}
    If the absolute magnitude of Star $A$ is greater than the absolute magnitude of Star $B$, then:
    \begin{choices}
        \wrongchoice{$A$ appears brighter than $B$.}
        \wrongchoice{$B$ appears brighter than $A$.}
        \wrongchoice{$A$ is more luminous than $B$.}
        \wrongchoice{$B$ is more luminous than $A$.}
        \wrongchoice{$A$ is more distant than $B$.}
    \end{choices}
\end{question}
}

\element{jpierce}{
\begin{question}{E-Q78}
    The absolute magnitude of a star is the apparent magnitude the star would have if it were at a distance of:
    \begin{multicols}{2}
    \begin{choices}
        \wrongchoice{1 ly}
        \wrongchoice{1 pc}
        \wrongchoice{3.26 ly}
        \wrongchoice{10 pc}
        \wrongchoice{206265 AU}
    \end{choices}
    \end{multicols}
\end{question}
}

\element{jpierce}{
\begin{question}{E-Q79}
    The apparent magnitude of a star is:
    \begin{choices}
        \wrongchoice{independent of the star's distance.}
        \wrongchoice{a direct indication of the star's temperature.}
        \wrongchoice{a direct indication of the star's diameter.}
        \wrongchoice{used to assign the star to a constellation.}
        \wrongchoice{a measure of the amount of light we receive from it.}
    \end{choices}
\end{question}
}

\element{jpierce}{
\begin{question}{E-Q80}
    These Main Sequence stars all have the same parallax.  
    Which one would appear brightest to us?
    \begin{multicols}{3}
    \begin{choices}
        \wrongchoice{G2}
        \wrongchoice{B5}
        \wrongchoice{K3}
        \wrongchoice{A1}
        \wrongchoice{F8}
    \end{choices}
    \end{multicols}
\end{question}
}

\element{jpierce}{
\begin{question}{E-Q81}
    If a star has an absolute magnitude of 5 and a distance of 10 pc,
        then its apparent magnitude is:
    \begin{choices}
        \wrongchoice{less than 5.}
        \wrongchoice{5.}
        \wrongchoice{10.}
        \wrongchoice{15.}
        \wrongchoice{greater than 15.}
    \end{choices}
\end{question}
}

\element{jpierce}{
\begin{question}{E-Q82}
    Star $A$ and Star $B$ have the same apparent magnitude.
    If the absolute magnitude of Star $A$ is 5 and the absolute magnitude of Star $B$ is 3, then
    \begin{choices}
        \wrongchoice{Star $A$ appears brighter than Star $B$.}
        \wrongchoice{Star $A$ appears fainter than Star $B$.}
        \wrongchoice{Star $A$ is more luminous than Star $B$.}
        \wrongchoice{Star $A$ is closer than Star $B$.}
        \wrongchoice{Star $A$ is farther than Star $B$.}
    \end{choices}
\end{question}
}

\element{jpierce}{
\begin{question}{E-Q83}
    Star $A$ and Star $B$ have the same parallax.  
    If the apparent magnitude of Star $A$ is 2 and the apparent magnitude of Star $B$ is 3, then:
    \begin{choices}
        \wrongchoice{Star $A$ appears fainter than Star $B$.}
        \wrongchoice{Star $A$ is less luminous than Star $B$.}
        \wrongchoice{Star $A$ is more luminous than Star $B$.}
        \wrongchoice{Star $A$ is closer than Star $B$.}
        \wrongchoice{Star $A$ is farther than Star $B$.}
    \end{choices}
\end{question}
}

\element{jpierce}{
\begin{question}{E-Q84}
    If a star could be moved twice as far away,
        it would appear to be:
    \begin{choices}
        \wrongchoice{the same brightness.}
        \wrongchoice{twice as bright.}
        \wrongchoice{only half as bright.}
        \wrongchoice{four times as bright.}
        \wrongchoice{one-fourth as bright.}
    \end{choices}
\end{question}
}

\element{jpierce}{
\begin{question}{E-Q85}
    The parallax of a star is inversely related to its:
    \begin{choices}
        \wrongchoice{temperature.}
        \wrongchoice{distance.}
        \wrongchoice{luminosity.}
        \wrongchoice{diameter.}
        \wrongchoice{composition.}
    \end{choices}
\end{question}
}

\element{jpierce}{
\begin{question}{E-Q86}
    If a star has a parallax of \rule[-0.1pt]{4em}{0.1pt} arcseconds,
        then its distance is \rule[-0.1pt]{4em}{0.1pt}.
    \begin{choices}
        \wrongchoice{0.2; 20 parsecs}
        \wrongchoice{0.04; 25 lightyears}
        \wrongchoice{0.1; 10 lightyears}
        \wrongchoice{0.25; 4 parsecs}
        \wrongchoice{0.5; 0.5 parsecs}
    \end{choices}
\end{question}
}

\element{jpierce}{
\begin{question}{E-Q87}
    The parallaxes of nearby stars are produced by:
    \begin{choices}
        \wrongchoice{stellar motion across our line of sight.}
        \wrongchoice{stellar motion along our line of sight.}
        \wrongchoice{the rotation of the Galaxy.}
        \wrongchoice{Earth's rotation on its axis.}
        \wrongchoice{Earth's revolution around the sun.}
    \end{choices}
\end{question}
}


%% H:  Binary Stars and Stellar Masses
\element{jpierce}{
\begin{question}{E-Q88}
    \rule[-0.1pt]{4em}{0.1pt} binaries are studied using astrometry while \rule[-0.1pt]{4em}{0.1pt} binaries are studied using photometry.
    \begin{choices}
        \wrongchoice{Photometric; astrometric}
        \wrongchoice{Eclipsing; spectroscopic}
        \wrongchoice{Spectroscopic; visual}
        \wrongchoice{Visual; eclipsing}
        \wrongchoice{Apparent; absolute}
    \end{choices}
\end{question}
}

\element{jpierce}{
\begin{question}{E-Q89}
    If the H-beta line (lambda = \SI{4861.3}{\angstrom}) in a star's spectrum is observed at \SI{4860.8}{\angstrom},
        the line is \rule[-0.1pt]{4em}{0.1pt} and the star is \rule[-0.1pt]{4em}{0.1pt}. 
    \begin{choices}
        \wrongchoice{blueshifted; approaching}
        \wrongchoice{blueshifted; receding}
        \wrongchoice{redshifted; receding}
        \wrongchoice{redshifted; approaching}
        \wrongchoice{greenshifted; standing still}
    \end{choices}
\end{question}
}

\element{jpierce}{
\begin{question}{E-Q90}
    Periodic dips in the light curve of a star identify it as \rule[-0.1pt]{4em}{0.1pt} binary system.
    \begin{choices}
        \wrongchoice{a visual}
        \wrongchoice{an eclipsing}
        \wrongchoice{an astrometric}
        \wrongchoice{a spectroscopic}
        \wrongchoice{a massive}
    \end{choices}
\end{question}
}

\element{jpierce}{
\begin{question}{E-Q91}
    Binary star systems are the primary sources of information on stellar:
    \begin{choices}
        \wrongchoice{luminosities.}
        \wrongchoice{compositions.}
        \wrongchoice{masses.}
        \wrongchoice{ages.}
        \wrongchoice{structure.}
    \end{choices}
\end{question}
}

\element{jpierce}{
\begin{question}{E-Q92}
    Kepler's Third Law is applied to binary systems to determine stellar:
    \begin{choices}
        \wrongchoice{masses.}
        \wrongchoice{diameters.}
        \wrongchoice{temperatures.}
        \wrongchoice{luminosities.}
        \wrongchoice{ages.}
    \end{choices}
\end{question}
}

\element{jpierce}{
\begin{question}{E-Q93}
    A spectroscopic binary is a pair of stars in which the:
    \begin{choices}
        \wrongchoice{stars seem very close to one another, but are not really.}
        \wrongchoice{stars are very far apart.}
        \wrongchoice{spectral lines exhibit periodic Doppler shifts.}
        \wrongchoice{stars eclipse each other.}
        \wrongchoice{stars are of the same spectral type.}
    \end{choices}
\end{question}
}

\element{jpierce}{
\begin{question}{E-Q94}
    \rule[-0.1pt]{4em}{0.1pt} binary star is one whose members can be seen as separate stars through a telescope.
    \begin{choices}
        \wrongchoice{An eclipsing}
        \wrongchoice{A visual}
        \wrongchoice{A spectroscopic}
        \wrongchoice{A variable}
        \wrongchoice{A distinct}
    \end{choices}
\end{question}
}


%% I:  Stellar Properties and the HR Diagram
\element{jpierce}{
\begin{question}{E-Q95}
    Which of these is a Supergiant star?
    \begin{choices}
        \wrongchoice{M5 III}
        \wrongchoice{F3 I}
        \wrongchoice{K8 V}
        \wrongchoice{O9 II}
        \wrongchoice{G7 IV}
    \end{choices}
\end{question}
}

\element{jpierce}{
\begin{question}{E-Q96}
    A star is placed in a luminosity class by observing:
    \begin{choices}
        \wrongchoice{how sharp or fuzzy its spectral lines appear.}
        \wrongchoice{how far its spectral lines have been Doppler shifted.}
        \wrongchoice{how large the star appears.}
        \wrongchoice{how bright the star appears.}
        \wrongchoice{how red or blue the star appears.}
    \end{choices}
\end{question}
}

\element{jpierce}{
\begin{question}{E-Q97}
    Of these, which is the hottest star?
    \begin{choices}
        \wrongchoice{M5 III}
        \wrongchoice{F3 I}
        \wrongchoice{K8 V}
        \wrongchoice{O6 II}
        \wrongchoice{G7 IV}
    \end{choices}
\end{question}
}

\element{jpierce}{
\begin{question}{E-Q98}
    Most stars have roughly the same:
    \begin{choices}
        \wrongchoice{age.}
        \wrongchoice{radius.}
        \wrongchoice{luminosity.}
        \wrongchoice{composition.}
        \wrongchoice{distance.}
    \end{choices}
\end{question}
}

\element{jpierce}{
\begin{question}{E-Q99}
    The HR diagram is a plot of:
    \begin{choices}
        \wrongchoice{parallax vs. radius.}
        \wrongchoice{mass vs. color.}
        \wrongchoice{absolute magnitude vs. composition.}
        \wrongchoice{luminosity vs. temperature.}
        \wrongchoice{apparent magnitude vs. distance.}
    \end{choices}
\end{question}
}

\element{jpierce}{
\begin{question}{E-Q100}
    Where does a typical supergiant star appear on the HR diagram?
    \begin{choices}
        \wrongchoice{from the lower right to the upper left corner}
        \wrongchoice{from the lower left to the upper right corner}
        \wrongchoice{to the right of center}
        \wrongchoice{at the lower left}
        \wrongchoice{across the top}
    \end{choices}
\end{question}
}

\element{jpierce}{
\begin{question}{E-Q101}
    Which of these stellar properties does not increase along the Main Sequence from spectral type M to spectral type O?
    \begin{choices}
        \wrongchoice{surface temperature}
        \wrongchoice{absolute magnitude}
        \wrongchoice{mass}
        \wrongchoice{luminosity}
        \wrongchoice{radius}
    \end{choices}
\end{question}
}

\element{jpierce}{
\begin{question}{E-Q102}
    High mass Main Sequence stars have:
    \begin{choices}
        \wrongchoice{high luminosities and low temperatures.}
        \wrongchoice{low luminosities and low temperatures.}
        \wrongchoice{high luminosities and high temperatures.}
        \wrongchoice{low luminosities and high temperatures.}
        \wrongchoice{small radii and a reddish color.}
    \end{choices}
\end{question}
}

\element{jpierce}{
\begin{question}{E-Q103}
    The mass of a giant star can best be determined from its:
    \begin{choices}
        \wrongchoice{temperature.}
        \wrongchoice{luminosity.}
        \wrongchoice{position on the main sequence.}
        \wrongchoice{radius.}
        \wrongchoice{orbital motion.}
    \end{choices}
\end{question}
}

\element{jpierce}{
\begin{question}{E-Q104}
    On the HR diagram,
        the stars with largest radii are found in the:
    \begin{choices}
        \wrongchoice{center.}
        \wrongchoice{upper left corner.}
        \wrongchoice{lower left corner.}
        \wrongchoice{lower right corner.}
        \wrongchoice{upper right corner.}
    \end{choices}
\end{question}
}

\element{jpierce}{
\begin{question}{E-Q105}
    The spectrum of a star yields information that allows us to determine the star's:
    \begin{choices}
        \wrongchoice{age, temperature, luminosity, and composition.}
        \wrongchoice{velocity, age, temperature, and luminosity.}
        \wrongchoice{composition, velocity, age, and temperature.}
        \wrongchoice{luminosity, composition, velocity, and age.}
        \wrongchoice{temperature, luminosity, composition, and velocity.}
    \end{choices}
\end{question}
}

\element{jpierce}{
\begin{question}{E-Q106}
    If a star's luminosity class and spectral type are known,
        then \rule[-0.1pt]{4em}{0.1pt} can be found.
    \begin{choices}
        \wrongchoice{its luminosity}
        \wrongchoice{its temperature}
        \wrongchoice{its radius}
        \wrongchoice{its absolute magnitude}
        \wrongchoice{all of the above}
    \end{choices}
\end{question}
}

\element{jpierce}{
\begin{question}{E-Q107}
    Stars with low luminosity are either very \rule[-0.1pt]{4em}{0.1pt} or very \rule[-0.1pt]{4em}{0.1pt}.
    \begin{choices}
        \wrongchoice{hot; small}
        \wrongchoice{hot; large}
        \wrongchoice{cool; large}
        \wrongchoice{cool; small}
        \wrongchoice{blue; distant}
    \end{choices}
\end{question}
}

\element{jpierce}{
\begin{question}{E-Q108}
    If two stars have the same temperature but different radii,
        then the smaller star will be:
    \begin{choices}
        \wrongchoice{bluer.}
        \wrongchoice{redder.}
        \wrongchoice{less luminous.}
        \wrongchoice{more luminous.}
        \wrongchoice{farther away.}
    \end{choices}
\end{question}
}


%% J:  The Interstellar Medium, Star Formation, and Stellar Structure
\element{jpierce}{
\begin{question}{E-Q109}
    Emission nebulae appear \rule[-0.1pt]{4em}{0.1pt},
        reflection nebulae appear \rule[-0.1pt]{4em}{0.1pt},
        and dark nebulae appear \rule[-0.1pt]{4em}{0.1pt}.
    \begin{choices}
        \wrongchoice{blue; red; black}
        \wrongchoice{yellow; green; blue}
        \wrongchoice{green; yellow; black}
        \wrongchoice{orange; purple; brown}
        \wrongchoice{red; blue; black}
    \end{choices}
\end{question}
}

\element{jpierce}{
\begin{question}{E-Q110}
    A star that appears fainter,
        due to extinction of some of its light by interstellar dust,
        also appears to be:
    \begin{choices}
        \wrongchoice{bluer.}
        \wrongchoice{hotter.}
        \wrongchoice{redder.}
        \wrongchoice{receding.}
        \wrongchoice{approaching.}
    \end{choices}
\end{question}
}

111.  A young, hot star emits visible light that is scattered by dust
 grains in a nearby cloud; this scattered light is viewed by observers 
on Earth as
  A.  an emission nebula.
  B.  a dark nebula.
  C.  a reflection nebula.
  D.  a spiral nebula.
  E.  a dust nebula.



112.  \rule[-0.1pt]{4em}{0.1pt} are evidence of interstellar gas clouds.
  A.  Reflection nebulae and interstellar reddening
  B.  Emission nebulae and interstellar absorption
  C.  Dark nebulae and emission nebulae
  D.  Interstellar absorption and interstellar reddening
  E.  Interstellar reddening and emission nebulae



113.  Where the interstellar medium is observed as a reddish glow, as in the Lagoon Nebula, we call it
  A.  interstellar absorption.
  B.  interstellar reddening.
  C.  an emission nebula.
  D.  a reflection nebula.
  E.  a dark nebula.



114.  The principal energy source of a protostar is 
  A.  gravitational contraction.
  B.  the P-P chain.
  C.  the CNO cycle.
  D.  combustion of hydrogen.
  E.  electrostatic repulsion.



115.  Protostars form from
  A.  planets.
  B.  black holes.
  C.  gas and dust clouds.
  D.  main sequence stars.
  E.  white dwarfs.



116.  Compared to a normal star, a protostar has
  A.  a lower surface temperature and a larger radius.
  B.  a lower surface temperature and a smaller radius.
  C.  a higher surface temperature and a smaller radius.
  D.  a higher surface temperature and a larger radius.
  E.  more planets.



117.  The raw materials of star formation
  A.  are interstellar gas and dust.
  B.  are mostly hydrogen and helium.
  C.  cannot be detected.
  D.  A and B
  E.  B and C



118.  When newly formed stars first become visible in the sky,
    they occupy the \rule[-0.1pt]{4em}{0.1pt} region of the HR diagram.
  A.  white dwarf
  B.  giant
  C.  supergiant
  D.  main sequence
  E.  binary



119.  Stars spend most of their lifetimes as 
  A.  giant stars.
  B.  protostars.
  C.  binary stars.
  D.  main sequence stars.
  E.  supergiant stars.



120.  The stars that spend the longest time on the main sequence have spectral type \rule[-0.1pt]{4em}{0.1pt}.
  A.  A
  B.  B
  C.  O
  D.  M
  E.  G



121.  Low mass main sequence stars obtain energy from the \rule[-0.1pt]{4em}{0.1pt} while high mass main sequence stars utilize the \rule[-0.1pt]{4em}{0.1pt}.
  A.  Proton-Proton chain; CNO cycle
  B.  Proton-Proton chain; Triple Alpha process
  C.  CNO cycle; Triple Alpha process
  D.  CNO cycle; Proton-Proton chain
  E.  Triple Alpha process; Proton-Proton chain



122.  A star on the ZAMS is 
  A.  beginning the protostar phase.
  B.  beginning the main sequence phase.
  C.  ending the main sequence phase.
  D.  beginning the giant phase.
  E.  ending the giant phase.



123.  A one-solar-mass main sequence star 
  A.  should evolve at the same rate as the Sun.
  B.  should evolve more rapidly than the Sun.
  C.  should evolve more slowly than the Sun.
  D.  should have more hydrogen than the Sun.
  E.  should have more planets than the Sun.



124.  Repulsion of like charges requires that stellar nuclear reactions proceed in regions of 
  A.  high luminosity.
  B.  high mass.
  C.  high temperature.
  D.  low pressure.
  E.  low density.



125.  By the CNO cycle, stars convert 
  A.  hydrogen to helium.
  B.  hydrogen to carbon.
  C.  hydrogen to nitrogen.
  D.  hydrogen to oxygen.
  E.  hydrogen to cyanogen.



126.  Our knowledge of the interior structure of stars comes mostly from
  A.  observations of the sun.
  B.  observations of exploding stars.
  C.  observations of binary stars.
  D.  observations of newly formed stars.
  E.  computer models of stars.



127.  Compared to the stellar surface, the center of a star has
  A.  a higher temperature and a higher pressure.
  B.  a higher temperature and a lower pressure.
  C.  a lower temperature and a higher pressure.
  D.  a lower temperature and a lower pressure.
  E.  the same temperature and pressure.



128.  In a region of a star where the opacity is relatively high, energy transport will be accomplished primarily through
  A.  conduction.
  B.  convection.
  C.  radiation.
  D.  recombination.
  E.  excitation.


%% K:  Stellar Evolution and Stellar Endpoints


129.  A highly degenerate electron gas in a stellar core
  A.  exerts a pressure that is largely independent of its temperature.
  B.  obeys the ideal gas law.
  C.  exerts a pressure that is proportional to its temperature.
  D.  cannot possibly occur.
  E.  can occur if the core density becomes low enough.



130.  The helium flash occurs because
  A.  degeneracy disconnects the stellar thermostat before helium ignition occurs.
  B.  helium vaporizes so easily in stars.
  C.  helium is rapidly converted into hydrogen.
  D.  there is very little helium in stars.
  E.  helium stars tend to explode when they end their lives.



131.  Compared to the sun, a 0.5 solar mass main sequence star should
  A.  evolve more slowly.
  B.  emit more red photons.
  C.  have a higher surface temperature.
  D.  have a higher luminosity.
  E.  emit more blue photons.



132.  The Triple Alpha process occurs inside
  A.  main sequence stars.
  B.  giant stars.
  C.  white dwarfs.
  D.  protostars.
  E.  neutron stars.



133.  For stars similar to the Sun, the stellar evolution sequence proceeds from \rule[-0.1pt]{4em}{0.1pt} to \rule[-0.1pt]{4em}{0.1pt} to \rule[-0.1pt]{4em}{0.1pt}.
  A.  gas and dust cloud; giant star; main sequence star
  B.  gas and dust cloud; main sequence star; giant star
  C.  giant star; main sequence star; gas and dust cloud
  D.  giant star; gas and dust cloud; main sequence star
  E.  main sequence star; gas and dust cloud; giant star



134.  As a star evolves off the main sequence,
    its core \rule[-0.1pt]{4em}{0.1pt} and its envelope \rule[-0.1pt]{4em}{0.1pt}.
  A.  contracts; contracts
  B.  contracts; expands
  C.  expands; expands
  D.  expands; contracts
  E.  contracts; puffs away into space



135.  Future evolution of the sun will most likely produce a
  A.  supernova.
  B.  neutron star.
  C.  black hole.
  D.  white dwarf.
  E.  protostar.



136.  Low mass main sequence stars (less than 0.4 solar mass)
  A.  are completely convective.
  B.  are completely radiative.
  C.  are completely degenerate.
  D.  are burning helium.
  E.  do not radiate energy.



137.  The next step for a star that develops an iron core is
  A.  carbon detonation.
  B.  the helium flash.
  C.  expansion to the giant phase.
  D.  a supernova.
  E.  the cocoon nebula phase.



138.  A Type II supernova is produced
  A.  by the collapse of gas and dust clouds to form a new star.
  B.  by the explosion of a massive star.
  C.  during the formation of a planetary nebula.
  D.  by the collision of two stars.
  E.  by the collision of two planets.



139.  Why is a black hole black?
  A.  It emits black photons.
  B.  It is hidden by dust clouds.
  C.  It radiates only infrared rays.
  D.  Its escape velocity is greater than the speed of light.
  E.  It has no mass and therefore cannot radiate.



140.  A stellar core that has collapsed inside its Schwarzschild Radius will be
  A.  an accretion disc.
  B.  a black hole.
  C.  a white dwarf.
  D.  a neutron star.
  E.  a planet.



141.  The Crab Nebula provides a link between
  A.  planetary nebulae and white dwarfs.
  B.  black holes, neutron stars, and white dwarfs.
  C.  main sequence stars and giant stars.
  D.  supergiants and supernovae.
  E.  supernovae, neutron stars, and pulsars.



142.  Planetary nebulae are so named because
  A.  they are the nebulae out of which planets form.
  B.  they are created from the remains of planets that have exploded.
  C.  some of them resemble planets in appearance.
  D.  they are contained within our solar system.
  E.  some of them are believed to orbit the sun.



143.  In the interior of a neutron star, what prevents gravitational collapse from occurring?
  A.  nuclear reactions
  B.  ideal gas pressure
  C.  degenerate electron pressure
  D.  degenerate neutron pressure
  E.  nothing



144.  Pulsars are 
  A.  pulsating black holes.
  B.  vibrating white dwarfs.
  C.  spinning neutron stars.
  D.  rotating supergiants.
  E.  found inside planetary nebulae.



145.  The Chandrasekhar Limit applies to
  A.  the spin rate of a neutron star.
  B.  the radius of a black hole.
  C.  the luminosity of a supernova.
  D.  the mass of a white dwarf.
  E.  the frequency of a pulsar.



146.  A typical white dwarf has a mass equivalent to that of \rule[-0.1pt]{4em}{0.1pt} in a volume about equal to that of \rule[-0.1pt]{4em}{0.1pt}.
  A.  the Earth; the Sun
  B.  the Sun; the Earth
  C.  the Sun; the Earth's orbit
  D.  the Earth: the Moon
  E.  the Moon; the Earth


%% L:  The Milky Way


147.  The youngest stars in the Milky Way, found in the \rule[-0.1pt]{4em}{0.1pt}, have the \rule[-0.1pt]{4em}{0.1pt} metal contents and the \rule[-0.1pt]{4em}{0.1pt} orbits.
  A.  disk; lowest; most circular
  B.  spiral arms; highest; most circular
  C.  nucleus; lowest; most circular
  D.  globular clusters; highest; most elliptical
  E.  halo; lowest; most elliptical



148.  An estimate of the age of a star cluster can be obtained from
  A.  measurements of the cluster's parallax.
  B.  an HR diagram for the cluster.
  C.  a count of white dwarfs in the cluster.
  D.  a determination of the cluster's redshift.
  E.  timings of pulsars in the cluster.



149.  Compared to open clusters, the globular clusters have
  A.  fewer stars.
  B.  bluer stars.
  C.  a greater percentage of high mass stars.
  D.  stars with lower metal contents.
  E.  younger stars.



150.  Which of these groups best characterizes the spherical component of the Milky Way?
  A.  open clusters, the nuclear bulge, O and B stars
  B.  globular clusters, spiral arms, gas and dust
  C.  stellar associations, Pop I stars, star formation
  D.  halo stars, globular clusters, elliptical orbits
  E.  circular orbits, hot stars, Pop II stars



151.  The different chemical compositions of the various stellar populations are the result of
  A.  ionization.
  B.  nucleosynthesis.
  C.  interstellar absorption.
  D.  stellar collisions.
  E.  selective breeding.



152.  In our Galaxy, as time goes on, the abundance of heavy elements (relative to hydrogen) 
  A.  increases.
  B.  decreases.
  C.  remains the same.
  D.  fluctuates regularly, up and down.
  E.  fluctuates irregularly.



153.  Stars that have the most elliptical orbits about the galactic 
center and that are found in the halo are classed as \rule[-0.1pt]{4em}{0.1pt} stars.
  A.  Extreme Population I
  B.  Extreme Population II
  C.  E9
  D.  E0
  E.  H II



154.  Population I and Population II refer to
  A.  neutral and singly ionized elements.
  B.  different types of extraterrestrial aliens.
  C.  different metal abundances.
  D.  different groups of pulsars.
  E.  different spiral arms in the Galaxy.



155.  Our sun 
  A.  orbits the center of the Galaxy once every 250 years.
  B.  orbits the center of the Galaxy once every 250 thousand years.
  C.  orbits the center of the Galaxy once every 250 million years.
  D.  orbits the center of the Galaxy once every 250 billion years.
  E.  lies at the center of the Galaxy and therefore does not move.



156.  We cannot see visible light from the nucleus of our Galaxy because
  A.  our Galaxy has no nucleus.
  B.  the stars in the nucleus are too faint to see at this distance.
  C.  the nucleus never appears above our horizon as seen from earth.
  D.  interstellar dust absorbs the light from stars in the nucleus.
  E.  the nucleus is really a huge black hole, which does not radiate any visible light.



157.  In our Galaxy, the sun is located
  A.  in the nucleus.
  B.  in the halo.
  C.  in the disk.
  D.  at the center.
  E.  in a globular cluster.




158.  The Sun is apparently
  A.  a member of a stellar association.
  B.  a member of an open cluster.
  C.  a member of a globular cluster.
  D.  a part of the constellation Orion.
  E.  not a member of any identifiable stellar group (except the Milky Way).



%% M:  Galaxies


159.  The \rule[-0.1pt]{4em}{0.1pt} of a  Cepheid is related to its \rule[-0.1pt]{4em}{0.1pt}.
  A.  velocity; distance
  B.  temperature; apparent magnitude
  C.  distance; absolute magnitude
  D.  mass; velocity
  E.  luminosity; period




160.  Compared to a galaxy classed as Sa, a galaxy classed as Sb would
  A.  have a more prominent bar feature.
  B.  have more tightly wound spiral arms.
  C.  have a smaller nuclear bulge.
  D.  have a lower abundance of hot stars.
  E.  appear more elliptical.




161.  The Andromeda Galaxy is classed as 
  A.  E2.
  B.  E7.
  C.  SBa.
  D.  Sb.
  E.  Sc.



162.  Neither \rule[-0.1pt]{4em}{0.1pt} nor \rule[-0.1pt]{4em}{0.1pt} show evidence of recent star formation.
  A.  elliptical galaxies; open clusters
  B.  elliptical galaxies; globular clusters
  C.  spiral galaxies; globular clusters
  D.  spiral galaxies; open clusters
  E.  irregular galaxies; open clusters



163.  Which of these is not used as a distance indicator for other galaxies?
  A.  Cepheids
  B.  supernovae
  C.  supergiants
  D.  globular clusters
  E.  pulsars



164.  The \rule[-0.1pt]{4em}{0.1pt} provides a means of estimating the distances to galaxies.
  A.  Hubble Law
  B.  Period-Luminosity Law
  C.  Galactic Parallax Law
  D.  A and B
  E.  A and C



165.  The Local Group refers to the 
  A.  nearby planets.
  B.  nearby stars.
  C.  nearby galaxies.
  D.  Solar System.
  E.  Milky Way.



166.  The look-back time for a galaxy is
  A.  how long the light from the galaxy takes to reach Earth.
  B.  numerically equal to the galaxy's distance in parsecs.
  C.  smaller for more distant galaxies.
  D.  equal to the age of the galaxy.
  E.  impossible to determine unless the galaxy is in a cluster.



167.  The Magellanic Clouds are 
  A.  open clusters.
  B.  clusters of galaxies.
  C.  spiral arms in the Milky Way.
  D.  dust clouds in the Milky Way.
  E.  satellites of the Milky Way.



168.  Our galaxy has a diameter of about \rule[-0.1pt]{4em}{0.1pt} lightyears and is about \rule[-0.1pt]{4em}{0.1pt} lightyears from the Andromeda Galaxy.
  A.            800;          30,000
  B.         8,000;        300,000
  C.       80,000;     3,000,000
  D.     800,000;   30,000,000
  E.  8,000,000; 300,000,000


%% N:  Active Galaxies and Quasars


169.  Radio galaxies
  A.  emit much less synchrotron radiation than normal galaxies do.
  B.  emit much more radio energy than normal galaxies do.
  C.  contain abnormally large numbers of radios.
  D.  have been predicted, but never observed.
  E.  are only found along the Milky Way.



170.  Seyfert galaxies are \rule[-0.1pt]{4em}{0.1pt} with very bright cores.
  A.  giant elliptical galaxies
  B.  dwarf elliptical galaxies
  C.  supernova remnants
  D.  globular clusters
  E.  spiral galaxies



171.  If quasar redshifts are due to Doppler shifts, then the quasars cannot be very
  A.  luminous.
  B.  far away.
  C.  massive.
  D.  nearby.
  E.  large.



172.  Because some quasars have been observed to vary in brightness 
over periods as short as a few days, we assume these quasars must be 
very
  A.  small.
  B.  distant.
  C.  bright.
  D.  hot.
  E.  massive.



173.  High energy electrons spiraling in a magnetic field produce \rule[-0.1pt]{4em}{0.1pt} in radio galaxies.
  A.  synchrotron radiation
  B.  supernovae
  C.  thermal radiation
  D.  Doppler shifts
  E.  emission nebulae


%% O:  Cosmology


174.  The 3� Cosmic Background Radiation is
  A.  energy released by the rapid formation of the first galaxies.
  B.  energy released by the rapid formation of the first stars.
  C.  energy emitted by the primeval fireball at the instant of the Big Bang.
  D.  energy emitted by the primeval fireball after it had cooled to a few thousand degrees Kelvin.
  E.  the light emitted by the interstellar medium.



175.  Which of these cosmologies was supported by the discovery of the 3� Cosmic Background Radiation?
  A.  Big Bang Open universe
  B.  Big Bang Closed universe
  C.  Steady State theory
  D.  A and B
  E.  A and C



176.  A universe that appears the same to us no matter which direction in space we look is said to be
  A.  homogeneous.
  B.  isotropic.
  C.  universal.
  D.  cosmological.
  E.  heterogeneous.



177.  Where was the Earth when the Big Bang occurred?
  A.  Far, far away in another part of the universe.
  B.  Close by, but protected by surrounding dust clouds.
  C.  In a different galaxy.
  D.  In a different universe.
  E.  It did not yet exist.



178.  The Primeval Fireball was a state of \rule[-0.1pt]{4em}{0.1pt} and \rule[-0.1pt]{4em}{0.1pt}.
  A.  high density; low temperature
  B.  high density; high temperature
  C.  low density; high temperature
  D.  low density; low temperature
  E.  high inflation; high unemployment



179.  Following the nuclear fusion phase of the Big Bang, the universe was composed almost entirely of
  A.  helium.
  B.  hydrogen and helium.
  C.  carbon, nitrogen, and oxygen.
  D.  hydrogen and oxygen.
  E.  iron.



180.  The cosmological principle states that
  A.  we live at no special place in the universe.
  B.  we live at no special time in the history of the universe.
  C.  the universe appears essentially the same to observers at different locations.
  D.  B and C
  E.  A and C



181.  The cosmological principle results from the assumption that
  A.  the universe is homogeneous and isotropic.
  B.  the same physical laws apply everywhere in the universe.
  C.  the galaxies are receding from us.
  D.  the galaxies are approaching us.
  E.  all galaxies exhibit redshifts.



182.  According to current estimates of the average density of the visible universe,
    our universe had \rule[-0.1pt]{4em}{0.1pt} and will have \rule[-0.1pt]{4em}{0.1pt}.
  A.  a beginning; an end
  B.  a beginning; no end
  C.  no beginning; no end
  D.  no beginning; an end
  E.  an oscillatory past; a similar future



183.  The general expansion of the universe
  A.  explains the Period-Luminosity Law.
  B.  explains Hubble's Law.
  C.  explains the Inverse Square Law.
  D.  implies our Galaxy is expanding.
  E.  implies our Galaxy is contracting.



184.  Most astronomers would now agree that in the past, the galaxies were
  A.  much more massive.
  B.  closer together.
  C.  farther apart.
  D.  more numerous.
  E.  much smaller.



185.  The Oscillating Universe is a variation of the
  A.  Steady State Universe.
  B.  Big Bang Open Universe.
  C.  Big Bang Closed Universe.
  D.  Flat Universe.
  E.  Bumpy Universe.



186.  Most astronomers today reject the \rule[-0.1pt]{4em}{0.1pt} theory because it \rule[-0.1pt]{4em}{0.1pt}.
  A.  Steady State: did not predict the 3� Cosmic Background Radiation
  B.  Big Bang: did not predict the 3� Cosmic Background Radiation
  C.  Steady State: contradicts the Hubble Law
  D.  Big Bang: contradicts the Hubble Law
  E.  Oscillating Universe: predicts the universe is contracting



187.  Which of these is not a feature of the Steady State cosmology?
  A.  The universe has no beginning.
  B.  The universe has no end.
  C.  The galaxies are all rushing apart.
  D.  The galaxies are all growing larger.
  E.  Matter is created continuously in the intergalactic voids.


%% P:  The Solar System

  
Matching:  PRINT the letter of the best matching answer in each blank.

188.  \rule[-0.1pt]{4em}{0.1pt} The second planet from the sun

189.  \rule[-0.1pt]{4em}{0.1pt} Now a dwarf planet 

190.  \rule[-0.1pt]{4em}{0.1pt} Orbits between Uranus and Jupiter

191.  \rule[-0.1pt]{4em}{0.1pt} The planet larger than Venus but smaller than Neptune in size
192.  \rule[-0.1pt]{4em}{0.1pt} The largest Jovian planet


Choices:
  A.  Mercury
  B.  Mars
  C.  Jupiter
  D.  Earth
  E.  Uranus
  S.  Pluto
  T.  Neptune
  W.  Venus
  X.  Saturn



%% Q:  Perspectives


193.  The nuclei of heavy atoms composing your body were probably manufactured
  A.  in the interiors of massive stars.
  B.  in interstellar space.
  C.  in the Big Bang.
  D.  inside a black hole.
  E.  by Dow Chemical Company.



194.  Which of the following is the most distant from earth?
  A.  the Crab Nebula
  B.  the Sun
  C.  the star, Alpha Centauri
  D.  Pluto
  E.  the Andromeda Galaxy



195.  Which of these is the largest, in general?
  A.  a star
  B.  a planet
  C.  a galaxy
  D.  a globular cluster
  E.  the Solar System



196.  Our Galaxy contains about \rule[-0.1pt]{4em}{0.1pt}.
  A.  one star
  B.  400 stars
  C.  700 thousand stars
  D.  300 million stars
  E.  200 billion stars



197.  Our universe refers to \rule[-0.1pt]{4em}{0.1pt}.
  A.  the earth
  B.  the sun and its planets
  C.  the Milky Way
  D.  our local star cluster
  E.  everything there is



198.  The Astronomical Unit is the distance
  A.  from the earth to the moon.
  B.  from the earth to the sun.
  C.  from the earth's center to its surface.
  D.  traveled by light in one year.
  E.  of a star with a parallax of one arcsecond.



199.  The Main Sequence is 
  A.  the evolutionary track followed by most stars.
  B.  the stellar spectral classification sequence.
  C.  another name for the Milky Way.
  D.  a group of stars characterized by hydrogen burning in the stellar core.
  E.  a numerical sequence that predicts stellar distances.



200.  If two stars are in the same cluster, then they
  A.  have the same surface temperature.
  B.  are the same age.
  C.  are the same color.
  D.  have the same brightness.
  E.  have the same luminosity.


