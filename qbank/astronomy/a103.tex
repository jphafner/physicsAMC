
%% http://physics.gmu.edu/~jevans/astr103/Exams/a103_sample_exam.html

Vocabulary Questions

    Mark the letter of the correct answer in the appropriate blank on your Scantron Form. Each of the 30 questions in this section is worth 1 point for a total of 30 points possible in this section. The total points possible on this exam is 80.

    scientific (powers-of-ten) notation
    a. All of spacetime and all the matter and energy that occupy spacetime.
    b. Representation of numbers as a product of a number lying between one and ten times a power of ten.
    c. The amount of time since the light we see from a distant object was emitted.
    d. A postulated scheme composed of observations, concepts, and laws concerning some aspect of nature that explains observed phenomena and relationships between them and has predicative capabilities.

    synodic period
    a. The apparent shift in position of a foreground object relative to background objects due to a change of position of the observer.
    b. The time between successive configurations of a planet or the Moon as seen by an observer on the Earth.
    c. An imaginary sphere used to show the apparent location of objects in the sky.
    d. The Earth's orbital plane projected on the celestial sphere or it is the annual apparent path of the Sun on the celestial sphere.

    astronomical unit
    a. The distance light travels in a vacuum in one year.
    b. The gravitationally bound collection of 100 to 400 billion stars of which the Sun is a member.
    c. The mean distance from the Sun to the Earth used as a unit of distance in the Solar System.
    d. The Sun and all the objects that orbit it, such as planets and comets.

    parallax
    a. An imaginary sphere used to show the apparent location of objects in the sky.
    b. The distance at which the Earth/Sun distance will subtend an angle of 1 arc second.
    c. The distance traveled by light in one year.
    d. The apparent shift in position of a foreground object relative to background objects due to a change of position of the observer.

    celestial sphere
    a. The Earth's orbital plane projected on the celestial sphere or it is the annual apparent path of the Sun on the celestial sphere.
    b. The 88 regions with definite east-west and north-south boundaries that cover the whole sky.
    c. The time interval between successive configurations of a planet or the Moon as seen by an observer on the Earth.
    d. An imaginary sphere used to show the apparent location of objects in the sky.

    vernal equinox
    a. The time each year when the Sun crosses the celestial equator moving from south to north.
    b. The projection of the Earth's geographic equator on to the sky.
    c. The Earth's axis of rotation extended through the north/south geographic poles until it intersects the celestial sphere.
    d. The Earth's orbital plane projected on the celestial sphere or it is the annual apparent path of the Sun on the celestial sphere.

    ecliptic
    a. The rate in seconds of arc per year at which a star moves across the plane of the sky.
    b. The Earth's orbital plane projected on the celestial sphere or it is the annual apparent path of the Sun on the celestial sphere.
    c. An imaginary sphere used to show the apparent location of objects in the sky.
    d. It is the distance between two adjacent peaks of a wave.

    semi-major axis
    a. One-half the length of the major axis of an ellipse, which is used as a measure of the size of the ellipse.
    b. An imaginary sphere used to show the apparent location of objects in the sky.
    c. A closed curve for which the sum of the distances from two points on the figure's major axis, called foci, to any point on the figure is a constant.
    d. For an ellipse, it is the ratio of the distance of a focus from the center to the semi-major axis, which is used as a measure of the shape of the ellipse.

    eccentricity
    a. The point in a planet's orbit where it is closest to the Sun.
    b. A closed curve for which the sum of the distances from two points on the figure's major axis, called foci, to any point on the figure is a constant.
    c. One-half the length of the major axis of an ellipse, which is used as a measure of the size of the ellipse.
    d. For an ellipse, it is the ratio of the distance of a focus from the center to the semi-major axis, which is used as a measure of the shape of the ellipse.

    mass
    a. For a moving body, it is the time rate of change of the velocity of the body measured in a particular direction and frame of reference; it is produced by forces.
    b. A quantitative measure of a body�s inertia, that is, the resistance a body offers to a change in its state of motion.
    c. A push or pull that changes a body's state of motion.
    c. A fundamental measure of the quantity of motion possessed by a body that depends on the body�s velocity and mass.

    momentum
    a. Fundamental measure of the quantity of motion possessed by a body that depends on the body�s velocity and mass.
    b. Empirical laws that describe the physical conditions under which matter will produce light having one of three different spectra of colors.
    c. Energy a body possesses by virtue of its state of motion.
    c. A quantitative measure of the average kinetic energy of the particles composing a body.

    kinetic energy
    a. A quantitative measure of the average kinetic energy of the particles composing a body.
    b. The number of protons in the nucleus of the atom of a particular chemical element.
    c. Energy a body possesses by virtue of its state of motion.
    d. Energy a body possesses by virtue of its position in a field of force.

    random thermal motion
    a. Atoms or molecules behaving like small solid spheres darting about rapidly and colliding with neighbors many hundreds of millions of times per second while exchanging momentum and kinetic energy.
    b. Mathematical laws that characterize the radiation emitted by an ideal radiator, particularly the dependence on temperature.
    c. Energy a body possesses by virtue of its state of motion.
    d. Discrete representation of electromagnetic radiation which carries a discrete amount of energy depending on the wavelength.

    photon
    a. Atoms or molecules behaving like small solid spheres darting about rapidly and colliding with neighbors many hundreds of millions of times per second while exchanging momentum and kinetic energy.
    b. Energy a body possesses by virtue of its position in a field of force.
    c. Discrete representation of electromagnetic radiation which carries a discrete amount of energy depending on the wavelength.
    d. The sum of the kinetic energies in random thermal motion of all the particles composing a body.

    thermal radiation laws
    a. Atoms or molecules behaving like small solid spheres darting about rapidly and colliding with neighbors many hundreds of millions of times per second while exchanging momentum and kinetic energy.
    b. It is the distance between two adjacent peaks of a wave.
    c. Mathematical laws that characterize the radiation emitted by an ideal radiator, particularly the dependence on temperature.
    d. The apparent shift in position of a foreground object relative to background objects due to a change of position of the observer.

    impact cratering
    a. The system of tightly clustered small particles of varying compositions that orbit the equatorial regions of the Jovian planets.
    b. Various terrain shaping events, such as plate motion, volcanoes, and rift valley formation, arising from the out flow of thermal energy from the body of a terrestrial planet.
    c. Terrain shaping mechanism that arises from multiple meteoroid or larger debris impacts over time.
    d. A small body of primordial dust and ice from which the planets formed.

    Jovian planet
    a. A small body of primordial dust and ice from which the planets formed.
    b. One of a group of Solar System planets characterized by their far position relative to the Sun, large masses, low mean densities, and light element compositions.
    c. The cloud of gas and dust from which the Sun and the rest of the Solar System formed.
    d. Region around a planet defined and occupied by its magnetic field including the possibility of charged particle belts.

    protoplanetary disk
    a. A small body composed of an ice and dust mixture in orbit about the Sun. When orbiting near the Sun, solar radiation vaporizes the icy material giving rise to a coma, tails, and a hydrogen envelope. b. The region within two to three planetary radii in which the tidal force on an object is comparable to the gravitational force holding it together. c. An equatorial disk of gas, dust, and small solid particles surrounding a protoplanet or newly formed planet. d. Process by which the heavier elements in a molten protoplanet sink toward its center while lighter elements rise toward its surface.
    luminosity
    a. Relatively dark spots of varying sizes found generally in groups on the solar surface that contain intense magnetic fields.
    b. Rate (per unit of time) that radiant energy is emitted over all wavelengths from the entire surface of a star.
    c. The limiting surface surrounding a black hole inside of which nothing can escape and thus it represents the last communication point with spacetime outside.
    d. Gas and dust material that lies in between stars in the disk of our Galaxy and other spiral galaxies.

    sunspot
    a. Relatively dark spots of varying sizes found generally in groups on the solar surface that contain intense magnetic fields.
    b. The limiting surface surrounding a black hole inside of which nothing can escape and thus it represents the last communication point with spacetime outside.
    c. An imperfection in the Earth�s atmosphere seen against the surface of the Sun.
    d. The final stage in the life of a massive star when it exhausts all nuclear fuels, implodes, followed by an explosion of most of the stars mass resulting in sharp increase in intrinsic brightness.

    proper motion
    a. The apparent shift in position of a foreground object relative to background objects due to a change of position of the observer.
    b. As viewed from the Earth, the apparent motion of a planet near opposition when it moves westward relative to the background stars.
    c. The conical motion of the Earth�s axis of rotation about the perpendicular to Earth�s orbital plane over the course of approximately 26,000 years.
    d. The rate in seconds of arc per year at which a star moves across the plane of the sky.

    apparent magnitude
    a. A logarithmic measure of the brightness of a star as it appears in the sky.
    b. Rapidly rotating neutron star emitting an intense beam of radiation that is sweep around, like a lighthouse beam, by the rotation.
    c. An otherwise continuous spectrum in which appears a discrete number of dark absorption lines located at discrete wavelengths.
    d. A body in a gravitational encounter that acquires sufficient kinetic energy to escape the gravitational attraction of the other body or bodies in the encounter.

    H-R diagram
    a. Gas and dust material that lies in between stars in the disk of our Galaxy and other spiral galaxies.
    b. The final stage in the life of a massive star when it exhausts all nuclear fuels, implodes, followed by an explosion of most of the stars mass resulting in sharp increase in intrinsic brightness.
    c. The limiting surface surrounding a black hole inside of which nothing can escape and thus it represents the last communication point with spacetime outside.
    d. Diagram for stars in which the star's luminosity, or it�s equivalent, is plotted against the star's surface temperature, or it�s equivalent, and conveys information about the structure and evolution of stars.

    interstellar matter
    a. A group of several tens to several tens of thousands of stars bound together by their mutual gravitational attraction for periods longer than tens of millions of years.
    b. A body in a gravitational encounter that acquires sufficient kinetic energy to escape the gravitational attraction of the other body or bodies in the encounter.
    c. Diagram for stars in which the star's luminosity, or it�s equivalent, is plotted against the star's surface temperature, or it�s equivalent, and conveys information about the structure and evolution of stars.
    d. Gas and dust material that lies in between stars in the disk of our Galaxy and other spiral galaxies.

    proton-proton chain
    a. The state for a star in which it generates as much thermonuclear energy in its deep interior as is radiated from its surface.
    b. Diagram for stars in which the star's luminosity, or it�s equivalent, is plotted against the star's surface temperature, or it�s equivalent, and conveys information about the structure and evolution of stars.
    c. Sequence of three thermonuclear reactions that constitute primary form for hydrogen to helium conversion in stars.
    d. Locus of all stars that have ceased contraction from the interstellar medium and have initiated hydrogen fusion in their cores.

    thermal equilibrium
    a. A logarithmic measure of the brightness of a star as it appears in the sky.
    b. The state for a star in which it generates as much thermonuclear energy in its deep interior as is radiated from its surface.
    c. Locus of all stars that have ceased contraction from the interstellar medium and have initiated hydrogen fusion in their cores.
    d. Sequence of three thermonuclear reactions that constitute primary form for hydrogen to helium conversion in stars.

    pulsar
    a. Sequence of three thermonuclear reactions that constitute primary form for hydrogen to helium conversion in stars.
    b. Rapidly rotating neutron star emitting an intense beam of radiation that is sweep around, like a lighthouse beam, by the rotation.
    c. The luminosity classification for a star in the H-R diagram that is large, cool, and a high-luminosity star.
    d. Cloud of interstellar gas in which hydrogen is primarily singly ionized due to absorption of ultraviolet photons from young hot stars forming in the cloud.

    event horizon
    a. The final stage in the life of a massive star when it exhausts all nuclear fuels, implodes, followed by an explosion of most of the stars mass resulting in sharp increase in intrinsic brightness.
    b. Rapidly rotating neutron star emitting an intense beam of radiation that is sweep around, like a lighthouse beam, by the rotation.
    c. Gas and dust material that lies in between stars in the disk of our Galaxy and other spiral galaxies.
    d. The limiting surface surrounding a black hole inside of which nothing can escape and thus it represents the last communication point with spacetime outside.

    star cluster
    a. Rate (per unit of time) that radiant energy is emitted over all wavelengths from the entire surface of a star.
    b. Cloud of interstellar gas in which hydrogen is primarily singly ionized due to absorption of ultraviolet photons from young hot stars forming in the cloud.
    c. A collapsing fragment of gas and dust destined to become a self-gravitating and self-luminous gaseous body burning hydrogen.
    d. A group of several tens to several tens of thousands of stars bound together by their mutual gravitational attraction for periods longer than tens of millions of years.

    X-ray binary
    a. A close binary system containing an accreting neutron star whose accretion disk is fed by matter transferring from the companion star in the system.
    b. Matter in the form of gas and dust that lies in between stars in the disk of our Galaxy and other spiral galaxies from which new generations of stars will form.
    c. A luminous shell of gas ejected from an old, low mass star late in its evolution.
    d. Locus of all stars that have ceased contraction from the interstellar medium and have initiated hydrogen burning.

Multiple-Choice Questions

    Mark the letter of the correct answer in the appropriate blank on your Scantron Form. Each of the 50 questions in this section is worth 1 point. The total points possible on this exam is 80.

    In an investigation of a scientific theory, a logical argument can be shown to falsify one of the hypotheses upon which the theory is constructed. Therefore, the scientific community should respond to this theory in the following way.
    a. Since the hypothesis is known to be nonprovable from the beginning, the scientific community can continue to use this scientific theory with complete confidence.
    b. Since either the inclusion or exclusion of hypotheses in a scientific theory is arbitrary, the scientific community can continue to use this scientific theory with complete confidence.
    c. Assuming that the theory is constructed on several hypotheses, falsification of one means that modification, even major modification, of the theory is the appropriate next step for the scientific community.
    d. Regardless of the number of hypotheses underlying the theory, the scientific community must abandon all aspects of the theory.

    The planets as observed by the Babylonians and later the Greeks
    a. display retrograde motion over a portion of their synodic periods, in that they reverse their normal west to east motion relative to background stars.
    b. display variations in brightness, particularly Venus, over their synodic periods.
    c. display angular changes in position (or elongation) on the celestial sphere, which are not uniform over their synodic periods.
    d. all of the above.

    The point directly overhead is called
    a. the celestial equator.
    b. the zenith.
    c. the vernal equinox.
    d. the celestial sphere.

    The Sun crosses the celestial equator each year moving from south to north at the time of the
    a. summer solstice.
    b. winter solstice.
    c. vernal equinox.
    d. autumnal equinox.

    In astronomy, which term refers specifically to the orbital motion of a planet around the Sun?
    a. revolution.
    b. rotation.
    c. rebellion.
    d. spin.

    In modern astronomy, the constellations are
    a. a small number of well-defined groups of stars in our sky, each having a limited extent.
    b. 12 specific regions of our sky, through which the planets and Moon appear to move.
    c. nearby galaxies, carefully labeled for the convenience of astronomers.
    d. 88 regions with definite east-west and north-south boundaries, which cover the whole sky.

    The parsec
    a. is equal to 3.26 light years.
    b. is the distance of a star whose parallactic displacement is one second of arc.
    c. is a unit of distance characteristic of the separations between stars in the solar neighborhood.
    d. all of the above.

    If you were able to travel out into space until the angular distance between the Earth and the Sun was one second of arc, how far would you be from the Sun? (Assume that the Earth-Sun line is at right angles to your line of sight.)
    a. 1 ly.
    b. 1 km.
    c. 1 pc.
    d. 1 AU.

    Although the Ptolemaic geocentric system was a dominate influence on western thought for a very long time,
    a. it was never believed to represent reality; most natural philosophers in the centuries following Ptolemy were fairly certain that the system was Sun-centered but could not prove that to be the case.
    b. it was of very little practical value to astronomy, since the Ptolemaic system possesses no predictive capability.
    c. it was unable to persuade the intellectual community to believe that the planetary and stellar systems were any more than holes in the celestial sphere through which light from the "anti-Sun" was visible.
    d. it was not the only conceptual model in as much as the Greek natural philosopher Aristarchus had devised a heliocentric model in the 3rd century BC.

    Precession is
    a. the occasional reversal of the direction of spin of the Earth.
    b. the daily spinning motion of the Earth.
    c. the motion of the Earth along its orbital path.
    d. a very slow conical motion of the Earth's axis of rotation.

    The average distance from the Earth to the Sun, 150,000,000 km, can be written in scientific notation as
    a. 1.5 x 108 km.
    b. 1.5 x 106 km.
    c. 1.5 x 109 km.
    d. 0.15 x 10-8 km.

    The tilt of the Earth's equator relative to its orbital plane is about
    a. 23.5o.
    b. 5.2o.
    c. 0.5 arc minutes.
    d. 90o.

    A claim can be made that Kepler was the first in the line of modern scientists. What did he do or contribute to deserve such a designation?
    a. He published two important books on astronomy, one in 1543 and the other in 1687.
    b. He taught mathematics and astronomy in the University of Tubingen, which was one of the first universities in Europe.
    c. He boldly extend his laws of planetary motion derived basically for Mars to all planets, i.e., he made them universal laws.
    d. He invented the law of gravity.

    One of the consequences of Kepler's second law is that
    a. we can measure the sidereal period for any body in motion anywhere in the universe.
    b. planets move at a uniform rate around the Sun even though the orbit is an ellipse.
    c. the Sun is located at the center of the ellipse rather than at either of the foci along the major axis.
    d. planets move slowest when farthest from the Sun and fastest when closest to the Sun.

    From his harmonic law, Kepler could
    a. predict the existence of Uranus, Neptune, and Pluto.
    b. determine the mass of the Sun.
    c. determine a relative size for the known Solar System.
    d. measure the orbital periods of the superior planets.

    Venus shows phases similar to those of the Moon, and also changes apparent size with time. According to Galileo, who first saw this, these observations show that
    a. Venus orbits the Sun.
    b. Venus, like the Moon, orbits the Earth.
    c. Venus orbits the Moon.
    d. the Moon really orbits the Sun and not the Earth.

    If the Earth's spin axis were to be perpendicular to the Earth's orbital plane (the ecliptic plane), the season and seasonal variation
    a. would remain the same in severity as they are at present but each season would last twice as long.
    b. would remain the same as they are at present.
    c. would become much more severe.
    d. would be nonexistent.

    Why is the period between two successive full moons NOT equal to the Moon's orbital period, or sidereal month?
    a. because the Moon's orbit is elliptical, and the Moon therefore moves irregularly round the Earth.
    b. these two time intervals are not related, since full moon time depends upon the Moon's rotation period about its own axis.
    c. because the Earth-Moon system is also orbiting the Sun.
    d. because the Moon's orbit is inclined at about 5o to the Earth's orbital plane.

    Temperature is a measure of
    a. the gravitational attraction of atoms for each other.
    b. the mass of all the atoms composing a body.
    c. the pressure exerted by protons and neutrons on the walls of a container.
    d. the average kinetic energy of the atoms and molecules composing the body.

    In comparison to cool bodies, hot bodies
    a. emit more radiant energy in the blue region of the electromagnetic spectrum than in the red.
    b. are made up of heavier elements such as iron or the other elements around iron in the periodic table.
    c. have spectra which are emission spectra rather than continuous.
    d. can not approximate a blackbody.

    The region of stars like the Sun responsible for the emission of the luminosity is called the
    a. energy generating core.
    b. corona.
    c. photosphere.
    d. hydrogen convection zone.

    For the Sun
    a. the spectrum of the photosphere is an absorption spectrum.
    b. the Balmer series of hydrogen is clearly visible in the photospheric spectrum.
    c. limb darkening implies that temperature declines outwards through the photosphere.
    d. all of the above.

    Stars are
    a. self-gravitating and self-luminous gaseous bodies.
    b. bodies like the Sun, some being more massive and considerably more being less massive.
    c. responsible for the evolution of the chemical elements beyond hydrogen and helium in the Universe.
    d. all of the above.

    The fundamental method of determining the distances of stars depends on the
    a. phenomenon of colors from a prism.
    b. Doppler effect.
    c. parallactic phenomenon.
    d. limited wavelength sensitivity of photographic emulsions.

    Apparent magnitude is a measure of
    a. the brightness of a star as it appears in the sky.
    b. the temperature of the stars photosphere.
    c. the brightness a star would have if it was located at a distance of 32.6 ly or 10 pc from the Sun.
    d. all of the above.

    Surface temperatures for stars
    a. can be determined from application of the Wien's law, the Stefan-Boltzmann law, or Planck's law to the spectral energy distribution for a star.
    b. can be determined from the spectral type.
    c. lie between 3000o to 50,000o K.
    d. all of the above.

    The fact that strengths of absorption lines due to neutral, singly-ionized and doubly-ionized atoms vary from star to star is a consequence of.
    a. different temperatures in the photospheres of stars.
    b. different chemical compositions in the photospheres of stars.
    c. different luminosities for stars.
    d. all of the above.

    From the shape, width, and strength of absorption lines in spectra, astronomers can infer
    a. random thermal motions, existence of streams of gas, and rotation of the star
    b. the chemical composition of a star's photosphere
    c. the variation in temperature and density down through the star's photosphere
    d. all of the above

    Two stars in a binary system orbit around a common point, called the center of mass, which is
    a. closer to the more massive star.
    b. always exactly midway between the two stars.
    c. always inside one of the stars.
    d. closer to the less massive star.

    The H-R diagram for the 30 apparently brightest stars contrasted with that for the stars within 15 ly of the Sun shows that
    a. the vast majority of the nearby stars are faint dwarf stars.
    b. among the bright stars most are intrinsically brighter than the Sun.
    c. among the bright stars they appear bright not because they are nearby but because they are intrinsically bright.
    d. all of the above.

    Variable stars are
    a. stars that for some reason speed up and slow down as they orbit the center of our Galaxy.
    b. stars that change brightness because they are pulsating or in some cases there is an explosive ejection of matter.
    c. stars that are either loosing mass or gaining mass.
    d. all of the above.

    Star clusters
    a. are observed to be of three types--globular clusters, open clusters, and OB associations.
    b. are all located in the halo of our Galaxy.
    c. are remarkable similar in that their brightest stars are always blue in color.
    d. all of the above.

    H I and H II regions are distinguished from each other in that
    a. although both are glowing in visible light, H I regions are quite hot, while H II regions are actually quite cold.
    b. hydrogen is primarily singly ionized in H II regions, while hydrogen is in atomic form in H I regions.
    c. H I regions are composed of solid dust grains, while H II regions are composed of very cold hydrogen gas.
    d. all of the above.

    The primary chemical constituent of the Universe as deduced from studies of stars and gaseous nebula is
    a. carbon.
    b. hydrogen.
    c. iron.
    d. oxygen and silicon.

    The hydrogen fusion process, or hydrogen "burning," in which four hydrogen nuclei (protons) are converted into one helium nucleus accompanied by the conversion of mass directly into energy, is accomplished in stars
    a. only by the proton-proton chain.
    b. by either the proton-proton chain or the carbon-nitrogen-oxygen cycle.
    c. is an extremely short phase in the life of most stars.
    d. is an extremely disruptive phase in the stars life such that it pulsates.

    For stars in general, successive stages of thermonuclear burning
    a. are viable energy sources for shorter and shorter periods of time as synthesis proceeds toward heavier elements.
    b. are only possible if gravitational contraction can raise core temperatures higher and higher.
    c. use the end product or "ash" of the preceding stage as the fuel.
    d. all of the above.

    The rate at which stars evolve
    a. is determined by their chemical composition.
    b. is determined by their mass with low-mass stars being the longest-lived stars.
    c. is determined by a number of factors most of which we do not know.
    d. is determined by their age with old stars evolving the fastest.

    The Crab Nebula is an example of what type of astronomical phenomenon?
    a. a planetary nebula, a shell of gas leaving an old star.
    b. a supernova explosion.
    c. a gas and dust cloud, the formation region for new stars.
    d. a spiral galaxy, a collection of 100 billion or more stars.

    Einstein's special relativity theory is based on two assumptions, which are usually stated as
    a. physical laws have the same form for all inertial (uniformly moving) observers; the measured speed of light is the same for all inertial observers, independent of the motion of the source or observer.
    b. forces in nature must decrease in value as the inverse square of the distance between the bodies; forces must also increase as the product of the masses of the bodies involved.
    c. space and time are absolutes independent of any considerations concerning matter or its distribution in the universe; time may only go forward never backward.
    d. there is no center for the universe; there is no boundary to the universe.

    A black hole is
    a. a star with a temperature of 0o K.
    b. at the centers of all stars, providing their energy by gravitational collapse.
    c. strongly curved spacetime.
    d. all of the above.

    Studies of the stellar composition of our Galaxy suggest that
    a. most of its stars are main-sequence F through M stars
    . b. white dwarfs are the second most populous group of stars in the Galaxy.
    c. red giants are at most about one-half of one percent of all stars in the Galaxy.
    d. all of the above.

    The majority of the luminosity of the Galaxy is provided by
    a. primarily the red and blue supergiants.
    b. late-type main-sequence stars.
    c. O and B stars, since they are intrinsically so luminous.
    d. stars intrinsically brighter than the red and white dwarfs.

    The possible presence of a very large amount of unseen or dark matter in the halo of our Galaxy is deduced from
    a. the unexpected absence of luminous matter (stars, etc.) beyond a certain distance.
    b. the rotation curve of our Galaxy, which shows orbital speeds in the outer parts of the Galaxy to decrease in an unexpected Keplerian fashion.
    c. the unexpected high amount of interstellar absorption in certain directions.
    d. the rotation curve of our Galaxy, which indicates higher than expected orbital speeds in the outer regions of the Galaxy.

    Astronomers have used variations of the inverse-square technique
    a. to estimate the gravitational repulsion between nearby stars.
    b. to determine the large-scale structure of the universe.
    c. to establish a set of astronomical objects as rungs on a distance ladder that is supported by the direct measure of stellar parallax for nearby stars.
    d. all of the above.

    In spiral galaxies, the visible arms stand out over the disk stars because of the brightness of
    a. globular clusters.
    b. open clusters and red dwarfs.
    c. red supergiants and giants.
    d. blue supergiants and H II regions.

    Which of the following is the most common state in which matter occurs in the universe?
    a. plasma.
    b. liquid.
    c. molecular compound.
    d. solid.

    The significance of the Hubble sequence is that it is
    a. an evolutionary sequence for galaxies.
    b. a sequence of increasing mass.
    c. a dynamical and population sequence.
    d. a chemical evolution sequence.

    Like our Galaxy, other galaxies
    a. have much of their mass tied up in stars that make only small contributions to their luminosities.
    b. owe their luminosity to their intrinsically bright stellar members.
    c. possess stars of differing ages.
    d. all of the above.

    The spiral-density-wave theory
    a. postulates that expanding compressional waves spread out from the centers of spiral galaxies, and where interstellar matter encounters these compressional waves, it is compressed, which in turn initiates star formation.
    b. states that a newly formed star, or protostar, first appears on the H-R diagram as a bright, cool object in the red-giant region.
    c. shows that the equation of state for a perfect gas is one for which the pressure in the gas is proportional to the gas' density and temperature.
    d. all of the above.

    Dividing normal spirals and the barred spirals into subgroups a, b, and c is based on which of the following considerations?
    a. population types in the spiral arms and nucleus.
    b. number of H II regions in the spiral arms.
    c. density of blue and red supergiants in the spiral arms, nucleus and halo.
    d. tightness of the arms and the relative size of the nucleus to the disk.

    Which of the following is thought to be correct concerning galaxies?
    a. The spectrum of a galaxy has the general appearance of being the composite of the spectra of a large number of stars.
    b. The velocity-distance relation exhibited by galaxies in which their blueshifts (which are used to obtain their velocities) are inversely proportional to their distances squared.
    c. Elliptical galaxies have a larger percentage of their mass in the form of interstellar matter than do spiral galaxies.
    d. all of the above.

    The redshift of the absorption lines in the spectra of very distant galaxies has been interpreted by astronomers to be due to the
    a. reddening of light moving through intergalactic space by intergalactic dust.
    b. gravitational redshift of thousands of galaxies falling into black holes.
    c. the "tiredness" of light traveling over the immense distances of the universe.
    d. the expansion of space in the universe.

    Quasars
    a. appear to increase in number with distance or back in time due to the look-back effect meaning that they were more prevalent in the early universe.
    b. appear to be the large central galaxy of a rich cluster of galaxies which is cannibalizing neighboring small galaxies.
    c. appear to be the strongest radio sources in the universe far eclipsing the radio galaxies.
    d. all of the above.

    Hubble's constant is
    a. the magnitude of the gravitational attraction between superclusters of galaxies.
    b. one of the most important constants found in general relativity.
    c. the relationship between mass and luminosity for main sequence stars.
    d. the slope of the straight-line velocity-distance relation for galaxies in the Hubble law.

    Which of the following statements regarding the structure of the universe is true?
    a. While the typical sizes of galaxies is on the order of tens to hundreds of thousands of light years, the typical sizes of the superclusters is tens to hundreds of millions of light years.
    b. On the scale of billions of light years, galaxies are more homogeneously distributed than they are on smaller scales.
    c. On the scale size of the visible universe, individual galaxies are insignificantly small (about one thousandth of a percent).
    d. all of the above.

    Which of the following statements is true?
    a. The cosmic background radiation appears to be coming from all directions in space with the same intensity.
    b. The ghost-like world of the cosmic background radiation exists all about us, and yet does not interact with the vast majority of our physical existence.
    c. Because the universe appears to be isotropic we assume it to be homogeneous in the cosmological principle.
    d. all of the above.

    The three types of geometry that arise in consideration of Friedmann's cosmological models are
    a. Euclidean.
    b. spherical.
    c. hyperbolic.
    d. all of the above.

    Which of the following is true?
    a. The universe has no center and no edge.
    b. It is through the theory of general relativity that we hope to understand why the universe started off so very flat and smooth in order that it possess those properties now, since general relativity is a complete theory whose mathematical laws were in operation before, during, and after the big bang.
    c. According to Einstein's concept of space and time, there is space beyond the universe, and there is time before and after the universe, i.e. space and time are absolutes independent of the universe.
    d. all of the above.

    Several problems exist in trying to fit Friedmann's big bang cosmological models to the observed universe. These problems are known as the flatness problem--the present geometry of the universe is incredibly close to Euclidean--and the smoothness problem--the universe is amazingly homogeneous. Why are these considered problems?
    a. A flat Euclidean and homogeneous universe is highly improbable since it is but one of a double infinity of models in Friedmann's cosmology.
    b. Hubble's law is inconsistent with a flat, homogeneous universe in as much as spacetime should be warped according to the law.
    c. The problem really stems from the attitude of certain cosmologists who are biased in their opinions against a flat, homogeneous universe.
    d. all of the above.

    de Sitter's cosmological model differed from that of Einstein in what ways?
    a. de Sitter's universe contains no matter whereas Einstein's does.
    b. de Sitter's universe is open whereas Einstein's is closed.
    c. de Sitter's universe predicts a velocity-distance relationship whereas Einstein's universe is static.
    d. all of the above.


