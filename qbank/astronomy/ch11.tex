

Astronomy
Learning Objectives and Study Questions for
Chapter 11
1.  Calculate a star's distance from Earth given its parallax or vice versa.
2.  Determine which of two stars is brighter, and by how many times, given their magnitudes .
3.  Use the distance-magnitude relationship to calculate a star's M, m, or d given the other two quantities.
4.  Determine the relative brightnesses and temperatures of two stars that are the same distance away but have different spectral or luminosity classes (for example: a G2V and a G2II ; or an A5V and a K5V).
5.  Locate the fields where the following types of stars would plot on an H-R diagram: supergiants, giants, main-sequence stars, white dwarfs, and red dwarfs.
6.  Determine which is brighter (at a given distance)—an F0 white dwarf or an F 0 main-sequence star—and explain why. Similarly, determine which is brighter—an F0 giant or an F0 main-sequence star—and explain why.
7.  Knowing the orbital distance and period of two stars in a binary system as well as their individual distances from t he system's center of mass, calculate both their total mass and individual masses.
8.  Sketch the shape of an eclipsing binary system's light curve from a series of drawings of the orbiting stars (see Figure 11-13, p. 334).


1.  A star that shows 0.05 arcseconds parallax is _____ parsecs away from Earth. 
A.  0.05
B.  5
C.  20
D.  50
E.  500

2.  A star that is 10 light years away from Earth displays _____ arcseconds of parallax.
A.  0.1
B.  0.326
C.  10
D.  32.6
E.  100

3.  Star α has an apparent magnitude of 7 and star β has an apparent magnitude of 3. Which is brighter from Earth?
A.  α
B.  β
C.  Both are the same brightness
D.  It depends on the observer
E.  indeterminate, cannot tell from the data given distances

4.  Star α has an apparent magnitude of 7 and star β has an apparent magnitude of 3. Which is truly brighter?
A.  α
B.  β
C.  both are the same brightness
D.  it depends on the observer


A.
indeterminate, canno
t tell from the data given
5.
Star 
δ
has an apparent magnitude (m) of 7 and is found, from parallax, to be 100 pc away. 
What is this star’s absolute magnitude (M)?
A.
2
B.
7
C.
12
D.
70
E.
indeterminate, cannot tell from the data given
6.
A star cluster has four prominent stars whose spectral classes are: 
F9, A3, G2, and K0. Which 
of these stars is the hottest?
A.
F9
B.
A3
C.
G2
D.
K0
E.
indeterminate, canno
t tell from the data given
7.
Sol is in luminosity class V, meaning it is a _____.
A.
main sequence star
B.
subgiant 
C.
giant
D.
bright giant
E.
supergiant
8.
Some stars, like Procyon B, 
are hotter than Sol and yet much dimmer (M = 12.5 vs. 5). Such 
stars must be
_____.
A.
much farther away
B.
much smaller (white dwarfs)
C.
subgiants 
D.
near the ends of their lives
E.
much younger 
9.
Spectroscopic parallax 
enables us to determine a star’s _____ using its 
spectrum and an H
-
R 
diagram.
A.
energy output
B.
age
C.
composition
D.
distance
E.
fate 
10.
In a binary system, two stars are seen to be separated by 4.0AU and to have an orbital 
period of 3.27 years.  What is the 
total mass 
of the stars in this system?
A.
1.2
B.
1.5
C.
2.5
D.
6.0
E.
19.6
11.
To calculate the masses of the individual stars in a binary system we need to kn
ow their 
separation distance, a
, orbital period, p, and _____.
A.
distances from common center of mass
B.
absolute magnitudes
C.
ages
D.
numbers of planets
E.
spectral clas
