
%% Practice Exam for ASTR I, EXAM 5 on Cosmology
%%--------------------------------------------------

%% this section contains 18 problems

%% ANSWERS:
% 1.  c
% 2.  e
% 3.  a
% 4.  e
% 5.  c
% 6.  e
% 7.  a
% 8.  a
% 9.  a
% 10. d
% 11. c
% 12. b
% 13. a
% 14. a
% 15. c
% 16. d
% 17. d
% 18. a

\element{astr}{
\begin{question}{ASTRI-5-Q01}
    Which one of the following is the most complete statement of the Cosmological Principle?
    \begin{choices}
        \wrongchoice{the universe appears the same to all observers}
        \wrongchoice{the appearance of the universe depends on the direction of observation}
      \correctchoice{on a large scale, the universe appears the same to all observers}
        \wrongchoice{the universe is unchanging throughout all time}
        \wrongchoice{the distribution of matter is uniform, but the appearance depends on the direction of observation}
    \end{choices}
\end{question}
}

\element{astr}{
\begin{questionmult}{ASTRI-5-Q02}
    Which of the following are possible universes according to the solutions of Einstein’s equations for the geometry of the universe?
    \begin{choices}
      \correctchoice{open}
      \correctchoice{closed}
      \correctchoice{flat}
        %\wrongchoice{a and b only}
        %\correctchoice{a, b, and c}
    \end{choices}
\end{questionmult}
}

\element{astr}{
\begin{question}{ASTRI-5-Q03}
    A universe with a positive curvature is considered a:
    \begin{multicols}{2}
    \begin{choices}
      \correctchoice{closed universe}
        \wrongchoice{open universe}
        \wrongchoice{static universe}
        \wrongchoice{flat universe}
    \end{choices}
    \end{multicols}
\end{question}
}

\element{astr}{
\begin{question}{ASTRI-5-Q04}
    The critical density for the universe is
    \begin{choices}
        \wrongchoice{the density of matter in the Milky Way}
        \wrongchoice{the density at which an object becomes a black hole}
        \wrongchoice{the number density of galaxies in a typical cluster of galaxies}
        \wrongchoice{the density at the center of the Big Bang}
      \correctchoice{the density of a flat universe}
    \end{choices}
\end{question}
}

\element{astr}{
\begin{question}{ASTRI-5-Q05}
    If the density of the universe is greater than the critical density,
        the universe will:
    \begin{choices}
        \wrongchoice{continue to expand forever}
        \wrongchoice{eventually stop expanding}
      \correctchoice{eventually stop expanding and begin contraction}
        \wrongchoice{continue to contract forever}
        \wrongchoice{eventually stop contracting and begin expanding}
    \end{choices}
\end{question}
}

\element{astr}{
\begin{question}{ASTRI-5-Q06}
    The critical density is roughly:
    \begin{choices}
        \wrongchoice{the density of air at sea level on Earth}
        \wrongchoice{the density at the center of the Sun}
        \wrongchoice{the density of a white dwarf}
        \wrongchoice{the density of the interstellar medium ($\approx 1$ proton per cubic centimeter)}
      \correctchoice{about 1 proton per cubic meter}
    \end{choices}
\end{question}
}

\element{astr}{
\begin{question}{ASTRI-5-Q07}
    The density of the observable matter in the universe indicates that the universe is:
    \begin{multicols}{2}
    \begin{choices}
      \correctchoice{open}
        \wrongchoice{closed}
        \wrongchoice{flat}
        \wrongchoice{static}
    \end{choices}
    \end{multicols}
\end{question}
}

\element{astr}{
\begin{question}{ASTRI-5-Q08}
    To determine whether the expansion of the universe is decelerating or accelerating requires
    \begin{choices}
      \correctchoice{observing very faint galaxies}
        \wrongchoice{observing very bright galaxies}
        \wrongchoice{observing nearby galaxies}
        \wrongchoice{observing the motions of binary galaxies}
        \wrongchoice{observing the microwave background radiation}
    \end{choices}
\end{question}
}

\element{astr}{
\begin{question}{ASTRI-5-Q09}
    A large amount of deuterium in the universe would mean the universe is:
    \begin{choices}
      \correctchoice{open}
        \wrongchoice{closed}
        \wrongchoice{flat}
        \wrongchoice{static}
    \end{choices}
\end{question}
}

\element{astr}{
\begin{question}{ASTRI-5-Q10}
    What percentage of the mass of the universe was formed into helium during the Big Bang?
    \begin{multicols}{2}
    \begin{choices}
        \wrongchoice{\SI{1}{\percent}}
        \wrongchoice{\SI{5}{\percent}}
        \wrongchoice{\SI{10}{\percent}}
      \correctchoice{\SI{25}{\percent}}
        \wrongchoice{\SI{50}{\percent}}
    \end{choices}
    \end{multicols}
\end{question}
}

\element{astr}{
\begin{question}{ASTRI-5-Q11}
    The temperature of the universe 1 second after the Big Bang was approximately:
    \begin{multicols}{2}
    \begin{choices}
        \wrongchoice{10 (to the 6th power) K}
        \wrongchoice{10 (to the 8th power) K}
      \correctchoice{10 (to the 10th power) K}
        \wrongchoice{10 (to the 15th power) K}
        \wrongchoice{10 (to the 20th power) K}
    \end{choices}
    \end{multicols}
\end{question}
}

\element{astr}{
\begin{question}{ASTRI-5-Q12}
    The assumption that the general structure of the universe is the same everywhere is the assumption of
    \begin{multicols}{2}
    \begin{choices}
        \wrongchoice{isotropy}
      \correctchoice{homogeneity}
        \wrongchoice{regularity}
        \wrongchoice{universality}
        \wrongchoice{constancy}
    \end{choices}
    \end{multicols}
\end{question}
}

\element{astr}{
\begin{question}{ASTRI-5-Q13}
    When we observe distant galaxies we are observing:
    \begin{multicols}{2}
    \begin{choices}
      \correctchoice{very young objects}
        \wrongchoice{very old objects}
        \wrongchoice{objects having the approximate age of the Milky Way}
        \wrongchoice{distant galaxies; no statement may be made about age}
    \end{choices}
    \end{multicols}
\end{question}
}

\element{astr}{
\begin{question}{ASTRI-5-Q14}
    Except for hydrogen, the most abundant element formed in the Big Bang was:
    \begin{multicols}{2}
    \begin{choices}
      \correctchoice{helium}
        \wrongchoice{lithium}
        \wrongchoice{oxygen}
        \wrongchoice{carbon}
        \wrongchoice{nitrogen}
    \end{choices}
    \end{multicols}
\end{question}
}

\element{astr}{
\begin{question}{ASTRI-5-Q15}
    Deuterium in the universe was produced:
    \begin{choices}
        \wrongchoice{by fusion in stellar interiors}
        \wrongchoice{by fission in stellar interiors}
      \correctchoice{during the initial Big Bang}
        \wrongchoice{by high density radiation near stars}
        \wrongchoice{none of the above, it only exists on the Earth}
    \end{choices}
\end{question}
}

\element{astr}{
\begin{question}{ASTRI-5-Q16}
    Olbers' paradox results from asking the question:
    \begin{choices}
        \wrongchoice{What is truth?}
        \wrongchoice{What is a quasar?}
        \wrongchoice{What is the nature of the Universe?}
      \correctchoice{Why is the night sky dark?}
        \wrongchoice{What am I doing here?}
    \end{choices}
\end{question}
}

\element{astr}{
\begin{question}{ASTRI-5-Q17}
    A difficulty with inflationary models of the universe is that they fail to explain:
    \begin{choices}
        \wrongchoice{where the universe came from}
        \wrongchoice{the origin of life}
        \wrongchoice{the formation of the solar system}
      \correctchoice{how galaxies, clusters and superclusters formed}
        \wrongchoice{the origin of the microwave background radiation}
    \end{choices}
\end{question}
}

\element{astr}{
\begin{question}{ASTRI-5-Q18}
    Grand Unified Theories connect all but which one of the four forces in nature?
    \begin{choices}
      \correctchoice{gravity}
        \wrongchoice{electromagnetism}
        \wrongchoice{weak nuclear}
        \wrongchoice{strong nuclear}
    \end{choices}
\end{question}
}

\endinput


