
%% University of Northern Iowa
%%  Morgans Astronomy Exams
%%--------------------------------------------------

%% this section contains XX problems

\element{morgans}{
\begin{question}{test2H-Q01}
    Which of the following is \emph{not} produced during the proton-proton reaction?
    \begin{choices}
        \wrongchoice{helium}
        \wrongchoice{deuterium}
      \correctchoice{gallium}
        \wrongchoice{positron}
    \end{choices}
\end{question}
}

\element{morgans}{
\begin{question}{test2H-Q02}
    The temperature of the photosphere is roughly:
    \begin{choices}
      \correctchoice{5000 K.}
        \wrongchoice{50,000 K.}
        \wrongchoice{100,000 K.}
        \wrongchoice{1,000,000 K.}
    \end{choices}
\end{question}
}

\element{morgans}{
\begin{question}{test2H-Q03}
    Which of the following statements about the Solar corona is \emph{false}?
    \begin{choices}
        \wrongchoice{it is a source of X rays}
      \correctchoice{it has a smooth, spherical shape}
        \wrongchoice{it has a temperature of 1 to 2 million K}
        \wrongchoice{prominences are found there}
    \end{choices}
\end{question}
}

\element{morgans}{
\begin{question}{test2H-Q04}
    If it were possible to observe stellar parallax from another planet,
        which planet listed below would provide the most accurate distances to the farthest stars?
    \begin{choices}
        \wrongchoice{Venus}
        \wrongchoice{Mars}
        \wrongchoice{Jupiter}
        \wrongchoice{Saturn}
      \correctchoice{Pluto}
    \end{choices}
\end{question}
}

\element{morgans}{
\begin{question}{test2H-Q05}
    Who is known for classifying over 200,000 stars and producing the Henry Draper Catalog?
    \begin{choices}
        \wrongchoice{Henry Draper}
        \wrongchoice{Edward C. Pickering}
        \wrongchoice{Henrietta Leavitt}
      \correctchoice{Annie J. Cannon}
    \end{choices}
\end{question}
}

\element{morgans}{
\begin{question}{test2H-Q06}
    Which of the following is NOT in itself useful for determining stellar temperature?
    \begin{choices}
        \wrongchoice{spectral class}
        \wrongchoice{color}
      \correctchoice{absolute magnitude}
        \wrongchoice{degree of ionization}
        \wrongchoice{Wien's Law}
    \end{choices}
\end{question}
}

\element{morgans}{
\begin{question}{test2H-Q07}
    Star $A$ is a G3V, while Star $B$ is a K3I. 
    Which star is more luminous?
    \begin{choices}
        \wrongchoice{$A$}
      \correctchoice{$B$}
        \wrongchoice{they have the same luminosity}
        \wrongchoice{not possible to determine luminosity from the information}
    \end{choices}
\end{question}
}

\element{morgans}{
\begin{question}{test2H-Q08}
    Which of the following represents the approximate range of stellar masses?
    \begin{choices}
        \wrongchoice{all stars have roughly the same mass}
        \wrongchoice{1.0 to 10 solar masses}
        \wrongchoice{0.1 to 10 solar masses}
      \correctchoice{0.05 to 50 solar masses}
        \wrongchoice{0.01 to 1000 solar masses}
    \end{choices}
\end{question}
}

\element{morgans}{
\begin{question}{test2H-Q09}
    Low mass stars burn hydrogen in the \rule[-0.1pt]{4em}{0.1pt};
        massive stars burn hydrogen in the \rule[-0.1pt]{4em}{0.1pt}.
    \begin{choices}
        \wrongchoice{proton-proton chain; proton-proton chain}
        \wrongchoice{carbon-nitrogen-oxygen cycle; carbon-nitrogen-oxygen cycle}
      \correctchoice{proton-proton chain; carbon-nitrogen-oxygen cycle}
        \wrongchoice{carbon-nitrogen-oxygen cycle; proton-proton chain}
        \wrongchoice{triple-alpha reaction; proton-proton chain}
    \end{choices}
\end{question}
}

\element{morgans}{
\begin{question}{test2H-Q10}
    The production of more and more massive chemical elements requires ever increasing:
    \begin{choices}
        \wrongchoice{amounts of hydrogen.}
        \wrongchoice{helium.}
        \wrongchoice{iron.}
        \wrongchoice{time.}
      \correctchoice{temperature.}
    \end{choices}
\end{question}
}

\element{morgans}{
\begin{question}{test2H-Q11}
    Stars on the upper end of the main sequence next evolve into:
    \begin{choices}
        \wrongchoice{red dwarfs.}
        \wrongchoice{lower main sequence stars.}
        \wrongchoice{solar-type stars.}
        \wrongchoice{white dwarfs.}
      \correctchoice{red giants.}
    \end{choices}
\end{question}
}

\element{morgans}{
\begin{question}{test2H-Q12}
    Protostars in dark,
        dusty regions may be studied in the \rule[-0.1pt]{4em}{0.1pt} spectral region.
    \begin{choices}
        \wrongchoice{X-ray}
        \wrongchoice{ultraviolet}
        \wrongchoice{visual}
      \correctchoice{infrared}
        \wrongchoice{gamma-ray}
    \end{choices}
\end{question}
}

\element{morgans}{
\begin{question}{test2H-Q13}
    As a star converts its hydrogen to helium,
        the position of the star in the H-R diagram moves mostly toward:
    \begin{choices}
        \wrongchoice{higher density.}
        \wrongchoice{lower temperature.}
      \correctchoice{higher luminosity.}
        \wrongchoice{lower luminosity.}
        \wrongchoice{lower radius.}
    \end{choices}
\end{question}
}

\element{morgans}{
\begin{question}{test2H-Q14}
    Which of the following objects is electron degenerate?
    \begin{choices}
        \wrongchoice{A main sequence star.}
        \wrongchoice{A black hole.}
      \correctchoice{A white dwarf.}
        \wrongchoice{A neutron star.}
    \end{choices}
\end{question}
}

\element{morgans}{
\begin{question}{test2H-Q15}
    In the most massive stars,
        the heaviest element which will be produced in the core will be:
    \begin{choices}
        \wrongchoice{helium.}
        \wrongchoice{oxygen.}
        \wrongchoice{silicon.}
      \correctchoice{iron.}
    \end{choices}
\end{question}
}

\element{morgans}{
\begin{question}{test2H-Q16}
    A star which has a main sequence mass of 10 solar masses will most likely end up as:
    \begin{choices}
        \wrongchoice{a T Tauri star.}
        \wrongchoice{a white dwarf.}
      \correctchoice{a neutron star.}
        \wrongchoice{a black hole.}
    \end{choices}
\end{question}
}

\element{morgans}{
\begin{question}{test2H-Q17}
    In a galaxy,
        supernova will be observed on average \rule[-0.1pt]{4em}{0.1pt} every century.
    \begin{choices}
        \wrongchoice{1000 times}
        \wrongchoice{10 times}
      \correctchoice{once}
        \wrongchoice{hardly ever}
    \end{choices}
\end{question}
}

\element{morgans}{
\begin{question}{test2H-Q18}
    How can astronomers determine which type of supernovae they are observing?
    \begin{choices}
        \wrongchoice{Type I supernovae fade much more quickly than Type II}
        \wrongchoice{the brightnesses are different}
      \correctchoice{the spectral features are different}
        \wrongchoice{by determining exactly which object produced the supernova}
    \end{choices}
\end{question}
}

\element{morgans}{
\begin{question}{test2H-Q19}
    A millisecond pulsar may realistically have a frequency of approximately:
    \begin{choices}
      \correctchoice{a million pulses per second.}
        \wrongchoice{1000 pulses per second.}
        \wrongchoice{1 pulse per second.}
        \wrongchoice{1 pulse per 10 seconds.}
        \wrongchoice{1 pulse per minute.}
    \end{choices}
\end{question}
}

\element{morgans}{
\begin{question}{test2H-Q20}
    Which of the following lists the stellar remnants in order of decreasing maximum mass?
    \begin{choices}
        \wrongchoice{neutron star, white dwarf, black hole}
      \correctchoice{black hole, neutron star, white dwarf}
        \wrongchoice{white dwarf, black hole, neutron star}
        \wrongchoice{black hole, white dwarf, neutron star}
        \wrongchoice{they all have approximately the same mass}
    \end{choices}
\end{question}
}

\begin{comment}
    Fill In
    Place the most appropriate word or words in the blank. You may have to click on the blank to activate it before you start typing in your answer.
     
    A rotating disk of gas surrounding a compact object formed by material falling inward is called a(n) .

    The specific nuclear reaction which produces energy in the Sun is called the .

    The two most abundant elements in the Sun are .

    The most important use of binary stars is in determining the property of .

    The specific hydrogen burning reaction occurring in stars more massive than the Sun is the .

    The very hot, extended outer atmosphere of the Sun and other cool main sequence stars is called the .

    A star in the stage would be found to the left of the upper part of the main sequence on an H-R diagram..

    Hydrogen burning for a 1 M star lasts for years.

    A star with a main sequence mass of 40 or more solar masses will end up as a(n) .

    White dwarf stars cannot be more massive than .
\end{comment}


\endinput


