

PRACTICE EXAM 4 on Galaxies
 
   1. The general shape of our galaxy is nearest to that of a(n)
      a.  pear.
      b.  egg.
      c.  ball.
      d.  fried egg.
      
   2. The distance of the Sun from the center of the Galaxy is
      nearest to __________ parsec(s) or __________ light years.
      a.  1; 3.26
      b.  100; 326
      c.  1,000; 3,260
      d.  10,000; 32,600
      e.  30,000; 100,000
         
   3. Which of the following objects or techniques is most useful to
      determine the distance to the most distant objects in the Galaxy?
      a.  solar-type stars
      b.  Cepheid variables
      c.  RR Lyrae variables
      d.  main sequence fitting methods
      e.  white dwarf stars
      
   4. The period of revolution of the Sun about the galactic center is closest to
      a.  3 million years.
      b.  25 million years.
      c.  250 million years.
      d.  2 billion years.
      e.  25 billion years.
      
   5. Spiral arms appear to be prominent in spiral galaxies because
      a.  all the stars are distributed in a spiral pattern.
      b.  cool stars are distributed in a spiral pattern while the
          hot stars are spread more uniformly.
      c.  hot stars are distributed in a spiral pattern while the
          cool stars are spread more uniformly.
      d.  globular clusters are distributed in a spiral pattern
          around the Galaxy.
      
   6. Our ability to detect distant stars in our galaxy is limited because of
      a.  absorption by dust in the Galaxy.
      b.  the existence of strong 21-cm radiation in the Galaxy.
      c.  the existence of many bright nebulae in the Galaxy.
      d.  none of the above; there is no limit in our ability to detect distant stars.
      
   7. The primary use of observations of 21-cm radiation is to determine
      a.  distances to distant stars.
      b.  the chemical composition of the interstellar medium.
      c.  the spiral structure of the Galaxy.
      d.  the distance of the Sun from the center of the Galaxy.
      
   8. The nucleus of the Galaxy is composed primarily of
      a.  hot stars.
      b.  cool stars.
      c.  hydrogen gas.
      d.  planetary nebulae.
      e.  solar-type stars.
      
   9. The most massive component of the Milky Way Galaxy is
      a.  the disk.
      b.  the nucleus.
      c.  the halo.
      d.  the spiral arms.
      
  10. Astronomers can usually differentiate Population II stars near the Sun from Population I stars by
      a.  the colors of the stars.
      b.  the amount of interstellar reddening in each.
      c.  the strength of radio emission from each.
      d.  the difference in Doppler shift for each.
      e.  none of the above; it is not possible to differentiate these two classes of stars.
      
  11. The density wave which is thought to travel throughout the Galaxy is most analogous to
      a.  a water wave.
      b.  a sound wave.
      c.  a light wave.
       
  12. As evidenced by the concentration of gas and dust in the spiral arms, a density wave has its strongest effect on
      a.  stars.
      b.  gas and dust.
      c.  star clusters.
      d.  binary stars.
      e.  other nearby galaxies.
       
  13. The observed distribution of globular clusters indicates that the gas cloud from which the Galaxy formed was
      a.  disk-shaped.
      b.  spherical.
      c.  elliptical.
 
  14. We know the stars observed in globular clusters all have relatively low masses because
      a.  the more massive ones have all evolved and are ``dead'' stars.
      b.  the high mass stars were ejected from the clusters by tidal interactions with the Milky Way.
      c.  the high mass stars were ejected from the cluster in collisions with the Magellanic Clouds.
      d.  the high mass stars are too faint to observe.
      e.  only low mass stars formed in globular clusters.
      
  15. Important differences between stars of Pop. I and Pop. II are that
      a.  Pop. II stars are young and move slowly around the Galaxy.
      b.  Pop. II stars are metal poor and young.
      c.  Pop. II stars are metal poor stars in the galactic halo.
      d.  Pop. II stars are slowly moving stars in the galactic disc.
      e.  Pop. II stars are young, metal rich stars in the galactic nucleus.
      
  16. An argument against the evolution of galaxies from spiral to elliptical is the fact that
      a.  both types of galaxies show current star formation.
      b.  neither type of galaxy shows current star formation.
      c.  both types of galaxies contain equally old stars (10 billion years).
      d.  the oldest stars in elliptical galaxies are much older than the oldest stars in spiral galaxies.
      
  17.  The Magellanic Clouds are
      a.  irregular galaxies.
      b.  spiral galaxies.
      c.  elliptical galaxies.
      d.  large clouds of gas and dust.
             
  18. The distances to the nearest galaxies can be determined using
      a.  spectroscopic parallax.
      b.  main sequence fitting.
      c.  proper motions.
      d.  RR Lyrae stars.
      e.  Cepheid variables.
       
   19. Which of the following is NOT an example of a standard candle?
      a.  Cepheids
      b.  supergiants
      c.  the brightest galaxy in a cluster
      d.  supernovae
      e.  pulsars
      
  20. The Tully-Fisher method for finding distances to galaxies relates absolute magnitude and
      a.  redshift.
      b.  rotational velocity.
      c.  magnetic field.
      d.  angular size.
      e.  apparent magnitude.
      
 21. What is the easiest way of measuring the rotation of a spiral galaxy?
      a.  pick out individual stars, and measure their Doppler shifts
      b.  use 21-cm emission to measure the motion of the gas in the disk
      c.  look for supernova, and observe how they expand
      d.  determine the location and distribution of globular clusters within the galaxy
      
 22. Which type of galaxies, have a wider range of luminosities and sizes?
      a.  Ellipticals
      b.  Spirals
      c.  SO
      
 23. Galaxies in clusters
      a.  tend to move randomly within the cluster, and eventually will leave the cluster.
      b.  do not orbit, but are fixed in space.
      c.  orbit about the center of the cluster.
      d.  often merge in the center to form a giant spiral galaxy.
                
 24. The most numerous galaxies in the Local Group are _______ galaxies.
      a.  elliptical
      b.  spiral
      c.  irregular
      
 25. Most of the galaxies in rich clusters are
      a.  normal spirals.
      b.  barred spirals.
      c.  elliptical.
      d.  irregular.
      e.  divided equally among all types.
      
 26. The X rays observed to come from many clusters of galaxies are explained by
      a.  a black hole in the cluster center.
      b.  mass transfer onto neutron galaxies.
      c.  highly energetic collisions of galaxies.
      d.  a hot intergalactic medium.
      e.  hot galactic halos.
      
 27. Spectra of distant galaxies show
      a.  a large red shift.
      b.  a large blue shift.
      c.  no spectral shift.
      d.  a small red shift.
      e.  a small blue shift.
 
 28. The Hubble law may be expressed mathematically as
      a.  v = Hd.
      b.  v = H/d.
      c.  v = Hd2.
      d.  v = H/d2.
      e.  a very complicated formula.
      
 29. The Hubble constant is related to the __________ of the universe.
      a.  size
      b.  mass
      c.  age
      d.  luminosity
      
 30. The fact that jets are NOT observed in all radio galaxies
      a.  proves that there is not a black hole at their core.
      b.  proves that there is a black hole at their core.
      c.  indicates that our orientation to the galaxy may prevent us from seeing them.
      d.  the question is incorrect, jets are seen around all radio galaxies.
      
 31. The radio emission in a typical radio galaxy is produced by _______ emission.
      a.  thermal
      b.  synchrotron
      c.  black hole
      d.  cyclotron
      e.  thermonuclear
      
 32. Seyfert galaxies are characterized by
      a.  small reddish nuclei.
      b.  large reddish nuclei.
      c.  large bluish nuclei.
      d.  small bluish nuclei.
      
 33. The broad spectral lines formed in Seyfert galaxy nuclei are indicative of
      a.  rapid motion by the galaxy away from the observer.
      b.  rapid motion by the galaxy towards the observer.
      c.  the presence of magnetic fields.
      d.  large turbulent velocities in the nucleus.
      e.  a degenerate nucleus.
      
 34. The most conspicuous observed property of quasars is their
      a.  low apparent magnitude (high apparent brightness).
      b.  high redshift.
      c.  high blueshift.
      d.  dramatic appearance on photographs.
      e.  strong spiral structure.
 
 35. The radio emission from quasars shows characteristics of
      a.  thermal emission from a hot body.
      b.  thermal emission from a cool body.
      c.  nonthermal synchrotron emission.
      d.  thermal synchrotron emission.
      
 36. The active region of a quasar whose light output varies considerably in a month is
      a.  a light-month in size.
      b.  a light-year in size.
      c.  a parsec in size.
      d.  the size of the Milky Way.
      e.  unknown in size.
      
 37. Which of the following properties is NOT present in both quasars and Seyfert galaxy nuclei?
      a.  complex absorption lines
      b.  blue color
      c.  10 % of them are radio sources
      d.  variability of light
      e.  similar emission spectra
      
 
 
 
 
 
 
 
 
 
 
Answer Key   for Practice Exam 3 on Galaxies
 
   1. d        
   2. d        
   3. b        
   4. c        
   5. c        
   6. a        
   7. c        
   8. b        
   9. c        
  10. d        
  11. b        
  12. b        
  13. b
  14. a      
  15. c       
  16. c        
  17. a        
  18. e        
  19. e        
  20. b        
  21. b        
  22. a        
  23. c        
  24. a        
  25. c        
  26. d       
  27. a        
  28. a        
  29. c        
  30. c        
  31. b        
  32. d       
  33. d        
  34. b       
  35. c        
  36. a    
  37. a
 
