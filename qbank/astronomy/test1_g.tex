
%% University of Northern Iowa
%%  Morgans Astronomy Exams
%%--------------------------------------------------

%% this section contains 15 problems

\element{morgans}{
\begin{question}{test1G-Q01}
    If you are at a latitude of 33 degrees south,
        and an object has a declination of 27 degrees north,
        how high above the horizon is it? 
    \begin{choices}
        \wrongchoice{33 degrees}
        \wrongchoice{60 degrees}
      \correctchoice{30 degrees}
        \wrongchoice{84 degrees}
        \wrongchoice{Not visible from your location}
    \end{choices}
\end{question}
}

\element{morgans}{
\begin{question}{test1G-Q02}
    You notice that on the autumnal equinox (September 21) the Sun is 62 degrees above the northern horizon. 
    What is the latitude of your location?
    \begin{choices}
        \wrongchoice{\ang{0}}
        \wrongchoice{\ang{10} south}
        \wrongchoice{\ang{42.5} north}
        \wrongchoice{\ang{62} south}
      \correctchoice{\ang{28} south}
    \end{choices}
\end{question}
}

\element{morgans}{
\begin{question}{test1G-Q03}
    Tonight is a Full Moon. 
    The Sun has a RA of 15h 25m. 
    What is the RA of the Moon? 
    \begin{choices}
        \wrongchoice{0h}
        \wrongchoice{15h 25m}
      \correctchoice{3h 25m}
        \wrongchoice{12h}
        \wrongchoice{21h 35m}
    \end{choices}
\end{question}
}

\element{morgans}{
\begin{question}{test1G-Q04}
    A comet has a declination of 52 degrees south. 
    %% NOTE: change location to NYC landmark!!
    How high above the horizon will it be when viewed from Cedar Falls? 
    \begin{choices}
        \wrongchoice{52 degrees}
        \wrongchoice{80.5 degrees}
        \wrongchoice{9.5 degrees}
        \wrongchoice{97.5 degrees}
      \correctchoice{Not visible from your location}
    \end{choices}
\end{question}
}

\element{morgans}{
\begin{question}{test1G-Q05}
    A comet is orbiting the Sun with a period of 1000 years. 
    What is its average distance from the Sun? 
    \begin{choices}
        \wrongchoice{1000 A.U.}
      \correctchoice{100 A.U.}
        \wrongchoice{10 A.U.}
        \wrongchoice{1,000,000 A.U.}
        \wrongchoice{Can't determine that with the information given}
    \end{choices}
\end{question}
}

\element{morgans}{
\begin{question}{test1G-Q06}
    Object $A$ has 5 times the mass as object $B$. 
    You hit both of them with the same force. 
    Which of the following is correct?
    \begin{choices}
      \correctchoice{Object $B$ will be accelerated 5 times more than object $A$}
        \wrongchoice{Object $A$ will be accelerated 5 times more than object $B$}
        \wrongchoice{Object $B$ will be accelerated 25 times more than object $A$}
        \wrongchoice{Object $A$ will be accelerated 25 times more than object $B$}
        \wrongchoice{Can't determine that with the information given}
    \end{choices}
\end{question}
}

\element{morgans}{
\begin{question}{test1G-Q07}
    You are on the planet Gumby,
        which has a mass that is 4 times the mass of the Earth,
        and a radius that is 4 times the radius of the Earth. 
    How does the force of gravity on the surface of Gumby compare to the force of gravity on the surface of the Earth?
    \begin{choices}
        \wrongchoice{It is the same as on the Earth}
      \correctchoice{It is 1/4 of the Earth's}
        \wrongchoice{It is 4 times the Earth's}
        \wrongchoice{It is 1/16 of the Earth's}
        \wrongchoice{It is 16 times the Earth's}
    \end{choices}
\end{question}
}

\element{morgans}{
\begin{question}{test1G-Q08}
    You leave the planet Gumby and head for the planet known as Pokey. 
    Pokey has a radius that is 1/8 that of the Earth,
        and a mass that is 2 times that of the Earth's. 
    How does the force of gravity on the surface of Pokey compare to the force of gravity on the surface of the Earth?
    \begin{choices}
        \wrongchoice{It is 4 times the Earth's}
        \wrongchoice{It is 8 times the Earth's}
        \wrongchoice{It is 16 times the Earth's}
        \wrongchoice{It is 64 times the Earth's}
      \correctchoice{It is 128 times the Earth's}
    \end{choices}
\end{question}
}

\element{morgans}{
\begin{question}{test1G-Q09}
    You are on a planet that has a mass equal to that of the Earth. 
    However, the surface gravity is 16 times greater than that of the Earth. 
    How does the radius of this planet compare to that of the Earth?
    \begin{choices}
        \wrongchoice{It is 16 times greater}
        \wrongchoice{It is 16 times smaller (1/16)}
        \wrongchoice{It is 4 times greater}
      \correctchoice{It is 4 times smaller (1/4)}
        \wrongchoice{You can't figure that out with the information provided}
    \end{choices}
\end{question}
}

\element{morgans}{
\begin{question}{test1G-Q10}
    Two photons were zooming across the Universe. 
    One of them is an ultraviolet photon,
        while the other is a microwave photon. 
    Which has the greater speed?
    \begin{choices}
        \wrongchoice{The ultraviolet since it has a higher energy}
        \wrongchoice{The microwave since it has a longer wavelength}
      \correctchoice{They have the same speed}
        \wrongchoice{You can't figure that out with the information provided}
    \end{choices}
\end{question}
}

\element{morgans}{
\begin{question}{test1G-Q11}
    %Of the two photons described in the previous problem,
    Two photons were zooming across the Universe. 
    One of them is an ultraviolet photon,
        while the other is a microwave photon. 
    Which has a larger frequency?
    \begin{choices}
      \correctchoice{The ultraviolet one}
        \wrongchoice{The microwave one}
        \wrongchoice{They both have the same frequency}
        \wrongchoice{You can't figure that out with the information provided}
    \end{choices}
\end{question}
}

\element{morgans}{
\begin{question}{test1G-Q12}
    You have the choice of purchasing a telescope with a 6 inch diameter opening,
        or a pair of binoculars which has 2 four inch diameter openings. 
    Which will gather more light? 
    (assume all openings are circular)
    \begin{choices}
      \correctchoice{The telescope}
        \wrongchoice{The binoculars}
        \wrongchoice{They will both gather the same amount of light}
        \wrongchoice{You can't figure that out with the information provided.}
    \end{choices}
\end{question}
}

\element{morgans}{
\begin{question}{test1G-Q13}
    You are at a latitude of 27 degrees north. 
    How high above the horizon is Polaris? 
    \begin{choices}
      \correctchoice{27 degrees}
        \wrongchoice{63 degrees}
        \wrongchoice{90 degrees}
        \wrongchoice{42.5 degrees}
        \wrongchoice{You can't see Polaris from your location}
    \end{choices}
\end{question}
}

\element{morgans}{
\begin{question}{test1G-Q14}
    If Mars is at opposition,
        what time will it rise?
    \begin{choices}
        \wrongchoice{Noon}
      \correctchoice{6 pm}
        \wrongchoice{Midnight}
        \wrongchoice{6 am}
    \end{choices}
\end{question}
}

\element{morgans}{
\begin{question}{test1G-Q15}
    If Venus is at greatest Eastern Elongation,
        it will be easily visible:
    \begin{choices}
        \wrongchoice{in the morning sky}
        \wrongchoice{around midnight}
        \wrongchoice{around noon}
      \correctchoice{in the evening sky}
    \end{choices}
\end{question}
}

\begin{comment}
    Fill In
    Place the most appropriate value in the blank. You may have to click on the blank to activate it before you start typing in your answer.
     
    You are at a latitude of 25 degrees south and you see an object located 65 degrees above the northern horizon. What is the objects declination? .

    You are at a latitude of 33 degrees south and Mars is at a declination of 17 degrees north. How high above the horizon is it when it is on your meridian? 

    When will the first quarter moon set? .

    If a planet is at an average distance of 25 A.U. from the Sun, what is its orbital period? (You'll need a calculator for this one) .

    If an object's mass is doubled, how much more force will you need to give it to have it accelerate at the same rate before its mass was doubled? .

    A photon with a wavelength twice the length of another photon will have the energy as the other. (Give how many times more or less)

    If you triple the diameter of the light gathering region of a telescope, you will gather times more light.

    A planet with an orbital period of 12 years will be on average A.U. from the Sun.

    You are the planet Astronomica, where the force of gravity is equal to the force of gravity on the Earth. The mass of Astronomica is 4 times the mass of the Earth, and the radius is times the radius of the Earth.

    Does your brain hurt now? 
\end{comment}

\endinput


