
%% University of Northern Iowa
%%  Morgans Astronomy Exams
%%--------------------------------------------------

%% this section contains XX problems

\element{morgans}{
\begin{question}{test1F-Q01}
    A useful device to show the apparent location of objects in the sky is known as the:
    \begin{choices}
        \wrongchoice{zodiac.}
      \correctchoice{celestial sphere.}
        \wrongchoice{Milky Way.}
        \wrongchoice{astronomical unit.}
    \end{choices}
\end{question}
}

\element{morgans}{
\begin{question}{test1F-Q02}
    You notice that on the summer solstice (June 21) the Sun is 33 1/2 degrees above the northern horizon. 
    What is the latitude of your location?
    \begin{choices}
        \wrongchoice{0}
        \wrongchoice{10 north}
        \wrongchoice{23 1/2 south}
        \wrongchoice{33 north}
      \correctchoice{33 south}
    \end{choices}
\end{question}
}

\element{morgans}{
\begin{question}{test1F-Q03}
    At the vernal equinox, an observer on the equator in Quito,
        Equador will observe the Sun to be \rule[-0.1pt]{4em}{0.1pt}
        degrees above the horizon at noon. 
    \begin{choices}
        \wrongchoice{0}
        \wrongchoice{30}
        \wrongchoice{45}
        \wrongchoice{60}
      \correctchoice{90}
    \end{choices}
\end{question}
}

\element{morgans}{
\begin{question}{test1F-Q04}
    The length of sidereal day is:
    \begin{choices}
        \wrongchoice{equal to the length of the Earth's orbital period.}
      \correctchoice{equal to the length of the Earth's rotation period.}
        \wrongchoice{the time interval from one vernal equinox to the next.}
        \wrongchoice{twelve lunar months.}
    \end{choices}
\end{question}
}

\element{morgans}{
\begin{question}{test1F-Q05}
    Ptolemy:
    \begin{choices}
        \wrongchoice{invented calculus and used it to predict the positions of the planets at any given time.}
      \correctchoice{wrote the Almagest, and other books summarizing the astronomical knowledge of earlier cultures.}
        \wrongchoice{was the first of the great Greek astronomers.}
        \wrongchoice{was the first to detect stellar parallax.}
    \end{choices}
\end{question}
}

\element{morgans}{
\begin{question}{test1F-Q06}
    Which of the following was \emph{not} an important contribution of Brahe's? 
    \begin{choices}
         \wrongchoice{improvements in instruments}
       \correctchoice{discovery of the rings around Saturn}
         \wrongchoice{very high precision observations}
         \wrongchoice{systematic observing over long periods of time}
    \end{choices}
\end{question}
}

\element{morgans}{
\begin{question}{test1F-Q07}
    If the Sun is at one focus of a planetary orbit,
        what is at the other focus?
    \begin{choices}
        \wrongchoice{the orbiting planet}
        \wrongchoice{a comet (e.g. like Halley's Comet)}
        \wrongchoice{a dead, burned-out (and therefore not readily visible) star}
      \correctchoice{nothing}
        \wrongchoice{no one knows}
    \end{choices}
\end{question}
}

\element{morgans}{
\begin{question}{test1F-Q08}
    Acceleration is defined as:
    \begin{choices}
      \correctchoice{the rate of change of velocity.}
        \wrongchoice{the rate of change of position.}
        \wrongchoice{the rate of change of distance.}
        \wrongchoice{how fast an object moves.}
        \wrongchoice{how fast an object changes position.}
    \end{choices}
\end{question}
}

\element{morgans}{
\begin{question}{test1F-Q09}
    If a person were on a planet having the mass of the Earth but double its size,
        the person's weight would be:
    \begin{choices}
        \wrongchoice{double.}
        \wrongchoice{cut in half.}
        \wrongchoice{increased by 4 times.}
      \correctchoice{decreased by 4 times.}
        \wrongchoice{unchanged.}
    \end{choices}
\end{question}
}

\element{morgans}{
\begin{question}{test1F-Q10}
    The surface gravity of a planet depends upon:
    \begin{choices}
        \wrongchoice{your mass and the planet's mass.}
        \wrongchoice{your mass and the planet's radius.}
      \correctchoice{the planet's mass and radius.}
        \wrongchoice{your mass and the planets orbital period.}
        \wrongchoice{the planet's mass and orbital period.}
    \end{choices}
\end{question}
}

\element{morgans}{
\begin{question}{test1F-Q11}
    The speed of light through a vacuum is:
    \begin{choices}
        \wrongchoice{greatest for red light due to its long wavelength.}
        \wrongchoice{greatest for blue light due to its higher photon energy.}
        \wrongchoice{greatest for only the shortest wavelength photons.}
      \correctchoice{the same for all types of light.}
    \end{choices}
\end{question}
}

\element{morgans}{
\begin{question}{test1F-Q12}
    If an electron moves from a lower energy level to the next higher energy level, then:
    \begin{choices}
      \correctchoice{the atom has become excited.}
        \wrongchoice{the atom has become ionized.}
        \wrongchoice{the atom's light will be bluer.}
        \wrongchoice{the atom's light will be redder.}
    \end{choices}
\end{question}
}

\element{morgans}{
\begin{question}{test1F-Q13}
    For which of the following types of telescopes is interferometry commonly used?
    \begin{choices}
        \wrongchoice{X-ray detector}
        \wrongchoice{ultraviolet telescope in orbit}
        \wrongchoice{visible telescope}
      \correctchoice{multiple radio telescope dishes}
    \end{choices}
\end{question}
}

\element{morgans}{
\begin{question}{test1F-Q14}
    Interferometry is:
    \begin{choices}
        \wrongchoice{The measure of the surface area of a telescope.}
        \wrongchoice{the way that telescope optics are tested---by the use of interference patterns.}
        \wrongchoice{the amount of interference caused by different atmospheric conditions.}
      \correctchoice{the method used to combine light from several different telescopes to make a stronger, clearer image.}
    \end{choices}
\end{question}
}

\element{morgans}{
\begin{question}{test1F-Q15}
    The Hubble Space Telescope has a diameter of nearly:
    \begin{choices}
        \wrongchoice{20 inches (0.5 meter).}
        \wrongchoice{50 inches (1.3 meters).}
        \wrongchoice{75 inches (1.9 meters).}
      \correctchoice{100 inches (2.5 meters).}
        \wrongchoice{200 inches (5 meters).}
    \end{choices}
\end{question}
}

\begin{commemnt}
    Fill In
    Place the most appropriate word or words in the blank. You may have to click on the blank to activate it before you start typing in your answer.
     
    The average distance of the Earth from the Sun is called a(n) .

    The apparent path of the Sun across the celestial sphere is called .

    When the Moon is between full and first or third quarter phase, it is said to be .

    The shape of a planetary orbit is called a(n) .

    Detailed, accurate observations of the relative positions of planets were made by .

    A blue shift in a spectra is due to motion (away from or towards) the light source .

    The measure of the number of peaks of a light wave that pass by each second is known as the .

    A(n) spectrum is produced by a high temperature gas having a low density.

    The type of telescope which uses a mirror as the main optical element is a(n) .

    The IUE satellite studies objects that emit light.
\end{commemnt}

\endinput



