
%% University of Northern Iowa
%%  Morgans Astronomy Exams
%%--------------------------------------------------

%% this section contains 15 problems

\element{morgans}{
\begin{question}{test3B-Q01}
    The diameter of the disk of our galaxy is closest to \rule[-0.1pt]{4em}{0.1pt} parsec(s),
        or \rule[-0.1pt]{4em}{0.1pt} light years.
    \begin{choices}
        \wrongchoice{1; 3.26}
        \wrongchoice{100; 326}
        \wrongchoice{1,000; 3260}
        \wrongchoice{18,000; 53,000}
      \correctchoice{30,000; 100,000}
    \end{choices}
\end{question}
}

\element{morgans}{
\begin{question}{test3B-Q02}
    Harlow Shapley:
    \begin{choices}
      \correctchoice{used Cepheids to determine the Sun's location in the Galaxy.}
        \wrongchoice{discovered the period - luminosity relations in Cepheids and RR Lyrae stars.}
        \wrongchoice{used globular clusters to determine the direction and location of the center of the Galaxy.}
        \wrongchoice{used radio telescope observations of hydrogen to map out the spiral structure of the Galaxy.}
    \end{choices}
\end{question}
}

\element{morgans}{
\begin{question}{test3B-Q03}
    21-cm radiation is important to astronomy because:
    \begin{choices}
        \wrongchoice{the interstellar medium is opaque (foggy) at 21-cm.}
      \correctchoice{it corresponds to interstellar hydrogen (H I).}
        \wrongchoice{stars emit strongly at 21-cm, thus allowing them to be seen at large distances.}
        \wrongchoice{nebulae emit strongly at 21-cm.}
        \wrongchoice{black holes emit large amounts of radio radiation at 21-cm.}
    \end{choices}
\end{question}
}

\element{morgans}{
\begin{question}{test3B-Q04}
    The Galaxy has:
    \begin{choices}
        \wrongchoice{one spiral arm.}
        \wrongchoice{two spiral arms.}
        \wrongchoice{four spiral arms.}
      \correctchoice{an unknown number of spiral arms.}
    \end{choices}
\end{question}
}

\element{morgans}{
\begin{question}{test3B-Q05}
    In comparison with the Sun,
        stars located in the galactic halo are expected to have:
    \begin{choices}
        \wrongchoice{low metal abundances.}
        \wrongchoice{metal abundances similar to those in the Sun.}
        \wrongchoice{metal abundances higher than those in the Sun.}
        \wrongchoice{none of the above; we have no idea what the metal abundance should be.}
    \end{choices}
\end{question}
}

\element{morgans}{
\begin{question}{test3B-Q06}
    The disk-like shape of our galaxy provides evidence that the early Galaxy:
    \begin{choices}
        \wrongchoice{was a strong X-ray emitter.}
        \wrongchoice{had a high temperature.}
        \wrongchoice{had a strong magnetic field.}
      \correctchoice{rotated.}
        \wrongchoice{had a high abundance of heavy elements.}
    \end{choices}
\end{question}
}

\element{morgans}{
\begin{question}{test3B-Q07}
    The resolution to the question of the nature of the ``nebulae'' came from the discovery of \rule[-0.1pt]{4em}{0.1pt} within the ``Andromeda nebulae''. 
    \begin{choices}
        \wrongchoice{solar-type stars}
        \wrongchoice{T-Tauri stars}
        \wrongchoice{RR Lyrae stars}
      \correctchoice{Cepheid variables}
        \wrongchoice{planetary nebulae}
    \end{choices}
\end{question}
}

\element{morgans}{
\begin{question}{test3B-Q08}
    Which of the following is \emph{not} an example of a standard candle?
    \begin{choices}
        \wrongchoice{Cepheids}
        \wrongchoice{supergiants}
        \wrongchoice{the brightest galaxy in a cluster}
        \wrongchoice{supernovae}
      \correctchoice{pulsars}
    \end{choices}
\end{question}
}

\element{morgans}{
\begin{question}{test3B-Q09}
    A galaxy whose overall color is reddish would probably be:
    \begin{choices}
        \wrongchoice{spiral.}
      \correctchoice{elliptical.}
        \wrongchoice{irregular.}
        \wrongchoice{indeterminate; color has nothing to do with galaxy type.}
    \end{choices}
\end{question}
}

\element{morgans}{
\begin{question}{test3B-Q10}
    The Local Group has a diameter of roughly:
    \begin{choices}
        \wrongchoice{100 parsecs.}
        \wrongchoice{1 kiloparsec.}
        \wrongchoice{100 kiloparsecs.}
      \correctchoice{1 megaparsec}
    \end{choices}
\end{question}
}

\element{morgans}{
\begin{question}{test3B-Q11}
    When a large galaxy pulls off material from a smaller galaxy,
        and basically ``eats up'' the smaller galaxy,
        this is known as:
    \begin{choices}
        \wrongchoice{active galaxies.}
        \wrongchoice{supernova.}
        \wrongchoice{galaxy collisions.}
      \correctchoice{galactic cannibalism.}
    \end{choices}
\end{question}
}

\element{morgans}{
\begin{question}{test3B-Q12}
    If galaxy $B$ is 9 times as distant as galaxy $A$,
        the velocity of $B$ will be  \rule[-0.1pt]{4em}{0.1pt} times as great as that of galaxy $A$ according to Hubble's law.
    \begin{choices}
        \wrongchoice{3}
      \correctchoice{9}
        \wrongchoice{1/3}
        \wrongchoice{1/9}
    \end{choices}
\end{question}
}

\element{morgans}{
\begin{question}{test3B-Q13}
    At the time of the big bang the universe was:
    \begin{choices}
        \wrongchoice{cool with a low density.}
        \wrongchoice{cool with a high density.}
      \correctchoice{hot with a high density.}
        \wrongchoice{hot with a low density.}
    \end{choices}
\end{question}
}

\element{morgans}{
\begin{question}{test3B-Q14}
    The value of the Hubble constant is equal to the slope of the line relating:
    \begin{choices}
      \correctchoice{velocity and distance.}
        \wrongchoice{distance and mass.}
        \wrongchoice{mass and luminosity.}
        \wrongchoice{velocity and mass.}
        \wrongchoice{rotation speed and distance.}
    \end{choices}
\end{question}
}

\element{morgans}{
\begin{question}{test3B-Q15}
    A typical radio lobe or jet is the size of the:
    \begin{choices}
        \wrongchoice{Sun.}
        \wrongchoice{solar system.}
        \wrongchoice{distance to the nearest star.}
        \wrongchoice{size of the Milky Way.}
      \correctchoice{distance from the Milky Way to M31.}
    \end{choices}
\end{question}
}

\element{morgans}{
\begin{question}{test3B-Q16}
    The most conspicuous observed property of quasars is their:
    \begin{choices}
        \wrongchoice{low apparent magnitude (high apparent brightness).}
      \correctchoice{high redshift.}
        \wrongchoice{high blueshift.}
        \wrongchoice{dramatic appearance on photographs.}
        \wrongchoice{strong spiral structure.}
    \end{choices}
\end{question}
}

\element{morgans}{
\begin{question}{test3B-Q17}
    A quasar with a redshift of 3.5 is seen near a large spiral galaxy with a redshift of 0.7. What does this imply about the nature of quasars?
    \begin{choices}
        \wrongchoice{the quasar is not at a cosmological distance, since these objects are both at the same distance}
        \wrongchoice{the quasar is at a cosmological distance, but so is the large spiral}
      \correctchoice{the quasar is only located near the same line of sight as the spiral galaxy and is really much closer than the spiral}
        \wrongchoice{the quasar is only located near the same line of sight as the spiral galaxy and is really much further away than the spiral}
    \end{choices}
\end{question}
}

\element{morgans}{
\begin{question}{test3B-Q18}
    In models of a non-accelerating universe,
        where the density of the universe is greater than the critical density the universe will:
    \begin{choices}
        \wrongchoice{continue to expand forever.}
        \wrongchoice{eventually stop expanding.}
      \correctchoice{eventually stop expanding and begin contraction.}
        \wrongchoice{continue to contract forever.}
        \wrongchoice{eventually stop contracting and begin expanding.}
    \end{choices}
\end{question}
}

\element{morgans}{
\begin{question}{test3B-Q19}
    What percentage of the mass of the universe was formed into helium during the big bang?
    \begin{choices}
        \wrongchoice{\SI{1}{\percent}}
        \wrongchoice{\SI{5}{\percent}}
        \wrongchoice{\SI{15}{\percent}}
      \correctchoice{\SI{25}{\percent}}
        \wrongchoice{\SI{50}{\percent}}
    \end{choices}
\end{question}
}

\element{morgans}{
\begin{question}{test3B-Q20}
    Grand Unified Theories connect which three forces?
    \begin{choices}
        \wrongchoice{gravitation, electromagnetism, weak nuclear}
      \correctchoice{electromagnetism, weak nuclear, strong nuclear}
        \wrongchoice{weak nuclear, strong nuclear, gravitation}
        \wrongchoice{strong nuclear, gravitation, electromagnetism}
    \end{choices}
\end{question}
}

    Fill In
    Place the most appropriate word or words in the blank. You may have to click on the blank to activate it before you start typing in your answer.
     
    The diffuse band of light stretching across the sky is called the .

    The most prominent objects in the halo of the galaxy are the , spherical groups of stars.

    The Sun belongs to Population .

    In the famous 1920 astronomical debate, the primary spokesman for the view that the "nebulae'' were outside our galaxy was .

    A cluster of galaxy clusters is called a(n) .

    The smearing of the image of a distant galaxy by a nearer cluster or large galaxy is known as .

    The age of the universe is estimated by astronomers to be years.

    An emission line active galaxy having an extreme radial velocity is known as a(n) .

    Quasar absorption lines are best explained as produced by .

    The word that describes the uniform distribution of matter throughout the universe is .

