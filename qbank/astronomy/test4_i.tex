
%% university of northern iowa
%%  morgans astronomy exams
%%--------------------------------------------------

%% this section contains 20 problems

\element{morgans}{
\begin{question}{test4A-q01}
    What did the NEAR spacecraft do at the end of its mission?
    \begin{choices}
        \wrongchoice{crashed into the Moon.}
        \wrongchoice{crashed into Jupiter.}
        \wrongchoice{crashed into an asteroid.}
      \correctchoice{landed on the surface of an asteroid.}
    \end{choices}
\end{question}
}

\element{morgans}{
\begin{question}{test4A-q02}
    Which of the following processes is likely to have been the most important in the formation of Earth's atmosphere?
    \begin{choices}
        \wrongchoice{vaporization of colliding comets}
        \wrongchoice{gravitational capture of surrounding gasses}
        \wrongchoice{capture of surrounding charged particles by the magnetic field}
        \wrongchoice{respiration of living organisms}
      \correctchoice{outgassing from the interior}
    \end{choices}
\end{question}
}

\element{morgans}{
\begin{question}{test4A-q03}
    Aurora are associated with:
    \begin{choices}
        \wrongchoice{volcanism.}
        \wrongchoice{earthquakes.}
      \correctchoice{magnetic fields.}
        \wrongchoice{weather systems.}
    \end{choices}
\end{question}
}

\element{morgans}{
\begin{question}{test4A-q04}
    Which of the following processes of erosion occurs today on the lunar surface?
    \begin{choices}
        \wrongchoice{plate tectonics}
        \wrongchoice{lava flows}
        \wrongchoice{wind}
      \correctchoice{meteoritic impacts}
    \end{choices}
\end{question}
}

\element{morgans}{
\begin{question}{test4A-q05}
    Mercury has a high surface temperature because of its:
    \begin{choices}
        \wrongchoice{strong greenhouse effect.}
      \correctchoice{proximity to the Sun.}
        \wrongchoice{eccentric orbit.}
        \wrongchoice{spin-orbit coupling.}
        \wrongchoice{rapid rotation.}
    \end{choices}
\end{question}
}

\element{morgans}{
\begin{question}{test4A-q06}
    Which of the following is NOT one of the surface regions on Venus?
    \begin{choices}
        \wrongchoice{rolling plains}
        \wrongchoice{lowlands}
        \wrongchoice{highlands (continents)}
      \correctchoice{scarps}
    \end{choices}
\end{question}
}

\element{morgans}{
\begin{question}{test4A-q07}
    If indeed the Earth has the same total amount of carbon dioxide as Venus,
        why is such a small amount found in the atmosphere?
    \begin{choices}
        \wrongchoice{the statement is false---Venus and the Earth do NOT have the same total amount of carbon dioxide}
        \wrongchoice{the statement is false---the amount of carbon dioxide in the atmosphere of the Earth is the same as for Venus' atmosphere}
        \wrongchoice{the high temperature of Venus excites the carbon dioxide more than does the lower temperature on Earth}
        \wrongchoice{the plants on Earth have decreased the amount of carbon dioxide to where it is today}
      \correctchoice{the Earth's lower temperature allows the carbon dioxide to be held in the oceans and rocks}
    \end{choices}
\end{question}
}

\element{morgans}{
\begin{question}{test4A-q08}
    The primary purpose of the Magellan spacecraft is to study Venus':
    \begin{choices}
        \wrongchoice{atmosphere.}
      \correctchoice{surface.}
        \wrongchoice{magnetic field.}
        \wrongchoice{chemical composition.}
        \wrongchoice{orbital characteristics.}
    \end{choices}
\end{question}
}

\element{morgans}{
\begin{question}{test4A-q09}
    A major reason for the size of volcanos on Mars is:
    \begin{choices}
      \correctchoice{the probable lack of continental drift on Mars.}
        \wrongchoice{the rapid motion of continental plates on Mars.}
        \wrongchoice{the large number of strong eruptions present on Mars.}
        \wrongchoice{flowing water (in the past) eroded them away.}
    \end{choices}
\end{question}
}

\element{morgans}{
\begin{question}{test4A-q10}
    Which has the oldest surface?
    \begin{choices}
      \correctchoice{Mercury}
        \wrongchoice{Venus}
        \wrongchoice{Earth}
        \wrongchoice{Mars}
    \end{choices}
\end{question}
}

\element{morgans}{
\begin{question}{test4A-q11}
    Neptune was discovered by means of:
    \begin{choices}
      \correctchoice{theoretical computations of its position combined with visual observing.}
        \wrongchoice{photography of the entire sky and blind searching.}
        \wrongchoice{observations of the sky from spacecraft.}
        \wrongchoice{lunar occultations.}
        \wrongchoice{interferometry.}
    \end{choices}
\end{question}
}

\element{morgans}{
\begin{question}{test4A-q12}
    Saturn's density is:
    \begin{choices}
      \correctchoice{less than that of Jupiter.}
        \wrongchoice{about that of Jupiter.}
        \wrongchoice{similar to the Earth's.}
        \wrongchoice{greater than that of the Earth.}
    \end{choices}
\end{question}
}

\element{morgans}{
\begin{question}{test4A-q13}
    The Galilean satellite showing the largest number of craters is:
    \begin{choices}
        \wrongchoice{Io.}
        \wrongchoice{Europa.}
      \correctchoice{Callisto.}
        \wrongchoice{Ganymede.}
    \end{choices}
\end{question}
}

\element{morgans}{
\begin{question}{test4A-q14}
    Which planet is most similar to Neptune?
    \begin{choices}
        \wrongchoice{Mercury}
        \wrongchoice{Jupiter}
        \wrongchoice{Saturn}
      \correctchoice{Uranus}
        \wrongchoice{Pluto}
    \end{choices}
\end{question}
}

\element{morgans}{
\begin{question}{test4A-q15}
    Neptune emits \rule[-0.1pt]{4em}{0.1pt} energy than Uranus.
    \begin{choices}
        \wrongchoice{less}
        \wrongchoice{the same}
      \correctchoice{more}
    \end{choices}
\end{question}
}

\element{morgans}{
\begin{question}{test4A-q16}
    Neptune's ring system is most similar to the ring system of:
    \begin{choices}
        \wrongchoice{Jupiter.}
        \wrongchoice{Saturn.}
      \correctchoice{Uranus.}
        \wrongchoice{the Earth.}
    \end{choices}
\end{question}
}

\element{morgans}{
\begin{question}{test4A-q17}
    Most asteroids are of the:
    \begin{choices}
        \wrongchoice{S---class (silicon).}
        \wrongchoice{M---class (metallic).}
      \correctchoice{C---class (carbonaceous).}
        \wrongchoice{U---class (unknown).}
    \end{choices}
\end{question}
}

\element{morgans}{
\begin{question}{test4A-q18}
    A comet's nucleus has a diameter of roughly:
    \begin{choices}
      \correctchoice{10 km.}
        \wrongchoice{100 km.}
        \wrongchoice{1,000 km.}
        \wrongchoice{10,000 km.}
        \wrongchoice{100,000 km.}
    \end{choices}
\end{question}
}

\element{morgans}{
\begin{question}{test4A-q19}
    Most meteorites come from:
    \begin{choices}
      \correctchoice{asteroids.}
        \wrongchoice{comets.}
        \wrongchoice{the Moon.}
        \wrongchoice{random rocks spread throughout the solar system.}
        \wrongchoice{an unknown source.}
    \end{choices}
\end{question}
}

\element{morgans}{
\begin{question}{test4A-q20}
    Which of the following is responsible for maintaining thin ring systems?
    \begin{choices}
      \correctchoice{shepherd satellites}
        \wrongchoice{magnetic fields}
        \wrongchoice{impacts with moons}
        \wrongchoice{solar winds}
    \end{choices}
\end{question}
}

\begin{comment}
    Fill In
    Place the most appropriate word or words in the blank. You may have to click on the blank to activate it before you start typing in your answer.
    are the two elements which primarily make up the cores of terrestrial planets. (list two)

    is the process by which a planet obtains its atmosphere of volatile elements from its interior.

    is the only planet in the inner solar system without any evidence for ice at its poles.

    is the age of the Earth in years.

    Mercury's appearance is different from that of Moon because of its extensive system of .

    The permanent layer of ice just below the surface of certain regions on the Earth and probably Mars is called .

    Uranus has known satellites.

    Pluto is most similar to , a satellite of Neptune.

    The extended head of a comet formed as the comet approaches the Sun is called the .

    The group of planets with the largest sizes in the solar system are called the
\end{comment}


\endinput


