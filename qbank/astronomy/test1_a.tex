
%% university of northern iowa
%%  morgans astronomy exams
%%--------------------------------------------------

%% this section contains 15 problems

\element{morgans}{
\begin{question}{test1a-q01}
    planets:
    \begin{choices}
        \wrongchoice{move rapidly across the sky relative to the stars.}
        \wrongchoice{are stationary relative to the stars.}
        \wrongchoice{all move at the same rate relative to the stars.}
      \correctchoice{move slowly relative to the stars.}
        \wrongchoice{can appear anywhere in the sky.}
    \end{choices}
\end{question}
}

\element{morgans}{
\begin{question}{test1A-Q02}
    Which of the following statements about planets is \emph{false}? 
    \begin{choices}
        %% NOTE: A,B,C,E, No D
      \correctchoice{none are visible to observers on the Earth.}
        \wrongchoice{they move relative to the stars.}
        \wrongchoice{they are found along the zodiac.}
        \wrongchoice{they do not twinkle as stars do.}
    \end{choices}
\end{question}
}

\element{morgans}{
\begin{question}{test1A-Q03}
    Precession is:
    \begin{choices}
        \wrongchoice{the accuracy with which numbers are given in astronomy.}
      \correctchoice{the slow motion of the Earth's rotation axis on the celestial sphere.}
        \wrongchoice{the apparent backward motion of planets on the celestial sphere.}
        \wrongchoice{the daily eastward motion of the Sun around the celestial sphere.}
    \end{choices}
\end{question}
}

\element{morgans}{
\begin{question}{test1A-Q04}
    If sunset is at 6:00 p.m., and if the Moon sets at midnight, what phase is it? 
    \begin{choices}
        \wrongchoice{new}
      \correctchoice{first quarter}
        \wrongchoice{full}
        \wrongchoice{third quarter}
    \end{choices}
\end{question}
}

\element{morgans}{
\begin{question}{test1A-Q05}
    The apparent lunar motion is:
    \begin{choices}
        \wrongchoice{south to north.}
        \wrongchoice{along the celestial equator.}
        \wrongchoice{exactly along the ecliptic.}
        \wrongchoice{nearly along the celestial equator.}
      \correctchoice{nearly along the ecliptic.}
    \end{choices}
\end{question}
}

\element{morgans}{
\begin{question}{test1A-Q06}
    The Almagest was written by:
    \begin{choices}
        \wrongchoice{Plato}
        \wrongchoice{Aristotle}
        \wrongchoice{Hipparchus}
      \correctchoice{Ptolomy}
        \wrongchoice{Pythagorus}
    \end{choices}
\end{question}
}

\element{morgans}{
\begin{question}{test1A-Q07}
    Whose astronomical observations did Kepler use to assist in his development of the laws of planetary motion?
    \begin{choices}
        \wrongchoice{Hipparchus}
        \wrongchoice{Ptolomy}
        \wrongchoice{Aristotle}
      \correctchoice{Tycho Brahe}
        \wrongchoice{Galileo}
    \end{choices}
\end{question}
}

\element{morgans}{
\begin{question}{test1A-Q08}
    Which of the following did Galileo \emph{not} observe?
    \begin{choices}
        \wrongchoice{sunspots}
      \correctchoice{the moons of Mars}
        \wrongchoice{the phases of Venus}
        \wrongchoice{the craters on the Moon}
    \end{choices}
\end{question}
}

\element{morgans}{
\begin{question}{test1A-Q09}
    Which of the following would \emph{not} occur if the Earth's mass were doubled? 
    (NOTE: the radius remains the same)
    \begin{choices}
      \correctchoice{your mass would double}
        \wrongchoice{your weight would double}
        \wrongchoice{the surface gravity would double}
        \wrongchoice{the escape velocity would increase}
    \end{choices}
\end{question}
}

\element{morgans}{
\begin{question}{test1A-Q10}
    If in a violent moment you kick a wall, your foot will hurt. 
    This is best explained by:
    \begin{choices}
        \wrongchoice{Newton's first law of motion.}
        \wrongchoice{Newton's second law of motion.}
      \correctchoice{Newton's third law of motion.}
        \wrongchoice{the universal law of gravity.}
    \end{choices}
\end{question}
}

\element{morgans}{
\begin{question}{test1A-Q11}
    You land your spaceship on a planet called Kepleria. 
    You do notice that the radius of Kepleria is half that of the Earth. 
    You also notice that the gravitational acceleration on the surface is equivalent to that of the Earth. 
    What does this information tell you?
    \begin{choices}
        \wrongchoice{the mass of Kepleria is 2 times greater than the Earth's}
        \wrongchoice{the mass of Kepleria is 4 times greater than the Earth's}
        \wrongchoice{the mass of Kepleria is 2 times smaller than the Earth's}
      \correctchoice{the mass of Kepleria is 4 times smaller than the Earth's}
    \end{choices}
\end{question}
}

\element{morgans}{
\begin{question}{test1A-Q12}
    Radio waves have:
    \begin{choices}
        \wrongchoice{high energy and long wavelength.}
      \correctchoice{low energy and long wavelength.}
        \wrongchoice{low energy and short wavelength.}
        \wrongchoice{high energy and short wavelength.}
    \end{choices}
\end{question}
}

\element{morgans}{
\begin{question}{test1A-Q13}
    The degree of ionization of an atom (i.e. the number of electrons lost) depends on:
    \begin{choices}
        \wrongchoice{the level of the ground state.}
      \correctchoice{the temperature of the gas.}
        \wrongchoice{the energy difference between the ground state and the first excited state.}
        \wrongchoice{the distance to the observer.}
    \end{choices}
\end{question}
}

\element{morgans}{
\begin{question}{test1A-Q14}
    If telescope $A$ has a mirror twice the diameter of telescope $B$,
        the light gathering power of telescope $A$ will be \rule[-0.1pt]{4em}{0.1pt} times that of $B$?
    \begin{multicols}{4}
    \begin{choices}
        \wrongchoice{$2$}
      \correctchoice{$4$}
        \wrongchoice{$8$}
        \wrongchoice{$16$}
    \end{choices}
    \end{multicols}
\end{question}
}

\element{morgans}{
\begin{question}{test1A-Q15}
    Which of the following is \emph{not} a characteristic of a CCD (charge coupled device)?
    \begin{choices}
        \wrongchoice{can detect variations in brightness}
        \wrongchoice{can record up to \SI{80}{\percent} or more of the photons that strike it}
        \wrongchoice{can be used over several wavelength ranges}
      \correctchoice{can be used to view a large field with a single exposure}
    \end{choices}
\end{question}
}

\begin{comment}
    Fill In
    Place the most appropriate word or words in the blank. You may have to click on the blank to activate it before you start typing in your answer.
    The east-west path of the Moon on the celestial sphere is close to the.

    The projection of the Earth's equator onto the celestial sphere is called the .

    The occasion when the Sun crosses the celestial equator is called .

    The planetary configuration in which some planet, the Sun and the Earth are in order and in a straight line is called .

    Relative to a planet's orbit, the Sun is located at the orbit's .

    In a period of one month, the amount of area swept out by the line joining a planet and the Sun when the planet is near to the Sun will be amount of area swept out when the planet is far from the Sun.

    Any change in the motion of an object is known as .

    A(n) is a particle of light having wave properties but which also acts as a discrete unit of energy.

    A(n) spectrum is produced when a low density, low temperature gas is in front of a higher temperature source of radiation.

    Using telescopes in pairs or groups to increase the resolving power is known as .
\end{comment}

\endinput


