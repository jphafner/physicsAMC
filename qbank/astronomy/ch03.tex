
%% Astronomy:
%%--------------------------------------------------

%% Chapter 03: Learning Objectives and Study Questions for
%%------------------------------------------------------------


% 1. Relate the wave and particle (photon) models for light in terms of their key quantities and the relationships between them (c = fλ , E = hf or E = hc/λ). [Remember, c = 3.00 × 10 8 m/s, a quantity that you should know.]
% 2. Arrange the following types of "light" in their correct relative positions in the E-M spectrum: visible; infrared; radio; ultraviolet; X-rays; and gamma rays.
% 3. Predict whether telescopes designed to gather a specific type of "light" should be placed on Earth or in orbit on the basis of "atmospheric windows".
% 4. Calculate how a telescope's light gathering ability, resolution, and magnification will change in response to changes in the diameter and focal length of its objective lens or mirror, and the focal length of its eyepiece.
% 5. Describe the basic difference between reflectors and refractors, and explain why nearly all large telescopes are reflectors.
% 6. Explain how adaptive optics is used to improve the "seeing" of many ground-based optical telescopes and how interferometry is used to improve the resolution of radio telescopes.


\element{astr}{
\begin{question}{ch03-q01}
    Different colors of light correspond to electromagnetic waves with different \rule[-0.1pt]{4em}{0.1pt}.
    \begin{choices}
        \wrongchoice{speeds}
        \wrongchoice{amplitudes}
        \wrongchoice{intensities}
        \wrongchoice{wavelengths}
        \wrongchoice{phases}
    \end{choices}
\end{question}
}

\element{astr}{
\begin{question}{ch03-q02}
    Which of the following colors of visible light is emitted by electrons vibrating at the highest frequency?
    \begin{choices}
        \wrongchoice{red}
        \wrongchoice{orange}
        \wrongchoice{yellow}
        \wrongchoice{green}
        \wrongchoice{blue}
    \end{choices}
\end{question}
}

3. About how many seconds does it take for light to travel 1.50×10 11 m through space from
the Sun to Earth? Remember, d = v×t.
A. 8
B. 50
C. 500
D. 4500
E. indeterminate, cannot tell
4. The intensity or brightness of light corresponds to the ____ of photons striking a surface
each second.
A. speed
B. energy
C. charge

D. momentum
E. number
5. Which of the following parts of the electromagnetic spectrum has photons that carry the
largest energies?
A. X-ray
B. ultraviolet
C. visible
D. infrared
E. radio
6. Which of the following types of telescopes would “see” well whether it was built on Earth’s
surface or in space?
A. γ-ray
B. X-ray
C. ultraviolet
D. infrared
E. radio
7. Which of the following is not an important function of a telescope for creating an image?
A. increasing brightness
B. enhancing colors
C. improving resolution
D. magnifying
E. all are important
8. Because a telescope’s ability to brighten an image is proportional to the area of its light-
gathering surface, increasing the diameter of its lens or mirror three times increases image
brightness by a factor of _____ times.
A. 3
B. 6
C. 9
D. 12
E. 18
9. If a telescope can resolve objects one arc second apart, a second telescope that has twice
the diameter of the first can resolve objects that are _____ arc seconds apart.
A. 4
B. 2
C. 0.5
D. 0.25
E. 0.0625
10. Refracting telescopes focus light using _____.
A. lenses
B. magnetic fields
C. mirrors


D. metal cylinders
E. metal dishes
11. Because of the difficulties inherent in making and supporting large lenses, virtually all large
optical telescopes used today are _____.
A. refractors
B. reflectors
C. collimators
D. space-based
E. interferometers
12. In order to eliminate twinkling and improve “seeing”, telescopes use all of the following
techniques except _____.
A. adaptive optics
B. high altitude locations
C. active optics
D. Schmidt correctors
E. orbital locations
13. Because of the long wavelengths of radio waves, individual radio telescopes tend to have
much poorer _____ than comparable optical telescopes.
A. brightness
B. penetration of interstellar gas and dust
C. magnification
D. resolution
E. dispersion
14. Data from separate radio telescopes can be combined using computers to create, in effect,
a single large telescope in a technique called _____.
A. refractometry
B. active optics
C. Interferometry
D. networking
E. collimation
15. Because of their high energies, X-ray photons must be focused using _____.
A. fluorite lenses
B. dense mirrors
C. magnetic fields
D. diffraction gratings
E. metal cylinders


\endinput



