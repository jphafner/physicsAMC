
%% University of Northern Iowa
%%  Morgans Astronomy Exams
%%--------------------------------------------------

%% this section contains XX problems

\element{morgans}{
\begin{question}{test1D-Q01}
    Measurements made on the celestial sphere are made in units of:
    \begin{choices}
        \wrongchoice{miles.}
        \wrongchoice{kilometers.}
      \correctchoice{degrees.}
        \wrongchoice{light years.}
    \end{choices}
\end{question}
}

\element{morgans}{
\begin{question}{test1D-Q02}
    If Vega rises tonight at 2:00 a.m.,
        at what time should you be looking for it two months from tonight? 
    \begin{choices}
      \correctchoice{10:00 p.m.}
        \wrongchoice{11:00 p.m.}
        \wrongchoice{midnight}
        \wrongchoice{2:00 a.m.}
        \wrongchoice{4:00 a.m.}
    \end{choices}
\end{question}
}

\element{morgans}{
\begin{question}{test1D-Q03}
    How many degrees above the horizon would the Sun be on the summer solstice (June 21) for an observer at the Earth's north pole?
    \begin{choices}
        \wrongchoice{0}
        \wrongchoice{90}
      \correctchoice{23 1/2}
        \wrongchoice{66 1/2}
        \wrongchoice{The Sun is not visible on that day.}
    \end{choices}
\end{question}
}

\element{morgans}{
\begin{question}{test1D-Q04}
    Eclipses do not occur each month because:
    \begin{choices}
        \wrongchoice{the Moon is always in the ecliptic.}
        \wrongchoice{the Moon is never in the ecliptic.}
        \wrongchoice{the Earth's axis is tilted to the ecliptic.}
        \wrongchoice{the Moon's orbit is in the ecliptic.}
      \correctchoice{the Moon's orbit is not always in the ecliptic.}
    \end{choices}
\end{question}
}

\element{morgans}{
\begin{question}{test1D-Q05}
    When an inferior planet is located on the far side of the Sun, it is at:
    \begin{choices}
        \wrongchoice{greatest elongation.}
        \wrongchoice{inferior conjunction.}
      \correctchoice{superior conjunction.}
        \wrongchoice{quadrature.}
    \end{choices}
\end{question}
}

\element{morgans}{
\begin{question}{test1D-Q06}
    Tycho Brahe's principal contribution to astronomy was:
    \begin{choices}
        \wrongchoice{his noble blood.}
        \wrongchoice{his suggested model for the solar system (which had a fixed Earth with the Sun revolving about it but the rest of the planets revolving about the Sun).}
      \correctchoice{the accuracy of his observations and the completeness of his records.}
        \wrongchoice{his choice of Galileo as an assistant.}
    \end{choices}
\end{question}
}

\element{morgans}{
\begin{question}{test1D-Q07}
    In simple language, Kepler's second law means that:
    \begin{choices}
      \correctchoice{a planet moves more rapidly when near the Sun than when farther away.}
        \wrongchoice{planets close to the Sun have shorter periods than those farther away.}
        \wrongchoice{the Sun is at the center of planetary orbits.}
        \wrongchoice{slowly moving planets are close to the Sun.}
    \end{choices}
\end{question}
}

\element{morgans}{
\begin{question}{test1D-Q08}
    Which of the following has the greatest mass?
    \begin{choices}[o]
        \wrongchoice{100 lbs of goose feathers}
        \wrongchoice{100 lbs of lead}
        \wrongchoice{a 100 lb person}
      \correctchoice{they all have the same mass}
    \end{choices}
\end{question}
}

\element{morgans}{
\begin{question}{test1D-Q09}
    If there were another Earth-mass planet in our solar system at 1 a.u. from the Sun,
        but with double the Earth's radius,
    \begin{choices}
        \wrongchoice{the gravitational attraction to the Sun would double.}
        \wrongchoice{the gravitational attraction to the Sun would decrease by two times.}
        \wrongchoice{the gravitational attraction to the Sun would increase by four times.}
        \wrongchoice{the gravitational attraction to the Sun would decrease by four times.}
      \correctchoice{there would be no change in the gravitational attraction between the planet and the Sun.}
    \end{choices}
\end{question}
}

\element{morgans}{
\begin{question}{test1D-Q10}
    Planets Romulus and Remus orbit a star which is identical to the Sun. 
    Romulus is located 5 A. U. from the star,
        while Remus is located 10 A. U. from the star. 
    Which of the following statements is true?
    \begin{choices}
      \correctchoice{Romulus' orbital period will be less than Remus'}
        \wrongchoice{Romulus' orbital period will be greater than Remus'}
        \wrongchoice{Romulus' orbital period will equal Remus'}
        \wrongchoice{it is not possible to determine the relationship between these planet's orbits and their periods, since Kepler's laws only work in our solar system}
    \end{choices}
\end{question}
}

\element{morgans}{
\begin{question}{test1D-Q11}
    Which of the following lists of types of light is in the correct order from lowest to highest energy? 
    (NOTE: Not all types of light are included in each list.)
    \begin{choices}
        \wrongchoice{gamma ray, blue, green, red, microwave, radio}
        \wrongchoice{X ray, ultraviolet, infrared, yellow, red, radio}
      \correctchoice{radio, microwave, blue, ultraviolet, gamma ray}
        \wrongchoice{radio, X ray, green, red, blue, infrared}
    \end{choices}
\end{question}
}

\element{morgans}{
\begin{question}{test1D-Q12}
    If there was a cool gas cloud of hydrogen and helium between a continuous light source and an observer, what type of spectrum would the observer detect?
    \begin{choices}
        \wrongchoice{an absorption spectrum showing only lines corresponding to hydrogen}
        \wrongchoice{an absorption spectrum showing only lines corresponding to helium}
        \wrongchoice{an emission spectrum of helium combined with an absorption spectrum of hydrogen}
        \wrongchoice{an emission spectrum containing lines corresponding to both helium and hydrogen}
      \correctchoice{an absorption spectrum with lines corresponding to both helium and hydrogen}
    \end{choices}
\end{question}
}

\element{morgans}{
\begin{question}{test1D-Q13}
    The primary purpose of a telescope is:
    \begin{choices}
      \correctchoice{to gather light}
        \wrongchoice{to magnify objects}
        \wrongchoice{to make objects look clearer}
        \wrongchoice{to look in your neighbors bedroom}
    \end{choices}
\end{question}
}

\element{morgans}{
\begin{question}{test1D-Q14}
    What must be used to detect gamma rays?
    \begin{choices}
        \wrongchoice{a well ground mirror}
      \correctchoice{a satellite}
        \wrongchoice{an array of radio telescopes}
        \wrongchoice{an interferometer}
    \end{choices}
\end{question}
}

\element{morgans}{
\begin{question}{test1D-Q15}
    In which spectral region is it possible for astronomers to observe through clouds?
    \begin{choices}
        \wrongchoice{visual}
      \correctchoice{radio}
        \wrongchoice{ultraviolet}
        \wrongchoice{X ray}
        \wrongchoice{gamma ray}
    \end{choices}
\end{question}
}

\begin{comment}
    Fill In
    Place the most appropriate word or words in the blank. You may have to click on the blank to activate it before you start typing in your answer.
     
    The spin of the Earth is called .

    The north-south line passing through both poles and the point directly above the observer is called the .

    The planetary configuration in which the Sun, Earth, and some planet are in that order and in a straight line is called .

    The apparent change in the position of a star caused by a change in the location of the observer is called .

    Empirical laws of planetary motion were discovered by .

    The orbital period of a planet is related to its average distance from the Sun, according to law.

    If a light source is moving away from an observer, the light will be .

    The type of electromagnetic wave which is adjacent to visible light but at a higher energy is radiation.

    The name of any atomic state other than the ground state is the state.

    The is the term for the clarity of an image, or the ability to see fine detail.
\end{comment}

\endinput



