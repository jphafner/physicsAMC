
%% University of Northern Iowa
%%  Morgans Astronomy Exams
%%--------------------------------------------------

%% this section contains 15 problems

\element{morgans}{
\begin{question}{test3A-Q01}
    The general shape of our galaxy is nearest to that of a(n):
    \begin{choices}
        \wrongchoice{pear.}
        \wrongchoice{hard boiled egg.}
        \wrongchoice{ball.}
      \correctchoice{fried egg.}
    \end{choices}
\end{question}
}

\element{morgans}{
\begin{question}{test3A-Q02}
    The solar system is located within:
    \begin{choices}
        \wrongchoice{the galactic halo.}
      \correctchoice{the galactic disc.}
        \wrongchoice{the galactic nucleus.}
        \wrongchoice{none of the above; the solar system is not located within a galaxy.}
    \end{choices}
\end{question}
}

\element{morgans}{
\begin{question}{test3A-Q03}
    Our ability to detect distant stars in our galaxy is limited because of:
    \begin{choices}
      \correctchoice{absorption by dust in the Galaxy.}
        \wrongchoice{the existence of strong 21-cm radiation in the Galaxy.}
        \wrongchoice{the existence of many bright nebulae in the Galaxy.}
        \wrongchoice{none of the above; there is no limit in our ability to detect distant stars.}
    \end{choices}
\end{question}
}

\element{morgans}{
\begin{question}{test3A-Q04}
    The number of stars in the Galaxy is thought to be nearest:
    \begin{choices}
        \wrongchoice{one million}
        \wrongchoice{one billion}
        \wrongchoice{ten billion}
      \correctchoice{few hundred billion}
        \wrongchoice{few hundred trillion}
    \end{choices}
\end{question}
}

\element{morgans}{
\begin{question}{test3A-Q05}
    In comparison with the Sun, which one of the following statements is true?
    \begin{choices}
      \correctchoice{stars in the halo are deficient in heavy elements}
        \wrongchoice{stars in the galactic disk are deficient in heavy elements}
        \wrongchoice{stars in the nucleus have large amounts of heavy elements}
        \wrongchoice{all chemical elements are distributed more or less uniformly throughout all parts of the Galaxy}
    \end{choices}
\end{question}
}

\element{morgans}{
\begin{question}{test3A-Q06}
    The oldest objects in the Galaxy are located:
    \begin{choices}
      \correctchoice{in the halo.}
        \wrongchoice{in the spiral arms.}
        \wrongchoice{near the Sun.}
        \wrongchoice{in the galactic center.}
    \end{choices}
\end{question}
}

\element{morgans}{
\begin{question}{test3A-Q07}
    In the famous 1920 astronomical debate,
        the primary spokesman for the view that the "nebulae'' were outside our galaxy was:
    \begin{choices}
        \wrongchoice{Hubble.}
        \wrongchoice{Shapley.}
      \correctchoice{Curtis.}
        \wrongchoice{Wilson.}
        \wrongchoice{Russell.}
    \end{choices}
\end{question}
}

\element{morgans}{
\begin{question}{test3A-Q08}
    Of the items listed below,
        which provides astronomers with the best distance determinations for galaxies?
    \begin{choices}
        \wrongchoice{parallax.}
        \wrongchoice{main sequence fitting.}
        \wrongchoice{proper motions.}
        \wrongchoice{RR Lyrae stars.}
      \correctchoice{Cepheid variables.}
    \end{choices}
\end{question}
}

\element{morgans}{
\begin{question}{test3A-Q09}
    Which type of galaxies,
        has a wide range of sizes?
    \begin{choices}
        \wrongchoice{Ellipticals}
        \wrongchoice{Spirals}
        \wrongchoice{SO}
    \end{choices}
\end{question}
}

\element{morgans}{
\begin{question}{test3A-Q10}
    The most numerous galaxies in the Local Group are \rule[-0.1pt]{4em}{0.1pt} galaxies.
    \begin{choices}
      \correctchoice{dwarf elliptical}
        \wrongchoice{spiral}
        \wrongchoice{giant elliptical}
    \end{choices}
\end{question}
}

\element{morgans}{
\begin{question}{test3A-Q11}
    If the motions of galaxies located in the outer regions of their clusters are measured,
        then:
    \begin{choices}
        \wrongchoice{the masses of the galaxies are the only quantities that can be determined.}
        \wrongchoice{the mass of the material between the galaxies is the only quantity that is measured.}
      \correctchoice{the masses of both the galaxies and the material between them is measured.}
        \wrongchoice{the mass of the university is known.}
    \end{choices}
\end{question}
}

\element{morgans}{
\begin{question}{test3A-Q12}
    Distances to the furthest galaxies can be determined most accurately using:
    \begin{choices}
        \wrongchoice{Cepheids.}
        \wrongchoice{novae.}
      \correctchoice{supernovae.}
        \wrongchoice{RR Lyrae stars.}
    \end{choices}
\end{question}
}

\element{morgans}{
\begin{question}{test3A-Q13}
    It is important that we first determine distances to Cepheids in our Galaxy,
        because:
    \begin{choices}
      \correctchoice{it is a stepping stone to determining even greater distances.}
        \wrongchoice{it is the closest cluster to the Andromeda galaxy.}
        \wrongchoice{it is the most distant globular cluster in our galaxy.}
        \wrongchoice{it is the most distant cluster of galaxies yet observed.}
    \end{choices}
\end{question}
}

\element{morgans}{
\begin{question}{test3A-Q14}
    The 3 K radiation was caused by:
    \begin{choices}
        \wrongchoice{the formation of galaxies.}
      \correctchoice{the big bang.}
        \wrongchoice{for formation of stars.}
        \wrongchoice{the formation of the planets.}
        \wrongchoice{the formation of supernovae.}
    \end{choices}
\end{question}
}

\element{morgans}{
\begin{question}{test3A-Q15}
    Cen $A$ is:
    \begin{choices}
        \wrongchoice{the radio source at the center of our galaxy.}
        \wrongchoice{a quasar.}
      \correctchoice{a radio galaxy.}
        \wrongchoice{a Seyfert galaxy.}
        \wrongchoice{a BL Lac object.}
    \end{choices}
\end{question}
}

\element{morgans}{
\begin{question}{test3A-Q16}
    The broad spectral lines formed in Seyfert galaxy nuclei are indicative of:
    \begin{choices}
        \wrongchoice{rapid motion by the galaxy away from the observer.}
        \wrongchoice{rapid motion by the galaxy towards the observer.}
        \wrongchoice{the presence of magnetic fields.}
      \correctchoice{large turbulent velocities in the nucleus.}
        \wrongchoice{a degenerate nucleus.}
    \end{choices}
\end{question}
}

\element{morgans}{
\begin{question}{test3A-Q17}
    The great distances of quasars implied by their large redshifts together with their faint apparent magnitude tells us that quasars are:
    \begin{choices}
        \wrongchoice{very massive.}
        \wrongchoice{very large in diameter.}
        \wrongchoice{high temperature objects.}
        \wrongchoice{binary galaxies.}
      \correctchoice{highly luminous.}
    \end{choices}
\end{question}
}

\element{morgans}{
\begin{question}{test3A-Q18}
    The critical density is defined as:
    \begin{choices}
      \correctchoice{the density to close the universe.}
        \wrongchoice{the density at which an object becomes a black hole.}
        \wrongchoice{the number density of galaxies in a typical cluster of galaxies.}
        \wrongchoice{the density at the center of the Big Bang.}
        \wrongchoice{the density of matter in the Milky Way.}
    \end{choices}
\end{question}
}

\element{morgans}{
\begin{question}{test3A-Q19}
    BL Lac galaxy spectra have:
    \begin{choices}
        \wrongchoice{no spectral features.}
        \wrongchoice{absorption features only.}
        \wrongchoice{emission features only.}
        \wrongchoice{both absorption and emission features.}
    \end{choices}
\end{question}
}

\element{morgans}{
\begin{question}{test3A-Q20}
    Olbers' paradox results from asking the question:
    \begin{choices}
        \wrongchoice{what is truth?}
        \wrongchoice{what is a quasar?}
        \wrongchoice{what is the nature of the Universe?}
        \wrongchoice{why is the night sky dark?}
        \wrongchoice{what am I doing here?}
    \end{choices}
\end{question}
}

\begin{comment}
    Fill-In
    Place the most appropriate word or words in the blank. You may have to click on the blank to activate it before you start typing in your answer.
     
    The name of the galaxy which contains the Sun is .

    Electromagnetic radiation at a wavelength of 21-cm is produced by .

    Globular clusters belong to Population .

    A general term for any astronomical object whose absolute magnitude can be inferred from its other observed characteristics, and which is therefore useful as a distance indicator is .

    The galaxy cluster to which our galaxy belongs is called the .

    Matter whose presence is inferred from its gravitational effect, but can not be seen is known as .

    The value of the temperature of the remnant radiation from the big bang is .

    An active galaxy, that looks like a spiral galaxy with an unusually bright nucleus is called a(n) .

    Radiation produced by the synchrotron process can be distinguished from thermal radiation because the synchrotron radiation is .

    The postulate that the universe is both homogeneous and isotropic is called the
\end{comment}

