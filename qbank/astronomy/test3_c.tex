
%% University of Northern Iowa
%%  Morgans Astronomy Exams
%%--------------------------------------------------

%% this section contains 15 problems

\element{morgans}{
\begin{question}{test3C-Q01}
    The distance of the Sun from the center of the Galaxy is nearest
        to \rule[-0.1pt]{4em}{0.1pt} parsec(s) or \rule[-0.1pt]{4em}{0.1pt} light years.
    \begin{choices}
        \wrongchoice{1; 3.26}
        \wrongchoice{100; 326}
        \wrongchoice{1,000; 3,260}
      \correctchoice{8,000; 26,000}
        \wrongchoice{80,000; 260,000}
    \end{choices}
\end{question}
}

\element{morgans}{
\begin{question}{test3C-Q02}
    The position of the Sun in the Galaxy was determined by Shapley by measuring the positions of:
    \begin{choices}
        \wrongchoice{supergiants.}
        \wrongchoice{galactic (open) clusters.}
        \wrongchoice{stellar associations.}
        \wrongchoice{globular clusters.}
        \wrongchoice{T Tauri stars.}
    \end{choices}
\end{question}
}

\element{morgans}{
\begin{question}{test3C-Q03}
    21-cm radiation is produced by:
    \begin{choices}
        \wrongchoice{dust.}
        \wrongchoice{oxygen.}
        \wrongchoice{nitrogen.}
        \wrongchoice{hydrogen.}
        \wrongchoice{helium.}
    \end{choices}
\end{question}
}

\element{morgans}{
\begin{question}{test3C-Q04}
    The nucleus of the Galaxy is composed primarily of:
    \begin{choices}
        \wrongchoice{hot stars.}
        \wrongchoice{cool stars.}
        \wrongchoice{hydrogen gas.}
        \wrongchoice{planetary nebulae.}
        \wrongchoice{solar-type stars.}
    \end{choices}
\end{question}
}

\element{morgans}{
\begin{question}{test3C-Q05}
    Stars which contribute most to the chemical enrichment of the interstellar medium are stars which are:
    \begin{choices}
        \wrongchoice{less massive than the Sun.}
        \wrongchoice{solar mass stars.}
      \correctchoice{more massive than the Sun.}
        \wrongchoice{none of the above; stars do not enrich the interstellar medium, galaxies do.}
    \end{choices}
\end{question}
}

\element{morgans}{
\begin{question}{test3C-Q06}
    We know the stars observed in globular clusters all have relatively low masses because:
    \begin{choices}
      \correctchoice{the more massive ones have all evolved and are ``dead'' stars.}
        \wrongchoice{the high mass stars were ejected from the clusters by tidal interactions with the Milky Way.}
        \wrongchoice{the high mass stars were ejected from the cluster in collisions with the Magellanic Clouds.}
        \wrongchoice{the high mass stars are too faint to observe.}
        \wrongchoice{only low mass stars formed in globular clusters.}
    \end{choices}
\end{question}
}

\element{morgans}{
\begin{question}{test3C-Q07}
    Hubble's resolution of the question of the nature of the ``nebulae'' hinged on our understanding of the:
    \begin{choices}
        \wrongchoice{H-R diagram.}
      \correctchoice{period-luminosity law.}
        \wrongchoice{mass-luminosity law.}
        \wrongchoice{mass-radius relation.}
        \wrongchoice{temperature-luminosity relation.}
    \end{choices}
\end{question}
}

\element{morgans}{
\begin{question}{test3C-Q08}
    Stars in the Instability Strip are unstable because they are no longer in:
    \begin{choices}
        \wrongchoice{thermal equilibrium.}
      \correctchoice{hydrostatic equilibrium.}
        \wrongchoice{convective equilibrium.}
        \wrongchoice{luminosity equilibrium.}
    \end{choices}
\end{question}
}

\element{morgans}{
\begin{question}{test3C-Q09}
    A galaxy whose overall color is bluish includes:
    \begin{choices}
      \correctchoice{Population I objects.}
        \wrongchoice{Population II objects.}
        \wrongchoice{Population III objects.}
    \end{choices}
\end{question}
}

\element{morgans}{
\begin{question}{test3C-Q10}
    The largest galaxy near our own (also called M31) is known as the:
    \begin{choices}
        \wrongchoice{Milky Way.}
      \correctchoice{Andromeda Galaxy.}
        \wrongchoice{Large Magellanic Cloud.}
        \wrongchoice{Small Magellanic Cloud.}
        \wrongchoice{Local Group.}
    \end{choices}
\end{question}
}

\element{morgans}{
\begin{question}{test3C-Q11}
    Which type of galaxy has the largest percentage of gas and dust in it?
    \begin{choices}
        \wrongchoice{Spirals}
      \correctchoice{Irregulars}
        \wrongchoice{Barred Spirals}
        \wrongchoice{Ellipticals}
    \end{choices}
\end{question}
}

\element{morgans}{
\begin{question}{test3C-Q12}
    The Hubble law may be expressed mathematically as:
    \begin{choices}
      \correctchoice{v = Hd.}
        \wrongchoice{v = H/d.}
        \wrongchoice{v = Hd2.}
        \wrongchoice{v = H/d2.}
        \wrongchoice{a very complicated formula.}
    \end{choices}
\end{question}
}

\element{morgans}{
\begin{question}{test3C-Q13}
    The microwave background radiation has a spectrum similar to that of:
    \begin{choices}
        \wrongchoice{synchrotron radiation from a hot body.}
      \correctchoice{a cool black body.}
        \wrongchoice{an emission nebula.}
        \wrongchoice{the Sun.}
    \end{choices}
\end{question}
}

\element{morgans}{
\begin{question}{test3C-Q14}
    The Hubble constant is currently thought to have a value of:
    \begin{choices}
        \wrongchoice{10 km/sec/kpc}
      \corRectchoice{70 km/sec/Mpc}
        \wrongchoice{100 km/sec/Mpc}
        \wrongchoice{150 km/sec/Mpc}
        \wrongchoice{greater than 200 km/sec/Mpc}
    \end{choices}
\end{question}
}

\element{morgans}{
\begin{question}{test3C-Q15}
    Radio \rule[-0.1pt]{4em}{0.1pt} are seen in many radio galaxies,
        often associated with jets that originate in the galaxy's core.
    \begin{choices}
        \wrongchoice{stations}
      \correctchoice{lobes}
        \wrongchoice{stars}
        \wrongchoice{bubbles}
        \wrongchoice{disks}
    \end{choices}
\end{question}
}

\element{morgans}{
\begin{question}{test3C-Q16}
    The redshifts of emission lines in quasar spectra are best explained as caused by:
    \begin{choices}
      \correctchoice{a large velocity for the quasar.}
        \wrongchoice{a strong gravitational field.}
        \wrongchoice{a black hole in the quasar.}
        \wrongchoice{rapid rotation.}
        \wrongchoice{turbulence in the nucleus.}
    \end{choices}
\end{question}
}

\element{morgans}{
\begin{question}{test3C-Q17}
    Broad emission lines in quasars are caused by:
    \begin{choices}
         \wrongchoice{internal motions of thousands of km/sec.}
       \correctchoice{rapid rotation.}
         \wrongchoice{recession velocities near the speed of light.}
         \wrongchoice{nuclei composed of degenerate matter.}
         \wrongchoice{strong magnetic fields.}
    \end{choices}
\end{question}
}

\element{morgans}{
\begin{question}{test3C-Q18}
    If the density of the universe is equal to the critical density then the universe has a curvature that is:
    \begin{choices}
        \wrongchoice{spherical.}
        \wrongchoice{hyperbolic.}
      \correctchoice{flat.}
        \wrongchoice{there is no term to describe that condition.}
    \end{choices}
\end{question}
}

\element{morgans}{
\begin{question}{test3C-Q19}
    The assumption that the general structure of the universe is the same everywhere is the assumption of:
    \begin{choices}
        \wrongchoice{isotropy.}
      \correctchoice{homogeneity.}
        \wrongchoice{regularity.}
        \wrongchoice{universality.}
        \wrongchoice{constancy.}
    \end{choices}
\end{question}
}

\element{morgans}{
\begin{question}{test3C-Q20}
    Which force is \emph{not} included in the Grand Unification Theory?
    \begin{choices}
        \wrongchoice{Electromagnetic}
        \wrongchoice{Strong nuclear}
        \wrongchoice{Weak nuclear}
      \correctchoice{Gravity}
    \end{choices}
\end{question}
}

\begin{comment}
    Fill In
    Place the most appropriate word or words in the blank. You may have to click on the blank to activate it before you start typing in your answer.
     
    The extended spherical volume of a galaxy is called the galactic .

    The Sun is located approximately kpc from the Galactic center.

    Open clusters are members of Population .

    The primary spokesman for the view that the "nebulae" were inside our galaxy was .

    The size of the galaxy cluster of which we are a member is approximately .

    Cluster of clusters of galaxies are known as .

    The accepted value of the Hubble constant is about .

    Strange objects that are thought to be the nuclei of young galaxies with very high red shifts are called .

    An active galaxy with no features in its spectra is known as a(n)  .

    The word for the idea that the universe appears the same in all directions is
\end{comment}


\endinput



