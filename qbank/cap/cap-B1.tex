

%% CAP High School Prize Examination
%%----------------------------------------


%% this section contains 40 problems


%% CAP Exam 2015
%%----------------------------------------


%% CAP Exam 2013
%%----------------------------------------
\element{cap}{ %% cap-B1
\begin{question}{CAP-A-2013-q15}
    Adrianne has a device that emits photons with a wavelength of $\lambda=\SI{1.498}{\kilo\meter}$.
    Kwan has a similar device that emits photons at a frequency of $f=\SI{201}{\kilo\hertz}$.
    Whose device emits photons in air at the highest frequency?
    \begin{choices}
        \wrongchoice{Adrianne's}
      \correctchoice{Kwan's}
        \wrongchoice{The frequencies are the same.}
        \wrongchoice{There is not enough information to compare.}
    \end{choices}
\end{question}
}

\element{cap}{ %% cap-B1
\begin{question}{CAP-A-2013-q20}
    Two interfering waves have the same wavelength, frequency, and amplitude. 
    They are traveling in the same direction but are 90 degrees out of phase. 
    Compared to the individual waves,
        what can be said about the resultant wave?
    \begin{choices}
        \wrongchoice{It will have the same amplitude and velocity, but a different wavelength.}
        \wrongchoice{It will have the same amplitude and wavelength, but a different velocity.}
      \correctchoice{It will have the same wavelength and velocity, but a different amplitude.}
        \wrongchoice{It will have the same amplitude and frequency, but a different velocity.}
        \wrongchoice{It will have the same frequency and velocity, but a different wavelength.}
    \end{choices}
\end{question}
}


%% CAP Exam 2012
%%----------------------------------------
\element{cap}{ %% cap-B1
\begin{question}{CAP-A-2012-q14}
    Consider a metal rod firmly attached to a wall. 
    \begin{center}
    \begin{tikzpicture}
        %% wall
        \draw (0,-1) -- (0,1);
        \node[anchor=west,fill,pattern=north east lines,minimum width=0.05cm, minimum height=2cm] at (0,0) {};
        %% rod
        \draw[fill=white!90!black] (0,-1ex) rectangle (-5,1ex);
        %% hammer
        \begin{scope}[rotate=-45,xshift=-4.5cm,yshift=-5cm]
            %% handle
            \draw[fill=white!90!black] (-1ex,0) -- (-0.5ex,2) -- (0.5ex,2) -- (1ex,0) -- cycle;
            %% mallet
            \draw[fill=white!60!black] (-2ex,2) -- (-1.5ex,2.2) -- (1.5ex,2.2) -- (2ex,2) -- cycle;
            \draw[fill] (-4ex,2.2) -- (-4ex,2.7) -- (4ex,2.7) -- (4ex,2.2) -- cycle;
        \end{scope}
    \end{tikzpicture}
    \end{center}
    When you strike the rod with a hammer,
        which kind of wave Which of the following is the closest to the total mass will you excite?
    \begin{choices}
        \wrongchoice{A longitudinal wave.}
        \wrongchoice{A transverse wave.}
      \correctchoice{Either kind, or both, depending on where and how you hit the rod.}
        \wrongchoice{No wave will be excited, the metal is too strong.}
    \end{choices}
\end{question}
}

\element{cap}{ %% cap-B1
\begin{question}{CAP-A-2012-q21}
    Which of the following radiation types has the longest wavelength?
    \begin{choices}
      \correctchoice{Radio Waves}
        \wrongchoice{Visible light}
        \wrongchoice{X rays}
        \wrongchoice{Infrared light}
        \wrongchoice{All of the provided have the same wavelength.}
    \end{choices}
\end{question}
}


%% CAP Exam 2009
%%----------------------------------------
\element{cap}{ %% cap-B1
\begin{question}{CAP-A-2009-q18}
    Two sinusoidal waves traveling at the same speed in opposite directions interfere to produce a standing wave with the wave function
        $y = \left(\SI{1.50}{\meter}\right) \sin\left(0.400x\right)\cos\left(200t\right)$,
        where $x$ is in meters and $t$ is in seconds. 
    The speed of propagation of each of the interfering waves is:
    \begin{multicols}{3}
    \begin{choices}
        \wrongchoice{\SI{159}{\meter\per\second}}
        \wrongchoice{\SI{200}{\meter\per\second}}
        \wrongchoice{\SI{300}{\meter\per\second}}
        \wrongchoice{\SI{47.7}{\meter\per\second}}
      \correctchoice{\SI{500}{\meter\per\second}}
    \end{choices}
    \end{multicols}
\end{question}
}


\endinput


