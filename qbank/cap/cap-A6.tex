

%% CAP High School Prize Examination
%%----------------------------------------


%% this section contains 40 problems


%% CAP Exam 2015
%%----------------------------------------


%% CAP Exam 2013
%%----------------------------------------
\element{cap}{ %% cap-A6
\begin{question}{CAP-A-2013-q10}
    A uniform rod is suspended on two thin strings as shown below. 
    \begin{center}
    \begin{tikzpicture}
        %% ceiling
        \draw (-3,0) -- (3,0);
        \node[anchor=south,fill,pattern=north east lines,minimum width=6cm, minimum height=0.05cm] at (0,0) {};
        %% large rod
        \draw[line width=3pt] (-2,-3) -- (2,-1.5);
        %% strings
        \draw (-2,-3) -- (-2,0);
        \draw (2,-1.5) -- (2,0);
    \end{tikzpicture}
    \end{center}
    Which string has a larger tension force?
    \begin{choices}
        \wrongchoice{The right one}
        \wrongchoice{The left one}
      \correctchoice{The tension force is the same on both}
        \wrongchoice{It depends on the angle}
    \end{choices}
\end{question}
}

\element{cap}{ %% cap-A6
\begin{question}{CAP-A-2013-q21}
    A spinning ice skater has an initial kinetic energy $\frac{1}{2} I\omega^2$.
    She pulls in her outstretched arms,
        decreasing her moment of inertia to $\frac{1}{4} I$.
    What is her new angular speed?
    \begin{multicols}{3}
    \begin{choices}
        \wrongchoice{$\dfrac{\omega}{4}$}
        \wrongchoice{$\dfrac{\omega}{2}$}
        \wrongchoice{$\omega$}
        \wrongchoice{$2\omega$}
      \correctchoice{$4\omega$}
    \end{choices}
    \end{multicols}
\end{question}
}


%% CAP Exam 2012
%%----------------------------------------
\element{cap}{ %% cap-A6
\begin{question}{CAP-A-2012-q16}
    A rod ($AB$) is attached to a fixed point ($C$) using a light rope ($AC$). 
    The other end of the rod ($B$) is sitting on ice with negligible friction and the system is in stationary position. 
    Which of the following can be the equilibrium configuration of this system?
    \begin{multicols}{2}
    \begin{choices}
        \AMCboxDimensions{down=-1cm}
        \wrongchoice{
            \begin{tikzpicture}
                %% floor
                \draw (0,0) -- (3,0);
                \node[anchor=north,fill,pattern=north east lines,minimum width=3cm, minimum height=0.05cm] at (1.5,0) {};
                %% coordinate A, B, C
                \coordinate (A) at (0.5,2);
                \coordinate (B) at (+2,0);
                \coordinate (C) at (+2,2);
                %% lables
                \draw[fill] (A) circle (1.5pt) node[anchor=east] {$A$};
                \draw[fill] (B) circle (1.5pt) node[anchor=south west] {$B$};
                \draw[fill] (C) circle (1.5pt) node[anchor=west] {$C$};
                %% Rod AB, rope AC
                \draw[very thick] (A) -- (B);
                \draw[thin] (A) -- (C);
                \draw[dashed] (C) -- (B);
            \end{tikzpicture}
        }
        \wrongchoice{
            \begin{tikzpicture}
                %% floor
                \draw (0,0) -- (3,0);
                \node[anchor=north,fill,pattern=north east lines,minimum width=3cm, minimum height=0.05cm] at (1.5,0) {};
                %% coordinate A, B, C
                \coordinate (A) at (0.5,1.5);
                \coordinate (B) at (+2.5,0);
                \coordinate (C) at (1.5,2);
                %% lables
                \draw[fill] (A) circle (1.5pt) node[anchor=east] {$A$};
                \draw[fill] (B) circle (1.5pt) node[anchor=south west] {$B$};
                \draw[fill] (C) circle (1.5pt) node[anchor=west] {$C$};
                %% Rod AB, rope AC
                \draw[dashed] (C) -- (1.5,0);
                \draw[very thick] (A) -- (B);
                \draw[thin] (A) -- (C);
            \end{tikzpicture}
        }
        %% ANS: is C
        \correctchoice{
            \begin{tikzpicture}
                %% floor
                \draw (0,0) -- (3,0);
                \node[anchor=north,fill,pattern=north east lines,minimum width=3cm, minimum height=0.05cm] at (1.5,0) {};
                %% coordinate A, B, C
                \coordinate (A) at (0.5,1.25);
                \coordinate (B) at (+2.75,0);
                \coordinate (C) at (0.5,2);
                %% lables
                \draw[fill] (A) circle (1.5pt) node[anchor=east] {$A$};
                \draw[fill] (B) circle (1.5pt) node[anchor=south] {$B$};
                \draw[fill] (C) circle (1.5pt) node[anchor=west] {$C$};
                %% Rod AB, rope AC
                \draw[dashed] (C) -- (0.5,0);
                \draw[very thick] (A) -- (B);
                \draw[thin] (A) -- (C);
            \end{tikzpicture}
        }
        \wrongchoice{
            \begin{tikzpicture}
                %% floor
                \draw (0,0) -- (3,0);
                \node[anchor=north,fill,pattern=north east lines,minimum width=3cm, minimum height=0.05cm] at (1.5,0) {};
                %% coordinate A, B, C
                \coordinate (A) at (1,1);
                \coordinate (B) at (+2.75,0);
                \coordinate (C) at (0.25,2);
                %% lables
                \draw[fill] (A) circle (1.5pt) node[anchor=south west] {$A$};
                \draw[fill] (B) circle (1.5pt) node[anchor=south] {$B$};
                \draw[fill] (C) circle (1.5pt) node[anchor=west] {$C$};
                %% Rod AB, rope AC
                \draw[dashed] (C) -- (0.25,0);
                \draw[very thick] (A) -- (B);
                \draw[thin] (A) -- (C);
            \end{tikzpicture}
        }
    \end{choices}
    \end{multicols}
\end{question}
}


%% CAP Exam 2010
%%----------------------------------------
\element{cap}{ %% cap-A6
\begin{question}{CAP-A-2010-q20}
    Two gear wheels of the same thickness, but with one twice the diameter of the other,
        are mounted on parallel light axles far enough apart not to mesh the teeth of the wheels. 
    The larger wheel is spun with angular velocity $\Omega$ and the wheels are then moved together so they mesh. 
    What is the subsequent angular velocity of the larger wheel?
    \begin{multicols}{4}
    \begin{choices}
        \wrongchoice{$\dfrac{\Omega}{2}$}
        \wrongchoice{$\dfrac{\Omega}{5}$}
        \wrongchoice{$\dfrac{2\Omega}{5}$}
      \correctchoice{$\dfrac{4\Omega}{5}$}
    \end{choices}
    \end{multicols}
\end{question}
}


\endinput


