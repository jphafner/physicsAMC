

%% CAP High School Prize Examination
%%----------------------------------------


%% this section contains 40 problems


%% CAP Exam 2015
%%----------------------------------------
\element{cap}{ %% cap-B2
\begin{question}{CAP-A-2015-q14}
    An observer is running toward a mirror with a speed of \SI{15}{\kilo\meter\per\hour},
        and the mirror is moving toward the observer with a speed of \SI{10}{\kilo\meter\per\hour}. 
    What is the relative speed with which the observer sees his image approaching him?
    \begin{multicols}{3}
    \begin{choices}
        \wrongchoice{\SI{10}{\kilo\meter\per\hour}}
        \wrongchoice{\SI{20}{\kilo\meter\per\hour}}
        \wrongchoice{\SI{25}{\kilo\meter\per\hour}}
        \wrongchoice{\SI{30}{\kilo\meter\per\hour}}
      \correctchoice{\SI{50}{\kilo\meter\per\hour}}
    \end{choices}
    \end{multicols}
\end{question}
}

\element{cap}{ %% cap-B2
\begin{question}{CAP-A-2015-q18}
    A parallel beam of light of frequency \SI{6.9e14}{\hertz} enters a glass plate with an index of refraction $n=1.5$.
    The frequency of light in the glass is:
    \begin{multicols}{2}
    \begin{choices}
        \wrongchoice{\SI{4.6e14}{\hertz}}
      \correctchoice{\SI{6.9e14}{\hertz}}
        \wrongchoice{\SI{10.36e14}{\hertz}}
        \wrongchoice{\SI{13.8e14}{\hertz}}
    \end{choices}
    \end{multicols}
\end{question}
}


%% CAP Exam 2014
%%----------------------------------------
\element{cap}{ %% cap-B2
\begin{question}{CAP-A-2014-q22}
    What will happen to the magnitude of the optical power of a lens when it is placed in water ($n=1.33$) compared to its power in the air ($n=1.00$)?
    \begin{choices}
        \wrongchoice{It will increase.}
      \correctchoice{It will decrease.}
        \wrongchoice{It will stay the same.}
        \wrongchoice{It will depend on whether the lens is converging or diverging.}
    \end{choices}
\end{question}
}


%% CAP Exam 2013
%%----------------------------------------
\element{cap}{ %% cap-B2
\begin{question}{CAP-A-2013-q24}
    You are given two lenses placed close together: a converging lens with focal length \SI{+10}{\centi\meter},
        and a diverging lens with focal length \SI{-20}{\centi\meter}.
    Which of the following would produce a virtual image that is larger than the object?
    \begin{choices}
      \correctchoice{Placing the object \SI{5}{\centi\meter} from the converging lens.}
        \wrongchoice{Placing the object \SI{15}{\centi\meter} from the converging lens.}
        \wrongchoice{Placing the object \SI{25}{\centi\meter} from the converging lens.}
        \wrongchoice{Placing the object \SI{15}{\centi\meter} from the diverging lens.}
        \wrongchoice{Placing the object \SI{25}{\centi\meter} from the diverging lens.}
    \end{choices}
\end{question}
}


%% CAP Exam 2010
%%----------------------------------------
\element{cap}{ %% cap-B2
\begin{question}{CAP-A-2010-q06}
    Two identical glasses, one completely filled with water,
        and the other filled with vodka, are viewed directly from above.
    The refractive index of vodka is slightly greater than the refractive index of water. 
    Which glass appears to have a greater depth of fluid?
    \begin{choices}
      \correctchoice{the water glass;}
        \wrongchoice{the vodka glass;}
        \wrongchoice{neither glass because the apparent depth is the same as the real depth in both glasses;}
        \wrongchoice{neither glass, because the effect of the clear fluid is to reduce the apparent depth by the same factor in both glasses.}
    \end{choices}
\end{question}
}


\endinput


