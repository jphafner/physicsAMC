

%% CAP High School Prize Examination
%%----------------------------------------


%% this section contains 40 problems


%% CAP Exam 2015
%%----------------------------------------
\element{cap}{ %% cap-A1
\begin{question}{CAP-A-2015-q13}
    A passenger of a train that is moving with the speed of \SI{25}{\meter\per\second} sees from his seat far from a window that it takes 6 seconds for another train to pass the window completely. 
    If the length of the second train is \SI{300}{\meter} and it is going in the opposite direction,
        what is its velocity?
    \begin{multicols}{3}
    \begin{choices}
        \wrongchoice{\SI{15}{\meter\per\second}}
        \wrongchoice{\SI{20}{\meter\per\second}}
      \correctchoice{\SI{25}{\meter\per\second}}
        \wrongchoice{\SI{30}{\meter\per\second}}
        \wrongchoice{\SI{35}{\meter\per\second}}
    \end{choices}
    \end{multicols}
\end{question}
}


%% CAP Exam 2014
%%----------------------------------------
\element{cap}{ %% cap-A1
\begin{question}{CAP-A-2014-q15}
    In 1929, Edwin Hubble discovered that the universe is expanding. 
    He observed that galaxies far away from us are moving away at a speed that is proportional to their distance from us 
        (you can assume the constant of proportionality is time-independent). 
    For a galaxy that obeys Hubble's law,
        which of the following can be the graph of distance (from Earth) versus time? 
    For each plot, $t=0$ corresponds to the present.
    \begin{multicols}{2}
    \begin{choices}
        \AMCboxDimensions{down=-2.5em}
        \wrongchoice{
            \begin{tikzpicture}
                \begin{axis}[
                    axis y line=left, 
                    axis x line=bottom, 
                    axis line style={->},
                    xlabel={time},
                    xtick=\empty,
                    ylabel={distance},
                    ytick=\empty,
                    xmin=0,xmax=11,
                    ymin=0,ymax=11,
                    width=1.0\columnwidth,
                ]
                \addplot[line width=1pt] plot coordinates { (0,2) (10,10) };
                \end{axis}
            \end{tikzpicture}
        }
        %% NOTE: ANS is B
        \correctchoice{
            \begin{tikzpicture}
                \begin{axis}[
                    axis y line=left, 
                    axis x line=bottom, 
                    axis line style={->},
                    xlabel={time},
                    xtick=\empty,
                    ylabel={distance},
                    ytick=\empty,
                    xmin=0,xmax=11,
                    ymin=0,ymax=11,
                    width=1.0\columnwidth,
                ]
                \addplot[line width=1pt,domain=0:10] { 2 + 0.08*x*x };
                \end{axis}
            \end{tikzpicture}
        }
        \wrongchoice{
            \begin{tikzpicture}
                \begin{axis}[
                    axis y line=left, 
                    axis x line=bottom, 
                    axis line style={->},
                    xlabel={time},
                    xtick=\empty,
                    ylabel={distance},
                    ytick=\empty,
                    xmin=0,xmax=11,
                    ymin=0,ymax=11,
                    width=1.0\columnwidth,
                ]
                \addplot[line width=1pt,domain=0:10] { 10 - 0.08*(x-10)*(x-10) };
                \end{axis}
            \end{tikzpicture}
        }
        \wrongchoice{
            \begin{tikzpicture}
                \begin{axis}[
                    axis y line=left, 
                    axis x line=bottom, 
                    axis line style={->},
                    xlabel={time},
                    xtick=\empty,
                    ylabel={distance},
                    ytick=\empty,
                    xmin=0,xmax=11,
                    ymin=0,ymax=11,
                    width=1.0\columnwidth,
                ]
                %\addplot[line width=1pt,domain=0:10] { 2 + exp(x)/2200 };
                \addplot[line width=1pt,domain=0:10] { 2 + 0.0008*x*x*x*x };
                \end{axis}
            \end{tikzpicture}
        }
    \end{choices}
    \end{multicols}
\end{question}
}


%% CAP Exam 2013
%%----------------------------------------
\element{cap}{ %% cap-A1
\begin{question}{CAP-A-2013-q19}
    Car $A$ and car $B$ are both traveling down a straight highway at \SI{90}{\kilo\meter\per\hour}.
    Car $A$ is only \SI{6.0}{\meter} behind car $B$.
    The driver of car $B$ brakes,
        slowing down with a constant deceleration of \SI{2.0}{\meter\per\second\squared}.
    After \SI{1.2}{\second},
        the driver of car $A$ begins to brake,
        also at \SI{2.0}{\meter\per\second\squared},
        but this is insufficient and the cars eventually collide,
        both still moving forward. 
    What is the relative velocity of the two cars at the moment of the collision?
    \begin{multicols}{3}
    \begin{choices}
      \correctchoice{\SI{2.4}{\meter\per\second}}
        \wrongchoice{\SI{5.0}{\meter\per\second}}
        \wrongchoice{\SI{9.5}{\meter\per\second}}
        \wrongchoice{\SI{21}{\meter\per\second}}
        \wrongchoice{\SI{24}{\meter\per\second}}
    \end{choices}
    \end{multicols}
\end{question}
}


%% CAP Exam 2012
%%----------------------------------------
\element{cap}{ %% cap-A1
\begin{question}{CAP-A-2012-q09}
    A detector far away from the source of a wave is detecting pulses of that wave every 0.2 second. 
    If the detector starts to move towards the source at a speed of \SI{6.0}{\kilo\meter\per\hour},
        then it would detect a total of 18200 pulses per hour. 
    What is the speed of the wave?
    \begin{multicols}{2}
    \begin{choices}
        \wrongchoice{\SI{100}{\meter\per\second}}
      \correctchoice{\SI{150}{\meter\per\second}}
        \wrongchoice{\SI{200}{\meter\per\second}}
        \wrongchoice{\SI{300}{\meter\per\second}}
    \end{choices}
    \end{multicols}
\end{question}
}

\element{cap}{ %% cap-A1
\begin{question}{CAP-A-2012-q15}
    Assume that you videotaped the fall of a ball due to gravity in vacuum,
        and are now playing the video in reverse at the same speed. 
    Which of the following statements is correct about the acceleration of the ball seen in these conditions,
        compared to that of the actual falling ball?
    \begin{choices}
      \correctchoice{The accelerations are the same in both cases.}
        \wrongchoice{They have the same value but opposite directions.}
        \wrongchoice{They have different values but the same direction.}
        \wrongchoice{Both the values and directions differ.}
    \end{choices}
\end{question}
}


%% CAP Exam 2011
%%----------------------------------------
\element{cap}{ %% cap-A1
\begin{questionmult}{CAP-A-2011-q21}
    If you toss a ball up, at the highest point:
    \begin{choices}
      \correctchoice{The velocity changes direction.}
        \wrongchoice{The acceleration changes direction.}
        \wrongchoice{The acceleration is zero.}
        \wrongchoice{Both velocity and acceleration are zero.}
        %\wrongchoice{More than one of the above is correct.}
    \end{choices}
\end{questionmult}
}


\endinput


