
%%--------------------------------------------------
%% Holt: Multiple Choice Questions
%%--------------------------------------------------


%% Chapter 16: Electric Forces and Field
%%--------------------------------------------------


%% Holt Multiple Choice Questions
%%--------------------------------------------------
\element{holt-mc}{
\begin{question}{ch16-mc-Q01}
    In which way is the electric field similar to the graviational field?
    \begin{choices}
        \wrongchoice{Electric force is proportional to the mass of the object.}
        \wrongchoice{Electric force is similar in strength to gravitational force.}
        \wrongchoice{Electric force is both attractive and repulsive.}
        \wrongchoice{Electric force decreases in strength as the distance between the charges increases}
    \end{choices}
\end{question}
}

\element{holt-mc}{
\begin{question}{ch16-mc-Q02}
    What must the charges be for $A$ and $B$ in the figure below so that they produce the electric field lines shown?
    \begin{center}
    \begin{tikzpicture}
        %% NOTE: insert diagram
    \end{tikzpicture}
    \end{center}
    \begin{choices}
        \wrongchoice{$A$ and $B$ must both be positive}
        \wrongchoice{$A$ and $B$ must both be negative}
        \wrongchoice{$A$ must be negative, and $B$ must be positive}
        \wrongchoice{$A$ must be positive, and $B$ must be negative}
    \end{choices}
\end{question}
}

\element{holt-mc}{
\begin{question}{ch16-mc-Q03}
    Which activity does not produce the same results as the other three?
    \begin{choices}
        \wrongchoice{sliding over a plastic covered automobile seat}
        \wrongchoice{walking across a woolen carpet}
        \wrongchoice{scraping food from a metal bowl with a metal spoon}
        \wrongchoice{brushing dry hair with a plastic comb}
    \end{choices}
\end{question}
}

\element{holt-mc}{
\begin{question}{ch16-mc-Q04}
    By how much does the electric force between two charges change when the distance between them is doubled?
    \begin{multicols}{2}
    \begin{choices}
        \wrongchoice{\num{4}}
        \wrongchoice{\num{2}}
        \wrongchoice{\num{1/2}}
        \wrongchoice{\num{1/4}}
    \end{choices}
    \end{multicols}
\end{question}
}

\element{holt-mc}{
\begin{question}{ch16-mc-Q05}
    A negatively charged object is brought close to the surface of a conductor,
        whose opposite side is then grounded?
    What is this process of charging called?
    \begin{choices}
        \wrongchoice{charging by contact}
        \wrongchoice{charging by induction}
        \wrongchoice{charging by conduction}
        \wrongchoice{charging by polarization}
    \end{choices}
\end{question}
}

\element{holt-mc}{
\begin{question}{ch16-mc-Q06}
    A negatively charged object is brought close to the surface of a conductor,
        whose opposite side is then grounded?
    What kind of charge is left on the conductor's surface?
    \begin{choices}
        \wrongchoice{neutral}
        \wrongchoice{negative}
        \wrongchoice{positive}
        \wrongchoice{both positive and negative}
    \end{choices}
\end{question}
}

\element{holt-mc}{
\begin{question}{ch16-mc-Q07}
    The graph shows the electric field strength at different distances from the center of the charged conducting sphere of a Van de Graaff generator.
    \begin{center}
    \begin{tikzpicture}
    \end{tikzpicture}
    \end{center}
    What is the electric field strength \SI{2.0}{\meter} from the center of the conducting sphere?
    \begin{multicols}{2}
    \begin{choices}
        \wrongchoice{\SI{0}{\newton\per\coulomb}}
        \wrongchoice{\SI{5.0e2}{\newton\per\coulomb}}
        \wrongchoice{\SI{5.0e3}{\newton\per\coulomb}}
        \wrongchoice{\SI{7.2e3}{\newton\per\coulomb}}
    \end{choices}
    \end{multicols}
\end{question}
}

\element{holt-mc}{
\begin{question}{ch16-mc-Q08}
    The graph shows the electric field strength at different distances from the center of the charged conducting sphere of a Van de Graaff generator.
    \begin{center}
    \begin{tikzpicture}
    \end{tikzpicture}
    \end{center}
    What is the strength of the electric field at the surface of the conducting sphere?
    \begin{multicols}{2}
    \begin{choices}
        \wrongchoice{\SI{0}{\newton\per\coulomb}}
        \wrongchoice{\SI{1.5e2}{\newton\per\coulomb}}
        \wrongchoice{\SI{2.0e2}{\newton\per\coulomb}}
        \wrongchoice{\SI{7.2e3}{\newton\per\coulomb}}
    \end{choices}
    \end{multicols}
\end{question}
}

\element{holt-mc}{
\begin{question}{ch16-mc-Q09}
    The graph shows the electric field strength at different distances from the center of the charged conducting sphere of a Van de Graaff generator.
    \begin{center}
    \begin{tikzpicture}
    \end{tikzpicture}
    \end{center}
    What is the strength of the electric field inside the conducting sphere?
    \begin{multicols}{2}
    \begin{choices}
        \wrongchoice{\SI{0}{\newton\per\coulomb}}
        \wrongchoice{\SI{1.5e2}{\newton\per\coulomb}}
        \wrongchoice{\SI{2.0e2}{\newton\per\coulomb}}
        \wrongchoice{\SI{7.2e3}{\newton\per\coulomb}}
    \end{choices}
    \end{multicols}
\end{question}
}

\element{holt-mc}{
\begin{question}{ch16-mc-Q10}
    The graph shows the electric field strength at different distances from the center of the charged conducting sphere of a Van de Graaff generator.
    \begin{center}
    \begin{tikzpicture}
    \end{tikzpicture}
    \end{center}
    What is the radius of the conducting sphere?
    \begin{multicols}{2}
    \begin{choices}
        \wrongchoice{\SI{0.5}{\meter}}
        \wrongchoice{\SI{1.0}{\meter}}
        \wrongchoice{\SI{1.5}{\meter}}
        \wrongchoice{\SI{2.0}{\meter}}
    \end{choices}
    \end{multicols}
\end{question}
}


\endinput

