
%%--------------------------------------------------
%% Holt: Multiple Choice Questions
%%--------------------------------------------------


%% Chapter 22: Subatomic Physics
%%--------------------------------------------------


%% Holt Multiple Choice Questions
%%--------------------------------------------------
\element{holt-mc}{
\begin{question}{ch22-mc-Q01}
    Which of the following statements correctly describes a nucleus with the symbol \ce{^{14}_{6}C}?
    \begin{choices}
        \wrongchoice{It is the nucleus of a cobalt atom with eight protons and six neutrons}
        \wrongchoice{It is the nucleus of a carbon atom with eight protons and six neutrons}
        \wrongchoice{It is the nucleus of a carbon atom with six protons and eight neutrons}
        \wrongchoice{It is the nucleus of a carbon atom with six protons and fourteen neutrons}
    \end{choices}
\end{question}
}

\element{holt-mc}{
\begin{question}{ch22-mc-Q02}
    One unified mass unit (u) is equivalent to a mass of \SI{1.66e-27}{\kilo\gram}.
    What is the equivalent rest energy?
    \begin{multicols}{2}
    \begin{choices}
        \wrongchoice{\SI{8.27e-46}{\kilo\gram}}
        \wrongchoice{\SI{4.98e-19}{\kilo\gram}}
        \wrongchoice{\SI{1.49e-10}{\kilo\gram}}
        \wrongchoice{\SI{9.31e8}{\kilo\gram}}
    \end{choices}
    \end{multicols}
\end{question}
}

\element{holt-mc}{
\begin{question}{ch22-mc-Q03}
    What kind of force holds protons and neutrons together in a nucleus?
    \begin{choices}
        \wrongchoice{electric force}
        \wrongchoice{gravitational force}
        \wrongchoice{binding force}
        \wrongchoice{strong force}
    \end{choices}
\end{question}
}

\element{holt-mc}{
\begin{question}{ch22-mc-Q04}
    What type of nuclear decay most often produces the greatest mass loss?
    \begin{choices}
        \wrongchoice{alpha ($\alpha$) decay}
        \wrongchoice{beta ($\beta$) decay}
        \wrongchoice{gamma ($\gamma$) decay}
        \wrongchoice{alpha, beta and gamma all produce the same mass loss}
    \end{choices}
\end{question}
}

\element{holt-mc}{
\begin{question}{ch22-mc-Q05}
    A nuclear reaction of major historical note took place in 1932, when a berylium target was bombarded with alpha particles.
    Analysis of the experiment showed that the following reaction took place:
        \ce{^{4}_{2}He + ^{9}_{4}Be -> ^{12}_{6}C + X}.
    What is \ce{X} in this reaction?
    \begin{multicols}{2}
    \begin{choices}
        \wrongchoice{\ce{^{0}_{1}e}}
        \wrongchoice{\ce{^{0}_{-1}p}}
        \wrongchoice{\ce{^{1}_{0}n}}
        \wrongchoice{\ce{^{1}_{1}p}}
    \end{choices}
    \end{multicols}
\end{question}
}

\element{holt-mc}{
\begin{question}{ch22-mc-Q06}
    What fraction of a radioactive sample has decayed after two half-lives have elapsed?
    \begin{multicols}{2}
    \begin{choices}
        \wrongchoice{$\frac{1}{4}$}
        \wrongchoice{$\frac{1}{2}$}
        \wrongchoice{$\frac{3}{4}$}
        \wrongchoice{The whole sample has decayed}
    \end{choices}
    \end{multicols}
\end{question}
}

\element{holt-mc}{
\begin{question}{ch22-mc-Q07}
    A sample of organic material is found to contain \SI{18}{\gram} of carbon-14.
    Based on samples of pottery found at a dig,
        investigators believe the material is about \num{23 000} years old.
    The half-life of carbon-14 is \SI{5715}{\year}.
    Estimate what percentage of the material's carbon-14 has decayed?
    \begin{multicols}{2}
    \begin{choices}
        \wrongchoice{\SI{4.0}{\percent}}
        \wrongchoice{\SI{25}{\percent}}
        \wrongchoice{\SI{75}{\percent}}
        \wrongchoice{\SI{94}{\percent}}
    \end{choices}
    \end{multicols}
\end{question}
}

\element{holt-mc}{
\begin{question}{ch22-mc-Q08}
    The half-life of radium-228 is \SI{5.76}{\year}.
    At some instant, a sample contains \num{2.0e9} nuclei.
    Calculate the decay and the activity of the sample.
    \begin{choices}
        \wrongchoice{}
        %\wrongchoice{$\lambda$=\SI{3.81e-9}{\per\second}; activity = \SI{2.1e-10}{\$=}
    \end{choices}
\end{question}
}

\element{holt-mc}{
\begin{question}{ch22-mc-Q09}
    What must be true in order for a nuclear reaction to happen naturally?
    \begin{choices}
        \wrongchoice{The nucleus must release energy in the reaction}
        \wrongchoice{The binding eneryg per nucleon must decrease in the reaction}
        \wrongchoice{The binding energy per nucleon must increase in the reaction}
        \wrongchoice{There must be an input of energy to cause the reactoin.}
    \end{choices}
\end{question}
}

\element{holt-mc}{
\begin{question}{ch22-mc-Q10}
    Which is theweakest of teh four fundamental interactions? 
    What must be true in order for a nuclear reaction to happen naturally?
    \begin{choices}
        \wrongchoice{electromagnetic}
        \wrongchoice{gravitatoinal}
        \wrongchoice{strong}
        \wrongchoice{weak}
    \end{choices}
\end{question}
}

\element{holt-mc}{
\begin{question}{ch22-mc-Q11}
    Which of the following choices does \emph{not} correctly match a fundamnetal interaction with its mediating particle?
    \begin{choices}
        \wrongchoice{strong, gluons}
        \wrongchoice{electromagnetic, electron}
        \wrongchoice{weak: $W$ and $Z$ bosons}
        \wrongchoice{gravitational: graviton}
    \end{choices}
\end{question}
}

\element{holt-mc}{
\begin{question}{ch22-mc-Q12}
    What is the charge of a baryond containing one up quark (u) and two down quarks (d)?
    \begin{mutlicols}{4}
    \begin{choices}
        \wrongchoice{\num{-1}}
        \wrongchoice{\num{0}}
        \wrongchoice{\num[retain-epxlicit-plus]{+1}
        \wrongchoice{\num[retain-epxlicit-plus]{}}
   \end{choices}
    \end{kmutlicols}
\end{question}
}



\endinput

