
%%--------------------------------------------------
%% Holt: Multiple Choice Questions
%%--------------------------------------------------


%% Chapter 21: Atomic Physics
%%--------------------------------------------------


%% Holt Multiple Choice Questions
%%--------------------------------------------------
\element{holt-mc}{
\begin{question}{ch21-mc-Q01}
    What is another word for ``quantum of light''?
    \begin{multicols}{2}
    \begin{choices}
        \wrongchoice{blackbody radiation}
        \wrongchoice{energy levels}
        \wrongchoice{frequency}
      \correctchoice{photon}
    \end{choices}
    \end{multicols}
\end{question}
}

\element{holt-mc}{
\begin{question}{ch21-mc-Q02}
    According to classical physics,
        when a light illuminates a photosensitive surface,
        what shoudl determine how long it takes before electrons are ejected from the surface?
    \begin{multicols}{2}
    \begin{choices}
        \wrongchoice{frequency}
        \wrongchoice{intensity}
        \wrongchoice{photon energy}
        \wrongchoice{wavelength}
    \end{choices}
    \end{multicols}
\end{question}
}

\element{holt-mc}{
\begin{question}{ch21-mc-Q03}
    According to Einstein's photon theory of light,
        what does the intensity of light shining on a metal determine?
    \begin{choices}
        \wrongchoice{the number of photons hitting the metal in a given time interval}
        \wrongchoice{the energy of photons hitting the metal}
        \wrongchoice{whether or not photoelectrons will be emitted}
        \wrongchoice{$KE_{max}$ of emitted photoelectrons}
    \end{choices}
\end{question}
}

\element{holt-mc}{
\begin{question}{ch21-mc-Q04}
    An X-ray photon is scattered by a stationar electron.
    How does the frequency of this scattered photon compare to its frequency before being scattered?
    \begin{choices}
        \wrongchoice{The new frequency is higher.}
        \wrongchoice{The new frequency is lower.}
        \wrongchoice{The frequency stays the same.}
        \wrongchoice{The scattered photon has no frequency.}
    \end{choices}
\end{question}
}

\element{holt-mc}{
\begin{question}{ch21-mc-Q05}
    Which of the following summarizes Thomson's model of the atom?
    \begin{choices}
        \wrongchoice{Atoms are hard, uniform, indestructable spheres.}
        \wrongchoice{Electrons are embedded in a sphere of positive charge.}
        \wrongchoice{Electrons orbit the nucleus in the same way that planets orbit the sun.}
        \wrongchoice{Electrons exist only at discrete energy levels.}
    \end{choices}
\end{question}
}

\element{holt-mc}{
\begin{question}{ch21-mc-Q06}
    What happens when an electron moves from a higher energy level to a lower energy level in an atom?
    \begin{choices}
        \wrongchoice{Energy is absorbed from a source outside the atom.}
        \wrongchoice{The energy contained in the electromagnetic field inside the atom increases.}
        \wrongchoice{Energy is released across a continuous range of values.}
        \wrongchoice{A photon is emitted with energy equal to the difference in energy between the two levels.}
    \end{choices}
\end{question}
}

\element{holt-mc}{
\begin{question}{ch21-mc-Q07}
    \begin{center}
    \begin{tikzpicture}
    \end{tikzpicture}
    \end{center}
    What is the frequency of the photon emitted when an electron jumps from $E_5$ to $E_2$?
    \begin{multicols}{2}
    \begin{choices}
        \wrongchoice{\SI{2.86}{\eV}}
        \wrongchoice{\SI{6.15e14}{\hertz}}
        \wrongchoice{\SI{6.90e14}{\hertz}}
        \wrongchoice{\SI{4.31e33}{\hertz}}
    \end{choices}
    \end{multicols}
\end{question}
}

\element{holt-mc}{
\begin{question}{ch21-mc-Q08}
    \begin{center}
    \begin{tikzpicture}
    \end{tikzpicture}
    \end{center}
    What is the frequency of the photon emitted when an electron jumps from $E_2$ to $E_3$?
    \begin{multicols}{2}
    \begin{choices}
        \wrongchoice{\SI{1.89}{\eV}}
        \wrongchoice{\SI{4.56e14}{\hertz}}
        \wrongchoice{\SI{6.89e14}{\hertz}}
        \wrongchoice{\SI{2.85e33}{\hertz}}
    \end{choices}
    \end{multicols}
\end{question}
}

\element{holt-mc}{
\begin{question}{ch21-mc-Q09}
    \begin{center}
    \begin{tikzpicture}
    \end{tikzpicture}
    \end{center}
    What type of spectrum is created by appliying a high potential difference to a pure atomic gas?
    \begin{choices}
        \wrongchoice{an emission spectrum}
        \wrongchoice{an absorption spectrum}
        \wrongchoice{a continuous spectrum}
        \wrongchoice{a visible spectrum}
    \end{choices}
\end{question}
}

\element{holt-mc}{
\begin{question}{ch21-mc-Q10}
    What type of spectrum is used to identify elements in the atmosphere of stars?
    \begin{choices}
        \wrongchoice{an emission spectrum}
        \wrongchoice{an absorption spectrum}
        \wrongchoice{a continuous spectrum}
        \wrongchoice{a visible spectrum}
    \end{choices}
\end{question}
}

\element{holt-mc}{
\begin{question}{ch21-mc-Q11}
    What is the speed of a proton ($m=\SI{1.67e-27}{\kilo\gram}$) with a de Broglie wavelength of \SI{4.00e-14}{\meter}?
    \begin{multicols}{2}
    \begin{choices}
        \wrongchoice{\SI{1.59e-30}{\meter\per\second}}
        \wrongchoice{\SI{1.01e-7}{\meter\per\second}}
        \wrongchoice{\SI{9.93e6}{\meter\per\second}}
        \wrongchoice{\SI{1.01e7}{\meter\per\second}}
    \end{choices}
    \end{multicols}
\end{question}
}

\element{holt-mc}{
\begin{question}{ch21-mc-Q12}
    What does Heisenberg's uncertainty principle state?
    \begin{choices}
        \wrongchoice{It is important to simultaneously measure a particle's position and momentum with infinite accuracy.}
        \wrongchoice{It is important to measure both a particle's position and its momentum.}
        \wrongchoice{The more accurately we know a particle's position, the more accuractely we know the particle's momentum.}
        \wrongchoice{All measurements are uncertain.}
    \end{choices}
\end{question}
}


\endinput

