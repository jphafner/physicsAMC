
%% http://www.sosmath.com/algebra/solve/solve0/solve0.html

%% Linear Equations

%% Equations constainting radicals

%% Equations constainting absolute values

%% Quadratic equations

%% Equations involving fractions

%% Exponential equations

%% Logarithmic equations

%% Trigonometric equations

%% Linear Equatoins: solve for x

%% The international System of Units
%%--------------------------------------------------

%% Solve the following equations for the variable indicated.
%% There should be enough room to do one step at a time
%%------------------------------------------------------------
\element{SI}{
\begin{questionmult}{SI-Q01}
\luaexec{
    %% Question
    local Q = [[
        Solve the following equations for the variable indicated.
        $v=\frac{x}{t}$, find $t$.
    ]]
    %% Random Permutations
    local tab1 = {}
    for i=1,22 do
        tab1[i] = i
    end
    local tab2 = {}
    for i=1,7 do
        tab2[i] = i
    end
    tab1 = permute(tab1,22,22)
    tab2 = permute(tab2,7,7)
    %% Random correct vs wrong
    local n1 = math.random(1,3)
    local n2 = 4 - n1
    %% Print Question
    tex.print( Q )
    %% Print MC Options
    tex.print( BeginMulticols )
        tex.print( BeginChoices )
            for i=1,n2 do
                tex.print( string.format(CorrectChoice,defined[tab2[i]]) )
            end
            for i=1,n1 do
                tex.print( string.format(WrongChoice,derived[tab1[i]]) )
            end
        tex.print( EndChoices )
    tex.print( EndMulticols )
}
\end{questionmult}
}


%% Evaluate the following using the information given.
%% Try algebraically solving for the unknown variable first
%%------------------------------------------------------------
\element{SI}{
\begin{questionmult}{SI-Q02}
\luaexec{
    %% Question
    local Q = [[
        Evaluate the following using the information given:
        $v_f = v_i + a t$, find $a$ given $v_i=2$, $v_f=4$ and $t=2$.
    ]]
    %% Random Permutations
    local tab1 = {}
    for i=1,22 do
        tab1[i] = i
    end
    local tab2 = {}
    for i=1,7 do
        tab2[i] = i
    end
    tab1 = permute(tab1,22,22)
    tab2 = permute(tab2,7,7)
    %% Random correct vs wrong
    local n1 = math.random(1,3)
    local n2 = 4 - n1
    %% Print Question
    tex.print( Q )
    %% Print MC Options
    tex.print( BeginMulticols )
        tex.print( BeginChoices )
            for i=1,n1 do
                tex.print( string.format(CorrectChoice,derived[tab1[i]]) )
            end
            for i=1,n2 do
                tex.print( string.format(WrongChoice,defined[tab2[i]]) )
            end
        tex.print( EndChoices )
    tex.print( EndMulticols )
}
\end{questionmult}
}

