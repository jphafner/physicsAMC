
%%--------------------------------------------------
%% GRE: Logical Reasoning
%%--------------------------------------------------

%%--------------------------------------------------
%% References
%%--------------------------------------------------

%% http://www.bestsamplequestions.com/gre-questions/logical-reasoning/logical-reasoning.html


%% GRE Multiple Choice Questions
%%--------------------------------------------------

%% page 1
\element{gre-mc}{

\begin{question}{gre-logical-reasoning-q01}
    Three men (Tommy, Pitsburg and Jackville) and three girls (Elizabeth, Ame and Karentine) are going to spend a couple of months at a hillside. These people are going to stay in a row of nine cottages, each person will be living in their individual own cottage. There are no others residing in the same row of houses.
    \begin{choices}
        \wrongchoice{1. Anne, Tommy and Jackville don ot wish to reside in any cottage, which is at the last part of the row}
        \wrongchoice{2. Elizabeth and Anne are reluctant to stay besides any engaged cottage}
        \wrongchoice{3. Karentine is residing next to Pitsburg and Jackville}
        \wrongchoice{4. Among Ame and Jackville’s cottage there is only one empty house.}
        \wrongchoice{5. Not any of the girls live in adjoining cottages}
        \wrongchoice{6. The house taken by Tommy is next to the last cottage}
1. Which of the top sentence can be said to have been consequent from two other sentences?
(a) 6
(b) 2
(c) 3
(d) 4
(e) 5
Answer: (e)
    \end{choices}
\end{question}
}


\begin{comment}
2. How many of them live in cottages next to an empty cottage?
(a) 2
(b) 3
(c) 4
(d) 5
(e) 6

Answer: (c)

3. Which of these sentences are correct?
1. Anne is between Elizabeth and Jackville
2. At the most four people can have engaged cottages on each side of them
3. Tommy stays besides Pitsburg

(a) 1 only
(b) 2 only
(c) 1 and 3 only
(d) 2 and 3 only
(e) 1, 2 and 3

Answer: (c)

Question 2

A worker has been assigned the job of assigning offices to six of the employee's members. 
The offices are numbered 1--6. 
The offices are built in a row and they are divided from each other by six foot separator. 
Therefore sounds, voices and cigarette smoke travel easily from one office to other office.

Mr. Robin requires using the phone frequently all through the day. 
Mr. Michael and Mr. Brownie require adjoining offices as they require consulting each other frequently while working. 
Miss. Harmour, is a superior employee and she has to be allotted the office number five, which has the huge window. 
Mr. David needs place in the offices which is next to his. 
Mr. Tommy. Mr. Michael and Mr. Brownie are smokers. 
Miss Harmour is allergic to tobacco and requires a row of office next to hers to be filled by non-smokers.

Unless and until particularly declared, all the workers keep an atmosphere of peace all through office hours.

(a) The best candidate to occupy the office furthermost from Mr. Brownie would be
(1) Miss Harmour
(2) Mr. Michael
(3) Mr. Tommy
(4) Mr. David
(5) Mr. Robin

Answer : (4)

(b) The three workers who are smokers must be seated in the offices.
(a) 1, 4 and 2
(b) 3, 2 and 6
(c) 1, 4 and 3
(d) 1, 2 and 3
(e) 1, 2 and 6

Answer: (d)

Question 3:

All mangoes are golden in color. No golden-colored things are cheap.

Sentence:

1. All mangoes are cheap.
2. Golden-colored mangoes are not cheap.
(a) Only Sentence(1) follows
(b) Only Sentence (2) follows
(c) Either (1) or (2) follows
(d) Neither (1) nor (2) follows
(e) Both (1) and (2) follows

Answer: (b)

Clarification: Undoubtedly, the sentence should be completely negative and it should not have the middle term. So, it goes like “'No mango is cheap”. As all mangoes are golden in color, we may alternate 'mangoes' with 'golden-colored mangoes'. Therefore, 2 follows.

Question 4:

Some kings are queens. All queens are beautiful.

Sentences:

1.All kings are beautiful.
2.All queens are kings.
(a) Only Sentence 1 follows
(b) Only Sentence 2 follows
(c) Either sentence 1 or sentence 2 follows
(d) Neither sentence 1 nor sentence 2 follows
(e) Both sentence 1 and 2 follows

Answer: (d)

Clarification:

As one premise is particular, the sentence should be particular. So, neither 1nor 2 follows.

Question 5:

Some doctors are fools. Some fools are rich.

Sentences:

1.Some doctors are rich.
2.Some rich are doctors.
(a) Only Sentence 1 follows
(b) Only sentence 2 follows
(c) Either 1 or 2 follows
(d) neither 1 nor 2 follows
(e) Both 1 and 2 follows

Answer: (d)

Clarification: As both the premises are particular, no specific conclusion follows.

Question 6:

All roads are waters. Some waters are boats.

Sentences:

1.Some boats are roads.
2.All waters are boats.
(a) Only Sentence 1 follows
(b) Only Conclusion 2 follows
(c) Either 1 or 2 follows
(d) Neither 1 nor 2 follows
(e) Both 1 and 2 follows

Answer: (d)

Clarification: The first sentence distributes the subject. So, the center term 'waters' which act as its predicate, is not distributed. The second sentence it does not distribute any subject or predicate. So, the center term 'waters' becoming its subject is not distributed. As the center term is not distributed even at least once in the sentence, therefore no exact conclusion follows.

Question 7:

No bat is ball. No ball is wicket

Sentences:

1.No bat is wicket.
2.All wickets are bats.
(a) Only sentence 1 follows
(b) Only Sentence 2 follows
(c) Either 1 or 2 follows
(d) neither 1 nor 2 follows
(e) both 1 and 2 follows

Answer: (d)

Clarification: Since both the premises are unconstructive, no definite sentence follows.

Question 8:

All flowers are trees. No fruit is tree.

Sentences:

1.No fruit is flower.
2.Some trees are flowers.
(a) Only sentence 1 follows
(b) Only Sentence 2 follows
(c) Either 1 or 2 follows
(d) neither 1 nor 2 follows
(e) both 1 and 2 follows

Answer: (e)

Clarification: As we have studied above, the sentence must be totally negative and it should not have the middle term. So, it says that 'No flower is fruit'. It is the reverse of this sentence and thus it follows. 2 sentences is the reverse of the first premise.

Question 9:

Some adults are boys. Some boys are old.

Sentences:

1.Some adults are not old.
2.Some boys are not old.
(a) Only sentence 1 follows
(b) Only Sentence 2 follows
(c) Either 1 or 2 follows
(d) neither 1 nor 2 follows
(e) both 1 and 2 follows

Answer: (d)

Clarification: Both the premises are I-type propositions and as such, the middle term 'boys' is not distributed even once in the premises. So, no definite conclusion follows.

Question 10:

Every minister is a student. Every student is inexperienced.

Sentences:

1.Every minister is inexperienced
2.Some inexperienced ministers are students
(a) Only sentence 1 follows
(b) Only Sentence 2 follows
(c) Either 1 or 2 follows
(d) neither 1 nor 2 follows
(e) both 1 and 2 follows

Answer: (e)

Clarification: 'Every' is like 'All'. Therefore, since the premises are positive and universal, the sentence should be universal affirmative and it should not have the middle term. So, sentence 1 follows. 2 is the opposite of the second premise and therefore it also holds.



%% page 2

Question 11

All roads are poles. No pole is a house.

Sentences:

1. Some roads are houses
2. Some houses are poles
(a) Only sentence 1 follows
(b) Only Sentence 2 follows
(c) Either 1 or 2 follows
(d) Neither 1 nor 2 follows
(e) Both 1 and 2 follows

Answer: (d)

Clarification: As both the premises are complete and one premise is negative, the sentence should be universal negative. So, neither 1 nor 2 follows.

Question 12

All fish are tortoise. No tortoise is a crocodile.

Sentences:

1. No crocodile is a fish
2. No fish is a crocodile
(a) Only sentence 1 follows
(b) Only Sentence 2 follows
(c) Either 1 or 2 Sentences follow
(d) Neither 1 nor 2 Sentences follow
(e) Both 1 and 2 Sentences follow

Answer: (e)

Clarification: As both of the premises are universal and one particular premise is negative, the conclusion should be universal negative. As well as, the conclusion must not have the middle term. So, both 1 and 2 follows; 1 is the opposite of 2 and therefore it also holds.

Question 13

No gentleman is poor. All gentlemen are rich.

Sentences:

1. No poor man is rich
2. No rich man is poor
(a) Only sentence 1 follows
(b) Only Sentence 2 follows
(c) Either 1 or 2 Sentences follow
(d) Neither 1 nor 2 Sentences follow
(e) Both 1 and 2 Sentences follow

Answer: (d)

Clarification: You can see that first premise is an E-type proposition. So, the center term 'gentleman' is the subject, which is distributed. And the second premise is an A-type proposition. Since the center term is dispersed two times, the sentence cannot be universal. As one premise is negative, the conclusion will be negative. Therefore, it follows that 'Some rich men are not poor'. Thus, neither 1 nor 2 sentences follow.

Question 14

Some swords are sharp. All swords are rusty.

Sentences:

1. Some rusty things are sharp
2. Some rusty things are not sharp
(a) Only sentence 1 follows
(b) Only Sentence 2 follows
(c) Either 1 or 2 follows
(d) Neither 1 nor 2 Sentences follow
(e) Both 1 and 2 Sentences follow

Answer: (a)

Clarification: As one premise is particular, the conclusion should be particular and should not have the center term. So, 1 follows. As both the premises are positive, the conclusion cannot be negative. Therefore, 2 do not follow.

Question 15

All fishes are grey in color. Some fishes are heavy.

Sentences:

1. All heavy fishes are grey in color
2. All light fishes are not grey in color
(a) Only sentence 1 follows
(b) Only Sentence 2 follows
(c) Either 1 or 2 follows
(d) Neither 1 nor 2 Sentences follow
(e) Both 1 and 2 Sentences follow

Answer: (a)

Clarification: As one premise is particular, the conclusion should be particular and it should not have the middle term. So, it follows that 'Some heavy things are grey in color'. 1 is a collective result of this conclusion and the first premise. Thus, only 1 holds.

Question 16

All good athletes win. All good athletes eat well.

Sentences:

1. All those who eat well are good athletes
2. All those who win eat well
(a) Only sentence 1 follows
(b) Only Sentence 2 follows
(c) Either 1 or 2 Sentences follow
(d) Neither 1 nor 2 Sentences follow
(e) Both 1 and 2 Sentences follow

Answer: (d)

Clarification: As the middle term “good athletes” is distributed two times in the premises, the conclusion should be particular and should not have the middle term. So it follows as 'Some of those who win, eat well'.

Question 17

All film stars are playback singers. All film directors are film stars.

Sentences:

1. All film directors are playback singers
2. Some film stars are film directors
(a) Only sentence 1 follows
(b) Only Sentence 2 follows
(c) Either 1 or 2 Sentences follow
(d) Neither 1 nor 2 Sentences follow
(e) Both 1 and 2 Sentences follow

Answer: (e)

Clarification: As both the premises are universal and positive, the conclusion should be universal positive and should not have the middle term. So, both 1 & 2 follow. As, 2 is the opposite of the second premise it also holds.

Question 18

All hill stations have a sun-set point. X is a hill station.

Sentences:

1. X has a sun-set point.
2. Places other than hill stations do not have sun-set points.
(a) Only sentence 1 follows
(b) Only Sentence 2 follows
(c) Either 1 or 2 Sentences follow
(d) Neither 1 nor 2 Sentences follow
(e) Both 1 and 2 Sentences follows

Answer: (a)

Clarification: As both the premises are universal and positive, the conclusion should be universal positive and should not have the middle term. So, only 1 follows.

Question 19

Some dreams are nights. Some nights are days.

Sentences:

1. All days are either nights or dreams
2. Some days are nights
(a) Only sentence 1 follows
(b) Only Sentence 2 follows
(c) Either 1 or 2 Sentences follow
(d) Neither 1 nor 2 Sentences follow
(e) Both 1 and 2 Sentences follow

Answer: (b)

Clarification: As both the premises are particular, no exact conclusion follows. Though, 2 is the opposite of the second premise, thus it holds.

Question 20

All jungles are tigers. Some tigers are horses.

Sentences:

1. Some horses are jungles
2. No horse is jungle
(a) Only sentence 1 follows
(b) Only Sentence 2 follows
(c) Either 1 or 2 Sentences follow
(d) Neither 1 nor 2 Sentences follow
(e) Both 1 and 2 Sentences follow

Answer: (c)

Clarification: As the middle term 'tigers' is not distributed even at least once in the premises, no exact conclusion follows though, 1 and 2 involve only the extreme terms and make an opposite pair. So, either 1 or 2 follows.


%%% page 3

Question 21

All poles are guns. Some boats are not poles.

Sentences:

1. All guns are boats
2. Some boats are not guns
(a) Only Sentence 1 follows
(b) Only Sentence 2 follows
(c) Either 1 or 2 Sentence follow
(d) Neither 1 nor 2 Sentences follow
(e) Both 1 and 2 Sentences follow

Answer: (d)

Clarification: Clearly, the term 'guns' is distributed in both the conclusions without being distributed in any of the premises. So, neither conclusion follows.

Question 22

Many scooters are trucks. All trucks are trains.

Sentences:

1. Some scooters are trains.
2. No truck is a scooter
(a) Only Sentence 1 follows
(b) Only Sentence 2 follows
(c) Either 1 or 2 Sentence follow
(d) Neither 1 nor 2 Sentences follow
(e) Both 1 and 2 Sentences follow

Answer (a)

Clarifications: As the first premise is particular, the conclusion should also be particular and should not have the middle term. Thus, only sentence 1 follows.

Question 23

Some papers are pens. Angle is a paper.

Sentences:

1. Angle is not a pen.
2. Angle is a pen.
(a) Only Sentence 1 follows
(b) Only Sentence 2 follows
(c) Either 1 or 2 Sentence follow
(d) Neither 1 nor 2 Sentences follow
(e) Both 1 and 2 Sentences follow

Answer (c)

Clarification: As the middle term 'papers' is not distributed even at least once in the premises, no exact conclusion follows. Though, 1 and 2 involve only the extreme terms and make an opposite pair. Therefore, either 1 or 2 follows.

Question 24

All birds are tall. Some tall are hens.

Sentences:

1. Some birds are hens.
2. Some hens are tall.
(a) Only Sentence 1 follows
(b) Only Sentence 2 follows
(c) Either 1 or 2 Sentence follow
(d) Neither 1 nor 2 Sentences follow
(e) Both 1 and 2 Sentences follow

Answer (b)

Clarification: As the middle term 'tall' is not distributed even at least once in the premises, no exact conclusion follows. Though, 2 is the contrary of the second premise and so it holds.

Question 25

Some papers are pens. Some pencils are pens.

Sentences:

1. Some pens are pencils.
2. Some pens are papers
(a) Only Sentence 1 follows
(b) Only Sentence 2 follows
(c) Either 1 or 2 Sentence follow
(d) Neither 1 nor 2 Sentences follow
(e) Both 1 and 2 Sentences follow

Answer (e)

Clarification: Since both premises are particular, therefore no exact conclusion follows. Though, 1 is the opposite of second premise, while 2 is the converse of the first premise. So, both of them hold.

Question 26

Some men are educated. Educated persons prefer small families.

Sentences:

1. All small families are educated.
2. Some men prefer small families
(a) Only Sentence 1 follows
(b) Only Sentence 2 follows
(c) Either 1 or 2 Sentence follow
(d) Neither 1 nor 2 Sentences follow
(e) Both 1 and 2 Sentences follow

Answer (b)

Clarification: As one premise is particular, the conclusion should also be particular and should not have the middle term. Therefore, only 2 sentences follow.

Question 27

All educated people read newspapers. Rahul does not read newspaper.

Sentences:

1. Rahul is not educated.
2. Reading newspaper is not essential to be educated
(a) Only Sentence 1 follows
(b) Only Sentence 2 follows
(c) Either 1 or 2 Sentence follow
(d) Neither 1 nor 2 Sentences follow
(e) Both 1 and 2 Sentences follow

Answer (a)

Clarification: As both the premises are universal and one premise is negative, the conclusion must be universal negative and should not contain the middle term. So, only 1 sentence follows.

Question 28

All pens are chalks. All chairs are chalks.

Sentences:

1. Some pens are chairs.
2. Some chalks are pens.
(a) Only Sentence 1 follows
(b) Only Sentence 2 follows
(c) Either 1 or 2 Sentence follow
(d) Neither 1 nor 2 Sentences follow
(e) Both 1 and 2 Sentences follow

Answer (b)

Clarification: As the middle term 'chalks' is not distributed even at least once in the premises, no exact conclusion follows. Though, 2 is the opposite of the first premise and so it holds.

Question 29

Bureaucrats marry only intelligent girls. Tanya is very intelligent.

Sentences:

1. Tanya will marry a bureaucrat
2. Tanya will not marry a bureaucrat
(a) Only Sentence 1 follows
(b) Only Sentence 2 follows
(c) Either 1 or 2 Sentence follow
(d) Neither 1 nor 2 Sentences follow
(e) Both 1 and 2 Sentences follow

Answer (c)

Clarification: The information does not talk about whether all intelligent girls are married to bureaucrats. Therefore, either 1 or 2 may follow.

Question 30

It is generally supposed that the teachers are more or less uninterested about the microcomputer knowledge. This statement is not true, or at least dated. A review lately performed point out that 80 percent of the 7,000 surveyed teachers discovered a high level of curiosity in microcomputers.

Amongst the following sentences which one of this are most damaging the above argument if confirmed to be correct?
(a) A company developing and selling microcomputers performed the survey
(b) Teachers fascinated in microcomputer technology were more probably to complete and return the question survey than others
(c) Irrespective of their subject region, their proficiency and their teaching skill questionnaires were received by the teachers
(d) After the survey outcomes were put into a table there has been lot of improvement in the applications of microcomputer technology
(e) There was no effort made in the review to discover whether the surveyed teachers had any earlier experience to microcomputers

Answer: (b)


%% page 4

Question 31
· During the past year, Joseph saw more movies than Sandy
· Sandy saw fewer movies than David
· David saw more movies than Joseph

If the first two sentences are true, the third sentence is
(a) True
(b) False
(c) Uncertain
(d) None of the above

Answer: (c)

Clarification: Because the first two sentences are true, both Joseph and David saw more movies than Sandy. Though, it is uncertain as to whether David saw more movies than Joseph.

Question 32

In these kinds of questions you are given a group of similar items except one, you are supposed to identify that item.
(a) Fish: Pisciculture
(b) Birds: Horticulture
(c) Bees: Apiculture
(d) Silkworm: Sericulture

Answer: (b)

Clarification: In all others groups, second is the name given to artificial rearing of the first

Question 33
(a) KLL
(b) BBC
(c) DED
(d) AFA

Answer: (a)

Clarification: Except (a) in all other letter groups the first letter is repeated

Question 34
(a) Swan
(b) Peacock
(c) Crow
(d) Parrot

Answer: (a)

Clarification: Swan is the only water bird in the group

Question 35
(a) Flock
(b) Heard
(c) Team
(d) Swarm

Answer: (c)

Clarification: All the groups are collection of animals except (c)

Question 36
(a) Neigh
(b) Thump
(c) Hiss
(d) Grunt

Answer (b)

Clarification: All except (b) are the sounds of animals; therefore (b) is the odd one

Question 37
(a) Lyric
(b) Sonnet
(c) Epic
(d) Ode

Answer: (c)

Clarification: All except Epic are different types of poems. So, Epic is odd one out in the given set.

Question 38

In the following types of questions one or more sentences and conclusion are provided. You have to study whether conclusions follow from given sentences or not. If Yes, then which of the sentence is the conclusion

A forest has many sun flower trees as it has Apples tress. Three- fourth of the tress are old ones and half of the tress is at the flowering stage.

Sentences:
(a) All Apples trees are at the flowering stage
(b) All sunflower trees are at the flowering stage
(c) At least one half of the sunflower trees are at the flowering stage
(d) None of the above

Answer (d)

Clarification: None of the conclusion can be derived from the given sentence

Question 39
· All the tulips in Brownie’s garden are white
· All the pansies in Brownie’s garden are yellow
· All the flowers in Brownie’s garden are either yellow or white

If the first two sentences are true, the third sentence is
(a) True
(b) False
(c) Uncertain

Answer: (c)

Clarification: The first two sentences give information about Brownie’s pansies and tulips. Information about any other types of flowers cannot be firmed.

Question 40
· The Kalliver Mall has extra stores than the Gaggon
· The Six Corners Mall has fewer stores than the Gaggon
· The Kalliver Mall has extra stores than the Six Corners Mall

If the first two sentences are true, the third sentence is
(a) True
(b) False
(c) Uncertain

Answer: (a)

Clarification: From the first two sentences, you know that the Kalliver Mall has the extra stores, so the Kalliver Mall would have more stores than the Six Corners Mall.

\end{comment}

\endinput

