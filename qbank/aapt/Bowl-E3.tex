

%% AAPT Physics Bowl Exams Questions
%%----------------------------------------


%% This section has XX problems


%% PhysicsBowl 2015
%%----------------------------------------
\element{aapt}{ %% Bowl-E3
\begin{question}{bowl-2015-q49}
    The kinetic energy associated with an electron is twice its rest energy.
    At what speed is the electron traveling?
    \begin{multicols}{2}
    \begin{choices}
      \correctchoice{\SI{2.83e8}{\meter\per\second}}
        \wrongchoice{\SI{2.67e8}{\meter\per\second}}
        \wrongchoice{\SI{2.60e8}{\meter\per\second}}
        \wrongchoice{\SI{2.25e8}{\meter\per\second}}
        \wrongchoice{\SI{2.12e8}{\meter\per\second}}
    \end{choices}
    \end{multicols}
\end{question}
}


%% PhysicsBowl 2014
%%----------------------------------------
\element{aapt}{ %% Bowl-E3
\begin{question}{bowl-2014-q50}
    Two clocks, $A$ and $B$, are synchronized on Earth.
    Clock $A$ is placed onto a space ship that leaves Earth in a straight line with speed of \SI{2.40e8}{\meter\per\second}.
    On Earth, a scientist with clock $B$ has her telescope fixed directly on clock $A$.
    If each clock started at $t=\SI{0}{\second}$,
        what time does the scientist observe on clock $A$ when clock $B$ reads $t=\SI{90}{\second}$?
    Assume the time of acceleration for the ship leaving the Earth was negligible.
    \begin{multicols}{3}
    \begin{choices}
        \wrongchoice{\SI{24}{\second}}
      \correctchoice{\SI{30}{\second}}
        \wrongchoice{\SI{50}{\second}}
        \wrongchoice{\SI{54}{\second}}
        \wrongchoice{\SI{72}{\second}}
    \end{choices}
    \end{multicols}
\end{question}
}


%% PhysicsBowl 2013
%%----------------------------------------
\element{aapt}{ %% Bowl-E3
\begin{question}{bowl-2013-q49}
    Two electrons move with the magnitude of their linear momentum having a ratio of $2:1$.
    If the slower electron moves with speed of \SI{1.2e8}{\meter\per\second},
        what is the speed of the faster moving electron?
    \begin{multicols}{2}
    \begin{choices}
        \wrongchoice{\SI{2.67e8}{\meter\per\second}}
        \wrongchoice{\SI{2.40e8}{\meter\per\second}}
        \wrongchoice{\SI{2.24e8}{\meter\per\second}}
      \correctchoice{\SI{1.97e8}{\meter\per\second}}
        \wrongchoice{\SI{1.56e8}{\meter\per\second}}
    \end{choices}
    \end{multicols}
\end{question}
}


%% PhysicsBowl 2012
%%----------------------------------------
\element{aapt}{ %% Bowl-E3
\begin{question}{bowl-2012-q45}
    A stationary atom of mass \SI{4.00e-26}{\kilo\gram} spontaneously
        emits a photon of energy \SI{10.0}{\eV}.
    Which one of the following choices best represents the speed
        of the atom after emitting the photon?
    \begin{multicols}{2}
    \begin{choices}
        \wrongchoice{\SI{4.00e7}{\meter\per\second}}
        \wrongchoice{\SI{8.97e3}{\meter\per\second}}
      \correctchoice{\SI{1.33e-1}{\meter\per\second}}
        \wrongchoice{\SI{1.58e-4}{\meter\per\second}}
        \wrongchoice{\SI{2.50e-8}{\meter\per\second}}
    \end{choices}
    \end{multicols}
\end{question}
}


%% PhysicsBowl 2011
%%----------------------------------------
\element{aapt}{ %% Bowl-E3
\begin{question}{bowl-2011-q02}
    Albert Einstein's most famous equation $E=mc^2$.
    The unit for the quantity represented by $E$ can
        be written as which of the following options?
    \begin{multicols}{2}
    \begin{choices}
        \wrongchoice{second (\si{\second})}
        \wrongchoice{newton (\si{\newton})}
        \wrongchoice{kilogram (\si{\kilo\gram})}
        \wrongchoice{meter (\si{\meter})}
      \correctchoice{joule (\si{\joule})}
    \end{choices}
    \end{multicols}
\end{question}
}

\element{aapt}{ %% Bowl-E3
\begin{question}{bowl-2011-q50}
    A person sets a one-meter long stick so that is makes a \ang{30} angle with the $x$-axis.
    An observer in a space ship moving along the $x$-axis measures the stick to be \ang{60} with the $x$-axis.
    With what speed is the space ship moving in terms of the speed of light, $c$?
    \begin{multicols}{3}
    \begin{choices}
        \wrongchoice{$\dfrac{1}{2}c$}
        \wrongchoice{$\dfrac{\sqrt{2}}{3}c$}
        \wrongchoice{$\dfrac{3}{4}c$}
        \wrongchoice{$\dfrac{2}{\sqrt{5}}c$}
      \correctchoice{$\dfrac{2\sqrt{2}}{3}c$}
    \end{choices}
    \end{multicols}
\end{question}
}


%% PhysicsBowl 2010
%%----------------------------------------
\element{aapt}{ %% Bowl-E3
\begin{question}{bowl-2010-q50}
    An object of mass $m$ is initially at rest.
    After this object is accelerated to a speed of \SI{2.40e8}{\meter\per\second},
        it collides with and sticks to a second object of mass $m$ at rest.
    Immediately after the collision what is the common speed of the two masses?
    \begin{multicols}{2}
    \begin{choices}
        \wrongchoice{\SI{2.25e8}{\meter\per\second}}
        \wrongchoice{\SI{1.85e8}{\meter\per\second}}
        \wrongchoice{\SI{1.66e8}{\meter\per\second}}
      \correctchoice{\SI{1.50e8}{\meter\per\second}}
        \wrongchoice{\SI{1.20e8}{\meter\per\second}}
    \end{choices}
    \end{multicols}
\end{question}
}


%% PhysicsBowl 2009
%%----------------------------------------
\element{aapt}{ %% Bowl-E3
\begin{question}{bowl-2009-q50}
    What is the magnitude of the linear momentum
        of an electron moving in a straight line if
        it has \SI{3.2e-13}{\joule} of kinetic energy?
    \begin{multicols}{2}
    \begin{choices}
        \wrongchoice{\SI{0}{\kilo\gram\meter\per\second}}
        \wrongchoice{\SI{2.6e-22}{\kilo\gram\meter\per\second}}
        \wrongchoice{\SI{7.6e-22}{\kilo\gram\meter\per\second}}
      \correctchoice{\SI{1.3e-21}{\kilo\gram\meter\per\second}}
        \wrongchoice{\SI{1.9e-12}{\kilo\gram\meter\per\second}}
    \end{choices}
    \end{multicols}
\end{question}
}


%% PhysicsBowl 2008
%%----------------------------------------
\element{aapt}{ %% Bowl-E3
\begin{question}{bowl-2008-q45}
    Electron \#1 moves with speed $\num{0.30}c$ where $c$ is the speed of light.
    Electron \#2 moves with speed $\num{0.60}c$.
    What is the ratio of the kinetic energy of electron \#2 to electron \#1?
    \begin{multicols}{3}
    \begin{choices}
        \wrongchoice{\num{1.19}}
        \wrongchoice{\num{1.32}}
        \wrongchoice{\num{2.00}}
        \wrongchoice{\num{4.00}}
      \correctchoice{\num{5.18}}
    \end{choices}
    \end{multicols}
\end{question}
}


%% PhysicsBowl 2007
%%----------------------------------------
\element{aapt}{ %% Bowl-E3
\begin{question}{bowl-2007-q47}
    Two spaceships travel along paths that are at right angles to each other.
    Each ship travels at $\num{0.60}c$ where $c$ is the speed of light in a vacuum according to a stationary observer.
    If one of the ships turns on a green laser and aims it at a right angle to the direction of its travel,
        with what speed does the other speed record the speed of the green light?
    \begin{multicols}{2}
    \begin{choices}
        \wrongchoice{$\num{0.40}c$}
        \wrongchoice{$\num{0.85}c$}
      \correctchoice{$\num{1.00}c$}
        \wrongchoice{$\num{1.17}c$}
        \wrongchoice{More information is required about the direction
            that the light is traveling in order to answer the question.}
    \end{choices}
    \end{multicols}
\end{question}
}

\element{aapt}{ %% Bowl-E3
\begin{question}{bowl-2007-q48}
    How fast must an observer more so that a stationary object appears to be one-half of its proper length?
    \begin{multicols}{3}
    \begin{choices}
        \wrongchoice{$\num{0.50}c$}
        \wrongchoice{$\num{0.67}c$}
        \wrongchoice{$\num{0.75}c$}
      \correctchoice{$\num{0.87}c$}
        \wrongchoice{$\num{0.93}c$}
    \end{choices}
    \end{multicols}
\end{question}
}



%% PhysicsBowl 2006
%%----------------------------------------
\element{aapt}{ %% Bowl-E3
\begin{question}{bowl-2006-q35}
    A scientist claims to have perfected a technique in which he can spontaneously convert an electron completely into energy in the laboratory without any other material required. 
    What is the conclusion about this claim from our current understanding of physics?
    \begin{choices}
        \wrongchoice{This is possible because Einstein's equation says that mass and energy are equivalent\ldots it is just very difficult to achieve with electrons}
        \wrongchoice{This is possible and it is done all the time in the high-energy physics labs.}
        \wrongchoice{The scientist is almost correct\ldots except that in converting the electron to energy, an electron's anti-particle is produced in the process as well.}
        \wrongchoice{The scientist is almost correct\ldots except that in converting the electron to energy, a proton is produced in the process as well.}
      \correctchoice{This is not possible because charge conservation would be violated.}
    \end{choices}
\end{question}
}


%% PhysicsBowl 1997
%%----------------------------------------
\element{aapt}{ %% Bowl-E3
\begin{question}{bowl-1997-q38}
    A vehicle traveling at a constant $\num{0.60}\,c$ passes a clock at a street intersection.
    For a \SI{1}{\second} interval on that clock,
        what interval will the driver of the vehicle measure?
    \begin{multicols}{3}
    \begin{choices}
        \wrongchoice{\SI{0.60}{\second}}
        \wrongchoice{\SI{0.80}{\second}}
        \wrongchoice{\SI{1}{\second}}
      \correctchoice{\SI{1.25}{\second}}
        \wrongchoice{\SI{1.67}{\second}}
    \end{choices}
    \end{multicols}
\end{question}
}


%% PhysicsBowl 1996
%%----------------------------------------
\element{aapt}{ %% Bowl-E3
\begin{question}{bowl-1996-q28}
    A photon with frequency $f$ behaves as if it had a mass equal to:
    \begin{multicols}{3}
    \begin{choices}
        \wrongchoice{$hfc^2$}
      \correctchoice{$\dfrac{hf}{c^2}$}
        \wrongchoice{$\dfrac{c^2}{hf}$}
        \wrongchoice{$\dfrac{fc^2}{h}$}
        \wrongchoice{$\dfrac{h}{fc^2}$}
    \end{choices}
    \end{multicols}
\end{question}
}


%% PhysicsBowl 1995
%%----------------------------------------
\element{aapt}{ %% Bowl-E3
\begin{question}{bowl-1995-q33}
    An atomic particle of mass $m$ moving at speed $v$ is found to have wavelength $\lambda$.
    What is the wavelength of a second particle with speed $3v$ and the same mass?
    \begin{multicols}{3}
    \begin{choices}
        \wrongchoice{$\frac{1}{9}\lambda$}
      \correctchoice{$\frac{1}{3}\lambda$}
        \wrongchoice{$\lambda$}
        \wrongchoice{$3\,\lambda$}
        \wrongchoice{$9\,\lambda$}
    \end{choices}
    \end{multicols}
\end{question}
}


\endinput


