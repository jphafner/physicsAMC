

%% AAPT Physics Bowl Exams Questions
%%----------------------------------------


%% This section has XX problems


%% PhysicsBowl 2015
%%----------------------------------------


%% PhysicsBowl 2014
%%----------------------------------------
\element{aapt}{ %% Bowl-E2
\begin{question}{bowl-2014-q41}
    ``No two electrons in an atom can have an identical set of
        the four quantum numbers.'' is a statement most closely
        associated with which of the following scientists?
    \begin{multicols}{2}
    \begin{choices}
        \wrongchoice{Albert Einstein}
        \wrongchoice{Enrico Fermi}
        \wrongchoice{Sheldon Cooper}
      \correctchoice{Wolfgang Pauli}
        \wrongchoice{Issac Newton}
    \end{choices}
    \end{multicols}
\end{question}
}


%% PhysicsBowl 2013
%%----------------------------------------
\element{aapt}{ %% Bowl-E2
\begin{question}{bowl-2013-q21}
    The following nuclear reaction occurs:
    \begin{equation}
        \ce{^{131}_{53}I -> ^{131}_{54}Xe + ^{A}_{Z}X}
    \end{equation}
    What is \ce{^{A}_{Z}X}?
    \begin{multicols}{2}
    \begin{choices}
        \wrongchoice{a neutron}
        \wrongchoice{a proton}
        \wrongchoice{a positron}
        \wrongchoice{an alpha particle}
      \correctchoice{an electron}
    \end{choices}
    \end{multicols}
\end{question}
}


%% PhysicsBowl 2012
%%----------------------------------------
\element{aapt}{ %% Bowl-E2
\begin{question}{bowl-2012-q20}
    ``Both the position and momentum of an electron cannot be known
        exactly at the same instant of time.''
    To whom is this concept attributed?
    \begin{multicols}{2}
    \begin{choices}
        %% NOTE: added first names
        \wrongchoice{Wolfgang Pauli}
        \wrongchoice{Louis de Broglie}
        \wrongchoice{Albert Einstein}
        \wrongchoice{Paul Dirac}
      \correctchoice{Werner Heisenberg}
    \end{choices}
    \end{multicols}
\end{question}
}

\element{aapt}{ %% Bowl-E2
\begin{question}{bowl-2012-q41}
    A hypothetical radioactive substance Aaptinium decays via alpha-emission
        into Physicbowlium.
    The decay constant for this alpha-emission is \SI{20}{\per\second}.
    Which one of the following statements correctly compares Physicbowlium
        to Aaptinium?
    \begin{choices}
        \wrongchoice{Physicsbowlium has 4 fewer protons and 2 fewer neutrons than Aaptinium}
        \wrongchoice{Physicsbowlium has 4 fewer neutron and 2 fewer protons than Aaptinium}
      \correctchoice{Physicsbowlium has 2 fewer protons and 2 fewer neutrons than Aaptinium}
        \wrongchoice{Physicsbowlium has 4 fewer protons than, and the same number of neutrons as, Aaptinium}
        \wrongchoice{Physicsbowlium has 2 fewer protons than, and the same number of neutrons as, Aaptinium}
    \end{choices}
\end{question}
}

\element{aapt}{ %% Bowl-E2
\begin{question}{bowl-2012-q42}
    A hypothetical radioactive substance Aaptinium decays via alpha-emission into Physicbowlium.
    The decay constant for this alpha-emission is \SI{20}{\per\second}.
    What is the half-life of Aaptinium?
    \begin{multicols}{3}
    \begin{choices}
      \correctchoice{\SI{0.035}{\second}}
        \wrongchoice{\SI{0.050}{\second}}
        \wrongchoice{\SI{0.100}{\second}}
        \wrongchoice{\SI{0.297}{\second}}
        \wrongchoice{\SI{0.693}{\second}}
    \end{choices}
    \end{multicols}
\end{question}
}


%% PhysicsBowl 2010
%%----------------------------------------
\element{aapt}{ %% Bowl-E2
\begin{question}{bowl-2010-q44}
    A radioactive sample of gas has a half-life of \SI{100}{\second}.
    If there are initially \num{10 000} of these gas molecules in a closed container,
        approximately how many of the molecules after a time of \SI{250}{\second} elapses?
    \begin{multicols}{3}
    \begin{choices}
        \wrongchoice{\num{2500}}
        \wrongchoice{\num{2190}}
      \correctchoice{\num{1770}}
        \wrongchoice{\num{1560}}
        \wrongchoice{\num{1250}}
    \end{choices}
    \end{multicols}
\end{question}
}

\element{aapt}{ %% Bowl-E2
\begin{question}{bowl-2010-q46}
    For the following nuclear reaction,
        what is the unknown labeled by $X$?
    \begin{equation*}
        ^{22}_{11}Na + X \rightarrow ^{22}_{10}Ne + \nu_e
    \end{equation*}
    \begin{multicols}{2}
    \begin{choices}
        \wrongchoice{A proton}
      \correctchoice{An electron}
        \wrongchoice{A neutron}
        \wrongchoice{An alpha particle}
        \wrongchoice{A positron}
    \end{choices}
    \end{multicols}
\end{question}
}


%% PhysicsBowl 2008
%%----------------------------------------
\element{aapt}{ %% Bowl-E2
\begin{question}{bowl-2008-q25}
    The following nuclear reaction occurs:
    \begin{displaymath}
        \ce{^{4}_2 He + ^{9}_{4} Be -> ^{12}_{4} C + ^{A}_{B} X} .
    \end{displaymath}
    What is \ce{^{A}_{Z} X}?
    \begin{multicols}{2}
    \begin{choices}
        \wrongchoice{a proton}
        \wrongchoice{an electron}
        \wrongchoice{a positron}
        \wrongchoice{an alpha particle}
      \correctchoice{a neutron}
    \end{choices}
    \end{multicols}
\end{question}
}

\element{aapt}{ %% Bowl-E2
\begin{question}{bowl-2008-q26}
    If the principle quantum number of an electron is $n=4$,
        how many possible values of the orbital magnetic quantum number $m_l$ are there for this electron?
    \begin{multicols}{3}
    \begin{choices}
        \wrongchoice{\num{3}}
        \wrongchoice{\num{4}}
      \correctchoice{\num{7}}
        \wrongchoice{\num{9}}
        \wrongchoice{\num{16}}
    \end{choices}
    \end{multicols}
\end{question}
}

\element{aapt}{ %% Bowl-E2
\begin{question}{bowl-2008-q47}
    A radioactive sample decays with a half-life of \SI{2.0}{\year}.
    Approximately how much time must pass so that only $\frac{1}{3}$ of the original sample remains?
    \begin{multicols}{3}
    \begin{choices}
        \wrongchoice{\SI{6.0}{\year}}
        \wrongchoice{\SI{3.4}{\year}}
      \correctchoice{\SI{3.2}{\year}}
        \wrongchoice{\SI{3.0}{\year}}
        \wrongchoice{\SI{2.8}{\year}}
    \end{choices}
    \end{multicols}
\end{question}
}


%% PhysicsBowl 2007
%%----------------------------------------
\element{aapt}{ %% Bowl-E2
\begin{question}{bowl-2007-q50}
    A gas undergoes radioactive decay with time constant $\tau$.
    A sample of \num{10 000} particles is put into a container.
    After one time constant has passed, the experimenter places another \num{10 000} particles into the original container.
    How much time passes from the addition of the particles until the container of gas reaches \num{10 000} particles again?
    \begin{multicols}{3}
    \begin{choices}
      \correctchoice{$\left(\num{0.405}\right)\tau$}
        \wrongchoice{$\left(\num{0.500}\right)\tau$}
        \wrongchoice{$\left(\num{0.693}\right)\tau$}
        \wrongchoice{$\tau$}
        \wrongchoice{$2\tau$}
    \end{choices}
    \end{multicols}
\end{question}
}


%% PhysicsBowl 2006
%%----------------------------------------
\element{aapt}{ %% Bowl-E2
\begin{question}{bowl-2006-q40}
    A new element,
        named Physicsbowlium (symbol \ce{Phys}) is discovered to undergo double alpha decay and beta decay simultaneously.
    Amazingly, this causes the material to decay into an element called Onlyonatestium (symbol \ce{Oo}).
    What is the correct representation of the (\ce{Oo})?
    \begin{equation*}
        \ce{^{2006}_{200}Phys -> ^{4}_{2}\alpha + ^{4}_{2}\alpha + ^{0}_{-1}e + ^{?}_{?}Oo }
    \end{equation*}
    \begin{multicols}{3}
    \begin{choices}
        \wrongchoice{\ce{^{1998}_{195}Oo}}
        \wrongchoice{\ce{^{2006}_{195}Oo}}
        \wrongchoice{\ce{^{1998}_{203}Oo}}
        \wrongchoice{\ce{^{2014}_{203}Oo}}
      \correctchoice{\ce{^{1998}_{197}Oo}}
    \end{choices}
    \end{multicols}
\end{question}
}


%% PhysicsBowl 2000
%%----------------------------------------
\element{aapt}{ %% Bowl-E2
\begin{question}{bowl-2000-q02}
    The following equation is an example of what kind of nuclear reaction?
    \begin{equation*}
        \ce{ ^{12}_{6}C + ^{4}_{2}He -> ^{16}_{8}O + Energy}
    \end{equation*}
    \begin{multicols}{2}
    \begin{choices}
        \wrongchoice{fission}
      \correctchoice{fussion}
        \wrongchoice{alpha decay}
        \wrongchoice{beta decay}
        \wrongchoice{positron decay}
    \end{choices}
    \end{multicols}
\end{question}
}


%% PhysicsBowl 1999
%%----------------------------------------
\element{aapt}{ %% Bowl-E2
\begin{question}{bowl-1999-q35}
    The following equation is an example of what kind of nuclear reaction?
    \begin{equation*}
        \ce{^{235}_{92}U + ^1_0 n -> ^{133}_{51}Sb + ^{99}_{41}Nb + 4 ^1_0n}
    \end{equation*}
    \begin{multicols}{2}
    \begin{choices}
      \correctchoice{fission}
        \wrongchoice{fussion}
        \wrongchoice{alpha decay}
        \wrongchoice{beta decay}
        \wrongchoice{positron decay}
    \end{choices}
    \end{multicols}
\end{question}
}


%% PhysicsBowl 1998
%%----------------------------------------
\element{aapt}{ %% Bowl-E2
\begin{question}{bowl-1998-q11}
    In the nuclear reaction below,
    \begin{equation*}
         \ce{^{6}_{3}Li + $?$ -> ^{7}_{3}Li} \, ,
    \end{equation*}
    the ``$?$'' represents:
    \begin{multicols}{2}
    \begin{choices}
        \wrongchoice{an alpha particle}
        \wrongchoice{a deuteron}
        \wrongchoice{an electron}
      \correctchoice{a neutron}
        \wrongchoice{a proton}
    \end{choices}
    \end{multicols}
\end{question}
}

\element{aapt}{ %% Bowl-E2
\begin{question}{bowl-1998-q12}
    An alpha particle is the same as:
    \begin{multicols}{2}
    \begin{choices}
      \correctchoice{a helium nucleus}
        \wrongchoice{a positron}
        \wrongchoice{an electron}
        \wrongchoice{a high energy photon}
        \wrongchoice{a deuteron}
    \end{choices}
    \end{multicols}
\end{question}
}


%% PhysicsBowl 1996
%%----------------------------------------
\element{aapt}{ %% Bowl-E2
\begin{question}{bowl-1996-q13}
    A potassium \ce{^{40}_{19}K} nucleus emits a beta ($\beta$) and becomes:
    \begin{multicols}{3}
    \begin{choices}
        \wrongchoice{\ce{^{36}_{17}Cl}}
        \wrongchoice{\ce{^{44}_{21}Sc}}
        \wrongchoice{\ce{^{40}_{18}Ar}}
        \wrongchoice{\ce{^{40}_{19}K}}
      \correctchoice{\ce{^{40}_{20}Ca}}
    \end{choices}
    \end{multicols}
\end{question}
}


%% PhysicsBowl 1995
%%----------------------------------------
\element{aapt}{ %% Bowl-E2
\begin{question}{bowl-1995-q22}
    A radioactive element has a half-life of \SI{4.0}{\hour}.
    Approximately how much of the radioactive element will remain after \SI{12.0}{\hour}?
    \begin{multicols}{3}
    \begin{choices}
        \wrongchoice{$\dfrac{1}{16}$}
      \correctchoice{$\dfrac{1}{8}$}
        \wrongchoice{$\dfrac{1}{6}$}
        \wrongchoice{$\dfrac{1}{4}$}
        \wrongchoice{$\dfrac{1}{3}$}
    \end{choices}
    \end{multicols}
\end{question}
}

\element{aapt}{ %% Bowl-E2
\begin{question}{bowl-1995-q28}
    A radon \ce{^{220}_{86}Rn} nucleus emits an alpha particle and becomes a:
    \begin{multicols}{3}
    \begin{choices}
      \correctchoice{\ce{^{216}_{84}Po}}
        \wrongchoice{\ce{^{220}_{85}At}}
        \wrongchoice{\ce{^{220}_{86}Rn}}
        \wrongchoice{\ce{^{220}_{87}Fr}}
        \wrongchoice{\ce{^{224}_{88}Ra}}
    \end{choices}
    \end{multicols}
\end{question}
}


%% PhysicsBowl 1994
%%----------------------------------------
\element{aapt}{ %% Bowl-E2
\begin{question}{bowl-1994-q10}
    Which device can be used to detect nuclear radiation?
    \begin{multicols}{2}
    \begin{choices}
        \wrongchoice{cyclotron}
      \correctchoice{photographic film}
        \wrongchoice{betatron}
        \wrongchoice{synchrotron}
        \wrongchoice{Van de Graaff}
    \end{choices}
    \end{multicols}
\end{question}
}

\element{aapt}{ %% Bowl-E2
\begin{question}{bowl-1994-q22}
    A radioactive oxygen, $^{15}_{8}\mathrm{O}$, nucleus emits a positron and becomes:
    \begin{multicols}{3}
    \begin{choices}
        \wrongchoice{$^{14}_{7}\mathrm{N}$}
        \wrongchoice{$^{15}_{7}\mathrm{N}$}
        \wrongchoice{$^{15}_{8}\mathrm{O}$}
      \correctchoice{$^{14}_{9}\mathrm{F}$}
        \wrongchoice{$^{15}_{9}\mathrm{F}$}
    \end{choices}
    \end{multicols}
\end{question}
}

\element{aapt}{ %% Bowl-E2
\begin{question}{bowl-1994-q24}
    A sample of radioactive material has an initial activity of \SI{10 000}{\per\minute}.
    \SI{30}{\minute} later, its activity is \SI{2 500}{\per\minute}.
    The half-life of the material is:
    \begin{multicols}{3}
    \begin{choices}
        \wrongchoice{\SI{7.5}{\minute}}
        \wrongchoice{\SI{10}{\minute}}
      \correctchoice{\SI{15}{\minute}}
        \wrongchoice{\SI{20}{\minute}}
        \wrongchoice{\SI{22}{\minute}}
    \end{choices}
    \end{multicols}
\end{question}
}


\endinput


