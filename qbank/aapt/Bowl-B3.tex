

%% AAPT Physics Bowl Exams Questions
%%----------------------------------------


%% This section has XX problems


%% PhysicsBowl 2015
%%----------------------------------------


%% PhysicsBowl 2014
%%----------------------------------------
\element{aapt}{ %% Bowl-B3
\begin{question}{bowl-2014-q42}
    Plane-polarized light with intensity $I$ is incident on a single polarizing sheet.
    If the intensity of the light becomes $\frac{1}{4}I$ after passing through the polarizer,
        what is the angle between the transmission axis of the polarizer and the polarizer and the polarization plane of the incident light?
    \begin{multicols}{3}
    \begin{choices}
        \wrongchoice{\ang{75}}
        \wrongchoice{\ang{67.5}}
      \correctchoice{\ang{60}}
        \wrongchoice{\ang{30}}
        \wrongchoice{\ang{22.5}}
    \end{choices}
    \end{multicols}
\end{question}
}

\element{aapt}{ %% Bowl-B3
\begin{question}{bowl-2014-q48}
    A laser beam with wavelength \SI{632.8}{\nano\meter} shines onto a newly fabricated single slit.
    As a result, the width of the principal bright region on a viewing screen \SI{1.25}{\meter} away is \SI{1.00}{\meter}.
    Which of the following best represents the size of the single slit opening?
    \begin{multicols}{3}
    \begin{choices}
        \wrongchoice{\SI{0.79}{\micro\meter}}
        \wrongchoice{\SI{0.85}{\micro\meter}}
        \wrongchoice{\SI{1.01}{\micro\meter}}
        \wrongchoice{\SI{1.58}{\micro\meter}}
      \correctchoice{\SI{1.70}{\micro\meter}}
    \end{choices}
    \end{multicols}
\end{question}
}


%% PhysicsBowl 2013
%%----------------------------------------
\element{aapt}{ %% Bowl-B3
\begin{question}{bowl-2013-q24}
    Which one of the following statements best describes Huygens' Principle?
    \begin{choices}
        \wrongchoice{An additional pressure is transmitted undiminished to all points in the fluid and to the walls of the container.}
      \correctchoice{Each point on a wavefront acts as a source of secondary spherical wavelets (new waves).}
        \wrongchoice{For every action force, there is an equal but opposite reaction force.}
        \wrongchoice{It is impossible to have a process which has the sole result of transferring energy from a low temperature reservoir to a high temperature reservoir.}
        \wrongchoice{A time-changing magnetic field has an associated induced electric field.}
    \end{choices}
\end{question}
}


%% PhysicsBowl 2012
%%----------------------------------------
\element{aapt}{ %% Bowl-B3
\begin{question}{bowl-2012-q30}
    Coherent light of wavelength \SI{550}{\nano\meter} shines
        on a double slit apparatus that has point slits
        spaced by a distance of \SI{42.4}{\micro\meter}.
    In theory, what is the maximum order bright fringe
        that can be viewed?
    %% d \sin\theta = m\lambda
    %% m_{max} at \theta=\ang{90}
    %% m_{max} = \frac{d}{\lambda} = 77.1
    \begin{multicols}{3}
    \begin{choices}
        \wrongchoice{\num{1297}}
      \correctchoice{\num{77}}
        \wrongchoice{\num{12}}
        \wrongchoice{\num{8}}
        \wrongchoice{\num{1}}
    \end{choices}
    \end{multicols}
\end{question}
}

\element{aapt}{ %% Bowl-B3
\begin{question}{bowl-2012-q49}
    A beam of unpolarized light traveling in air strikes a piece of optically flat glass at an angle of incidence of \ang{58}.
    Some of the light is reflected while the remainder is transmitted into the glass.
    The reflected beam is \SI{100}{\percent} polarized parallel to the surface of the glass.
    What is the index of refraction for the glass?
    \begin{multicols}{3}
    \begin{choices}
      \correctchoice{\num{1.60}}
        \wrongchoice{\num{1.53}}
        \wrongchoice{\num{1.47}}
        \wrongchoice{\num{1.38}}
        \wrongchoice{\num{1.18}}
    \end{choices}
    \end{multicols}
\end{question}
}


%% PhysicsBowl 2008
%%----------------------------------------
\element{aapt}{ %% Bowl-B3
\begin{question}{bowl-2008-q43}
    Unpolarized light of intensity $I_0$ enters a polarizer-analyzer system in which the angle between the transmission axes of the polarizer and analyzer is \ang{30}.
    What is the intensity of the light leaving the analyzer?
    \begin{multicols}{3}
    \begin{choices}
      \correctchoice{$\dfrac{3}{8}I_0$}
        \wrongchoice{$\dfrac{1}{8}I_0$}
        \wrongchoice{$\dfrac{3}{4}I_0$}
        \wrongchoice{$\dfrac{1}{4}I_0$}
        \wrongchoice{$\dfrac{1}{2}I_0$}
    \end{choices}
    \end{multicols}
\end{question}
}


%% PhysicsBowl 2006
%%----------------------------------------
\element{aapt}{ %% Bowl-B3
\begin{question}{bowl-2006-q45}
    In a Young's double-slit experiment, the slit separation is doubled.
    To maintain the same fringe spacing on the screen,
        the screen-to-slit distance $D$ must be changed to:
    \begin{multicols}{3}
    \begin{choices}
        \wrongchoice{$\dfrac{D}{2}$}
        \wrongchoice{$\dfrac{D}{2}$}
        \wrongchoice{$\sqrt{2} D$}
      \correctchoice{$2 D$}
        \wrongchoice{$4 D$}
    \end{choices}
    \end{multicols}
\end{question}
}

\element{aapt}{ %% Bowl-B3
\begin{question}{bowl-2006-q48}
    Unpolarized light of intensity $I_0$ is incident onto a series of 3 polarizers.
    The angles of the polarizers are set to
        \ang{0}, \ang{45}, \ang{90} as measured from the vertical.
    What is the intensity of the light as it exits from the third polarizer?
    \begin{multicols}{3}
    \begin{choices}
        \wrongchoice{\num{0}}
        \wrongchoice{\num{1/2}}
        \wrongchoice{\num{1/4}}
      \correctchoice{\num{1/8}}
        \wrongchoice{\num{1/16}}
    \end{choices}
    \end{multicols}
\end{question}
}


%% PhysicsBowl 2005
%%----------------------------------------
\element{aapt}{ %% Bowl-B3
\begin{question}{bowl-2005-q15}
    %In the electromagnetic spectrum,
    %    rank the following electromagnetic waves in terms of increasing wavelength
    Which option correctly ranks the electromagnetic waves in terms of increasing wavelength?
    \begin{choices}
        \wrongchoice{Ultraviolet, X-ray, Radio waves}
        \wrongchoice{Ultraviolet, Radio Waves, X-ray}
        \wrongchoice{Radio waves, Ultraviolet, X-ray}
        \wrongchoice{Radio waves, X-ray, Ultraviolet}
      \correctchoice{X-ray, Ultraviolet, Radio waves}
    \end{choices}
\end{question}
}


%% PhysicsBowl 2000
%%----------------------------------------
\element{aapt}{ %% Bowl-B3
\begin{question}{bowl-2000-q30}
    Monochromatic light falls on a single slit \SI{0.01}{\centi\meter} wide
        and develops a first-order minimum (dark band) \SI{0.59}{\centi\meter}
        from the center of the central bright band on a screen that is one meter away.
    Determine the wavelength of the light.
    \begin{multicols}{2}
    \begin{choices}
        \wrongchoice{\SI{1.18e-2}{\centi\meter}}
        \wrongchoice{\SI{5.90e-3}{\centi\meter}}
        \wrongchoice{\SI{1.18e-4}{\centi\meter}}
      \correctchoice{\SI{5.90e-5}{\centi\meter}}
        \wrongchoice{\SI{1.18e-6}{\centi\meter}}
    \end{choices}
    \end{multicols}
\end{question}
}


%% PhysicsBowl 1997
%%----------------------------------------
\element{aapt}{ %% Bowl-B3
\begin{questionmult}{bowl-1997-q29}
    A student performs an experiment similar to Young's Double Slit Experiment.
    Coherent light passes through two narrow slits and produces a
        pattern of alternating bright and dark lines on a screen.
    Which of the following would cause the bright lines on the screen to be further apart?
    \begin{choices}
        \wrongchoice{Increasing the distance between the slits.}
      \correctchoice{Decreasing the distance between the slits.}
        \wrongchoice{Decreasing the wavelength of the light.}
    \end{choices}
\end{questionmult}
}


%% PhysicsBowl 1994
%%----------------------------------------
\element{aapt}{ %% Bowl-B3
\begin{question}{bowl-1994-q32}
    In Young's double slit experiment,
        second and higher order bright bands can overlap.
    Which third order band would occur at the same location as a second order band of wavelength \SI{660}{\nano\meter}?
    \begin{multicols}{3}
    \begin{choices}
        \wrongchoice{\SI{1320}{\nano\meter}}
        \wrongchoice{\SI{990}{\nano\meter}}
        \wrongchoice{\SI{495}{\nano\meter}}
      \correctchoice{\SI{440}{\nano\meter}}
        \wrongchoice{\SI{330}{\nano\meter}}
    \end{choices}
    \end{multicols}
\end{question}
}


\endinput


