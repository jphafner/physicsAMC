

%% AAPT Physics Bowl Exams Questions
%%----------------------------------------


%% This section has XX problems


%% PhysicsBowl 2015
%%----------------------------------------


%% PhysicsBowl 2013
%%----------------------------------------
\element{aapt}{ %% Bowl-E1
\begin{question}{bowl-2013-q36}
    The principal quantum number of an electron is $n=5$.
    How many possible values of the orbital magnetic quantum number $m_l$ are there for this electron?
    \begin{multicols}{3}
    \begin{choices}
        \wrongchoice{\num{25}}
        \wrongchoice{\num{11}}
      \correctchoice{\num{9}}
        \wrongchoice{\num{5}}
        \wrongchoice{\num{4}}
    \end{choices}
    \end{multicols}
\end{question}
}


%% PhysicsBowl 1997
%%----------------------------------------
\element{aapt}{ %% Bowl-E1
\begin{question}{bowl-1997-q36}
    A student performs the Photoelectric Effect Experiment and obtains the data depicted in the accompanying graph of $E_{Km}$,
        the maximum kinetic energy of the photoelectrons in \si{\eV}, versus $f$, the frequency of the photons in \SI{10e14}{\hertz}.
    \begin{center}
    \begin{tikzpicture}
        \begin{axis}[
            axis y line=left,
            axis x line=middle,
            axis line style={->},
            xlabel={frequency},
            x unit=\SI{e14}{\hertz},
            xtick={0,2,4,6,8,10},
            x label style={anchor=north east,yshift=-2.5em},
            ylabel={$E_{Km}$},
            y unit=\si{\eV},
            ytick={-1,0,1,2},
            grid=major,
            xmin=0,xmax=10,
            ymin=-1.5,ymax=2.5,
            width=0.8\columnwidth,
            height=0.5\columnwidth,
        ]
        %% NOTE: \phi = 3.63e14 * 6.63e-34 / 1.6e-19
        \addplot[line width=1pt,mark=\empty] plot coordinates { (3.63,0) (10,2.64) };
        \end{axis}
    \end{tikzpicture}
    \end{center}
    What is the approximate work function for this material?
    \begin{multicols}{2}
    \begin{choices}
      \correctchoice{\SI{1.5}{\eV}}
        \wrongchoice{\SI{2.0}{\eV}}
        \wrongchoice{\SI{2.7}{\eV}}
        \wrongchoice{\SI{4.0}{\eV}}
        \wrongchoice{\SI{6.0}{\eV}}
        %% NOTE: is this distractor too much?
        \wrongchoice{\SI{3.6}{\eV}}
    \end{choices}
    \end{multicols}
\end{question}
}


\endinput


