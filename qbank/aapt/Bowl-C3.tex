

%% AAPT Physics Bowl Exams Questions
%%----------------------------------------


%% This section has XX problems


%% PhysicsBowl 2015
%%----------------------------------------
\element{aapt}{ %% Bowl-C3
\begin{question}{bowl-2015-q19}
    Ten moles of helium gas are enclosed in a container at a pressure of
        \SI{1.00}{atm} and at a temperature of \SI{400}{\kelvin}.
    Which one of the following choices best represents the density of this gas sample?
    \begin{multicols}{2}
    \begin{choices}
        \wrongchoice{\SI{0.012}{\kilo\gram\per\meter\cubed}}
      \correctchoice{\SI{0.12}{\kilo\gram\per\meter\cubed}}
        \wrongchoice{\SI{1.2}{\kilo\gram\per\meter\cubed}}
        \wrongchoice{\SI{120}{\kilo\gram\per\meter\cubed}}
        \wrongchoice{\SI{1.2e4}{\kilo\gram\per\meter\cubed}}
    \end{choices}
    \end{multicols}
\end{question}
}

\element{aapt}{ %% Bowl-C3
\begin{question}{bowl-2015-q25}
    A gas undergoes the unusual process $M\rightarrow{}N$ in the pressure vs. volume graph shown.
    \begin{center}
    \begin{tikzpicture}
        \begin{axis}[
            axis y line=left,
            axis x line=bottom,
            axis line style={->},
            xlabel={volume},
            xtick=\empty,
            ylabel={pressure},
            ytick=\empty,
            grid=major,
            xmin=0,xmax=10,
            ymin=0,ymax=10,
            width=0.8\columnwidth,
            height=0.5\columnwidth,
        ]
        %% line
        \addplot[line width=1pt,smooth,tension=0.7,mark=\empty] plot coordinates {(6,8) (5,1) (3,3) (2,2) };
        %% arrows
        \draw[very thick,-latex] (axis cs:5.70,4.5) -- ++(260:0.5);
        \draw[very thick,-latex] (axis cs:3,3) -- ++(180:0.5);
        %% labels
        \fill (axis cs:2,2) circle (1.5pt) node[anchor=north] {$N$};
        \fill (axis cs:6,8) circle (1.5pt) node[anchor=south] {$M$};
        \end{axis}
    \end{tikzpicture}
    \end{center}
    Which one of the following choices properly represents the signs of the internal energy change of the gas, $\Delta U$,
        the total energy transferred as heat to the gas, $Q$,
        and the total work done on the gas by the surroundings, $W$, for this process?
    \begin{center}
    \begin{tabu}{cX[c]X[c]X[c]}
        \toprule
        \makebox[1.5em][c]{\textnumero}
            & $\Delta U$
            & $Q$ & $W$ \\
        \bottomrule
    \end{tabu}
    \end{center}
    \begin{choices}
        \wrongchoice{\begin{tabu}{X[c]X[c]X[c]} Positive & Negative & Positive \\ \end{tabu}}
        \wrongchoice{\begin{tabu}{X[c]X[c]X[c]} Negative & Positive & Negative \\ \end{tabu}}
        \wrongchoice{\begin{tabu}{X[c]X[c]X[c]} Negative & Negative & Negative \\ \end{tabu}}
        \wrongchoice{\begin{tabu}{X[c]X[c]X[c]} Positive & Positive & Negative \\ \end{tabu}}
      \correctchoice{\begin{tabu}{X[c]X[c]X[c]} Negative & Negative & Positive \\ \end{tabu}}
    \end{choices}
\end{question}
}

\element{aapt}{ %% Bowl-C3
\begin{question}{bowl-2015-q47}
    One mole of a diatomic ideal gas undergoes a reversible adiabatic process.
    The pressure and volume initially are given as $P=\SI{2.0}{atm}$ and $V=\SI{30}{\liter}$.
    If the volume is halved during the adiabatic process,
        how much work was done on the gas sample by the surroundings?
    \begin{multicols}{3}
    \begin{choices}
        \wrongchoice{\SI{6790}{\joule}}
        \wrongchoice{\SI{5530}{\joule}}
      \correctchoice{\SI{4850}{\joule}}
        \wrongchoice{\SI{4200}{\joule}}
        \wrongchoice{\SI{3040}{\joule}}
    \end{choices}
    \end{multicols}
\end{question}
}


%% PhysicsBowl 2014
%%----------------------------------------
\element{aapt}{ %% Bowl-C3
\begin{question}{bowl-2014-q46}
    A monatomic, ideal gas undergoes an isobaric process.
    During the process,
        the gas performs \SI{80}{\joule} of work on the surrounding.
    Which one of the following statements best describes the head exchange during this process?
    \begin{choices}
      \correctchoice{\SI{200}{\joule} of energy was added to the gas.}
        \wrongchoice{\SI{200}{\joule} of energy was removed from the gas.}
        \wrongchoice{\SI{80}{\joule} of energy was added to the gas.}
        \wrongchoice{\SI{80}{\joule} of energy was removed from the gas.}
        \wrongchoice{It cannot be determined without knowing the change in temperature for the gas.}
    \end{choices}
\end{question}
}


%% PhysicsBowl 2013
%%----------------------------------------
\element{aapt}{ %% Bowl-C3
\begin{question}{bowl-2013-q31}
    A monatomic ideal gas undergoes a reversible isothermal
        expansion in an enclosed container.
    Which one of the following quantities associated with
        the gas has a value of \emph{zero}?
    \begin{choices}
        \wrongchoice{Heat}
        \wrongchoice{Entropy change}
        \wrongchoice{Work done}
      \correctchoice{Internal energy change}
        \wrongchoice{Pressure change}
    \end{choices}
\end{question}
}

\element{aapt}{ %% Bowl-C3
\begin{question}{bowl-2013-q41}
    An engine operates between a low temperature of \SI{273}{\degreeCelsius} and a high temperature of \SI{546}{\degreeCelsius}.
    What is the maximum theoretical efficiency of this engine?
    \begin{multicols}{3}
    \begin{choices}
      \correctchoice{$\dfrac{1}{3}$}
        \wrongchoice{$\dfrac{2}{3}$}
        \wrongchoice{$\dfrac{1}{4}$}
        \wrongchoice{$\dfrac{1}{2}$}
        \wrongchoice{$\dfrac{3}{4}$}
    \end{choices}
    \end{multicols}
\end{question}
}

\element{aapt}{ %% Bowl-C3
\begin{question}{bowl-2013-q48}
    Two identical samples of a monatomic ideal gas are to undergo reversible processes.
    Which one of the following choices is a correct statement about the heat associated with the process?
    \begin{description}[itemsep=1ex]
        \item[Process 1:] An isochoric pressure doubling
        \item[Process 2:] An isobaric volume doubling
    \end{description}
    \begin{choices}
      \correctchoice{There is less heat associated with Process 1 than Process 2.}
        \wrongchoice{The heat is the same non-zero value for Process 1 and 2.}
        \wrongchoice{There is more heat associated with Process 1 than Process 2.}
        \wrongchoice{The heat is zero for Process 1 and 2.}
        \wrongchoice{More information is required to determine the relationship for heats.}
    \end{choices}
\end{question}
}


%% PhysicsBowl 2012
%%----------------------------------------
\element{aapt}{ %% Bowl-C3
\begin{question}{bowl-2012-q27}
    The pressure inside a container with two moles of
        an ideal gas is \num{0.75}~atmospheres.
    The temperature of the gas is \SI{100}{\degreeCelsius}.
    The container maintains constant volume as the pressure is tripled.
    Which one of the following choices best represents
        the temperature of the gas after the pressure is tripled?
    \begin{multicols}{3}
    \begin{choices}
        \wrongchoice{\SI{33}{\degreeCelsius}}
        \wrongchoice{\SI{300}{\degreeCelsius}}
        \wrongchoice{\SI{573}{\degreeCelsius}}
      \correctchoice{\SI{847}{\degreeCelsius}}
        \wrongchoice{\SI{1119}{\degreeCelsius}}
    \end{choices}
    \end{multicols}
\end{question}
}

\element{aapt}{ %% Bowl-C3
\begin{question}{bowl-2012-q44}
    A monatomic ideal gas is the working substance for a refrigerator that undergoes the cycle process ($ABCDA$) shown in the $PV$ diagram.
    \begin{center}
    \begin{tikzpicture}
        \begin{axis}[
            axis y line=left,
            axis x line=bottom,
            axis line style={->},
            xlabel={volume},
            xtick={0,1,2},
            xticklabels={0,$V_0$,$2V_0$},
            ylabel={pressure},
            ytick={0,1,2},
            yticklabels={0,$P_0$,$2P_0$},
            xmin=0,xmax=2.5,
            ymin=0,ymax=2.5,
            width=0.8\columnwidth,
            height=0.5\columnwidth,
        ]
        %% process
        \addplot[line width=1pt,mark=\empty] plot coordinates { (1,1) (2,1) (2,2) (1,2) (1,1) };
        %% labels
        \node[anchor=north east] at (axis cs:1,1) {$A$};
        \node[anchor=north west] at (axis cs:2,1) {$B$};
        \node[anchor=south west] at (axis cs:2,2) {$C$};
        \node[anchor=south east] at (axis cs:1,2) {$D$};
        %% arrows
        \draw[line width=1pt,-latex] (axis cs:2,2) -- (axis cs:1.5,2);
        \draw[line width=1pt,-latex] (axis cs:1,2) -- (axis cs:1,1.5);
        \draw[line width=1pt,-latex] (axis cs:1,1) -- (axis cs:1.5,1);
        \draw[line width=1pt,-latex] (axis cs:2,1) -- (axis cs:2,1.5);
        %% dashed
        \draw[dashed] (axis cs:0,1) -- (axis cs:1,1);
        \draw[dashed] (axis cs:0,2) -- (axis cs:1,2);
        \draw[dashed] (axis cs:1,0) -- (axis cs:1,1);
        \draw[dashed] (axis cs:2,0) -- (axis cs:2,1);
        \end{axis}
    \end{tikzpicture}
    \end{center}
    The process are all isochoric or isobaric with pressure between $P_0$
        and $2P_0$ and volume between $V_0$ and $2V_0$.
    What is the coefficient of performance for this refrigerator?
    \begin{multicols}{3}
    \begin{choices}
        \wrongchoice{$\dfrac{1}{4}$}
        \wrongchoice{$\dfrac{1}{3}$}
        \wrongchoice{$\dfrac{4}{3}$}
      \correctchoice{$\dfrac{11}{2}$}
        \wrongchoice{$\dfrac{13}{2}$}
    \end{choices}
    \end{multicols}
\end{question}
}


%% PhysicsBowl 2011
%%----------------------------------------
\element{aapt}{ %% Bowl-C3
\begin{question}{bowl-2011-q44}
    One mole of monatomic ideal gas undergoes an isobaric expansion.
    In the process, the temperature of the gas increases from \SI{300}{\kelvin} to \SI{500}{\kelvin}.
    Which one of the following choices best represents the amount of work done by the gas on the surroundings?
    \begin{multicols}{3}
    \begin{choices}
        \wrongchoice{\SI{665}{\joule}}
        \wrongchoice{\SI{1110}{\joule}}
      \correctchoice{\SI{1660}{\joule}}
        \wrongchoice{\SI{2490}{\joule}}
        \wrongchoice{\SI{4160}{\joule}}
    \end{choices}
    \end{multicols}
\end{question}
}


%% PhysicsBowl 2010
%%----------------------------------------
\element{aapt}{ %% Bowl-C3
\begin{question}{bowl-2010-q05}
    When computed with proper MKS units, the Universal Gas
        Constant divided by Boltzmann's constant is equal to:
    \begin{choices}
        \wrongchoice{the speed of light.}
        \wrongchoice{Planck's constant.}
      \correctchoice{Avogadro's number.}
        \wrongchoice{the permittivity of free space.}
        \wrongchoice{the Universal Gravitational constant.}
    \end{choices}
\end{question}
}


%% PhysicsBowl 2009
%%----------------------------------------
\element{aapt}{ %% Bowl-C3
\begin{question}{bowl-2009-q33}
    An ideal gas undergoes an isobaric expansion followed by an isochoric cooling.
    Which of the following statements \emph{must} be true after the completion of these processes?
    \begin{choices}
      \correctchoice{The final pressure is less than the original pressure.}
        \wrongchoice{The final volume is less than the original volume.}
        \wrongchoice{The final temperature is less than the original temperature.}
        \wrongchoice{The total quantity of heat, $Q$, associated with these processes is positive.}
        \wrongchoice{The internal energy of the gas is unchanged.}
    \end{choices}
\end{question}
}


%% PhysicsBowl 2008
%%----------------------------------------
\element{aapt}{ %% Bowl-C3
\begin{question}{bowl-2008-q44}
    A mole of a monatomic ideal gas has pressure $P$,
        volume $V$, and temperature $T$.
    Which of the following processes would result in the greatest amount of energy added to the gas from heat?
    \begin{choices}
      \correctchoice{A process doubling the temperature at constant pressure.}
        \wrongchoice{An adiabatic free expansion doubling the volume.}
        \wrongchoice{A process doubling the pressure at constant volume.}
        \wrongchoice{An adiabatic expansion doubling the volume.}
        \wrongchoice{A process doubling the volume at constant temperature.}
    \end{choices}
\end{question}
}


%% PhysicsBowl 2007
%%----------------------------------------
\element{aapt}{ %% Bowl-C3
\begin{question}{bowl-2007-q40}
    Which is not true of an isochoric process on an enclosed ideal gas in which the pressure decreases?
    \begin{choices}
        \wrongchoice{The work done is zero.}
        \wrongchoice{The internal energy of the gas decreases.}
      \correctchoice{The heat is zero.}
        \wrongchoice{The rms speed of the gas molecules decreases.}
        \wrongchoice{The gas temperature decreases.}
    \end{choices}
\end{question}
}

\element{aapt}{ %% Bowl-C3
\begin{question}{bowl-2007-q46}
    An ideal gas undergoes a reversible isothermal expansion at $T=\SI{300}{\kelvin}$.
    The total change in entropy of the gas is \SI{2.5}{\joule\per\kelvin}.
    How much work was done by the environment on the gas during this process?
    \begin{multicols}{2}
    \begin{choices}
      \correctchoice{\SI{-750}{\joule}}
        \wrongchoice{\SI{-120}{\joule}}
        \wrongchoice{\SI{120}{\joule}}
        \wrongchoice{\SI{750}{\joule}}
        \wrongchoice{More information is required to answer this question.}
    \end{choices}
    \end{multicols}
\end{question}
}


%% PhysicsBowl 2006
%%----------------------------------------
\element{aapt}{ %% Bowl-C3
\begin{question}{bowl-2006-q26}
    In the Pressure versus Volume graph shown,
        in the process of going from $a$ to $b$ \SI{60}{\joule} of heat are added,
        and in the process of going from $b$ to $d$ \SI{20}{\joule} of heat are added.
    \begin{center}
    \begin{tikzpicture}
        \begin{axis}[
            axis y line=left,
            axis x line=bottom,
            axis line style={->},
            xlabel={volume},
            x unit=\si{\meter\cubed},
            xtick={2,5},
            ylabel={pressure},
            y unit=\si{\pascal},
            ytick={3,8},
            xmin=0,xmax=6,
            ymin=0,ymax=9,
            clip=false,
            width=0.8\columnwidth,
            height=0.5\columnwidth,
        ]
        %% process
        \addplot[line width=1pt,mark=\empty] plot coordinates { (2,3) (2,8) (5,8) (5,3) (2,3) };
        %% labels
        \node[anchor=north west] at (axis cs:2,3) {$A$};
        \node[anchor=south west] at (axis cs:2,8) {$B$};
        \node[anchor=north west] at (axis cs:5,3) {$C$};
        \node[anchor=south west] at (axis cs:5,8) {$D$};
        %% arrows
        \draw[line width=1pt,-latex] (axis cs:2,3) -- (axis cs:2,4);
        \draw[line width=1pt,-latex] (axis cs:2,8) -- (axis cs:3,8);
        \draw[line width=1pt,-latex] (axis cs:2,3) -- (axis cs:3,3);
        \draw[line width=1pt,-latex] (axis cs:5,3) -- (axis cs:5,4);
        %% dashed
        \draw[dashed] (axis cs:0,3) -- (axis cs:2,3);
        \draw[dashed] (axis cs:0,8) -- (axis cs:2,8);
        \draw[dashed] (axis cs:2,0) -- (axis cs:2,3);
        \draw[dashed] (axis cs:5,0) -- (axis cs:5,3);
        \end{axis}
    \end{tikzpicture}
    \end{center}
    In the process of going $A$ to $C$ to $D$,
        what is the total heat added?
    \begin{multicols}{3}
    \begin{choices}
        \wrongchoice{\SI{80}{\joule}}
      \correctchoice{\SI{65}{\joule}}
        \wrongchoice{\SI{60}{\joule}}
        \wrongchoice{\SI{56}{\joule}}
        \wrongchoice{\SI{47}{\joule}}
    \end{choices}
    \end{multicols}
\end{question}
}

\element{aapt}{ %% Bowl-C3
\begin{question}{bowl-2006-q43}
    A \SI{10}{\ohm} resistor is connected to a \SI{12}{\volt} battery.
    If the temperature (\SI{300}{\kelvin}) and mass of the resistor (\SI{20}{\gram}) remain constant,
        what is the change in entropy of the resistor during \SI{30}{\second} of operation of the circuit?
    \begin{multicols}{2}
    \begin{choices}
        \wrongchoice{\SI{432}{\joule\per\kelvin}}
        \wrongchoice{\SI{1.44}{\joule\per\kelvin}}
      \correctchoice{\SI{0}{\joule\per\kelvin}}
        \wrongchoice{\SI{-1.44}{\joule\per\kelvin}}
        \wrongchoice{\SI{-432}{\joule\per\kelvin}}
    \end{choices}
    \end{multicols}
\end{question}
}

\element{aapt}{ %% Bowl-C3
\begin{question}{bowl-2006-q44}
    A refrigerator must operate when the outside environment is at \SI{22}{\degreeCelsius} to maintain an environment at \SI{2}{\degreeCelsius} inside the refrigerator.
    During one cycle,
        the refrigerator expels \SI{1200}{\joule} of energy to the outside environment while \SI{400}{\joule} of work were done.
    What is the coefficient of performance of this refrigerator?
    \begin{multicols}{3}
    \begin{choices}
      \correctchoice{\num{2.00}}
        \wrongchoice{\num{3.00}}
        \wrongchoice{\num{10.00}}
        \wrongchoice{\num{13.75}}
        \wrongchoice{\num{14.75}}
    \end{choices}
    \end{multicols}
\end{question}
}


%% PhysicsBowl 2005
%%----------------------------------------
\element{aapt}{ %% Bowl-C3
\begin{question}{bowl-2005-q42}
    A sample of a monatomic ideal gas starts with pressure,
        volume, temperature, and number of moles given as $P_0$, $V_0$, $T_0$, $n$, respectively.
    The gas has molar specific heats at constant pressure and constant volume of $c_p$, $c_v$.
    \begin{description}
        \item[Process 1:] The gas is compressed to half of its volume at constant pressure.
        \item[Process 2:] The gas is expanded isothermally such that the gas does work on the environment equal to the amount of work done onto the gas by the environment during process 1.
    \end{description}
    What is the entropy change of the gas during Process 2?
    \begin{multicols}{2}
    \begin{choices}
        \wrongchoice{$nc_p \ln\left(\dfrac{1}{2}\right)$}
        \wrongchoice{$nc_p \ln\left(2\right)$}
      \correctchoice{$\dfrac{P_0 V_0}{T_0}$}
        \wrongchoice{$0$}
        \wrongchoice{none of the provided}
    \end{choices}
    \end{multicols}
\end{question}
}


%% PhysicsBowl 1998
%%----------------------------------------
\element{aapt}{ %% Bowl-C3
\begin{question}{bowl-1998-q07}
    A heat engine takes in \SI{200}{\joule} of thermal energy and performs \SI{50}{\joule} of work in each cycle.
    What is its efficiency?
    \begin{multicols}{3}
    \begin{choices}
        \wrongchoice{\SI{500}{\percent}}
        \wrongchoice{\SI{400}{\percent}}
      \correctchoice{\SI{25}{\percent}}
        \wrongchoice{\SI{20}{\percent}}
        \wrongchoice{\SI{12}{\percent}}
    \end{choices}
    \end{multicols}
\end{question}
}


%% PhysicsBowl 1997
%%----------------------------------------
\element{aapt}{ %% Bowl-C3
\begin{question}{bowl-1997-q15}
    Which of the following is always true for an isothermal process of an ideal gas?
    \begin{choices}
      \correctchoice{The internal energy does not change.}
        \wrongchoice{No heat flows into or out of the system.}
        \wrongchoice{The pressure does not change.}
        \wrongchoice{The volume does not change.}
        \wrongchoice{No work is done by or on the system.}
    \end{choices}
\end{question}
}


%% PhysicsBowl 1996
%%----------------------------------------
\element{aapt}{ %% Bowl-C3
\begin{question}{bowl-1996-q22}
    When an ideal gas is isothermally compressed:
    \begin{choices}
      \correctchoice{thermal energy flows from the gas to the surroundings.}
        \wrongchoice{the temperature of the gas decreases.}
        \wrongchoice{no thermal energy enters or leaves the gas.}
        \wrongchoice{the temperature of the gas increases.}
        \wrongchoice{thermal energy flows from the surroundings to the gas.}
    \end{choices}
\end{question}
}

\element{aapt}{ %% Bowl-C3
\begin{question}{bowl-1995-q24}
    The theoretical (Carnot) efficiency of a heat engine operating between \SI{600}{\degreeCelsius} and \SI{100}{\degreeCelsius} is:
    \begin{multicols}{3}
    \begin{choices}
        \wrongchoice{\SI{16.7}{\percent}}
        \wrongchoice{\SI{20.0}{\percent}}
        \wrongchoice{\SI{42.0}{\percent}}
      \correctchoice{\SI{57.3}{\percent}}
        \wrongchoice{\SI{83.3}{\percent}}
    \end{choices}
    \end{multicols}
\end{question}
}


\endinput


