
%%--------------------------------------------------
%% Schaum's Outline of Applied Physics
%%--------------------------------------------------


%% Chapter 2: Vectors
%%--------------------------------------------------


%% Schaum's Numeric Problems
%%--------------------------------------------------
\element{schaums-num}{
\begin{questionmultx}{ch02-num-Q01}
    Find the length, $l$, of the hypotenuse of a right triangle whose legs are \SI{483}{\meter} and \SI{620}{\meter} long.
    Report $l/\si{\meter}$ below.
    \AMCnumericChoices{785.93}{
        vertical=false,
        digits=5,decimals=2,sign=true,
        borderwidth=0pt,backgroundcol=white,approx=5
    }
\end{questionmultx}
}

\element{schaums-num}{
\begin{questionmultx}{ch02-num-Q02}
    The hypotenuse of a right triangle is \SI{28}{\centi\meter} long and the length of one of the legs is \SI{23}{\centi\meter}.
    Find the length of the other leg.
    \AMCnumericChoices{785.93}{
        vertical=false,
        digits=5,decimals=2,sign=true,
        borderwidth=0pt,backgroundcol=white,approx=5
    }
\end{questionmultx}
}

\begin{comment}
2.3. Find the values of the unknown sides and angles in the right triangles for which the following data are known (see
Fig. 2-8).
(a)θ = 45 ◦ , a = 10 (d)a = 3, b = 4, c = 5
(b)θ = 15 ◦ , b = 4 (e)a = 5, b = 12, c = 13◦
(c)θ = 25 , c = 5

2.4. Two forces, one of 10 N and the other of 6 N, act on a body. The directions of the forces are not known.
(a) What is the minimum magnitude of the resultant of these forces? (b) What is the maximum magnitude?

2.5. A man drives 10 km to the north and then 20 km to the east. What are the magnitude and direction of his displacement
from the starting point?

2.6. Find the magnitude and direction of the resultant force produced by a vertically upward force of 40 N and a
horizontal force of 30 N.

2.7. Find the vertical and horizontal components of a 50-N force that is directed 50 ◦ above the horizontal.


2.8. A woman pushes a lawn mower with a force of 80 N. If the handle of the lawn mower is 40 ◦ above the horizontal,
how much downward force is being exerted on the ground?

2.9. An airplane is heading northeast at a velocity of 550 km/h. What is the northward component of its velocity? The
eastward component?

2.10. A boat heads northwest at 20 km/h in a river that flows east at 6 km/h. What are the magnitude and direction of the
boat’s velocity relative to the earth’s surface?

2.11. An airplane whose velocity is 120 km/h has just taken off from a runway. A car driving at 100 km/h on the runway is
able to remain just below the airplane. At what angle is the airplane climbing?

2.12. A boat moving at 36 km/h is crossing a river in which the current is flowing at 12 km/h. In what direction should the
boat head if it is to reach a point on the other side of the river directly opposite its starting point?

2.13. An airplane flies 400 km west from city A to city B, then 300 km northwest to city C, and finally 100 km north to
city D. How far it is from city A to city D? In what direction must the airplane head to return directly to city A from
city D?

2.14. Find the magnitude and direction of the resultant of a 5-N force that acts at an angle of 37 ◦ clockwise from the +x
axis, a 3-N force that acts at an angle of 180 ◦ clockwise from the +x axis, and a 7-N force that acts at an angle of
225 ◦ clockwise from the +x axis.

2.15. Find the magnitude and direction of the resultant of a 60-N force that acts at an angle of 45 ◦ clockwise from the +y
axis, a 20-N force that acts at an angle of 90 ◦ clockwise from the +y axis, and a 40-N force that acts at an angle of
300 ◦ clockwise from the +y axis.


\end{comment}



\endinput

