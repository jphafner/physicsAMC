
%%--------------------------------------------------
%% Schaum's Outline of Applied Physics
%%--------------------------------------------------


%% Chapter 4: Motion in a Vertical Plane
%%--------------------------------------------------


%% Schaum's Numeric Problems
%%--------------------------------------------------
\element{schaums-num}{
\begin{questionmultx}{ch04-num-Q01}
    Find the length, $l$, of the hypotenuse of a right triangle whose legs are \SI{483}{\meter} and \SI{620}{\meter} long.
    Report $l/\si{\meter}$ below.
    \AMCnumericChoices{785.93}{
        vertical=false,
        digits=5,decimals=2,sign=true,
        borderwidth=0pt,backgroundcol=white,approx=5
    }
\end{questionmultx}
}


4.1. A hunter aims a rifle directly at a squirrel on a branch of a tree. The squirrel sees the flash of the rifle’s firing. Should
the squirrel stay where it is or drop from the branch in free fall the moment the rifle is fired?
4.2. A stone is dropped from the edge of a cliff. (a) What is its velocity in meters per second 3 s later? (b) How far does it
fall in this time? (c) How far will it fall in the next second?
4.3. A girl throws a ball 20 m vertically into the air. (a) How long does she have to wait to catch it on the way down?
(b) What was its initial velocity? (c) What will be its final velocity?
4.4. The Sears Tower in Chicago is 1454 ft high. (a) How long would it take an object dropped from the top of the building
to reach the ground? (b) What would the object’s final velocity be?
4.5. A body in free fall reaches the ground in 5 s. (a) From what height in meters was it dropped? (b) What is its final
velocity? (c) How far did it fall in the last second of its descent?
4.6. The acceleration of gravity on the surface of Mars is 3.7 m/s 2 . If a ball is thrown vertically downward at 10 m/s from
a cliff on Mars, (a) What will its speed be after 1 s? (b) After 3 s?
4.7. A ball is thrown vertically upward with a velocity of 12 m/s. (a) At what height is the ball 1 s later? (b) 2 s later?
(c) What is the maximum height the ball reaches?
4.8. A ball is thrown horizontally with a velocity of 12 m/s. (a) How far has the ball fallen 1 s later? (b) 2 s later?
4.9. A ball is thrown vertically downward with a velocity of 12 m/s. (a) How far has the ball fallen 1 s later? (b) 2 s later?
4.10. A pear is thrown vertically downward from a cliff 48 m high and reaches the foot of the cliff 2 s later. Find the initial
velocity of the pear.
4.11. A rifle with a muzzle velocity of 200 m/s is fired with its barrel horizontally at a height of 1.5 m above the ground. (a)
How long is the bullet in the air? (b) How far away from the rifle does the bullet strike the ground? (c) If the muzzle
velocity were 150 m/s, would there be any difference in these answers?
4.12. A ball is rolled off the edge of a table 3 ft high with a horizontal velocity of 4 ft/s. With what velocity does it strike
the floor?
4.13. A rescue line is to be thrown horizontally from the upper deck of a ship 30 m above sea level to a lifeboat 30 m away.
What should the velocity of the line be?
4.14. If the initial velocity of a projectile were doubled, how would its maximum range be affected?
4.15. A crossbow can shoot an arrow 100 m vertically upward. (a) What is its maximum horizontal range? (b) With what
velocity will the arrow strike the ground when shot vertically upward and when shot so as to have the maximum range?


4.16. A shell is fired at an angle of 40 ◦ above the horizontal at a velocity of 300 m/s. (a) What is its range? (b) What is its
time of flight?
4.17. A golf ball leaves the club 40 m/s at an angle of 55 ◦ above the horizontal. (a) What is its range? (b) What is its time
of flight?
4.18. A rifle bullet has a muzzle velocity of 200 m/s. (a) At what angle should the rifle be pointed to achieve the maximum
range? (b) What is that range? (c) What is the time of flight at the maximum range?
4.19. An arrow leaves a bow at 25 m/s. (a) What is its maximum range? (b) At what two angles can it be sent above the
horizontal to reach a target 50 m away?
4.20. A soccer ball that was kicked at an angle of 40 ◦ above the horizontal strikes the ground 25 m away. What was its
initial velocity?


4.1. The squirrel should stay where it is. If it drops, it will fall with the same acceleration as the bullet and so will be struck.
4.2. (a) 29.4 m/s (b) 44.1 m (c) 34.3 m
4.3. (a) 4.04 s (b) 19.8 m/s (c) 19.8 m/s
4.4. (a) 9.5 s (b) 305 ft/s
4.5. (a) 123 m (b) 49 m/s (c) 44 m
4.6. (a) 13.7 m/s (b) 43.3 m/s
4.7. (a) 7.1 m (b) 4.4 m (c) 7.3 m
4.8. (a) 4.9 m (b) 19.6 m
4.9. (a) 16.9 m (b) 43.6 m


4.10. 14 m/s
4.11. (a) 0.56 s (b) 112 m (c) The time would be unchanged since it depends only on the height of the rifle, but the distance would be reduced to 84 m.
4.12. 14.4 ft/s
4.13. 12 m/s
4.14. The maximum range would be four times greater.
4.15. (a) 200 m (b) 44 m/s; 44 m/s
4.16. (a) 9.044 km (b) 39 s
4.17. (a) 153 m (b) 6.7 s
4.18. (a) 45 ◦ (b) 4.08 km (c) 29
4.19. (a) 64 m (b) 26◦, 64 ◦
4.20. 16 m/s




\endinput

