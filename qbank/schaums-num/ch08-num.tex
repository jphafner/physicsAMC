
%%--------------------------------------------------
%% Schaum's Outline of Applied Physics
%%--------------------------------------------------


%% Chapter 8: Momentum
%%--------------------------------------------------


%% Schaum's Numeric Problems
%%--------------------------------------------------
\element{schaums-num}{
\begin{questionmultx}{ch08-num-Q01}
    Find the length, $l$, of the hypotenuse of a right triangle whose legs are \SI{483}{\meter} and \SI{620}{\meter} long.
    Report $l/\si{\meter}$ below.
    \AMCnumericChoices{785.93}{
        vertical=false,
        digits=5,decimals=2,sign=true,
        borderwidth=0pt,backgroundcol=white,approx=5
    }
\end{questionmultx}
}


8.1. Find the momentum of a 100-kg ostrich running at 15 m/s.
8.2. Find the momentum of a 3200-lb car moving at 60 mi/h (88 ft/s).
8.3. An object at rest explodes and breaks up into two parts that fly off. Must they move in opposite directions?
8.4. A moving object strikes a stationary one. After the collision, must they move in the same direction?
8.5. A 2500-kg truck crashes into a wall at 40 km/h and comes to a stop in 0.5 s. What is the average force on the truck?
8.6. A 5-kg rifle and a 7-kg rifle fire identical bullets with the same muzzle velocities. Compare the recoil momenta and recoil velocities of the two rifles.
8.7. An empty dump truck is coasting with its engine off along a level road when rain starts to fall.
    (a) Neglecting friction, what (if anything) happens to the velocity of the truck?
    (b) The rain stops and the collected water leaks out. What (if anything) happens to the velocity of the truck now?
8.8. A 60-kg woman dives horizontally from a 250-kg boat with a velocity of 2 m/s. What is the recoil velocity of the boat?
8.9. Four 50-kg girls simultaneously dive horizontally at 2.5 m/s from the same side of a boat, whose recoil velocity is 0.1 m/s. What is the mass of the boat?
8.10. A spacecraft’s motors provide a total thrust of 1.5 MN. If the exhaust speed is 2.5 km/s, at what rate is fuel being used?


8.11. A rocket whose initial weight is 10,000 lb uses 50 lb/s of fuel that it exhausts at 8000 ft/s. Find the initial acceleration
of the rocket.
8.12. An unoccupied 1200-kg car has coasted down a hill and is moving along a level road at 10 m/s. A 6000-kg truck
moving in the opposite direction collides head-on with it. What was the truck’s velocity if both vehicles come to a
stop after the collision?
8.13. A 50-kg boy at rest on roller skates catches a 0.6-kg ball moving toward him at 30 m/s. How fast does he move
backward as a result?
8.14. A 1200-kg car traveling at 10 m/s overtakes a 1000-kg car traveling at 8 m/s and collides with it. (a) If the two cars
stick together, what is their final velocity? (b) How much kinetic energy is lost? What percentage of the original KE
is this?
8.15. The cars of Prob. 8.14 are moving in opposite directions and collide head-on. (a) If they stick together, what is their
final velocity? (b) How much kinetic energy is lost? What percentage of the original KE is this?
8.16. A 1-lb stone moving south at 8 ft/s collides with a 5-lb lump of clay moving west at 3 ft/s and becomes embedded in
the clay. Find the velocity (magnitude and direction) of the composite body.
8.17. A 1200-kg car moving east at 30 km/h collides with an 1800-kg car moving north at 20 km/h. The cars stick together
after the collision. Find the velocity (magnitude and direction) of the wreckage.
8.18. A 10-kg iron ball rolling at 2 m/s on a horizontal surface strikes a 1-kg wooden ball of the same size that is at rest.
(a) What is the velocity of each ball after the collision? (b) What proportion of the iron ball’s initial KE was transferred
to the wooden ball?
8.19. A 46-g golf ball flies off at 1.5 times the velocity of the clubhead that struck it. (a) What was the mass of the clubhead?
(b) What proportion of the clubhead’s initial KE was transferred to the ball?
8.20. A 5-kg ball moving at 6 m/s strikes a 3-kg ball initially at rest. The 5-kg ball continues moving in the same direction
at 2 m/s. (a) Find the velocity and direction of the 3-kg ball. (b) Find the coefficient of restitution.
8.21. A 5-kg ball moving to the right at 3 m/s overtakes and collides with a 10-kg ball moving to the right at 2.4 m/s. Find
the final velocities of the two balls if the coefficient of restitution is 0.8.
8.22. A rubber ball is dropped on the ground from a height of 5 m. If the coefficient of restitution is 0.7, find the height to
which the ball rebounds.
8.23. A ball dropped from a height of 3 m bounces up to a height of 1 m. Find the coefficient of restitution.



8.1. 1500 kg·m/s
8.2. 8800 slug·ft/s
8.3. Yes
8.4. No; no; they can move in any direction provided that the vector sum of their momenta equals the initial momentum of the first object.
8.5. 18,500 N
8.6. The recoil momenta are the same, but the lighter rifle has a higher recoil velocity.
8.7. (a) The truck’s velocity decreases as rainwater accumulates in it, since the total momentum must remain constant despite the increase in mass.
    (b) The reduced velocity does not change because the water that leaks out carries with it the momentum it had gained.
8.8. 0.48 m/s
8.9. 5000 kg
8.10. 600 kg/s
8.11. 8 ft/s 2
8.12. 2 m/s
8.13. 0.36 m/s
8.14. (a) 9.09 m/s (b) 1100 J; 1.2%
8.15. (a) 1.82 m/s (b) 88.4 kJ; 96%
8.16. 2.83 ft/s at 28 ◦ south of west
8.17. 17 km/h to the northeast
8.18. (a) v 1 = 1.6 m/s, v 2 = 3.6 m/s (b) 0.33 = 33%
8.19. (a) 138 g (b) 0.75 = 75%
8.20. (a) 6.67 m/s in the same direction as the 5-kg ball (b) 0.78
8.21. 2.3 m/s; 2.8 m/s
8.22. 2.45 m
8.23. 0.58



\endinput

