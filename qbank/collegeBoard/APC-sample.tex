

%% this section contains XX problems
%%----------------------------------------


%% AP Physics C: Sample Questions
%%----------------------------------------
\element{AP}{
\begin{question}{APC-sample-Q01}
    The speed $v$ of an automobile moving on a straight road is given in meters per second as a function of time $t$ in seconds by the following equation:
    \begin{equation*}
        v\left( t \right) = 4 + 2 t^3
    \end{equation*}
    What is the acceleration of the automobile at $t=\SI{2}{\second}$?
    \begin{multicols}{2}
    \begin{choices}
        \wrongchoice{\SI{12}{\meter\per\second\squared}}
        \wrongchoice{\SI{16}{\meter\per\second\squared}}
        \wrongchoice{\SI{20}{\meter\per\second\squared}}
      \correctchoice{\SI{24}{\meter\per\second\squared}}
        \wrongchoice{\SI{28}{\meter\per\second\squared}}
    \end{choices}
    \end{multicols}
\end{question}
}

\element{AP}{
\begin{question}{APC-sample-Q02}
    The speed $v$ of an automobile moving on a straight road is given in meters per second as a function of time $t$ in seconds by the following equation:
    \begin{equation*}
        v\left( t \right) = 4 + 2 t^3
    \end{equation*}
    How far has the automobile traveled in the interval between $t=\SI{0}{\second}$ and $t=\SI{2}{\second}$?
    \begin{multicols}{2}
    \begin{choices}
      \correctchoice{\SI{16}{\meter}}
        \wrongchoice{\SI{20}{\meter}}
        \wrongchoice{\SI{24}{\meter}}
        \wrongchoice{\SI{32}{\meter}}
        \wrongchoice{\SI{72}{\meter}}
    \end{choices}
    \end{multicols}
\end{question}
}

\element{AP}{
\begin{question}{APC-sample-Q03}
    If a particle moves in a plane so that its position is described by the functions
        $x = A\cos\omega t$ and $y = A\sin\omega t$, the particle is:
    \begin{choices}
      \correctchoice{moving with constant speed along a circle}
        \wrongchoice{moving with varying speed along a circle}
        \wrongchoice{moving with constant acceleration along a straight line}
        \wrongchoice{moving along a parabola}
        \wrongchoice{oscillating back and forth along a straight line}
    \end{choices}
\end{question}
}

\element{AP}{
\begin{question}{APC-sample-Q04}
    A system in equilibrium consists of an object of weight $W$ that hangs from three ropes,
        as shown below. 
    \begin{center}
    \begin{tikzpicture}
        %% NOTE: insert diagram
    \end{tikzpicture}
    \end{center}
    The tensions in the ropes are $_1$, $T_2$, and $T_3$. 
    Which of the following are correct values of $T_2$ and $T_3$?
    \begin{choices}
        \wrongchoice{$T_2 = W\tan\ang{60}$, $T_3 = \dfrac{W}{\cos\ang{60}}$}
        \wrongchoice{$T_2 = W\tan\ang{60}$, $T_3 = \dfrac{W}{\sin\ang{60}}$}
        \wrongchoice{$T_2 = W\tan\ang{60}$, $T_3 = W\sin\ang{60}$}
        \wrongchoice{$T_2 = \dfrac{W}{\tan\ang{60}}$, $T_3 = \dfrac{W}{\cos\ang{60}}$}
      \correctchoice{$T_2 = \dfrac{W}{\tan\ang{60}}$, $T_3 = \dfrac{W}{\sin\ang{60}}$}
    \end{choices}
\end{question}
}

\element{AP}{
\begin{question}{APC-sample-Q05}
    The constant force $\vec{F}$ with components $F_x=\SI{3}{\newton}$ and $F_y=\SI{4}{\newton}$,
        shown below, acts on a body while that body moves from the point
        $P(x=\SI{2}{\meter}, y=\SI{2}{\meter})$ to the point $Q(x=\SI{14}{\meter},y=\SI{1}{\meter})$.
    \begin{center}
    \begin{tikzpicture}
        %% NOTE: insert diagram
    \end{tikzpicture}
    \end{center}
    How much work does the force do on the body during this process?
    \begin{multicols}{2}
    \begin{choices}
      \correctchoice{\SI{16}{\joule}}
        \wrongchoice{\SI{30}{\joule}}
        \wrongchoice{\SI{46}{\joule}}
        \wrongchoice{\SI{56}{\joule}}
        \wrongchoice{\SI{65}{\joule}}
    \end{choices}
    \end{multicols}
\end{question}
}

\element{AP}{
\begin{question}{APC-sample-Q06}
    The sum of all the external forces on a system of particles is zero. 
    Which of the following must be true of the system?
    \begin{choices}
        \wrongchoice{The total mechanical energy is constant.j}
        \wrongchoice{The total potential energy is constant.j}
        \wrongchoice{The total kinetic energy is constant.j}
      \correctchoice{The total linear momentum is constant.j}
        \wrongchoice{It is in static equilibrium.j}
    \end{choices}
\end{question}
}

\element{AP}{
\begin{question}{APC-sample-Q07}
    A toy cannon is fixed to a small cart and both move to the right with speed $v$ along a straight track, as shown below.
    \begin{center}
    \begin{tikzpicture}
        %% NOTE: insert diagram
    \end{tikzpicture}
    \end{center}
    The cannon points in the direction of motion.
    When the cannon fires a projectile the cart and cannon are brought to rest. 
    If $M$ is the mass of the cart and cannon combined without the projectile,
        and $m$ is the mass of the projectile,
        what is the speed of the projectile relative to the ground immediately after it is fired?
    \begin{multicols}{2}
    \begin{choices}
        \wrongchoice{$\dfrac{Mv}{m}$}
      \correctchoice{$\dfrac{(M+m)v}{m}$}
        \wrongchoice{$\dfrac{(M-m)v}{m}$}
        \wrongchoice{$\dfrac{m}{M}$}
        \wrongchoice{$\dfrac{mv}{M-m}$}
    \end{choices}
    \end{multicols}
\end{question}
}

\element{AP}{
\begin{question}{APC-sample-Q08}
    A disk $X$ rotates freely with angular velocity $\omega$ on frictionless bearings,
        as shown below. 
    \begin{center}
    \begin{tikzpicture}
        %% NOTE: insert diagram
    \end{tikzpicture}
    \end{center}
    A second identical disk $Y$,
        initially not rotating,
        is placed on $X$ so that both disks rotate together without slipping. 
    When the disks are rotating together,
        which of the following is half what it was before?
    \begin{choices}
        \wrongchoice{Moment of inertia of $X$}
        \wrongchoice{Moment of inertia of $Y$}
      \correctchoice{Angular velocity of $X$}
        \wrongchoice{Angular velocity of $Y$}
        \wrongchoice{Angular momentum of both disks}
    \end{choices}
\end{question}
}

\element{AP}{
\begin{question}{APC-sample-Q09}
    The ring and the disk shown below have identical masses, radii, and velocities,
        and are not attached to each other. 
    \begin{center}
    \begin{tikzpicture}
        %% NOTE: insert diagram
    \end{tikzpicture}
    \end{center}
    If the ring and the disk each roll without slipping up an inclined plane,
        how will the distances that they move up the plane before coming to rest compare?
    \begin{choices}
      \correctchoice{The ring will move farther than will the disk.}
        \wrongchoice{The disk will move farther than will the ring.}
        \wrongchoice{The ring and the disk will move equal distances.}
        \wrongchoice{The relative distances depend on the angle of elevation of the plane.}
        \wrongchoice{The relative distances depend on the length of the plane.}
    \end{choices}
\end{question}
}

\element{AP}{
\begin{question}{APC-sample-Q10}
    Let $g$ be the acceleration due to gravity at the surface of a planet of radius $R$. 
    Which of the following is a dimensionally correct formula for the minimum kinetic energy $K$ that a projectile of mass m must have at the planet’s surface if the projectile is to escape from the planet's gravitational field?
    \begin{multicols}{2}
    \begin{choices}
        \wrongchoice{$K=\sqrt{gR}$}
      \correctchoice{$K=mgR$}
        \wrongchoice{$K=\dfrac{mg}{R}$}
        \wrongchoice{$K=m\sqrt{\dfrac{g}{R}}$}
        \wrongchoice{$K=gR$}
    \end{choices}
    \end{multicols}
\end{question}
}


\endinput


