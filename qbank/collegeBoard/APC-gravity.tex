
%% Cracking the AP Physics B and C
%%----------------------------------------

%% Chapter 07: Gravity
%%----------------------------------------
\element{AP}{
\begin{question}{gravity-Q01}
    If the distance between two point paticles is doubled,
        then the gravitational force between them:
    \begin{choices}
        \wrongchoice{decreases by a factor of $4$.}
        \wrongchoice{decreases by a factor of $2$.}
        \wrongchoice{increases by a factor of $2$.}
        \wrongchoice{increases by a factor of $4$.}
        \wrongchoice{Cannot be determined without knowing the masses.}
    \end{choices}
\end{question}
}

\element{AP}{
\begin{question}{gravity-Q02}
    At the surface of the earth, and object of mass $m$ has weight $w$.
    If this object is transported to an altitude that's twice the radius of the earth,
        then, at the new location,
    \begin{choices}
        \wrongchoice{its mass is $m/2$ and its weight is $w/2$.}
        \wrongchoice{its mass is $m$ and its weight is $w/2$.}
        \wrongchoice{its mass is $m/2$ and its weight is $w/4$.}
      \correctchoice{its mass is $m$ and its weight is $w/4$.}
        \wrongchoice{its mass is $m$ and its weight is $w/9$.}
    \end{choices}
\end{question}
}

\element{AP}{
\begin{question}{gravity-Q03}
    A moon of mass $m$ orbits a planet of mass $100 m$.
    Let the strength of the gravitational force exerted by the planet on the moon be denoted by $F_1$,
        and let the strength of the gravitational force exerted by the moon on the planet $F_2$.
    Which of the following is true?
    \begin{multicols}{2}
    \begin{choices}
        \wrongchoice{$F_1 = 100 F_2$}
        \wrongchoice{$F_1 = 10 F_2$}
        \wrongchoice{$F_1 = F_2$}
        \wrongchoice{$F_2 = 10 F_1$}
        \wrongchoice{$F_2 = 100 F_1$}
    \end{choices}
    \end{multicols}
\end{question}
}

\element{AP}{
\begin{question}{gravity-Q04}
    The planet Pluto has $1/500$ the mass and $1/15$ the radius of Earth.
    What is the value of $g$ (in \si{\meter\per\second\squared}) on the surface of Pluto?
    \begin{multicols}{2}
    \begin{choices}
        \wrongchoice{\SI{50/225}{\meter\per\second\squared}}
        \wrongchoice{\SI{50/15}{\meter\per\second\squared}}
        \wrongchoice{\SI{15/50}{\meter\per\second\squared}}
        \wrongchoice{\SI{225/50}{\meter\per\second\squared}}
        %% NOTE: E is a dup and needs to be changed
        \wrongchoice{\SI{50/225}{\meter\per\second\squared}}
    \end{choices}
    \end{multicols}
\end{question}
}

\element{AP}{
\begin{question}{gravity-Q05}
    A satellite is currently orbitting Earth in a circular orbit of radius $R$,
        its kinetic energy is $K_1$.
    If the satellite is moved and enters a new circular orbit of radius $2R$,
        what will be its new kinetic energy?
    \begin{multicols}{2}
    \begin{choices}
        \wrongchoice{$\dfrac{K_1}{4}$}
        \wrongchoice{$\dfrac{K_1}{2}$}
        \wrongchoice{$K_1$}
        \wrongchoice{$2 K_1$}
        \wrongchoice{$4 K_1$}
    \end{choices}
    \end{multicols}
\end{question}
}

\element{AP}{
\begin{question}{gravity-Q06}
    A moon of Jupiter has a nearly circular orbit of radius $R$
        and an orbit period of $T$.
    Which of the following expressions gives the mass of Jupiter?
    \begin{multicols}{2}
    \begin{choices}
        \wrongchoice{$\dfrac{2\pi R}{T}$}
        \wrongchoice{$\dfrac{4\pi^2 R}{T^2}$}
        \wrongchoice{$\dfrac{2\pi R^2}{GT^2}$}
        \wrongchoice{$\dfrac{4\pi R^2}{GT^2}$}
        \wrongchoice{$\dfrac{2\pi^2 R^3}{GT^2}$}
    \end{choices}
    \end{multicols}
\end{question}
}

\element{AP}{
\begin{question}{gravity-Q07}
    Two large bodies, Body $A$ of mass $m$ and body $B$ of mass $4m$,
        are separated by a distance $R$.
    At what distance from body $A$, along teh line joining the bodies,
        would the gravitational force on the object be equal to zero?
    (Ignore the presence of any other bodies.)
    \begin{multicols}{2}
    \begin{choices}
        \wrongchoice{$\dfrac{R}{16}$}
        \wrongchoice{$\dfrac{R}{8}$}
        \wrongchoice{$\dfrac{R}{5}$}
        \wrongchoice{$\dfrac{R}{4}$}
        \wrongchoice{$\dfrac{R}{3}$}
    \end{choices}
    \end{multicols}
\end{question}
}

\element{AP}{
\begin{question}{gravity-Q08}
    The mean distance from saturn to the Sun is 9 times greater than the
        mean distance from Earth to the Sun.
    How long is a Saturn year?
    \begin{multicols}{2}
    \begin{choices}
        \wrongchoice{\num{18} Earth years}
        \wrongchoice{\num{27} Earth years}
        \wrongchoice{\num{81} Earth years}
        \wrongchoice{\num{243} Earth years}
        \wrongchoice{\num{729} Earth years}
    \end{choices}
    \end{multicols}
\end{question}
}

\element{AP}{
\begin{question}{gravity-Q09}
    The Moon has mass $M$ and radius $R$.
    A small object is dropped from a distance of $3R$ from the Moon's center.
    The object's impact speed when it strikes the surface of teh Moon is equal to
    \begin{equation*}
        \sqrt{\frac{kGM}{R}}\quad\text{for k =}
    \end{equation*}
    \begin{multicols}{2}
    \begin{choices}
        \wrongchoice{$\dfrac{1}{3}$}
        \wrongchoice{$\dfrac{2}{3}$}
        \wrongchoice{$\dfrac{3}{4}$}
        \wrongchoice{$\dfrac{4}{3}$}
        \wrongchoice{$\dfrac{3}{2}$}
    \end{choices}
    \end{multicols}
\end{question}
}

\element{AP}{
\begin{question}{gravity-Q10}
    A planet orbits the Sun in an elliptical orbit of eccentricity $e$.
    What is the radio of the planet's speed at the periohelium to its speed at aphelion?
    \begin{multicols}{2}
    \begin{choices}
        \wrongchoice{$\dfrac{1}{1-e}$}
        \wrongchoice{$\dfrac{e}{1-e}$}
        \wrongchoice{$\dfrac{1}{1+e}$}
        \wrongchoice{$\dfrac{e}{1+e}$}
        \wrongchoice{$\dfrac{1+e}{1-e}$}
    \end{choices}
    \end{multicols}
\end{question}
}


\endinput


