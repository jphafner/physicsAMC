

%% Cracking the AP Physics B and C
%%----------------------------------------

%% Chapter 08: Oscillations
%%----------------------------------------
\element{AP}{
\begin{questionmult}{oscillations-Q01}
    Which of the following is/are characteristics of simple harmonic motion?
    \begin{choices}
        \wrongchoice{The acceleration is constant.}
      \correctchoice{The restoring force is proportional to the displacement.}
      \correctchoice{The frequency is independent of the amplitude.}
    \end{choices}
\end{questionmult}
}

\element{AP}{
\begin{question}{oscillations-Q02}
    A block attached to an ideal spring undergoes simple harmonic motion.
    The acceleration of the block has its maximum magnitude at the point where:
    \begin{choices}
        \wrongchoice{The speed is the maximum.}
        \wrongchoice{The potential energy is the minimum.}
      \correctchoice{The speed is the minimum.}
        \wrongchoice{The restoring force is the minimum.}
        \wrongchoice{The kinetic energy is the maximum.}
    \end{choices}
\end{question}
}

\element{AP}{
\begin{question}{oscillations-Q03}
    A block attached to an ideal spring undergoes simple harmonic motion
        about its equilibrium position ($x=0$) with amplitude $A$.
    What fraction of the total energy is in the form of kinetic energy
        when the block is at position $x=\frac{1}{2}A$?
    \begin{multicols}
    \begin{choices}
        \wrongchoice{$\frac{1}{3}$}
        \wrongchoice{$\frac{3}{8}$}
        \wrongchoice{$\frac{1}{2}$}
        \wrongchoice{$\frac{2}{3}$}
        \wrongchoice{$\frac{3}{4}$}
    \end{choices}
    \end{multicols}
\end{question}
}

\element{AP}{
\begin{question}{oscillations-Q04}
    A student measures the maximum speed of a block undergoing simple harmonic oscillations
        of amplitude $A$ on the end of an idea spring.
    If the block is replaced by one with twice the mass but the amplitude of its oscillations
        remains the same, then the maximum speed of the block will:
    \begin{choices}
        \wrongchoice{decrease by a factor of $4$.}
        \wrongchoice{decrease by a factor of $2$.}
        \wrongchoice{decrease by a factor of $\sqrt{2}$.}
        \wrongchoice{remain the same.}
        \wrongchoice{increase by a factor of $2$.}
    \end{choices}
\end{question}
}

\element{AP}{
\begin{question}{oscillations-Q05}
    A spring-block simple harmonic oscillator is set up so that the oscillations are vertical.
    The period of the motion is $T$.
    If the spring and block are taken to the surface of the Moon,
        where the gravitational acceleration is $\frac{1}{6}$ of its values here,
        then the vertical oscillations will have a period of:
    \begin{multicols}{2}
    \begin{choices}
        \wrongchoice{$\dfrac{T}{6}$}
        \wrongchoice{$\dfrac{T}{3}$}
        \wrongchoice{$\dfrac{T}{\sqrt{6}}$}
      \correctchoice{$T$}
        \wrongchoice{$\sqrt{6}T$}
    \end{choices}
    \end{multicols}
\end{question}
}

\element{AP}{
\begin{question}{oscillations-Q06}
    A linear spring of force ocnstant $k$ is used in a physics lab experiment.
    A block of mass $m$ is attached to the spring and the resulting frequency, $f$,
        of the simple harmonic oscillations is measured.
    Blocks of various masses are used in different trials,
        and in each case, the corresponding frequency is measured and recorded.
    If $f^2$ is plotted versus $1/m$, the graph will be a straight line with slope:
    \begin{multicols}{2}
    \begin{choices}
        \wrongchoice{$\dfrac{4\pi^2}{k^2}$}
        \wrongchoice{$\dfrac{4\pi^2}{k}$}
        \wrongchoice{$4\pi^2k$}
        \wrongchoice{$\dfrac{k}{4\pi^2}$}
        \wrongchoice{$\dfrac{k^2}{4\pi^2}$}
    \end{choices}
    \end{multicols}
\end{question}
}

\element{AP}{
\begin{question}{oscillations-Q07}
    A block of mass $m=\SI{4}{\kilo\gram}$ on a frictionless,
        horizontal table is attached to one end of a spring of
        force constant $k=\SI{400}{\newton\per\meter}$ and undergoes
        simple harmonic oscillations about its equilibrium position
        ($x=0$) with amplitude of $A=\SI{6}{\centi\meter}$.
    If the block is at $x=\SI{6}{\centi\meter}$ at time $t=0$,
        then which of the following equations
        (with $x$ in centimeters and $t$ in seconds)
        gives the block's position as a function of time?
    \begin{multicols}{2}
    \begin{choices}
        \wrongchoice{$x=6\sin\left(10 t + \frac{1}{2}\pi\right)$}
        \wrongchoice{$x=6\sin\left(10\pi t + \frac{1}{2}\pi\right)$}
        \wrongchoice{$x=6\sin\left(10\pi t - \frac{1}{2}\pi\right)$}
        \wrongchoice{$x=6\sin\left(10 t\right)$}
        \wrongchoice{$x=6\sin\left(10 t - \frac{1}{2}\pi\right)$}
    \end{choices}
    \end{multicols}
\end{question}
}

\element{AP}{
\begin{question}{oscillations-Q08}
    A block attached to an ideal spring undergoes simple harmonic motion
        about its equilibrium position with amplitude $A$ and angular frequency $\omega$.
    What is the maximum magnitude of the block's velocity?
    \begin{multicols}{2}
    \begin{choices}
        \wrongchoice{$A\omega$}
        \wrongchoice{$A^2\omega$}
        \wrongchoice{$A\omega^2$}
        \wrongchoice{$\dfrac{A}{\omega}$}
        \wrongchoice{$\dfrac{A}{\omega^2}$}
    \end{choices}
    \end{multicols}
\end{question}
}

\element{AP}{
\begin{question}{oscillations-Q09}
    A simple pendulum swings about the vertical equilibrium position
        with a maximum angular displacement of \ang{5} and period $T$.
    If the same pendulum is given a maximum angular displacement of \ang{10},
        then which of the following best gives the period of the oscillations?
    \begin{multicols}{2}
    \begin{choices}
        \wrongchoice{$\dfrac{T}{2}$}
        \wrongchoice{$\dfrac{T}{\sqrt{2}}$}
        \wrongchoice{$T$}
        \wrongchoice{$\sqrt{2}T$}
        \wrongchoice{$2T$}
    \end{choices}
    \end{multicols}
\end{question}
}

\element{AP}{
\begin{question}{oscillations-Q10}
    A simple pendulum of lenght $L$ and mass $m$ swings about the vertical
        equilibrium position $(\theta=0)$ with a maximum angular displacement of $\theta_{max}$.
    What is the tension in the connecting rod when the pendulum's angular displacement is $\theta=\theta_{max}$?
    \begin{multicols}{2}
    \begin{choices}
        \wrongchoice{$mg\sin\theta_{max}$}
        \wrongchoice{$mg\cos\theta_{max}$}
        \wrongchoice{$mgL\sin\theta_{max}$}
        \wrongchoice{$mgL\cos\theta_{max}$}
        \wrongchoice{$mgL\left(1-\cos\theta_{max}\right)$}
    \end{choices}
    \end{multicols}
\end{question}
}


\endinput


