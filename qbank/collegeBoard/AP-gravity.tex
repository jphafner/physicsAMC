

%% this section contains XX problems
%%----------------------------------------


%% Jacobs 5 steps to a 5
%%------------------------------
\element{AP}{
\begin{question}{Jacobs-Q16}
    A satellite orbits the moon far from its surface in
        a circle of radius $r$.
    If a second satellite has a greater speed,
        yet still needs to maintain a circular orbit around
        the moon, how should the second satellite orbit?
    \begin{multicols}{2}
    \begin{choices}
        \wrongchoice{with a radius of $r$}
        \wrongchoice{with a radius greater than $r$}
      \correctchoice{with a radius less than $r$}
        \wrongchoice{Only an eccentric elliptical orbit can
            be maintained with a larger speed.}
        \wrongchoice{No orbit at all can be maintained with
            a larger speed.}
    \end{choices}
    \end{multicols}
\end{question}
}


%% Sample AP 1 Questions
%%------------------------------
\element{AP}{
\begin{question}{sample1-Q01}
    Two solid spheres of radius $R$ made of the same steel are placed in contact,
        as shown in the figures below.
    \begin{center}
        \includegraphics[keepaspectratio]{sample-AP1-Q01}
    \end{center}
    The magnitude of the gravitational force that they exert on each other is $F_1$.
    When two other solid spheres of radius $3R$ made of this steel are placed
        in contact, what is the magnitude of the gravitational force that
        they exert on each other?
    \begin{multicols}{2}
    \begin{choices}
      \correctchoice{$81 F_1$}
        \wrongchoice{$9 F_1$}
        \wrongchoice{$3 F_1$}
        \wrongchoice{$  F_1$}
    \end{choices}
    \end{multicols}
\end{question}
}

\element{AP}{
\begin{question}{sample1-Q04}
    While traveling in its elliptical orbit around the Sun, Mars gains
        speed during the part of the orbit where it is getting closer
        to the Sun.
    Which of the following can be used to explain this gain in speed?
    \begin{choices}
      \correctchoice{As Mars gets closer to the Sun, th Mars-Sun system loses
                     potential energy and Mars gains kinetic energy.}
        \wrongchoice{A component of the gravitational force exerted on Mars
                     is perpendicular to the direction of motion, causing an
                     acceleration and hence a gain in speed along that direction.}
        \wrongchoice{The torque exerted on Mars by the Sun during this segment of
                     the orbit increases the Mars-Sun system's angular momentum}
        \wrongchoice{The centripetal force exerted on mars is greater than the
                     gravitational force during this segment of the orbit,
                     causing Mars to gain speed as it gets closer to the Sun.}
    \end{choices}
\end{question}
}


%% 2004-APB
%%------------------------------
\element{AP}{
\begin{question}{2004-APB-Q10}
    A new planet is discovered that has twice the Earth's mass and twice the
        Earth's radius.
    On the surface of this new planet, a person who weighs \SI{500}{\newton}
        on Earth would experience a gravitational force of
    \begin{multicols}{3}
    \begin{choices}
        \wrongchoice{\SI{125}{\newton}}
      \correctchoice{\SI{250}{\newton}}
        \wrongchoice{\SI{500}{\newton}}
        \wrongchoice{\SI{1000}{\newton}}
        \wrongchoice{\SI{2000}{\newton}}
    \end{choices}
    \end{multicols}
\end{question}
}

\element{AP}{
\begin{question}{2004-APB-Q67}
    A satellite of mass $m$ and speed $v$ moves in a stable,
        circular orbit around a planet of mass $M$.
    What is the radius of the satellite's orbit?
    \begin{multicols}{2}
    \begin{choices}
        \wrongchoice{$\dfrac{G M}{m v}$}
        \wrongchoice{$\dfrac{G v}{m M}$}
      \correctchoice{$\dfrac{G M}{v^2}$}
        \wrongchoice{$\dfrac{G m M}{v}$}
        \wrongchoice{$\dfrac{G m M}{v^2}$}
    \end{choices}
    \end{multicols}
\end{question}
}

