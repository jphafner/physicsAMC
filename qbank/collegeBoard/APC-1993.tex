
%% this section contains XX problems
%%----------------------------------------


%% AP Physics C: Mechanics 1993
%%------------------------------
\element{AP}{
\begin{question}{APC-1993-Q01}
    In the absence of air friction,
        an object dropped near the surface of the Earth experiences a
        constant acceleration of about \SI{9.8}{\meter\per\second\squared}.
    This means that the:
    \begin{choices}
        \wrongchoice{speed of the object increases \SI{9.8}{\meter\per\second} during each second.}
        \wrongchoice{speed of the object as it falls is \SI{9.8}{\meter\per\second}.}
        \wrongchoice{object falls \SI{9.8}{\meter} during each second.}
        \wrongchoice{object falls \SI{9.8}{\meter} during the first second only.}
        \wrongchoice{derivative of the distance with respect to time for the object equals \SI{9.8}{\meter\per\second\squared}.}
    \end{choices}
\end{question}
}

\element{AP}{
\begin{question}{APC-1993-Q02}
    A \SI{500}{\kilo\gram} sports car accelerates uniformly from rest,
        reaching a speed of \SI{30}{\meter\per\second} in \SI{6}{\second}.
    During the \SI{6}{\second}, the car has traveled a distance of:
    \begin{multicols}{2}
    \begin{choices}
        \wrongchoice{\SI{15}{\meter}}
        \wrongchoice{\SI{30}{\meter}}
        \wrongchoice{\SI{60}{\meter}}
        \wrongchoice{\SI{90}{\meter}}
        \wrongchoice{\SI{180}{\meter}}
    \end{choices}
    \end{multicols}
\end{question}
}

\element{AP}{
\begin{question}{APC-1993-Q03}
    At a particular instant,
        a stationary observer on the ground sees a package falling with
        speed $v_1$ at an angle to the vertical. 
    \begin{center}
    \begin{tikzpicture}
        %% NOTE: diagram
    \end{tikzpicture}
    \end{center}
    To a pilot flying horizontally at constant speed relative to the ground,
        the package appears to be falling vertically with a speed $v_2$ at that instant. 
    What is the speed of the pilot relative to the ground?
    \begin{multicols}{2}
    \begin{choices}
        \wrongchoice{$v_1 + v_2$}
        \wrongchoice{$v_1 - v_2$}
        \wrongchoice{$v_2 - v_1$}
        \wrongchoice{$\sqrt{v_1^2 - v_2^2}$}
        \wrongchoice{$\sqrt{v_1^2 + v_2^2}$}
    \end{choices}
    \end{multicols}
\end{question}
}

\element{AP}{
\begin{question}{APC-1993-Q04}
    A ball initially moves horizontally with velocity $v_i$, as shown below. 
    \begin{center}
    \begin{tikzpicture}
        %% NOTE: diagram
    \end{tikzpicture}
    \end{center}
    It is then struck by a stick. 
    After leaving the stick, the ball moves vertically with a velocity $v_f$,
        which is smaller in magnitude than $v_i$.
    Which of the following vectors best represents the direction
        of the average force that the stick exerts on the ball?
    \begin{multicols}{2}
    \begin{choices}
        \wrongchoice{
            \begin{tikzpicture}
            \end{tikzpicture}
        }
    \end{choices}
    \end{multicols}
\end{question}
}

\element{AP}{
\begin{question}{APC-1993-Q05}
    If $F_1$ is the magnitude of the force exerted by the Earth
        on a satellite in orbit about the Earth and $F_2$ is the
        magnitude of the force exerted by the satellite on the Earth,
        then which of the following is true?
    \begin{choices}
        \wrongchoice{$F_1$ is much greater than $F_2$.}
        \wrongchoice{$F_1$ is slightly greater than $F_2$ .}
        \wrongchoice{$F_1$ is equal to $F_2$.}
        \wrongchoice{$F_2$ is slightly greater than $F_1$}
        \wrongchoice{$F_2$ is much greater than $F_1$}
    \end{choices}
\end{question}
}

\element{AP}{
\begin{question}{APC-1993-Q06}
    A ball is thrown upward. 
    At a height of 10 meters above the ground,
        the ball has a potential energy of \SI{50}{\joule}
        (with the potential energy equal to zero at ground level)
        and is moving upward with a kinetic energy of \SI{50}{\joule}.
    Air friction is negligible. 
    The maximum height reached by the ball is most nearly
    \begin{multicols}{2}
    \begin{choices}
        \wrongchoice{\SI{10}{\meter}}
        \wrongchoice{\SI{20}{\meter}}
        \wrongchoice{\SI{30}{\meter}}
        \wrongchoice{\SI{40}{\meter}}
        \wrongchoice{\SI{50}{\meter}}
    \end{choices}
    \end{multicols}
\end{question}
}
        
\newcommand{\APCNineteenNinetyThreeQEight}{
\begin{tikzpicture}
\end{tikzpicture}
}

\element{AP}{
\begin{question}{APC-1993-Q07}
    A block on a horizontal frictionless plane is attached to a spring,
        as shown below.
    The block oscillates along the
    $x$-axis with simple harmonic motion of amplitude $A$.
    \begin{center}
        \APCNineteenNinetyThreeQEight
    \end{center}
    Which of the following statements about the block is correct?
    \begin{choices}
        \wrongchoice{At $x = 0$, its velocity is zero.}
        \wrongchoice{At $x = 0$, its acceleration is at a maximum.}
        \wrongchoice{At $x = A$, its displacement is at a maximum.}
        \wrongchoice{At $x = A$, its velocity is at a maximum.}
        \wrongchoice{At $x = A$, its acceleration is zero.}
    \end{choices}
\end{question}
}

\element{AP}{
\begin{question}{APC-1993-Q08}
    A block on a horizontal frictionless plane is attached to a spring,
        as shown below.
    \begin{center}
        \APCNineteenNinetyThreeQEight
    \end{center}
    The block oscillates along the
    $x$-axis with simple harmonic motion of amplitude $A$.
    Which of the following statements about energy is correct?
    \begin{choices}
        \wrongchoice{The potential energy of the spring is at a minimum at $x = 0$.}
        \wrongchoice{The potential energy of the Spring is at a minimum at $x = A$.}
        \wrongchoice{The kinetic energy of the block is at a minimum at $x =0$.}
        \wrongchoice{The kinetic energy of the block is at a maximum at $x = A$.}
        \wrongchoice{The kinetic energy of the block is always equal to the potential energy of the spring.}
    \end{choices}
\end{question}
}

\element{AP}{
\begin{question}{APC-1993-Q09}
    Two \SI{0.60}{\kilo\gram} objects are connected by a thread that passes over a light,
        frictionless pulley, as shown below.
    \begin{center}
        \begin{tikzpicture}
        \end{tikzpicture}
    \end{center}
    The objects are initially held at rest. 
    If a third object with a mass of \SI{0.30}{\kilo\gram}
        is added on top of one of the \SI{0.60}{\kilo\gram}
        objects as shown and the objects are released,
        the magnitude of the acceleration of the \SI{0.30}{\kilo\gram} object is most nearly:
    \begin{multicols}{2}
    \begin{choices}
        \wrongchoice{\SI{10.0}{\meter\per\second\squared}}
        \wrongchoice{\SI{6.0}{\meter\per\second\squared}}
        \wrongchoice{\SI{3.0}{\meter\per\second\squared}}
        \wrongchoice{\SI{2.0}{\meter\per\second\squared}}
        \wrongchoice{\SI{1.0}{\meter\per\second\squared}}
    \end{choices}
    \end{multicols}
\end{question}
}

\element{AP}{
\begin{question}{APC-1993-Q10}
    During a certain time interval,
        a constant force delivers an average power of \SI{4}{\watt} to an object. 
    If the object has an average speed of \SI{2}{\meter\per\second}
        and the force acts in the direction of motion of the object,
        the magnitude of the force is
    \begin{multicols}{2}
    \begin{choices}
        \wrongchoice{\SI{16}{\newton}}
        \wrongchoice{\SI{8}{\newton}}
        \wrongchoice{\SI{6}{\newton}}
        \wrongchoice{\SI{4}{\newton}}
        \wrongchoice{\SI{2}{\newton}}
    \end{choices}
    \end{multicols}
\end{question}
}

\element{AP}{
\begin{question}{APC-1993-Q11}
    Two balls are on a frictionless horizontal tabletop. 
    Ball $X$ initially moves at \SI{10}{\meter\per\second},
        as shown in Figure I below. 
    It then collides elastically with identical ball $Y$,
        which is initially at rest. 
    \begin{center}
    \begin{tikzpicture}
        %% NOTE: diagram
    \end{tikzpicture}
    \end{center}
    After the collision, ball $X$ moves at \SI{6}{\meter\per\second}
        along a path at 53 0 to its original direction, as shown in Figure II above. 
    Which of the following diagrams best represents the motion of ball $Y$ after the collision?
    \begin{multicols}{2}
    \begin{choices}
        \wrongchoice{
            \begin{tikzpicture}
            \end{tikzpicture}
        }
    \end{choices}
    \end{multicols}
\end{question}
}

\newcommand{\APCNineteenNinetyThreeQTwelve}{
\begin{tikzpicture}
\end{tikzpicture}
}

\element{AP}{
\begin{question}{APC-1993-Q12}
    An ant of mass $m$ clings to the rim of a flywheel of radius $r$, as shown above. 
    The flywheel rotates clockwise on a horizontal shaft $S$ with constant angular velocity $\omega$.
    As the wheel rotates, the ant revolves past the stationary points I, II, III, and IV. 
    The ant can adhere to the wheel with a force much greater than its own weight.
    \begin{center}
        \APCNineteenNinetyThreeQTwelve
    \end{center}
    It will be most difficult for the ant to adhere to the wheel as it
        revolves past which of the four points?
    \begin{choices}
        \wrongchoice{I}
        \wrongchoice{II}
        \wrongchoice{III}
        \wrongchoice{IV}
        \wrongchoice{It will be equally difficult for the ant to adhere to the wheel at all points}
    \end{choices}
\end{question}
}

\element{AP}{
\begin{question}{APC-1993-Q13}
    An ant of mass $m$ clings to the rim of a flywheel of radius $r$, as shown above. 
    The flywheel rotates clockwise on a horizontal shaft $S$ with constant angular velocity $\omega$.
    As the wheel rotates, the ant revolves past the stationary points I, II, III, and IV. 
    The ant can adhere to the wheel with a force much greater than its own weight.
    \begin{center}
        \APCNineteenNinetyThreeQTwelve
    \end{center}
    What is the magnitude of the minimum adhesion force necessary
        for the ant to stay on the flywheel at point III?
    \begin{multicols}{2}
    \begin{choices}
        \wrongchoice{$mg$}
        \wrongchoice{$m\omega^2 r$}
        \wrongchoice{$m\omega^2 r 2 + mg$}
        \wrongchoice{$m\omega^2 r - mg$}
        \wrongchoice{$m\omega^2 r + mg$}
    \end{choices}
    \end{multicols}
\end{question}
}

\element{AP}{
\begin{question}{APC-1993-Q14}
    A weight lifter lifts a mass $m$ at constant speed to a height $h$ in time $t$.
    How much work is done by the weight lifter?
    \begin{multicols}{2}
    \begin{choices}
        \wrongchoice{$mg$}
        \wrongchoice{$mh$}
        \wrongchoice{$mgh$}
        \wrongchoice{$mght$}
        \wrongchoice{$mgh/t$}
    \end{choices}
    \end{multicols}
\end{question}
}

\element{AP}{
\begin{question}{APC-1993-Q15}
    A conservative force has the potential energy function $U(x)$,
        shown by the graph below.
    \begin{center}
    \begin{tikzpicture}
        %% NOTE: pgfplots
    \end{tikzpicture}
    \end{center}
    A particle moving in one dimension under the influence of this
        force has kinetic energy \SI{1.0}{\joule} when it is at position $x_1$.
    Which of the following is a correct statement about the motion of the particle?
    \begin{choices}
        \wrongchoice{It oscillates with maximum position $x_2$ and minimum position $x 0$.}
        \wrongchoice{It moves to the right of $x_3$ and does not return.}
        \wrongchoice{It moves to the left of $x_0$ and does not return.}
        \wrongchoice{It comes to rest at either $x_0$ or $x_2$.}
        \wrongchoice{It cannot reach either $x_0$ or $x_2$.}
    \end{choices}
\end{question}
}

\element{AP}{
\begin{question}{APC-1993-Q16}
    A balloon of mass $M$ is floating motionless in the air. 
    A person of mass less than $M$ is on a rope ladder hanging from the balloon. 
    The person begins to climb the ladder at a uniform speed $v$ relative to the ground. 
    How does the balloon move relative to the ground?
    \begin{choices}
        \wrongchoice{Up with speed $v$}
        \wrongchoice{Up with a speed less than $v$}
        \wrongchoice{Down with speed $v$}
        \wrongchoice{Down with a speed less than $v$}
        \wrongchoice{The balloon does not move.}
    \end{choices}
\end{question}
}

\element{AP}{
\begin{question}{APC-1993-Q17}
    If one knows only the constant resultant force acting on an object
        and the time during which this force acts, one can determine the:
    \begin{choices}
        \wrongchoice{change in momentum of the object}
        \wrongchoice{change in velocity of the object}
        \wrongchoice{change in kinetic energy of the object}
        \wrongchoice{mass of the object}
        \wrongchoice{acceleration of the object}
    \end{choices}
\end{question}
}

\element{AP}{
\begin{question}{APC-1993-Q18}
    When an object is moved from rest at point $A$ to rest at point $B$ 
        in a gravitational field, the net work done by the field depends
        on the mass of the object and
    \begin{choices}
        \wrongchoice{the positions of $A$ and $B$ only}
        \wrongchoice{the path taken between $A$ and $B$ only}
        \wrongchoice{both the positions of $A$ and $B$ and the path taken between them}
        \wrongchoice{the velocity of the object as it moves between $A$ and $B$}
        \wrongchoice{the nature of the external force moving the object from $A$ to $B$}
    \end{choices}
\end{question}
}

\element{AP}{
\begin{question}{APC-1993-Q19}
    An object is shot vertically upward into the air with a positive initial velocity. 
    Which of the following correctly describes the velocity and
        acceleration of the object at its maximum elevation?
    \begin{choices}
        %% NOTE: table
        \wrongchoice{ }
    \end{choices}
\end{question}
}

\element{AP}{
\begin{question}{APC-1993-Q20}
    A turntable that is initially at rest is set in motion
        with a constant angular acceleration $\alpha$.
    What is the angular velocity of the turntable after
        it has made one complete revolution?
    \begin{multicols}{2}
    \begin{choices}
        \wrongchoice{$\sqrt{2\alpha}$}
        \wrongchoice{$\sqrt{2\pi\alpha}$}
        \wrongchoice{$\sqrt{4\pi\alpha}$}
        \wrongchoice{$2\alpha$}
        \wrongchoice{$4\pi\alpha$}
    \end{choices}
    \end{multicols}
\end{question}
}

\element{AP}{
\begin{question}{APC-1993-Q21}
    An object of mass $m$ is moving with speed $v_0$ to the
        right on a horizontal frictionless surface,
        as shown below, when it explodes into two pieces. 
    \begin{center}
    \begin{tikzpicture}
        %% NOTE: 
    \end{tikzpicture}
    \end{center}
    Subsequently, one piece of mass 2/5 $m$ moves with a speed $v_0/2$ to the left.
    The speed of the other piece of the object is:
    \begin{multicols}{2}
    \begin{choices}
        \wrongchoice{$v_0/2$}
        \wrongchoice{$v_0/3$}
        \wrongchoice{$7 v_0/5$}
        \wrongchoice{$3 v_0/2$}
        \wrongchoice{$2 v_0$}
    \end{choices}
    \end{multicols}
\end{question}
}

\element{AP}{
\begin{question}{APC-1993-Q22}
    A newly discovered planet has twice the mass of the Earth,
        but the acceleration due to gravity on the new planet's
        surface is exactly the same as the acceleration due to gravity on the Earth's surface. 
    The radius of the new planet in terms of the radius $R$ of Earth is:
    \begin{multicols}{2}
    \begin{choices}
        \wrongchoice{$\frac{1}{2} R$}
        \wrongchoice{$\frac{\sqrt{2}}{2} R$}
        \wrongchoice{$\sqrt{2} R$}
        \wrongchoice{$2 R$}
        \wrongchoice{$4 R$}
    \end{choices}
    \end{multicols}
\end{question}
}

\newcommand{\APCNineteenNinetyThreeQTwentyThree}{
\begin{tikzpicture}
    %% NOTE: 
\end{tikzpicture}
}

\element{AP}{
\begin{question}{APC-1993-Q23}
    Two identical massless springs are hung from a horizontal support. 
    A block of mass \SI{1.2}{\kilo\gram} is suspended from the pair of springs, as shown below. 
    When the block is in equilibrium, each spring is stretched an additional \SI{0.15}{\meter}.
    \begin{center}
        \APCNineteenNinetyThreeQTwentyThree
    \end{center}
    The force constant of each spring is most nearly:
    \begin{multicols}{2}
    \begin{choices}
        \wrongchoice{\SI{40}{\newton\per\meter}}
        \wrongchoice{\SI{48}{\newton\per\meter}}
        \wrongchoice{\SI{60}{\newton\per\meter}}
        \wrongchoice{\SI{80}{\newton\per\meter}}
        \wrongchoice{\SI{96}{\newton\per\meter}}
    \end{choices}
    \end{multicols}
\end{question}
}

\element{AP}{
\begin{questionmult}{APC-1993-Q24}
    Two identical massless springs are hung from a horizontal support. 
    A block of mass \SI{1.2}{\kilo\gram} is suspended from the pair of springs, as shown below. 
    When the block is in equilibrium, each spring is stretched an additional \SI{0.15}{\meter}.
    \begin{center}
        \APCNineteenNinetyThreeQTwentyThree
    \end{center}
    When the block is set into oscillation with amplitude $A$,
        it passes through its equilibrium point with a speed $v$. 
    In which of the following cases will the block,
        when oscillating with amplitude $A$,
        also have speed $v$ when it passes through its equilibrium point?
    \begin{choices}
        \wrongchoice{The block is hung from only one of the two springs.}
        \wrongchoice{The block is hung from the same two springs,
            but the springs are connected in series rather than in parallel.}
        \wrongchoice{A \SI{0.5}{\kilo\gram} mass is attached to the block.}
    \end{choices}
\end{questionmult}
}

\element{AP}{
\begin{question}{APC-1993-Q25}
    A spring-loaded gun can fire a projectile to a height $h$ if it is fired straight up. 
    If the same gun is pointed at an angle of \ang{45} from the vertical,
        what maximum height can now be reached by the projectile?
    \begin{multicols}{2}
    \begin{choices}
        \wrongchoice{$h/4$}
        \wrongchoice{$\frac{h}{2\sqrt{2}}$}
        \wrongchoice{$h/2$}
        \wrongchoice{$\frac{h}{\sqrt{2}}$}
        \wrongchoice{$h$}
    \end{choices}
    \end{multicols}
\end{question}
}

\element{AP}{
\begin{question}{APC-1993-Q26}
    The rigid body shown in the diagram below consists of a vertical support post and two horizontal crossbars with spheres attached. 
    \begin{center}
    \begin{tikzpicture}
    \end{tikzpicture}
    \end{center}
    The masses of the spheres and the lengths of the crossbars are indicated in the diagram. 
    The body rotates about a vertical axis along the support post with constant angular speed $\omega$.
    If the masses of the support post and the crossbars are negligible,
        what is the ratio of the angular momentum of the two upper spheres
        to that of the two lower spheres?
    \begin{multicols}{2}
    \begin{choices}
        \wrongchoice{$2/1$}
        \wrongchoice{$1/1$}
        \wrongchoice{$1/2$}
        \wrongchoice{$1/4$}
        \wrongchoice{$1/8$}
    \end{choices}
    \end{multicols}
\end{question}
}

\newcommand{\APCNineteenNinetyThreeQTwentySeven}{
\begin{tikzpicture}
\end{tikzpicture}
}

\element{AP}{
\begin{question}{APC-1993-Q27}
    A ball is thrown and follows a parabolic path, as shown below.
    \begin{center}
        \APCNineteenNinetyThreeQTwentySeven
    \end{center}
    Air friction is negligible. 
    Point $Q$ is the highest point on the path.

    \vskip0.5\baselineskip
    Which of the following best indicates the direction of the acceleration,
        if any, of the ball at point $Q$?
    \begin{multicols}{2}
    \begin{choices}
        \wrongchoice{
            \begin{tikzpicture}
            \end{tikzpicture}
        }
    \end{choices}
    \end{multicols}
\end{question}
}

\element{AP}{
\begin{question}{APC-1993-Q28}
    A ball is thrown and follows a parabolic path, as shown below.
    \begin{center}
        \APCNineteenNinetyThreeQTwentySeven
    \end{center}
    Air friction is negligible. 
    Point $Q$ is the highest point on the path.

    \vskip0.5\baselineskip
    Which of the following best indicates the direction
        of the net force on the ball at point $P$?
    \begin{multicols}{2}
    \begin{choices}
        \wrongchoice{
            \begin{tikzpicture}
            \end{tikzpicture}
        }
    \end{choices}
    \end{multicols}
\end{question}
}

\newcommand{\APCNineteenNinetyThreeQTwentyNine}{
\begin{tikzpicture}
\end{tikzpicture}
}

\element{AP}{
\begin{question}{APC-1993-Q29}
    A \SI{5}{\kilo\gram} sphere is connected to a \SI{10}{\kilo\gram}   
        sphere by a rigid rod of negligible mass, as shown below.
    \begin{center}
        \APCNineteenNinetyThreeQTwentyNine
    \end{center}
    Which of the five lettered points represents the center of mass of the sphere-rod combination?
    \begin{multicols}{2}
    \begin{choices}
        \wrongchoice{A}
        \wrongchoice{B}
        \wrongchoice{C}
        \wrongchoice{D}
        \wrongchoice{E}
    \end{choices}
    \end{multicols}
\end{question}
}

\element{AP}{
\begin{question}{APC-1993-Q30}
    A \SI{5}{\kilo\gram} sphere is connected to a \SI{10}{\kilo\gram}   
        sphere by a rigid rod of negligible mass, as shown below.
    \begin{center}
        \APCNineteenNinetyThreeQTwentyNine
    \end{center}
    The sphere-rod combination can be pivoted about an axis that is perpendicular
        to the plane of the page and that passes through one of the five lettered points. 
    Through which point should the axis pass for the moment of inertia of the
        sphere-rod combination about this axis to be greatest?
    \begin{multicols}{2}
    \begin{choices}
        \wrongchoice{A}
        \wrongchoice{B}
        \wrongchoice{C}
        \wrongchoice{D}
        \wrongchoice{E}
    \end{choices}
    \end{multicols}
\end{question}
}

\element{AP}{
\begin{question}{APC-1993-Q31}
    A small mass is released from rest at a very great distance from a larger stationary mass. 
    Which of the following
    graphs best represents the gravitational potential energy $U$
        of the system of the two masses as a function of time $t$?
    \begin{multicols}{2}
    \begin{choices}
        \wrongchoice{
            \begin{tikzpicture}
            \end{tikzpicture}
        }
    \end{choices}
    \end{multicols}
\end{question}
}

\element{AP}{
\begin{question}{APC-1993-Q32}
    A satellite $S$ is in an elliptical orbit around a planet $P$,
        as shown below, with $r_1$ and $r_2$ being its closest and farthest distances,
        respectively, from the center of the planet. 
    \begin{center}
    \begin{tikzpicture}
    \end{tikzpicture}
    \end{center}
    If the satellite has a speed $v_1$ at its closest distance,
        what is its speed at its farthest distance?
    \begin{multicols}{2}
    \begin{choices}
        \wrongchoice{$\dfrac{r_1}{r_2} v_1$}
        \wrongchoice{$\dfrac{r_2}{r_1} v_1$}
        \wrongchoice{$\dfrac{\left(r_1-r_2\right) v_1$}
        \wrongchoice{$\dfrac{r_1 + r_2}{2} v_1$}
        \wrongchoice{$\dfrac{r_2 - r_1}{r_1 + r_2} v_1$}
    \end{choices}
    \end{multicols}
\end{question}
}

\element{AP}{
\begin{question}{APC-1993-Q33}
    A simple pendulum consists of a \SI{1.0}{\kilo\gram}
        brass bob on a string about \SI{1.0}{\meter} long. 
    It has a period of \SI{2.0}{\second}.
    The pendulum would have a period of \SI{1.0}{\second} if the:
    \begin{choices}
        \wrongchoice{string were replaced by one about \SI{0.25}{\meter} long}
        \wrongchoice{string were replaced by one about \SI{2.0}{\meter} long}
        \wrongchoice{bob were replaced by a \SI{0.25}{\kilo\gram} brass sphere}
        \wrongchoice{bob were replaced by a \SI{4.0}{\kilo\gram} brass sphere}
        \wrongchoice{amplitude of the motion were increased}
    \end{choices}
\end{question}
}

\element{AP}{
\begin{question}{APC-1993-Q34}
    A block of mass 5 kilograms lies on an inclined plane, as shown below.
    \begin{center}
    \begin{tikzpicture}
    \end{tikzpicture}
    \end{center}
    The horizontal and vertical supports for the plane have
        lengths of \SI{4}{\meter} and \SI{3}{\meter}, respectively. 
    The coefficient of friction between the plane and the block is \num{0.3}. 
    The magnitude of the force $F$ necessary to pull the block up the plane
        with constant speed is most nearly
    \begin{multicols}{2}
    \begin{choices}
        \wrongchoice{\SI{30}{\newton}}
        \wrongchoice{\SI{42}{\newton}}
        \wrongchoice{\SI{49}{\newton}}
        \wrongchoice{\SI{50}{\newton}}
        \wrongchoice{\SI{58}{\newton}}
    \end{choices}
    \end{multicols}
\end{question}
}

\element{AP}{
\begin{question}{APC-1993-Q35}
    A rod of negligible mass is pivoted at a point that is off-center,  
        so that length $l_1$ is different from length $l_2$. 
    The figures below show two cases in which masses are suspended
        from the ends of the rod. In each case the unknown mass $m$
        is balanced by a known mass, $M_1$ or $M_2$, so that the rod remains horizontal. 
    \begin{center}
    \begin{tikzpicture}
    \end{tikzpicture}
    \end{center}
    What is the value of m in terms of the known masses?
    \begin{multicols}{2}
    \begin{choices}
        \wrongchoice{$M_1 + M_2$}
        \wrongchoice{$\frac{1}{2} \left( M_1 + M_2 \right)$}
        \wrongchoice{$M_1 M_2$}
        \wrongchoice{$\frac{1}{2} M_1 M_2$}
        \wrongchoice{$\sqrt{ M_1 M_2 }$}
    \end{choices}
    \end{multicols}
\end{question}
}


\endinput

