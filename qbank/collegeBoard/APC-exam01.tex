
%% Princeton Review: AP Physics B and C
%%----------------------------------------


%% AP Physics C: Mechanics
%%----------------------------------------
\element{AP}{
\begin{question}{APC-exam01-Q01}
    A rock is dropped off a cliff and falls the first half
        of the distance to the ground in $t_1$ seconds.
    If it falls the second half of the distance in $t_2$ seconds,
        what is the value of $t_2/t_1$?
    (Ignore air resistance)
    \begin{multicols}{2}
    \begin{choices}
        \wrongchoice{$\dfrac{1}{2\sqrt{2}}$}
        \wrongchoice{$\dfrac{1}{\sqrt{2}}$}
        \wrongchoice{$\dfrac{1}{2}$}
        \wrongchoice{$1-\dfrac{1}{\sqrt{2}}$}
        \wrongchoice{$\sqrt{2}-1$}
    \end{choices}
    \end{multicols}
\end{question}
}

\element{AP}{
\begin{question}{APC-exam01-Q02}
    A box of mass $m$ slides on a horizontal surface with initial speed $v_0$.
    It feels no forces other than gravity and the force from the surface.
    If the coefficient of kinetic friction between the box and the surface is $\mu$,
        how far does the box slide before coming to rest?
    \begin{multicols}{2}
    \begin{choices}
        \wrongchoice{$\dfrac{v_0^2}{2\mu g}$}
        \wrongchoice{$\dfrac{v_0^2}{\mu g}$}
        \wrongchoice{$\dfrac{2 v_0^2}{\mu g}$}
        \wrongchoice{$\dfrac{mv_0^2}{\mu g}$}
        \wrongchoice{$\dfrac{2mv_0^2}{\mu g}$}
    \end{choices}
    \end{multicols}
\end{question}
}

\element{AP}{
\begin{question}{APC-exam01-Q03}
    An object initially at rest experiences a time-varying acceleration
        given by $a=\SI{2}{\meter\per\second\squared}\,t$ for $t\geq 0$.
    How far does the object travel in the first \SI{3}{\second}?
    \begin{multicols}{2}
    \begin{choices}
        \wrongchoice{\SI{9}{\meter}}
        \wrongchoice{\SI{12}{\meter}}
        \wrongchoice{\SI{18}{\meter}}
        \wrongchoice{\SI{24}{\meter}}
        \wrongchoice{\SI{27}{\meter}}
    \end{choices}
    \end{multicols}
\end{question}
}

\element{AP}{
\begin{questionmult}{APC-exam01-Q04}
    Which of the following conditions will ensure
        that angular momentum is conserved?
    \begin{choices}
        \wrongchoice{Conservation of linear momentum}
        \wrongchoice{Zero net external force}
      \correctchoice{Zero net external torque}
    \end{choices}
\end{questionmult}
}

\element{AP}{
\begin{question}{APC-exam01-Q05}
    In the figure shown, a tension force $\mathbf{F}_1$ causes
        a particle of mass $m$ to move with constant angular speed 
        $\omega$ in a circular path (perpendicular to the page) of radius $R$.
    Which of the following expressions gives the magnitude of $\mathbf{F}_1$?
    \begin{multicols}{2}
    \begin{choices}
        \wrongchoice{$m\omega^2 R$}
        \wrongchoice{$m\sqrt{\omega^2 R^2 - g^2}$}
        \wrongchoice{$m\sqrt{\omega^2 R^2 + g^2}$}
        \wrongchoice{$m\left(\omega^2 R - g\right)$}
        \wrongchoice{$m\left(\omega^2 R + g\right)$}
    \end{choices}
    \end{multicols}
\end{question}
}

\element{AP}{
\begin{question}{APC-exam01-Q06}
    An object (mass = $m$) above the surface of the Moon
        (mass = $M$) is dropped from an altitude $h$
        equal to the Moon's radius ($R$).
    With what speed will the object strike the lunar surface?
    \begin{multicols}{2}
    \begin{choices}
        \wrongchoice{$\sqrt{\dfrac{GM}{R}$}
        \wrongchoice{$\sqrt{\dfrac{GM}{2R}$}
        \wrongchoice{$\sqrt{\dfrac{2GM}{R}$}
        \wrongchoice{$\sqrt{\dfrac{2GMm}{R}$}
        \wrongchoice{$\sqrt{\dfrac{GMm}{2R}$}
    \end{choices}
    \end{multicols}
\end{question}
}

\element{AP}{
\begin{question}{APC-exam01-Q07}
    The figure shows a linear spring anchored to the ceiling.
    If the mass of the block hanging from its lower end is doubled,
        what effect will this change have on the block's equilibrium 
        position and oscillation period?
    \begin{multicols}{2}
    \begin{choices}
        \wrongchoice{Equilibrium position: Lowered by a factor of $\sqrt{2}$.
                     Oscillation period: Decreased by a factor of $\sqrt{2}$}
        \wrongchoice{Equilibrium position: Lowered by a factor of $\sqrt{2}$.
                     Oscillation period: Increased by a factor of $\sqrt{2}$}
        \wrongchoice{Equilibrium position: Lowered by a factor of $\sqrt{2}$.
                     Oscillation period: Increased by a factor of $2$}
        \wrongchoice{Equilibrium position: Lowered by a factor of $2$.
                     Oscillation period: Decreased by a factor of $\sqrt{2}$}
      \correctchoice{Equilibrium position: Lowered by a factor of $2$.
                     Oscillation period: Increased by a factor of $\sqrt{2}$}
    \end{choices}
    \end{multicols}
\end{question}
}

\element{AP}{
\begin{question}{APC-exam01-Q08}
    A uniform cylinder of mass $m$ and radius $r$ unrolls without slipping from two strings tied to a vertical support.
    \begin{center}
    \begin{tikzpicture}
    \end{tikzpicture}
    \end{center}
    If the rotational inertia of the cylinder is $\frac{1}{2}mr^2$, find the acceleration of its center of mass.
    \begin{multicols}{2}
    \begin{choices}
        \wrongchoice{$\frac{1}{4}g$}
        \wrongchoice{$\frac{1}{2}g$}
        \wrongchoice{$\frac{1}{3}g$}
        \wrongchoice{$\frac{2}{3}g$}
        \wrongchoice{$\frac{3}{4}g$}
    \end{choices}
    \end{multicols}
\end{question}
}

\element{AP}{
\begin{question}{APC-exam01-Q09}
    A uniform cylinder, initially at rest on a frictionless, horizontal surface,
        is pulled by a constant force $\mathbf{F}$ from time $t=0$ to time $t=T$.
    From time $t=T$ on, this force is removed.
    Which of the following graphs best illustrates the speed, $v$,
        of the cylinder's center of mass from $t=0$ to $t=2T$?
    \begin{multicols}{2}
    \begin{choices}
        \wrongchoice{
            \begin{tikzpicture}
            \end{tikzpicture}
        }
    \end{choices}
    \end{multicols}
\end{question}
}

\element{AP}{
\begin{question}{APC-exam01-Q10}
    An engine provides \SI{10}{\kilo\watt} of power to lift a heavy load
        at constant velocity a distance of \SI{20}{\meter} in \SI{5}{\second}.
    What is the mass of the object being lifted?
    \begin{multicols}{2}
    \begin{choices}
        \wrongchoice{\SI{100}{\kilo\gram}}
        \wrongchoice{\SI{150}{\kilo\gram}}
        \wrongchoice{\SI{200}{\kilo\gram}}
        \wrongchoice{\SI{250}{\kilo\gram}}
        \wrongchoice{\SI{500}{\kilo\gram}}
    \end{choices}
    \end{multicols}
\end{question}
}

\element{AP}{
\begin{question}{APC-exam01-Q11}
    A satellite is in circular orbit around the earth.
    If the work required to lift the satellite to its orbit height
        is equal to the satellite's kinetic energy while in this orbit,
        how high above the surface of the earth (radius = $R$) is the satellite?
    \begin{multicols}{2}
    \begin{choices}
        \wrongchoice{$\frac{1}{2}R$}
        \wrongchoice{$\frac{2}{3}R$}
        \wrongchoice{$R$}
        \wrongchoice{$\frac{3}{2}R$}
        \wrongchoice{$2 R$}
    \end{choices}
    \end{multicols}
\end{question}
}

\element{AP}{
\begin{question}{APC-exam01-Q12}
    The figure below shows a uniform bar of mass $M$ resting on two supports.
    A block of mass $\frac{1}{2}M$ is placed on the bar twice as far from Support 2 as from Support 1.
    If $\mathbf{F}_1$ and $\mathbf{F}_2$ denot the downward forces on Support 1 and Support 2, respectively, what is the value of $F_2/F_1$?
    \begin{multicols}{2}
    \begin{choices}
        \wrongchoice{$\sfrac{1}{2}$}
        \wrongchoice{$\sfrac{2}{3}$}
        \wrongchoice{$\sfrac{3}{4}$}
        \wrongchoice{$\sfrac{4}{5}$}
        \wrongchoice{$\sfrac{5}{6}$}
    \end{choices}
    \end{multicols}
\end{question}
}

\element{AP}{
\begin{question}{APC-exam01-Q13}
    A rubber ball (mass = \SI{0.08}{\kilo\gram}) is dropped from a height of \SI{2}{\meter},
        and after bouncing off the floow,
        rises almost to its original height.
    If the impact time with the floor is measured to be \SI{0.04}{\second},
        what average force did the floor exert on the ball?
    \begin{multicols}{2}
    \begin{choices}
        \wrongchoice{\SI{0.16}{\newton}}
        \wrongchoice{\SI{16}{\newton}}
        \wrongchoice{\SI{25}{\newton}}
        \wrongchoice{\SI{36}{\newton}}
        \wrongchoice{\SI{64}{\newton}}
    \end{choices}
    \end{multicols}
\end{question}
}

\element{AP}{
\begin{question}{APC-exam01-Q14}
    A disk of radius \SI{0.1}{\meter} initially at rest undergoes an angular acceleration of \SI{2.0}{\radian\per\second\squared}.
    If the disk only rotates,
        find the total distance travelled by a point on the rim of the disk in \SI{4.0}{\second}.
    \begin{multicols}{2}
    \begin{choices}
        \wrongchoice{\SI{0.4}{\meter}}
        \wrongchoice{\SI{0.8}{\meter}}
        \wrongchoice{\SI{1.2}{\meter}}
        \wrongchoice{\SI{1.6}{\meter}}
        \wrongchoice{\SI{2.0}{\meter}}
    \end{choices}
    \end{multicols}
\end{question}
}

\element{AP}{
\begin{question}{APC-exam01-Q15}
    In the figure below, a small ball slides down a frictionless quarter-circular slide of radius $R$.
    \begin{center}
    \begin{tikzpicture}
    \end{tikzpicture}
    \end{center}
    If the ball starts from rest at a height equal to $2R$ above a horizontal surface,
        finds its equilibrium distplacement, $x$,
        at the moment is strikes the surface.
    \begin{multicols}{2}
    \begin{choices}
        \wrongchoice{\SI{0.4}{\meter}}
        \wrongchoice{\SI{0.8}{\meter}}
        \wrongchoice{\SI{1.2}{\meter}}
        \wrongchoice{\SI{1.6}{\meter}}
        \wrongchoice{\SI{2.0}{\meter}}
    \end{choices}
    \end{multicols}
\end{question}
}

\element{AP}{
\begin{question}{APC-exam01-Q16}
    The figure below shows a particle executing, uniform circular motino in a circle of radius $R$.
    \begin{center}
    \begin{tikzpicture}
    \end{tikzpicture}
    \end{center}
    Light sources (not shown) cause shadows of the particle to be projected onto two mutally perpendicular screens.
    The positive directions for $x$ and $y$ along the screens are denoted by the arrows.
    When the shadow on Screen 1 is at position $x=-0.5\,R$ and moving in the $+x$ direction,
        what is true about the position and velocity of the shadow on Screen 2 at that same instant?
    \begin{choices}
        \wrongchoice{$y=-0.866\,R$; velocity in $-y$ direction}
        \wrongchoice{$y=-0.866\,R$; velocity in $+y$ direction}
        \wrongchoice{$y=-0.5\,R$; velocity in $-y$ direction}
        \wrongchoice{$y=+0.866\,R$; velocity in $-y$ direction}
        \wrongchoice{$y=+0.866\,R$; velocity in $+y$ direction}
    \end{choices}
\end{question}
}

\element{AP}{
\begin{question}{APC-exam01-Q17}
    The figure shows a view from above of two objects attached to the end of rigid massless rod at rest on a fricitonless table.
    \begin{center}
    \begin{tikzpicture}
    \end{tikzpicture}
    \end{center}
    When a force $\mathbf{F}$ is applied as shown,
        the resulting rotational acceleration of the road about its center is $kF/(mL)$.
    What is $k$?
    \begin{multicols}{2}
    \begin{choices}
        \wrongchoice{$\dfrac{3}{8}$}
        \wrongchoice{$\dfrac{1}{2}$}
        \wrongchoice{$\dfrac{5}{8}$}
        \wrongchoice{$\dfrac{3}{4}$}
        \wrongchoice{$\dfrac{5}{6}$}
    \end{choices}
    \end{multicols}
\end{question}
}

\element{AP}{
\begin{questionmult}{APC-exam01-Q18}
    A lightweight toy car crashes head-on into a heavier toy truck.
    Which of the following statements is true as a result of the collision?
    \begin{choices}
        \wrongchoice{The car will experience a greater impulse than the truck?}
        \wrongchoice{The car will experience a greater change in momentum than the truck.}
      \correctchoice{The magnitude of the acceleration experienced by the car will be greater than that experienced by the truck.}
    \end{choices}
\end{questionmullt}
}

\element{AP}{
\begin{question}{APC-exam01-Q19}
    A homogenous bar is lying on a flat table.
    Besides the gravitional and normal forces (which cancel),
        the bar is acted upono by exactly two other external forces,
        $\mathbf{F}_1$ and $\mathbf{F}_2$,
        which are parallel to the surface of the table.
    If the net force on the rod is zero,
        which one of the following is also ttru?
    \begin{choices}
        \wrongchoice{The net torque on the bar must also be zero.}
        \wrongchoice{The bar cannot accelerate translationally or rotationally.}
        \wrongchoice{The bar can accelerate translationally if $\mathbf{F}_1$ and $\mathbf{F}_2$ are not applied at the same point.}
        \wrongchoice{The net torque will be zero if $\mathbf{F}_1$ and $\mathbf{F}_2$ are applied at the same point.}
        \wrongchoice{None of the above.}
    \end{choices}
\end{question}
}

\element{AP}{
\begin{question}{APC-exam01-Q20}
    An astronaut lands on a planet whose mass and radius are each twice that of Earth.
    If the astronaut weighs \SI{800}{\newton} on Earth,
        how much will he weigh on this planet?
    \begin{multicols}{2}
    \begin{choices}
        \wrongchoice{\SI{200}{\newton}}
        \wrongchoice{\SI{400}{\newton}}
        \wrongchoice{\SI{800}{\newton}}
        \wrongchoice{\SI{1600}{\newton}}
        \wrongchoice{\SI{3200}{\newton}}
    \end{choices}
    \end{multicols}
\end{question}
}

\element{AP}{
\begin{question}{APC-exam01-Q21}
    A particle of mass $m=\SI{1.0}{\kilo\gram}$ is acted upon by a variable force,
        $F(x)$, whose strength is given by the graph given below.
    \begin{center}
    \begin{tikzpicture}
    \end{tikzpicture}
    \end{center}
    If the particle's speed was zero at $x=0$,
        what is its speed at $x=\SI{4}{\meter}$?
    \begin{multicols}{2}
    \begin{choices}
        \wrongchoice{\SI{5.0}{\meter\per\second}}
        \wrongchoice{\SI{8.7}{\meter\per\second}}
        \wrongchoice{\SI{10}{\meter\per\second}}
        \wrongchoice{\SI{14}{\meter\per\second}}
        \wrongchoice{\SI{20}{\meter\per\second}}
    \end{choices}
    \end{multicols}
\end{question}
}

\element{AP}{
\begin{question}{APC-exam01-Q22}
    The radius of a collapsing spinning star (assumed to be a uniform sphere)
        decreases to $\sfrac{1}{16}$ its initial value.
    What is the ratio of the final rotatinoal kinetic energy to
        the intial rotational kinetic energy?
    \begin{multicols}{2}
    \begin{choices}
        \wrongchoice{$4$}
        \wrongchoice{$16$}
        \wrongchoice{$16^2$}
        \wrongchoice{$16^3$}
        \wrongchoice{$16^4$}
    \end{choices}
    \end{multicols}
\end{question}
}

\element{AP}{
\begin{question}{APC-exam01-Q23}
    An object is projected with an initial velocity of magnitude $v_0=\SI{40}{\meter\per\second}$ toward a vertical wall as shown in the figure below.
    \begin{center}
    \begin{tikzpicture}
    \end{tikzpicture}
    \end{center}
    The object will hit the wall closest to which point?
    \begin{multicols}{2}
    \begin{choices}
        \wrongchoice{$A$}
        \wrongchoice{$B$}
        \wrongchoice{$C$}
        \wrongchoice{$D$}
        \wrongchoice{$E$}
    \end{choices}
    \end{multicols}
\end{question}
}

\element{AP}{
\begin{question}{APC-exam01-Q24}
    If $L$, $M$, $T$ denot the dimensions of length, mass and time,
        respectively, what are the dimensions of impulse?
    \begin{multicols}{2}
    \begin{choices}
        \wrongchoice{$\dfrac{LM}{T^3}$}
        \wrongchoice{$\dfrac{LM}{T^2}$}
        \wrongchoice{$\dfrac{LM}{T}$}
        \wrongchoice{$\dfrac{L^2M}{T^2}$}
        \wrongchoice{$\dfrac{M^2L}{T}$}
    \end{choices}
    \end{multicols}
\end{question}
}

\element{AP}{
\begin{question}{APC-exam01-Q25}
    The figure shown is a view from above of two clay balls moving twoard each other on a frictionless surface.
    \begin{center}
    \begin{tikzpicture}
    \end{tikzpicture}
    \end{center}
    They collide perfectly inelastically at the indicated point and ar observed to then move in the direction indicated by the post-collision velocity vector, $v^{\prime}$.
    If $m_1 = 2 m_2$, what is $v_2$?
    \begin{multicols}{2}
    \begin{choices}
        \wrongchoice{$v_1 \dfrac{\sin\ang{45}}{2\sin\ang{60}$}
        \wrongchoice{$v_1 \dfrac{\cos\ang{45}}{2\cos\ang{60}$}
        \wrongchoice{$v_1 \dfrac{2\cos\ang{45}}{\cos\ang{60}$}
        \wrongchoice{$v_1 \dfrac{2\sin\ang{45}}{\sin\ang{60}$}
        \wrongchoice{$v_1 \dfrac{\sin\ang{45}}{\sin\ang{60}$}
    \end{choices}
    \end{multicols}
\end{question}
}

\element{AP}{
\begin{question}{APC-exam01-Q26}
    In the figure below,
        the coefficient of static friction between two blocks is \num{0.80}.
    \begin{center}
    \begin{tikzpicture}
    \end{tikzpicture}
    \end{center}
    If the blocks oscillate with a frequency of \SI{2.0}{\hertz},
        what is the maximum amplitude of the oscillations if the small block is not to sip on the large block?
    They collide perfectly inelastically at the indicated point and ar observed to then move in the direction indicated by the post-collision velocity vector, $v^{\prime}$.
    \begin{multicols}{2}
    \begin{choices}
        \wrongchoice{\SI{3.1}{\centi\meter}}
        \wrongchoice{\SI{5.0}{\centi\meter}}
        \wrongchoice{\SI{6.2}{\centi\meter}}
        \wrongchoice{\SI{7.5}{\centi\meter}}
        \wrongchoice{\SI{9.4}{\centi\meter}}
    \end{choices}
    \end{multicols}
\end{question}
}

\element{AP}{
\begin{question}{APC-exam01-Q27}
    When two objects collide,
        the ratio of the relative speed after the collision to the relative speed before the collision is called the \emph{coefficient of restitution, e}.
    If a ball is dropped from height $H_1$,
        onto a stationary floor,
        and the ball rebounds to height $H_2$,
        what is the coefficient of resistutino of the collision?
    \begin{multicols}{2}
    \begin{choices}
        \wrongchoice{$\dfrac{H_2}{H_1}$}
        \wrongchoice{$\dfrac{H_2}{H_1}$}
        \wrongchoice{$\sqrt{\dfrac{H_1}{H_2}}$}
        \wrongchoice{$\sqrt{\dfrac{H_2}{H_1}}$}
        \wrongchoice{$\left(\dfrac{H_1}{H_2}\right)^2$}
    \end{choices}
    \end{multicols}
\end{question}
}

\element{AP}{
\begin{question}{APC-exam01-Q28}
    The figure below shows a square metal plate of side length \SI{40}{\centi\meter} and uniform density,
        lying flat on a table.
    A force $\mathbf{F}$ of magnitude \SI{10}{\newton} is applied at one of the corners,
        as shown.
    \begin{center}
    \begin{tikzpicture}
    \end{tikzpicture}
    \end{center}
    Determine the torque produced by $\mathbf{F}$ relative to the center of rotation.
    \begin{multicols}{2}
    \begin{choices}
        \wrongchoice{$\dfrac{H_2}{H_1}$}
        \wrongchoice{$\dfrac{H_2}{H_1}$}
        \wrongchoice{$\sqrt{\dfrac{H_1}{H_2}}$}
        \wrongchoice{$\sqrt{\dfrac{H_2}{H_1}}$}
        \wrongchoice{$\left(\dfrac{H_1}{H_2}\right)^2$}
    \end{choices}
    \end{multicols}
\end{question}
}

\element{AP}{
\begin{question}{APC-exam01-Q29}
    A small block of mass $m=\SI{2.0}{\kilo\gram}$ is pushed from initial point
        $(x_i,z_i)=(\SI{0}{\meter},\SI{0}{\meter})$ upward to the final point
        $(x_f,z_f)=(\SI{3}{\meter},\SI{3}{\meter})$ along the path indicated.
    Path 1 is a portion of the parabola $z=x^2$,
        and Path 2 is a quarter circle whose equation if $(x-2)^2 + (z-2)^2=2$.
    \begin{center}
    \begin{tikzpicture}
        %% NOTE:
    \end{tikzpicture}
    \end{center}
    How much work is done by gravity during this displacement?
    \begin{multicols}{2}
    \begin{choices}
        \wrongchoice{\SI{60}{\joule}}
        \wrongchoice{\SI{80}{\joule}}
        \wrongchoice{\SI{90}{\joule}}
        \wrongchoice{\SI{100}{\joule}}
        \wrongchoice{\SI{120}{\joule}}
    \end{choices}
    \end{multicols}
\end{question}
}

\element{AP}{
\begin{question}{APC-exam01-Q30}
    In the figure shown, the block (mass=$m$) is at rest at $x=A$.
    \begin{center}
    \begin{tikzpicture}
        %% NOTE:
    \end{tikzpicture}
    \end{center}
    As it moves back toward the wall due to the force exerted by the stretched spring,
        it is also acted upon by a frictional force whose strength is given by the expressoin $bx$,
        where $b$ is a positive constant.
    What is the block's speed when it first passes through the equilibrium position $(x=0)$?
    \begin{multicols}{2}
    \begin{choices}
        \wrongchoice{$A\sqrt{\dfrac{k+b}{m}}$}
        \wrongchoice{$A\sqrt{\dfrac{k-b}{m}}$}
        \wrongchoice{$A\sqrt{\dfrac{\frac{1}{2}k+b}{m}}$}
        \wrongchoice{$A\sqrt{\dfrac{\frac{1}{2}k-b}{m}}$}
        \wrongchoice{$A\sqrt{\dfrac{\frac{1}{2}(k-b)}{m}}$}
    \end{choices}
    \end{multicols}
\end{question}
}

\element{AP}{
\begin{question}{APC-exam01-Q31}
    The rod shown below can pivot abou the point $x=0$ and rotates in a plane perpendicular to the page.
    \begin{center}
    \begin{tikzpicture}
        %% NOTE:
    \end{tikzpicture}
    \end{center}
    Its linear density, $\lambda$, increases with $x$ such that $\lambda(x)=kx$,
        where $k$ is a positive constant.
    Determine the rod's momentum of inertia in terms of its length, $L$,
        and its total mass, $M$.
    \begin{multicols}{2}
    \begin{choices}
        \wrongchoice{$\frac{1}{6} ML^2$}
        \wrongchoice{$\frac{1}{4} ML^2$}
        \wrongchoice{$\frac{1}{3} ML^2$}
        \wrongchoice{$\frac{1}{2} ML^2$}
        \wrongchoice{$2 ML^2$}
    \end{choices}
    \end{multicols}
\end{question}
}

\element{AP}{
\begin{question}{APC-exam01-Q32}
    A particle is subjected to a conservative force whose potential energy function is
    \begin{equation}
        U(x) = (x-2)^2 = 12 x
    \end{equation}
    where $U$ is given in joules when $x$ is measured in meters.
    Which of the following represents a position of stable equilibrium?
    \begin{multicols}{2}
    \begin{choices}
        \wrongchoice{$x=\SI{-4}{\meter}$}
        \wrongchoice{$x=\SI{-2}{\meter}$}
        \wrongchoice{$x=\SI{0}{\meter}$}
        \wrongchoice{$x=\SI{2}{\meter}$}
        \wrongchoice{$x=\SI{4}{\meter}$}
    \end{choices}
    \end{multicols}
\end{question}
}

\element{AP}{
\begin{question}{APC-exam01-Q33}
    At what angle to the horizontal should an ideal projectile
        be launched so that its horizontal displacement (the range)
        is equal to its maximum vertical displacement?
    \begin{multicols}{2}
    \begin{choices}
        \wrongchoice{$\sin^{-1}\left(\dfrac{1}{g}\right)$}
        \wrongchoice{$\cos^{-1}\left(\dfrac{1}{g}\right)$}
        \wrongchoice{\ang{45}}
        \wrongchoice{$\tan^{-1}\left(\num{2}\right)$}
        \wrongchoice{$\tan^{-1}\left(\num{4}\right)$}
    \end{choices}
    \end{multicols}
\end{question}
}

\element{AP}{
\begin{question}{APC-exam01-Q34}
    A projectile's kinetic energy is changing at a rate of \SI{-6.0}{\joule\per\second}
        when its speed is \SI{3.0}{\meter\per\second}.
    What is the magnitude of the force on the particle at this moment?
    \begin{multicols}{2}
    \begin{choices}
        \wrongchoice{\SI{0.5}{\newton}}
        \wrongchoice{\SI{2.0}{\newton}}
        \wrongchoice{\SI{4.5}{\newton}}
        \wrongchoice{\SI{9.0}{\newton}}
        \wrongchoice{\SI{18}{\newton}}
    \end{choices}
    \end{multicols}
\end{question}
}

\element{AP}{
\begin{question}{APC-exam01-Q35}
    An object of mass \SI{2}{\kilo\gram} is acted upon by three external forces,
        each of magnitude \SI{4}{\newton}.
    Which of the following could \emph{not} be the resulting acceleration of the object?
    \begin{multicols}{2}
    \begin{choices}
        \wrongchoice{\SI{0}{\meter\per\second\squared}}
        \wrongchoice{\SI{2}{\meter\per\second\squared}}
        \wrongchoice{\SI{4}{\meter\per\second\squared}}
        \wrongchoice{\SI{6}{\meter\per\second\squared}}
        \wrongchoice{\SI{8}{\meter\per\second\squared}}
    \end{choices}
    \end{multicols}
\end{question}
}


%% AP Physics C: Electricity and Magnetism
%%--------------------------------------------------
\element{AP}{
\begin{question}{APC-exam01-Q36}
    A nonconducting sphere is given a nonzero net electric charge,  
        $+Q$, and then brought close to a neutral conducting sphere of the same radius.
    Which of the following will be true?
    \begin{multicols}{2}
    \begin{choices}
        \wrongchoice{An electric field will be inducted within the conducting sphere.}
        \wrongchoice{The conducting sphere will develop a net electric charge of $-Q$.}
        \wrongchoice{The spheres will experience an electrostatic attraction.}
        \wrongchoice{The spheres will experience an electrostatic repulsion.}
        \wrongchoice{The spheres will experience no electrostatic interaction.}
    \end{choices}
    \end{multicols}
\end{question}
}


\endinput

