

%% this section contains XX problems
%%----------------------------------------


%% AP Physics B 1998
%%------------------------------
\element{AP}{
\begin{question}{APB-1998-Q01}
    A solid metal ball and a hollow plastic ball of the same
        external radius are released from rest in a large vacuum chamber.
    When each has fallen \SI{1}{\meter}, they both have the same:
    \begin{multicols}{2}
    \begin{choices}
        \wrongchoice{inertia}
        \wrongchoice{speed}
        \wrongchoice{momentum}
        \wrongchoice{kinetic energy}
        \wrongchoice{change in potential energy}
    \end{choices}
    \end{multicols}
\end{question}
}

\element{AP}{
\begin{question}{APB-1998-Q02}
    A student weighing \SI{700}{\newton} climbs at constant speed
        to the top of an \SI{8}{\meter} vertical rope in \SI{10}{\second}.
    The average power expended by the student to overcome gravity is most nearly:
    \begin{multicols}{2}
    \begin{choices}
        \wrongchoice{\SI{1.1}{\watt}}
        \wrongchoice{\SI{87.5}{\watt}}
        \wrongchoice{\SI{560}{\watt}}
        \wrongchoice{\SI{875}{\watt}}
        \wrongchoice{\SI{5 600}{\watt}}
    \end{choices}
    \end{multicols}
\end{question}
}

\element{AP}{
\begin{question}{APB-1998-Q03}
    A railroad car of mass $m$ is moving at speed $v$ when it
        collides with a second railroad car of mass $M$ which is at rest.
    The two cars lock together instantaneously and move along the track.
    What is the speed of the cars immediately after the collision?
    \begin{multicols}{2}
    \begin{choices}
        \wrongchoice{$\dfrac{v}{2}$}
        \wrongchoice{$\dfrac{mv}{M}$}
        \wrongchoice{$\dfrac{Mv}{m}$}
        \wrongchoice{$\dfrac{\left(m+M\right)}{m}$}
        \wrongchoice{$\dfrac{mv}{m+M}$}
    \end{choices}
    \end{multicols}
\end{question}
}

\element{AP}{
\begin{question}{APB-1998-Q04}
    An open cart on a level surface is rolling without frictional loss through a vertical downpour of rain, as shown below.
    \begin{center}
    \begin{tikzpicture}
    \end{tikzpicture}
    \end{center}
    As the cart rolls, an appreciable amount of rainwater accumlates in the cart.
    The speed of the car will:
    \begin{choices}
        \wrongchoice{increase because of conservation of momentum}
        \wrongchoice{increase because of conservation of mechanical energy}
      \correctchoice{decrease because of conservation of momentum}
        \wrongchoice{decrease because of conservation of mechanical energy}
        \wrongchoice{remain the same because the raindrops are falling perpendicular to the direction of the cart's motion.}
    \end{choices}
\end{question}
}

\element{AP}{
\begin{questionmult}{APB-1998-Q05}
    Units of power include which of the following?
    \begin{choices}
      \correctchoice{Watt (\si{\watt})}
      \correctchoice{Joule per second (\si{\joule\per\second})}
        \wrongchoice{kilowatt-hour (\si{\kilo\watt\hour})}
    \end{choices}
\end{questionmult}
}

\element{AP}{
\begin{question}{APB-1998-Q06}
    A \SI{2}{\kilo\gram} object moves in a circle of radius
        \SI{4}{\meter} at a constant speed of \SI{3}{\meter\per\second}.
    A net force of \SI{4.5}{\newton} acts on the object.
    What is the angular momentum of the object with
        respect to an axis perpendicular to the circle and through its center.
    \begin{multicols}{2}
    \begin{choices}
        \wrongchoice{\SI{9}{\newton\meter\per\kilo\gram}}
        \wrongchoice{\SI{12}{\meter\squared\per\second}}
        \wrongchoice{\SI{13.5}{\kilo\gram\meter\squared\per\second\squared}}
        \wrongchoice{\SI{18}{\newton\meter\per\kilo\gram}}
        \wrongchoice{\SI{24}{\kilo\gram\meter\squared\per\second}}
    \end{choices}
    \end{multicols}
\end{question}
}

\element{AP}{
\begin{questionmult}{APB-1998-Q07}
    Three forces act on an object.
    If the object is in translational equilibrium,
        which of the following must be true?
    \begin{choices}
      \correctchoice{The vector sum of the three forces must be equal.}
        \wrongchoice{The magnitude of the three forces must be equal.}
        \wrongchoice{All three forces must be parallel.}
    \end{choices}
\end{questionmult}
}

\element{AP}{
\begin{question}{APB-1998-Q08}
    The graph below represents the potential energy $U$
        as a function of displacement $x$
        for an object on the end of a spring oscillating
        in simple harmonic motion with amplitude $x_0$.
    \begin{center}
    \begin{tikzpicture}
    \end{tikzpicture}
    \end{center}
    Which of the following graphs represents the kinetic energy $K$
        of the object as a function of displacement $x$?
    \begin{multicols}{2}
    \begin{choices}
        \wrongchoice{
            \begin{tikzpicture}
            \end{tikzpicture}
        }
    \end{choices}
    \end{multicols}
\end{question}
}

\element{AP}{
\begin{question}{APB-1998-Q09}
    A child pushes horizontally on a box of mass $m$ which
        moves with constant speed $v$ across a horizontal floor.
    The coefficient of friction between the box and the floor is $\mu$.
    At what rate does the child do work on the box?
    \begin{multicols}{2}
    \begin{choices}
        \wrongchoice{$\mu m g v$}
        \wrongchoice{$m g v$}
        \wrongchoice{$v/\mu m g$}
        \wrongchoice{$\mu m g/v$}
        \wrongchoice{$\mu m v^2$}
    \end{choices}
    \end{multicols}
\end{question}
}

\element{AP}{
\begin{question}{APB-1998-Q10}
    Quantum transitions that result in the characteristic
        sharp lines of the X-ray spectrum always involve:
    \begin{choices}
        \wrongchoice{the inner electron shells}
        \wrongchoice{electron energy levels that have the same principal quantum number}
        \wrongchoice{emission of beta particles from the nucleus}
        \wrongchoice{neutrons within the nucleus}
        \wrongchoice{protons within the nucleus}
    \end{choices}
\end{question}
}

\element{AP}{
\begin{questionmult}{APB-1998-Q11}
    Which of the following experiments provided evidence
        that electrons exhibit wave properties?
    \begin{choices}
        \wrongchoice{Millikan oil-drop experiment}
      \correctchoice{Davisson-Germer electron-diffraction experiment}
        \wrongchoice{J. J. Thomson's measurement of the charge-to-mass ratio of electrons}
    \end{choices}
\end{questionmult}
}

\element{AP}{
\begin{questionmult}{APB-1998-Q12}
    Quantities that are conserved in all nuclear reactions include which of the following?
    \begin{choices}
      \correctchoice{Electric charge}
        \wrongchoice{Number of nuclei}
        \wrongchoice{Number of protons}
    \end{choices}
\end{questionmult}
}

\element{AP}{
\begin{question}{APB-1998-Q13}
    Which of the following is true about the net force on an
        uncharged conducting sphere in a uniform electric field?
    \begin{choices}
        \wrongchoice{It is zero}
        \wrongchoice{It is in the direction of the field.}
        \wrongchoice{It is in the direction opposite the field.}
        \wrongchoice{It produces a torque on the sphere about the direction of the field.}
        \wrongchoice{It causes the sphere to oscillate about an equilibrium position.}
    \end{choices}
\end{question}
}

\element{AP}{
\begin{question}{APB-1998-Q14}
    Two parallel conducting
    \begin{choices}
        \wrongchoice{}
    \end{choices}
\end{question}
}



\endinput


