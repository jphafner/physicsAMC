

%% this section contains XX problems

%% AP Physics B practice workbook
%%--------------------------------------------------

%% Section B: Oscillations
%%--------------------------------------------------

%% Page 269
\element{AP}{
\begin{question}{oscillations-Q01}
    A car of mass $m$, traveling at speed $v$,
        stops in time $t$ when maximum braking force is applied. 
    Assuming the braking force is independent of mass,
        what time would be required to stop a car of mass $2m$ traveling at speed $v$?
    \begin{multicols}{2}
    \begin{choices}
        \wrongchoice{$\frac{1}{2} t$}
        \wrongchoice{$t$}
        \wrongchoice{$\sqrt{2} t$}
      \correctchoice{$2 t$}
        \wrongchoice{$4 t$}
    \end{choices}
    \end{multicols}
\end{question}
}



1.
A mass m, attached to a horizontal massless spring with spring constant k, is set into simple harmonic motion.
Its maximum displacement from its equilibrium position is A. What is the mass’s speed as it passes through its
equilibrium position?
( A ) 0
k
m
( B ) A
( C ) A
m
k
( D )
1 k
A m
( E )
1 m
A k
2. The period of a spring-mass system undergoing simple harmonic motion is T. If the amplitude of the spring-
mass system’s motion is doubled, the period will be:
(A) 1⁄4 T (B) 1⁄2 T (C) T (D) 2T (E) 4T
3. A simple pendulum of mass m and length L has a period of oscillation T at angular amplitude θ = 5° measured
from its equilibrium position. If the amplitude is changed to 10° and everything else remains constant, the new
period of the pendulum would be approximately.
(A) 2T (B) (√2) T (C) T (D) T / (√2) (E) T / 2
4. A mass m is attached to a spring with a spring constant k. If the mass is set into simple harmonic motion by a
displacement d from its equilibrium position, what would be the speed, v, of the mass when it returns to the
equilibrium position?
( A ) v =
kd
m
( B ) v 2 =
kd
m
( C ) v =
kd
mg
( D ) v 2 =
mgd
k
( E ) v = d
k
m
5. A mass on the end of a spring oscillates with the displacement vs.
time graph shown. Which of the following statements about its
motion is INCORRECT?
(A) The amplitude of the oscillation is 0.08 m.
(B) The frequency of oscillation is 0.5 Hz.
(C) The mass achieves a maximum in speed at 1 sec.
(D) The period of oscillation is 2 sec.
(E) The mass experiences a maximum in the size of the
acceleration at t=1.5 sec
6. What is the period of a simple pendulum if the cord length is 67 cm and the pendulum bob has a mass of 2.4 kg.
(A) 0.259 s (B) 1.63 s
(C) 3.86 s
(D) 16.3 s
(E) 24.3 s
7.
If the mass of a simple pendulum is doubled but its length remains constant, its period is multiplied by a factor of
1
1
(A) 2
(B)
2
(C) 1
(D)
2
(E) 2
8. Which of the following is true for a system consisting of a mass oscillating on the end of an ideal spring?
(A) The kinetic and potential energies are equal to each other at all times.
(B) The kinetic and potential energies are both constant.
(C) The maximum potential energy is achieved when the mass passes through its equilibrium position.
(D) The maximum kinetic energy and maximum potential energy are equal, but occur at different times.
(E) The maximum kinetic energy occurs at maximum displacement of the mass from its equilibrium position
9. The length of a simple pendulum with a period on Earth of one second is most nearly
(A) 0.12 m
(B) 0.25 m (C) 0.50 m
(D) 1.0 m
(E) 10.0 m




Questions 10-11: A block oscillates without friction on the end of a spring as shown above. The minimum and
maximum lengths of the spring as it oscillates are, respectively, x min and x max . The graphs below can represent
quantities associated with the oscillation as functions of the length x of the spring.
10. Which graph can represent the total mechanical energy of the block-spring system as a function of x ?
(A) A
(B) B
(C) C
(D) D
(E) E
11. Which graph can represent the kinetic energy of the block as a function of x ?
(A) A
(B) B
(C) C
(D) D
(E) E
12. An object swings on the end of a cord as a simple pendulum with period T. Another object oscillates up and
down on the end of a vertical spring also with period T. If the masses of both objects are doubled, what are the
new values for the Periods?
Pendulum
Mass on Spring
T
2
T 2
(A)
(B)
(C) T
T
(D) T 2
T 2
T
T
T
(E) T 2
294
2



13. An object is attached to a spring and oscillates with amplitude A
and period T, as represented on the graph. The nature of the
velocity v and acceleration a of the object at time T/4 is best
represented by which of the following?
(A) v > 0, a > 0
(B) v > 0, a < 0
(C) v > 0, a = 0
(D) v = 0, a < 0
(E) v = 0, a = 0
14. When an object oscillating in simple harmonic motion is at its maximum displacement from the equilibrium
position. Which of the following is true of the values of its speed and the magnitude of the restoring force?
Speed
Restoring Force
(A) Zero
Maximum
(B) Zero
Zero
(C) 1⁄2 maximum
1⁄2 maximum
(D) Maximum
1⁄2 maximum
(E) Maximum
Zero
15. A particle oscillates up and down in simple harmonic
motion. Its height y as a function of time t is shown in the
diagram. At what time t does the particle achieve its
maximum positive acceleration?
(A) 1 s
(B) 2 s
(C) 3 s
(D) 4 s
(E) None of the above, because the acceleration is
constant
16. The graph shown represents the potential energy U as a function of
displacement x for an object on the end of a spring oscillating in
simple harmonic motion with amplitude x ο . Which of the
following graphs represents the kinetic energy K of the object as a
function of displacement x ?



A sphere of mass m 1 , which is attached to a spring, is displaced downward from its equilibrium position as shown
above left and released from rest. A sphere of mass m 2 , which is suspended from a string of length L, is displaced to
the right as shown above right and released from rest so that it swings as a simple pendulum with small amplitude.
Assume that both spheres undergo simple harmonic motion
17. Which of the following is true for both spheres?
(A) The maximum kinetic energy is attained as the sphere passes through its equilibrium position
(B) The maximum kinetic energy is attained as the sphere reaches its point of release.
(C) The minimum gravitational potential energy is attained as the sphere passes through its equilibrium position.
(D) The maximum gravitational potential energy is attained when the sphere reaches its point of release.
(E) The maximum total energy is attained only as the sphere passes through its equilibrium position.
18. If both spheres have the same period of oscillation, which of the following is an expression for the spring
constant
(A) L / m 1 g
(B) g / m 2 L
(C) m 1 L / g
(D) m 2 g / L
(E) m 1 g / L
19. A block attached to the lower end of a vertical spring oscillates up and down. If the spring obeys
Hooke's law, the period of oscillation depends on which of the following?
I. Mass of the block
II. Amplitude of the oscillation
III. Force constant of the spring
(A) I only
(B) II only
(C) III only
(D) I and II
(E) I and III



20. A simple pendulum and a mass hanging on a spring both have a period of 1 s when set into small oscillatory
motion on Earth. They are taken to Planet X, which has the same diameter as Earth but twice the mass. Which of the
following statements is true about the periods of the two objects on Planet X compared to their periods on Earth?
(A) Both are shorter.
(B) Both are the same.
(C) Both are longer.
(D) The period of the mass on the spring is shorter; that of the pendulum is the same.
(E) The period of the pendulum is shorter; that of the mass on the spring is the same
21. A simple pendulum of length l, whose bob has mass m, oscillates with a period T. If the bob is replaced by one
of mass 4m, the period of oscillation is
1
(A) 4 T
1
(B) 2 T
(C) T
(D) 2T
(E)4T
Questions 22-23
A 0.l -kilogram block is attached to an initially unstretched spring of force constant k = 40 newtons per meter as
shown above. The block is released from rest at time t = 0.
22. What is the amplitude, in meters, of the resulting simple harmonic motion of the block?
(A)
1
m
40
(B)
1
m
20
(C)
1
m
4
(D)
1
m
2
(E) 1m
23. What will the resulting period of oscillation be?
π
(A)
40
s
π
(B)
20
s
π
(C)
10
s
π
(D)
5
s
π
(E)
4
s
24. A ball is dropped from a height of 10 meters onto a hard surface so that the collision at the surface may be
assumed elastic. Under such conditions the motion of the ball is
(A) simple harmonic with a period of about 1.4 s
(B) simple harmonic with a period of about 2.8 s
(C) simple harmonic with an amplitude of 5 m
(D) periodic with a period of about 2.8 s but not simple harmonic
(E) motion with constant momentum



Questions 25-26 refer to the graph below of the displacement x versus time t for a particle in simple harmonic
motion.
25. Which of the following graphs shows the kinetic energy K of the particle as a function of time t for one cycle of
motion?
26. Which of the following graphs shows the kinetic energy K of the particle as a function of its displacement x ?



27. A mass m is attached to a vertical spring stretching it distance d. Then, the mass is set oscillating on a spring
with an amplitude of A, the period of oscillation is proportional to
(A)
d
g
(B)
g
d
(C)
d
mg
(D)
m 2 g
d
(E)
m
g
28. Two objects of equal mass hang from independent springs of unequal spring constant and oscillate up and down.
The spring of greater spring constant must have the
(A) smaller amplitude of oscillation
(B) larger amplitude of oscillation
(C) shorter period of oscillation
(D) longer period of oscillation
(E) lower frequency of oscillation
Questions 29-30. A block on a horizontal frictionless plane is attached to a spring, as shown above. The block
oscillates along the x-axis with simple harmonic motion of amplitude A.
29. Which of the following statements about the block is correct?
(A) At x = 0, its velocity is zero.
(B) At x = 0, its acceleration is at a maximum.
(C) At x = A, its displacement is at a maximum.
(D) At x = A, its velocity is at a maximum.
(E) At x = A, its acceleration is zero.
30. Which of the following statements about energy is correct?
(A) The potential energy of the spring is at a minimum at x = 0.
(B) The potential energy of the spring is at a minimum at x = A.
(C) The kinetic energy of the block is at a minimum at x =0.
(D) The kinetic energy of the block is at a maximum at x = A.
(E) The kinetic energy of the block is always equal to the potential energy of the spring.
31. A simple pendulum consists of a l.0-kilogram brass bob on a string about 1.0 meter long. It has a period of 2.0
seconds. The pendulum would have a period of 1.0 second if the
(A) string were replaced by one about 0.25 meter long
(B) string were replaced by one about 2.0 meters long
(C) bob were replaced by a 0.25-kg brass sphere
(D) bob were replaced by a 4.0-kg brass sphere
(E) amplitude of the motion were increased
32. A pendulum with a period of 1 s on Earth, where the acceleration due to gravity is g, is taken to another planet,
where its period is 2 s. The acceleration due to gravity on the other planet is most nearly
(A) g/4
(B) g/2
(C) g
(D) 2g
(E) 4g
33. A frictionless pendulum of length 3 m swings with an amplitude of 10°. At its maximum displacement, the
potential energy of the pendulum is 10 J. What is the kinetic energy of the pendulum when its potential energy
is 5 J ?
(A) 3.3 J
(B) 5 J
(C) 6.7 J
(D) 10 J
(E) 15 J



34. An ideal massless spring is fixed to the wall at one end, as shown above. A block of mass M attached to the
other end of the spring oscillates with amplitude A on a frictionless, horizontal surface. The maximum speed of
the block is v m . The force constant of the spring is
(A)
Mg
A
(B)
Mgv m
2 A
(C)
Mv m 2
2 A
(D)
Mv m 2
A 2
(E)
Mv m 2
2A 2
35. A simple pendulum has a period of 2 s for small amplitude oscillations. The length of the pendulum is most
nearly
(A) 1/6 m (B) 1/4 m
(C) 1/2 m
(D) 1 m
(E) 2 m
36. A mass M suspended by a spring with force constant k has a period T when set into oscillation on Earth. Its
period on Mars, whose mass is about 1/9 and radius 1/2 that of Earth, is most nearly
(A) T/3
(B) 2T/3
(C) T
(D) 3T/2
(E) 3T
37. A mass M suspended on a string of length L has a period T when set into oscillation on Earth. Its period on
Mars, whose mass is about 1/9 and radius 1/2 that of Earth, is most nearly
(A) T/3
(B) 2T/3
(C) T
(D) 3T/2
(E) 3T
38. A 1.0 kg mass is attached to the end of a vertical ideal spring with a force constant of 400 N/m. The mass is set
in simple harmonic motion with an amplitude of 10 cm. The speed of the 1.0 kg mass at the equilibrium
position is
(A) 2 m/s
(B) 4 m/s
(C) 20 m/s
(D) 40 m/s
(E) 200 m/s
39. An object of mass m hanging from a spring of spring constant k oscillates with a certain frequency. What is the
length of a simple pendulum that has the same frequency of oscillation?
(A) mk / g (B) mg / k (C) kg / m (D) k / mg (E) g / mk
40. A platform of mass 2 kg is supported by a spring of negligible mass as shown.
The platform oscillates with a period of 3 s when the platform is pushed down
and released. What must be the mass of a block that when placed on the
platform doubles the period of oscillation to 6 s?
(A) 1 kg (B) 2 kg (C) 4 kg (D) 6 kg (E) 8 kg







\endinput


