
%% this section contains XX problems
%%----------------------------------------


%% AP Physics C: Mechanics 1984
%%------------------------------
\element{AP}{
\begin{question}{APC-1984-Q36}
    A parallel-plate capacitor has a capacitance $C_0$. 
    A second parallel-plate capacitor has plates with
        twice the area and twice the separation. 
    The capacitance of the second capacitor is most nearly:
    \begin{multicols}{2}
    \begin{choices}
        \wrongchoice{$\frac{1}{4} C_0$}
        \wrongchoice{$\frac{1}{2} C_0$}
        \wrongchoice{$C_0$}
        \wrongchoice{$2 C_0$}
        \wrongchoice{$4 C_0$}
    \end{choices}
    \end{multicols}
\end{question}
}

\element{AP}{
\begin{question}{APC-1984-Q37}
    When lighted, a \SI{100}{\watt} light bulb operating on a
        \SI{110}{\volt} household circuit has a resistance closest to:
    \begin{multicols}{2}
    \begin{choices}
        \wrongchoice{\SI{e-2}{\ohm}}
        \wrongchoice{\SI{e-1}{\ohm}}
        \wrongchoice{\SI{e0}{\ohm}}
        \wrongchoice{\SI{e1}{\ohm}}
        \wrongchoice{\SI{e2}{\ohm}}
    \end{choices}
    \end{multicols}
\end{question}
}

\element{AP}{
\begin{question}{APC-1984-Q38}
    If $I$ is current, $t$ is time, $E$ is electric field intensity,
        and $x$ is distance, the ratio of $\int t\,\mathrm{d}t$
        to $\int E\,\mathrm{d}x$ may be expressed in:
    \begin{multicols}{2}
    \begin{choices}
        \wrongchoice{coulombs (\si{\coulomb})}
        \wrongchoice{joules (\si{\joule})}
        \wrongchoice{newtons (\si{\newton})}
        \wrongchoice{farads (\si{\farad})}
        \wrongchoice{henrys (\si{\henry})}
    \end{choices}
    \end{multicols}
\end{question}
}

\element{AP}{
\begin{question}{APC-1984-Q39}
    The electric field $E$ just outside the
        surface of a charged conductor is:
    \begin{choices}
        \wrongchoice{directed perpendicular to the surface}
        \wrongchoice{directed parallel to the surface}
        \wrongchoice{independent of the surface charge density}
        \wrongchoice{zero}
        \wrongchoice{infinite}
    \end{choices}
\end{question}
}

\element{AP}{
\begin{question}{APC-1984-Q40}
    A closed surface, in the shape of a cube of side $a$,
        is oriented as shown above in a region where there
        is a constant electric field of magnitude $E$ parallel to the $x$-axis. 
    \begin{center}
    \begin{tikzpicture}
    \end{tikzpicture}
    \end{center}
    The total electric flux through the cubical surface is:
    \begin{multicols}{2}
    \begin{choices}
        \wrongchoice{$-Ea^2$}
        \wrongchoice{zero}
        \wrongchoice{$Ea^2$}
        \wrongchoice{$2Ea^2$}
        \wrongchoice{$6Ea^2$}
    \end{choices}
    \end{multicols}
\end{question}
}

\element{AP}{
\begin{question}{APC-1984-Q41}
    In the circuit shown above, what is the resistance $R$?
    \begin{multicols}{2}
    \begin{choices}
        \wrongchoice{\SI{3}{\ohm}}
        \wrongchoice{\SI{4}{\ohm}}
        \wrongchoice{\SI{6}{\ohm}}
        \wrongchoice{\SI{12}{\ohm}}
        \wrongchoice{\SI{18}{\ohm}}
    \end{choices}
    \end{multicols}
\end{question}
}

\element{AP}{
\begin{question}{APC-1984-Q42}
    The figure below shows a spherical distribution of
        charge of radius $R$ and constant charge density $\rho$. 
    \begin{center}
    \begin{tikzpicture}
    \end{tikzpicture}
    \end{center}
    Which of the following graphs best represents the
        electric field strength $E$ as a function of the
        distance $r$ from the center of the sphere?
    \begin{multicols}{2}
    \begin{choices}
        \wrongchoice{
            \begin{tikzpicture}
            \end{tikzpicture}
        }
    \end{choices}
    \end{multicols}
\end{question}
}

\element{AP}{
\begin{question}{APC-1984-Q43}
    Points $R$ and $S$ are each the same distance $d$
        from two unequal charges, $+Q$ and $+2Q$, as shown below. 
    \begin{center}
    \begin{tikzpicture}
    \end{tikzpicture}
    \end{center}
    The work required to move a charge $-Q$ from point $R$ to point $S$ is
    \begin{choices}
        \wrongchoice{dependent on the path taken from $R$ to $S$}
        \wrongchoice{directly proportional to the distance between $R$ and $S$}
        \wrongchoice{positive}
        \wrongchoice{zero}
        \wrongchoice{negative}
    \end{choices}
\end{question}
}

\element{AP}{
\begin{question}{APC-1984-Q44}
    The five resistors shown below have the lengths and
        cross-sectional areas indicated and are made of
        material with the same resistivity. 
    Which has the greatest resistance?
    \begin{multicols}{2}
    \begin{choices}
        \wrongchoice{
            \begin{tikzpicture}
            \end{tikzpicutre}
        }
    \end{choices}
    \end{multicols}
\end{question}
}

\element{AP}{
\begin{question}{APC-1984-Q45}
    A \SI{12}{\volt} storage battery, with an internal resistance of \SI{2}{\ohm},
        is being charged by a current of \SI{2}{\ampere} as
        shown in the diagram below.
    \begin{center}
    \begin{tikzpicture}
    \end{tikzpicture}
    \end{center}
    Under these circumstances,
        a voltmeter connected across the terminals of the battery will read:
    \begin{multicols}{2}
    \begin{choices}
        \wrongchoice{\SI{4}{\volt}}
        \wrongchoice{\SI{8}{\volt}}
        \wrongchoice{\SI{10}{\volt}}
        \wrongchoice{\SI{12}{\volt}}
        \wrongchoice{\SI{16}{\volt}}
    \end{choices}
    \end{multicols}
\end{question}
}

\element{AP}{
\begin{question}{APC-1984-Q46}
    A galvanometer has a resistance of \SI{99}{\ohm} and
        deflects full scale when a current of \SI{e-3}{\ampere} passes through it. 
    In order to convert this galvanometer into an ammeter
        with a full-scale deflection of \SI{0.1}{\ampere},
        one should connect a resistance of:
    \begin{choices}
        \wrongchoice{\SI{1}{\ohm} in series with it}
        \wrongchoice{\SI{901}{\ohm} in series with it}
        \wrongchoice{\SI{9 900}{\ohm} in series with it}
        \wrongchoice{\SI{1}{\ohm} in parallel with it}
        \wrongchoice{\SI{9 900}{\ohm} in parallel with it}
    \end{choices}
\end{question}
}

\element{AP}{
\begin{question}{APC-1984-Q47}
    Two long, parallel wires, fixed in space, carry currents $I_1$ and $I_2$. 
    The force of attraction has magnitude $F$. 
    What currents will give an attractive force of magnitude $4F$?
    \begin{choices}
        \wrongchoice{$2 I_1$ and $\frac{1}{2} I_2$}
        \wrongchoice{$I_1$ and $\frac{1}{4} I_2$}
        \wrongchoice{$\frac{1}{2} I_1$ and $\frac{1}{2} I_2$}
        \wrongchoice{$2 I_1$ and $2 I_2$}
        \wrongchoice{$4 I_1$ and $4 I_2$}
    \end{choices}
\end{question}
}

\element{AP}{
\begin{question}{APC-1984-Q48}
    In each of the following situations,
        a bar magnet is aligned along the axis of a conducting loop. 
    The magnet and the loop move with the indicated velocities. 
    In which situation will the bar magnet \emph{not} induce
        a current in the conducting loop?
    \begin{multicols}{2}
    \begin{choices}
        \wrongchoice{
            \begin{tikzpicture}
            \end{tikzpicture}
        }
    \end{choices}
    \end{multicols}
\end{question}
}

\newcommand{\APCNineteenEightyFourQFortyNine}{
\begin{tikzpicture}
\end{tikzpicture}
}

\element{AP}{
\begin{question}{APC-1984-Q49}
    The ends of a metal bar rest on two horizontal north-south rails as shown below.
    \begin{center}
        \APCNineteenEightyFourQFortyNine
    \end{center}
    The bar may slide without friction freely with its length horizontal and lying east and west as shown above.
    There is a magnetic field parallel to the rails and directed north.

    If the bar is pushed northward on the rails, the electromotive force induced in the bar as a result of the magnetic field will
    \begin{choices}
        \wrongchoice{be directed upward}
        \wrongchoice{be zero}
        \wrongchoice{produce a westward current}
        \wrongchoice{produce an eastward current}
        \wrongchoice{stop the motion of the bar}
    \end{choices}
\end{question}
}

\element{AP}{
\begin{question}{APC-1984-Q50}
    The ends of a metal bar rest on two horizontal north-south rails as shown below.
    \begin{center}
        \APCNineteenEightyFourQFortyNine
    \end{center}
    The bar may slide without friction freely with its length horizontal and lying east and west as shown above.
    There is a magnetic field parallel to the rails and directed north.

    A battery is connected between the rails and causes the electrons in the bar to drift to the east. 
    The resulting magnetic force on the bar is directed:
    \begin{multicols}{2}
    \begin{choices}
        \wrongchoice{north}
        \wrongchoice{south}
        \wrongchoice{east}
        \wrongchoice{west}
        \wrongchoice{vertically}
    \end{choices}
    \end{multicols}
\end{question}
}

\element{AP}{
\begin{question}{APC-1984-Q51}
    A charged particle is projected with its initial velocity parallel to a uniform magnetic field. 
    The resulting path is a:
    \begin{choices}
        \wrongchoice{spiral}
        \wrongchoice{parabolic arc}
        \wrongchoice{circular arc}
        \wrongchoice{straight line parallel to the field}
        \wrongchoice{straight line perpendicular to the field}
    \end{choices}
\end{question}
}

\newcommand{\APCNineteenEightyFourQFiftyTwo}{
\begin{tikzpicture}
\end{tikzpicture}
}

\element{AP}{
\begin{question}{APC-1984-Q52}
    Two positive charges of magnitude $q$ are each a distance d from the origin $A$ of a coordinate system as shown below.
    \begin{center}
        \APCNineteenEightyFourQFiftyTwo
    \end{center}
    At which of the following points is the electric field least in magnitude?

    53. At which of the following points is the electric potential greatest in magnitude?
    \begin{multicols}{5}
    \begin{choices}[o]
        \wrongchoice{A}
        \wrongchoice{B}
        \wrongchoice{C}
        \wrongchoice{D}
        \wrongchoice{E}
    \end{choices}
    \end{multicols}
\end{question}
}

\element{AP}{
\begin{question}{APC-1984-Q53}
    Two positive charges of magnitude $q$ are each a distance d from the origin $A$ of a coordinate system as shown below.
    \begin{center}
        \APCNineteenEightyFourQFiftyTwo
    \end{center}
    At which of the following points is the electric potential greatest in magnitude?
    \begin{multicols}{5}
    \begin{choices}[o]
        \wrongchoice{A}
        \wrongchoice{B}
        \wrongchoice{C}
        \wrongchoice{D}
        \wrongchoice{E}
    \end{choices}
    \end{multicols}
\end{question}
}

\newcommand{\APCNineteenEightyFourQFiftyFour}{
\begin{tikzpicture}
\end{tikzpicture}
}

\element{AP}{
\begin{question}{APC-1984-Q54}
    The batteries in each of the circuits shown above are identical and the wires have negligible resistance.
    \begin{center}
        \APCNineteenEightyFourQFiftyFour
    \end{center}
    In which circuit is the current furnished by the battery the greatest?
    \begin{multicols}{5}
    \begin{choices}[o]
        \wrongchoice{A}
        \wrongchoice{B}
        \wrongchoice{C}
        \wrongchoice{D}
        \wrongchoice{E}
    \end{choices}
    \end{multicols}
\end{question}
}

\element{AP}{
\begin{question}{APC-1984-Q55}
    The batteries in each of the circuits shown above are identical and the wires have negligible resistance.
    \begin{center}
        \APCNineteenEightyFourQFiftyFour
    \end{center}
    In which circuit is the equivalent resistance connected to the battery the greatest?
    \begin{multicols}{5}
    \begin{choices}[o]
        \wrongchoice{A}
        \wrongchoice{B}
        \wrongchoice{C}
        \wrongchoice{D}
        \wrongchoice{E}
    \end{choices}
    \end{multicols}
\end{question}
}

\element{AP}{
\begin{question}{APC-1984-Q56}
    The batteries in each of the circuits shown above are identical and the wires have negligible resistance.
    \begin{center}
        \APCNineteenEightyFourQFiftyFour
    \end{center}
    Which circuit dissipates the least power?
    \begin{multicols}{5}
    \begin{choices}[o]
        \wrongchoice{A}
        \wrongchoice{B}
        \wrongchoice{C}
        \wrongchoice{D}
        \wrongchoice{E}
    \end{choices}
    \end{multicols}
\end{question}
}


%% NOTE: create common command for graphs and circuits

\element{AP}{
\begin{question}{APC-1984-Q57}
    refer to the circuit shown below.
    Assume the capacitor C is initially uncharged. 
    The following graphs may represent different quantities
        related to the circuit as functions of time $t$
        after the switch $S$ is closed.
    \begin{center}
        \begin{circuitikz}
        \end{circuitikz}
        %\APCNineteenEightyFourQFiftyFour
    \end{center}
    Which graph best represents the voltage versus time across the resistor R?
    \begin{multicols}{5}
    \begin{choices}[o]
        \wrongchoice{A}
        \wrongchoice{B}
        \wrongchoice{C}
        \wrongchoice{D}
        \wrongchoice{E}
    \end{choices}
    \end{multicols}
\end{question}
}

\element{AP}{
\begin{question}{APC-1984-Q58}
    refer to the circuit shown below.
    Assume the capacitor C is initially uncharged. 
    The following graphs may represent different quantities
        related to the circuit as functions of time $t$
        after the switch $S$ is closed.
    \begin{center}
        \begin{circuitikz}
        \end{circuitikz}
        %\APCNineteenEightyFourQFiftyFour
    \end{center}
    Which graph best represents the current versus time in the circuit?
    \begin{multicols}{5}
    \begin{choices}[o]
        \wrongchoice{A}
        \wrongchoice{B}
        \wrongchoice{C}
        \wrongchoice{D}
        \wrongchoice{E}
    \end{choices}
    \end{multicols}
\end{question}
}

\element{AP}{
\begin{question}{APC-1984-Q59}
    refer to the circuit shown below.
    Assume the capacitor C is initially uncharged. 
    The following graphs may represent different quantities
        related to the circuit as functions of time $t$
        after the switch $S$ is closed.
    \begin{center}
        \begin{circuitikz}
        \end{circuitikz}
        %\APCNineteenEightyFourQFiftyFour
    \end{center}
    Which graph best represents the voltage across the capacitor versus time?
    \begin{multicols}{5}
    \begin{choices}[o]
        \wrongchoice{A}
        \wrongchoice{B}
        \wrongchoice{C}
        \wrongchoice{D}
        \wrongchoice{E}
    \end{choices}
    \end{multicols}
\end{question}
}

\element{AP}{
\begin{question}{APC-1984-Q60}
    Three \SI{6}{\micro\farad} capacitors are connected in
        series with a \SI{6}{\volt} battery.
    The equivalent capacitance of the set of capacitors is:
    \begin{multicols}{2}
    \begin{choices}
        \wrongchoice{\SI{0.5}{\micro\farad}}
        \wrongchoice{\SI{2}{\micro\farad}}
        \wrongchoice{\SI{3}{\micro\farad}}
        \wrongchoice{\SI{9}{\micro\farad}} 
        \wrongchoice{\SI{18}{\micro\farad}}
    \end{choices}
    \end{multicols}
\end{question}
}

\element{AP}{
\begin{question}{APC-1984-Q61}
    Three \SI{6}{\micro\farad} capacitors are connected in
        series with a \SI{6}{\volt} battery.
    The energy stored in each capacitor is:
    \begin{multicols}{2}
    \begin{choices}
        \wrongchoice{\SI{4}{\micro\joule}}
        \wrongchoice{\SI{6}{\micro\joule}}
        \wrongchoice{\SI{12}{\micro\joule}}
        \wrongchoice{\SI{18}{\micro\joule}}
        \wrongchoice{\SI{36}{\micro\joule}}
    \end{choices}
    \end{multicols}
\end{question}
}

\element{AP}{
\begin{question}{APC-1984-Q62}
    A rigid insulated rod, with two unequal charges attached to its ends,
        is placed in a uniform electric field $E$ as shown below.
    \begin{center}
    \begin{tikzpicture}
    \end{tikzpicture}
    \end{center}
    The rod experiences a:
    \begin{choices}
        \wrongchoice{net force to the left and a clockwise rotation}
        \wrongchoice{net force to the left and a counterclockwise rotation}
        \wrongchoice{net force to the right and a clockwise rotation}
        \wrongchoice{net force to the right and a counterclockwise rotation}
        \wrongchoice{rotation, but no net force}
    \end{choices}
\end{question}
}

\element{AP}{
\begin{question}{APC-1984-Q63}
    A loop of wire is pulled with constant velocity $v$ to the right
        through a region of space where there is a uniform
        magnetic field $B$ directed into the page, as shown below. 
    \begin{center}
    \begin{tikzpicture}
    \end{tikzpicture}
    \end{center}
    The magnetic force on the loop is:
    \begin{choices}
        \wrongchoice{directed to the left both as it enters and as it leaves the region}
        \wrongchoice{directed to the right both as it enters and as it leaves the region}
        \wrongchoice{directed to the left as it enters the region and to the right as it leaves}
        \wrongchoice{directed to the right as it enters the region and to the left as it leaves}
        \wrongchoice{zero at all times}
    \end{choices}
\end{question}
}

\element{AP}{
\begin{question}{APC-1984-Q64}
    The electric field of two long coaxial cylinders is represented by lines of force as shown below. 
    \begin{center}
    \begin{tikzpicture}
    \end{tikzpicture}
    \end{center}
    The charge on the inner cylinder is $+Q$.
    The charge on the outer cylinder is:
    \begin{multicols}{3}
    \begin{choices}
        \wrongchoice{$+3Q$}
        \wrongchoice{$+Q$}
        \wrongchoice{$0$}
        \wrongchoice{$- Q$}
        \wrongchoice{$-3Q$}
    \end{choices}
    \end{multicols}
\end{question}
}

\element{AP}{
\begin{question}{APC-1984-Q65}
    Two identical conducting spheres are charged to $+2Q$ and $-Q$,
        respectively, and are separated by a distance $d$
        (much greater than the radii of the spheres) as shown below.
    \begin{center}
    \begin{tikzpicture}
    \end{tikzpicture}
    \end{center}
    The magnitude of the force of attraction on the left sphere is $F_1$.
    After the two spheres are made to touch and then are re-separated by distance $d$,
        the magnitude of the force on the left sphere is $F_2$. 
    Which of the following relationships is correct?
    \begin{multicols}{2}
    \begin{choices}
        \wrongchoice{$2 F_1 = F_2$}
        \wrongchoice{$F_1 = F_2$}
        \wrongchoice{$F_1 = 2 F_2$}
        \wrongchoice{$F_1 = 4 F_2$}
        \wrongchoice{$F_1 = 8 F_2$}
    \end{choices}
    \end{multicols}
\end{question}
}

\element{AP}{
\begin{question}{APC-1984-Q66}
    An isolated capacitor with air between its plates has
        a potential difference $V_0$ and a charge $Q_0$.
    After the space between the plates is filled with oil,
        the difference in potential is $V$ and the charge is $Q$. 
    Which of the following pairs of relationships is correct?
    \begin{multicols}{2}
    \begin{choices}
        \wrongchoice{$Q=Q_0$ and $V>V_0$}
        \wrongchoice{$Q=Q_0$ and $V<V_0$}
        \wrongchoice{$Q>Q_0$ and $V=V_0$}
        \wrongchoice{$Q<Q_0$ and $V<V_0$}
        \wrongchoice{$Q>Q_0$ and $V>V_0$}
    \end{choices}
    \end{multicols}
\end{question}
}

\element{AP}{
\begin{question}{APC-1984-Q67}
    A solid cylindrical conductor of radius $R$ carries
        a current $I$ uniformly distributed throughout its interior. 
    Which of the following graphs best represents the magnetic
        field intensity as a function of $r$,
        the radial distance from the axis of the cylinder?
    \begin{multicols}{2}
    \begin{choices}
        \wrongchoice{
            \begin{tikzpicture}
            \end{tikzpicture}
        }
    \end{choices}
    \end{multicols}
\end{question}
}

\element{AP}{
\begin{question}{APC-1984-Q68}
    The cross section below shows a long solenoid of length $l$
        and radius $r$ consisting of $N$ closely wound turns of wire. 
    \begin{center}
    \begin{tikzpicture}
    \end{tikzpicture}
    \end{center}
    When the current in the wire is $I$,
        the magnetic field within this solenoid has magnitude $B_0$. 
    A solenoid with the same number of turns $N$,
        length $l$, and current $I$, but with radius $2r$,
        would have a magnetic field of magnitude most nearly equal to:
    \begin{multicols}{2}
    \begin{choices}
        \wrongchoice{$\dfrac{B_0}{4}$}
        \wrongchoice{$\dfrac{B_0}{2}$}
        \wrongchoice{$B_0$}
        \wrongchoice{$2B_0$}
        \wrongchoice{$4B_0$}
    \end{choices}
    \end{multicols}
\end{question}
}

\element{AP}{
\begin{question}{APC-1984-Q69}
    In the circuit shown below,
        the capacitor $C$ is first charged by throwing switch $S$ to the left,
        and then discharged by throwing $S$ to the right. 
    \begin{center}
    \begin{circuitikz}
    \end{circuitikz}
    \end{center}
    The time constant for discharge could be increased by which of the following?
    \begin{choices}
        \wrongchoice{Placing another capacitor in parallel with $C$}
        \wrongchoice{Placing another capacitor in series with $C$}
        \wrongchoice{Placing another resistor in parallel with the resistor $R$}
        \wrongchoice{Increasing battery emf, $\epsilon$}
        \wrongchoice{Decreasing battery emf, $\epsilon$}
    \end{choices}
\end{question}
}

\element{AP}{
\begin{question}{APC-1984-Q70}
    Concentric conducting spheres of radii $a$ and $2a$ bear equal
        but opposite charges $+Q$ and $-Q$, respectively. 
    \begin{center}
    \begin{circuitikz}
    \end{circuitikz}
    \end{center}
    Which of the following graphs best represents the 
        electric potential $V$ as a function of $r$?
    \begin{multicols}{2}
    \begin{choices}
        \wrongchoice{
            \begin{tikzpicture}
            \end{tikzpicture}
        }
    \end{choices}
    \end{multicols}
\end{question}
}


\endinput

