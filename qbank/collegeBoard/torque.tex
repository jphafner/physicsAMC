

%% this section contains XX problems

%% AP Physics B practice workbook
%%--------------------------------------------------

%% Section B: Torque
%%--------------------------------------------------
\element{AP}{
\begin{question}{torque-Q01}
    A uniform meterstick of mass \SI{0.20}{\kilo\gram} is pivoted at the \SI{40}{\centi\meter} mark. 
    Where should one hang a mass of \SI{0.50}{\kilo\gram} to balance the stick?
    \begin{multicols}{3}
    \begin{choices}
        \wrongchoice{\SI{16}{\centi\meter}}
      \correctchoice{\SI{36}{\centi\meter}}
        \wrongchoice{\SI{44}{\centi\meter}}
        \wrongchoice{\SI{46}{\centi\meter}}
        \wrongchoice{\SI{54}{\centi\meter}}
    \end{choices}
    \end{multicols}
\end{question}
}

\element{AP}{
\begin{question}{torque-Q02}
    A uniform meterstick is balanced at its midpoint with several forces applied as shown below. 
    If the stick is in equilibrium,
        the magnitude of the force $X$ in newtons (\si{\newton}) is:
    \begin{multicols}{2}
    \begin{choices}
      \correctchoice{\SI{50}{\newton}}
        \wrongchoice{\SI{100}{\newton}}
        \wrongchoice{\SI{200}{\newton}}
        \wrongchoice{\SI{300}{\newton}} 
        \wrongchoice{impossible to determine without the weight of the stick}
    \end{choices}
    \end{multicols}
\end{question}
}

\element{AP}{
\begin{question}{torque-Q03}
    A door (seen from above in the figures below) has hinges on the left hand side. 
    Which force produces the largest torque? 
    The magnitudes of all forces are equal.
    \begin{multicols}{2}
    \begin{choices}
        %% NOTE: ans is C
        \wrongchoice{
            \begin{tikzpicture}
            \end{tikzpicture}
        }
    \end{choices}
    \end{multicols}
\end{question}
}

\element{AP}{
\begin{question}{torque-Q04}
    A meterstick is supported at each side by a spring scale. 
    A heavy mass is then hung on the meterstick so that the spring scale on the left hand side reads four times the value of the spring scale on the right hand side. 
    If the mass of the meterstick is negligible compared to the hanging mass,
        how far from the right hand side is the large mass hanging.
    \begin{multicols}{2}
    \begin{choices}
        \wrongchoice{\SI{25}{\centi\meter}}
        \wrongchoice{\SI{50}{\centi\meter}}
        \wrongchoice{\SI{67}{\centi\meter}}
        \wrongchoice{\SI{75}{\centi\meter}}
      \correctchoice{\SI{80}{\centi\meter}}
    \end{choices}
    \end{multicols}
\end{question}
}

\element{AP}{
\begin{question}{torque-Q05}
    A uniform meter stick has a \SI{45.0}{\gram} mass placed at the \SI{20}{\centi\meter} mark as shown in the figure. 
    If a pivot is placed at the \SI{42.5}{\centi\meter} mark and the meter stick remains horizontal in static equilibrium,
        what is the mass of the meter stick?
    \begin{multicols}{2}
    \begin{choices}
        \wrongchoice{\SI{18.0}{\gram}}
        \wrongchoice{\SI{45.0}{\gram}}
        \wrongchoice{\SI{72.0}{\gram}}
        \wrongchoice{\SI{120.0}{\gram}}
      \correctchoice{\SI{135.0}{\gram}}
    \end{choices}
    \end{multicols}
\end{question}
}

\element{AP}{
\begin{question}{torque-Q06}
    A massless rigid rod of length $3d$ is pivoted at a fixed point $W$,
        and two forces each of magnitude $F$ are applied vertically upward as shown.
    \begin{center}
    \begin{tikzpicture}
        %% NOTE:
    \end{tikzpicture}
    \end{center}
    A third vertical force of magnitude $F$ may be applied,
        either upward or downward, at one of the labeled points.
    With the proper choice of direction at each point,
        the rod can be in equilibrium if the third force of magnitude $F$ is applied at point:
    \begin{multicols}{2}
    \begin{choices}
        \wrongchoice{$W$ only}
        \wrongchoice{$Y$ only}
      \correctchoice{$V$ or $X$ only}
        \wrongchoice{$V$ or $Y$ only}
        \wrongchoice{$V$, $W$, or $X$}
    \end{choices}
    \end{multicols}
\end{question}
}

\element{AP}{
\begin{question}{torque-Q07}
    A massless rigid rod of length $3d$ is pivoted at a fixed point $W$,
        and two forces each of magnitude $F$ are applied vertically upward as shown.
    \begin{center}
    \begin{tikzpicture}
        %% NOTE:
    \end{tikzpicture}
    \end{center}
    A \SI{5}{\meter} uniform plank of mass \SI{100}{\kilo\gram} rests on the top of a building with \SI{2}{\meter} extended over the edge as shown. 
    How far can a 50-kilogram person venture past the edge of the building on the plank before the plank just begins to tip?
    \begin{multicols}{2}
    \begin{choices}
        \wrongchoice{\SI{0.5}{\meter}}
      \correctchoice{\SI{1}{\meter}}
        \wrongchoice{\SI{1.5}{\meter}}
        \wrongchoice{\SI{2}{\meter}}
        \wrongchoice{It is impossible to make the plank tip since the person would have to be more than \SI{2}{\meter} from the edge of the building.}
    \end{choices}
    \end{multicols}
\end{question}
}

\element{AP}{
\begin{question}{torque-Q08}
    A massless rigid rod of length $3d$ is pivoted at a fixed point $W$,
        and two forces each of magnitude $F$ are applied vertically upward as shown.
    \begin{center}
    \begin{tikzpicture}
        %% NOTE:
    \end{tikzpicture}
    \end{center}
    To weigh a fish, a person hangs a tackle box of mass \SI{3.5}{\kilo\gram} and a cooler of mass \SI{5}{\kilo\gram} from the ends of a uniform rigid pole that is suspended by a rope attached to its center. 
    The system balances when the fish hangs at a point $\frac{1}{4}$ of the rod's length from the tackle box. 
    What is the mass of the fish? 
    \begin{multicols}{2}
    \begin{choices}
        \wrongchoice{\SI{1.5}{\kilo\gram}}
        \wrongchoice{\SI{2}{\kilo\gram}}
      \correctchoice{\SI{3}{\kilo\gram}}
        \wrongchoice{\SI{6}{\kilo\gram}}
        \wrongchoice{\SI{6.5}{\kilo\gram}}
    \end{choices}
    \end{multicols}
\end{question}
}

\element{AP}{
\begin{question}{torque-Q09}
    Two objects, of masses \SI{6}{\kilo\gram} and \SI{8}{\kilo\gram},
        are hung from the ends of a stick that is \SI{70}{\centi\meter} long and has marks every \SI{10}{\centi\meter},
        as shown. 
    \begin{center}
    \begin{tikzpicture}
        %% NOTE:
    \end{tikzpicture}
    \end{center}
    If the mass of the stick is negligible,
        at which of the points indicated should a cord be attached if the stick is to remain horizontal when suspended from the cord?
    \begin{multicols}{3}
    \begin{choices}
        \wrongchoice{$A$}
        \wrongchoice{$B$}
        \wrongchoice{$C$}
      \correctchoice{$D$}
        \wrongchoice{$E$}
    \end{choices}
    \end{multicols}
\end{question}
}

\element{AP}{
\begin{question}{torque-Q10}
    A wheel of radius R and negligible mass is mounted on a horizontal frictionless axle so that the wheel is in a vertical plane. 
    Three small objects having masses $m$, $M$, and $2M$, respectively,
        are mounted on the rim of the wheel, as shown. 
    \begin{center}
    \begin{tikzpicture}
        %% NOTE:
    \end{tikzpicture}
    \end{center}
    If the system is in static equilibrium, what is the value of $m$ in terms of $M$?
    \begin{multicols}{3}
    \begin{choices}
        \wrongchoice{$\dfrac{M}{2}$}
        \wrongchoice{$M$}
      \correctchoice{$\dfrac{3M}{2}$}
        \wrongchoice{$2M$}
        \wrongchoice{$\dfrac{5M}{2}$}
    \end{choices}
    \end{multicols}
\end{question}
}

\element{AP}{
\begin{question}{torque-Q11}
    A rod on a horizontal tabletop is pivoted at one end and is free to rotate without friction about a vertical axis, as shown. 
    \begin{center}
    \begin{tikzpicture}
        %% NOTE:
    \end{tikzpicture}
    \end{center}
    A force $F$ is applied at the other end,
        at an angle $\theta$ to the rod. 
    If $F$ were to be applied perpendicular to the rod,
        at what distance from the axis should it be applied in order to produce the same torque?
    \begin{multicols}{3}
    \begin{choices}
      \correctchoice{$L\sin\theta$}
        \wrongchoice{$L\cos\theta$}
        \wrongchoice{$L$}
        \wrongchoice{$L\tan\theta$}
        \wrongchoice{$2 L$}
    \end{choices}
    \end{multicols}
\end{question}
}

\element{AP}{
\begin{question}{torque-Q12}
    %% Questions 12-13
    A horizontal, uniform board of weight \SI{125}{\newton} and length \SI{4}{\meter} is supported by vertical chains at each end. 
    A person weighing \SI{500}{\newton} is sitting on the board. 
    The tension in the right chain is \SI{250}{\newton}.

    What is the tension in the left chain?
    \begin{multicols}{3}
    \begin{choices}
        \wrongchoice{\SI{250}{\newton}}
      \correctchoice{\SI{375}{\newton}}
        \wrongchoice{\SI{500}{\newton}}
        \wrongchoice{\SI{625}{\newton}}
        \wrongchoice{\SI{875}{\newton}}
    \end{choices}
    \end{multicols}
\end{question}
}

\element{AP}{
\begin{question}{torque-Q13}
    %% Questions 12-13
    A horizontal, uniform board of weight \SI{125}{\newton} and length \SI{4}{\meter} is supported by vertical chains at each end. 
    A person weighing \SI{500}{\newton} is sitting on the board. 
    The tension in the right chain is \SI{250}{\newton}.

    How far from the left end of the board is the person sitting?
    \begin{multicols}{3}
    \begin{choices}
        \wrongchoice{\SI{0.4}{\meter}} 
      \correctchoice{\SI{1.5}{\meter}} 
        \wrongchoice{\SI{2}{\meter}} 
        \wrongchoice{\SI{2.5}{\meter}} 
        \wrongchoice{\SI{3}{\meter}}
    \end{choices}
    \end{multicols}
\end{question}
}

\element{AP}{
\begin{question}{torque-Q14}
    Torque is the rotational analogue of:
    %\begin{multicols}{2}
    \begin{choices}
        \wrongchoice{kinetic energy}
        \wrongchoice{linear momentum}
        \wrongchoice{acceleration}
      \correctchoice{force}
        \wrongchoice{mass}
    \end{choices}
    %\end{multicols}
\end{question}
}

\element{AP}{
\begin{question}{torque-Q15}
    A square piece of plywood on a horizontal tabletop is subjected to the two horizontal forces shown. 
    Where should a third force of magnitude \SI{5}{\newton} be applied to put the piece of plywood into equilibrium?
    \begin{multicols}{2}
    \begin{choices}
        %% NOTE: ans is A
        \wrongchoice{
            \begin{tikzpicture}
            \end{tikzpicture}
        }
    \end{choices}
    \end{multicols}
\end{question}
}

\element{AP}{
\begin{question}{torque-Q16}
    A uniform rigid bar of weight $W$ is supported in a horizontal orientation as shown by a rope that makes a \ang{30} angle with the horizontal. 
    The force exerted on the bar at point $O$,
        where it is pivoted, is best represented by a vector whose direction is which of the following?
    \begin{multicols}{2}
    \begin{choices}
        %% NOTE: ans is B
        \wrongchoice{
            \begin{tikzpicture}
            \end{tikzpicture}
        }
    \end{choices}
    \end{multicols}
\end{question}
}

\element{AP}{
\begin{question}{torque-Q17}
    In which of the following diagrams is the torque about point $O$ equal in magnitude to the torque about point $X$ in the diagram? 
    (All forces lie in the plane of the paper.)
    \begin{multicols}{2}
    \begin{choices}
        %% NOTE: ans is C
        \wrongchoice{
            \begin{tikzpicture}
            \end{tikzpicture}
        }
    \end{choices}
    \end{multicols}
\end{question}
}

\element{AP}{
\begin{question}{torque-Q18}
    A rod of length $L$ and of negligible mass is pivoted at a point that is off-center with lengths shown in the figure below. 
    \begin{center}
    \begin{tikzpicture}
        %% NOTE:
    \end{tikzpicture}
    \end{center}
    The figures show two cases in which masses are suspended from the ends of the rod. 
    In each case the unknown mass $m$ is balanced by a known mass, $M_1$ or $M_2$,
        so that the rod remains horizontal. 
    What is the value of $m$ in terms of the known masses?
    \begin{multicols}{2}
    \begin{choices}
        \wrongchoice{$M_1 + M_2$}
        \wrongchoice{$\frac{1}{2} \left( M_1 + M_2 \right)$}
        \wrongchoice{$M_1 M_2$}
        \wrongchoice{$\frac{1}{2} M_1 M_2$}
      \correctchoice{$\sqrt{ M_1 M_2 }$}
    \end{choices}
    \end{multicols}
\end{question}
}

\element{AP}{
\begin{question}{torque-Q19}
    A system of two wheels fixed to each other is free to rotate about a frictionless axis through the common center of the wheels and perpendicular to the page. 
    Four forces are exerted tangentially to the rims of the wheels, as shown. 
    \begin{center}
    \begin{tikzpicture}
        %% NOTE:
    \end{tikzpicture}
    \end{center}
    The magnitude of the net torque on the system about the axis is:
    \begin{multicols}{3}
    \begin{choices}
        \wrongchoice{zero}
        \wrongchoice{$ FR$}
      \correctchoice{$2 FR$}
        \wrongchoice{$5 FR$}
        \wrongchoice{$14 FR$}
    \end{choices}
    \end{multicols}
\end{question}
}

\element{AP}{
\begin{question}{torque-Q20}
    For the wheel-and-axle system shown,
    \begin{center}
    \begin{tikzpicture}
        %% NOTE:
    \end{tikzpicture}
    \end{center}
        which of the following expresses the condition required for the system to be in static equilibrium?
    \begin{multicols}{2}
    \begin{choices}
        \wrongchoice{$m_1 = m_2$}
      \correctchoice{$a m_2 = b m_2$}
        \wrongchoice{$a m_2 = b m_1$}
        \wrongchoice{$a^2 m_1 = b^2 m_2$}
        \wrongchoice{$b^2 m_1 = a^2 m_2$}
    \end{choices}
    \end{multicols}
\end{question}
}

\element{AP}{
\begin{question}{torque-Q21}
    A meterstick of negligible mass is placed on a fulcrum at the \SI{0.60}{\meter} mark,
        with a \SI{2.0}{\kilo\gram} mass hung at the \SI{0}{\meter} mark and a \SI{1.0}{\kilo\gram} mass hung at the \SI{1.0}{\meter} mark.
    The meterstick is released from rest in a horizontal position.
    Immediately after release,
        the magnitude of the net torque on the meterstick about the fulcrum is most nearly:
    \begin{multicols}{2}
    \begin{choices}
        \wrongchoice{\SI{2.0}{\newton\per\meter}} 
      \correctchoice{\SI{8.0}{\newton\per\meter}} 
        \wrongchoice{\SI{10}{\newton\per\meter}} 
        \wrongchoice{\SI{14}{\newton\per\meter}} 
        \wrongchoice{\SI{16}{\newton\per\meter}}
    \end{choices}
    \end{multicols}
\end{question}
}


\endinput

