

%% this section contains XX problems

%% AP Physics B practice workbook
%%--------------------------------------------------

%% Section B: Work
%%--------------------------------------------------
\element{AP}{
\begin{question}{work-Q01}
    A mass $m$ attached to a horizontal massless spring with spring constant $k$,
        is set into simple harmonic motion. 
    Its maximum displacement from its equilibrium position is $A$.
    \begin{center}
    \begin{tikzpicture}
        %% NOTE:
    \end{tikzpicture}
    \end{center}
    What is the masses speed as it passes through its equilibrium position?
    \begin{multicols}{2}
    \begin{choices}
        \wrongchoice{$0$}
      \correctchoice{$A \sqrt{\dfrac{k}{m}}$}
        \wrongchoice{$A \sqrt{\dfrac{m}{k}}$}
        \wrongchoice{$\dfrac{1}{A} \sqrt{\dfrac{k}{m}}$}
        \wrongchoice{$\dfrac{1}{A} \sqrt{\dfrac{m}{k}}$}
    \end{choices}
    \end{multicols}
\end{question}
}

\element{AP}{
\begin{question}{work-Q02}
    A force $F$ at an angle $\theta$ above the horizontal is used to pull a heavy suitcase of weight mg a distance d along a level floor at constant velocity. 
    The coefficient of friction between the floor and the suitcase is $\mu$. 
    The work done by the frictional force is:
    \begin{multicols}{2}
    \begin{choices}
      \correctchoice{$- Fd\cos\theta$}
        \wrongchoice{$mgh - Fd\cos\theta$}
        \wrongchoice{$- \mu Fd\cos\theta$}
        \wrongchoice{$- \mu mgd$}
        \wrongchoice{$- \mu mgd \cos\theta$}
    \end{choices}
    \end{multicols}
\end{question}
}

\element{AP}{
\begin{question}{work-Q03}
    If the unit for force is $F$, the unit for velocity $V$,
        and the unit for time $T$, then the unit for energy is:
    \begin{multicols}{2}
    \begin{choices}
      \correctchoice{$\dfrac{FV^2}{T^2}$}
        \wrongchoice{$FVT$}
        \wrongchoice{$\dfrac{F}{T}$}
        \wrongchoice{$\dfrac{FV}{T}$}
        \wrongchoice{$\dfrac{F}{T^2}$}
    \end{choices}
    \end{multicols}
\end{question}
}

\element{AP}{
\begin{question}{work-Q04}
    A force of \SI{10}{\newton} stretches a spring that has a spring constant of \SI{20}{\newton\per\meter}. 
    The potential energy stored in the spring is:
    \begin{multicols}{2}
    \begin{choices}
      \correctchoice{\SI{2.5}{\joule}}
        \wrongchoice{\SI{5.0}{\joule}}
        \wrongchoice{\SI{10}{\joule}}
        \wrongchoice{\SI{40}{\joule}}
        \wrongchoice{\SI{200}{\joule}}
    \end{choices}
    \end{multicols}
\end{question}
}

\element{AP}{
\begin{question}{work-Q05}
    A \SI{2}{\kilo\gram} ball is attached to a \SI{0.80}{\meter} string and whirled in a horizontal circle at a constant speed of \SI{6}{\meter\per\second}. 
    The work done on the ball during each revolution is:
    \begin{multicols}{2}
    \begin{choices}
        \wrongchoice{\SI{450}{\joule}}
        \wrongchoice{\SI{90}{\joule}}
        \wrongchoice{\SI{72}{\joule}}
        \wrongchoice{\SI{16}{\joule}}
      \correctchoice{zero}
    \end{choices}
    \end{multicols}
\end{question}
}

\element{AP}{
\begin{question}{work-Q06}
    A pendulum bob of mass $m$ on a cord of length $L$ is pulled sideways until the cord makes an angle $\theta$ with the vertical as shown in the figure below.
    \begin{center}
    \begin{tikzpicture}
        %% NOTE:
    \end{tikzpicture}
    \end{center}
    The change in potential energy of the bob during the displacement is:
    \begin{multicols}{2}
    \begin{choices}
      \correctchoice{$mgL (1-\cos\theta)$}
        \wrongchoice{$mgL (1-\sin\theta)$}
        \wrongchoice{$mgL \sin\theta$}
        \wrongchoice{$mgL \cos\theta$}
        \wrongchoice{$2mgL (1-\sin\theta)$}
    \end{choices}
    \end{multicols}
\end{question}
}

\element{AP}{
\begin{question}{work-Q07}
    A force $F$ directed at an angle $\theta$ above the horizontal is used to pull a crate a distance $D$ across a level floor.
    The work done by the force $F$ is
    \begin{multicols}{2}
    \begin{choices}
        \wrongchoice{$FD$}
      \correctchoice{$FD\cos\theta$}
        \wrongchoice{$FD\sin\theta$}
        \wrongchoice{$mg\sin\theta$}
        \wrongchoice{$mgD\cos\theta$}
    \end{choices}
    \end{multicols}
\end{question}
}

\element{AP}{
\begin{question}{work-Q08}
    A compressed spring has \SI{16}{\joule} of potential energy. 
    What is the maximum speed it can impart to a \SI{2}{\kilo\gram} object?
    \begin{multicols}{2}
    \begin{choices}
        \wrongchoice{\SI{2.8}{\meter\per\second}}
      \correctchoice{\SI{4.0}{\meter\per\second}}
        \wrongchoice{\SI{5.6}{\meter\per\second}}
        \wrongchoice{\SI{8.0}{\meter\per\second}}
        \wrongchoice{\SI{16}{\meter\per\second}}
    \end{choices}
    \end{multicols}
\end{question}
}

\element{AP}{
\begin{question}{work-Q09}
    A softball player catches a ball of mass $m$,
        which is moving towards her with horizontal speed $V$. 
    While bringing the ball to rest,
        her hand moved back a distance $d$. 
    Assuming constant deceleration,
        the horizontal force exerted on the ball by the hand is
    \begin{multicols}{2}
    \begin{choices}
      \correctchoice{$\dfrac{mV^2}{2d}$}
        \wrongchoice{$\dfrac{mV^2}{d}$}
        \wrongchoice{$mVd$}
        \wrongchoice{$\dfrac{2mV}{d}$}
        \wrongchoice{$\dfrac{mV}{d}$}
    \end{choices}
    \end{multicols}
\end{question}
}

\element{AP}{
\begin{question}{work-Q10}
    A \SI{3}{\kilo\gram} block with initial speed \SI{4}{\meter\per\second} slides across a rough horizontal floor before coming to rest. 
    The frictional force acting on the block is \SI{3}{\newton}. 
    How far does the block slide before coming to rest?
    \begin{multicols}{2}
    \begin{choices}
        \wrongchoice{\SI{1.0}{\meter}}
        \wrongchoice{\SI{2.0}{\meter}}
        \wrongchoice{\SI{4.0}{\meter}}
      \correctchoice{\SI{8.0}{\meter}}
        \wrongchoice{\SI{16}{\meter}}
    \end{choices}
    \end{multicols}
\end{question}
}

\element{AP}{
\begin{question}{work-Q11}
    A construction laborer holds a \SI{20}{\kilo\gram} sheet of wallboard \SI{3}{\meter} above the floor for \SI{4}{\second}. 
    During these \SI{4}{\second} how much power was expended on the wallboard?
    \begin{multicols}{2}
    \begin{choices}
        \wrongchoice{\SI{2400}{\watt}}
        \wrongchoice{\SI{340}{\watt}}
        \wrongchoice{\SI{27}{\watt}}
        \wrongchoice{\SI{15}{\watt}}
        %\wrongchoice{zero} ??
      \correctchoice{none of the provided}
    \end{choices}
    \end{multicols}
\end{question}
}

\element{AP}{
\begin{question}{work-Q12}
    A pendulum is pulled to one side and released. 
    It swings freely to the opposite side and stops. 
    Which of the following might best represent graphs of kinetic energy ($E_k$),
        potential energy ($E_p$) and total mechanical energy ($E_T$)
    \begin{multicols}{2}
    \begin{choices}
        %% NOTE: ans is C
        \wrongchoice{
            \begin{tikzpicture}
            \end{tikzpicture}
        }
    \end{choices}
    \end{multicols}
\end{question}
}

\element{AP}{
\begin{question}{work-Q13}
    %% Problems 13 and 14 refer to the following situation:
    A car of mass $m$ slides across a patch of ice at a speed $v$ with its brakes locked. 
    It the hits dry pavement and skids to a stop in a distance $d$. 
    The coefficient of kinetic friction between the tires and the dry road is $\mu$.

    If the car has a mass of $2m$,
        it would have skidded a distance of:
    \begin{multicols}{2}
    \begin{choices}
        \wrongchoice{$\frac{1}{2} d$}
      \correctchoice{$d$}
        \wrongchoice{$\sqrt{2} d$}
        \wrongchoice{$2 d$}
        \wrongchoice{$4 d$}
    \end{choices}
    \end{multicols}
\end{question}
}

\element{AP}{
\begin{question}{work-Q14}
    %% Problems 13 and 14 refer to the following situation:
    A car of mass $m$ slides across a patch of ice at a speed $v$ with its brakes locked. 
    It the hits dry pavement and skids to a stop in a distance $d$. 
    The coefficient of kinetic friction between the tires and the dry road is $\mu$.

    If the car has a speed of $2v$,
        it would have skidded a distance of:
    \begin{multicols}{2}
    \begin{choices}
        \wrongchoice{$\frac{1}{2} d$}
        \wrongchoice{$d$}
        \wrongchoice{$\sqrt{2} d$}
        \wrongchoice{$2 d$}
      \correctchoice{$4 d$}
    \end{choices}
    \end{multicols}
\end{question}
}

\element{AP}{
\begin{question}{work-Q15}
    A ball is thrown vertically upwards with a velocity $v$ and an initial kinetic energy $E_k$. 
    When half way to the top of its flight,
        it has a velocity and kinetic energy respectively of:
    \begin{multicols}{2}
    \begin{choices}
        \wrongchoice{$\dfrac{v}{2}$, $\dfrac{E_k}{2}$}
      \correctchoice{$\dfrac{v}{\sqrt{2}}$, $\dfrac{E_k}{2}$}
        \wrongchoice{$\dfrac{v}{4}$, $\dfrac{E_k}{2}$}
        \wrongchoice{$\dfrac{v}{2}$, $\dfrac{E_k}{\sqrt{2}}$}
        \wrongchoice{$\dfrac{v}{\sqrt{2}}$, $\dfrac{E_k}{\sqrt{2}}$}
    \end{choices}
    \end{multicols}
\end{question}
}

\element{AP}{
\begin{question}{work-Q16}
    A football is kicked off the ground a distance of \SI{50}{\yard} downfield. 
    Neglecting air resistance,
        which of the following statements would be \emph{incorrect} when the football reaches the highest point?
    \begin{choices}
      \correctchoice{all of the balls original kinetic energy has been changed into potential energy}
        \wrongchoice{the balls horizontal velocity is the same as when it left the kickers foot}
        \wrongchoice{the ball will have been in the air one-half of its total flight time}
        \wrongchoice{the ball has an acceleration of g}
        \wrongchoice{the vertical component of the velocity is equal to zero}
    \end{choices}
\end{question}
}

\element{AP}{
\begin{question}{work-Q17}
    A mass $m$ is attached to a spring with a spring constant $k$. 
    If the mass is set into motion by a displacement $d$ from its equilibrium position,
        what would be the speed, $v$,
        of the mass when it returns to equilibrium position?
    \begin{multicols}{2}
    \begin{choices}
        \wrongchoice{$v = \sqrt{\dfrac{kd}{m}}$}
        \wrongchoice{$v^2 = \dfrac{kd}{m}$}
        \wrongchoice{$v = \dfrac{kd}{mg}$}
        \wrongchoice{$v^2 = \dfrac{mgd}{k}$}
      \correctchoice{$v = d \sqrt{\dfrac{k}{m}}$}
    \end{choices}
    \end{multicols}
\end{question}
}

\element{AP}{
\begin{question}{work-Q18}
    If $M$ represents units of mass, $L$ represents units of length,
        and $T$ represents units of time, the dimensions of power would be:
    \begin{multicols}{2}
    \begin{choices}
        \wrongchoice{$\dfrac{ML}{T^2}$}
        \wrongchoice{$\dfrac{ML^2}{T^2}$}
      \correctchoice{$\dfrac{ML^2}{T^3}$}
        \wrongchoice{$\dfrac{ML}{T}$}
        \wrongchoice{$\dfrac{ML^2}{T}$}
    \end{choices}
    \end{multicols}
\end{question}
}

\element{AP}{
\begin{question}{work-Q19}
    An automobile engine delivers \SI{24 000}{\watt} of power to a car's driving wheels. 
    If the car maintains a constant speed of \SI{30}{\meter\per\second},
        what is the magnitude of the retarding force acting on the car?
    \begin{multicols}{2}
    \begin{choices}
      \correctchoice{\SI{800}{\newton}}
        \wrongchoice{\SI{960}{\newton}}
        \wrongchoice{\SI{1950}{\newton}}
        \wrongchoice{\SI{720 000}{\newton}}
        \wrongchoice{\SI{1 560 000}{\newton}}
    \end{choices}
    \end{multicols}
\end{question}
}

\element{AP}{
\begin{question}{work-Q20}
    A fan blows the air and gives it kinetic energy. 
    An hour after the fan has been turned off,
        what has happened to the kinetic energy of the air?
    \begin{choices}
        \wrongchoice{it disappears}
        \wrongchoice{it turns into potential energy}
      \correctchoice{it turns into thermal energy}
        \wrongchoice{it turns into sound energy}
        \wrongchoice{it turns into electrical energy}
    \end{choices}
\end{question}
}

\element{AP}{
\begin{question}{work-Q21}
    A box of old textbooks is on the middle shelf in the bookroom \SI{1.3}{\meter} from the floor. 
    If the janitor relocates the box to a shelf that is \SI{2.6}{\meter} from the floor,
        how much work does he do on the box? 
    The box has a mass of \SI{10}{\kilo\gram}.
    \begin{multicols}{2}
    \begin{choices}
        \wrongchoice{\SI{13}{\joule}}
        \wrongchoice{\SI{26}{\joule}}
        \wrongchoice{\SI{52}{\joule}}
      \correctchoice{\SI{130}{\joule}}
        \wrongchoice{\SI{260}{\joule}}
    \end{choices}
    \end{multicols}
\end{question}
}

\element{AP}{
\begin{question}{work-Q22}
    A mass, $M$, is at rest on a frictionless surface,
        connected to an ideal horizontal spring that is unstretched. 
    A person extends the spring \SI{30}{\centi\meter} from equilibrium and holds it by applying a \SI{10}{\newton} force. 
    The spring is brought back to equilibrium and the mass connected to it is now doubled to $2M$. 
    If the spring is extended back \SI{30}{\centi\meter} from equilibrium,
        what is the necessary force applied by the person to hold the mass stationary there?
    \begin{multicols}{2}
    \begin{choices}
        \wrongchoice{\SI{20}{\newton}}
        \wrongchoice{\SI{14.1}{\newton}}
      \correctchoice{\SI{10}{\newton}}
        \wrongchoice{\SI{7.07}{\newton}}
        \wrongchoice{\SI{5}{\newton}}
    \end{choices}
    \end{multicols}
\end{question}
}

\element{AP}{
\begin{question}{work-Q23}
    A deliveryman moves 10 cartons from the sidewalk,
        along a \SI{10}{\meter} ramp to a loading dock,
        which is \SI{1.5}{\meter} above the sidewalk. 
    If each carton has a mass of \SI{25}{\kilo\gram},
        what is the total work done by the deliveryman on the cartons to move them to the loading dock?
    \begin{multicols}{2}
    \begin{choices}
        \wrongchoice{\SI{2500}{\joule}}
      \correctchoice{\SI{3750}{\joule}}
        \wrongchoice{\SI{10000}{\joule}}
        \wrongchoice{\SI{25000}{\joule}}
        \wrongchoice{\SI{37500}{\joule}}
    \end{choices}
    \end{multicols}
\end{question}
}

\element{AP}{
\begin{question}{work-Q24}
    A rock is dropped from the top of a tall tower. 
    Half a second later another rock, twice as massive as the first, is dropped. 
    Ignoring air resistance,
    \begin{choices}
      \correctchoice{the distance between the rocks increases while both are falling.}
        \wrongchoice{the acceleration is greater for the more massive rock.}
        \wrongchoice{the speed of both rocks is constant while they fall.}
        \wrongchoice{they strike the ground more than half a second apart.}
        \wrongchoice{they strike the ground with the same kinetic energy.}
    \end{choices}
\end{question}
}

\element{AP}{
\begin{question}{work-Q25}
    A \SI{60.0}{\kilo\gram} ball of clay is tossed vertically in the air with an initial speed of \SI{4.60}{\meter\per\second}. 
    Ignoring air resistance,
        what is the change in its potential energy when it reaches its highest point?
    \begin{multicols}{2}
    \begin{choices}
        \wrongchoice{\SI{0}{\joule}}
        \wrongchoice{\SI{45}{\joule}}
        \wrongchoice{\SI{280}{\joule}}
      \correctchoice{\SI{635}{\joule}}
        \wrongchoice{\SI{2700}{\joule}}
    \end{choices}
    \end{multicols}
\end{question}
}

\element{AP}{
\begin{question}{work-Q26}
    Which of the following is true for a system consisting of a mass oscillating on the end of an ideal spring?
    \begin{choices}
        \wrongchoice{The kinetic and potential energies are equal to each other at all times.}
        \wrongchoice{The kinetic and potential energies are both constant.}
        \wrongchoice{The maximum potential energy is achieved when the mass passes through its equilibrium position.}
      \correctchoice{The maximum kinetic energy and maximum potential energy are equal, but occur at different times.}
        \wrongchoice{The maximum kinetic energy occurs at maximum displacement of the mass from its equilibrium position}
    \end{choices}
\end{question}
}

\element{AP}{
\begin{question}{work-Q27}
    From the top of a high cliff,
        a ball is thrown horizontally with initial speed $v_0$. 
    Which of the following graphs best represents the ball's kinetic energy $K$ as a function of time $t$?
    \begin{multicols}{2}
    \begin{choices}
        %% NOTE: ans is E
        \wrongchoice{
            \begin{tikzpicture}
            \end{tikzpicture}
        }
    \end{choices}
    \end{multicols}
\end{question}
}

\element{AP}{
\begin{question}{work-Q28}
    A person pushes a box across a horizontal surface at a constant speed of \SI{0.5}{\meter\per\second}. 
    The box has a mass of \SI{40}{\kilo\gram},
        and the coefficient of sliding friction is \num{0.25}. 
    The power supplied to the box by the person is:
    \begin{multicols}{2}
    \begin{choices}
        \wrongchoice{\SI{0.2}{\watt}}
        \wrongchoice{\SI{5}{\watt}}
      \correctchoice{\SI{50}{\watt}}
        \wrongchoice{\SI{100}{\watt}}
        \wrongchoice{\SI{200}{\watt}}
    \end{choices}
    \end{multicols}
\end{question}
}

\element{AP}{
\begin{question}{work-Q29}
    A horizontal force $F$ is used to pull a \SI{5}{\kilo\gram} block across a floor at a constant speed of \SI{3}{\meter\per\second}.
    The frictional force between the block and the floor is \SI{10}{\newton}. 
    The work done by the force $F$ in \SI{1}{\minute} is most nearly
    \begin{multicols}{2}
    \begin{choices}
        \wrongchoice{\SI{0}{\joule}}
        \wrongchoice{\SI{30}{\joule}}
        \wrongchoice{\SI{600}{\joule}}
        \wrongchoice{\SI{1 350}{\joule}}
      \correctchoice{\SI{1 800}{\joule}}
    \end{choices}
    \end{multicols}
\end{question}
}

\newcommand{\workQThirtyA}{
\begin{tikzpicture}
    %% NOTE:
\end{tikzpicture}
}

\newcommand{\workQThirtyB}{
\begin{tikzpicture}
    %% NOTE:
\end{tikzpicture}
}

\newcommand{\workQThirtyC}{
\begin{tikzpicture}
    %% NOTE:
\end{tikzpicture}
}

\newcommand{\workQThirtyD}{
\begin{tikzpicture}
    %% NOTE:
\end{tikzpicture}
}

\newcommand{\workQThirtyE}{
\begin{tikzpicture}
    %% NOTE:
\end{tikzpicture}
}

\element{AP}{
\begin{question}{work-Q30}
    %Questions 30-31:
    A block oscillates without friction on the end of a spring as shown. 
    The minimum and maximum lengths of the spring as it oscillates are,
        respectively, $x_{min}$ and $x_{max}$. 
    The graphs below can represent quantities associated with the oscillation as functions of the length x of the spring.
    \begin{center}
    \begin{tikzpicture}
        %% NOTE:
    \end{tikzpicture}
    \end{center}
    Which graph can represent the total mechanical energy of the block-spring system as a function of x ?
    \begin{multicols}{2}
    \begin{choices}
        \wrongchoice{\workQThirtyA}
        \wrongchoice{\workQThirtyB}
        \wrongchoice{\workQThirtyC}
        \wrongchoice{\workQThirtyD}
      \correctchoice{\workQThirtyE}
    \end{choices}
    \end{multicols}
\end{question}
}

\element{AP}{
\begin{question}{work-Q31}
    %Questions 30-31:
    A block oscillates without friction on the end of a spring as shown. 
    The minimum and maximum lengths of the spring as it oscillates are,
        respectively, $x_{min}$ and $x_{max}$. 
    The graphs below can represent quantities associated with the oscillation as functions of the length x of the spring.
    \begin{center}
    \begin{tikzpicture}
        %% NOTE:
    \end{tikzpicture}
    \end{center}
    Which graph can represent the kinetic energy of the block as a function of $x$?
    \begin{multicols}{2}
    \begin{choices}
        \wrongchoice{\workQThirtyA}
        \wrongchoice{\workQThirtyB}
        \wrongchoice{\workQThirtyC}
      \correctchoice{\workQThirtyD}
        \wrongchoice{\workQThirtyE}
    \end{choices}
    \end{multicols}
\end{question}
}

\newcommand{\workQThrityTwo}{
\begin{tikzpicture}
    %% NOTE:
\end{tikzpicture}
}

\element{AP}{
\begin{question}{work-Q32}
    %% Questions 32--33
    A ball swings freely back and forth in an arc from point I to point IV, as shown. 
    Point II is the lowest point in the path,
        III is located 0.5 meter above II, and IV is I meter above II.
    Air resistance is negligible.
    \begin{center}
        \workQThrityTwo
    \end{center}
    If the potential energy is zero at point II,
        where will the kinetic and potential energies of the ball be equal?
    \begin{choices}
        \wrongchoice{At point II}
        \wrongchoice{At some point between II and III}
      \correctchoice{At point III}
        \wrongchoice{At some point between III and IV}
        \wrongchoice{At point IV}
    \end{choices}
\end{question}
}

\element{AP}{
\begin{question}{work-Q33}
    %% Questions 32--33
    A ball swings freely back and forth in an arc from point I to point IV, as shown. 
    Point II is the lowest point in the path,
        III is located 0.5 meter above II, and IV is I meter above II.
    Air resistance is negligible.
    \begin{center}
        \workQThrityTwo
    \end{center}
    The speed of the ball at point II is most nearly:
    \begin{multicols}{2}
    \begin{choices}
        \wrongchoice{\SI{3.0}{\meter\per\second}}
      \correctchoice{\SI{4.5}{\meter\per\second}}
        \wrongchoice{\SI{9.8}{\meter\per\second}}
        \wrongchoice{\SI{14}{\meter\per\second}}
        \wrongchoice{\SI{20}{\meter\per\second}}
    \end{choices}
    \end{multicols}
\end{question}
}

\element{AP}{
\begin{question}{work-Q34}
    An ideal spring obeys Hooke's law, $F = -kx$. 
    A mass of \SI{0.50}{\kilo\gram} hung vertically from this spring stretches the spring \SI{0.075}{\meter}. 
    The value of the force constant for the spring is most nearly:
    \begin{multicols}{2}
    \begin{choices}
        \wrongchoice{\SI{0.33}{\newton\per\meter}}
        \wrongchoice{\SI{0.66}{\newton\per\meter}}
        \wrongchoice{\SI{6.6}{\newton\per\meter}}
        \wrongchoice{\SI{33}{\newton\per\meter}}
      \correctchoice{\SI{66}{\newton\per\meter}}
    \end{choices}
    \end{multicols}
\end{question}
}

\element{AP}{
\begin{question}{work-Q35}
    The figure shows a rough semicircular track whose ends are at a vertical height $h$. 
    \begin{center}
    \begin{tikzpicture}
        %% NOTE:
    \end{tikzpicture}
    \end{center}
    A block placed at point $P$ at one end of the track is released from rest and slides past the bottom of the track. 
    Which of the following is true of the height to which the block rises on the other side of the track?
    \begin{choices}
        \wrongchoice{It is equal to $\frac{h}{2\pi}$}
        \wrongchoice{It is equal to $\frac{h}{4}$}
        \wrongchoice{It is equal to $\frac{h}{2}$}
        \wrongchoice{It is equal to $h$}
      \correctchoice{It is between zero and $h$; the exact height depends on how much energy is lost to friction.}
    \end{choices}
\end{question}
}

\element{AP}{
\begin{question}{work-Q36}
    A weight lifter lifts a mass $m$ at constant speed to a height h in time $t$. 
    What is the average power output of the weight lifter?
    \begin{multicols}{2}
    \begin{choices}
        \wrongchoice{$mg$}
        \wrongchoice{$mh$}
        \wrongchoice{$mgh$}
        \wrongchoice{$mght$}
      \correctchoice{$\dfrac{mgh}{t}$}
    \end{choices}
    \end{multicols}
\end{question}
}

\element{AP}{
\begin{question}{work-Q37}
    A block of mass m slides on a horizontal frictionless table with an initial speed $v_0$. 
    It then compresses a spring of force constant $k$ and is brought to rest. 
    How much is the spring compressed from its natural length?
    \begin{multicols}{2}
    \begin{choices}
        \wrongchoice{$\dfrac{v_0^2}{2g}$}
        \wrongchoice{$v_0 \dfrac{mg}{k}$}
        \wrongchoice{$v_0 \dfrac{m}{k}$}
      \correctchoice{$v_0 \sqrt{\dfrac{m}{k}}$}
        \wrongchoice{$v_0 \sqrt{\dfrac{k}{m}}$}
    \end{choices}
    \end{multicols}
\end{question}
}

\newcommand{\workQThirtyEight}{
\begin{tikzpicture}
\end{tikzpicture}
}

\element{AP}{
\begin{question}{work-Q38}
    %Questions 38-40
    A plane \SI{5}{\meter} in length is inclined at an angle of \ang{37}, as shown. 
    A block of weight \SI{20}{\newton} is placed at the top of the plane and allowed to slide down.
    \begin{center}
        \workQThirtyEight
    \end{center}
    The mass of the block is most nearly:
    \begin{multicols}{2}
    \begin{choices}
        \wrongchoice{\SI{1.0}{\kilo\gram}}
        \wrongchoice{\SI{1.2}{\kilo\gram}}
        \wrongchoice{\SI{1.6}{\kilo\gram}}
      \correctchoice{\SI{2.0}{\kilo\gram}}
        \wrongchoice{\SI{2.5}{\kilo\gram}}
    \end{choices}
    \end{multicols}
\end{question}
}

\element{AP}{
\begin{question}{work-Q39}
    %Questions 38-40
    A plane \SI{5}{\meter} in length is inclined at an angle of \ang{37}, as shown. 
    A block of weight \SI{20}{\newton} is placed at the top of the plane and allowed to slide down.
    \begin{center}
        \workQThirtyEight
    \end{center}
    The magnitude of the normal force exerted on the block by the plane is most nearly:
    \begin{multicols}{2}
    \begin{choices}
        \wrongchoice{\SI{10}{\newton}}
        \wrongchoice{\SI{12}{\newton}}
      \correctchoice{\SI{16}{\newton}}
        \wrongchoice{\SI{20}{\newton}}
        \wrongchoice{\SI{33}{\newton}}
    \end{choices}
    \end{multicols}
\end{question}
}

\element{AP}{
\begin{question}{work-Q40}
    %Questions 38-40
    A plane \SI{5}{\meter} in length is inclined at an angle of \ang{37}, as shown. 
    A block of weight \SI{20}{\newton} is placed at the top of the plane and allowed to slide down.
    \begin{center}
        \workQThirtyEight
    \end{center}
    The work done on the block by the gravitational force during the \SI{5}{\meter} slide down the plane is most nearly:
    \begin{multicols}{2}
    \begin{choices}
        \wrongchoice{\SI{20}{\joule}}
      \correctchoice{\SI{60}{\joule}}
        \wrongchoice{\SI{80}{\joule}}
        \wrongchoice{\SI{100}{\joule}}
        \wrongchoice{\SI{130}{\joule}}
    \end{choices}
    \end{multicols}
\end{question}
}

\element{AP}{
\begin{question}{work-Q41}
    A student weighing \SI{700}{\newton} climbs at constant speed to the top of an \SI{8}{\meter} vertical rope in \SI{10}{\second}. 
    The average power expended by the student to overcome gravity is most nearly:
    \begin{multicols}{2}
    \begin{choices}
        \wrongchoice{\SI{1.1}{\watt}}
        \wrongchoice{\SI{87.5}{\watt}}
      \correctchoice{\SI{560}{\watt}}
        \wrongchoice{\SI{875}{\watt}}
        \wrongchoice{\SI{5,600}{\watt}}
    \end{choices}
    \end{multicols}
\end{question}
}

\element{AP}{
\begin{question}{work-Q42}
    The graph shown represents the potential energy $U$ as a function of displacement $x$ for an object on the end of a spring moving back and forth with amplitude $x_0$. 
    \begin{center}
    \begin{tikzpicture}
        %% NOTE:
    \end{tikzpicture}
    \end{center}
    Which of the following graphs represents the kinetic energy $K$ of the object as a function of displacement $x$?
    \begin{multicols}{2}
    \begin{choices}
        %% NOTE: ans is D
        \wrongchoice{
            \begin{tikzpicture}
            \end{tikzpicture}
        }
    \end{choices}
    \end{multicols}
\end{question}
}

\element{AP}{
\begin{question}{work-Q43}
    A child pushes horizontally on a box of mass $m$ which moves with constant speed $v$ across a horizontal floor.
    The coefficient of friction between the box and the floor is $\mu$. 
    At what rate does the child do work on the box?
    \begin{multicols}{2}
    \begin{choices}
      \correctchoice{$\mu mgv$}
        \wrongchoice{$mgv$}
        \wrongchoice{$\dfrac{\mu mg}{v}$}
        \wrongchoice{$\dfrac{\mu mg}{v}$}
        \wrongchoice{$\mu mv^2$}
    \end{choices}
    \end{multicols}
\end{question}
}

\element{AP}{
\begin{question}{work-Q44}
    A block of mass \SI{3.0}{\kilo\gram} is hung from a spring,
        causing it to stretch \SI{12}{\centi\meter} at equilibrium, as shown. 
    The \SI{3.0}{\kilo\gram} block is then replaced by a \SI{4.0}{\kilo\gram} block,
        and the new block is released from the position shown,
        at which the spring is unstretched. 
    \begin{center}
    \begin{tikzpicture}
        %% NOTE: tikz springs
    \end{tikzpicture}
    \end{center}
    How far will the \SI{4.0}{\kilo\gram} block fall before its direction is reversed?
    \begin{multicols}{2}
    \begin{choices}
        \wrongchoice{\SI{9}{\centi\meter}}
        \wrongchoice{\SI{18}{\centi\meter}}
        \wrongchoice{\SI{24}{\centi\meter}}
      \correctchoice{\SI{32}{\centi\meter}}
        \wrongchoice{\SI{48}{\centi\meter}}
    \end{choices}
    \end{multicols}
\end{question}
}

\element{AP}{
\begin{question}{work-Q45}
    What is the kinetic energy of a satellite of mass m that orbits the Earth,
        of mass $M$, in a circular orbit of radius $R$?
    \begin{multicols}{2}
    \begin{choices}
        \wrongchoice{Zero}
      \correctchoice{$\dfrac{GMm}{2R}$}
        \wrongchoice{$\dfrac{GMm}{4R}$}
        \wrongchoice{$\dfrac{GMm}{2R^2}$}
        \wrongchoice{$\dfrac{GMm}{R^2}$}
    \end{choices}
    \end{multicols}
\end{question}
}

\element{AP}{
\begin{question}{work-Q46}
    %% Questions 46-47
    A rock of mass $m$ is thrown horizontally off a building from a height $h$, as shown above. 
    The speed of the rock as it leaves the thrower's hand at the edge of the building is $v_0$.
    \begin{center}
    \begin{tikzpicture}
        %% NOTE: steal from kinematics
    \end{tikzpicture}
    \end{center}
    How much time does it take the rock to travel from the edge of the building to the ground?
    \begin{multicols}{2}
    \begin{choices}
        \wrongchoice{$\sqrt{hv_0}$}
        \wrongchoice{$\dfrac{h}{v_0}$}
        \wrongchoice{$\dfrac{h v_0}{g}$}
        \wrongchoice{$\dfrac{2h}{g}$}
      \correctchoice{$\sqrt{\dfrac{2h}{g}}$}
    \end{choices}
    \end{multicols}
\end{question}
}

\element{AP}{
\begin{question}{work-Q47}
    %% Questions 46-47
    A rock of mass $m$ is thrown horizontally off a building from a height $h$, as shown above. 
    The speed of the rock as it leaves the thrower's hand at the edge of the building is $v_0$.
    \begin{center}
    \begin{tikzpicture}
        %% NOTE: steal from kinematics
    \end{tikzpicture}
    \end{center}
    What is the kinetic energy of the rock just before it hits the ground?
    \begin{multicols}{2}
    \begin{choices}
        \wrongchoice{$mgh$}
        \wrongchoice{$\frac{1}{2} mv_0^2$}
        \wrongchoice{$\frac{1}{2} mv_0^2 - mgh$}
      \correctchoice{$\frac{1}{2} mv_0^2 + mgh$}
        \wrongchoice{$mgh - \frac{1}{2} mv_0^2$}
    \end{choices}
    \end{multicols}
\end{question}
}

\element{AP}{
\begin{question}{work-Q48}
    A sphere of mass $m_1$,
        which is attached to a spring,
        is displaced downward from its equilibrium position as shown above left and released from rest. 
    A sphere of mass $m_2$,
        which is suspended from a string of length $L$,
        is displaced to the right as shown above right and released from rest so that it swings as a simple pendulum with small amplitude. 
    Assume that both spheres undergo simple harmonic motion.
    \begin{center}
    \begin{tikzpicture}
        %% NOTE: pendulum and spring
    \end{tikzpicture}
    \end{center}
    Which of the following is true for both spheres?
    \begin{choices}
      \correctchoice{The maximum kinetic energy is attained as the sphere passes through its equilibrium position.}
        \wrongchoice{The maximum kinetic energy is attained as the sphere reaches its point of release.}
        \wrongchoice{The minimum gravitational potential energy is attained as the sphere passes through its equilibrium position.}
        \wrongchoice{The maximum gravitational potential energy is attained when the sphere reaches its point of release.}
        \wrongchoice{The maximum total energy is attained only as the sphere passes through its equilibrium position.}
    \end{choices}
\end{question}
}

\element{AP}{
\begin{question}{work-Q49}
    %% Questions 49-50
    An object of mass $m$ is initially at rest and free to move without friction in any direction in the $xy$-plane. 
    A constant net force of magnitude $F$ directed in the $+x$ direction acts on the object for \SI{1}{\second}. 
    Immediately thereafter a constant net force of the same magnitude $F$ directed in the $+y$ direction acts on the object for \SI{1}{\second}. 
    After this, no forces act on the object.

    Which of the following vectors could represent the velocity of the object at the end of \SI{3}{\second},
        assuming the scales on the $x$ and $y$ axes are equal?
    \begin{multicols}{2}
    \begin{choices}
        %% NOTE: ans is C
        \wrongchoice{
            \begin{tikzpicture}
            \end{tikzpicture}
        }
    \end{choices}
    \end{multicols}
\end{question}
}

\element{AP}{
\begin{question}{work-Q50}
    %% Questions 49-50
    An object of mass $m$ is initially at rest and free to move without friction in any direction in the $xy$-plane. 
    A constant net force of magnitude $F$ directed in the $+x$ direction acts on the object for \SI{1}{\second}. 
    Immediately thereafter a constant net force of the same magnitude $F$ directed in the $+y$ direction acts on the object for \SI{1}{\second}. 
    After this, no forces act on the object.

    Which of the following graphs best represents the kinetic energy $K$ of the object as a function of time?
    \begin{multicols}{2}
    \begin{choices}
        %% NOTE: ans is B
        \wrongchoice{
            \begin{tikzpicture}
            \end{tikzpicture}
        }
    \end{choices}
    \end{multicols}
\end{question}
}

\element{AP}{
\begin{question}{work-Q51}
    A constant force of \SI{900}{\newton} pushes a \SI{100}{\kilo\gram} mass up the inclined plane shown at a uniform speed of \SI{4}{\meter\per\second}.
    \begin{center}
    \begin{tikzpicture}
        %% NOTE:
    \end{tikzpicture}
    \end{center}
    The power developed by the \SI{900}{\newton} force is most nearly:
    \begin{multicols}{2}
    \begin{choices}
        \wrongchoice{\SI{400}{\watt}}
        \wrongchoice{\SI{800}{\watt}}
        \wrongchoice{\SI{900}{\watt}}
        \wrongchoice{\SI{1000}{\watt}}
      \correctchoice{\SI{3600}{\watt}}
    \end{choices}
    \end{multicols}
\end{question}
}

\element{AP}{
\begin{question}{work-Q52}
    An object of mass $m$ is lifted at constant velocity a vertical distance $H$ in time $T$. 
    The power supplied by the lifting force is:
    \begin{multicols}{2}
    \begin{choices}
        \wrongchoice{$mgHT$}
      \correctchoice{$\dfrac{mgH}{T}$}
        \wrongchoice{$\dfrac{mg}{HT}$}
        \wrongchoice{$\dfrac{mgT}{H}$}
        \wrongchoice{$zero$}
    \end{choices}
    \end{multicols}
\end{question}
}

\element{AP}{
\begin{question}{work-Q53}
    A system consists of two objects having masses $m_1$ and $m_2$ $(m_1 < m_2)$.
    The objects are connected by a massless string,
        hung over a pulley as shown, and then released.
    \begin{center}
    \begin{tikzpicture}
        %% NOTE:
    \end{tikzpicture}
    \end{center}
    When the object of mass $m_2$ has descended a distance $h$,
        the potential energy of the system has decreased by:
    \begin{multicols}{2}
    \begin{choices}
      \correctchoice{$gh (m_2 - m_1)$}
        \wrongchoice{$gh m_2$}
        \wrongchoice{$gh \left( m_1 + m_2 \right)$}
        \wrongchoice{$\dfrac{gh}{2} \left( m_1 + m_2 \right)$}
        \wrongchoice{zero}
    \end{choices}
    \end{multicols}
\end{question}
}

\element{AP}{
\begin{question}{work-Q54}
    The following graphs, all drawn to the same scale,
        represent the net force $F$ as a function of displacement $x$
        for an object that moves along a straight line. 
    Which graph represents the force that will cause the greatest change
        in the kinetic energy of the object from $x=0$ to $x=x_1$?
    \begin{multicols}{2}
    \begin{choices}
        %% NOTE: ans is E
        \wrongchoice{
            \begin{tikzpicture}
            \end{tikzpicture}
        }
    \end{choices}
    \end{multicols}
\end{question}
}

\element{AP}{
\begin{question}{work-Q55}
    From the top of a \SI{70}{\meter} high building,
        a \SI{1}{\kilo\gram} ball is thrown directly downward with an initial speed of \SI{10}{\meter\per\second}.
    If the ball reaches the ground with a speed of \SI{30}{\meter\per\second},
        the energy lost to friction is most nearly:
    \begin{multicols}{2}
    \begin{choices}
        \wrongchoice{\SI{0}{\joule}}
        \wrongchoice{\SI{100}{\joule}}
      \correctchoice{\SI{300}{\joule}}
        \wrongchoice{\SI{400}{\joule}}
        \wrongchoice{\SI{700}{\joule}}
    \end{choices}
    \end{multicols}
\end{question}
}

\element{AP}{
\begin{question}{work-Q56}
    A pendulum consists of a ball of mass $m$ suspended at the end of a massless cord of length $L$ as shown. 
    \begin{center}
    \begin{tikzpicture}
        %% NOTE:
    \end{tikzpicture}
    \end{center}
    The pendulum is drawn aside through an angle of \ang{60} with the vertical and released. 
    At the low point of its swing,
        the speed of the pendulum ball is:
    \begin{multicols}{2}
    \begin{choices}
      \correctchoice{$\sqrt{gL}$}
        \wrongchoice{$\sqrt{2gL}$}
        \wrongchoice{$\frac{1}{2}gL$}
        \wrongchoice{$gL$}
        \wrongchoice{$2gL$}
    \end{choices}
    \end{multicols}
\end{question}
}

\element{AP}{
\begin{question}{work-Q57}
    A rock is lifted for a certain time by a force $F$ that is greater in magnitude than the rock's weight $W$. 
    The change in kinetic energy of the rock during this time is equal to the:
    \begin{choices}
      \correctchoice{work done by the net force $(F-W)$}
        \wrongchoice{work done by F alone}
        \wrongchoice{work done by W alone}
        \wrongchoice{difference in the momentum of the rock before and after this time}
        \wrongchoice{difference in the potential energy of the rock before and after this time.}
    \end{choices}
\end{question}
}

\element{AP}{
\begin{question}{work-Q58}
    A ball is thrown upward. 
    At a height of \SI{10}{\meter} above the ground,
        the ball has a potential energy of \SI{50}{\joule} (with the potential energy equal to zero at ground level) and is moving upward with a kinetic energy of \SI{50}{\joule}. 
    Air friction is negligible. 
    The maximum height reached by the ball is most nearly:
    \begin{multicols}{2}
    \begin{choices}
        \wrongchoice{\SI{10}{\meter}}
      \correctchoice{\SI{20}{\meter}}
        \wrongchoice{\SI{30}{\meter}}
        \wrongchoice{\SI{40}{\meter}}
        \wrongchoice{\SI{50}{\meter}}
    \end{choices}
    \end{multicols}
\end{question}
}

\element{AP}{
\begin{question}{work-Q59}
    A block on a horizontal frictionless plane is attached to a spring, as shown. 
    \begin{center}
    \begin{tikzpicture}
        %% NOTE:
    \end{tikzpicture}
    \end{center}
    The block oscillates along the $x$-axis with amplitude $A$.
    Which of the following statements about energy is correct?
    \begin{choices}
      \correctchoice{The potential energy of the spring is at a minimum at $x = 0$.}
        \wrongchoice{The potential energy of the spring is at a minimum at $x = A$.}
        \wrongchoice{The kinetic energy of the block is at a minimum at $x = 0$.}
        \wrongchoice{The kinetic energy of the block is at a maximum at $x = A$.}
        \wrongchoice{The kinetic energy of the block is always equal to the potential energy of the spring.}
    \end{choices}
\end{question}
}

\element{AP}{
\begin{question}{work-Q60}
    During a certain time interval,
        a constant force delivers an average power of \SI{4}{\watt} to an object. 
    If the object has an average speed of \SI{2}{\meter\per\second} and the force acts in the direction of motion of the object,
        the magnitude of the force is:
    \begin{multicols}{2}
    \begin{choices}
        \wrongchoice{\SI{16}{\newton}}
        \wrongchoice{\SI{8}{\newton}}
        \wrongchoice{\SI{6}{\newton}}
        \wrongchoice{\SI{4}{\newton}}
      \correctchoice{\SI{2}{\newton}}
    \end{choices}
    \end{multicols}
\end{question}
}

\element{AP}{
\begin{question}{work-Q61}
    A spring-loaded gun can fire a projectile to a height $h$ if it is fired straight up. 
    If the same gun is pointed at an angle of \ang{45} from the vertical,
        what maximum height can now be reached by the projectile?
    \begin{multicols}{2}
    \begin{choices}
        \wrongchoice{$\dfrac{h}{4}$}
        \wrongchoice{$\dfrac{h}{2\sqrt{2}}$}
      \correctchoice{$\dfrac{h}{2}$}
        \wrongchoice{$\dfrac{h}{\sqrt{2}}$}
        \wrongchoice{$h$}
    \end{choices}
    \end{multicols}
\end{question}
}

\element{AP}{
\begin{question}{work-Q62}
    A force $F$ is exerted by a broom handle on the head of the broom,
        which has a mass $m$.
    The handle is at an angle $\theta$ to the horizontal, as shown. 
    \begin{center}
    \begin{tikzpicture}
        %% NOTE:
    \end{tikzpicture}
    \end{center}
    The work done by the force on the head of the broom as it moves a distance $d$ across a horizontal floor is:
    \begin{multicols}{2}
    \begin{choices}
        \wrongchoice{$Fd \sin\theta$}
      \correctchoice{$Fd \cos\theta$}
        \wrongchoice{$Fm \cos\theta$}
        \wrongchoice{$Fm \tan\theta$}
        \wrongchoice{$Fmd \sin\theta$}
    \end{choices}
    \end{multicols}
\end{question}
}

\newcommand{\workQSixtyThree}{
\begin{tikzpicture}
    %% NOTE:
\end{tikzpicture}
}

\element{AP}{
\begin{question}{work-Q63}
    %% Question 63-64
    A spring has a force constant of \SI{100}{\newton\per\meter} and an unstretched length of \SI{0.07}{\meter}.
    One end is attached to a post that is free to rotate in the center of a smooth table,
        as shown in the top view. 
    The other end is attached to a \SI{1}{\kilo\gram} disc moving in uniform circular motion on the table,
        which stretches the spring by \SI{0.03}{\meter}.
    Friction is negligible.
    \begin{center}
        \workQSixtyThree
    \end{center}
    What is the centripetal force on the disc?
    \begin{multicols}{3}
    \begin{choices}
        \wrongchoice{\SI{0.3}{\newton}}
      \correctchoice{\SI{3}{\newton}}
        \wrongchoice{\SI{10}{\newton}}
        \wrongchoice{\SI{300}{\newton}}
        \wrongchoice{\SI{1,000 }{\newton}}
    \end{choices}
    \end{multicols}
\end{question}
}

\element{AP}{
\begin{question}{work-Q64}
    %% Question 63-64
    A spring has a force constant of \SI{100}{\newton\per\meter} and an unstretched length of \SI{0.07}{\meter}.
    One end is attached to a post that is free to rotate in the center of a smooth table,
        as shown in the top view. 
    The other end is attached to a \SI{1}{\kilo\gram} disc moving in uniform circular motion on the table,
        which stretches the spring by \SI{0.03}{\meter}.
    Friction is negligible.
    \begin{center}
        \workQSixtyThree
    \end{center}
    What is the work done on the disc by the spring during one full circle?
    \begin{multicols}{3}
    \begin{choices}
      \correctchoice{\SI{0}{\joule}}
        \wrongchoice{\SI{94}{\joule}}
        \wrongchoice{\SI{186}{\joule}}
        \wrongchoice{\SI{314}{\joule}}
        \wrongchoice{\SI{628}{\joule}}
    \end{choices}
    \end{multicols}
\end{question}
}

\element{AP}{
\begin{question}{work-Q65}
    A frictionless pendulum of length \SI{3}{\meter} swings with an amplitude of \ang{10}.
    At its maximum displacement,
        the potential energy of the pendulum is \SI{10}{\joule}.
    What is the kinetic energy of the pendulum when its potential energy is \SI{5}{\joule}?
    \begin{multicols}{3}
    \begin{choices}
        \wrongchoice{\SI{3.3}{\joule}}
      \correctchoice{\SI{5}{\joule}}
        \wrongchoice{\SI{6.7}{\joule}}
        \wrongchoice{\SI{10}{\joule}}
        \wrongchoice{\SI{15}{\joule}}
    \end{choices}
    \end{multicols}
\end{question}
}

\element{AP}{
\begin{question}{work-Q66}
    A descending elevator of mass \SI{1 000}{\kilo\gram} is uniformly decelerated to rest over a distance of \SI{8}{\meter} by a cable in which the tension is \SI{11 000}{\newton}. 
    \begin{center}
    \begin{tikzpicture}
        %% NOTE:
    \end{tikzpicture}
    \end{center}
    The speed $v_i$ of the elevator at the beginning of the \SI{8}{\meter} descent is most nearly:
    \begin{multicols}{3}
    \begin{choices}
      \correctchoice{\SI{4}{\meter\per\second}}
        \wrongchoice{\SI{10}{\meter\per\second}}
        \wrongchoice{\SI{13}{\meter\per\second}}
        \wrongchoice{\SI{16}{\meter\per\second}}
        \wrongchoice{\SI{21}{\meter\per\second}}
    \end{choices}
    \end{multicols}
\end{question}
}

\element{AP}{
\begin{question}{work-Q67}
    An ideal massless spring is fixed to the wall at one end, as shown. 
    \begin{center}
    \begin{tikzpicture}
        %% NOTE:
    \end{tikzpicture}
    \end{center}
    A block of mass $M$ attached to the other end of the spring oscillates with amplitude $A$ on a frictionless, horizontal surface. 
    The maximum speed of the block is $v_m$.
    The force constant of the spring is:
    \begin{multicols}{3}
    \begin{choices}
        \wrongchoice{$\dfrac{Mg}{A}$}
        \wrongchoice{$\dfrac{Mgv_m}{2A}$}
        \wrongchoice{$\dfrac{Mv_m^2}{2A}$}
      \correctchoice{$\dfrac{Mv_m^2}{A^2}$}
        \wrongchoice{$\dfrac{Mv_m^2}{2A^2}$}
    \end{choices}
    \end{multicols}
\end{question}
}

\element{AP}{
\begin{question}{work-Q68}
    A \SI{1000}{\watt} electric motor lifts a \SI{100}{\kilo\gram} safe at constant velocity. 
    The vertical distance through which the motor can raise the safe in \SI{10}{\second} is most nearly:
    \begin{multicols}{3}
    \begin{choices}
        \wrongchoice{\SI{1}{\meter}} 
        \wrongchoice{\SI{3}{\meter}} 
      \correctchoice{\SI{10}{\meter}} 
        \wrongchoice{\SI{32}{\meter}} 
        \wrongchoice{\SI{100}{\meter}}
    \end{choices}
    \end{multicols}
\end{question}
}

\element{AP}{
\begin{question}{work-Q69}
    A deliveryman moves 10 cartons from the sidewalk,
        along a \SI{10}{\meter} ramp to a loading dock,
        which is \SI{1.5}{\meter} above the sidewalk. 
    \begin{center}
    \begin{tikzpicture}
        %% NOTE:
    \end{tikzpicture}
    \end{center}
    If each carton has a mass of \SI{25}{\kilo\gram},
        what is the total work done by the deliveryman on the cartons to move them to the loading dock?
    \begin{multicols}{2}
    \begin{choices}
        \wrongchoice{\SI{2500}{\joule}}
      \correctchoice{\SI{3750}{\joule}}
        \wrongchoice{\SI{10000}{\joule}}
        \wrongchoice{\SI{25000}{\joule}}
        \wrongchoice{\SI{37500}{\joule}}
    \end{choices}
    \end{multicols}
\end{question}
}

\element{AP}{
\begin{question}{work-Q70}
    A \SI{60.0}{\kilo\gram} ball of clay is tossed vertically in the air with an initial speed of \SI{4.60}{\meter\per\second}. 
    Ignoring air resistance,
        what is the change in its potential energy when it reaches its highest point?
    \begin{multicols}{2}
    \begin{choices}
        \wrongchoice{\SI{0}{\joule}}
        \wrongchoice{\SI{45}{\joule}}
        \wrongchoice{\SI{280}{\joule}}
      \correctchoice{\SI{635}{\joule}}
        \wrongchoice{\SI{2700}{\joule}
    \end{choices}
    \end{multicols}
\end{question}
}

\element{AP}{
\begin{question}{work-Q71}
    A \SI{500}{\kilo\gram} car is moving at \SI{28}{\meter\per\second}.
    The driver sees a barrier ahead. 
    If the car takes \SI{95}{\meter} to come to rest,
        what is the magnitude of the minimum average net force necessary to stop?
    \begin{multicols}{2}
    \begin{choices}
        \wrongchoice{\SI{47.5}{\newton}} 
        \wrongchoice{\SI{1400}{\newton}} 
      \correctchoice{\SI{2060}{\newton}} 
        \wrongchoice{\SI{19600}{\newton}} 
        \wrongchoice{\SI{133000}{\newton}}
    \end{choices}
    \end{multicols}
\end{question}
}

\element{AP}{
\begin{question}{work-Q72}
    A person pushes a block of mass $M=\SI{6.0}{\kilo\gram}$ with a constant speed of \SI{5.0}{\meter\per\second} straight up a flat surface inclined \ang{30.0} above the horizontal. The coefficient of kinetic friction between the block and the surface is μ = 0.40.
    What is the net force acting on the block?
    \begin{multicols}{2}
    \begin{choices}
      \correctchoice{\SI{0}{\newton}}
        \wrongchoice{\SI{21}{\newton}}
        \wrongchoice{\SI{30}{\newton}}
        \wrongchoice{\SI{51}{\newton}}
        \wrongchoice{\SI{76}{\newton}}
    \end{choices}
    \end{multicols}
\end{question}
}

\element{AP}{
\begin{question}{work-Q73}
    A block of mass $M$ on a horizontal surface is connected to the end of a massless spring of spring constant $k$. 
    The block is pulled a distance $x$ from equilibrium and when released from rest,
        the block moves toward equilibrium. 
    \begin{center}
    \begin{tikzpicture}
        %% NOTE:
    \end{tikzpicture}
    \end{center}
    What coefficient of kinetic friction between the surface and the block would allow the block to return to equilibrium and stop?
    \begin{multicols}{2}
    \begin{choices}
        \wrongchoice{$\dfrac{kx^2}{2Mg}$}
        \wrongchoice{$\dfrac{kx}{Mg}$}
      \correctchoice{$\dfrac{kx}{2Mg}$}
        \wrongchoice{$\dfrac{Mg}{2kx}$}
        \wrongchoice{$\dfrac{k}{4Mgx}$}
    \end{choices}
    \end{multicols}
\end{question}
}

\element{AP}{
\begin{question}{work-Q74}
    An object is dropped from rest from a certain height. 
    Air resistance is negligible. 
    After falling a distance $d$,
        the object's kinetic energy is proportional to which of the following?
    \begin{multicols}{3}
    \begin{choices}
        \wrongchoice{$\dfrac{1}{d^2}$}
        \wrongchoice{$\dfrac{1}{d}$}
        \wrongchoice{$\sqrt{d}$}
      \correctchoice{$d$}
        \wrongchoice{$d^2$}
    \end{choices}
    \end{multicols}
\end{question}
}

\element{AP}{
\begin{question}{work-Q75}
    An object is projected vertically upward from ground level. 
    It rises to a maximum height $H$. 
    If air resistance is negligible,
        which of the following must be true for the object when it is at a height $H/2$?
    \begin{choices}
        \wrongchoice{Its speed is half of its initial speed.}
      \correctchoice{Its kinetic energy is half of its initial kinetic energy.}
        \wrongchoice{Its potential energy is half of its initial potential energy.}
        \wrongchoice{Its total mechanical energy is half of its initial value.}
        \wrongchoice{Its total mechanical energy is half of its value at the highest point.}
    \end{choices}
\end{question}
}

\element{AP}{
\begin{question}{work-Q76}
    A particle $P$ moves around the circle of radius $R$ under the influence of a radial force of magnitude $F$ as shown. 
    \begin{center}
    \begin{tikzpicture}
        %% NOTE:
    \end{tikzpicture}
    \end{center}
    What is the work done by the radial force as the particle moves from position 1 to position 2 halfway around the circle?
    \begin{multicols}{3}
    \begin{choices}
      \correctchoice{Zero}
        \wrongchoice{$RF$}
        \wrongchoice{$2RF$}
        \wrongchoice{$\pi RF$}
        \wrongchoice{$2\pi RF$}
    \end{choices}
    \end{multicols}
\end{question}
}


\endinput


