
%% XXXXXX Questions used on the
%% NYSED Physics Regents Examination
%%--------------------------------------------------

%% this section contains XX problems


%% Section June2017
%%--------------------
\element{nysed}{
\begin{question}{June2017-Q01}
    A unit used for a vector quantity is:
    \begin{multicols}{2}
    \begin{choices}
        \wrongchoice{watt}
      \correctchoice{newton}
        \wrongchoice{kilogram}
        \wrongchoice{second}
    \end{choices}
    \end{multicols}
\end{question}
}

\element{nysed}{
\begin{question}{June2017-Q02}
    A displacement vector with a magnitude of \SI{20}{\meter} meters could have perpendicular components with magnitudes of:
    \begin{multicols}{2}
    \begin{choices}
        \wrongchoice{\SI{10}{\meter} and \SI{10}{\meter}}
        \wrongchoice{\SI{12}{\meter} and \SI{8}{\meter}}
        \wrongchoice{\SI{12}{\meter} and \SI{16}{\meter}}
        \wrongchoice{\SI{16}{\meter} and \SI{8.0}{\meter}}
    \end{choices}
    \end{multicols}
\end{question}
}

\element{nysed}{
\begin{question}{June2017-Q03}
    A hiker travels \SI{1.0}{\kilo\meter} south,
        turns and travels \SI{3.0}{\kilo\meter} west,
        and then turns and travels \SI{3.0}{\kilo\meter} north. 
    What is the total distance traveled by the hiker?
    \begin{multicols}{2}
    \begin{choices}
        \wrongchoice{\SI{3.2}{\kilo\gram}} 
        \wrongchoice{\SI{3.6}{\kilo\gram}}
        \wrongchoice{\SI{5.0}{\kilo\gram}}
        \wrongchoice{\SI{7.0}{\kilo\gram}}
    \end{choices}
    \end{multicols}
\end{question}
}

\element{nysed}{
\begin{question}{June2017-Q04}
    A car with an initial velocity of \SI{16.0}{\meter\per\second} second east
        slows uniformly to \SI{6.0}{\meter\per\second} east in \SI{4.0}{\second}.
    What is the acceleration of the car during this \SI{4.0}{\second}interval?
    \begin{multicols}{2}
    \begin{choices}
        \wrongchoice{\SI{2.5}{\meter\per\second\squared} west}
        \wrongchoice{\SI{2.5}{\meter\per\second\squared} east}
        \wrongchoice{\SI{4.0}{\meter\per\second\squared} west}
        \wrongchoice{\SI{4.0}{\meter\per\second\squared} east}
    \end{choices}
    \end{multicols}
\end{question}
}

\element{nysed}{
\begin{question}{June2017-Q05}
    On the surface of planet $X$,
        a body with a mass of \SI{10}{\kilo\gram} weighs \SI{40}{\newton}.
    The magnitude of the acceleration due to gravity on the surface of planet $X$ is:
    \begin{multicols}{2}
    \begin{choices}
        \wrongchoice{\SI{4.0e3}{\meter\per\second\squared}}
        \wrongchoice{\SI{4.0e2}{\meter\per\second\squared}}
        \wrongchoice{\SI{9.8}{\meter\per\second\squared}}
        \wrongchoice{\SI{4.0}{\meter\per\second\squared}}
    \end{choices}
    \end{multicols}
\end{question}
}

\element{nysed}{
\begin{question}{June2017-Q06}
    A car traveling in a straight line at an initial speed of \SI{8.0}{\meter\per\second}
        accelerates uniformly to a speed of \SI{14}{\meter\per\second} over a distance of \SI{44}{\meter}.
    What is the magnitude of the acceleration of the car?
    \begin{multicols}{2}
    \begin{choices}
        \wrongchoice{\SI{0.41}{\meter\per\second\squared}}
        \wrongchoice{\SI{1.5}{\meter\per\second\squared}}
        \wrongchoice{\SI{3.0}{\meter\per\second\squared}}
        \wrongchoice{\SI{2.2}{\meter\per\second\squared}}
    \end{choices}
    \end{multicols}
\end{question}
}

\element{nysed}{
\begin{question}{June2017-Q07}
    An object starts from rest and falls freely for \SI{40}{\meter} near the surface of planet $P$.
    If the time of fall is \SI{4.0}{\second},
        what is the magnitude of the acceleration due to gravity on planet $P$?
    \begin{multicols}{2}
    \begin{choices}
          \wrongchoice{\SI{0}{\meter\per\second\squared}}
          \wrongchoice{\SI{1.3}{\meter\per\second\squared}}
          \wrongchoice{\SI{5.0}{\meter\per\second\squared}}
          \wrongchoice{\SI{10.}{\meter\per\second\squared}}
    \end{choices}
    \end{multicols}
\end{question}
}

\element{nysed}{
\begin{question}{June2017-Q08}
    If a block is in equilibrium,
        the magnitude of the block's acceleration is:
    \begin{choices}
        \wrongchoice{zero}
        \wrongchoice{decreasing}
        \wrongchoice{increasing}
        \wrongchoice{constant, but not zero}
    \end{choices}
\end{question}
}

\element{nysed}{
\begin{question}{June2017-Q09}
    The diagram below shows a light ray striking a plane mirror.
    \begin{center}
    \begin{tikzpicture}
        %% TODO
    \end{tikzpicture}
    \end{center}
    What is the angle of reflection?
    \begin{multicols}{2}
    \begin{choices}
        \wrongchoice{\ang{30.}}
        \wrongchoice{\ang{60.}}
        \wrongchoice{\ang{90.}}
        \wrongchoice{\ang{120.}}
    \end{choices}
    \end{multicols}
\end{question}
}

\element{nysed}{
\begin{question}{June2017-Q10}
    An electric field exerts an electrostatic force of magnitude \SI{1.5e-14}{\newton} on an electron within the field. 
    What is the magnitude of the electric field strength at the location of the electron?
    \begin{multicols}{2}
    \begin{choices}
        \wrongchoice{\SI{2.4e-33}{\newton\per\coulomb}}
        \wrongchoice{\SI{1.1e-5}{\newton\per\coulomb}}
        \wrongchoice{\SI{9.4e4}{\newton\per\coulomb}}
        \wrongchoice{\SI{1.6e16}{\newton\per\coulomb}}
    \end{choices}
    \end{multicols}
\end{question}
}

\element{nysed}{
\begin{question}{June2017-Q11}
    A \SI{7.0}{\kilo\gram} cart, $A$,
        and a \SI{3.0}{\kilo\gram} cart, $B$,
        are initially held together at rest on a horizontal, frictionless surface. 
        When a compressed spring attached to one of the carts is released, the carts are pushed apart. 
    After the spring is released, the speed of cart $B$ is \SI{6.0}{\meter\per\second},
        as represented in the diagram below.
    \begin{center}
    \begin{tikzpicture}
        %% TODO
    \end{tikzpicture}
    \end{center}
    What is the speed of cart $A$ after the spring is released? 
    \begin{multicols}{2}
    \begin{choices}
        \wrongchoice{\SI{14}{\meter\per\second}}
        \wrongchoice{\SI{6.0}{\meter\per\second}} 
        \wrongchoice{\SI{3.0}{\meter\per\second}} 
        \wrongchoice{\SI{2.6}{\meter\per\second}}
    \end{choices}
    \end{multicols}
\end{question}
}

\element{nysed}{
\begin{question}{June2017-Q12}
    An electron in a magnetic field travels at constant speed in the circular path represented in the diagram below.
    \begin{center}
    \begin{tikzpicture}
        %% TODO
    \end{tikzpicture}
    \end{center}
    Which arrow represents the direction of the net force acting on the electron when the electron is at position $A$?
    \begin{multicols}{2}
    \begin{choices}
        %% (1) Up, (2) Down, (3) Left, (4), right
        \wrongchoice{
            \begin{tikzpicture}
            \end{tikzpicture}
        }
    \end{choices}
    \end{multicols}
\end{question}
}

\element{nysed}{
\begin{question}{June2017-Q13}
    The potential difference between two points, $A$ and $B$, in an electric field is \SI{2.00}{\volt}.
    The energy required to move a charge of \SI{8.00e-19}{\coulomb} from point $A$ to point $B$ is:
    \begin{multicols}{2}
    \begin{choices}
        \wrongchoice{\SI{4.00e-19}{\joule}} 
        \wrongchoice{\SI{1.60e-18}{\joule}} 
        \wrongchoice{\SI{6.25e17}{\joule}} 
        \wrongchoice{\SI{2.50e18}{\joule}}
    \end{choices}
    \end{multicols}
\end{question}
}

\element{nysed}{
\begin{question}{June2017-Q14}
    Which statement describes the gravitational force and the electrostatic force between two charged particles?
    \begin{choices}
        \wrongchoice{The gravitational force may be either attractive or repulsive, whereas the electrostatic force must be attractive.}
        \wrongchoice{The gravitational force must be attractive, whereas the electrostatic force may be either attractive or repulsive.}
        \wrongchoice{Both forces may be either attractive or repulsive.}
        \wrongchoice{Both forces must be attractive.}
    \end{choices}
\end{question}
}

\element{nysed}{
\begin{question}{June2017-Q15}
    An electrostatic force exists between two \SI{+3.20e-19}{\coulomb} point charges separated by a distance of \SI{0.030}{\meter}. 
    As the distance between the two point charges is decreased,
        the electrostatic force of:
    \begin{choices}
        \wrongchoice{attraction between the two charges decreases}
        \wrongchoice{attraction between the two charges increases}
        \wrongchoice{repulsion between the two charges decreases}
        \wrongchoice{repulsion between the two charges increases}
    \end{choices}
\end{question}
}

\element{nysed}{
\begin{question}{June2017-Q16}
    What is the energy of the photon emitted when an electron in a mercury atom drops from energy level $f$ to energy level $b$?
    \begin{multicols}{2}
    \begin{choices}
        \wrongchoice{\SI{8.42}{\eV}} 
        \wrongchoice{\SI{5.74}{\eV}} 
        \wrongchoice{\SI{3.06}{\eV}} 
        \wrongchoice{\SI{2.68}{\eV}}
    \end{choices}
    \end{multicols}
\end{question}
}

\element{nysed}{
\begin{question}{June2017-Q17}
    An observer counts $4$ complete water waves passing by the end of a dock every \SI{10}{\second}.
    What is the frequency of the waves?
    \begin{multicols}{2}
    \begin{choices}
        \wrongchoice{\SI{0.40}{\hertz}} 
        \wrongchoice{\SI{2.5}{\hertz}} 
        \wrongchoice{\SI{40.}{\hertz}} 
        \wrongchoice{\SI{4.0}{\hertz}}
    \end{choices}
    \end{multicols}
\end{question}
}

\element{nysed}{
\begin{question}{June2017-Q18}
    Copper is a metal commonly used for electrical wiring in houses. 
    Which metal conducts electricity better than copper at \SI{20}{\degreeCelsius}? 
    \begin{choices}
        \wrongchoice{aluminum}
        \wrongchoice{gold}
        \wrongchoice{nichrome}
        \wrongchoice{silver}
    \end{choices}
\end{question}
}

\element{nysed}{
\begin{question}{June2017-Q19}
    A motor does \SI{20}{\joule} joules of work on a block,
        accelerating the block vertically upward. 
    Neglecting friction, if the gravitational potential energy of the block increases by \SI{15}{\Joule},
        its kinetic energy
    \begin{choices}
        \wrongchoice{decreases by \SI{5}{\joule}} 
        \wrongchoice{increases by \SI{5}{\joule}} 
        \wrongchoice{decreases by \SI{35}{\joule}} 
        \wrongchoice{increases by \SI{35}{\joule}}
    \end{choices}
\end{question}
}

\element{nysed}{
\begin{question}{June2017-Q20}
    When only one lightbulb blows out, an entire string of decorative lights goes out. 
    The lights in this string must be connected in:
    \begin{choices}
        \wrongchoice{parallel with one current pathway}
        \wrongchoice{parallel with multiple current pathways}
        \wrongchoice{series with one current pathway}
        \wrongchoice{series with multiple current pathways}
    \end{choices}
\end{question}
}

\element{nysed}{
\begin{question}{June2017-Q21}
    An electric toaster is rated \SI{1200}{\watt} at \SI{120}{\Volt}.
    What is the total electrical energy used to operate the toaster for \SI{30}{\second}?
    \begin{multicols}{2}
    \begin{choices}
        \wrongchoice{\SI{1.8e3}{\joule}} 
        \wrongchoice{\SI{1.8e4}{\joule}} 
        \wrongchoice{\SI{3.6e3}{\joule}} 
        \wrongchoice{\SI{3.6e4}{\joule}}
    \end{choices}
    \end{multicols}
\end{question}
}

\element{nysed}{
\begin{question}{June2017-Q22}
    What is the rate at which work is done in lifting a \SI{35}{\kilo\gram}
        object vertically at a constant speed of \SI{5.0}{\meter\per\second}? 
    \begin{multicols}{2}
    \begin{choices}
        \wrongchoice{\SI{1700}{\watt}} 
        \wrongchoice{\SI{340}{\watt}} 
        \wrongchoice{\SI{180}{\watt}} 
        \wrongchoice{\SI{7.0}{\watt}}
    \end{choices}
    \end{multicols}
\end{question}
}

\element{nysed}{
\begin{question}{June2017-Q23}
    When a wave travels through a medium, the wave transfers:
    \begin{choices}
        \wrongchoice{mass, only}
        \wrongchoice{energy, only}
        \wrongchoice{both mass and energy}
        \wrongchoice{neither mass nor energy}
    \end{choices}
\end{question}
}

\element{nysed}{
\begin{question}{June2017-Q24}
    Glass may shatter when exposed to sound of a particular frequency. 
    This phenomenon is an example of:
    \begin{choices}
        \wrongchoice{refraction}
        \wrongchoice{resonance}
        \wrongchoice{diffraction}
        \wrongchoice{the Doppler effect}
    \end{choices}
\end{question}
}

\element{nysed}{
\begin{question}{June2017-Q25}
    Which waves require a material medium for transmission?
    \begin{choices}
        \wrongchoice{light waves}
        \wrongchoice{radio waves}
        \wrongchoice{sound waves}
        \wrongchoice{microwaves}
    \end{choices}
\end{question}
}

\element{nysed}{
\begin{question}{June2017-Q26}
    Which type of oscillation would most likely produce an electromagnetic wave? 
    \begin{choices}
        \wrongchoice{a vibrating tuning fork}
        \wrongchoice{a washing machine agitator at work}
        \wrongchoice{a swinging pendulum}
        \wrongchoice{an electron traveling back and forth in a wire}
    \end{choices}
\end{question}
}

\element{nysed}{
\begin{question}{June2017-Q27}
    If monochromatic light passes from water into air with an angle of incidence of \ang{35},
        which characteristic of the light will remain the same? 
    \begin{multicols}{2}
    \begin{choices}
        \wrongchoice{frequency }
        \wrongchoice{wavelength }
        \wrongchoice{speed }
        \wrongchoice{direction}
    \end{choices}
    \end{multicols}
\end{question}
}

\element{nysed}{
\begin{question}{June2017-Q28}
    The absolute index of refraction of medium $Y$ is twice as great as the absolute index of refraction of medium $X$. 
    As a light ray travels from medium $X$ into medium $Y$, the speed of the light ray is 
    \begin{multicols}{2}
    \begin{choices}
        \wrongchoice{halved}
        \wrongchoice{doubled}
        \wrongchoice{quartered}
        \wrongchoice{quadruple}
    \end{choices}
    \end{multicols}
\end{question}
}

\element{nysed}{
\begin{question}{June2017-Q29}
    The diagram below shows a transverse wave moving toward the right along a rope.
    \begin{center}
    \begin{tikzpicture}
        %% TODO
    \end{tikzpicture}
    \end{center}
    At the instant shown, point $P$ on the rope is moving toward the:
    \begin{multicols}{2}
    \begin{choices}
        \wrongchoice{bottom of the page}
        \wrongchoice{top of the page}
        \wrongchoice{left}
        \wrongchoice{right}
    \end{choices}
    \end{multicols}
\end{question}
}

\element{nysed}{
\begin{question}{June2017-Q30}
    When an isolated conductor is placed in the vicinity of a positive charge, the conductor is attracted to the charge. 
    The charge of the conductor 
    \begin{choices}
        \wrongchoice{must be positive}
        \wrongchoice{must be negative}
        \wrongchoice{could be neutral or positive}
        \wrongchoice{could be neutral or negative}
    \end{choices}
\end{question}
}

\element{nysed}{
\begin{question}{June2017-Q31}
    The quarks that compose a baryon may have charges of:
    \begin{choices}
        \wrongchoice{$+\dfrac{2}{3}$e, $+\dfrac{2}{3}$e, and $-\dfrac{1}{3}$e}
        \wrongchoice{$+\dfrac{1}{3}$e, $-\dfrac{1}{3}$e, and $+\dfrac{2}{3}$e}
        \wrongchoice{$-1$e, $-1$e, and $0$}
        \wrongchoice{$+\dfrac{2}{3}$e, $+\dfrac{2}{3}$e, and $0$}
    \end{choices}
\end{question}
}

\element{nysed}{
\begin{question}{June2017-Q32}
    A rubber block weighing \SI{60}{\Newton} is resting on a horizontal surface of dry asphalt. 
    What is the magnitude of the minimum force needed to start the rubber block moving across the dry asphalt?
    \begin{multicols}{2}
    \begin{choices}
        \wrongchoice{\SI{32}{\newton}} 
        \wrongchoice{\SI{40}{\newton}}
        \wrongchoice{\SI{51}{\newton}}
        \wrongchoice{\SI{60}{\newton}}
    \end{choices}
    \end{multicols}
\end{question}
}

\element{nysed}{
\begin{question}{June2017-Q33}
    The data table below lists the mass and speed of four different objects.
    Which object has the greatest inertia? 
    %% TODO: Table in options, look for examples for formatting
    \begin{multicols}{2}
    \begin{choices}
        \wrongchoice{}
    \end{choices}
    \end{multicols}
\end{question}
}

\element{nysed}{
\begin{question}{June2017-Q34}
    The electroscope shown in the diagram below is made completely of metal and consists of a knob, a stem, and leaves. 
    A positively charged rod is brought near the knob of the electroscope and then removed.
    \begin{center}
    \begin{tikzpicture}
        %% TODO
    \end{tikzpicture}
    \end{center}
    The motion of the leaves results from electrons moving from the:
    \begin{choices}
        \wrongchoice{leaves to the knob, only}
        \wrongchoice{knob to the leaves, only}
        \wrongchoice{leaves to the knob and then back to the leaves}
        \wrongchoice{knob to the leaves and then back to the knob}
    \end{choices}
\end{question}
}

\element{nysed}{
\begin{question}{June2017-Q35}
    Which circuit diagram represents the correct way to measure the current in a resistor?
    \begin{multicols}{2}
    \begin{choices}
        \wrongchoice{
            \begin{circuitikz}
                %% TODO
            \end{circuitikz}
        }
    \end{choices}
    \end{multicols}
\end{question}
}

%% Part B-1
\element{nysed}{
\begin{question}{June2017-Q36}
    The height of a typical kitchen table is approximately:
    \begin{multicols}{2}
    \begin{choices}
        \wrongchoice{\SI{e-2}{\meter}} 
        \wrongchoice{\SI{e0}{\meter}} 
        \wrongchoice{\SI{e1}{\meter}} 
        \wrongchoice{\SI{e2}{\meter}} 
    \end{choices}
    \end{multicols}
\end{question}
}

\element{nysed}{
\begin{question}{June2017-Q37}
    A ball is thrown with a velocity of \SI{35}{\meter\per\second} at an angle of \ang{30}  above the horizontal. 
    Which quantity has a magnitude of zero when the ball is at the highest point in its trajectory? 
    \begin{choices}
        \wrongchoice{the acceleration of the ball}
        \wrongchoice{the momentum of the ball}
        \wrongchoice{the horizontal component of the ball's velocity}
        \wrongchoice{the vertical component of the ball's velocity}
    \end{choices}
\end{question}
}

\element{nysed}{
\begin{question}{June2017-Q38}
    The graph below represents the relationship between velocity and time of travel for a toy car moving in a straight line.
    \begin{center}
    \begin{tikzpicture}
        %% TODO
    \end{tikzpicture}
    \end{center}
    The shaded area under the line represents the toy car's:
    \begin{multicols}{2}
    \begin{choices}
        \wrongchoice{displacement}
        \wrongchoice{momentum}
        \wrongchoice{acceleration}
        \wrongchoice{speed}
    \end{choices}
    \end{multicols}{2}
\end{question}
}

\element{nysed}{
\begin{question}{June2017-Q39}
    A spring stores \SI{10}{\Joule} of elastic potential energy when it is compressed \SI{0.20}{\meter}.
    What is the spring constant of the spring? 
    \begin{multicols}{2}
    \begin{choices}
        \wrongchoice{\SI{5.0e1}{\newton\per\meter}} 
        \wrongchoice{\SI{1.0e2}{\newton\per\meter}} 
        \wrongchoice{\SI{2.5e2}{\newton\per\meter}} 
        \wrongchoice{\SI{5.0e2}{\newton\per\meter}}
    \end{choices}
    \end{multicols}
\end{question}
}

\element{nysed}{
\begin{question}{June2017-Q40}
    Base your answers to questions 40 and 41 on the information below and on your knowledge of physics.

    A cannonball with a mass of \SI{1.0}{\kilo\gram} is fired horizontally from a \SI{500}{\kilo\gram} cannon,
        initially at rest, on a horizontal, frictionless surface. 
    The cannonball is acted on by an average force of \SI{8.0e3}{\newton} for \SI{1.0e-1}{\second}.
    %% Start question
    What is the magnitude of the change in momentum of the cannonball during firing? 
    \begin{multicols}{2}
    \begin{choices}
        \wrongchoice{\SI{0}{\kilo\gram\meter\per\second}} 
        \wrongchoice{\SI{8.0e2}{\kilo\gram\meter\per\second}} 
        \wrongchoice{\SI{8.0e3}{\kilo\gram\meter\per\second}} 
        \wrongchoice{\SI{8.0e4}{\kilo\gram\meter\per\second}}
    \end{choices}
    \end{multicols}
\end{question}
}

\element{nysed}{
\begin{question}{June2017-Q41}
    Base your answers to questions 40 and 41 on the information below and on your knowledge of physics.

    \siA cannonball with a mass of \SI{1.0}{\kilo\gram} is fired horizontally from a \SI{500}{\kilo\gram} cannon,
        initially at rest, on a horizontal, frictionless surface. 
    The cannonball is acted on by an average force of \SI{8.0e3}{\newton} for \SI{1.0e-1}{\second}.
    %% Start question
    What is the magnitude of the average net force acting on the cannon? 
    \begin{multicols}{2}
    \begin{choices}
        \wrongchoice{\SI{1.6}{\newton}} 
        \wrongchoice{\SI{16}{\newton}} 
        \wrongchoice{\SI{8.0e3}{\newton}} 
        \wrongchoice{\SI{4.0e6}{\newton}}
    \end{choices}
    \end{multicols}
\end{question}
}

\element{nysed}{
\begin{question}{June2017-Q42}
    A metal sphere, $X$, has an initial net charge of \SI{-6e-6}{\coulomb} and an identical sphere, $Y$,
        has an initial net charge of \SI{+2e-6}{\coulomb}. 
    The spheres touch each other and then separate. 
    What is the net charge on sphere X after the spheres have separated?
    \begin{multicols}{2}
    \begin{choices}
        \wrongchoice{\SI{0}{\coulomb}} 
        \wrongchoice{\SI{-2e-6}{\coulomb}} 
        \wrongchoice{\SI{-4e-6}{\coulomb}} 
        \wrongchoice{\SI{-6e-6}{\coulomb}} 
    \end{choices}
    \end{multicols}
\end{question}
}

\element{nysed}{
\begin{question}{June2017-Q43}
    A constant eastward horizontal force of \SI{70}{\newton} is applied to a \SI{20}{\kilo\gram} crate moving toward the east on a level floor. 
    If the frictional force on the crate has a magnitude of \SI{10}{\newton},
        what is the magnitude of the crate's acceleration?
    \begin{multicols}{2}
    \begin{choices}
        \wrongchoice{\SI{0.50}{\meter\per\second\squared}} 
        \wrongchoice{\SI{3.5}{\meter\per\second\squared}} 
        \wrongchoice{\SI{3.0}{\meter\per\second\squared}} 
        \wrongchoice{\SI{4.0}{\meter\per\second\squared}} 
    \end{choices}
    \end{multicols}
\end{question}
}

\element{nysed}{
\begin{question}{June2017-Q44}
    Which graph represents the relationship between the energy of photons and the wavelengths of photons in a vacuum?
    \begin{multicols}{2}
    \begin{choices}
        \wrongchoice{
            \begin{tikzpicture}
                %% TODO
            \end{tikzpicture}
        }
    \end{choices}
    \end{multicols}
\end{question}
}

\element{nysed}{
\begin{question}{June2017-Q45}
    Base your answers to questions 45 and 46 on the information and diagram below and on your knowledge of physics.

    One end of a long spring is attached to a wall. 
    A student vibrates the other end of the spring vertically,
        creating a wave that moves to the wall and reflects back toward the student,
        resulting in a standing wave in the spring, as represented below.
    \begin{center}
    \begin{tikzpicture}
        %% TODO
    \end{tikzpicture}
    \end{center}
    %% Start question
    What is the phase difference between the incident wave and the reflected wave at point $P$?
    \begin{multicols}{2}
    \begin{choices}
        \wrongchoice{\ang{0}}
        \wrongchoice{\ang{90}}
        \wrongchoice{\ang{180}}
        \wrongchoice{\ang{270}}
    \end{choices}
    \end{multicols}
\end{question}
}

\element{nysed}{
\begin{question}{June2017-Q46}
    Base your answers to questions 45 and 46 on the information and diagram below and on your knowledge of physics.

    One end of a long spring is attached to a wall. 
    A student vibrates the other end of the spring vertically,
        creating a wave that moves to the wall and reflects back toward the student,
        resulting in a standing wave in the spring, as represented below.
    \begin{center}
    \begin{tikzpicture}
        %% TODO
    \end{tikzpicture}
    \end{center}
    %% Start question
    What is the total number of antinodes on the standing wave in the diagram?
    \begin{multicols}{2}
    \begin{choices}
        \wrongchoice{6}
        \wrongchoice{2}
        \wrongchoice{3}
        \wrongchoice{4}
    \end{choices}
    \end{multicols}
\end{question}
}

\element{nysed}{
\begin{question}{June2017-Q47}
    The diagrams below represent four pieces of copper wire at \SI{20}{\degreeCelsius}. 
    For each piece of wire, $l$ represents a unit of length and $A$ represents a unit of cross-sectional area.
    %% TODO: rewrite to reflect formatting
    The piece of wire that has the greatest resistance is (1) wire 1 (3) wire 3 (2) wire 2 (4) wire 4
    \begin{center}
    \end{center}
    \begin{multicols}{2}
    \begin{choices}
        %% TODO: Format to multichoice
        \wrongchoice{
            \begin{tikzpicture}
                %% TODO
            \end{tikzpicture}
        }
    \end{choices}
    \end{multicols}
\end{question}
}

\element{nysed}{
\begin{question}{June2017-Q48}
    Base your answers to questions 48 and 49 on the diagram below,
        which represents two charged, identical metal spheres, and on your knowledge of physics.
    \begin{center}
    \begin{tikzpicture}
        %% TODO
    \end{tikzpicture}
    \end{center}
    %% Start question
    The number of excess elementary charges on sphere $A$ is:
    \begin{multicols}{2}
    \begin{choices}
      \wrongchoice{\num{6.4e-25}}
      \wrongchoice{\num{6.4e-19}}
      \wrongchoice{\num{2.5e13}}
      \wrongchoice{\num{5.0e13}}
    \end{choices}
    \end{multicols}
\end{question}
}

\element{nysed}{
\begin{question}{June2017-Q49}
    Base your answers to questions 48 and 49 on the diagram below,
        which represents two charged, identical metal spheres, and on your knowledge of physics.
    \begin{center}
    \begin{tikzpicture}
        %% TODO
    \end{tikzpicture}
    \end{center}
    %% Start question
    What is the magnitude of the electric force between the two spheres? 
    \begin{multicols}{2}
    \begin{choices}
        \wrongchoice{\SI{3.0e-12}{\newton}} 
        \wrongchoice{\SI{1.0e-6}{\newton}} 
        \wrongchoice{\SI{2.7e-2}{\newton}} 
        \wrongchoice{\SI{5.4e-2}{\newton}}
    \end{choices}
    \end{multicols}
\end{question}
}

\element{nysed}{
\begin{question}{June2017-Q50}
    The diagram below represents the wave fronts produced by a point source moving to the right in a uniform medium. 
    Observers are located at points $A$ and $B$.
    \begin{center}
    \begin{tikzpicture}
        %% TODO
    \end{tikzpicture}
    \end{center}
    Compared to the wave frequency and wavelength observed at point $A$,
        the wave observed at point $B$ has a:
    \begin{choices}
        \wrongchoice{higher frequency and a shorter wavelength}
        \wrongchoice{higher frequency and a longer wavelength}
        \wrongchoice{lower frequency and a shorter wavelength}
        \wrongchoice{lower frequency and a longer wavelengt}
    \end{choices}
\end{question}
}

\endinput


