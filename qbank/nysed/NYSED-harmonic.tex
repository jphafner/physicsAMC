
%% Vibration Questions used on the
%% NYSED Physics Regents Examination
%%--------------------------------------------------

%% this section contains 17 problems


%% Section June2015
%%--------------------
\element{nysed}{
\begin{question}{June2015-Q21}
    A duck floating on a lake oscillates up and down \num{5.0} times during a \SI{10}{\second} interval as a periodic wave passes by. 
    What is the frequency of the duck's oscillations?
    \begin{multicols}{2}
    \begin{choices}
        \wrongchoice{\SI{0.10}{\hertz}}
      \correctchoice{\SI{0.50}{\hertz}}
        \wrongchoice{\SI{2.0}{\hertz}}
        \wrongchoice{\SI{50.}{\hertz}}
    \end{choices}
    \end{multicols}
\end{question}
}


%% Section June2014
%%--------------------
\element{nysed}{
\begin{question}{June2014-Q20}
    A boy pushes his sister on a swing.
    What is the frequency of oscillation of his sister on the swing if the boy counts \num{90} complete swings in \SI{300}{\second}?
    \begin{multicols}{2}
    \begin{choices}
      \correctchoice{\SI{0.30}{\hertz}}
        \wrongchoice{\SI{1.5}{\hertz}}
        \wrongchoice{\SI{2.0}{\hertz}}
        \wrongchoice{\SI{18}{\hertz}}
    \end{choices}
    \end{multicols}
\end{question}
}

\element{nysed}{
\begin{question}{June2014-Q27}
    When air is blown across the top of an open water bottle,
        air molecules in the bottle vibrate at a particular frequency and sound is produces.
    This phenomenon is called:
    \begin{choices}
      \correctchoice{resonance}
        \wrongchoice{diffraction}
        \wrongchoice{the Doppler effect}
        \wrongchoice{refraction}
    \end{choices}
\end{question}
}


%% Section June2013
%%--------------------
\element{nysed}{
\begin{question}{June2013-Q25}
    One vibrating \SI{256}{\hertz} tuning fork transfers energy to another \SI{256}{\hertz} tuning fork,
        causing the second tuning fork to vibrate.
    This phenomenon is an example of
    \begin{multicols}{2}
    \begin{choices}
        \wrongchoice{diffraction}
        \wrongchoice{reflection}
        \wrongchoice{refraction}
      \correctchoice{resonance}
    \end{choices}
    \end{multicols}
\end{question}
}

\element{nysed}{
\begin{question}{June2013-Q29}
    A physics student takes her pulses and determines that her heart beats periodically \num{60} times in \SI{60}{\second}.
    The period of her heartbeat is:
    \begin{multicols}{2}
    \begin{choices}
        \wrongchoice{\SI{1}{\hertz}}
        \wrongchoice{\SI{60}{\hertz}}
      \correctchoice{\SI{1}{\second}}
        \wrongchoice{\SI{60}{\second}}
    \end{choices}
    \end{multicols}
\end{question}
}


%% Section June2012
%%--------------------
\element{nysed}{
\begin{question}{June2012-Q32}
    Ultrasound is a medical technique that transmits sound waves through soft tissue in the human body.
    Ultrasound waves can break kidney stones into tiny fragments,
        making it easier for them to be excreted without pain.
    The shattering of kidney stones with specific frequencies of sound waves is an application of which wave phenomenon?
    \begin{choices}
        \wrongchoice{the Doppler effect}
        \wrongchoice{reflection}
        \wrongchoice{refraction}
      \correctchoice{resonance}
    \end{choices}
\end{question}
}


%% Section June2011
%%--------------------
\element{nysed}{
\begin{question}{June2011-Q32}
    A \SI{256}{\hertz} vibrating tuning fork is brought near a nonvibrating \SI{256}{\hertz} tuning fork.
    The second tuning fork begins to vibrate.
    Which phenomenon causes the nonvibrating tuning fork to begin to vibrate?
    \begin{multicols}{2}
    \begin{choices}
        \wrongchoice{resistance}
      \correctchoice{resonance}
        \wrongchoice{refraction}
        \wrongchoice{reflection}
    \end{choices}
    \end{multicols}
\end{question}
}


%% Section June2010
%%--------------------
\element{nysed}{
\begin{question}{June2010-Q29}
    Playing a certain musical note on a trumpet causes the spring on a bottom of a nearby snare drum to vibrate.
    This phenomenon is an example of:
    \begin{multicols}{2}
    \begin{choices}
      \correctchoice{resonance}
        \wrongchoice{refraction}
        \wrongchoice{reflection}
        \wrongchoice{diffraction}
    \end{choices}
    \end{multicols}
\end{question}
}


%% Section June2009
%%--------------------
\element{nysed}{
\begin{question}{June2009-Q30}
    Sound waves strike a glass and cause it to shatter.
    This phenomenon illustrates:
    \begin{multicols}{2}
    \begin{choices}
      \correctchoice{resonance}
        \wrongchoice{refraction}
        \wrongchoice{reflection}
        \wrongchoice{diffraction}
    \end{choices}
    \end{multicols}
\end{question}
}


%% Section Jan2009
%%--------------------
\element{nysed}{
\begin{question}{Jan2009-Q22}
    A dampened fingertip rubbed around the rim of a crystal stemware glass causes the glass to vibrate and produce a musical note.
    This effect is due to:
    \begin{multicols}{2}
    \begin{choices}
      \correctchoice{resonance}
        \wrongchoice{refraction}
        \wrongchoice{reflection}
        \wrongchoice{rarefaction}
    \end{choices}
    \end{multicols}
\end{question}
}


%% Section June2008
%%--------------------
\element{nysed}{
\begin{question}{June2008-Q40}
    A nearby object may vibrate strongly when a specific frequency of sound is emitted from a loudspeaker.
    This phenomenon is called:
    \begin{multicols}{2}
    \begin{choices}
      \correctchoice{resonance}
        \wrongchoice{the Doppler effect}
        \wrongchoice{reflection}
        \wrongchoice{interference}
    \end{choices}
    \end{multicols}
\end{question}
}


%% Section Jan2008
%%--------------------
\element{nysed}{
\begin{question}{Jan2008-Q29}
    Resonance occurs when one vibrating object transfers energy to a second object causing it to vibrate.
    The energy transfer is most efficient when,
        compared to the first object, the second object has the same natural:
    \begin{multicols}{2}
    \begin{choices}
      \correctchoice{frequency}
        \wrongchoice{loudness}
        \wrongchoice{amplitude}
        \wrongchoice{speed}
    \end{choices}
    \end{multicols}
\end{question}
}


%% Section June2007
%%--------------------
\element{nysed}{
\begin{question}{June2007-Q28}
    A car traveling at \SI{70}{\kilo\meter\per\hour} accelerates to pass a truck.
    When the car reaches a speed of \SI{90}{\kilo\meter\per\hour} the driver hears the glove compartment door start to vibrate.
    By the time the speed of the car is \SI{100}{\kilo\meter\per\hour} the glove compartment door has stopped vibrating.
    This vibrating phenomenon is an example of:
    \begin{choices}
      \correctchoice{resonance}
        \wrongchoice{diffraction}
        \wrongchoice{destructive interference}
        \wrongchoice{the Doppler effect}
    \end{choices}
\end{question}
}


%% Section Jan2007
%%--------------------
\element{nysed}{
\begin{question}{Jan2007-Q29}
    Which wave phenomenon occurs when vibrations in one object cause vibrations in a second object?
    \begin{multicols}{2}
    \begin{choices}
      \correctchoice{resonance}
        \wrongchoice{reflection}
        \wrongchoice{intensity}
        \wrongchoice{tuning}
    \end{choices}
    \end{multicols}
\end{question}
}


%% Section June2006
%%--------------------
\element{nysed}{
\begin{question}{June2006-Q31}
    A girl on a swing may increase the amplitude of the swing's oscillations if she moves her legs at the natural frequency of the swing.
    This is an example of:
    \begin{choices}
      \correctchoice{resonance}
        \wrongchoice{the Doppler effect}
        \wrongchoice{wave transmission}
        \wrongchoice{destructive interference}
    \end{choices}
\end{question}
}


%% Section Jan2006
%%--------------------


%% Section June2005
%%--------------------


%% Section Jan2005
%%--------------------
\element{nysed}{
\begin{question}{Jan2005-Q14}
    A student in a band notices that a drum vibrates when another instrument emits a certain frequency note.
    This phenomenon illustrates:
    \begin{multicols}{2}
    \begin{choices}
      \correctchoice{resonance}
        \wrongchoice{reflection}
        \wrongchoice{refraction}
        \wrongchoice{diffraction}
    \end{choices}
    \end{multicols}
\end{question}
}


%% Section June2004
%%--------------------
\element{nysed}{
\begin{question}{June2004-Q29}
    Which phenomenon occurs when an object absorbs wave energy that matches the object's natural frequency?
    \begin{multicols}{2}
    \begin{choices}
      \correctchoice{resonance}
        \wrongchoice{interference}
        \wrongchoice{reflection}
        \wrongchoice{diffraction}
    \end{choices}
    \end{multicols}
\end{question}
}


%% Section Jan2004
%%--------------------


%% Section June2003
%%--------------------


%% Section Jan2003
%%--------------------

%% Nothing relevant before Jan2003


\endinput

