
%% XXXXXX Questions used on the
%% NYSED Physics Regents Examination
%%--------------------------------------------------

%% NOTE: electromagneticApplication?, 2dkinematics, electrostatics, Sound, light
%% Section June2015
%%--------------------
\element{nysed}{
\begin{question}{June2015-Q01}
    Which quantities are scalar?
    \begin{multicols}{2}
    \begin{choices}
      \correctchoice{speed and work}
        \wrongchoice{velocity and force}
        \wrongchoice{distance and acceleration}
        \wrongchoice{momentum and power}
    \end{choices}
    \end{multicols}
\end{question}
}

\element{nysed}{
\begin{question}{June2015-Q02}
   A \SI{3.00}{\kilogram} mass is thrown vertically upward with an initial speed of \SI{9.80}{\meter\per\second}.
   What is the maximum height this object will reach? [Neglect friction.]
    \begin{multicols}{2}
    \begin{choices}
        \wrongchoice{1.00 m}
      \correctchoice{4.90 m}
        \wrongchoice{9.80 m}
        \wrongchoice{19.6 m}
    \end{choices}
    \end{multicols}
\end{question}
}


\element{nysed}{
\begin{question}{June2015-Q03}
    An airplane traveling north at 220. meters per second encounters a 50.0-meters-per-second crosswind from west to east, as represented in the diagram below.
            \begin{tikzpicture}
                \draw[white] (0,0) rectangle (2,4);
                \draw[domain=0:10,samples=4,mark=*,only marks] plot ({1.0}, {0.02*\x*\x});
                \draw (0,0) -- (2,0) node[pos=0.5,anchor=north] {Earth's Surface};
            \end{tikzpicture}
    What is the resultant speed of the plane?
    \begin{multicols}{2}
    \begin{choices}
        \wrongchoice{170. m/s}
        \wrongchoice{214 m/s}
      \correctchoice{226 m/s}
        \wrongchoice{270. m/s}
    \end{choices}
    \end{multicols}
\end{question}
}

\element{nysed}{
\begin{question}{June2015-Q04}
    A \SI{160}{\kilogram} space vehicle is traveling along a straight line at a constant speed of \SI{800}{\meter\per\second}.
    The magnitude of the net force on the space vehicle is
    \begin{multicols}{2}
    \begin{choices}
        \correctchoice{\SI{0}{\newton}}
          \wrongchoice{\SI{1.60e2}{\newton}}
          \wrongchoice{\SI{8.00e2}{\newton}}
	  \wrongchoice{\SI{1.28e5}{\newton}}
    \end{choices}
    \end{multicols}
\end{question}
}

\element{nysed}{
\begin{question}{June2015-Q05}
   A student throws a \SI{5.0}{\newton} ball straight up.
   What is the net force on the ball at its maximum height?
    \begin{multicols}{2}
    \begin{choices}
        \wrongchoice{0.0 N}
        \wrongchoice{5.0 N, up}
      \correctchoice{5.0 N, down}
        \wrongchoice{9.8 N, down}
    \end{choices}
    \end{multicols}
\end{question}
}

\element{nysed}{
\begin{question}{June2015-Q06}
	A vertical spring has a spring constant of \SI{100}{\newton\per\meter}.
	When an object is attached to the bottom of the spring, the spring changes from its unstretched length of \SI{0.50}{\meter} to a length of \SI{0.65}{\meter}.
	The magnitude of the weight of the attached object is
        \wrongchoice{1.1 N}
      \correctchoice{2 15 N}
        \wrongchoice{50. N}
        \wrongchoice{4 65 N}
    \begin{multicols}{2}
    \begin{choices}
    \end{choices}
    \end{multicols}
\end{question}
}


\element{nysed}{
\begin{question}{June2015-Q07}
	A \SI{1.5}{\kilogram} cart initially moves at \SI{2.0}{\meter\per\second}.
	It is brought to rest by a constant net force in \SI{0.30}{\second}.
	What is the magnitude of the net force?
    \begin{multicols}{2}
    \begin{choices}
           \wrongchoice{\SI{0.40}{\newton}}
           \wrongchoice{\SI{0.90}{\newton}}
	 \correctchoice{\SI{10}{\newton}}
           \wrongchoice{\SI{15}{\newton}}
    \end{choices}
    \end{multicols}
\end{question}
}

\element{nysed}{
\begin{question}{June2015-Q08}
	Which characteristic of a light wave must increase as the light wave passes from glass into air?
    \begin{multicols}{2}
    \begin{choices}
        \wrongchoice{amplitude}
        \wrongchoice{frequency}
        \wrongchoice{period}
      \correctchoice{wavelength}
    \end{choices}
    \end{multicols}
\end{question}
}

\element{nysed}{
\begin{question}{June2015-Q09}
    As a \SI{5.0e2}{\newton} basketball player jumps from the floor up toward the basket,
	the magnitude of the force of her feet on the floor is \SI{1.0e3}{\newton}. 
    As she jumps, the magnitude of the force of the floor on her feet is
    \begin{multicols}{2}
    \begin{choices}
        \wrongchoice{\SI{5.0e2}{\newton}}
      \correctchoice{\SI{1.0e3}{\newton}}
        \wrongchoice{\SI{1.5e3}{\newton}}
        \wrongchoice{\SI{5.0e5}{\newton}}
    \end{choices}
    \end{multicols}
\end{question}
}

\element{nysed}{
\begin{question}{June2015-Q10}
	A \SI{0.0600}{\kilogram} ball traveling at \SI{60.0}{\meters\per\second} hits a concrete wall. 
	What speed must a \SI{0.0100}{\kilogram} bullet have in order to hit the wall with the same magnitude of momentum as the ball?
    \begin{multicols}{2}
    \begin{choices}
        \wrongchoice{\SI{3.60}{\meter\per\second}}
        \wrongchoice{\SI{6.00}{\meter\per\second}}
      \correctchoice{\SI{360}{\meter\per\second}}
        \wrongchoice{\SI{600}{\meter\per\second}}
    \end{choices}
    \end{multicols}
\end{question}
}

\element{nysed}{
\begin{question}{June2015-Q11}
    The Hubble telescope’s orbit is 5.6 × 105 meters above Earth’s surface. 
    The telescope has a mass of 1.1 × 104 kilograms. 
    Earth exerts a gravitational force of 9.1 × 104 newtons on the telescope. 
    The magnitude of Earth’s gravitational field strength at this location is
    \begin{multicols}{2}
    \begin{choices}
        \wrongchoice{1.5 × 10−20 N/kg}
        \wrongchoice{0.12 N/kg}
      \correctchoice{8.3 N/kg}
        \wrongchoice{9.8 N/kg}
    \end{choices}
    \end{multicols}
\end{question}
}

\element{nysed}{
\begin{question}{June2015-Q12}
    When two point charges are a distance d apart, the magnitude of the electrostatic force between them is F. 
    If the distance between the point charges is increased to 3d, the magnitude of the electrostatic force between the two charges will be
    \begin{multicols}{2}
    \begin{choices}
      \correctchoice{1/9 F}
        \wrongchoice{2 F}
        \wrongchoice{1/3F}
        \wrongchoice{4F}
    \end{choices}
    \end{multicols}
\end{question}
}

\element{nysed}{
\begin{question}{June2015-Q13}
    A radio operating at 3.0 volts and a constant temperature draws a current of 1.8 × 10−4 ampere.
    What is the resistance of the radio circuit?
    \begin{multicols}{2}
    \begin{choices}
       \correctchoice{1.7 × 104 Ω}
         \wrongchoice{3.0e1 \ohm}
         \wrongchoice{5.4 × 10−4 Ω}
         \wrongchoice{6.0}
    \end{choices}
    \end{multicols}
\end{question}
}


\element{nysed}{
\begin{question}{June2015-Q14}
    Which energy transformation occurs in an operating electric motor?
    \begin{multicols}{2}
    \begin{choices}
       \correctchoice{electrical → mechanical}
         \wrongchoice{mechanical → electrical}
         \wrongchoice{chemical → electrical}
         \wrongchoice{electrical → chemical}
    \end{choices}
    \end{multicols}
\end{question}
}


\element{nysed}{
\begin{question}{June2015-Q15}
    A block slides across a rough, horizontal tabletop.
    As the block comes to rest, there is an increase in the block-tabletop system’s
    \begin{multicols}{2}
    \begin{choices}
         \wrongchoice{gravitational potential energy}
         \wrongchoice{elastic potential energy}
         \wrongchoice{kinetic energy}
       \correctchoice{internal (thermal) energy}
    \end{choices}
    \end{multicols}
\end{question}
}

\element{nysed}{
\begin{question}{June2015-Q16}
How much work is required to move an electron
through a potential difference of 3.00 volts?
    \begin{multicols}{2}
    \begin{choices}
         \wrongchoice{5.33 × 10–20 J}
       \correctchoice{4.80 × 10–19 J}
         \wrongchoice{3.00 J}
         \wrongchoice{1.88 × 1019 J}
    \end{choices}
    \end{multicols}
\end{question}
}

\element{nysed}{
\begin{question}{June2015-Q17}
    During a laboratory experiment, a student finds that at 20° Celsius, a 6.0-meter length of copper wire has a resistance of 1.3 ohms. 
    The cross-sectional area of this wire is
    \begin{multicols}{2}
    \begin{choices}
       \correctchoice{7.9 × 10−8 m2}
         \wrongchoice{1.1 × 10 m 2}
         \wrongchoice{4.6 × 100 m2 −7}
         \wrongchoice{1.3 × 107 m2}
    \end{choices}
    \end{multicols}
\end{question}
}

\element{nysed}{
\begin{question}{June2015-Q18}
    A net charge of 5.0 coulombs passes a point on a conductor in 0.050 second. 
    The average current is
    \begin{multicols}{2}
    \begin{choices}
         \wrongchoice{8.0 × 10−8 A}
         \wrongchoice{1.0 x 10−2}
         \wrongchoice{2.5 × 10−1 A}
       \correctchoice{1.0 × 102 A}
    \end{choices}
    \end{multicols}
\end{question}
}

\element{nysed}{
\begin{question}{June2015-Q19}
    If several resistors are connected in series in an electric circuit, the potential difference across each resistor
    \begin{multicols}{2}
    \begin{choices}
       \correctchoice{varies directly with its resistance}
         \wrongchoice{varies inversely with its resistance}
         \wrongchoice{varies inversely with the square of its resistance}
         \wrongchoice{is independent of its resistance}
    \end{choices}
    \end{multicols}
\end{question}
}

\element{nysed}{
\begin{question}{June2015-Q20}
    The amplitude of a sound wave is most closely related to the sound’s
    \begin{multicols}{2}
    \begin{choices}
         \wrongchoice{speed}
         \wrongchoice{wavelength}
       \correctchoice{loudness}
         \wrongchoice{pitch}
    \end{choices}
    \end{multicols}
\end{question}
}

\element{nysed}{
\begin{question}{June2015-Q21}
    A duck floating on a lake oscillates up and down 5.0 times during a 10.-second interval as a periodic wave passes by. 
    What is the frequency of the duck’s oscillations?
    \begin{multicols}{2}
    \begin{choices}
         \wrongchoice{0.10 Hz}
       \correctchoice{0.50 Hz}
         \wrongchoice{2.0 Hz}
         \wrongchoice{50. Hz}
    \end{choices}
    \end{multicols}
\end{question}
}

\element{nysed}{
\begin{question}{June2015-Q22}
    Which diagram best represents the position of a ball,
        at equal time intervals,
        as it falls freely from rest near Earth's surface?
    \begin{multicols}{2}
    \begin{choices}
        %% NOTE: make ticker tape graph
        \wrongchoice{
            \begin{tikzpicture}
                \draw[white] (0,0) rectangle (2,4);
                \draw[domain=0:10,samples=4,mark=*,only marks] plot ({1.0}, {0.02*\x*\x});
                \draw (0,0) -- (2,0) node[pos=0.5,anchor=north] {Earth's Surface};
            \end{tikzpicture}
        }
        \wrongchoice{
            \begin{tikzpicture}
                \draw[white] (0,0) rectangle (1,-4);
                \draw[domain=0:10,samples=4,mark=*,only marks] plot ({1.0}, {-0.02*\x*\x});
                \draw (0,-4) -- (2,-4) node[pos=0.5,anchor=north] {Earth's Surface};
            \end{tikzpicture}
        }
        \wrongchoice{
            \begin{tikzpicture}
                \draw[white] (0,0) rectangle (1,-4);
                \draw[domain=0:10,samples=4,mark=*,only marks] plot ({1.0}, {0.2*\x});
                \draw (0,0) -- (2,0) node[pos=0.5,anchor=north] {Earth's Surface};
            \end{tikzpicture}
        }
        \wrongchoice{
            \begin{tikzpicture}
                \draw[white] (0,0) rectangle (1,-4);
                \draw[domain=0:10,samples=4,mark=*,only marks] plot ({1.0}, {0.04*(10-\x)*\x});
                \draw (0,0) -- (2,0) node[pos=0.5,anchor=north] {Earth's Surface};
            \end{tikzpicture}
        }
    \end{choices}
    \end{multicols}
\end{question}
}

\element{nysed}{
\begin{question}{June2015-Q23}
   A gamma ray and a microwave traveling in a vacuum have the same
    \begin{multicols}{2}
    \begin{choices}
        \wrongchoice{frequency}
        \wrongchoice{period}
      \correctchoice{speed}
        \wrongchoice{wavelength}
    \end{choices}
    \end{multicols}
\end{question}
}

\element{nysed}{
\begin{question}{June2015-Q24}
    A student produces a wave in a long spring by vibrating its end. As the frequency of the vibration is doubled, the wavelength in the spring is
    \begin{multicols}{2}
    \begin{choices}
        \wrongchoice{quartered}
      \correctchoice{halved}
        \wrongchoice{unchanged}
        \wrongchoice{doubled}
    \end{choices}
    \end{multicols}
\end{question}
}

\element{nysed}{
\begin{question}{June2015-Q25}
   Which two points on the wave shown in the diagram below are in phase with each other?
    \begin{center}
        \begin{tikzpicture}
        \end{tikzpicture}
    \end{center}
    \begin{multicols}{2}
    \begin{choices}
        \wrongchoice{A and B}
      \correctchoice{A and E}
        \wrongchoice{B and C}
        \wrongchoice{B and D}
    \end{choices}
    \end{multicols}
\end{question}
}

\element{nysed}{
\begin{question}{June2015-Q26}
    As a longitudinal wave moves through a medium, the particles of the medium
    \begin{multicols}{2}
    \begin{choices}
      \correctchoice{vibrate parallel to the direction of the wave’s propagation}
        \wrongchoice{vibrate perpendicular to the direction of the wave’s propagation}
        \wrongchoice{are transferred in the direction of the wave’s motion, only}
        \wrongchoice{are stationary}
    \end{choices}
    \end{multicols}
\end{question}
}

\element{nysed}{
\begin{question}{June2015-Q27}
    Wind blowing across suspended power lines may cause the power lines to vibrate at their natural frequency. 
    This often produces audible sound waves. 
    This phenomenon, often called an Aeolian harp, is an example of
    \begin{multicols}{2}
    \begin{choices}
        \wrongchoice{diffraction}
        \wrongchoice{the Doppler effect}
        \wrongchoice{refraction}
      \correctchoice{resonance}
    \end{choices}
    \end{multicols}
\end{question}
}

\element{nysed}{
\begin{question}{June2015-Q28}
    A student listens to music from a speaker in an adjoining room, as represented in the diagram below
    \begin{center}
    \begin{tikzpicture}

    \end{tikzpicture}
    \end{center}
    She notices that she does not have to be directly in front of the doorway to hear the music.
    This spreading of sound waves beyond the doorway is an example of
    \begin{multicols}{2}
    \begin{choices}
        \wrongchoice{the Doppler effect}
        \wrongchoice{resonance}
        \wrongchoice{refraction}
      \correctchoice{diffraction}
    \end{choices}
    \end{multicols}
\end{question}
}

\element{nysed}{
\begin{question}{June2015-Q29}
   What is the minimum energy required to ionize a hydrogen atom in the $n=3$ state?
    \begin{multicols}{2}
    \begin{choices}
        \wrongchoice{0.00 eV}
        \wrongchoice{0.66 eV}
      \correctchoice{1.51 eV}
        \wrongchoice{12.09 eV}
    \end{choices}
    \end{multicols}
\end{question}
}



%% NOTE: TODO: finish next two questions
\newcommand{\JuneTwentyFifteenQThirty}{
    \begin{tikzpicture}
    \end{tikzpicture}
}

\element{nysed}{
    Base your answers to questions 30 and 31 on the diagram below and on your knowledge of physics. 
    The diagram represents two small, charged, identical metal spheres, A and B that are separated by a distance of 2.0 meters.
    The diagram represents two small, charged,
        identical metal spheres, $A$ and $B$ that are separated
        by a distance of \SI{2.0}{\meter}.
    \begin{center}
        \JuneTwentyFifteenQThirty
    \end{center}
\begin{question}{June2015-Q30}
    What is the magnitude of the electrostatic force exerted by sphere A on sphere B?
    \begin{multicols}{2}
    \begin{choices}
        \wrongchoice{\SI{7.2e-3}{\newton}}
        \wrongchoice{\SI{3.6e-3}{\newton}}
        \wrongchoice{\SI{8.0e-13}{\newton}}
        \wrongchoice{\SI{4.0e-13}{\newton}}
    \end{choices}
    \end{multicols}
\end{question}
\begin{question}{June2015-Q31}
    The diagram represents two small, charged,
        identical metal spheres, $A$ and $B$ that are separated
        by a distance of \SI{2.0}{\meter}.
    \begin{center}
        \JuneTwentyFifteenQThirty
    \end{center}
    If the two spheres were touched together and then separated,
        the charge on sphere $A$ would be
    \begin{multicols}{2}
    \begin{choices}
        \wrongchoice{\SI{-3.0e-7}{\coulomb}}
        \wrongchoice{\SI{-6.0e-6}{\coulomb}}
        \wrongchoice{\SI{-1.3e-6}{\coulomb}}
        \wrongchoice{\SI{-2.6e-6}{\coulomb}}
    \end{choices}
    \end{multicols}
\end{question}
}

\element{nysed}{
\begin{question}{June2015-Q32}
    The horn of a moving vehicle produces a sound of constant frequency. 
    Two stationary observers, $A$ and $C$,
    and the vehicle's driver, $B$, positioned as represented
        in the diagram below, hear the sound of the horn.
    \begin{center}
        %% NOTE: insert graphic
        %\includegraphics[width=0.9\columnwidth,keepaspectratio]{June2015-Q32}
    \end{center}
    Compared to the frequency of the sound of the horn heard by driver $B$,
        the frequency heard by observer $A$ is:
    \begin{choices}
        \wrongchoice{lower and the frequency heard by observer $C$ is lower}
        \wrongchoice{lower and the frequency heard by observer $C$ is higher}
        \wrongchoice{higher and the frequency heard by observer $C$ is lower}
        \wrongchoice{higher and the frequency heard by observer $C$ is higher}
    \end{choices}
\end{question}
}

\element{nysed}{
\begin{question}{June2015-Q33}
    A different force is applied to each of four different blocks on a frictionless, horizontal surface. 
    In which diagram does the block have the greatest inertia 2.0 seconds after starting from rest?
    \begin{center}
    \end{center}
    Which ray best represents the path of the reflected light ray?
    \begin{multicols}{4}
    \begin{choices}[o]
        \wrongchoice{
	    \begin{tikzpicture}
		%% NOTE:
	    \end{tikzpicture}
        }
        \wrongchoice{
	    \begin{tikzpicture}
		%% NOTE:
	    \end{tikzpicture}
        }
        \wrongchoice{
	    \begin{tikzpicture}
		%% NOTE:
	    \end{tikzpicture}
        }
        \correctchoice{
	    \begin{tikzpicture}
		%% NOTE:
	    \end{tikzpicture}
        }
    \end{choices}
    \end{multicols}
\end{question}
}

\element{nysed}{
\begin{question}{June2015-Q34}
    The diagram below shows a ray of monochromatic light
        incident on a boundary between air and glass.
    \begin{center}
    \begin{tikzpicture}
        %% NOTE:
    \end{tikzpicture}
    \end{center}
    Which ray best represents the path of the reflected light ray?
    \begin{multicols}{4}
    \begin{choices}[o]
      \correctchoice{$A}
        \wrongchoice{$B}
        \wrongchoice{$C}
        \wrongchoice{$D}
    \end{choices}
    \end{multicols}
\end{question}
}

\element{nysed}{
\begin{question}{June2015-Q35}
    Two pulses approach each other in the same medium. 
    The diagram below represents the displacements caused by each pulse.
    \begin{center}
    \begin{tikzpicture}
        %% NOTE:
    \end{tikzpicture}
    \end{center}
    Which diagram best represents the resultant displacement of the medium as the pulses pass through each other?
    \begin{multicols}{4}
    \begin{choices}[o]
        \wrongchoice{
	    \begin{tikzpicture}
		%% NOTE:
	    \end{tikzpicture}
        }
        \correctchoice{
	    \begin{tikzpicture}
		%% NOTE:
	    \end{tikzpicture}
        }
        \wrongchoice{
	    \begin{tikzpicture}
		%% NOTE:
	    \end{tikzpicture}
        }
        \wrongchoice{
	    \begin{tikzpicture}
		%% NOTE:
	    \end{tikzpicture}
        }
    \end{choices}
    \end{multicols}
\end{question}
}


%%
%%  Part B-1
%%

\element{nysed}{
\begin{question}{June2015-Q36}
     The diameter of an automobile tire is closest to
    \begin{multicols}{4}
    \begin{choices}[o]
        \wrongchoice{\SI{e-2}{\meter}}
      \correctchoice{\SI{e0}{\meter}}
        \wrongchoice{\SI{e1}{\meter}}
        \wrongchoice{\SI{e2}{\meter}}
    \end{choices}
    \end{multicols}
\end{question}
}

\element{nysed}{
\begin{question}{June2015-Q37}
    The vector diagram below represents the velocity of a car traveling 24 meters per second 35° east of north.
    \begin{center}
        \begin{tikzpicture}
        \end{tikzpicture}
    \end{center}
    What is the magnitude of the component of the car's velocity that is directed eastward?
    \begin{multicols}{4}
    \begin{choices}[o]
      \correctchoice{14 m/s}
        \wrongchoice{20. m/s}
        \wrongchoice{29 m/s}
        \wrongchoice{42 m/}
    \end{choices}
    \end{multicols}
\end{question}
}

\element{nysed}{
\begin{question}{June2015-Q38}
    Without air resistance, a kicked ball would reach a maximum height of 6.7 meters and land 38 meters away. 
    With air resistance, the ball would travel
    \begin{multicols}{4}
    \begin{choices}[o]
        \wrongchoice{6.7 m vertically and more than 38 m horizontally}
        \wrongchoice{38 m horizontally and less than 6.7 m vertically}
        \wrongchoice{more than 6.7 m vertically and less than 38 m horizontally}
      \correctchoice{less than 38 m horizontally and less than 6.7 m vertically}
    \end{choices}
    \end{multicols}
\end{question}
}

\element{nysed}{
\begin{question}{June2015-Q39}
    \begin{multicols}{4}
    \begin{choices}[o]
        \wrongchoice{6.7 m vertically and more than 38 m horizontally}
        \wrongchoice{38 m horizontally and less than 6.7 m vertically}
        \wrongchoice{more than 6.7 m vertically and less than 38 m horizontally}
      \correctchoice{less than 38 m horizontally and less than 6.7 m vertically}
    \end{choices}
    \end{multicols}
\end{question}
}

\element{nysed}{
\begin{question}{June2015-Q40}
    A car, initially traveling at 15 meters per second north, accelerates to 25 meters per second north in 4.0 seconds. 
    The magnitude of the average acceleration is
    \begin{multicols}{4}
    \begin{choices}[o]
      \correctchoice{2.5 m/s2}
        \wrongchoice{6.3 m/s}
        \wrongchoice{10. m/s2}
        \wrongchoice{20. m/s2}
    \end{choices}
    \end{multicols}
\end{question}
}

\element{nysed}{
\begin{question}{June2015-Q41}
   An object is in equilibrium. Which force vector diagram could represent the force(s) acting on the object?
    \begin{multicols}{4}
    \begin{choices}[o]
        \wrongchoice{
	  \begin{tikzpicture}
	  \end{tikzpicture}
        }
        \wrongchoice{
	  \begin{tikzpicture}
	  \end{tikzpicture}
        }
        \wrongchoice{
	  \begin{tikzpicture}
	  \end{tikzpicture}
        }
      \correctchoice{
	\begin{tikzpicture}
	\end{tikzpicture}
       }
    \end{choices}
    \end{multicols}
\end{question}
}

\element{nysed}{
\begin{question}{June2015-Q42}
    Which combination of fundamental units can be used to express the amount of work done on an object?
    \begin{multicols}{4}
    \begin{choices}[o]
        \wrongchoice{kg•m/s}
        \wrongchoice{kg•m/s2}
      \correctchoice{kg•m2/s2}
        \wrongchoice{kg•m2/s3}
    \end{choices}
    \end{choices}
    \end{multicols}
\end{question}
}

\element{nysed}{
\begin{question}{June2015-Q43}
    Which graph best represents the relationship between the potential energy stored in a spring and the change in the spring’s length from its equilibrium position?
    \begin{multicols}{4}
    \begin{choices}[o]
        \wrongchoice{kg•m/s}
        \wrongchoice{kg•m/s2}
      \correctchoice{kg•m2/s2}
        \wrongchoice{kg•m2/s3}
    \end{choices}
    \end{choices}
    \end{multicols}
\end{question}
}



\endinput

