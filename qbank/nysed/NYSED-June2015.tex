
%% XXXXXX Questions used on the
%% NYSED Physics Regents Examination
%%--------------------------------------------------

%% NOTE: electromagneticApplication?, 2dkinematics, electrostatics, Sound, light
%% Section June2015
%%--------------------
\element{nysed}{
\begin{question}{June2015-Q01}
    Which quantities are scalar?
    \begin{multicols}{2}
    \begin{choices}
      \correctchoice{speed and work}
        \wrongchoice{velocity and force}
        \wrongchoice{distance and acceleration}
        \wrongchoice{momentum and power}
    \end{choices}
    \end{multicols}
\end{question}
}

\element{nysed}{
\begin{question}{June2015-Q02}
   A \SI{3.00}{\kilogram} mass is thrown vertically upward with an initial speed of \SI{9.80}{\meter\per\second}.
   What is the maximum height this object will reach? [Neglect friction.]
    \begin{multicols}{2}
    \begin{choices}
        \wrongchoice{1.00 m}
      \correctchoice{4.90 m}
        \wrongchoice{9.80 m}
        \wrongchoice{19.6 m}
    \end{choices}
    \end{multicols}
\end{question}
}


\element{nysed}{
\begin{question}{June2015-Q03}
    An airplane traveling north at 220. meters per second encounters a 50.0-meters-per-second crosswind from west to east, as represented in the diagram below.
            \begin{tikzpicture}
                \draw[white] (0,0) rectangle (2,4);
                \draw[domain=0:10,samples=4,mark=*,only marks] plot ({1.0}, {0.02*\x*\x});
                \draw (0,0) -- (2,0) node[pos=0.5,anchor=north] {Earth's Surface};
            \end{tikzpicture}
    What is the resultant speed of the plane?
    \begin{multicols}{2}
    \begin{choices}
        \wrongchoice{170. m/s}
        \wrongchoice{214 m/s}
      \correctchoice{226 m/s}
        \wrongchoice{270. m/s}
    \end{choices}
    \end{multicols}
\end{question}
}

\element{nysed}{
\begin{question}{June2015-Q04}
    A \SI{160}{\kilogram} space vehicle is traveling along a straight line at a constant speed of \SI{800}{\meter\per\second}.
    The magnitude of the net force on the space vehicle is
    \begin{multicols}{2}
    \begin{choices}
      \correctchoice{0 N}
        \wrongchoice{1.60 10^2 N}
        \wrongchoice{8.00 × 102 N}
        \wrongchoice{1.28 × 105 N}
    \end{choices}
    \end{multicols}
\end{question}
}

\element{nysed}{
\begin{question}{June2015-Q05}
   A student throws a \SI{5.0}{\newton} ball straight up.
   What is the net force on the ball at its maximum height?
    \begin{multicols}{2}
    \begin{choices}
        \wrongchoice{0.0 N}
        \wrongchoice{5.0 N, up}
      \correctchoice{5.0 N, down}
        \wrongchoice{9.8 N, down}
    \end{choices}
    \end{multicols}
\end{question}
}

\element{nysed}{
\begin{question}{June2015-Q06}
	A vertical spring has a spring constant of \SI{100}{\newton\per\meter}.
	When an object is attached to the bottom of the spring, the spring changes from its unstretched length of \SI{0.50}{\meter} to a length of \SI{0.65}{\meter}.
	The magnitude of the weight of the attached object is
        \wrongchoice{1.1 N}
      \correctchoice{2 15 N}
        \wrongchoice{50. N}
        \wrongchoice{4 65 N}
    \begin{multicols}{2}
    \begin{choices}
    \end{choices}
    \end{multicols}
\end{question}
}


\element{nysed}{
\begin{question}{June2015-Q07}
	A \SI{1.5}{\kilogram} cart initially moves at \SI{2.0}{\meter\per\second}.
	It is brought to rest by a constant net force in \SI{0.30}{\second}.
	What is the magnitude of the net force?
    \begin{multicols}{2}
    \begin{choices}
        \wrongchoice{0.40 N}
        \wrongchoice{0.90 N}
      \correctchoice{10. N}
        \wrongchoice{15 N}
    \end{choices}
    \end{multicols}
\end{question}
}

\element{nysed}{
\begin{question}{June2015-Q08}
	Which characteristic of a light wave must increase as the light wave passes from glass into air?
    \begin{multicols}{2}
    \begin{choices}
        \wrongchoice{amplitude}
        \wrongchoice{frequency}
        \wrongchoice{period}
      \correctchoice{wavelength}
    \end{choices}
    \end{multicols}
\end{question}
}

\element{nysed}{
\begin{question}{June2015-Q09}
    As a 5.0 × 102-newton basketball player jumps from the floor up toward the basket,
	the magnitude of the force of her feet on the floor is 1.0 × 103 newtons. 
    As she jumps, the magnitude of the force of the floor on her feet is
    \begin{multicols}{2}
    \begin{choices}
        \wrongchoice{5.0 × 102 N}
      \correctchoice{1.0 × 103 N}
        \wrongchoice{1.5 × 103 N}
        \wrongchoice{5.0 × 105 N}
    \end{choices}
    \end{multicols}
\end{question}
}

\element{nysed}{
\begin{question}{June2015-Q10}
	A 0.0600-kilogram ball traveling at 60.0 meters
per second hits a concrete wall. What speed
must a 0.0100-kilogram bullet have in order to
hit the wall with the same magnitude of
momentum as the ball?
    \begin{multicols}{2}
    \begin{choices}
        \wrongchoice{3.60 m/s}
        \wrongchoice{6.00 m/s}
      \correctchoice{360. m/s}
        \wrongchoice{600. m/s}
    \end{choices}
    \end{multicols}
\end{question}
}

\element{nysed}{
\begin{question}{June2015-Q11}
    The Hubble telescope’s orbit is 5.6 × 105 meters above Earth’s surface. 
    The telescope has a mass of 1.1 × 104 kilograms. 
    Earth exerts a gravitational force of 9.1 × 104 newtons on the telescope. 
    The magnitude of Earth’s gravitational field strength at this location is
    \begin{multicols}{2}
    \begin{choices}
        \wrongchoice{1.5 × 10−20 N/kg}
        \wrongchoice{0.12 N/kg}
      \correctchoice{8.3 N/kg}
        \wrongchoice{9.8 N/kg}
    \end{choices}
    \end{multicols}
\end{question}
}

\element{nysed}{
\begin{question}{June2015-Q12}
    When two point charges are a distance d apart, the magnitude of the electrostatic force between them is F. 
    If the distance between the point charges is increased to 3d, the magnitude of the electrostatic force between the two charges will be
    \begin{multicols}{2}
    \begin{choices}
      \correctchoice{1/9 F}
        \wrongchoice{2 F}
        \wrongchoice{1/3F}
        \wrongchoice{4F}
    \end{choices}
    \end{multicols}
\end{question}
}

\element{nysed}{
\begin{question}{June2015-Q13}
    A radio operating at 3.0 volts and a constant temperature draws a current of 1.8 × 10−4 ampere.
    What is the resistance of the radio circuit?
    \begin{multicols}{2}
    \begin{choices}
       \correctchoice{1.7 × 104 Ω}
         \wrongchoice{3.0e1 \ohm}
         \wrongchoice{5.4 × 10−4 Ω}
         \wrongchoice{6.0}
    \end{choices}
    \end{multicols}
\end{question}
}

\element{nysed}{
\begin{question}{June2015-Q22}
    Which diagram best represents the position of a ball,
        at equal time intervals,
        as it falls freely from rest near Earth's surface?
    \begin{multicols}{2}
    \begin{choices}
        %% NOTE: make ticker tape graph
        \wrongchoice{
            \begin{tikzpicture}
                \draw[white] (0,0) rectangle (2,4);
                \draw[domain=0:10,samples=4,mark=*,only marks] plot ({1.0}, {0.02*\x*\x});
                \draw (0,0) -- (2,0) node[pos=0.5,anchor=north] {Earth's Surface};
            \end{tikzpicture}
        }
        \wrongchoice{
            \begin{tikzpicture}
                \draw[white] (0,0) rectangle (1,-4);
                \draw[domain=0:10,samples=4,mark=*,only marks] plot ({1.0}, {-0.02*\x*\x});
                \draw (0,-4) -- (2,-4) node[pos=0.5,anchor=north] {Earth's Surface};
            \end{tikzpicture}
        }
        \wrongchoice{
            \begin{tikzpicture}
                \draw[white] (0,0) rectangle (1,-4);
                \draw[domain=0:10,samples=4,mark=*,only marks] plot ({1.0}, {0.2*\x});
                \draw (0,0) -- (2,0) node[pos=0.5,anchor=north] {Earth's Surface};
            \end{tikzpicture}
        }
        \wrongchoice{
            \begin{tikzpicture}
                \draw[white] (0,0) rectangle (1,-4);
                \draw[domain=0:10,samples=4,mark=*,only marks] plot ({1.0}, {0.04*(10-\x)*\x});
                \draw (0,0) -- (2,0) node[pos=0.5,anchor=north] {Earth's Surface};
            \end{tikzpicture}
        }
    \end{choices}
    \end{multicols}
\end{question}
}

%% NOTE: TODO: finish next two questions
\newcommand{\JuneTwentyFifteenQThirty}{
    \begin{tikzpicture}
    \end{tikzpicture}
}

\element{nysed}{
\begin{question}{June2015-Q30}
    %Base your answers to questions 30 and 31 on the diagram
    %    below and on your knowledge of physics. 
    The diagram represents two small, charged,
        identical metal spheres, $A$ and $B$ that are separated
        by a distance of \SI{2.0}{\meter}.
    \begin{center}
        \JuneTwentyFifteenQThirty
    \end{center}
    What is the magnitude of the electrostatic force exerted by sphere A on sphere B?
    \begin{multicols}{2}
    \begin{choices}
        \wrongchoice{\SI{7.2e-3}{\newton}}
        \wrongchoice{\SI{3.6e-3}{\newton}}
        \wrongchoice{\SI{8.0e-13}{\newton}}
        \wrongchoice{\SI{4.0e-13}{\newton}}
    \end{choices}
    \end{multicols}
\end{question}
}

\element{nysed}{
\begin{question}{June2015-Q31}
    The diagram represents two small, charged,
        identical metal spheres, $A$ and $B$ that are separated
        by a distance of \SI{2.0}{\meter}.
    \begin{center}
        \JuneTwentyFifteenQThirty
    \end{center}
    If the two spheres were touched together and then separated,
        the charge on sphere $A$ would be
    \begin{multicols}{2}
    \begin{choices}
        \wrongchoice{\SI{-3.0e-7}{\coulomb}}
        \wrongchoice{\SI{-6.0e-6}{\coulomb}}
        \wrongchoice{\SI{-1.3e-6}{\coulomb}}
        \wrongchoice{\SI{-2.6e-6}{\coulomb}}
    \end{choices}
    \end{multicols}
\end{question}
}

\element{nysed}{
\begin{question}{June2015-Q32}
    The horn of a moving vehicle produces a sound of constant frequency. 
    Two stationary observers, $A$ and $C$,
    and the vehicle's driver, $B$, positioned as represented
        in the diagram below, hear the sound of the horn.
    \begin{center}
        %% NOTE: insert graphic
        %\includegraphics[width=0.9\columnwidth,keepaspectratio]{June2015-Q32}
    \end{center}
    Compared to the frequency of the sound of the horn heard by driver $B$,
        the frequency heard by observer $A$ is:
    \begin{choices}
        \wrongchoice{lower and the frequency heard by observer $C$ is lower}
        \wrongchoice{lower and the frequency heard by observer $C$ is higher}
        \wrongchoice{higher and the frequency heard by observer $C$ is lower}
        \wrongchoice{higher and the frequency heard by observer $C$ is higher}
    \end{choices}
\end{question}
}

\element{nysed}{
\begin{question}{June2015-Q34}
    The diagram below shows a ray of monochromatic light
        incident on a boundary between air and glass.
    \begin{center}
    \begin{tikzpicture}
        %% NOTE:
    \end{tikzpicture}
    \end{center}
    Which ray best represents the path of the reflected light ray?
    \begin{multicols}{4}
    \begin{choices}[o]
        \wrongchoice{$A
        \wrongchoice{$B
        \wrongchoice{$C
        \wrongchoice{$D
    \end{choices}
    \end{multicols}
\end{question}
}

\endinput

