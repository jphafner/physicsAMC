

%% Gravity
%%------------------------------------------------

\begin{question}[ID=gravity-A-Q01,topic=gravity,difficulty=A]
    Miranda orbits Uranus in \SI{34}{\hour}.
    What is Miranda's tangential velocity (orbital speed)?
\end{question}
\begin{solution}
    N/A
\end{solution}


\begin{question}[ID=gravity-A-Q01,topic=gravity,difficulty=A]
    Triton has a tangential velocity of \SI{4.39e3}{\meter\per\second}
        in its orbit around Neptune.
    What is the period of Triton's orbit in hours?
\end{question}
\begin{solution}
    \SI{141}{\hour}
\end{solution}


\begin{question}[ID=gravity-A-Q03,topic=gravity,difficulty=A]
    The space station orbits the Earth every \SI{90}{\minute}.
    What is the tangential velocity of the space station?
\end{question}
\begin{solution}
    N/A
\end{solution}


\begin{question}[ID=gravity-A-Q04,topic=gravity,difficulty=A]
    The first exosolar planet discovered, unofficially called Bellerophon,
        orbits nearby star 51 Pegasi with a tangential velocity
        of \SI{136}{\kilo\meter\per\second}.
    If the star has a mass of \num{1.11} times the mass of our sun,
        what is the period of the planet in days?
\end{question}
\begin{solution}
    \SI{4.26}{\day}
\end{solution}


\begin{question}[ID=gravity-B-Q01,topic=gravity,difficulty=B]
    A satellite with an orbital period of exactly
        \SI{24.0}{\hour} is always positioned over
        the same spot on Earth.
    At what distance above the surface of the Earth
        would a geosynchronous satellite orbit?
\end{question}
\begin{solution}
    N/A
\end{solution}


\begin{question}[ID=gravity-B-Q02,topic=gravity,difficulty=B]
    Use the data table on the back of this paper to calculate Neptune’s orbital period in Earth years.
\end{question}
\begin{solution}
    N/A
\end{solution}


\begin{question}[ID=gravity-B-Q03,topic=gravity,difficulty=B]
    If a \SI{1500}{\kilo\gram} satellite orbiting Saturn has a period
        of \SI{17.2}{\day}, what is the mean radius of the satellite's
        orbit?
\end{question}
\begin{solution}
    N/A
\end{solution}


\begin{question}[ID=gravity-B-Q04,topic=gravity,difficulty=B]
    What is the orbital radius of the Martian moon Deiomos if
        if orbits Mars in \SI{30.3}{\hour}?
\end{question}
\begin{solution}
    \SI{2.35e7}{\meter}
\end{solution}


\begin{question}[ID=gravity-B-Q05,topic=gravity,difficulty=B]
    The space station orbits the Earth every \SI{90}{\minute}.
    How high above the Earth's surface is the station's orbit?
\end{question}
\begin{solution}
    N/A
\end{solution}


\begin{question}[ID=gravity-C-Q01,topic=gravity,difficulty=C]
    At the sun's surface, the gravitational force between the
        sun and a \SI{5.00}{\kilo\gram} mass of hot gas has 
        a magnitude of \SI{1370}{\newton}.
    Assuming that the sun is spherical, what is the sun's
        mean radius?
\end{question}
\begin{solution}
    N/A
\end{solution}


\begin{question}[ID=gravity-C-Q02,topic=gravity,difficulty=C]
    How much would a \SI{65}{\kilo\gram} person weigh on Mars?
\end{question}
\begin{solution}
    N/A
\end{solution}


\begin{question}[ID=gravity-C-Q03,topic=gravity,difficulty=C]
    The gravitational force between Ganymede and Jupiter is
        \SI{1.636e22}{\newton}.
    The distance between the two bodies is \SI{1.071e6}{\kilo\meter}.
    What is Ganymede's mass?
\end{question}
\begin{solution}
    \SI{1.48e23}{\kilo\gram}
\end{solution}


\begin{question}[ID=gravity-C-Q04,topic=gravity,difficulty=C]
    At the surface of a red giant star, the gravitational force
        on \SI{1.00}{\kilo\gram} is only \SI{2.19e-3}{\newton}.
    If the red giant star has a mass of \SI{3.98e31}{\kilo\gram},
        what is its radius?
\end{question}
\begin{solution}
    \SI{1.10e12}{\meter}
\end{solution}


\endinput

