
%% Scientific Notation Problems
%%------------------------------------------------

\sisetup{
    retain-zero-exponent,
}

\begin{question}[ID=SN01,topic=units,difficulty=easy]
    Write the following using scientific notation
    \luaexec{
        beginTasks = [[\string\begin{tasks}(1)]]
        endTasks = [[\string\end{tasks}]]
        task = [[
            \string\task
                \string\begin{equation*}
                    \string\frac{(\string\num{\%.1e}) (\string\num{\%.1e})}
                                {(\string\num{\%.1e}) (\string\num{\%.1e})}
                \string\end{equation*}
        ]]
        local val1, val2, val3, val4 = 0,0,0,0
        tex.print(beginTasks)
        for i = 1,3 do 
            val1 = math.random(1,9) * math.pow( 10, math.random(-10,10) )
            val2 = math.random(1,9) * math.pow( 10, math.random(-10,10) )
            val3 = math.random(1,9) * math.pow( 10, math.random(-10,10) )
            val4 = math.random(1,9) * math.pow( 10, math.random(-10,10) )
            tex.print(string.format(task,val1,val2,val3,val4))
        end
        tex.print(endTasks)
    }
\end{question}
\begin{solution}
    use calculator
\end{solution}


%% Physical Quantities
%%------------------------------------------------

\begin{question}[ID=SN11-Sun,topic=units,difficulty=easy]
    The speed of light is \SI{3.00e8}{\meter\per\second}.
    If the sun is \SI{1.50e11}{\meter} from earth,
        how many seconds does it take light to reach the earth?
    Express your answer in scientific notation.
\end{question}
\begin{solution}
    \SI{5e2}{\second}
\end{solution}


\begin{question}[ID=SN12-Earth,topic=units,difficulty=easy]
    Given that the Earth's radius is \SI{6.3e6}{\meter} and
        its mass is \SI{5.97e24}{\kilo\gram}, calculate the
        average density of our planet?
    Express your answer using scientific notation.
    (Note: The volume of a sphere is $V=\frac{4}{3} \pi{} r^3$)
\end{question}
\begin{solution}
    \SI{5.7068e3}{\kilo\gram\per\meter\cubed}
\end{solution}


\begin{question}[ID=SN13-Saturn,topic=units,difficulty=easy]
    Given that Saturn's radius is \SI{5.85e7}{\meter} and
        its mass is \SI{5.68e26}{\kilo\gram}.
    Calculate the average density of Saturn.
    Express your answer using scientific notation.
    (Note: The volume of a sphere is $V=\frac{4}{3} \pi{} r^3$)
\end{question}
\begin{solution}
    \SI{712}{\kilo\gram\per\meter\cubed}
\end{solution}


\begin{question}[ID=SN14-MilkyWay,topic=units,difficulty=easy]
    What is the ratio of Milky Way radius to our solar system radius given that,
        the distance from pluto to the sun is \SI{5.9e12}{\meter}
        and the Milky Way disk radius is \SI{3.9e20}{\meter}.
    Round the coefficient to the nearest tenth using scientific notation.
\end{question}
\begin{solution}
    \num{0.0000000151} or \num{1.5e-8}
\end{solution}


\begin{question}[ID=SN16-Light,topic=units,difficulty=easy]
    The speed of light is now defined to be
        \SI{2.99792458e8}{\meter\per\second}.
    Express the speed of light using three, five and seven
        significant figures in scientific notation.
\end{question}
\begin{solution}
    \SI{3.00e8}{\meter\per\second},
    \SI{2.9979e8}{\meter\per\second},
    \SI{2.997925e8}{\meter\per\second},
\end{solution}

\endinput

