%% This defines the lua function mycommand()
%\directlua{require('./qbank/lua/common.lua')}


%% Numeric Difficulty A
%%------------------------------------------------

%\element{numeric}{
%\begin{questionmultx}{work-A-Q01}
%\luaexec{
%    %% Question
%    local Q = [[
%        A crate weighing \string\SI{\%d}{\string\newton}
%            sits at the bottom of an incline.
%        A worker pushes horizontally on the crate,
%            parallel to the ground with a force of
%            \string\SI{\%d}{\string\newton}.
%        The coefficient of friction between the crate and the floor is
%            \string\num{\%0.2f}.
%        What is the work done by the friction divided by
%            \string\SI{1}{\string\joule}?
%    ]]
%    %% Random Values
%    local w = math.random(80,150)
%    local f = math.random(50,150)
%    local c = round(math.random(),2)
%    local ans = 
%    %% Print Question
%    tex.print( string.format(Q, vel, tim) )
%    %tex.print( ans )
%    %% Print AMCnumeric
%    options = [[ digits=4, decimals=2, sign=True, approx=1 ]]
%    tex.print( BeginCenter )
%    tex.print( string.format(AMCnumeric, ans, options ) )
%    tex.print( EndCenter )
%}
%\end{questionmultx}
%}

\element{numeric}{
\begin{questionmultx}{work-A-Q02}
\luaexec{
    %% Question
    local Q = [[
        A shuffleboard player pushes a
            \string\SI{\%0.2f}{\string\kilo\string\gram}
            puck that is initially at rest such that a
            constant horizontal force of
            \string\SI{\%0.1f}{\string\newton}
            acts on it through a distance of
            \string\SI{\%0.1f}{\string\meter}.
        What is the speed of the puck when the force is removed
            divided by \string\SI{1}{\string\meter\string\per\string\second}?
        Ignore friction.
    ]]
    %% Random Values
    local m = 0.10 + round( math.random(),2 )
    local f = 1 + round( 8*math.random(),1 )
    local d = 1 + round( 8*math.random(),1 )
    local ans = math.sqrt( 2*f*d / m )
    %% Print Question
    tex.print( string.format(Q, m, f, d) )
    %tex.print( ans )
    %% Print AMCnumeric
    options = [[ digits=4, decimals=2, sign=True, approx=1 ]]
    tex.print( BeginCenter )
    tex.print( string.format(AMCnumeric, ans, options ) )
    tex.print( EndCenter )
}
\end{questionmultx}
}

\element{numeric}{
\begin{questionmultx}{work-A-Q03}
\luaexec{
    %% Question
    local Q = [[
        A shuffleboard player pushes a
            \string\SI{\%0.2f}{\string\kilo\string\gram}
            puck that is initially at rest such that a
            constant horizontal force of
            \string\SI{\%0.1f}{\string\newton}
            acts on it through a distance of
            \string\SI{\%0.1f}{\string\meter}.
        How much work would be required to bring the puck
            back to rest divided by \string\SI{1}{\string\joule}?
        Ignore friction.
    ]]
    %% Random Values
    local m = 0.10 + round( math.random(),2 )
    local f = 1 + round( 8*math.random(),1 )
    local d = 1 + round( 8*math.random(),1 )
    local ans = f * d
    %% Print Question
    tex.print( string.format(Q, m, f, d) )
    %tex.print( ans )
    %% Print AMCnumeric
    options = [[ digits=4, decimals=2, sign=True, approx=1 ]]
    tex.print( BeginCenter )
    tex.print( string.format(AMCnumeric, ans, options ) )
    tex.print( EndCenter )
}
\end{questionmultx}
}


\element{numeric}{
\begin{questionmultx}{work-A-Q04}
\luaexec{
    %% Question
    local Q = [[
        A \string\SI{\%d}{\string\kilo\string\gram}
            bicyclist rides his
            \string\SI{\%d}{\string\kilo\string\gram}
            bicycle with a speed of
            \string\SI{\%d}{\string\meter\string\per\string\second}.
        How much work must be done by the brakes to bring the bike
            and rider to a stop divided by
            \string\SI{1}{\string\joule}?
    ]]
    %% Random Values
    local m1 = math.random(40,150)
    local m2 = math.random(5,25)
    local v = math.random(5,20)
    local ans = 0.5 * (m1+m2) * v*v
    %% Print Question
    tex.print( string.format(Q, m1, m2, v) )
    %tex.print( ans )
    %% Print AMCnumeric
    options = [[ digits=4, decimals=2, sign=True, approx=1 ]]
    tex.print( BeginCenter )
    tex.print( string.format(AMCnumeric, ans, options ) )
    tex.print( EndCenter )
}
\end{questionmultx}
}


\element{numeric}{
\begin{questionmultx}{work-B-Q01}
\luaexec{
    %% Question
    local Q = [[
        In \string\SI{\%0.3f}{\string\second},
            through a distance of \string\SI{\%0.2f}{\string\meter},
            a test pilot's speed decreases from
            \string\SI{\%0.1f}{\string\meter\string\per\string\second} to
            \string\SI{\%0.1f}{\string\meter\string\per\string\second}.
        If the pilot's mass is
            \string\SI{\%0.1f}{\string\kilo\string\gram}.
        How much work is done against his body divided by
            \string\SI{1}{\string\joule}?
    ]]
    %% Random Values
    local t = 0.1 + round( math.random(), 3)
    local d = 5.0 + round( 3*math.random(), 2)
    local v1 = 50.0 + round( 50*math.random(), 1)
    local v2 = v1 - round( 20*math.random(), 1)
    local m  = 50.0 + round( 50*math.random(), 1)
    local ans = 0.5 * (m) * ( (v1*v1) - (v2*v2) )
    %% Print Question
    tex.print( string.format(Q, t, d, v1, v2, m) )
    %% Print AMCnumeric
    options = [[ digits=4, decimals=2, sign=True, approx=1 ]]
    tex.print( BeginCenter )
    tex.print( string.format(AMCnumeric, ans, options ) )
    tex.print( EndCenter )
}
\end{questionmultx}
}


\begin{comment}
\begin{question}[ID=work-B-Q04,topic=work,difficulty=B]
    A flight attendant pulls her \SI{70}{\newton} flight bag
        a distance of \SI{253}{\meter} along a level airport floor
        at a constant velocity.
    The force she exerts is \SI{40.0}{\newton} at an angle
        \ang{52} above the horizontal.
    Find the work done by the \emph{force of friction} on the flight bag.
\end{question}


\begin{question}[ID=work-B-Q05,topic=work,difficulty=B]
    A skier of mass \SI{70.0}{\kilo\gram} is pulled up a ski
        slope by a motor-driven cable.
    How much work is required to pul the skier \SI{60.0}{\meter}
        up a \ang{35} frictionless slope at a constant
        speed of \SI{2.0}{\meter\per\second}?
\end{question}


\begin{question}[ID=work-B-Q06,topic=work,difficulty=B]
    A stunt man crashes a car into a wall,
        starting at a speed of \SI{75}{\kilo\meter}.
    It takes \SI{0.212}{\second} for the vehicle to come to a stop.
    In that time, the car travels \SI{1.13}{\meter}.
    If the car has a mass of \SI{690}{\kilo\gram},
        how much work is done stopping the car?
\end{question}
\end{comment}


\endinput

