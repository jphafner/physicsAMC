

%% Centripetal
%%------------------------------------------------


\begin{question}[ID=rotational-A-Q01,topic=rotational-motion,difficulty=A]
    A car traveling at \SI{30.0}{\meter\per\second} undergoes
        a constant negative acceleration of magnitude
        \SI{2.00}{\meter\per\second\squared} when the brakes are applied.
    How many revolutions does each tire make before the car comes to a stop,
        assuming that the car does not skid and that the tires have radii
        of \SI{0.300}{\meter}?
\end{question}
\begin{solution}
    N/A
\end{solution}


\begin{question}[ID=rotational-A-Q02,topic=rotational-motion,difficulty=A]
    A \SI{2.00e3}{\kilo\gram} car rounds a circular turn of radius \SI{20.0}{\meter}.
    If the road is flat and the coefficient of friction between the tires
        and the road is \num{0.700}, how fast can the car go without skidding?
\end{question}
\begin{solution}
    N/A
\end{solution}


\begin{question}[ID=rotational-A-Q03,topic=rotational-motion,difficulty=A]
    A \SI{1.80e3}{\kilo\gram} car rounds a circular turn without skidding
        at a speed of \SI{26}{\meter\per\second}.
    If the road is flat and the coefficient of friction between the tires
        and the road is \num{0.8120}, what is the minimum radius of the turn?
\end{question}
\begin{solution}
    N/A
\end{solution}


\begin{question}[ID=rotational-B-Q01,topic=rotational-motion,difficulty=B]
    A copper block rests \SI{30.0}{\centi\meter} from the center of a steel turntable.
    The coefficient of static friction ($\mu_s$) between the block and
        the surface is \num{0.53}.
    The turntable starts from rest and rotates with a constant angular
        acceleration of \SI{0.50}{\radian\per\second\squared}.
    After what time interval will the block start to slip on the turntable?
\end{question}
\begin{solution}
    N/A
\end{solution}


\begin{question}[ID=rotational-B-Q02,topic=rotational-motion,difficulty=B]
    A bicycle slows down uniformly from $v_i = \SI{8.40}{\meter\per\second}$
        to rest over a distance of \SI{115}{\meter}.
    Each wheel has an outside diameter of \SI{68.0}{\centi\meter}.
    Determine:
    \begin{enumerate*}[label=\arabic*)]
        \item the initial angular velocity of the wheel
        \item the total number of revolutions each wheel rotates before coming to a stop
        \item the angular acceleration of the wheel
    \end{enumerate*}
\end{question}
\begin{solution}
    N/A
\end{solution}


\begin{question}[ID=rotational-B-Q03,topic=rotational-motion,difficulty=B]
    A wheel \SI{33}{\centi\meter} in diameter accelerates uniformly from
        \SI{240}{rpm} to \SI{360}{rpm} in \SI{6.5}{\second}.
    How far will a point on the edge of the wheel have traveled in this time?
\end{question}
\begin{solution}
    N/A
\end{solution}


\begin{question}[ID=rotational-B-Q04,topic=rotational-motion,difficulty=B]
    A cooling fan is turned off when it is running at
        \SI{850}{rev\per\minute}.
    It turns \num{1500} revolutions before it comes to a stop.  
    \begin{enumerate*}[label=\arabic*)]
        \item What is the fan's angular acceleration, assuming it is constant?
        \item How long did it take the fan to come to a complete stop?
    \end{enumerate*}
\end{question}
\begin{solution}
    N/A
\end{solution}


\begin{question}[ID=rotational-C-Q01,topic=rotational-motion,difficulty=C]
    A carousel at the county fair has three seats next to each other.
    Closest in is a Chicken Seat, which is \SI{5.0}{\meter} from the center.
    The next seat is the Unicorn, which is \SI{8.0}{\meter} from the center.
    The farther from the center is the Horsey chair, which is \SI{12}{\meter} from the center.
    The ride makes a revolution once every \SI{15}{\second}.
    \begin{enumerate*}[label=\arabic*)]
        \item What is the tangential speed of the three seats?
        \item  What is the centripetal acceleration on each of the seats?
    \end{enumerate*}
\end{question}
\begin{solution}
    N/A
\end{solution}


\begin{question}[ID=rotational-C-Q02,topic=rotational-motion,difficulty=C]
    Find the angular acceleration of a spinning amusement-park ride
        that initially travels at \SI{0.36}{\radian\per\second}
        then accelerates to \SI{0.84}{\radian\per\second} during
        a \SI{0.72}{\second} time interval.
\end{question}
\begin{solution}
    N/A
\end{solution}


\begin{question}[ID=rotational-C-Q03,topic=rotational-motion,difficulty=C]
    What is the instantaneous angular speed of a spinning amusement-park
        ride that accelerates from \SI{0.50}{\radian\per\second}
        at a constant angular acceleration of \SI{0.20}{\radian\per\second\squared}
        for \SI{1.0}{\second}?
\end{question}
\begin{solution}
    N/A
\end{solution}


\begin{question}[ID=rotational-C-Q04,topic=rotational-motion,difficulty=C]
    If a spinning amusement-park ride has an angular speed of
        \SI{1.2}{\radian\per\second}, what is the centripetal acceleration of a person
        standing \SI{12}{\meter} from the center of the ride?
\end{question}
\begin{solution}
    N/A
\end{solution}

\endinput

