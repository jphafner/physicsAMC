
%%--------------------------------------------------
%% CPO: Multiple Choice Questions
%%--------------------------------------------------


%% Chapter 12: Relativity
%%--------------------------------------------------


%% Learning Objectives
%%--------------------------------------------------

%% Describe how matter and energy are interchangeable. 
%% Describe the fastest speed in the universe. 
%% Explain the concept of antimatter.
%% Describe some consequences of special relativity. 
%% Describe some consequences of general relativity. 
%% Explain what a black hole is.
%% Explain what the big bang is.


%% CPO Multiple Choice Questions
%%--------------------------------------------------
\element{cpo-mc}{
\begin{question}{cpo-ch12-q01}
    The energy released in a nuclear reaction comes from:
    \begin{choices}
      \correctchoice{the conversion of mass to energy.}
        \wrongchoice{breaking chemical bonds between atoms.}
        \wrongchoice{heat produced by the rapid motion of protons and electrons.}
        \wrongchoice{atoms rapidly changing phase from solid to liquid.}
    \end{choices}
\end{question}
}

\element{cpo-mc}{
\begin{questionmult}{ch12-Q02}
    The speed of light is:
    \begin{choices}
      \correctchoice{always constant.}
      \correctchoice{\SI{300 000 000}{\meter\per\second}.}
      \correctchoice{represented by the symbol $c$.}
    \end{choices}
\end{questionmult}
}

\element{cpo-mc}{
\begin{question}{cpo-ch12-q03}
    what happens when antimatter meets an equal amount of normal matter?
    \begin{choices}
      \correctchoice{The antimatter and matter are converted to pure energy.}
        \wrongchoice{The antimatter and matter are converted to radioactive material.}
        \wrongchoice{The big bang is created.}
        \wrongchoice{A black hole is created.}
    \end{choices}
\end{question}
}

\element{cpo-mc}{
\begin{question}{cpo-ch12-q04}
    Einstein's formula $E=mc^2$, tells us that:
    \begin{choices}
        \wrongchoice{the speed of light is a constant.}
      \correctchoice{matter and energy can be turned into each other.}
        \wrongchoice{chemical reactions release energy.}
        \wrongchoice{forces move at the speed of light.}
    \end{choices}
\end{question}
}

\element{cpo-mc}{
\begin{question}{cpo-ch12-q05}
    In Einstein's formula, $E=mc^2$, the symbol $E$ stands for:
    \begin{multicols}{2}
    \begin{choices}
        \wrongchoice{Einstein}
      \correctchoice{energy}
        \wrongchoice{exothermic}
        \wrongchoice{endothermic}
    \end{choices}
    \end{multicols}
\end{question}
}

\element{cpo-mc}{
\begin{question}{cpo-ch12-q06}
    The antimatter twin of an electron is:
    \begin{choices}
      \correctchoice{positively charged.}
        \wrongchoice{negatively charged.}
        \wrongchoice{neutral.}
        \wrongchoice{of any charge.}
    \end{choices}
\end{question}
}

\element{cpo-mc}{
\begin{question}{cpo-ch12-q07}
    If you converted \SI{1}{\kilo\gram} of matter completely into energy,
        how much energy would you create?
    \begin{multicols}{2}
    \begin{choices}
        \wrongchoice{\SI{3e8}{\joule}}
        \wrongchoice{\SI{1e16}{\joule}}
        \wrongchoice{\SI{3e16}{\joule}}
      \correctchoice{\SI{9e16}{\joule}}
    \end{choices}
    \end{multicols}
\end{question}
}

\element{cpo-mc}{
\begin{question}{cpo-ch12-q08}
    Which of the following travels the fastest in a vacuum?
    \begin{choices}
        \wrongchoice{Light}
        \wrongchoice{Gravity}
        \wrongchoice{Radio waves}
      \correctchoice{Light, Gravity and Radio waves all travel at the same speed}
    \end{choices}
\end{question}
}

\element{cpo-mc}{
\begin{question}{cpo-ch12-q09}
    When an object approaches the speed of light, its mass:
    \begin{choices}
      \correctchoice{increases}
        \wrongchoice{decreases}
        \wrongchoice{stays the same}
        \wrongchoice{not enough information given.}
    \end{choices}
\end{question}
}

\element{cpo-mc}{
\begin{question}{cpo-ch12-q10}
    Jupiter is \SI{7.8e8}{\kilo\meter} (\SI{780 000 000}{\kilo\meter}) from the sun.
    The speed of light is \SI{3e5}{\kilo\meter\per\second} (\SI{300 000}{\kilo\meter\per\second}).
    How long does it take sunlight to travel from the Sun to Jupiter?
    \begin{multicols}{2}
    \begin{choices}
        \wrongchoice{\SI{3.8e-4}{\second}}
        \wrongchoice{\SI{1.0}{\second}}
      \correctchoice{\SI{2.6e3}{\second}}
        \wrongchoice{\SI{2.3e14}{\second}}
    \end{choices}
    \end{multicols}
\end{question}
}

\element{cpo-mc}{
\begin{question}{cpo-ch12-q11}
    Jupiter is \SI{7.8e8}{\kilo\meter} (\SI{780 000 000}{\kilo\meter}) from the sun.
    The speed of light is \SI{3e5}{\kilo\meter\per\second} (\SI{300 000}{\kilo\meter\per\second}).
    If the sun were to vanish,
        how long would it take before an observer on Jupiter would see the sun go out?
    \begin{multicols}{2}
    \begin{choices}
        \wrongchoice{\SI{0.000 38}{\second}}
        \wrongchoice{\SI{1.0}{\second}}
      \correctchoice{\SI{2 600}{\second}}
        \wrongchoice{\SI{2.3e14}{\second}}
    \end{choices}
    \end{multicols}
\end{question}
}

\element{cpo-mc}{
\begin{question}{cpo-ch12-q12}
    Jupiter is \SI{7.8e8}{\kilo\meter} (\SI{780 000 000}{\kilo\meter}) from the sun.
    The speed of light is \SI{3e5}{\kilo\meter\per\second} (\SI{300 000}{\kilo\meter\per\second}).
    If the sun were to vanish,
        how long would it take before Jupiter would fly out of orbit because the gravitational force between Jupiter and the sun disappears?
    \begin{multicols}{2}
    \begin{choices}
        \wrongchoice{\SI{0.000 38}{\second}}
        \wrongchoice{\SI{1.0}{\second}}
      \correctchoice{\SI{2 600}{\second}}
        \wrongchoice{\SI{2.3e14}{\second}}
    \end{choices}
    \end{multicols}
\end{question}
}

\element{cpo-mc}{
\begin{question}{cpo-ch12-q13}
    According to the theory of special relativity, a moving clock:
    \begin{multicols}{2}
    \begin{choices}
        \wrongchoice{will not run}
        \wrongchoice{runs faster}
        \wrongchoice{runs in reverse}
      \correctchoice{runs slower}
    \end{choices}
    \end{multicols}
\end{question}
}

\element{cpo-mc}{
\begin{questionmult}{ch12-Q14}
    Black holes:
    \begin{choices}
      \correctchoice{have such strong gravity that no light can escape.}
        \wrongchoice{are observable by refracting telescopes.}
        \wrongchoice{are the cause of the big bang.}
    \end{choices}
\end{questionmult}
}

\element{cpo-mc}{
\begin{question}{cpo-ch12-q15}
    Time moves more slowly for an object in motion than for one at rest.
    This effect is known as:
    \begin{multicols}{2}
    \begin{choices}
        \wrongchoice{special relativity}
      \correctchoice{time dilation}
        \wrongchoice{big bang}
        \wrongchoice{reference frame}
    \end{choices}
    \end{multicols}
\end{question}
}

\element{cpo-mc}{
\begin{questionmult}{ch12-Q16}
    Which of the following are effects of special relativity?
    \begin{choices}
      \correctchoice{Time runs slower for moving objects.}
      \correctchoice{Mass increases as an object gets closer to the speed of light.}
      \correctchoice{Space gets smaller for an object moving near the speed of light.}
    \end{choices}
\end{questionmult}
}

\element{cpo-mc}{
\begin{question}{cpo-ch12-q17}
    An unexpected result of Michelson and Morley's experiment in measuring the speed of light is that the speed of light:
    \begin{choices}
      \correctchoice{is always the same, independent of your relative motion.}
        \wrongchoice{increases as it travels away from you.}
        \wrongchoice{decreases as it travels away from you.}
        \wrongchoice{would appear to be zero if you traveled alongside it at the same speed.}
    \end{choices}
\end{question}
}

\element{cpo-mc}{
\begin{question}{cpo-ch12-q18}
    A 25 year old astronaut leaves Earth and travels for 20 years (as measured on Earth) at close to the speed of light.
    The biological age of the astronaut when he returned to Earth at the end of the 20 years would be:
    \begin{choices}
      \correctchoice{less than 45 years.}
        \wrongchoice{45 years.}
        \wrongchoice{greater than 45 years.}
        \wrongchoice{Not enough information given.}
    \end{choices}
\end{question}
}

\element{cpo-mc}{
\begin{question}{cpo-ch12-q19}
    Which of the following best describes what is ``relative'' in the theories of relativity?
    \begin{choices}
      \correctchoice{Measurement of time and motion change relative to their frame of reference.}
        \wrongchoice{Everything is relative.}
        \wrongchoice{The speed of light is relative.}
        \wrongchoice{Antimatter twins are relatives.}
    \end{choices}
\end{question}
}

\element{cpo-mc}{
\begin{question}{cpo-ch12-q20}
    What is the big bang?
    \begin{choices}
      \correctchoice{The hugely powerful explosion that created the universe.}
        \wrongchoice{The center of the Milky Way Galaxy.}
        \wrongchoice{Matter that travels faster than the speed of light.}
        \wrongchoice{The nuclear reaction produced by the sun.}
    \end{choices}
\end{question}
}

\element{cpo-mc}{
\begin{question}{cpo-ch12-q21}
    Because the universe is expanding, we can infer that it must have been \rule[-0.1pt]{4em}{0.1pt} at one time.
    \begin{multicols}{2}
    \begin{choices}
        \wrongchoice{slower}
      \correctchoice{smaller}
        \wrongchoice{larger}
        \wrongchoice{faster}
    \end{choices}
    \end{multicols}
\end{question}
}

\element{cpo-mc}{
\begin{question}{cpo-ch12-q22}
    Einstein's theory of general relativity describes gravity as an effect created by:
    \begin{choices}
        \wrongchoice{the Milky Way Galaxy.}
        \wrongchoice{a black hole.}
      \correctchoice{the curvature of space and time.}
        \wrongchoice{the big bang.}
    \end{choices}
\end{question}
}

\element{cpo-mc}{
\begin{question}{cpo-ch12-q23}
    According to Einstein's principle of equivalence,
        gravity is equivalent to:
    \begin{multicols}{2}
    \begin{choices}
        \wrongchoice{velocity}
      \correctchoice{acceleration}
        \wrongchoice{net force}
        \wrongchoice{energy}
    \end{choices}
    \end{multicols}
\end{question}
}

\element{cpo-mc}{
\begin{question}{cpo-ch12-q24}
    In a nuclear reactor, \SI{6.3e13}{\joule} of energy are produced in a fission reaction splitting uranium.
    The mass of uranium converted into energy is:
    \begin{multicols}{2}
    \begin{choices}
      \correctchoice{\SI{7e-4}{\kilo\gram}}
        \wrongchoice{\SI{1.0}{\kilo\gram}}
        \wrongchoice{\SI{2.1e5}{\kilo\gram}}
        \wrongchoice{\SI{1.89e22}{\kilo\gram}}
    \end{choices}
    \end{multicols}
\end{question}
}

\endinput

