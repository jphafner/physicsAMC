
%%--------------------------------------------------
%% CPO: Multiple Choice Questions
%%--------------------------------------------------


%% Chapter 7: Temperature, Energy and Matter
%%--------------------------------------------------


%% Learning Objectives
%%--------------------------------------------------

%% Describe the nature of matter at the atomic level. 
%% Explain how Brownian motion supports the theory that matter is made of tiny, indivisible particles. 
%% Distinguish between elements, compounds, and mixtures. 
%% Convert between temperature scales. 
%% Explain the relationship between temperature and the movement of particles in a system. 
%% Describe the relationship between temperature and states of matter. 
%% Explain the difference between temperature and thermal energy. 
%% Define heat. 
%% Calculate the heat required to raise the temperature of a material. 
%% Explain three methods of heat transfer and give examples of each. 
%% Give examples of thermal conductors and thermal insulators. 
%% Describe the direction of heat transfer between two objects


%% CPO Multiple Choice Questions
%%--------------------------------------------------
\element{cpo-mc}{
\begin{question}{cpo-ch07-q01}
    A substance with the same composition throughout containing two or more different kinds of elements chemically joined is know as a(n):
    \begin{multicols}{2}
    \begin{choices}
        \wrongchoice{atom}
        \wrongchoice{mixture}
      \correctchoice{compound}
        \wrongchoice{element}
    \end{choices}
    \end{multicols}
\end{question}
}

\element{cpo-mc}{
\begin{question}{cpo-ch07-q02}
    The smallest particle of an element that retains the chemical identity of the element is a(n):
    \begin{multicols}{2}
    \begin{choices}
        \wrongchoice{compound}
      \correctchoice{atom}
        \wrongchoice{molecule}
        \wrongchoice{mixture}
    \end{choices}
    \end{multicols}
\end{question}
}

\element{cpo-mc}{
\begin{question}{cpo-ch07-q03}
    A single atom has a diameter of about:
    \begin{multicols}{2}
    \begin{choices}
        \wrongchoice{\SI{e10}{\meter}}
        \wrongchoice{\SI[retain-zero-exponent]{e0}{\meter}}
      \correctchoice{\SI{e-10}{\meter}}
        \wrongchoice{\SI{e-1000}{\meter}}
    \end{choices}
    \end{multicols}
\end{question}
}

\element{cpo-mc}{
\begin{question}{cpo-ch07-q04}
    Of the following, the one that would be considered a mixture is:
    \begin{multicols}{2}
    \begin{choices}
      \correctchoice{fruit salad}
        \wrongchoice{oxygen gas}
        \wrongchoice{table salt (NaCl)}
        \wrongchoice{distilled water}
    \end{choices}
    \end{multicols}
\end{question}
}

\element{cpo-mc}{
\begin{question}{cpo-ch07-q05}
    The \emph{average} random kinetic energy of atoms or molecules within an object is a measure of the object's:
    \begin{multicols}{2}
    \begin{choices}
        \wrongchoice{heat}
      \correctchoice{temperature}
        \wrongchoice{thermal energy}
        \wrongchoice{specific heat}
    \end{choices}
    \end{multicols}
\end{question}
}

\element{cpo-mc}{
\begin{question}{cpo-ch07-q06}
    The temperature at which molecules in a substance have the lowest amount of energy possible is:
    \begin{multicols}{2}
    \begin{choices}
        \wrongchoice{\SI{0}{\degreeCelsius}}
        \wrongchoice{\SI{273}{\degreeCelsius}}
        \wrongchoice{\SI{-300}{\degree\Fahrenheit}}
      \correctchoice{\SI{0}{\kelvin}}
    \end{choices}
    \end{multicols}
\end{question}
}

\element{cpo-mc}{
\begin{question}{cpo-ch07-q07}
    Intermolecular forces between molecules of a substance are strongest when the material is a:
    \begin{multicols}{2}
    \begin{choices}
      \correctchoice{solid}
        \wrongchoice{liquid}
        \wrongchoice{gas}
        \wrongchoice{plasma}
    \end{choices}
    \end{multicols}
\end{question}
}

\element{cpo-mc}{
\begin{question}{cpo-ch07-q08}
    A temperature of \SI{11}{\degreeCelsius} is equivalent to:
    \begin{multicols}{2}
    \begin{choices}
        \wrongchoice{\SI{11}{\degree\Fahrenheit}}
        \wrongchoice{\SI{11}{\kelvin}}
        \wrongchoice{\SI{284}{\degree\Fahrenheit}}
      \correctchoice{\SI{284}{\kelvin}}
    \end{choices}
    \end{multicols}
\end{question}
}

\element{cpo-mc}{
\begin{question}{cpo-ch07-q09}
    A temperature of \SI{54}}{\degree\Fahrenheit} measures a temperature of about:
    \begin{multicols}{2}
    \begin{choices}
      \correctchoice{\SI{12}{\degreeCelsius}.}
        \wrongchoice{\SI{12}{\kelvin}}
        \wrongchoice{\SI{327}{\kelvin}}
        \wrongchoice{\SI{-219}{\kelvin}}
    \end{choices}
    \end{multicols}
\end{question}
}

\element{cpo-mc}{
\begin{question}{cpo-ch07-q10}
    A temperature of \SI{87}{\kelvin} is nearly equivalent to:
    \begin{multicols}{2}
    \begin{choices}
        \wrongchoice{\SI{87}{\degree\Fahrenheit}}
        \wrongchoice{\SI{360}{\degree\Fahrenheit}}
        \wrongchoice{\SI{360}{\degreeCelsius}}
        \wrongchoice{\SI{-303}{\degree\Fahrenheit}}
    \end{choices}
    \end{multicols}
\end{question}
}

\element{cpo-mc}{
\begin{question}{cpo-ch07-q11}
    The smallest change of temperature is represented by:
    \begin{multicols}{2}
    \begin{choices}
        \wrongchoice{\SI{1}{\degreeCelsius}}
        \wrongchoice{\SI{1}{\kelvin}}
      \correctchoice{\SI{1}{\degree\Fahrenheit}}
        \wrongchoice{either \SI{1}{\degreeCelsius} or \SI{1}{\kelvin}}
    \end{choices}
    \end{multicols}
\end{question}
}

\element{cpo-mc}{
\begin{question}{cpo-ch07-q12}
    The property that describes the amount of heat required to raise the temperature of \SI{1}{\kilo\gram} of a substance \SI{1}{\degreeCelsius} is its:
    \begin{choices}
        \wrongchoice{thermal conductivity}
      \correctchoice{specific heat}
        \wrongchoice{temperature variation}
        \wrongchoice{thermal energy}
    \end{choices}
\end{question}
}

\element{cpo-mc}{
\begin{question}{cpo-ch07-q13}
    According to the table below,
        the material requiring the most energy to raise its temperature from \SI{20}{\degreeCelsius} to \SI{40}{\degreeCelsius} is:
    \begin{center}
    %% NOTE: TODO: tabular siunit S 
    \begin{tabular}{rl}
        material    & specific heat/(\si{\joule\per\kilo\gram\per\degreeCelsius}) \\
        \midrule
        water       & 4184 \\
        aluminum    &  900 \\
        steel       &  470 \\
        silver      &  235 \\
        oil         & 1900 \\
        concrete    &  880 \\
        glass       &  800 \\
        gold        &  129 \\
        wood        & 2500 \\
    \end{tabular}
    \end{center}
    \begin{multicols}{2}
    \begin{choices}
      \correctchoice{water}
        \wrongchoice{gold}
        \wrongchoice{oil}
        \wrongchoice{wood}
    \end{choices}
    \end{multicols}
\end{question}
}

\element{cpo-mc}{
\begin{question}{cpo-ch07-q14}
    The flow of thermal energy is called:
    \begin{multicols}{2}
    \begin{choices}
        \wrongchoice{temperature}
        \wrongchoice{specific heat}
      \correctchoice{heat}
        \wrongchoice{thermal equilibrium}
    \end{choices}
    \end{multicols}
\end{question}
}

\element{cpo-mc}{
\begin{question}{cpo-ch07-q15}
    The \emph{sum} of all the kinetic energy of the atoms or molecules measures the object's:
    \begin{choices}
        \wrongchoice{specific heat}
        \wrongchoice{temperature}
      \correctchoice{thermal energy}
        \wrongchoice{thermal conductivity}
    \end{choices}
\end{question}
}

\element{cpo-mc}{
\begin{question}{cpo-ch07-q16}
    Of the following, the \emph{greatest} amount of thermal energy would be contained in:
    \begin{choices}
        \wrongchoice{an ice cube}
        \wrongchoice{a room full of air at \SI{100}{\degreeCelsius}}
        \wrongchoice{a cup of hot chocolate}
      \correctchoice{The North Atlantic Ocean}
    \end{choices}
\end{question}
}

\element{cpo-mc}{
\begin{question}{cpo-ch07-q17}
    The largest unit for measuring heat is the
    \begin{choices}
        \wrongchoice{degree Celsius (\si{\degreeCelsius}).}
        \wrongchoice{degree Fahrenheit (\si{\degree\Fahrenheit}).}
      \correctchoice{British thermal unit (btu).}
        \wrongchoice{calorie (cal)}
    \end{choices}
\end{question}
}

\element{cpo-mc}{
\begin{question}{cpo-ch07-q18}
    The specific heat of steel is \SI{470}{\joule\per\kilo\gram\per\degreeCelsius}.
    The amount of heat needed to raise the temperature of \SI{1.4}{\kilo\gram} of steel from \SI{12}{\degreeCelsius} to \SI{20}{\degreeCelsius} is:
    \begin{multicols}{2}
    \begin{choices}
        \wrongchoice{\SI{658}{\joule}}
      \correctchoice{\SI{5 260}{\joule}}
        \wrongchoice{\SI{7 900}{\joule}}
        \wrongchoice{\SI{13 200}{\joule}}
    \end{choices}
    \end{multicols}
\end{question}
}

\element{cpo-mc}{
\begin{question}{cpo-ch07-q19}
    The specific heat of oil is \SI{1 900}{\joule\per\kilo\gram\per\degreeCelsius}.
    If \SI{12 000}{\joule} of heat is added to \SI{2}{\kilo\gram} of oil at \SI{30}{\degreeCelsius} its temperature will become:
    \begin{multicols}{2}
    \begin{choices}
        \wrongchoice{\SI{0.11}{\degreeCelsius}}
        \wrongchoice{\SI{3.2}{\degreeCelsius}}
      \correctchoice{\SI{33.2}{\degreeCelsius}}
        \wrongchoice{\SI{95}{\degreeCelsius}}
    \end{choices}
    \end{multicols}
\end{question}
}

\element{cpo-mc}{
\begin{question}{cpo-ch07-q20}
    The graph below represents the temperature changes that occur as a gas changes from a temperature of \SI{200}{\degreeCelsius} to a solid at \SI{20}{\degreeCelsius}:
    \begin{center}
    \begin{tikzpicture}
        \begin{axis}[
            axis y line=left,
            axis x line=bottom,
            axis line style={->},
            ylabel={temperature},
            y unit=\si{\degreeCelsius},
            ytick={0,40,80,120,160,200},
            xlabel={time},
            x unit=\si{\minute},
            xtick={0,4,8,12,16,20,24,28,32,36},
            minor tick num=1,
            ymin=0,ymax=205,
            xmin=0,xmax=36.5,
            grid=both,
            width=0.8\columnwidth,
            height=0.5\columnwidth,
        ]
        \addplot[line width=1pt,domain=0:6]{200 - 10*x};
        \addplot[line width=1pt,domain=6:14]{140};
        \addplot[line width=1pt,domain=14:30]{140 - 5*(x-14)};
        \addplot[line width=1pt,domain=30:34]{60};
        \addplot[line width=1pt,domain=34:36]{60 - 20*(x-34)};
        \end{axis}
    \end{tikzpicture}
    \end{center}
    If the \SI{1.0}{\kilo\gram} sample of gas loses heat at a constant rate of \SI{2.0}{\joule\per\minute},
        the phase of the substance with the highest specific heat is:
    \begin{choices}
      \correctchoice{solid}
        \wrongchoice{liquid}
        \wrongchoice{gas}
        \wrongchoice{Cannot be determined from the graph.}
    \end{choices}
\end{question}
}

\element{cpo-mc}{
\begin{question}{cpo-ch07-q21}
    Airspaces between feathers of a down-filled coat causes the coat to be a good thermal:
    \begin{multicols}{2}
    \begin{choices}
        \wrongchoice{convector}
        \wrongchoice{conductor}
        \wrongchoice{radiator}
      \correctchoice{insulator}
    \end{choices}
    \end{multicols}
\end{question}
}

\element{cpo-mc}{
\begin{question}{cpo-ch07-q22}
    In nature, heat will always flow from a:
    \begin{choices}
        \wrongchoice{cold object to the warm object.}
        \wrongchoice{small object to the large object.}
      \correctchoice{warm object to the cold object.}
        \wrongchoice{large object to the small object.}
    \end{choices}
\end{question}
}

\element{cpo-mc}{
\begin{question}{cpo-ch07-q23}
    Heat energy from the sun is transferred to Earth by:
    \begin{multicols}{2}
    \begin{choices}
      \correctchoice{radiation}
        \wrongchoice{conduction}
        \wrongchoice{convection}
        \wrongchoice{insulation}
    \end{choices}
    \end{multicols}
\end{question}
}

\element{cpo-mc}{
\begin{question}{cpo-ch07-q24}
    Holding your hand above the flame of a candle,
        you will receive the most heat by means of:
    \begin{multicols}{2}
    \begin{choices}
        \wrongchoice{radiation}
        \wrongchoice{conduction}
      \correctchoice{convection}
        \wrongchoice{insulation}
    \end{choices}
    \end{multicols}
\end{question}
}

\element{cpo-mc}{
\begin{question}{cpo-ch07-q25}
    Materials that absorb radiation most effectively are also the best emitters of radiation.
    A wood stove for providing heat in a home will be most effective if it is:
    \begin{multicols}{2}
    \begin{choices}
        \wrongchoice{white}
      \correctchoice{black}
        \wrongchoice{silver}
        \wrongchoice{red}
    \end{choices}
    \end{multicols}
\end{question}
}

\element{cpo-mc}{
\begin{question}{cpo-ch07-q26}
    The difference you feel when holding metallic and foam containers filled with both hot liquid is caused mostly by the difference in the container's thermal:
    \begin{multicols}{2}
    \begin{choices}
        \wrongchoice{radiation}
      \correctchoice{conductivity}
        \wrongchoice{convection}
        \wrongchoice{energy}
    \end{choices}
    \end{multicols}
\end{question}
}


\endinput


