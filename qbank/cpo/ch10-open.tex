
%%--------------------------------------------------
%% CPO: AMC Open Free Response Questions
%%--------------------------------------------------


%% Chapter 10: Energy Flow and Systems
%%--------------------------------------------------


%% CPO Short Answer Questions
%%--------------------------------------------------
\element{cpo-short}{
\begin{question}{ch10-short-q01}
    While sitting next to a campfire, Anya noticed several different
        forms of energy being transformed from the potential
        chemical energy of the wood.
    Name two of the forms of energy she noticed.
    \AMCOpen{lines=3}{
        \wrongchoice[W]{w}\scoring{0}
        \wrongchoice[P]{p}\scoring{1}
        \correctchoice[C]{c}\scoring{2}
    }
    %% ANS: thermal, radiant, sound
\end{question}
}

\element{cpo-short}{
\begin{question}{ch10-short-q02}
    What is the basic source for all energy in the universe?
    Explain why?
    \AMCOpen{lines=3}{
        \wrongchoice[W]{w}\scoring{0}
        \wrongchoice[P]{p}\scoring{1}
        \correctchoice[C]{c}\scoring{2}
    }
\end{question}
}

\element{cpo-short}{
\begin{question}{ch10-short-q03}
    You and a friend carry bundles of shingles up a ladder to the roof.
    In \SI{10}{\minute}, your friend carries \num{10} bundles to the roof.
    You carry \num{10} bundles to the roof but take \SI{20}{\minute} to do it.
    How doe syou power compare to your friend's power?
    \AMCOpen{lines=3}{
        \wrongchoice[W]{w}\scoring{0}
        \wrongchoice[P]{p}\scoring{1}
        \correctchoice[C]{c}\scoring{2}
    }
    %% ANS: you are half as powerful
\end{question}
}


%% CPO Problem Questions
%%--------------------------------------------------
\element{cpo-problem}{
\begin{question}{ch10-problem-q01}
    Benjamin, who weights \SI{500}{\newton}, and Andrei, who weighs
        \SI{600}{\newton}, are standing at the bottom of a flight
        of stairs that is \SI{4}{\meter} high.
    Benjamin runs up the stairs in \SI{5}{\second}.
    Andrei runs up the same stairs in \SI{6}{\second}.
    Compare the work done and the power generated by each boy's run
        up the stairs.
    \emph{Support your answer by showing calculations for each quantity.}
    \begin{enumerate}
        \item Who does more work?
        \item Who is more powerful?
    \end{enumerate}
    \AMCOpen{lines=3}{
        \wrongchoice[W]{w}\scoring{0}
        \wrongchoice[P]{p}\scoring{1}
        \correctchoice[C]{c}\scoring{2}
    }
    %% ANS: Andrei does more work
    %% ANS: they generate the same power
\end{question}
}

\element{cpo-problem}{
\begin{question}{ch10-problem-q02}
    Gerry, who weighs \SI{500}{\newton}, is late for class.
    It takes her \SI{7.5}{\second} to run up two flights of stairs,
        a total of \SI{8.0}{\meter}.
    \begin{center}
        %% NOTE: add picture
    \end{center}
    How much power does Gerry expend in her run upstairs?
    \AMCOpen{lines=3}{
        \wrongchoice[W]{w}\scoring{0}
        \wrongchoice[P]{p}\scoring{1}
        \correctchoice[C]{c}\scoring{2}
    }
\end{question}
}

\element{cpo-problem}{
\begin{question}{ch10-problem-q03}
    An automobile uses \SI{75 000}{\joule} of chemical energy
        from gasoline to produce \SI{10 000}{\joule} of useful
        output energy.
    How much energy is lost to the system, and where did that
        energy go?
    \AMCOpen{lines=3}{
        \wrongchoice[W]{w}\scoring{0}
        \wrongchoice[P]{p}\scoring{1}
        \correctchoice[C]{c}\scoring{2}
    }
\end{question}
}

\element{cpo-problem}{
\begin{question}{ch10-problem-q04}
    Calculate the efficiency for a machine that requires \SI{400}{\joule}
        of input energy to produce \SI{300}{\joule} of useful
        output work.
    \AMCOpen{lines=3}{
        \wrongchoice[W]{w}\scoring{0}
        \wrongchoice[P]{p}\scoring{1}
        \correctchoice[C]{c}\scoring{2}
    }
\end{question}
}

\element{cpo-problem}{
\begin{question}{ch10-problem-q05}
    Brooke uses \SI{150}{\newton} of force to push a heavy box
        \SI{4}{\meter} in \SI{10}{\second}.
    How powerful is Brooke?
    \AMCOpen{lines=3}{
        \wrongchoice[W]{w}\scoring{0}
        \wrongchoice[P]{p}\scoring{1}
        \correctchoice[C]{c}\scoring{2}
    }
    %% ANS: \SI{60}{\watt}
\end{question}
}

\element{cpo-problem}{
\begin{question}{ch10-problem-q06}
    A baseball player stops a \SI{0.15}{\kilo\gram} baseball
        thrown at a speed of \SI{50}{\meter\per\second} in a
        distance of \SI{0.1}{\meter}.
    What force is required to do this?
    \AMCOpen{lines=3}{
        \wrongchoice[W]{w}\scoring{0}
        \wrongchoice[P]{p}\scoring{1}
        \correctchoice[C]{c}\scoring{2}
    }
    %% ANS: \SI{1 875}{\newton}
\end{question}
}

\element{cpo-problem}{
\begin{question}{ch10-problem-q07}
    A construction worker uses a rope and pulley system to lift
        \SI{2 000}{\newton} of lumber from the ground to a
        waiting helper on the second floor, \SI{4}{\meter} from
        the ground.
    To do this, she applies a \SI{200}{\newton} force on the rope
        of the rope and pulley.
    She pulls \SI{60}{\meter} of rope through the pulley before
        the load is lifted to the second floor.
    Based upon the information given, calculate the following
        for this rope and pulley:
    \begin{enumerate}
        \item work input
        \item work output
        \item efficiency
    \end{enumerate}
    \AMCOpen{lines=3}{
        \wrongchoice[W]{w}\scoring{0}
        \wrongchoice[P]{p}\scoring{1}
        \correctchoice[C]{c}\scoring{2}
    }
    %% ANS:
    %% A \SI{12 000}{\joule}
    %% B \SI{8 000}{\joule}
    %% C \SI{67}{\percent}
\end{question}
}

\element{cpo-problem}{
\begin{question}{ch10-problem-q08}
    \SI{1 000}{\joule} of work is put into a device that has
        two processes.
    The first process is \SI{90}{\percent} efficient, and the
        second process is \SI{50}{\percent} efficient.
    How much useful output work will be produced by this device?
    \begin{enumerate}
        \item work input
        \item work output
        \item efficiency
    \end{enumerate}
    \AMCOpen{lines=3}{
        \wrongchoice[W]{w}\scoring{0}
        \wrongchoice[P]{p}\scoring{1}
        \correctchoice[C]{c}\scoring{2}
    }
    %% ANS: \SI{450}{\joule}
\end{question}
}


%% CPO Essay Questions
%%--------------------------------------------------
\element{cpo-essay}{
\begin{question}{ch10-essay-q01}
    In a steady state food chain of living things, explain
        why you would find more herbivores than carnivores?
    \AMCOpen{lines=3}{
        \wrongchoice[W]{w}\scoring{0}
        \wrongchoice[P]{p}\scoring{1}
        \correctchoice[C]{c}\scoring{2}
    }
\end{question}
}

\element{cpo-essay}{
\begin{question}{ch10-essay-q02}
    Illustrate the water cycle using the terms \emph{condensation},
        \emph{evaporation}, \emph{groundwater}, \emph{rivers},
        and \emph{precipation}.
    whow the source of energy used to drive the water cycle.
    \AMCOpen{lines=3}{
        \wrongchoice[W]{w}\scoring{0}
        \wrongchoice[P]{p}\scoring{1}
        \correctchoice[C]{c}\scoring{2}
    }
\end{question}
}

\element{cpo-essay}{
\begin{question}{ch10-essay-q03}
    Use a diagram to show the flow of energy through a food chain.
    In the diagram, include the terms \emph{carnivore}, \emph{decomposer},
        \emph{herbivore}, \emph{producer}.
    \AMCOpen{lines=3}{
        \wrongchoice[W]{w}\scoring{0}
        \wrongchoice[P]{p}\scoring{1}
        \correctchoice[C]{c}\scoring{2}
    }
\end{question}
}

\endinput


