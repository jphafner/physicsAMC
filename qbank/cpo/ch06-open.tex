
%%--------------------------------------------------
%% CPO: AMC Open Free Response Questions
%%--------------------------------------------------


%% Chapter 6: Systems in Motion
%%--------------------------------------------------


%% CPO Short Answer Questions
%%--------------------------------------------------
\element{cpo-short}{
\begin{question}{ch06-short-q01}
    Name three factors that can affect the range of a projectile.
    \AMCOpen{lines=3}{
        \wrongchoice[W]{w}\scoring{0}
        \wrongchoice[P]{p}\scoring{1}
        \correctchoice[C]{c}\scoring{2}
    }
    %% ANS: velocity, angle, air resistance, gravity, initial height
\end{question}
}

\element{cpo-short}{
\begin{question}{ch06-short-q02}
    Of the following flying objects, list those that are projectiles:
    \begin{enumerate*}
        \item A jet plane just after take-off
        \item A baseball hit by a star baseball player
        \item A free-throw after being released by a basketball player
        \item A rock thrown from a bridge into a river
        \item A bird flying between trees
        \item A young woman diving from the starting block into a swimming pool
    \end{enumerate*}
    \AMCOpen{lines=3}{
        \wrongchoice[W]{w}\scoring{0}
        \wrongchoice[P]{p}\scoring{1}
        \correctchoice[C]{c}\scoring{2}
    }
    %% ANS: b,c,d,f
\end{question}
}

\element{cpo-short}{
\begin{question}{ch06-short-q03}
    Projectile motion may be separated into two components,
        vertical and horizontal.
    In addition to the difference in their directions,
        state another difference between these motion vectors.
    \AMCOpen{lines=3}{
        \wrongchoice[W]{w}\scoring{0}
        \wrongchoice[P]{p}\scoring{1}
        \correctchoice[C]{c}\scoring{2}
    }
    %% ANS: A>0 in vertical, A=0 in horizontal
\end{question}
}

\element{cpo-short}{
\begin{question}{ch06-short-q04}
    Malik and Destiny are seated on a moving merry-go-round.
    Malik is closer to the center than Destiny.
    Briefly describe the similarity or difference in their
        speeds in terms of angular  speed and linear speed.
    \AMCOpen{lines=3}{
        \wrongchoice[W]{w}\scoring{0}
        \wrongchoice[P]{p}\scoring{1}
        \correctchoice[C]{c}\scoring{2}
    }
\end{question}
}

\element{cpo-short}{
\begin{question}{ch06-short-q05}
    List examples of three objects that are revolving and
        three that are rotating.
    \AMCOpen{lines=3}{
        \wrongchoice[W]{w}\scoring{0}
        \wrongchoice[P]{p}\scoring{1}
        \correctchoice[C]{c}\scoring{2}
    }
\end{question}
}

\element{cpo-short}{
\begin{question}{ch06-short-q06}
    If the distance between two objects is doubled,
        what happens to the force of gravity between them?
    \AMCOpen{lines=3}{
        \wrongchoice[W]{w}\scoring{0}
        \wrongchoice[P]{p}\scoring{1}
        \correctchoice[C]{c}\scoring{2}
    }
\end{question}
}

\element{cpo-short}{
\begin{question}{ch06-short-q07}
    Briefly describe how you might lose weight simply by changing your location?
    \AMCOpen{lines=3}{
        \wrongchoice[W]{w}\scoring{0}
        \wrongchoice[P]{p}\scoring{1}
        \correctchoice[C]{c}\scoring{2}
    }
\end{question}
}

\element{cpo-short}{
\begin{question}{ch06-short-q08}
    Imagine you are twirling a ball tied to the end of a string
        in a counterclockwise direction above your head.
    Draw a circle to represent the path of the ball.
    On that circle draw arrows to represent the following:
    \begin{enumerate*}
        \item The direction of the speed of the ball; label it $V$.
        \item The direction of the acceleration of the ball; label it $A$.
        \item The direction of the centripetal force of the ball; label it $F$.
    \end{enumerate*}
    \AMCOpen{lines=3}{
        \wrongchoice[W]{w}\scoring{0}
        \wrongchoice[P]{p}\scoring{1}
        \correctchoice[C]{c}\scoring{2}
    }
\end{question}
}


%% CPO Problem Questions
%%--------------------------------------------------
\element{cpo-problem}{
\begin{question}{ch06-problem-q01}
    A bowling ball rolls off a horizontal loading platform \SI{2.0}{\meter}
        high with a speed of \SI{3.0}{\meter\per\second}.
    How long will it take to hit the ground.
    \AMCOpen{lines=3}{
        \wrongchoice[W]{w}\scoring{0}
        \wrongchoice[P]{p}\scoring{1}
        \correctchoice[C]{c}\scoring{2}
    }
    %% ANS: \SI{0.64}{\second}
\end{question}
}

\element{cpo-problem}{
\begin{question}{ch06-problem-q02}
    A quarterback throws a football, giving it a vertical velocity
        of \SI{9.2}{\meter\per\second} and a horizontal
        velocity of \SI{25.4}{\meter\per\second}.
    Calculate the vertical and horizontal components of the football's
        velocity after \SI{1.00}{\second}.
    \AMCOpen{lines=3}{
        \wrongchoice[W]{w}\scoring{0}
        \wrongchoice[P]{p}\scoring{1}
        \correctchoice[C]{c}\scoring{2}
    }
    %% ANS: v_y = \SI{0.6}{\meter\per\second}
    %% ANS: v_x = \SI{24.5}{\meter\per\second}
\end{question}
}

\element{cpo-problem}{
\begin{question}{ch06-problem-q03}
    A rocket is fired into the air with vertical velocity of
        \SI{58.8}{\meter\per\second} and a horizontal
        velocity of \SI{21.3}{\meter\per\second}.
    Assuming the rocket is in free-fall, calculate the
        range of the rocket.
    \AMCOpen{lines=3}{
        \wrongchoice[W]{w}\scoring{0}
        \wrongchoice[P]{p}\scoring{1}
        \correctchoice[C]{c}\scoring{2}
    }
    %% ANS: \SI{256}{\meter}
\end{question}
}

\element{cpo-problem}{
\begin{question}{ch06-problem-q04}
    Calculate the angular speed of a bicycle wheel that
        makes \num{240} rotations in \SI{6}{\minute}.
    \AMCOpen{lines=3}{
        \wrongchoice[W]{w}\scoring{0}
        \wrongchoice[P]{p}\scoring{1}
        \correctchoice[C]{c}\scoring{2}
    }
    %% ANS: 40 rpm
\end{question}
}

\element{cpo-problem}{
\begin{question}{ch06-problem-q05}
    Calculate the number of degrees a wheel has rotated when
        it has gone twice around.
    \AMCOpen{lines=3}{
        \wrongchoice[W]{w}\scoring{0}
        \wrongchoice[P]{p}\scoring{1}
        \correctchoice[C]{c}\scoring{2}
    }
    %% ANS: \ang{720}
\end{question}
}

\element{cpo-problem}{
\begin{question}{ch06-problem-q06}
    A barrel with a circumference of \SI{20}{\meter} rolls \SI{12}{\meter}
        in \SI{3}{\second}.
    Calculate the angular speed of the barrel.
    \AMCOpen{lines=3}{
        \wrongchoice[W]{w}\scoring{0}
        \wrongchoice[P]{p}\scoring{1}
        \correctchoice[C]{c}\scoring{2}
    }
    %% ANS: \SI{2}{rev\per\second}
\end{question}
}

\element{cpo-problem}{
\begin{question}{ch06-problem-q07}
    Use the information below to calculate the weight of a \SI{2 000}{\kilo\gram}
        car on Earth's surface.
    \begin{enumerate*}    
        \item $G$, the universal gravitational constant =
            \SI{6.67e-11}{\newton\meter\squared\per\kilo\gram\squared}
        \item Mass of Earth = \SI{5.97e24}{\kilo\gram}
        \item Radius of Earth = \SI{6.38e6}{\meter}
    \end{enumerate*}    
    \AMCOpen{lines=3}{
        \wrongchoice[W]{w}\scoring{0}
        \wrongchoice[P]{p}\scoring{1}
        \correctchoice[C]{c}\scoring{2}
    }
    %% ANS: \SI{1.96e4}{\newton}
\end{question}
}

\element{cpo-problem}{
\begin{question}{ch06-problem-q08}
    The centripetal force needed to cause a \SI{2 000}{\kilo\gram} truck to
        safely round a curve at \SI{15}{\meter\per\second} is
        \SI{13 000}{\newton}.
    If a \SI{2 000}{\kilo\gram} load is added to the truck but the
        driver cautiously slows to \SI{7.5}{\meter\per\second} in rounding
            the curve, how much force is required to safely negotiate the curve.
    \AMCOpen{lines=3}{
        \wrongchoice[W]{w}\scoring{0}
        \wrongchoice[P]{p}\scoring{1}
        \correctchoice[C]{c}\scoring{2}
    }
    %% ANS: \SI{6 500}{\newton}
\end{question}
}

\element{cpo-problem}{
\begin{question}{ch06-problem-q09}
    Two objects are placed near one another.
    The mass of one object is doubled while the distance
        between the objects is tripled.
    What net change takes place in the force of gravity between the objects?
    \AMCOpen{lines=3}{
        \wrongchoice[W]{w}\scoring{0}
        \wrongchoice[P]{p}\scoring{1}
        \correctchoice[C]{c}\scoring{2}
    }
    %% ANS: reduced by \frac{2}{9}
\end{question}
}


%% CPO Essay Questions
%%--------------------------------------------------
\element{cpo-essay}{
\begin{question}{ch06-essay-q01}
    Explain the statement, ``The horizontal and vertical components
        of a projectile's velocity are independent of each other.''
    \AMCOpen{lines=3}{
        \wrongchoice[W]{w}\scoring{0}
        \wrongchoice[P]{p}\scoring{1}
        \correctchoice[C]{c}\scoring{2}
    }
\end{question}
}

\element{cpo-essay}{
\begin{question}{ch06-essay-q02}
    Speedometers measure the speed of an automobile by measuring
        the angular speed of the wheels and converting it to linear
        speed.
    If Carlos customizes his car by using larger wheels than those used
        by the auto maker, this will affect the reading of his speedometer.
    Explain how \emph{and} why the speedometer's reading will be affected.
    \AMCOpen{lines=3}{
        \wrongchoice[W]{w}\scoring{0}
        \wrongchoice[P]{p}\scoring{1}
        \correctchoice[C]{c}\scoring{2}
    }
\end{question}
}

\element{cpo-essay}{
\begin{question}{ch06-essay-q03}
    Centripetal force must be applied to an automobile for it to
        follow a curved path.
    Explain why it is is a good idea to reduce speed when going
        around a sharp curve.
    In your discussion, mention the source of the centripetal force
        and the effect of both speed and the radius of the curve.
    \AMCOpen{lines=3}{
        \wrongchoice[W]{w}\scoring{0}
        \wrongchoice[P]{p}\scoring{1}
        \correctchoice[C]{c}\scoring{2}
    }
\end{question}
}

\element{cpo-essay}{
\begin{question}{ch06-essay-q04}
    Describe the relationship that exists for the force between
        two masses of uniform density and the distance between
        their centers.
    \AMCOpen{lines=3}{
        \wrongchoice[W]{w}\scoring{0}
        \wrongchoice[P]{p}\scoring{1}
        \correctchoice[C]{c}\scoring{2}
    }
\end{question}
}

\element{cpo-essay}{
\begin{question}{ch06-essay-q05}
    Describe how you would find the center of mass of an irregularly
        shaped piece of cardboard and how you could check to be sure
        your determination is correct.
    \AMCOpen{lines=3}{
        \wrongchoice[W]{w}\scoring{0}
        \wrongchoice[P]{p}\scoring{1}
        \correctchoice[C]{c}\scoring{2}
    }
\end{question}
}

\element{cpo-essay}{
\begin{question}{ch06-essay-q06}
    Explain why increasing the width between wheels of a car makes
        it more stable.
    \AMCOpen{lines=3}{
        \wrongchoice[W]{w}\scoring{0}
        \wrongchoice[P]{p}\scoring{1}
        \correctchoice[C]{c}\scoring{2}
    }
\end{question}
}

\element{cpo-essay}{
\begin{question}{ch06-essay-q07}
    Explain the difference between centripetal and centrifugal forces.
    \AMCOpen{lines=3}{
        \wrongchoice[W]{w}\scoring{0}
        \wrongchoice[P]{p}\scoring{1}
        \correctchoice[C]{c}\scoring{2}
    }
\end{question}
}

\element{cpo-essay}{
\begin{question}{ch06-essay-q08}
    Explain why in a skyscraper the center of gravity is lower than
        the center of mass, even though each level has the same mass.
    \AMCOpen{lines=3}{
        \wrongchoice[W]{w}\scoring{0}
        \wrongchoice[P]{p}\scoring{1}
        \correctchoice[C]{c}\scoring{2}
    }
\end{question}
}

\endinput


