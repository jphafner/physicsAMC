
%%--------------------------------------------------
%% CPO: Multiple Choice Questions
%%--------------------------------------------------


%% Chapter 8: Physical Properties of Matter
%%--------------------------------------------------


%% Learning Objectives
%%--------------------------------------------------

%% Distinguish between physical and chemical properties. 
%% Calculate the density of a substance. 
%% Calculate stress. 
%% Give examples of brittleness, elasticity, tensile strength, and thermal expansion. 
%% Explain why liquids are generally less dense than solids. 
%% Explain why solid water is less dense than liquid water. 
%% Describe the conditions necessary for an object to float. 
%% Explain what pressure is. 
%% Describe how energy conservation applies to fluids. 
%% Explain the meaning of viscosity. 
%% Describe the relationship between the pressure and volume of a gas.
%% Describe how changes in temperature affect gases.


%% CPO Multiple Choice Questions
%%--------------------------------------------------
\element{cpo-mc}{
\begin{question}{cpo-ch08-q01}
    Properties that can be seen through direct observation are called:
    \begin{multicols}{2}
    \begin{choices}
      \correctchoice{physical properties}
        \wrongchoice{chemical properties}
        \wrongchoice{chemical reactions}
        \wrongchoice{chemical equations}
    \end{choices}
    \end{multicols}
\end{question}
}

\element{cpo-mc}{
\begin{question}{cpo-ch08-q02}
    An example of a chemical property is the:
    \begin{choices}
        \wrongchoice{rough texture of sandpaper when touched.}
        \wrongchoice{tendency of water to exist as a solid at temperatures below \SI{0}{\degreeCelsius}.}
        \wrongchoice{color of a lemon.}
      \correctchoice{reaction of iron in the presence of oxygen.}
    \end{choices}
\end{question}
}

\element{cpo-mc}{
\begin{question}{cpo-ch08-q03}
    Change that is easily reversible is classified as \rule[-0.1pt]{4em}{0.1pt} change.
    \begin{multicols}{2}
    \begin{choices}
        \wrongchoice{chemical}
        \wrongchoice{nuclear}
      \correctchoice{physical}
        \wrongchoice{atomic}
    \end{choices}
    \end{multicols}
\end{question}
}

\element{cpo-mc}{
\begin{question}{cpo-ch08-q04}
    Density is calculated as:
    \begin{choices}
        \wrongchoice{mass times volume}
      \correctchoice{mass divided by volume}
        \wrongchoice{volume divided by mass}
        \wrongchoice{mass plus volume}
    \end{choices}
\end{question}
}

\element{cpo-mc}{
\begin{question}{cpo-ch08-q05}
    Density may be measured in units of:
    \begin{choices}
        \wrongchoice{kilogram meter (\si{\kilo\gram\meter})}
        \wrongchoice{kilogram per meter (\si{\kilo\gram\per\meter})}
        \wrongchoice{kilogram meter cubed (\si{\kilo\gram\meter\cubed})}
      \correctchoice{kilogram per meter cubed (\si{\kilo\gram\per\meter\cubed})}
    \end{choices}
\end{question}
}

\element{cpo-mc}{
\begin{question}{cpo-ch08-q06}
    The density of a material is dependent upon:
    \begin{choices}
        \wrongchoice{the mass of individual atoms of the material.}
        \wrongchoice{how the molecules of the material are ``packed''.}
      \correctchoice{\emph{both} the mass of the atoms and the ``packing'' of the molecules.}
        \wrongchoice{the product of the individual atomic masses and the space between molecules.}
    \end{choices}
\end{question}
}

\element{cpo-mc}{
\begin{question}{cpo-ch08-q07}
    Examples of amorphous solids include all of the follow \emph{except}:
    \begin{multicols}{2}
    \begin{choices}
      \correctchoice{salt}
        \wrongchoice{rubber}
        \wrongchoice{wax}
        \wrongchoice{glass}
    \end{choices}
    \end{multicols}
\end{question}
}

\element{cpo-mc}{
\begin{question}{cpo-ch08-q08}
    The ratio of the force acting through a material and the cross-section area through which the force is carried is known as:
    \begin{multicols}{2}
    \begin{choices}
        \wrongchoice{strength}
      \correctchoice{stress}
        \wrongchoice{strain}
        \wrongchoice{stretch}
    \end{choices}
    \end{multicols}
\end{question}
}

\element{cpo-mc}{
\begin{question}{cpo-ch08-q09}
    The metric unit for stress is the:
    \begin{multicols}{2}
    \begin{choices}
        \wrongchoice{newton (\si{\newton})}
        \wrongchoice{kilogram (\si{\kilo\gram})}
      \correctchoice{pascal (\si{\pascal})}
        \wrongchoice{square meter (\si{\meter\squared})}
    \end{choices}
    \end{multicols}
\end{question}
}

\element{cpo-mc}{
\begin{question}{cpo-ch08-q10}
    The amount of stress a material can withstand when stretched or subjected to tension is called:
    \begin{multicols}{2}
    \begin{choices}
        \wrongchoice{elasticity}
        \wrongchoice{brittleness}
      \correctchoice{tensile strength}
        \wrongchoice{deformation}
    \end{choices}
    \end{multicols}
\end{question}
}

\element{cpo-mc}{
\begin{question}{cpo-ch08-q11}
    The ability to be deformed and return to the original size and shape is called:
    \begin{multicols}{2}
    \begin{choices}
      \correctchoice{elasticity}
        \wrongchoice{flexibility}
        \wrongchoice{brittleness}
        \wrongchoice{tensile strength}
    \end{choices}
    \end{multicols}
\end{question}
}

\element{cpo-mc}{
\begin{question}{cpo-ch08-q12}
    Examples of physical changes include all of the following \emph{except}:
    \begin{choices}
        \wrongchoice{bending a metal bar}
        \wrongchoice{melting ice}
        \wrongchoice{dissolving sugar in water}
      \correctchoice{burning a match}
    \end{choices}
\end{question}
}

\element{cpo-mc}{
\begin{question}{cpo-ch08-q13}
    Referring to the chart below,
        the two objects or materials that have the same density are:
    \begin{center}
    %% NOTE: TODO: tabular siunitx S
    \begin{tabular}{rl}
        material    &   density [\si{\gram\per\centi\meter\cubed}] \\
        \midrule
        platinum    &   21500 \\
        lead        &   11300 \\
        steel       &   7800 \\
        titanium    &   4500 \\
        aluminum    &   2700 \\
        glass       &   2700 \\
        granite     &   2600 \\
        concrete    &   2300 \\
        plastic     &   2000 \\
        rubber      &   1200 \\
        liquid water&   1000 \\
        ice         &   920 \\
        oak (wood)  &   600 \\
        pine (wood) &   440 \\
        cork        &   120 \\
        air         &   0.9 \\
    \end{tabular}
    \end{center}
    \begin{choices}
        \wrongchoice{\SI{450}{\gram} of water and \SI{100}{\gram} of titanium.}
      \correctchoice{\SI{200}{\gram} of plastic golf tees and a plastic cube \SI{10}{\centi\meter} on edge.}
        \wrongchoice{\SI{10.0}{\gram} of rubber stoppers and \SI{20}{\centi\meter\cubed} of oak wood.}
        \wrongchoice{\SI{1.00}{\meter\cubed} of concrete and \SI{0.88}{\meter\cubed} of a granite statue.}
    \end{choices}
\end{question}
}

\element{cpo-mc}{
\begin{question}{cpo-ch08-q14}
    A titanium tube in a bicycle frame has a cross-section area of \SI{1.00e-4}{\meter\squared}.
    How much force would be required to break the tube if the rider ``crashes'' the bike?
    (tensile strength = \SI{9.00e8}{\pascal})
    \begin{multicols}{2}
    \begin{choices}
        \wrongchoice{\SI{1.11e-13}{\newton}}
      \correctchoice{\SI{9e4}{\newton}}
        \wrongchoice{\SI{9e6}{\newton}}
        \wrongchoice{\SI{9e10}{\newton}}
    \end{choices}
    \end{multicols}
\end{question}
}

\element{cpo-mc}{
\begin{question}{cpo-ch08-q15}
    A beachball floating in a swimming pool has a mass of \SI{0.30}{\kilo\gram} and a volume of \SI{0.15}{\meter\cubed}.
    The density of the floating ball is:
    \begin{multicols}{2}
    \begin{choices}
        \wrongchoice{\SI{0.045}{\kilo\gram\per\meter\cubed}}
        \wrongchoice{\SI{0.45}{\kilo\gram\per\meter\cubed}}
        \wrongchoice{\SI{0.50}{\kilo\gram\per\meter\cubed}}
      \correctchoice{\SI{2.0}{\kilo\gram\per\meter\cubed}}
    \end{choices}
    \end{multicols}
\end{question}
}

\element{cpo-mc}{
\begin{question}{cpo-ch08-q16}
    The density of air is \SI{0.9}{\kilo\gram\per\meter\cubed}.
    When a balloon is inflated with air,
        it has a volume of \SI{0.27}{\meter\cubed}.
    The mass of the air used to fill the balloon is:
    \begin{multicols}{2}
    \begin{choices}
      \correctchoice{\SI{0.2}{\kilo\gram}}
        \wrongchoice{\SI{0.3}{\kilo\gram}}
        \wrongchoice{\SI{1}{\kilo\gram}}
        \wrongchoice{\SI{3}{\kilo\gram}}
    \end{choices}
    \end{multicols}
\end{question}
}

\element{cpo-mc}{
\begin{question}{cpo-ch08-q17}
    The density of a diamond is \SI{3 500}{\kilo\gram\per\meter\cubed}.
    If a diamond with a mass of \SI{2}{\gram} is dropped into a graduated cylinder containing water,
        the water level in the graduated cylinder will rise:
    \begin{multicols}{2}
    \begin{choices}
      \correctchoice{\SI{0.57}{\centi\meter\cubed}}
        \wrongchoice{\SI{1.75}{\centi\meter\cubed}}
        \wrongchoice{\SI{3.50}{\centi\meter\cubed}}
        \wrongchoice{\SI{7.50}{\centi\meter\cubed}}
    \end{choices}
    \end{multicols}
\end{question}
}

\element{cpo-mc}{
\begin{question}{cpo-ch08-q18}
    A \SI{40}{\meter} long steel beam in bridge has coefficient of expansion of \SI{1.2e-5}{\per\degreeCelsius}.
    On a summer day, the temperature changes from \SI{10}{\degreeCelsius} to \SI{25}{\degreeCelsius}.
    The length of the beam will increase by:
    \begin{multicols}{2}
    \begin{choices}
        \wrongchoice{\SI{0.000 18}{\meter}.}
        \wrongchoice{\SI{0.000 48}{\meter}.}
      \correctchoice{\SI{0.0072}{\meter}.}
        \wrongchoice{\SI{40}{\meter}.}
    \end{choices}
    \end{multicols}
\end{question}
}

\element{cpo-mc}{
\begin{question}{cpo-ch08-q19}
    All of the following statements concerning fluid pressure are correct \emph{except}:
    \begin{choices}
        \wrongchoice{pressure exerts force on any surface touching a fluid.}
      \correctchoice{energy and pressure are not related.}
        \wrongchoice{pressure is derived from the collisions between atoms and molecules.}
        \wrongchoice{differences in pressure create potential energy.}
    \end{choices}
\end{question}
}

\element{cpo-mc}{
\begin{question}{cpo-ch08-q20}
    A scientific principle states that an upward force is exerted on an object in fluid equal to the weight of the fluid pushed aside by the object.
    The scientist credited for first recognizing this principle is:
    \begin{multicols}{2}
    \begin{choices}
      \correctchoice{Archimedes}
        \wrongchoice{Bernoulli}
        \wrongchoice{Newton}
        \wrongchoice{Galileo}
    \end{choices}
    \end{multicols}
\end{question}
}

\element{cpo-mc}{
\begin{question}{cpo-ch08-q21}
    The difficulty with which a fluid may be poured from a container is a measure of its:
    \begin{multicols}{2}
    \begin{choices}
        \wrongchoice{tensile strength}
        \wrongchoice{density}
      \correctchoice{viscosity}
        \wrongchoice{mass}
    \end{choices}
    \end{multicols}
\end{question}
}

\element{cpo-mc}{
\begin{question}{cpo-ch08-q22}
    The measure of the upward force applied by a fluid on an object is called:
    \begin{multicols}{2}
    \begin{choices}
        \wrongchoice{density}
      \correctchoice{buoyancy}
        \wrongchoice{weight}
        \wrongchoice{volume}
    \end{choices}
    \end{multicols}
\end{question}
}

\element{cpo-mc}{
\begin{question}{cpo-ch08-q23}
    Characteristics that are typical of \emph{most} fluids include all of the following \emph{except} fluids:
    \begin{choices}
      \correctchoice{are more dense than their solid phase.}
        \wrongchoice{are less rigidly organized than their solid phase.}
        \wrongchoice{exists at higher temperatures than their solid phase.}
        \wrongchoice{flow when a force is applied.}
    \end{choices}
\end{question}
}

\element{cpo-mc}{
\begin{question}{cpo-ch08-q24}
    Water is not typical of \emph{most} substances in its solid phase because:
    \begin{choices}
        \wrongchoice{it exists at lower temperature than its liquid phase.}
        \wrongchoice{molecules are more organized than its liquid phase.}
      \correctchoice{it is less dense than its liquid phase.}
        \wrongchoice{it has a crystalline structure.}
    \end{choices}
\end{question}
}

\element{cpo-mc}{
\begin{question}{cpo-ch08-q25}
    An object has a weight of \SI{2.5}{\newton} when suspended by a string attached to a spring scale.
    When the object, still suspended from the string,
        is held in a container of water without touching the sides or bottom of the container,
        the weight is \SI{1.5}{\newton}.
    The buoyant force of the water is:
    \begin{multicols}{2}
    \begin{choices}
      \correctchoice{\SI{1.0}{\newton}}
        \wrongchoice{\SI{1.7}{\newton}}
        \wrongchoice{\SI{3.8}{\newton}}
        \wrongchoice{\SI{4.0}{\newton}}
    \end{choices}
    \end{multicols}
\end{question}
}

\element{cpo-mc}{
\begin{question}{cpo-ch08-q26}
    All of the following statements about fluid pressure are correct \emph{except} fluid pressure:
    \begin{choices}
        \wrongchoice{is measure in pascals.}
        \wrongchoice{is the force per unit area.}
        \wrongchoice{is transmitted in all directions.}
      \correctchoice{in a container decreases with the depth of the fluid.}
    \end{choices}
\end{question}
}

\element{cpo-mc}{
\begin{question}{cpo-ch08-q27}
    Of the fluid characteristics named below,
        the one with the smallest overall effect on the energy of a fluid moving along a stream line is:
    \begin{multicols}{2}
    \begin{choices}
      \correctchoice{temperature}
        \wrongchoice{height}
        \wrongchoice{speed}
        \wrongchoice{pressure}
    \end{choices}
    \end{multicols}
\end{question}
}

\element{cpo-mc}{
\begin{question}{cpo-ch08-q28}
    The density of water is \SI{1 000}{\kilo\gram\per\meter\cubed}.
    An object with a volume of \SI{0.1}{\meter\cubed} submerged in water experiences a buoyant force of:
    \begin{multicols}{2}
    \begin{choices}
        \wrongchoice{\SI{100}{\newton}}
      \correctchoice{\SI{980}{\newton}}
        \wrongchoice{\SI{1 000}{\newton}}
        \wrongchoice{\SI{9 800}{\newton}}
    \end{choices}
    \end{multicols}
\end{question}
}

\element{cpo-mc}{
\begin{question}{cpo-ch08-q29}
    The density of water is \SI{1 000}{\kilo\gram\per\meter\cubed}.
    An pressure at the bottom of a swimming pool of water \SI{2.5}{\meter} deep is:
    \begin{multicols}{2}
    \begin{choices}
        \wrongchoice{\SI{400}{\newton\per\meter\squared}.}
        \wrongchoice{\SI{2 500}{\newton\per\meter\squared}.}
        \wrongchoice{\SI{9 800}{\newton\per\meter\squared}.}
      \correctchoice{\SI{24 500}{\newton\per\meter\squared}.}
    \end{choices}
    \end{multicols}
\end{question}
}

\element{cpo-mc}{
\begin{question}{cpo-ch08-q30}
    Tire pressure in an automobile tire is typically \SI{210 000}{\newton\per\meter\squared}.
    How much tire area must be in contact with road to support a \SI{2000}{\kilo\gram} vehicle.
    \begin{multicols}{2}
    \begin{choices}
        \wrongchoice{\SI{0.0095}{\meter\squared}}
      \correctchoice{\SI{0.093}{\meter\squared}}
        \wrongchoice{\SI{10.7}{\meter\squared}}
        \wrongchoice{\SI{105}{\meter\squared}}
    \end{choices}
    \end{multicols}
\end{question}
}

\element{cpo-mc}{
\begin{question}{cpo-ch08-q31}
    As the pressure on a gas at constant temperature increases, the volume decreases.
    This idea is stated in a scientific principle known as:
    \begin{multicols}{2}
    \begin{choices}
        \wrongchoice{Boyle's law}
      \correctchoice{Charles' law}
        \wrongchoice{Guy-Lussac's law}
        \wrongchoice{Kelvin's law}
    \end{choices}
    \end{multicols}
\end{question}
}

\element{cpo-mc}{
\begin{question}{cpo-ch08-q32}
    The gas that appears in greatest quantity in the atmosphere of Earth is:
    \begin{choices}
        \wrongchoice{argon (\ce{Ar})}
        \wrongchoice{carbon dioxide (\ce{CO2})}
      \correctchoice{nitrogen (\ce{N2})}
        \wrongchoice{oxygen (\ce{O2})}
    \end{choices}
\end{question}
}

\element{cpo-mc}{
\begin{question}{cpo-ch08-q33}
    ``The volume of a gas increases with increasing temperature'' is a statement of:
    \begin{multicols}{2}
    \begin{choices}
        \wrongchoice{Boyle's law}
      \correctchoice{Charles' law}
        \wrongchoice{Guy-Lussac's law}
        \wrongchoice{Kelvin's law}
    \end{choices}
    \end{multicols}
\end{question}
}

\element{cpo-mc}{
\begin{question}{cpo-ch08-q34}
    In the \SI{200}{\centi\meter\cubed} chamber of a bicycle pump the pressure is \SI{104}{\kilo\pascal}.
    As the handle is pushed down on the pump,
        the volume is reduced to \SI{50}{\centi\meter\cubed}.
    The pressure in the chamber of the pump is:
    \begin{multicols}{2}
    \begin{choices}
        \wrongchoice{\SI{26}{\kilo\pascal}}
        \wrongchoice{\SI{52}{\kilo\pascal}}
        \wrongchoice{\SI{208}{\kilo\pascal}}
      \correctchoice{\SI{416}{\kilo\pascal}}
    \end{choices}
    \end{multicols}
\end{question}
}

\element{cpo-mc}{
\begin{question}{cpo-ch08-q35}
    A balloon is filled with \SI{500}{\centi\meter\cubed} of air at \SI{27}{\degreeCelsius}.
    To what temperature must the air be changed to reduce the volume of the balloon to \SI{250}{\centi\meter\cubed}?
    \begin{multicols}{2}
    \begin{choices}
        \wrongchoice{\SI{13.5}{\degreeCelsius}}
        \wrongchoice{\SI{54}{\degreeCelsius}}
      \correctchoice{\SI{-123}{\degreeCelsius}}
        \wrongchoice{\SI{150}{\degreeCelsius}}
    \end{choices}
    \end{multicols}
\end{question}
}


\endinput


