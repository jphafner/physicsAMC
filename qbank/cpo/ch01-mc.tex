
%%--------------------------------------------------
%% CPO: Multiple Choice Questions
%%--------------------------------------------------


%% Chapter 1: Describing the Physical Universe
%%--------------------------------------------------


%% Learning Objectives
%%--------------------------------------------------

%% Explain what makes up the universe. 
%% Describe how the scientific method is used. 
%% Explain the effects of energy on a system. 
%% Express measurements in metric and English units. 
%% Distinguish between independent and dependent variables. 
%% Construct graphs. 
%% Convert between units of time. 
%% Define speed. 
%% Express an object’s speed using various units. 
%% Calculate speed, distance, or time given two of the three quantities. 
%% List the steps for solving physics problems.


%% CPO Multiple Choice Questions
%%--------------------------------------------------
\element{cpo-mc}{
\begin{question}{cpo-ch01-q01}
    All of these are very important parts of studying physics \emph{except}:
    \begin{choices}
        \wrongchoice{describing the organization of the universe.}
        \wrongchoice{understanding natural laws.}
      \correctchoice{memorizing complicated explanations.}
        \wrongchoice{deducing and applying natural laws.}
    \end{choices}
\end{question}
}

\element{cpo-mc}{
\begin{question}{cpo-ch01-q02}
    System variables can be observed and measured directly in a \rule[-0.1pt]{4em}{0.1pt} scale.
    \begin{multicols}{2}
    \begin{choices}
        \wrongchoice{time}
        \wrongchoice{microscopic}
      \correctchoice{macroscopic}
        \wrongchoice{mass}
    \end{choices}
    \end{multicols}
\end{question}
}

\element{cpo-mc}{
\begin{question}{cpo-ch01-q03}
    A carefully designed test done under controlled conditions
        is called a(n):
    \begin{multicols}{2}
    \begin{choices}
        \wrongchoice{natural law.}
      \correctchoice{experiment.}
        \wrongchoice{common law}
        \wrongchoice{analysis.}
    \end{choices}
    \end{multicols}
\end{question}
}

\element{cpo-mc}{
\begin{question}{cpo-ch01-q04}
    The mass of an object is determined by:
    \begin{choices}
      \correctchoice{finding the amount of matter it contains.}
        \wrongchoice{measuring its weight.}
        \wrongchoice{identifying the substance from which the object is made.}
        \wrongchoice{finding its dimensions.}
    \end{choices}
\end{question}
}

\element{cpo-mc}{
\begin{question}{cpo-ch01-q05}
    A factor that affects the behavior of a system is called a(n):
    \begin{multicols}{2}
    \begin{choices}
        \wrongchoice{natural law.}
        \wrongchoice{experiment.}
        \wrongchoice{analysis.}
      \correctchoice{variable.}
    \end{choices}
    \end{multicols}
\end{question}
}

\element{cpo-mc}{
\begin{question}{cpo-ch01-q06}
    The scientific method is a process used to solve many problems.
    One of the first steps is:
    \begin{choices}
        \wrongchoice{collecting data.}
        \wrongchoice{analyzing data.}
      \correctchoice{asking a question.}
        \wrongchoice{drawing a conclusion.}
    \end{choices}
\end{question}
}

\element{cpo-mc}{
\begin{question}{cpo-ch01-q07}
    A variable that remains unchanged throughout an experiment is called the:
    \begin{choices}
      \correctchoice{control variable.}
        \wrongchoice{experimental variable.}
        \wrongchoice{independent variable.}
        \wrongchoice{dependent variable.}
    \end{choices}
\end{question}
}

\element{cpo-mc}{
\begin{question}{cpo-ch01-q08}
    Compared to a laboratory cart at the top of a ramp,
        a cart at the bottom of the ramp has:
    \begin{choices}
        \wrongchoice{more energy and less stability.}
      \correctchoice{less energy and more stability.}
        \wrongchoice{more energy and more stability.}
        \wrongchoice{less energy and less stability.}
    \end{choices}
\end{question}
}

\element{cpo-mc}{
\begin{question}{cpo-ch01-q09}
    Of the following objects, the one which has the most mass is:
    \begin{choices}
      \correctchoice{the Goodyear blimp.}
        \wrongchoice{a silver dollar.}
        \wrongchoice{a piece of notebook paper.}
        \wrongchoice{a physics textbook.}
    \end{choices}
\end{question}
}

\element{cpo-mc}{
\begin{question}{cpo-ch01-q10}
    Robin measures the force needed to pull a wagon up an incline as more weight is added.
    In this investigation,
        weight is the \rule[-0.1pt]{4em}{0.1pt} variable.
    \begin{multicols}{2}
    \begin{choices}
        \wrongchoice{control}
        \wrongchoice{dependent}
      \correctchoice{independent}
        \wrongchoice{natural}
    \end{choices}
    \end{multicols}
\end{question}
}

\element{cpo-mc}{
\begin{question}{cpo-ch01-q11}
    The unit of time used most commonly by physicists and other scientists is the:
    \begin{multicols}{2}
    \begin{choices}
      \correctchoice{second}
        \wrongchoice{minute}
        \wrongchoice{hour}
        \wrongchoice{light year}
    \end{choices}
    \end{multicols}
\end{question}
}

\element{cpo-mc}{
\begin{questionmult}{cpo-ch01-Q12}
    The independent variable on a graph can be described as the variable:
    \begin{choices}
      \correctchoice{represented on the $x$-axis.}
      \correctchoice{causing the change in the experimental system.}
      \correctchoice{over which a scientist has direct control when designing the experiment.}
    \end{choices}
\end{questionmult}
}

\element{cpo-mc}{
\begin{question}{cpo-ch01-q13}
    The conversion factor for changing one unit of length to another in the metric system is a multiple of:
    \begin{multicols}{2}
    \begin{choices}
        \wrongchoice{3}
      \correctchoice{10}
        \wrongchoice{12}
        \wrongchoice{5280}
    \end{choices}
    \end{multicols}
\end{question}
}

\element{cpo-mc}{
\begin{question}{cpo-ch01-q14}
    Because it is based on factors of 10 and is easy to work with,
        scientists prefer to use the \rule[-0.1pt]{4em}{0.1pt} system.
    \begin{multicols}{2}
    \begin{choices}
      \correctchoice{metric}
        \wrongchoice{English}
        \wrongchoice{scientific}
        \wrongchoice{control}
    \end{choices}
    \end{multicols}
\end{question}
}

\element{cpo-mc}{
\begin{question}{cpo-ch01-q15}
    A graph may be described as all of the following \emph{except}:
    \begin{choices}
      \correctchoice{a tool to be interpreted \emph{only} by trained scientists and mathematicians.}
        \wrongchoice{used to describe the data collected from an experiment.}
        \wrongchoice{a picture showing how two variables are related.}
        \wrongchoice{easier to read than a table of numbers.}
    \end{choices}
\end{question}
}

\element{cpo-mc}{
\begin{question}{cpo-ch01-q16}
    The length of a new pencil is closest to:
    \begin{multicols}{2}
    \begin{choices}
        \wrongchoice{\SI{5}{\milli\meter}}
        \correctchoice{\SI{20}{\milli\meter}}
        \wrongchoice{\SI{1.5}{\meter}}
        \wrongchoice{\SI{2}{\kilo\meter}}
    \end{choices}
    \end{multicols}
\end{question}
}

\element{cpo-mc}{
\begin{question}{cpo-ch01-q17}
    %The number of seconds in one week is:
    One week is equivalent to:
    \begin{multicols}{2}
    \begin{choices}
        \wrongchoice{\SI{86 400}{\second}}
      \correctchoice{\SI{604 800}{\second}}
        \wrongchoice{\SI{31 557 600}{\second}}
        \wrongchoice{\SI{3 155 760 000}{\second}}
    \end{choices}
    \end{multicols}
\end{question}
}

\element{cpo-mc}{
\begin{question}{cpo-ch01-q18}
    Which of the lists show units arranged in order from smallest to largest?
    \begin{choices}
        \wrongchoice{millimeter, centimeter, kilometer, meter}
        \wrongchoice{centimeter, meter, kilometer, millimeter}
      \correctchoice{millimeter, centimeter, meter, kilometer}
        \wrongchoice{meter, kilometer, millimeter, centimeter}
    \end{choices}
\end{question}
}

\element{cpo-mc}{
\begin{question}{cpo-ch01-q19}
    Which of the following lists of mass units are arranged in order from smallest to largest?
    \begin{choices}
        \wrongchoice{gigagram, microgram, kilogram, megagram}
      \correctchoice{microgram, centigram, kilogram, gigagram}
        \wrongchoice{milligram, microgram, centigram, kilogram}
        \wrongchoice{megagram, kilogram, centigram, milligram}
    \end{choices}
\end{question}
}

\element{cpo-mc}{
\begin{question}{cpo-ch01-q20}
    How many seconds are in a stopwatch showing 1 hour, 3 minutes, and 5 seconds?
    \begin{multicols}{2}
    \begin{choices}
        \wrongchoice{\SI{68}{\second}}
        \wrongchoice{\SI{245}{\second}}
      \correctchoice{\SI{3 785}{\second}}
        \wrongchoice{\SI{10 805}{\second}}
    \end{choices}
    \end{multicols}
\end{question}
}

\element{cpo-mc}{
\begin{question}{cpo-ch01-q21}
    A rectangular solid has dimensions of $\SI{27}{\milli\meter}\times\SI{6.8}{\centi\meter}\times\SI{0.00025}{\kilo\meter}$.
    %The volume of the solid is \rule[-0.1pt]{4em}{0.1pt} cubic centimeters.
    The volume of the solid is:
    \begin{multicols}{2}
    \begin{choices}
        %% NOTE: changed formatting
        \wrongchoice{\SI{0.045 9}{\centi\meter\cubed}}
        \wrongchoice{\SI{0.345}{\centi\meter\cubed}}
        \wrongchoice{\SI{33.8}{\centi\meter\cubed}}
      \correctchoice{\SI{459}{\centi\meter\cubed}}
    \end{choices}
    \end{multicols}
\end{question}
}

\element{cpo-mc}{
\begin{question}{cpo-ch01-q22}
    Orlando measures the brightness of a flashlight bulb as he adds more batteries to the circuit.
    If he prepares a graph of the data:
    \begin{choices}
      \correctchoice{the number of batteries should be represented on the $x$-axis.}
        \wrongchoice{the brightness of the flashlight bulb should be represented on the $x$-axis.}
        \wrongchoice{it doesn't matter which variable he places on the $x$-axis.}
        \wrongchoice{he will need more information before deciding where to place the variables.}
    \end{choices}
\end{question}
}

\element{cpo-mc}{
\begin{question}{cpo-ch01-q23}
    On his way to a concert, John stops at the mall to buy some camera film.
    If you divide the distance he travels to the concert by the amount of time it took to get him home to his concert seat, you are calculating:
    \begin{multicols}{2}
    \begin{choices}
        \wrongchoice{speed}
        \wrongchoice{distance}
      \correctchoice{time interval}
        \wrongchoice{mixed units}
    \end{choices}
    \end{multicols}
\end{question}
}

\element{cpo-mc}{
\begin{question}{cpo-ch01-q24}
    If you know the distance traveled and the amount of time it took,
        speed may be calculated by:
    \begin{choices}
        \wrongchoice{dividing time by distance.}
        \wrongchoice{multiplying time by distance.}
      \correctchoice{dividing distance by time.}
        \wrongchoice{multiplying distance squared by time.}
    \end{choices}
\end{question}
}

\element{cpo-mc}{
\begin{question}{cpo-ch01-q25}
    Of the following, which equation does \emph{not} correctly represent a relationship between distance, time and speed?
    \begin{choices}
        \wrongchoice{Distance equals speed multiplied by time.}
      \correctchoice{Speed equals time multiplied by distance.}
        \wrongchoice{Time equals distance divided by speed.}
        \wrongchoice{Speed equals distance divided by time.}
    \end{choices}
\end{question}
}

\element{cpo-mc}{
\begin{question}{cpo-ch01-q26}
    The speed of a cheetah running 300 yards in 10 seconds is:
    \begin{multicols}{2}
    \begin{choices}
      \correctchoice{\SI{30}{\yard\per\second}}
        \wrongchoice{\SI{3 000}{\yard\per\second}}
        \wrongchoice{\SI{30 000}{\yard\per\second}}
        %% NOTE: I changed the wording on this
        \wrongchoice{Not defined}
    \end{choices}
    \end{multicols}
\end{question}
}

\element{cpo-mc}{
\begin{question}{cpo-ch01-q27}
    Doug rides a motorcycle at an average speed of \SI{42}{\mile\per\hour} for \SI{3.6}{\hour}.
    The distance he travels is about
    \begin{multicols}{2}
    \begin{choices}
        \wrongchoice{\SI{11}{\mile}}
        \wrongchoice{\SI{38}{\mile}}
        \wrongchoice{\SI{47}{\mile}}
      \correctchoice{\SI{150}{\mile}}
    \end{choices}
    \end{multicols}
\end{question}
}

\element{cpo-mc}{
\begin{question}{cpo-ch01-q28}
    Gwen rides her bicycle \SI{2.4}{\kilo\meter} up a steep hill in \SI{8}{\minute}.
    Her speed is:
    \begin{multicols}{2}
    \begin{choices}
      \correctchoice{\SI{0.3}{\kilo\meter\per\minute}}
        \wrongchoice{\SI{0.6}{\kilo\meter\per\minute}}
        \wrongchoice{\SI{3.3}{\kilo\meter\per\minute}}
        \wrongchoice{\SI{19}{\kilo\meter\per\minute}}
    \end{choices}
    \end{multicols}
\end{question}
}

\element{cpo-mc}{
\begin{question}{cpo-ch01-q29}
    A professional LPGA golfer walks at an average rate of \SI{3.20}{\foot\per\second} on the golf course.
    The amount of time required for her to walk from the tee to a green \SI{612}{\foot} away is:
    \begin{multicols}{2}
    \begin{choices}
        \wrongchoice{\SI{0.544}{\minute}}
        \wrongchoice{\SI{1.91}{\minute}}
        \wrongchoice{\SI{1 958}{\second}}
      \correctchoice{\SI{191}{\second}}
    \end{choices}
    \end{multicols}
\end{question}
}

\element{cpo-mc}{
\begin{question}{cpo-ch01-q30}
    A professional football quarterback throws a ball \SI{32}{\yard} down field to a receiver at a speed of \SI{60}{\mile\per\hour}.
    A mile equals \SI{1 760}{\yard}.
    Once the quarterback releases the ball, the football gets to the receiver in about:
    \begin{multicols}{2}
    \begin{choices}
      \correctchoice{\SI{1.1}{\second}}
        \wrongchoice{\SI{0.53}{\second}}
        \wrongchoice{\SI{0.92}{\second}}
        \wrongchoice{\SI{1.9}{\second}}
    \end{choices}
    \end{multicols}
\end{question}
}

\element{cpo-mc}{
\begin{question}{cpo-ch01-q31}
    Of the following, the largest unit of speed is:
    \begin{choices}
      \correctchoice{meter per second (\si{\meter\per\second})}
        \wrongchoice{kilometer per hour (\si{\kilo\meter\per\hour})}
        \wrongchoice{mile per hour (\si{\mile\per\hour})}
        \wrongchoice{inch per second (\si{\inch\per\second})}
    \end{choices}
\end{question}
}

\element{cpo-mc}{
\begin{question}{cpo-ch01-q32}
    The speed of a car traveling \SI{200}{\meter} in \SI{10}{\second} is equivalent to:
    \begin{multicols}{2}
    \begin{choices}
        \wrongchoice{\SI{20}{\yard\per\second}}
        \wrongchoice{\SI{2000}{\meter\per\second}}
      \correctchoice{\SI{72}{\kilo\meter\per\hour}}
        \wrongchoice{\SI{115}{\mile\per\hour}}
    \end{choices}
    \end{multicols}
\end{question}
}

\endinput

