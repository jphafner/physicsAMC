
%%--------------------------------------------------
%% CPO: AMC Open Free Response Questions
%%--------------------------------------------------


%% Chapter 2: The Laws of Motion
%%--------------------------------------------------


%% CPO Short Answer Questions
%%--------------------------------------------------
\element{cpo-short}{
\begin{question}{ch02-short-q01}
    You are a passenger sitting in the right side of a traveling car.
    The driver swerves to the left to avoid hitting a deer in the road.

    What motion would you experience as the passenger in the car?
    \AMCOpen{lines=3}{
        \wrongchoice[W]{w}\scoring{0}
        \wrongchoice[P]{p}\scoring{1}
        \correctchoice[C]{c}\scoring{2}
    }
\end{question}
}

\element{cpo-short}{
\begin{question}{ch02-short-q02}
    You are a passenger sitting in the right side of a traveling car.
    The driver swerves to the left to avoid hitting a deer in the road.

    Using Newton's first law, explain your motion as a passenger in
        the right side of the car.
    \AMCOpen{lines=3}{
        \wrongchoice[W]{w}\scoring{0}
        \wrongchoice[P]{p}\scoring{1}
        \correctchoice[C]{c}\scoring{2}
    }
\end{question}
}

\element{cpo-short}{
\begin{question}{ch02-short-q03}
    You are a passenger sitting in the right side of a traveling car.
    The driver swerves to the left to avoid hitting a deer in the road.

    Using Newton's second law, explain your motion as a passenger
        in the right side of the car.
    \AMCOpen{lines=3}{
        \wrongchoice[W]{w}\scoring{0}
        \wrongchoice[P]{p}\scoring{1}
        \correctchoice[C]{c}\scoring{2}
    }
\end{question}
}

\element{cpo-short}{
\begin{question}{ch02-short-q04}
    Describe two ways in which the motion of a body could be changed
        to cause it to accelerate.
    Give an example of each.
    \AMCOpen{lines=3}{
        \wrongchoice[W]{w}\scoring{0}
        \wrongchoice[P]{p}\scoring{1}
        \correctchoice[C]{c}\scoring{2}
    }
\end{question}
}

\element{cpo-short}{
\begin{question}{ch02-short-q05}
    Describe the difference between a positive and a negative acceleration.
    \AMCOpen{lines=3}{
        \wrongchoice[W]{w}\scoring{0}
        \wrongchoice[P]{p}\scoring{1}
        \correctchoice[C]{c}\scoring{2}
    }
\end{question}
}

\element{cpo-short}{
\begin{question}{ch02-short-q06}
    When you throw a ball up in the air, it travels up and then stops
        instantaneously before falling back down.
    Describe its velocity and acceleration at the point where it
        stops and changes direction to fall back down.
    \AMCOpen{lines=3}{
        \wrongchoice[W]{w}\scoring{0}
        \wrongchoice[P]{p}\scoring{1}
        \correctchoice[C]{c}\scoring{2}
    }
\end{question}
}

\element{cpo-short}{
\begin{question}{ch02-short-q07}
    When you drop a sheet of crumpled paper and a sheet of flat paper
        off a table at the same time, why does the flat
        sheet of paper hit the ground later?
    \AMCOpen{lines=3}{
        \wrongchoice[W]{w}\scoring{0}
        \wrongchoice[P]{p}\scoring{1}
        \correctchoice[C]{c}\scoring{2}
    }
\end{question}
}

\element{cpo-short}{
\begin{question}{ch02-short-q08}
    How can the distance a sprinter has run be determined from a graph
        of the sprinter's speed versus time?
    Assume the $x$-axis represents zero speed.
    \AMCOpen{lines=3}{
        \wrongchoice[W]{w}\scoring{0}
        \wrongchoice[P]{p}\scoring{1}
        \correctchoice[C]{c}\scoring{2}
    }
\end{question}
}

\element{cpo-short}{
\begin{question}{ch02-short-q09}
    The motion of a car is represented by a speed versus time graph.
    The line of the graph slopes down from left to right.
    What is the car doing?
    \AMCOpen{lines=3}{
        \wrongchoice[W]{w}\scoring{0}
        \wrongchoice[P]{p}\scoring{1}
        \correctchoice[C]{c}\scoring{2}
    }
\end{question}
}

%% CPO Problem Questions
%%--------------------------------------------------
\element{cpo-problem}{
\begin{question}{ch02-problem-q01}
    Tom pushes on a \SI{50}{\kilo\gram} box with a force of
        \SI{25}{\newton}.
    Assuming the surface on which the box moves is frictionless,
        at what rate does the box accelerate?
    \AMCOpen{lines=3}{
        \wrongchoice[W]{w}\scoring{0}
        \wrongchoice[P]{p}\scoring{1}
        \correctchoice[C]{c}\scoring{2}
    }
    %% ANS: \SI{0.5}{\meter\per\second\squared}
\end{question}
}

\element{cpo-problem}{
\begin{question}{ch02-problem-q02}
    Two body builders are involved in a weight lifting contest to
        determine who is the stronger.
    Ivan lifts \SI{480}{\pound}, Igor lifts \SI{2000}{\newton}.
    Who is stronger?
    \AMCOpen{lines=3}{
        \wrongchoice[W]{w}\scoring{0}
        \wrongchoice[P]{p}\scoring{1}
        \correctchoice[C]{c}\scoring{2}
    }
    %% ANS: \SI{165}{\centi\meter}
\end{question}
}

\element{cpo-problem}{
\begin{question}{ch02-problem-q03}
    In the graph below, the acceleration of an object is plotted against
        the net force on the object.
    What is the mass of the object?
    \begin{center}
        \begin{tikzpicture}
            \begin{axis}[
                axis y line=left,
                axis x line=bottom,
                axis line style={->},
                ylabel={acceleration},
                y unit=\si{\meter\per\second\squared},
                ytick={0,1,2,3,4},
                ymin=0,ymax=4.1,
                xlabel={net force},
                x unit=\si{\newton},
                xtick={0,1,2,3,4,5,6},
                xmin=0,xmax=6.1,
                grid=major,
                width=0.618\columnwidth,
                height=0.5\columnwidth,
                very thin,
            ]
            \addplot[black,line width=1pt,domain=0,6.1]{0.5 * x};
            \end{axis}
        \end{tikzpicture}
    \end{center}
    \AMCOpen{lines=3}{
        \wrongchoice[W]{w}\scoring{0}
        \wrongchoice[P]{p}\scoring{1}
        \correctchoice[C]{c}\scoring{2}
    }
    %% ANS: \SI{2}{\kilo\gram}
\end{question}
}

\element{cpo-problem}{
\begin{question}{ch02-problem-q04}
    An acorn falls from the top of an oak tree.
    If it takes \SI{2}{\second} for the acorn to fall,
        how tall is the tree?
    \AMCOpen{lines=3}{
        \wrongchoice[W]{w}\scoring{0}
        \wrongchoice[P]{p}\scoring{1}
        \correctchoice[C]{c}\scoring{2}
    }
    %% ANS: \SI{19.6}{\meter}
\end{question}
}

\element{cpo-problem}{
\begin{question}{ch02-problem-q05}
    What is the average speed of a ball thrown downward with an initial
        speed of \SI{4.9}{\meter\per\second} that falls for
        \SI{0.5}{\second}?
    How far does the ball fall?
    \AMCOpen{lines=3}{
        \wrongchoice[W]{w}\scoring{0}
        \wrongchoice[P]{p}\scoring{1}
        \correctchoice[C]{c}\scoring{2}
    }
    %% ANS: \SI{2.45}{\meter}
\end{question}
}

\element{cpo-problem}{
\begin{question}{ch02-problem-q06}
    To support a mass of \SI{1.0}{\kilo\gram} on Earth
        requires a force of \rule[-0.1pt]{4em}{0.1pt} pounds.
    \AMCOpen{lines=3}{
        \wrongchoice[W]{w}\scoring{0}
        \wrongchoice[P]{p}\scoring{1}
        \correctchoice[C]{c}\scoring{2}
    }
    %% ANS: \SI{1}{\mile}
\end{question}
}

\element{cpo-problem}{
\begin{question}{ch02-problem-q07}
    What is the weight in newtons of a \SI{15}{\kilo\gram}
        toddler on Earth?
    \AMCOpen{lines=3}{
        \wrongchoice[W]{w}\scoring{0}
        \wrongchoice[P]{p}\scoring{1}
        \correctchoice[C]{c}\scoring{2}
    }
    %% ANS: \SI{147}{\newton}
\end{question}
}

\element{cpo-problem}{
\begin{question}{ch02-problem-q08}
    Use the graph representing the distance versus time for a car
        moving in a straight line to find the speed of the car
        from $t=\SI{2.0}{\second}$ to $t=\SI{4.0}{\second}$.
    \begin{center}
        \begin{tikzpicture}
            \begin{axis}[
                axis y line=left,
                axis x line=bottom,
                axis line style={->},
                ylabel={distance},
                y unit=\si{\meter},
                ytick={0,10,20,30},
                ymin=0,ymax=31,
                xlabel={time},
                x unit=\si{\second},
                xtick={0,1,2,3,4,5,6},
                xmin=0,xmax=6.1,
                grid=major,
                width=0.618\columnwidth,
                height=0.5\columnwidth,
                very thin,
            ]
            \addplot[black,line width=1pt,domain=0,2]{5* x};
            \addplot[black,line width=1pt,domain=2,4]{10 + 10*(x-2)};
            \addplot[black,line width=1pt,domain=4,6]{30};
            \end{axis}
        \end{tikzpicture}
    \end{center}
    \AMCOpen{lines=3}{
        \wrongchoice[W]{w}\scoring{0}
        \wrongchoice[P]{p}\scoring{1}
        \correctchoice[C]{c}\scoring{2}
    }
    %% ANS: \SI{10}{\meter\per\second}
\end{question}
}

\element{cpo-problem}{
\begin{question}{ch02-problem-q09}
    Below is a graph representing the distance versus time for 
        an object moving in a straight line.
    Identify the interval during which the car is not moving.
    \begin{center}
        \begin{tikzpicture}
            \begin{axis}[
                axis y line=left,
                axis x line=bottom,
                axis line style={->},
                ylabel={position},
                y unit=\si{\meter},
                ytick={0,1,2,3,4},
                ymin=0,ymax=4.1,
                xlabel={time},
                x unit=\si{\second},
                xtick={0,1,2,3,4,5},
                xmin=0,xmax=5.1,
                grid=major,
                width=0.618\columnwidth,
                height=0.5\columnwidth,
                very thin,
            ]
            \addplot[black,line width=1pt,domain=0,2]{x};
            \addplot[black,line width=1pt,domain=2,4]{2};
            \addplot[black,line width=1pt,domain=4,5]{2 - (x-4)};
            \end{axis}
        \end{tikzpicture}
    \end{center}
    \AMCOpen{lines=3}{
        \wrongchoice[W]{w}\scoring{0}
        \wrongchoice[P]{p}\scoring{1}
        \correctchoice[C]{c}\scoring{2}
    }
    %% ANS: \SI{2}{\minute} to \SI{4}{\second}
\end{question}
}


%% CPO Essay Questions
%%--------------------------------------------------
\element{cpo-essay}{
\begin{question}{ch02-essay-q01}
    Explain the difference between mass and the weight of an object.
    \AMCOpen{lines=3}{
        \wrongchoice[W]{w}\scoring{0}
        \wrongchoice[P]{p}\scoring{1}
        \correctchoice[C]{c}\scoring{2}
    }
\end{question}
}

\element{cpo-essay}{
\begin{question}{ch02-essay-q02}
    The law of inertia states that no force is required to maintain the motion
        of a moving object.
    Explain why you must continue to pedal your bicycle even on a level surface
        to keep moving.
    \AMCOpen{lines=3}{
        \wrongchoice[W]{w}\scoring{0}
        \wrongchoice[P]{p}\scoring{1}
        \correctchoice[C]{c}\scoring{2}
    }
\end{question}
}

\endinput


