

%% http://highered.mheducation.com/sites/0072414944/student_view0/chapter1/multiple_choice_quiz_1.html


%% Alternative Multiple Choice Questions
%%--------------------------------------------------
\element{cpo-mc}{
\begin{question}{ch01-mc-Q101}
    An object or event is best described by noting its
    \begin{multicols}{2}
    \begin{choices}
        \wrongchoice{size.}
        \wrongchoice{properties.}
        \wrongchoice{referents.}
      \correctchoice{length, mass, or time.}
    \end{choices}
    \end{multicols}
\end{question}
}

\element{cpo-mc}{
\begin{question}{ch01-mc-Q102}
    How many decimeters are equal to the length of one meter, \SI{1}{\meter}?
    \begin{multicols}{2}
    \begin{choices}
        \wrongchoice{\num{1}}
      \correctchoice{\num{10}}
        \wrongchoice{\num{100}}
        \wrongchoice{\num{1000}}
    \end{choices}
    \end{multicols}
\end{question}
}

\element{cpo-mc}{
\begin{question}{ch01-mc-Q103}
    How many milliliters are needed to have a volume of one liter, \SI{1}{\liter}?
    \begin{multicols}{2}
    \begin{choices}
        \wrongchoice{\num{1}}
        \wrongchoice{\num{10}}
        \wrongchoice{\num{100}}
      \correctchoice{\num{1000}}
    \end{choices}
    \end{multicols}
\end{question}
}

\element{cpo-mc}{
\begin{question}{ch01-mc-Q104}
    Quantities, or measured properties that are capable of changing values are called:
    \begin{multicols}{2}
    \begin{choices}
        \wrongchoice{data.}
      \correctchoice{variables.}
        \wrongchoice{constants.}
        \wrongchoice{impossible.}
    \end{choices}
    \end{multicols}
\end{question}
}

\element{cpo-mc}{
\begin{question}{ch01-mc-Q105}
    The property of volume is a measure of
    \begin{choices}
        \wrongchoice{how much matter the object contains.}
        \wrongchoice{the compactness of matter in a given space.}
        \wrongchoice{the extent of the surface of the object.}
      \correctchoice{how much space the object occupies.}
    \end{choices}
\end{question}
}

\element{cpo-mc}{
\begin{question}{ch01-mc-Q106}
    A tentative scientific explanation which may or may not
        be rejected upon further experimentation is called a:
    \begin{multicols}{2}
    \begin{choices}
        \wrongchoice{theory.}
      \correctchoice{hypothesis.}
        \wrongchoice{model.}
        \wrongchoice{principle.}
    \end{choices}
    \end{multicols}
\end{question}
}

\element{cpo-mc}{
\begin{question}{ch01-mc-Q107}
    Which one of the following is not a fundamental property?
    \begin{multicols}{2}
    \begin{choices}
        \wrongchoice{mass}
        \wrongchoice{time}
      \correctchoice{weight}
        \wrongchoice{length}
    \end{choices}
    \end{multicols}
\end{question}
}

\element{cpo-mc}{
\begin{questionmult}{ch01-mc-Q108}
    Equations are used to
    \begin{choices}
      \correctchoice{describe a property.}
      \correctchoice{define a concept.}
      \correctchoice{describe how quantities change together.}
    \end{choices}
\end{questionmult}
}

\element{cpo-mc}{
\begin{question}{ch01-mc-Q109}
    How fast you are moving is a property of motion known as
    \begin{multicols}{2}
    \begin{choices}
      \correctchoice{speed.}
        \wrongchoice{velocity.}
        \wrongchoice{acceleration.}
        \wrongchoice{jerk}
    \end{choices}
    \end{multicols}
\end{question}
}

\element{cpo-mc}{
\begin{question}{ch01-mc-Q110}
    How fast you are moving in which direction is a property of motion known as
    \begin{multicols}{2}
    \begin{choices}
        \wrongchoice{speed.}
      \correctchoice{velocity.}
        \wrongchoice{acceleration.}
        \wrongchoice{jerk}
    \end{choices}
    \end{multicols}
\end{question}
}


%% Alternative Multiple Choice Questions
%%--------------------------------------------------
\element{cpo-mc}{
\begin{question}{ch02-mc-Q101}
    How fast you are changing your speed or direction of
        travel is a property of motion known as
    \begin{multicols}{2}
    \begin{choices}
        \wrongchoice{speed.}
        \wrongchoice{velocity.}
      \correctchoice{acceleration.}
        \wrongchoice{none of the provided.}
    \end{choices}
    \end{multicols}
\end{question}
}

\element{cpo-mc}{
\begin{question}{ch02-mc-Q102}
    The tendency of a moving object to remain in unchanging motion
        in the absence of an unbalanced force is called
    \begin{multicols}{2}
    \begin{choices}
      \correctchoice{inertia}
        \wrongchoice{free fall}
        \wrongchoice{acceleration}
        \wrongchoice{impulse}
    \end{choices}
    \end{multicols}
\end{question}
}

\element{cpo-mc}{
\begin{question}{ch02-mc-Q103}
    A ball rolling across the floor slows to a stop because
    \begin{choices}
      \correctchoice{there are unbalanced forces acting on it.}
        \wrongchoice{the force that started it moving wears out.}
        \wrongchoice{all the forces are balanced.}
        \wrongchoice{the net force equals zero.}
    \end{choices}
\end{question}
}

\element{cpo-mc}{
\begin{question}{ch02-mc-Q104}
    Ignoring air resistance,
        an object falling toward the surface of the earth has a velocity that is
    \begin{choices}
        \wrongchoice{constant.}
      \correctchoice{increasing.}
        \wrongchoice{decreasing.}
        \wrongchoice{acquired instantaneously.}
    \end{choices}
\end{question}
}

\element{cpo-mc}{
\begin{question}{ch02-mc-Q105}
    Ignoring air resistance,
        an object falling toward the surface of the earth has an acceleration that is
    \begin{choices}
      \correctchoice{constant.}
        \wrongchoice{increasing.}
        \wrongchoice{decreasing.}
        \wrongchoice{dependent on the weight of the object.}
    \end{choices}
\end{question}
}

\element{cpo-mc}{
\begin{question}{ch02-mc-Q106}
    Ignoring air resistance,
        a bullet fired horizontally has how many forces acting on it after leaving the rifle?
    \begin{choices}
        \wrongchoice{One, from the gunpowder explosion.}
      \correctchoice{One, from the pull of gravity.}
        \wrongchoice{Two, one from the gunpowder explosion and one from gravity.}
        \wrongchoice{Three, one from the gunpowder explosion, one from gravity, and one from the motion of the bullet.}
    \end{choices}
\end{question}
}

\element{cpo-mc}{
\begin{question}{ch02-mc-Q107}
    A heavy object and a light object are dropped from rest at the same time in a vacuum.
    The heavier object will reach the ground
    \begin{choices}
        \wrongchoice{before the lighter object}
      \correctchoice{at the same time as the lighter object}
        \wrongchoice{after the lighter object}
        \wrongchoice{It depends on the shape of the object}
    \end{choices}
\end{question}
}

\element{cpo-mc}{
\begin{questionmult}{ch02-mc-Q108}
    A change in the state of motion is evidence of
    \begin{choices}
      \correctchoice{a force.}
      \correctchoice{an applied force that is unbalanced.}
      \correctchoice{a force that is wearing down.}
      %\correctchoice{any of these.}
    \end{choices}
\end{questionmult}
}

\element{cpo-mc}{
\begin{question}{ch02-mc-Q109}
    What do you know for sure about the forces on the system
        of you and a bicycle as you pedal the bike at a constant speed in a straight line?
    \begin{choices}
        \wrongchoice{The applied force you are exerting on the pedal is greater than other forces.}
      \correctchoice{All forces are in balance.}
        \wrongchoice{The resisting forces of friction are greater than the applied forces.}
        \wrongchoice{Air and tire friction forces are less than the applied force.}
    \end{choices}
\end{question}
}

\element{cpo-mc}{
\begin{questionmult}{ch02-mc-Q110}
    What property of matter determines the resistance to a change
        in the state of motion for a given object?
    \begin{choices}
        \wrongchoice{weight}
        \wrongchoice{density}
      \correctchoice{mass}
        %\wrongchoice{any of these}
    \end{choices}
\end{questionmult}
}

\element{cpo-mc}{
\begin{question}{ch02-mc-Q111}
    With all other factors equal,
        if you double the unbalanced force on an object of a given mass the acceleration will be
    \begin{choices}
      \correctchoice{doubled.}
        \wrongchoice{increased fourfold.}
        \wrongchoice{increased by one-half.}
        \wrongchoice{increased by one-forth.}
    \end{choices}
\end{question}
}

\element{cpo-mc}{
\begin{question}{ch02-mc-Q112}
    With all other factors equal,
        if you double the mass of an object while a constant unbalanced force is applied,
        the acceleration will be
    \begin{choices}
        \wrongchoice{doubled.}
        \wrongchoice{increased fourfold.}
      \correctchoice{one-half as much.}
        \wrongchoice{one-fourth as much.}
    \end{choices}
\end{question}
}

\element{cpo-mc}{
\begin{question}{ch02-mc-Q113}
    Which of the following is a unit that can be used for a measure of weight?
    \begin{choices}
        \wrongchoice{kilogram (\si{\kilo\gram})}
      \correctchoice{newton (\si{\newton})}
        \wrongchoice{kilogram meter per second (\si{\kilo\gram\meter\per\second})}
        \wrongchoice{none of the provided}
    \end{choices}
\end{question}
}

\element{cpo-mc}{
\begin{question}{ch02-mc-Q114}
    Which of the following is a unit for a measure of resistance to a change of motion?
    \begin{choices}
        \wrongchoice{pound (\si{\pound})}
      \correctchoice{kilogram (\si{\kilo\gram})}
        \wrongchoice{newton (\si{\newton})}
        \wrongchoice{none of the provided}
    \end{choices}
\end{question}
}


%% Alternative Multiple Choice Questions
%%--------------------------------------------------
\element{cpo-mc}{
\begin{question}{ch04-mc-Q101}
    According to the scientific definition of work, pushing on a rock accomplishes no work unless there is
    \begin{choices}
        \wrongchoice{an applied force greater than its weight.}
        \wrongchoice{a net force greater than zero.}
        \wrongchoice{an opposing force.}
        \wrongchoice{movement in the same direction as the force.}
    \end{choices}
\end{question}
}

\element{cpo-mc}{
\begin{questionmult}{ch04-mc-Q102}
    The metric unit of a joule (\si{\joule}) is a unit of
    \begin{choices}
      \correctchoice{potential energy.}
      \correctchoice{work.}
      \correctchoice{kinetic energy.}
    \end{choices}
\end{questionmult}
}

\element{cpo-mc}{
\begin{question}{ch04-mc-Q103}
    Power is:
    \begin{choices}
        \wrongchoice{the rate at which energy is expended.}
        \wrongchoice{work per unit of time.}
        \wrongchoice{the rate at which work is done.}
        \wrongchoice{any of the above.}
    \end{choices}
\end{question}
}

\element{cpo-mc}{
\begin{question}{ch04-mc-Q104}
    A newton meter per second (\si{\newton\meter\per\second}) is a unit of:
    \begin{multicols}{2}
    \begin{choices}
        \wrongchoice{energy.}
      \correctchoice{power.}
        \wrongchoice{force.}
        \wrongchoice{work.}
    \end{choices}
    \end{multicols}
\end{question}
}

\element{cpo-mc}{
\begin{question}{ch04-mc-Q105}
    The kilowatt-hour (\si{\kilo\watt\hour}) is a unit of:
    \begin{multicols}{2}
    \begin{choices}
        \wrongchoice{power.}
        \wrongchoice{work.}
        \wrongchoice{time.}
        \wrongchoice{force.}
    \end{choices}
    \end{multicols}
\end{question}
}

\element{cpo-mc}{
\begin{question}{ch04-mc-Q106}
    The potential energy of a box on a shelf, relative to the floor, is a measure of
    \begin{choices}
        \wrongchoice{the work done putting the box on the shelf from the floor.}
        \wrongchoice{the weight of the box times the distance above the floor.}
        \wrongchoice{the energy the box has because of its position above the floor.}
        \wrongchoice{any of these.}
    \end{choices}
\end{question}
}

\element{cpo-mc}{
\begin{question}{ch04-mc-Q107}
    Which quantity has the greatest influence on the amount of
        kinetic energy that a large truck has while moving down the highway?
    \begin{multicols}{2}
    \begin{choices}
        \wrongchoice{mass}
        \wrongchoice{weight}
      \correctchoice{velocity}
        \wrongchoice{size}
    \end{choices}
    \end{multicols}
\end{question}
}

\element{cpo-mc}{
\begin{question}{ch04-mc-Q108}
    The law of conservation of energy is a statement that
    \begin{choices}
        \wrongchoice{energy must be conserved and you are breaking a law if you waste energy.}
        \wrongchoice{the supply of energy is limited so we must conserve.}
        \wrongchoice{the total amount of energy is constant.}
        \wrongchoice{energy cannot be used faster than it is created.}
    \end{choices}
\end{question}
}

\element{cpo-mc}{
\begin{question}{ch04-mc-Q109}
    Energy is:
    \begin{choices}
        \wrongchoice{the ability to do work.}
        \wrongchoice{the work needed to create potential or kinetic energy.}
        \wrongchoice{the work that can be done by an object with PE or KE.}
        \wrongchoice{all of the above}
    \end{choices}
\end{question}
}


%% Alternative Multiple Choice Questions
%%--------------------------------------------------
\element{cpo-mc}{
\begin{question}{ch05-mc-Q101}
    The immediate source of the force that accelerates a car over a road comes from
    \begin{multicols}{2}
    \begin{choices}
        \wrongchoice{the engine.}
        \wrongchoice{the tires.}
        \wrongchoice{the road.}
        \wrongchoice{all of these.}
    \end{choices}
    \end{multicols}
\end{question}
}


%% Alternative Multiple Choice Questions
%%--------------------------------------------------
\element{cpo-mc}{
\begin{question}{ch07-mc-Q101}
    The two temperature scales with the same interval step size are the
    \begin{choices}
        \wrongchoice{Celsius and Fahrenheit.}
        \wrongchoice{Fahrenheit and Kelvin.}
        \wrongchoice{Kelvin and Celsius.}
        \wrongchoice{This does not exist.}
    \end{choices}
\end{question}
}

\element{cpo-mc}{
\begin{question}{ch07-mc-Q102}
    Substance $A$ has a higher specific heat than substance $B$.
    With all other factors equal, which requires the most energy to heat equal masses of $A$ and $B$ to the same temperature?
    \begin{choices}
        \wrongchoice{Substance $A$}
        \wrongchoice{Substance $B$}
        \wrongchoice{Both require the same amount of heat.}
        \wrongchoice{Answer depends on the density of each substance.}
    \end{choices}
\end{question}
}

\element{cpo-mc}{
\begin{question}{ch07-mc-Q103}
    With all other factors equal, the most likely to burn your mouth when taken directly from an oven is a food with
    \begin{choices}
        \wrongchoice{higher specific heat.}
        \wrongchoice{lower specific heat.}
        \wrongchoice{Specific heat is not important in this situation.}
        \wrongchoice{More information is needed.}
    \end{choices}
\end{question}
}

\element{cpo-mc}{
\begin{question}{ch07-mc-Q104}
    A large and a small container of water with the same temperature have
    \begin{choices}
        \wrongchoice{the same total amounts of internal energy.}
        \wrongchoice{the same amounts of internal and external energy.}
        \wrongchoice{different amounts of heat.}
        \wrongchoice{the same amounts of all forms of energy.}
    \end{choices}
\end{question}
}

\element{cpo-mc}{
\begin{question}{ch07-mc-Q105}
    Anytime a temperature difference occurs, you can expect
    \begin{choices}
        \wrongchoice{cold to move to where it is warmer.}
        \wrongchoice{energy movement from higher temperature regions.}
        \wrongchoice{no energy movement unless it is warm enough, at least above the freezing temperature.}
        \wrongchoice{energy movement flowing slowly from cold to warmer regions.}
    \end{choices}
\end{question}
}

\element{cpo-mc}{
\begin{question}{ch07-mc-Q106}
    The transfer of energy from molecule to molecule is called
    \begin{choices}
        \wrongchoice{convection.}
        \wrongchoice{radiation.}
        \wrongchoice{conduction.}
        \wrongchoice{equilibrium.}
    \end{choices}
\end{question}
}


%% Alternative Multiple Choice Questions
%%--------------------------------------------------
\element{cpo-mc}{
\begin{question}{ch08-mc-Q101}
    Imagine a \SI{10}{\gram} chunk of aluminum ($d=\SI{2.7}{\gram\per\centi\meter\cubed}$) and a \SI{10}{\gram} chunk of iron ($d=\SI{7.9}{\gram\per\centi\meter\cubed}$).
    Which of the following is true?
    \begin{choices}
        \wrongchoice{The chunk of iron is smaller than the chunk of aluminum.}
        \wrongchoice{The chunk of iron is more massive than the chunk of aluminum.}
        \wrongchoice{The chunk of aluminum is smaller than the chunk of iron.}
        \wrongchoice{Both objects have the same volume.}
    \end{choices}
\end{question}
}

\element{cpo-mc}{
\begin{question}{ch08-mc-Q102}
    As you go to higher elevations above sea level the boiling point of water
    \begin{choices}
        \wrongchoice{decreases.}
        \wrongchoice{increases.}
        \wrongchoice{stays the same.}
        \wrongchoice{changes with the initial temperature of the water.}
    \end{choices}
\end{question}
}

\element{cpo-mc}{
\begin{question}{ch08-mc-Q103}
    As a solid goes through a phase change to a liquid, heat is absorbed and the temperature
    \begin{choices}
        \wrongchoice{increases.}
        \wrongchoice{decreases.}
        \wrongchoice{remains the same.}
        \wrongchoice{fluctuates.}
    \end{choices}
\end{question}
}

\element{cpo-mc}{
\begin{question}{ch08-mc-Q105}
    No water vapor is added to or removed from a sample of air that is cooling, so the relative humidity of this sample of air will
    \begin{choices}
        \wrongchoice{remain the same.}
        \wrongchoice{be lower.}
        \wrongchoice{be higher.}
        \wrongchoice{the answer depends on the temperature.}
    \end{choices}
\end{question}
}

\element{cpo-mc}{
\begin{question}{ch08-mc-Q106}
    Increasing the rate of heating under a pot of boiling water will
    \begin{choices}
        \wrongchoice{increase the temperature of the boiling water.}
        \wrongchoice{increase the rate of boiling, but not the temperature.}
        \wrongchoice{increase both the rate of boiling and the temperature of the boiling water.}
        \wrongchoice{all of the provided.}
    \end{choices}
\end{question}
}

\element{cpo-mc}{
\begin{question}{ch08-mc-Q107}
    What is the most likely temperature of the water at the bottom of Lake Superior in the winter?
    \begin{choices}
        \wrongchoice{\SI{0}{\degreeCelsius}}
      \correctchoice{\SI{4}{\degreeCelsius}}
        \wrongchoice{\SI{10}{\degreeCelsius}}
        \wrongchoice{The temperature is fairly variable.}
    \end{choices}
\end{question}
}

\element{cpo-mc}{
\begin{question}{ch08-mc-Q108}
    If you consider a very small portion of a material that is the same throughout,
        the density of the smaller sample will be
    \begin{choices}
        \wrongchoice{less than the larger sample.}
      \correctchoice{the same as the larger sample.}
        \wrongchoice{more than the larger sample.}
        \wrongchoice{dependent on the shape of the larger sample.}
    \end{choices}
\end{question}
}


%% Alternative Multiple Choice Questions
%%--------------------------------------------------
\element{cpo-mc}{
\begin{question}{ch09-mc-Q102}
    The outer electrons of an atom that participate in chemical bonding
        are referred to as what kind of electron?
    \begin{multicols}{2}
    \begin{choices}
        \wrongchoice{reactant}
        \wrongchoice{valence}
        \wrongchoice{net}
        \wrongchoice{product}
    \end{choices}
    \end{multicols}
\end{question}
}

\element{cpo-mc}{
\begin{question}{ch09-mc-Q103}
    The element sodium (\ce{Na}) has how many valence electrons?
    \begin{multicols}{2}
    \begin{choices}
        \wrongchoice{\num{1}}
        \wrongchoice{\num{2}}
        \wrongchoice{\num{7}}
        \wrongchoice{\num{8}}
    \end{choices}
    \end{multicols}
\end{question}
}

\element{cpo-mc}{
\begin{question}{ch09-mc-Q104}
    An atom is the:
    \begin{choices}
        \wrongchoice{smallest unit of an element that can exist alone.}
        \wrongchoice{smallest particle that participates in chemical bonding.}
        \wrongchoice{unit with equal numbers of protons and neutrons.}
        \wrongchoice{unit with unequal numbers of electrons and protons.}
    \end{choices}
\end{question}
}

\element{cpo-mc}{
\begin{question}{ch09-mc-Q105}
    The weighted average of the masses of the stable isotopes of an
        element as they occur in nature is called the:
    \begin{choices}
        \wrongchoice{atomic number.}
        \wrongchoice{atomic mass.}
        \wrongchoice{atomic weight.}
        \wrongchoice{mass number.}
    \end{choices}
\end{question}
}

\element{cpo-mc}{
\begin{question}{ch09-mc-Q106}
    The modern periodic law is based on:
    \begin{choices}
        \wrongchoice{atomic number.}
        \wrongchoice{atomic mass.}
        \wrongchoice{atomic weight.}
        \wrongchoice{chemical activity.}
    \end{choices}
\end{question}
}

\element{cpo-mc}{
\begin{question}{ch09-mc-Q107}
    The sum of the number of protons and neutrons in the nucleus of an atom is called the:
    \begin{choices}
        \wrongchoice{atomic number.}
        \wrongchoice{atomic mass.}
        \wrongchoice{atomic weight.}
        \wrongchoice{mass number.}
    \end{choices}
\end{question}
}

\element{cpo-mc}{
\begin{question}{ch09-mc-Q108}
    Each family, or group of elements in a vertical column of the
        periodic table has elements with chemical characteristics that are
    \begin{choices}
        \wrongchoice{exactly the same.}
        \wrongchoice{similar.}
        \wrongchoice{different.}
        \wrongchoice{exactly opposite.}
    \end{choices}
\end{question}
}

\element{cpo-mc}{
\begin{question}{ch09-mc-Q109}
    Isotopes are atoms of an element with identical chemical properties but with different
    \begin{choices}
        \wrongchoice{numbers of protons.}
        \wrongchoice{masses.}
        \wrongchoice{numbers of electrons.}
        \wrongchoice{atomic numbers.}
    \end{choices}
\end{question}
}

\element{cpo-mc}{
\begin{question}{ch09-mc-Q110}
    The masses of all isotopes are based on a comparison to the mass of a particular isotope of
    \begin{choices}
        \wrongchoice{hydrogen.}
        \wrongchoice{carbon.}
        \wrongchoice{oxygen.}
        \wrongchoice{uranium.}
    \end{choices}
\end{question}
}

\element{cpo-mc}{
\begin{question}{ch09-mc-Q111}
    How many naturally occurring elements are found on the earth in significant quantities?
    \begin{choices}
        \wrongchoice{\num{115}}
        \wrongchoice{\num{92}}
        \wrongchoice{\num{89}}
        \wrongchoice{\num{26}}
    \end{choices}
\end{question}
}

\element{cpo-mc}{
\begin{question}{ch09-mc-Q112}
    The Bohr model of the atom was able to explain the Balmer series because
    \begin{choices}
        \wrongchoice{larger orbits required electrons to have more negative energy in order to match the angular momentum.}
        \wrongchoice{differences between the energy levels of the orbits matched the difference between energy levels of the line spectra.}
        \wrongchoice{electrons were allowed to exist only in allowed orbits and nowhere else.}
        \wrongchoice{none of the above.}
    \end{choices}
\end{question}
}

\element{cpo-mc}{
\begin{question}{ch09-mc-Q113}
    The atomic number of an atom identifies the number of
    \begin{choices}
        \wrongchoice{protons.}
        \wrongchoice{neutrons.}
        \wrongchoice{quantum orbits.}
        \wrongchoice{excited states.}
    \end{choices}
\end{question}
}

\element{cpo-mc}{
\begin{question}{ch09-mc-Q114}
    Most of the volume of an atom is occupied by
    \begin{choices}
        \wrongchoice{electrons.}
        \wrongchoice{protons.}
        \wrongchoice{neutrons.}
        \wrongchoice{empty space}
    \end{choices}
\end{question}
}

\element{cpo-mc}{
\begin{question}{ch09-mc-Q115}
    The planetary model of an atom,
        with the nucleus playing the role of the Sun and the electrons playing the role of planets,
        is unacceptable because
    \begin{choices}
        \wrongchoice{the electrical attraction between a proton and an electron is too weak.}
        \wrongchoice{an electron is accelerating and would lose energy.}
        \wrongchoice{the nuclear attraction between a proton and an electron is too strong.}
        \wrongchoice{none of these because the planetary model is acceptable.}
    \end{choices}
\end{question}
}

\element{cpo-mc}{
\begin{question}{ch09-mc-Q116}
    The nucleus was discovered through experiments with
    \begin{choices}
        \wrongchoice{electricity.}
        \wrongchoice{light.}
        \wrongchoice{radio waves.}
        \wrongchoice{radioactivity.}
    \end{choices}
\end{question}
}

\element{cpo-mc}{
\begin{question}{ch09-mc-Q117}
    The first part of an atom to be discovered was the
    \begin{choices}
        \wrongchoice{proton.}
        \wrongchoice{neutron.}
        \wrongchoice{electron.}
        \wrongchoice{nucleus.}
    \end{choices}
\end{question}
}


%% Alternative Multiple Choice Questions
%%--------------------------------------------------
\element{cpo-mc}{
\begin{question}{ch11-mc-Q101}
    The empirical formula of \ce{C4H8} is:
    \begin{multicols}{2}
    \begin{choices}
        \wrongchoice{\ce{CH}.}
        \wrongchoice{\ce{CH2}.}
        \wrongchoice{\ce{C4H8}.}
        \wrongchoice{\ce{C2H4}.}
    \end{choices}
    \end{multicols}
\end{question}
}

\element{cpo-mc}{
\begin{question}{ch11-mc-Q102}
    The reaction \ce{2Mg + O2 -> 2 MgO} is an example of:
    \begin{choices}
        \wrongchoice{decomposition.}
        \wrongchoice{combination.}
        \wrongchoice{replacement.}
        \wrongchoice{ion exchange.}
    \end{choices}
\end{question}
}

\element{cpo-mc}{
\begin{question}{ch11-mc-Q103}
    Isotope $A$ has a half-life measured in minutes, whereas isotope $B$ has a half-life of millions of years.
    Which is more radioactive?
    \begin{choices}
        \wrongchoice{isotope $A$}
        \wrongchoice{isotope $B$}
        \wrongchoice{Both are equally dangerous.}
        \wrongchoice{It depends on the sample size.}
    \end{choices}
\end{question}
}

\element{cpo-mc}{
\begin{question}{ch11-mc-Q104}
    A measure of radiation that takes into account the possible biological damage produced by different types of radiation is called a
    \begin{choices}
        %% NOTE: SI Units format
        \wrongchoice{rem.}
        \wrongchoice{rad.}
        \wrongchoice{roentgen.}
        \wrongchoice{curie.}
    \end{choices}
\end{question}
}

\element{cpo-mc}{
\begin{question}{ch11-mc-Q105}
    The radioactive isotope $Z$ has a half-life of \SI{12}{\hour}.
    After \SI{2}{\day}, the fraction of the original amount remaining is:
    \begin{choices}
        \wrongchoice{\num{1/2}.}
        \wrongchoice{\num{1/4}.}
        \wrongchoice{\num{1/8}.}
        \wrongchoice{\num{1/16}.}
    \end{choices}
\end{question}
}

\element{cpo-mc}{
\begin{question}{ch11-mc-Q106}
    The number of oxygen atoms in \ce{Al2(SO4)3} is
    \begin{choices}
        \wrongchoice{\num{3}.}
        \wrongchoice{\num{4}.}
        \wrongchoice{\num{7}.}
      \correctchoice{\num{12}.}
    \end{choices}
\end{question}
}

\element{cpo-mc}{
\begin{question}{ch11-mc-Q107}
    When the equation __Li + __O2 → __Li2O is correctly balanced, the sum of the coefficients is
    \begin{choices}
        %% NOTE: reword using \ce{ }
        \wrongchoice{3.}
        \wrongchoice{5.}
        \wrongchoice{7.}
        \wrongchoice{12.}
    \end{choices}
\end{question}
}

\element{cpo-mc}{
\begin{question}{ch11-mc-Q108}
    The number of atoms in a molecule of ammonium sulfate, \ce{(NH4)2SO4}, is:
    \begin{choices}
        \wrongchoice{\num{4}.}
        \wrongchoice{\num{10}.}
        \wrongchoice{\num{14}.}
        \wrongchoice{\num{15}.}
    \end{choices}
\end{question}
}

\element{cpo-mc}{
\begin{question}{ch11-mc-Q109}
    Covalent bonds are formed when
    \begin{choices}
        \wrongchoice{electrons are transferred from the excited to the ground state.}
        \wrongchoice{electrons are transferred from the ground state to the excited state.}
        \wrongchoice{electrons are transferred between atoms.}
        \wrongchoice{atoms share electrons.}
    \end{choices}
\end{question}
}

\element{cpo-mc}{
\begin{question}{ch11-mc-Q110}
    Ionic bonds are formed when
    \begin{choices}
        \wrongchoice{electrons are transferred from the excited to the ground state.}
        \wrongchoice{electrons are transferred from the ground to the excited state.}
        \wrongchoice{electrons are transferred between atoms.}
        \wrongchoice{atoms share electrons.}
    \end{choices}
\end{question}
}

\element{cpo-mc}{
\begin{question}{ch11-mc-Q111}
    In a chemical reaction the element sodium (\ce{Na}) will
    \begin{choices}
        \wrongchoice{lose an electron.}
        \wrongchoice{lose two electrons.}
        \wrongchoice{gain an electron.}
        \wrongchoice{neither gain nor lose electrons.}
    \end{choices}
\end{question}
}

\element{cpo-mc}{
\begin{question}{ch11-mc-Q112}
    Which of the following represents the most stable outer orbital arrangement of electrons after a chemical reaction?
    \begin{choices}
        \wrongchoice{1 electron.}
        \wrongchoice{3 electrons.}
        \wrongchoice{6 electrons.}
        \wrongchoice{8 electrons.}
    \end{choices}
\end{question}
}


%% Alternative Multiple Choice Questions
%%--------------------------------------------------
\element{cpo-mc}{
\begin{question}{ch15-mc-Q101}
    How does an object become electrostatically charged?
    \begin{choices}
        \wrongchoice{from a transfer of protons or electrons}
        \wrongchoice{when electrons are transferred}
        \wrongchoice{when protons are transferred}
        \wrongchoice{from the creation or destruction of charge}
    \end{choices}
\end{question}
}

\element{cpo-mc}{
\begin{question}{ch15-mc-Q102}
    A coulomb is a unit of electrical:
    \begin{multicols}{2}
    \begin{choices}
      \correctchoice{charge.}
        \wrongchoice{potential difference.}
        \wrongchoice{current.}
        \wrongchoice{resistance.}
    \end{choices}
    \end{multicols}
\end{question}
}

\element{cpo-mc}{
\begin{question}{ch15-mc-Q103}
    A volt is a unit of electrical:
    \begin{multicols}{2}
    \begin{choices}
        \wrongchoice{charge.}
      \correctchoice{potential difference.}
        \wrongchoice{current.}
        \wrongchoice{resistance.}
    \end{choices}
    \end{multicols}
\end{question}
}

\element{cpo-mc}{
\begin{question}{ch15-mc-Q104}
    Which of the following requires a measure of time?
    \begin{multicols}{2}
    \begin{choices}
        \wrongchoice{volt (\si{\volt})}
        \wrongchoice{coulomb (\si{\coulomb})}
        \wrongchoice{joule (\si{\joule})}
      \correctchoice{watt (\si{\watt})}
    \end{choices}
    \end{multicols}
\end{question}
}

\element{cpo-mc}{
\begin{question}{ch15-mc-Q105}
    If an electric charge is somehow suddenly neutralized, the electric field that surrounds it will
    \begin{choices}
        \wrongchoice{immediately cease to exist.}
        \wrongchoice{collapse inward at the speed of light.}
        \wrongchoice{continue to exist until neutralized.}
        \wrongchoice{move off into space until it finds another charge}
    \end{choices}
\end{question}
}

\element{cpo-mc}{
\begin{question}{ch15-mc-Q106}
    A kilowatt-hour is a unit of:
    \begin{multicols}{2}
    \begin{choices}
        \wrongchoice{power.}
        \wrongchoice{charge.}
        \wrongchoice{work.}
        \wrongchoice{current.}
    \end{choices}
    \end{multicols}
\end{question}
}

\element{cpo-mc}{
\begin{question}{ch15-mc-Q107}
    If you multiply amps times volts the answer will have the unit
    \begin{multicols}{2}
    \begin{choices}
        \wrongchoice{ohm (\si{\ohm}).}
        \wrongchoice{joule (\si{\joule}).}
        \wrongchoice{amp (\si{\ampere}).}
        \wrongchoice{watt (\si{\watt}).}
    \end{choices}
    \end{multicols}
\end{question}
}


%% Alternative Multiple Choice Questions
%%--------------------------------------------------
\element{cpo-mc}{
\begin{question}{ch19-mc-Q101}
    If you increase the energy that goes into starting a vibration, you will increase the
    \begin{choices}
        \wrongchoice{frequency.}
        \wrongchoice{amplitude.}
        \wrongchoice{number of cycles per second.}
        \wrongchoice{wavelength.}
    \end{choices}
\end{question}
}

\element{cpo-mc}{
\begin{question}{ch19-mc-Q102}
    The time required for one cycle of any repeating event is called one
    \begin{choices}
        \wrongchoice{hertz.}
        \wrongchoice{period.}
        \wrongchoice{frequency.}
        \wrongchoice{amplitude.}
    \end{choices}
\end{question}
}

\element{cpo-mc}{
\begin{question}{ch19-mc-Q103}
    The time for one cycle and the frequency of a vibration have a relationship of the frequency varying
    \begin{choices}
        \wrongchoice{directly with the time.}
        \wrongchoice{directly with the amplitude.}
        \wrongchoice{inversely with the time.}
        \wrongchoice{inversely with the wavelength.}
    \end{choices}
\end{question}
}


%% Alternative Multiple Choice Questions
%%--------------------------------------------------
\element{cpo-mc}{
\begin{question}{ch20-mc-Q101}
    Particles of a material that move back and forth in the same direction the wave is moving are in what type of wave?
    \begin{multicols}{2}
    \begin{choices}
        \wrongchoice{longitudinal}
        \wrongchoice{transverse}
        \wrongchoice{torsional}
        \wrongchoice{standing}
    \end{choices}
    \end{multicols}
\end{question}
}

\element{cpo-mc}{
\begin{question}{ch20-mc-Q102}
    Particles of a material that move up and down perpendicular to the direction the wave is moving are in what type of wave?
    \begin{multicols}{2}
    \begin{choices}
        \wrongchoice{longitudinal}
        \wrongchoice{transverse}
        \wrongchoice{torsional}
        \wrongchoice{standing}
    \end{choices}
    \end{multicols}
\end{question}
}

\element{cpo-mc}{
\begin{question}{ch20-mc-Q103}
    The extent of displacement of a vibrating tuning fork is related to the resulting sound wave characteristic of
    \begin{multicols}{2}
    \begin{choices}
        \wrongchoice{frequency.}
        \wrongchoice{amplitude.}
        \wrongchoice{wavelength.}
        \wrongchoice{period.}
    \end{choices}
    \end{multicols}
\end{question}
}

\element{cpo-mc}{
\begin{question}{ch20-mc-Q104}
    An efficient transfer of energy that takes place at a natural frequency is known as
    \begin{multicols}{2}
    \begin{choices}
        \wrongchoice{resonance.}
        \wrongchoice{beats.}
        \wrongchoice{the Doppler effect.}
        \wrongchoice{reverberation.}
    \end{choices}
    \end{multicols}
\end{question}
}


%% Alternative Multiple Choice Questions
%%--------------------------------------------------
\element{cpo-mc}{
\begin{question}{ch21-mc-Q101}
    Sound waves travel faster in:
    \begin{multicols}{2}
    \begin{choices}
        \wrongchoice{icy cold air.}
        \wrongchoice{a vacuum.}
      \correctchoice{warm air.}
        \wrongchoice{cool air.}
    \end{choices}
    \end{multicols}
\end{question}
}

\element{cpo-mc}{
\begin{question}{ch21-mc-Q102}
    You hear a higher pitch when a sound wave has a greater
    \begin{multicols}{2}
    \begin{choices}
        \wrongchoice{amplitude.}
        \wrongchoice{velocity.}
        \wrongchoice{frequency.}
        \wrongchoice{wavelength.}
    \end{choices}
    \end{multicols}
\end{question}
}

\element{cpo-mc}{
\begin{question}{ch21-mc-Q103}
    Both constructive and destructive interference is necessary to produce the sound phenomena known as
    \begin{multicols}{2}
    \begin{choices}
        \wrongchoice{resonance.}
        \wrongchoice{refraction.}
        \wrongchoice{beats.}
        \wrongchoice{diffusion.}
    \end{choices}
    \end{multicols}
\end{question}
}


%% Alternative Multiple Choice Questions
%%--------------------------------------------------
\element{cpo-mc}{
\begin{question}{ch23-mc-Q101}
    A virtual image is:
    \begin{choices}
        \wrongchoice{produced by light rays.}
        \wrongchoice{your brain's interpretations of light rays.}
        \wrongchoice{found only on the Internet.}
        \wrongchoice{a real image produced by computer tricks with light.}
    \end{choices}
\end{question}
}

\element{cpo-mc}{
\begin{question}{ch23-mc-Q102}
    As the angle of an incident ray of light on a water surface increases, the angle of the refracted ray
    \begin{choices}
        \wrongchoice{also increases.}
        \wrongchoice{stays the same.}
        \wrongchoice{decreases.}
        \wrongchoice{none of the above.}
    \end{choices}
\end{question}
}

\element{cpo-mc}{
\begin{question}{ch23-mc-Q103}
    When viewed straight down (\ang{90} to the surface), an incident light ray moving from the water to air is refracted
    \begin{choices}
        \wrongchoice{away from the normal.}
        \wrongchoice{toward the normal.}
        \wrongchoice{not at all.}
        \wrongchoice{about 49°.}
    \end{choices}
\end{question}
}

\element{cpo-mc}{
\begin{question}{ch23-mc-Q104}
    A mirage is produced because
    \begin{choices}
        \wrongchoice{warm air has a higher index of refraction than cool air.}
        \wrongchoice{water from oceans and lakes is highly reflective.}
        \wrongchoice{images of water are reflected from the sky.}
        \wrongchoice{light travels faster through air than through water.}
    \end{choices}
\end{question}
}

\element{cpo-mc}{
\begin{question}{ch23-mc-Q105}
    The index of refraction is based on the ratio of the speed of light in
    \begin{choices}
        \wrongchoice{two transparent materials.}
        \wrongchoice{air to the speed of light in the transparent material.}
        \wrongchoice{water to the speed of light in the transparent material.}
        \wrongchoice{a vacuum to the speed of light in the transparent material.}
    \end{choices}
\end{question}
}

\element{cpo-mc}{
\begin{question}{ch23-mc-Q106}
    The shimmering that is observed over a hot surface is
    \begin{choices}
        \wrongchoice{heat.}
        \wrongchoice{reflections from rising heat.}
        \wrongchoice{changing refraction from the mixing of warm and cool air.}
        \wrongchoice{a mirage.}
    \end{choices}
\end{question}
}


%% NOTE: start here

%% Chapter 01
A cannonball is fired straight up at 50 m/s.
Neglecting air resistance, when it returns to its starting point its speed
    A)  is 50 m/s.
    B)  is more than 50 m/s.
    C)  is less than 50 m/s.
    D)  depends on how long it is in the air.

Doubling the distance between an orbiting satellite and the earth will result in what change in the gravitational attraction between the two?
    A)  one-half as much
    B)  one-fourth as much
    C)  twice as much
    D)  four times as much

%% Chapter 03
Most energy comes to and leaves the earth in the form of
    A)  nuclear energy.
    B)  chemical energy
    C)  radiant energy.
    D)  kinetic energy.

%% Chapter 04

%% Chapter 05

%% Chapter 06



Electrical, light, and chemical phenomena all involve movement or interactions of which part of an atom?
    A)  protons
    B)  neutrons
    C)  electrons
    D)  the nucleus

An amp is a unit of electrical
    A)  charge.
    B)  potential difference.
    C)  current.
    D)  resistance.

What travels through a conductor at near the speed of light when a current is established?
    A)  an electric field
    B)  electrons
    C)  protons
    D)  photons

%% Chapter 07
Light with the lowest frequency (longest wavelength) detected by your eyes is perceived as
    A)  blue.
    B)  green.
    C)  orange.
    D)  red.

According to the relationship between frequency and energy of light (E = hf), which color of light has more energy?
    A)  blue
    B)  green
    C)  orange
    D)  red

Which of the following cannot be explained with a wave theory of light?
    A)  polarization.
    B)  interference.
    C)  photoelectric effect.
    D)  all of the above.

Which of the following cannot be explained with a particle theory of light?
    A)  polarization
    B)  quantization of energy
    C)  photoelectric effect
    D)  all of the above

%% Chapter 08
Which of the following cannot be broken down to anything simpler?
    A)  water
    B)  table salt
    C)  silver
    D)  sugar

Elements combine in fixed mass ratios to form compounds. This must mean that elements
    A)  are made up of continuous matter without subunits.
    B)  are composed of discrete units called atoms.
    C)  have unambiguous atomic numbers.
    D)  are always chemically active.

The electron was discovered through experiments with
    A)  electricity.
    B)  light.
    C)  radio waves.
    D)  radioactivity.

The idea of matter waves, as reasoned by de Broglie, describes a wavelike behavior of any
    A)  particle, moving or not.
    B)  particle that is moving.
    C)  charged particle that is moving.
    D)  particle that is stationary.

%% Chapter 09
Ozone is a form of oxygen that is
    A)  monatomic.
    B)  diatonic.
    C)  triatomic.
    D)  tetratomic.

Chemical energy is stored within molecules as
    A)  thermal energy.
    B)  internal activation energy.
    C)  internal potential energy.
    D)  kinetic energy of the nucleus.

%% Chapter 10


For Questions 1 & 2: consider the chemical equation: CH4 + 2 O2 → CO2 + 2 H2O

The number of atoms on each side of the equation is
    A)  5.
    B)  6.
    C)  7.
    D)  9.

The total mass of products in the methane reaction above is
    A)  80 u.
    B)  64 u.
    C)  62 u.
    D)  48 u.

The formula weight of methane, CH4 is
    A)  13 u.
    B)  16 u.
    C)  52 u.
    D)  none of these.
The reaction: 2 NaI + Cl2 → 2 NaCl + I2 is an example of
    A)  decomposition.
    B)  combination.
    C)  replacement.
    D)  ion exchange.

The reaction: potassium chlorate Δ→ potassium chloride and oxygen gas is an example of
    A)  decomposition.
    B)  combination.
    C)  replacement.
    D)  ion exchange.

The equation 2 C2H5OH + __O2 → 4 CO2 + 6 H2O is balanced by making the coefficient of oxygen
    A)  14.
    B)  12.
    C)  7.
    D)  6.

%% Chapter 11
Water solutions of ionic substances that conduct electricity are called
    A)  electrical solutions.
    B)  polar solutions.
    C)  electrolytes.
    D)  indicators.

Which of the following are properties of basic solutions?
    A)  They turn the dye litmus red.
    B)  They taste sour.
    C)  They feel slippery.
    D)  They react with active metals to produce hydrogen gas.

Which of the following are properties of acidic solutions?
    A)  They turn the dye litmus red.
    B)  They taste sour.
    C)  They react with active metals to produce hydrogen gas.
    D)  All of these are true.

A bottle of whiskey contains 40% alcohol by volume.
This means that the whiskey contains 40 mL of alcohol
    A)  in every 100 mL of whiskey.
    B)  mixed with 100 mL of water.
    C)  mixed with 60 mL of whiskey.
    D)  in every 140 mL of whiskey.

The water hardness in an area is reported as 700 ppm total dissolved solids. This is the same concentration as
    A)  0.007%.
    B)  0.07%.
    C)  0.7%.
    D)  7%.

If the force of attraction between the ions in a solid is very strong, you would expect the solid to have
    A)  low solubility in water.
    B)  high solubility in water.
    C)  low solubility in a non-polar solvent.
    D)  high solubility in a non-polar solvent.

According to the modern definition, NH3 is
    A)  an acid because it contains hydrogen.
    B)  a base because it can accept H+ ions from water.
    C)  an acid because it raises the H+ ion concentration.
    D)  a base because it lowers the pH.

A solution with a pH of 2 is
    A)  twice as acidic as one with a pH of 1.
    B)  half as acidic as a solution with a pH of 1.
    C)  ten times as acidic as a solution with a pH of 1.
    D)  one-tenth as acidic as a solution with a pH of 1.

Which of the following solutions is likely to have a pH less than 7?
    A)  sodium chloride
    B)  ammonia
    C)  carbonic acid
    D)  pure water

%% Chapter 12
Hydrocarbons with single carbon-to-carbon bonds are known as
    A)  alkanes.
    B)  alkenes.
    C)  alkynes.
    D)  aromatic hydrocarbons.

The R in R-COOH or R-C=O stands for
    A)  a reactive atom.
    B)  a separate functional group.
    C)  any hydrocarbon group.
    D)  a rigid part of the molecule.

To which category does the molecule below belong?
    A)  alkane
    B)  alkene
    C)  alkyne
    D)  aromatic hydrocarbon

Sucrose is an example of a
    A)  monosaccharide.
    B)  disaccharide.
    C)  polysaccharide.
    D)  fruit sugar.

An alkane with 3 carbon atoms would have how many hydrogen atoms in the molecule?
    A)  4
    B)  6
    C)  8
    D)  10

The IUPAC name for the molecule below is
    A)  octane.
    B)  3-ethyl-4-methylpentane.
    C)  methylheptane.
    D)  3-ethyl-2-methylpentane.

Compounds with the general structural formula RCOOH are known as
    A)  esters.
    B)  carboxylic acids.
    C)  aldehydes.
    D)  fats.

Starch is a polymer formed by the linking of many
    A)  saccharides.
    B)  amino acids.
    C)  triglycerides.
    D)  peptides.

When wine ``goes bad,'' the ethanol is converted into
    A)  CH3COOH.
    B)  CH3OCH3.
    C)  CH3CH2OH.
    D)  CH3OH.

Organic compounds responsible for flavors in fruits and the scent of flowers are
    A)  esters.
    B)  ethers.
    C)  alcohols.
    D)  ketones.

%% Chapter 13
This type of radiation is released when decays to:
    A)  alpha
    B)  beta
    C)  gamma
    D)  all of these

The decay rate of a radioactive isotope can be increased by increasing the
    A)  temperature.
    B)  pressure.
    C)  size of the sample.
    D)  none of these.

The protons in a nucleus stay together due to the
    A)  electrostatic force of attraction.
    B)  nuclear force.
    C)  gravitational force.
    D)  binding force.

The isotope is most likely to emit
    A)  an alpha particle.
    B)  a beta particle.
    C)  a gamma ray.
    D)  It is not possible to predict.

When an isotope releases gamma radiation, the atomic number
    A)  decreases by two and the mass number decreases by four.
    B)  increases by one and the mass number remains the same.
    C)  and the mass number decrease by one.
    D)  and the mass number remain the same.

%% Chapter 14
Retrograde motion is a term used to describe the fact that
    A)  some planets travel in the opposite direction than the earth.
    B)  some planets rotate in a direction opposite of the earth's rotation.
    C)  planets revolve around the sun on epicycles.
    D)  planets sometimes appear to move backward in their path around the ecliptic.

The difference in brightness between two stars is related to
    A)  the amount of light and energy produced by the stars.
    B)  the relative size of the stars.
    C)  the distances between the stars.
    D)  all of these.

Which of the following depends on your location on the earth?
    A)  celestial pole
    B)  celestial meridian
    C)  celestial equator
    D)  all of these depend on your location.

The earth rotates
    A)  from west to east.
    B)  from east to west.
    C)  at the same rate as it revolves around the sun.
    D)  twice each day.

If the density of the universe is less than a critical value, then
    A)  it will eventually contract to a concentrated mass.
    B)  it will go on expanding forever.
    C)  it could expand to a fixed size and remain.
    D)  there's probably less dark matter than luminous matter.

The expected life span of a star the size of the sun is
    A)  10 billion years.
    B)  1 billion years.
    C)  100 million years.
    D)  10 million years.

Compared to a star like the sun, a blue-white star
    A)  is more massive.
    B)  has a higher surface temperature.
    C)  burns its fuel at a greater rate.
    D)  all of these.

The eventual fate of our sun is to become a
    A)  neutron star.
    B)  supernova.
    C)  white dwarf.
    D)  black hole.

The Milky Way galaxy is
    A)  spiral.
    B)  elliptical.
    C)  globular.
    D)  spherical.

Most of the stars plotted on a H-R diagram are
A)  red giant stars.
B)  white dwarf stars.
C)  cepheid variables.
D)  main sequence stars


%% Chapter 15
Which of the following statements about the length of a planet's day is true?
A)  The closer the planet is to the sun, the shorter is its day.
B)  The more massive the planet, the longer its day.
C)  The four gas giants barely rotate at all.
D)  None of these are true.

Which of the following is true about the surface of Mars?
    A)  There are numerous active volcanoes.
    B)  It is much too cold for liquid water to exist.
    C)  The polar ice caps are frozen water.
    D)  The greenish areas are due to low, scrubby vegetation.

Which planets do not have at least one moon?
    A)  Mercury
    B)  Venus
    C)  The first and second planet out from the sun.
    D)  All of the planets have at least one moon.

Most of the mass of the solar system is found in
    A)  the terrestrial planets.
    B)  the gas giants.
    C)  the comets and asteroids.
    D)  the sun.

The smallest planet in the solar system is
    A)  Mercury.
    B)  Mars.
    C)  Pluto.
    D)  Venus.

The atmosphere of Venus is predominantly
    A)  carbon dioxide.
    B)  nitrogen.
    C)  methane.
    D)  water vapor.

Remnants of asteroids or comets that survive the trip through the earth's atmosphere to strike the surface are called
    A)  meteors.
    B)  meteorites.
    C)  meteoroids.
    D)  ``shooting stars''.

Which of the following about Jupiter is not true?
    A)  Jupiter is mostly made of light elements like hydrogen and helium.
    B)  Much of the planet is composed of hydrogen compressed so greatly that it is liquid.
    C)  The colored bands in Jupiter's atmosphere are believed to result from atmospheric convection.
    D)  Because Jupiter is mostly hydrogen, it is only slightly more massive than the earth.


%% Chapter 16

During which month of the year does the earth receive the greatest average energy from the sun?
    A)  January
    B)  March
    C)  June
    D)  September

How frequently does the sun appear directly overhead in Mexico City (≈20° N latitude)?
    A)  daily
    B)  once a year
    C)  twice a year
    D)  four times a year

The slow wobble of the earth's axis is called
    A)  Coriolis.
    B)  solstice.
    C)  perigee.
    D)  precession.

An apparent solar day is
    A)  the interval between two consecutive solar noons.
    B)  slightly longer than a sidereal day.
    C)  rarely exactly 24 hours long.
    D)  all of these

Evidence that the earth is rotating is provided by
    A)  the motion of a pendulum.
    B)  seasonal climactic changes.
    C)  the varying solar day.
    D)  the movement of the constellations.

If your favorite constellation rose at 8:00 PM one night, when would it rise two weeks later?
    A)  at 8:00 PM
    B)  around 7:00 PM
    C)  around 9:00 PM
    D)  it depends on the constellation

North of the Tropic of Cancer, the sun appears directly overhead at noon
    A)  never.
    B)  daily.
    C)  twice a year, at the equinoxes.
    D)  twice a year, at the solstices.

South of the Antarctic Circle, the sun appears
    A)  directly overhead only once a year.
    B)  directly overhead twice yearly.
    C)  above the horizon all day at least once during December.
    D)  above the horizon all day at least once during June.

In the time 9:00 AM, the ``AM'' means
    A)  after morning.
    B)  after the meridian.
    C)  astronomical motion.
    D)  before the meridian.

During which phase of the moon would a solar eclipse occur?
    A)  full
    B)  first quarter
    C)  last quarter
    D)  new

