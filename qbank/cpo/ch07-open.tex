
%%--------------------------------------------------
%% CPO: AMC Open Free Response Questions
%%--------------------------------------------------


%% Chapter 7: Temperature, Energy and Matter
%%--------------------------------------------------


%% CPO Short Answer Questions
%%--------------------------------------------------
\element{cpo-short}{
\begin{question}{ch07-short-q01}
    List three physical properties that thermometers use
        to detect temperatures.
    \AMCOpen{lines=3}{
        \wrongchoice[W]{w}\scoring{0}
        \wrongchoice[P]{p}\scoring{1}
        \correctchoice[C]{c}\scoring{2}
    }
    %% ANS: color change, volume change, change in electrical resistance
\end{question}
}

\element{cpo-short}{
\begin{question}{ch07-short-q02}
    The graph of temperature versus time represents the changes
        in the temperature of a sample of gas as it is cooled at
        a constant rate.
    \begin{center}
        %% NOTE: add graph
    \end{center}
    What changes in the substance are occurring during the intervals
        from time \SI{6}{\minute} to \SI{14}{\minute} and from time
        \SI{30}{\minute} to \SI{34}{\minute}?
    \AMCOpen{lines=3}{
        \wrongchoice[W]{w}\scoring{0}
        \wrongchoice[P]{p}\scoring{1}
        \correctchoice[C]{c}\scoring{2}
    }
    %% ANS: gas to liquid and liquid to soli
\end{question}
}

\element{cpo-short}{
\begin{question}{ch07-short-q03}
    Name the two physical properties that determine the amount of thermal
        energy in a material whose specific heat is known.
    \AMCOpen{lines=3}{
        \wrongchoice[W]{w}\scoring{0}
        \wrongchoice[P]{p}\scoring{1}
        \correctchoice[C]{c}\scoring{2}
    }
    %% ANS: mass and temeprature
\end{question}
}

\element{cpo-short}{
\begin{question}{ch07-short-q04}
    Name three units commonly used for measure heat energy.
    Arrange them in order from smallest to largest unit.
    \AMCOpen{lines=3}{
        \wrongchoice[W]{w}\scoring{0}
        \wrongchoice[P]{p}\scoring{1}
        \correctchoice[C]{c}\scoring{2}
    }
\end{question}
}

\element{cpo-short}{
\begin{question}{ch07-short-q05}
    Heat may be transferred by conduction, convection and radiation.
    By which method is it transferred most rapidly?
    \AMCOpen{lines=3}{
        \wrongchoice[W]{w}\scoring{0}
        \wrongchoice[P]{p}\scoring{1}
        \correctchoice[C]{c}\scoring{2}
    }
    %% ANS: radiation
    %% NOTE: I consider this very advanced
\end{question}
}

\element{cpo-short}{
\begin{question}{ch07-short-q06}
    Explain in one sentence how thermal equilibrium may be established by
        heat flow between two object of different temperature.
    \AMCOpen{lines=3}{
        \wrongchoice[W]{w}\scoring{0}
        \wrongchoice[P]{p}\scoring{1}
        \correctchoice[C]{c}\scoring{2}
    }
\end{question}
}

\element{cpo-short}{
\begin{question}{ch07-short-q07}
    A thermos, a two-layered bottle, is effective in maintaining the temperature
        of a substance inside because a vacuum is maintained between the
        inside and the outside container.
    Name the type of heat transfer the thermos is \emph{least} effective in
        preventing.
    \AMCOpen{lines=3}{
        \wrongchoice[W]{w}\scoring{0}
        \wrongchoice[P]{p}\scoring{1}
        \correctchoice[C]{c}\scoring{2}
    }
    %% ANS: radiation
\end{question}
}


%% CPO Problem Questions
%%--------------------------------------------------
\element{cpo-problem}{
\begin{question}{ch07-problem-q01}
    The average body temperature for a human body is \ang{98.6}F.
    Calculate the equivalent temperature in degree Celsius.
    \AMCOpen{lines=3}{
        \wrongchoice[W]{w}\scoring{0}
        \wrongchoice[P]{p}\scoring{1}
        \correctchoice[C]{c}\scoring{2}
    }
    %% ANS: \SI{0.64}{\second}
\end{question}
}

\element{cpo-problem}{
\begin{question}{ch07-problem-q02}
    Room temperature is give as \SI{20}{\degreeCelsius}.
    What is this temperature given in kelvin (\si{\kelvin})?
    \AMCOpen{lines=3}{
        \wrongchoice[W]{w}\scoring{0}
        \wrongchoice[P]{p}\scoring{1}
        \correctchoice[C]{c}\scoring{2}
    }
    %% ANS: \SI{293}{\kelvin}
\end{question}
}

\element{cpo-problem}{
\begin{question}{ch07-problem-q03}
    Convert the temperature of \SI{233}{\kelvin} to the
        equivalent in degree Fahrenheit.
    \AMCOpen{lines=3}{
        \wrongchoice[W]{w}\scoring{0}
        \wrongchoice[P]{p}\scoring{1}
        \correctchoice[C]{c}\scoring{2}
    }
    %% ANS: \ang{-40}F
\end{question}
}

\element{cpo-problem}{
\begin{question}{ch07-problem-q04}
    The specific heat of concrete is \SI{880}{\joule\per\kilo\gram\per\degreeCelsius}.
    How much heat energy from the sun would be needed to raise the temperature
        of a \SI{1 500}{\kilo\gram} block on a concrete sidewalk from
        \SI{0}{\degreeCelsius} to \SI{20}{\degreeCelsius}?
    \AMCOpen{lines=3}{
        \wrongchoice[W]{w}\scoring{0}
        \wrongchoice[P]{p}\scoring{1}
        \correctchoice[C]{c}\scoring{2}
    }
    %% ANS: \SI{2.64e7}{\joule}
\end{question}
}

\element{cpo-problem}{
\begin{question}{ch07-problem-q05}
    A \SI{0.22}{\kilo\gram} block of aluminum with a temperature of 
        \SI{320}{\degreeCelsius} is added to a \SI{1.0}{\kilo\gram}
        of water with temperature of \SI{5}{\degreeCelsius}.
    Assuming no heat is lost to the air, what is the final
        temperature of the aluminum block?
    \AMCOpen{lines=3}{
        \wrongchoice[W]{w}\scoring{0}
        \wrongchoice[P]{p}\scoring{1}
        \correctchoice[C]{c}\scoring{2}
    }
    %% ANS: \SI{19}{\degreeCelsius}
\end{question}
}


%% CPO Essay Questions
%%--------------------------------------------------
\element{cpo-essay}{
\begin{question}{ch07-essay-q01}
    List and describe the four phases of matter.
    Give a common example of each.
    \AMCOpen{lines=3}{
        \wrongchoice[W]{w}\scoring{0}
        \wrongchoice[P]{p}\scoring{1}
        \correctchoice[C]{c}\scoring{2}
    }
\end{question}
}

\element{cpo-essay}{
\begin{question}{ch07-essay-q02}
    As water is heated from \SI{-20}{\degreeCelsius} to
        \SI{20}{\degreeCelsius}, there is a period of time
        during which the temperature does not rise.
    Identify the process occurring while the temperature is not
        rising and explain why the temperature does not rise.
    \AMCOpen{lines=3}{
        \wrongchoice[W]{w}\scoring{0}
        \wrongchoice[P]{p}\scoring{1}
        \correctchoice[C]{c}\scoring{2}
    }
\end{question}
}

\element{cpo-essay}{
\begin{question}{ch07-essay-q03}
    On a sunny day, why does the water of the ocean seem cool
        and the sand on the beach feel hot, yet in the evening
        the water feels warm and the sand cool?
    \AMCOpen{lines=3}{
        \wrongchoice[W]{w}\scoring{0}
        \wrongchoice[P]{p}\scoring{1}
        \correctchoice[C]{c}\scoring{2}
    }
\end{question}
}

\element{cpo-essay}{
\begin{question}{ch07-essay-q04}
    Explain why the specific heat of a dense material like gold
        is lower than the specific heat of a less dense material
        like aluminum.
    \AMCOpen{lines=3}{
        \wrongchoice[W]{w}\scoring{0}
        \wrongchoice[P]{p}\scoring{1}
        \correctchoice[C]{c}\scoring{2}
    }
\end{question}
}

\element{cpo-essay}{
\begin{question}{ch07-essay-q05}
    Jill and Sarah sit down together to enjoy a cup of cocoa with
        marshmallows.
    Sarah stirs her marshmallows in with a spoon made of gold.
    Jill stirs her cocoa using a stainless steel spoon.
    Sarah complains that her gold spoon is too hot to touch.
    Jill does not agree and thinks that Sarah is being whiney.
    Using your knowledge of physics, offer an excuse for Sarah.
    \AMCOpen{lines=3}{
        \wrongchoice[W]{w}\scoring{0}
        \wrongchoice[P]{p}\scoring{1}
        \correctchoice[C]{c}\scoring{2}
    }
\end{question}
}

\element{cpo-essay}{
\begin{question}{ch07-essay-q06}
    On a warm-than-average winter day, Tom is supposed to shovel snow from
        the sidewalk.
    Instead, he spreads black ashes from the wood stove on the snow.
    He claims this will clear the snow from the sidewalk.
    Explain why Tom may be correct.
    \AMCOpen{lines=3}{
        \wrongchoice[W]{w}\scoring{0}
        \wrongchoice[P]{p}\scoring{1}
        \correctchoice[C]{c}\scoring{2}
    }
\end{question}
}

\element{cpo-essay}{
\begin{question}{ch07-essay-q07}
    Two metal chains are located outside on a cold winter day.
    The temperature of both chairs are measured to be \ang{15}F.
    The seat of one chair is covered with a layer of styrofoam.
    the seat of the other is not.
    Explain why the seat without the styrofoam layer feels colder to sit on.
    \AMCOpen{lines=3}{
        \wrongchoice[W]{w}\scoring{0}
        \wrongchoice[P]{p}\scoring{1}
        \correctchoice[C]{c}\scoring{2}
    }
\end{question}
}

\endinput


