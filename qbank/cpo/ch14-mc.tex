
%%--------------------------------------------------
%% CPO: Multiple Choice Questions
%%--------------------------------------------------


%% Chapter 14: Electrical Systems
%%--------------------------------------------------


%% Learning Objectives
%%--------------------------------------------------

%% Describe a series circuit. 
%% Calculate the resistance and current in a series circuit. 
%% Explain how the voltage changes across each resistor in a series circuit. 
%% Describe how current divides in a parallel circuit.
%% Determine the voltage across and current through each branch of a parallel circuit. 
%% Explain why circuit breakers and fuses are used in homes. 
%% Calculate power in a circuit. 
%% Calculate the cost of running an appliance. 
%% Distinguish between alternating and direct current.


%% CPO Multiple Choice Questions
%%--------------------------------------------------
\element{cpo-mc}{
\begin{question}{cpo-ch14-q01}
    If there is a break at any point in a series circuit, the current will:
    \begin{choices}
      \correctchoice{stop everywhere in the circuit.}
        \wrongchoice{leak out of the break point.}
        \wrongchoice{be decreased by one-half.}
        \wrongchoice{continue through remaining unbroken circuit branches.}
    \end{choices}
\end{question}
}

\element{cpo-mc}{
\begin{question}{cpo-ch14-q02}
    When more devices are added to a series circuit,
        the total circuit resistance:
    \begin{choices}
        \wrongchoice{decreases.}
      \correctchoice{increases.}
        \wrongchoice{stays the same.}
        \wrongchoice{may increase or decrease, depending on the device.}
    \end{choices}
\end{question}
}

\element{cpo-mc}{
\begin{question}{cpo-ch14-q03}
    Suppose you connect more and more light bulbs in series to a battery.
    What happens to the brightness of each bulb as you add more bulbs?
    \begin{choices}
        \wrongchoice{The bulb grow brighter with each new bulb added.}
      \correctchoice{The bulbs grow dimmer with each new bulb added.}
        \wrongchoice{There is no change in brightness.}
        \wrongchoice{The brightness may increase or decrease depending on the type of light bulb.}
    \end{choices}
\end{question}
}

%% NOTE: circuit diagram used for Q04 to Q07
\newcommand{\circuitSeriesOne}{
    \ctikzset{bipoles/length=0.75cm}
    \begin{circuitikz}[scale=1.75]
        \draw (0,0) to [battery,l=\SI{9.0}{\volt}] (0,1)
                    to [R,l=\SI{1}{\ohm}] (1,1)
                    to [R,l=\SI{1}{\ohm}] (1,0)
                    to [R,l=\SI{1}{\ohm}] (0,0);
    \end{circuitikz}
}

\element{cpo-mc}{
\begin{question}{cpo-ch14-q04}
    The diagram below pictures three identical light bulbs,
        each with resistance of \SI{1}{\ohm},
        which are connected by resistance-free wires.
    A \SI{9}{\volt} battery supplies energy to the circuit.
    \begin{center}
        \circuitSeriesOne
    \end{center}
    What type of circuit is in the diagram?
    \begin{multicols}{2}
    \begin{choices}
      \correctchoice{Series}
        \wrongchoice{Parallel}
        \wrongchoice{Compound}
        \wrongchoice{Current}
    \end{choices}
    \end{multicols}
\end{question}
}

\element{cpo-mc}{
\begin{question}{cpo-ch14-q05}
    The diagram below pictures three identical light bulbs,
        each with resistance of \SI{1}{\ohm},
        which are connected by resistance-free wires.
    A \SI{9}{\volt} battery supplies energy to the circuit.
    \begin{center}
        \circuitSeriesOne
    \end{center}
    The total resistance for the circuit shown is:
    \begin{multicols}{2}
    \begin{choices}
        \wrongchoice{\SI{1}{\ohm}.}
        \wrongchoice{\SI{1}{\volt}.}
      \correctchoice{\SI{3}{\ohm}.}
        \wrongchoice{\SI{3}{\volt}.}
    \end{choices}
    \end{multicols}
\end{question}
}

\element{cpo-mc}{
\begin{question}{cpo-ch14-q06}
    The diagram below pictures three identical light bulbs,
        each with resistance of \SI{1}{\ohm},
        which are connected by resistance-free wires.
    A \SI{9}{\volt} battery supplies energy to the circuit.
    \begin{center}
        \circuitSeriesOne
    \end{center}
    The total current for the circuit shown is:
    \begin{multicols}{2}
    \begin{choices}
      \correctchoice{\SI{3}{\ampere}.}
        \wrongchoice{\SI{3}{\ohm}.}
        \wrongchoice{\SI{0.33}{\ampere}.}
        \wrongchoice{\SI{1}{\ohm}.}
    \end{choices}
    \end{multicols}
\end{question}
}

\element{cpo-mc}{
\begin{question}{cpo-ch14-q07}
    The diagram below pictures three identical light bulbs,
        each with resistance of \SI{1}{\ohm},
        which are connected by resistance-free wires.
    A \SI{9}{\volt} battery supplies energy to the circuit.
    \begin{center}
        \circuitSeriesOne
    \end{center}
    The voltage drop across each light bulb is:
    \begin{multicols}{2}
    \begin{choices}
      \correctchoice{\SI{3}{\volt}.}
        \wrongchoice{\SI{1}{\volt}.}
        \wrongchoice{\SI{0.33}{\volt}.}
        \wrongchoice{\SI{0}{\volt}.}
    \end{choices}
    \end{multicols}
\end{question}
}

\element{cpo-mc}{
\begin{question}{cpo-ch14-q08}
    The voltage drop across each resistor in the circuit below is:
    \begin{center}
    \ctikzset{bipoles/length=0.75cm}
    \begin{circuitikz}[scale=1.60]
        \draw (0,0) to [battery,l=\SI{9.0}{\volt}] (0,1) to [R,l=\SI{1.5}{\ohm}] (1,1) to [R,l=\SI{1.5}{\ohm}] (1,0) to [R,l=\SI{1.5}{\ohm}] (0,0);
    \end{circuitikz}
    \end{center}
    \begin{multicols}{2}
    \begin{choices}
        \wrongchoice{\SI{1.5}{\volt}}
      \correctchoice{\SI{3}{\volt}}
        \wrongchoice{\SI{2}{\volt}}
        \wrongchoice{\SI{9}{\volt}}
    \end{choices}
    \end{multicols}
\end{question}
}

\element{cpo-mc}{
\begin{question}{cpo-ch14-q09}
    The current in a series circuit:
    \begin{choices}
        \wrongchoice{decreases to zero as it travels through a circuit.}
      \correctchoice{is the same at all points in a circuit.}
        \wrongchoice{is greatest in the resistor with the highest resistance.}
        \wrongchoice{is greatest in the resistor with the lowest resistance.}
    \end{choices}
\end{question}
}

\element{cpo-mc}{
\begin{question}{cpo-ch14-q10}
    The current in each resistor in the circuit below is:
    \begin{center}
    \ctikzset{bipoles/length=0.75cm}
    \begin{circuitikz}[scale=1.60]
        \draw (0,0) to [battery,l=\SI{9.0}{\volt}] (0,1) to [R,l=\SI{3.0}{\ohm}] (1,1) to [R,l=\SI{3.0}{\ohm}] (1,0) to [R,l=\SI{3.0}{\ohm}] (0,0);
    \end{circuitikz}
    \end{center}
    \begin{multicols}{2}
    \begin{choices}
        \wrongchoice{\SI{0.33}{\ampere}}
      \correctchoice{\SI{1}{\ampere}}
        \wrongchoice{\SI{3}{\ampere}}
        \wrongchoice{\SI{9}{\ampere}}
    \end{choices}
    \end{multicols}
\end{question}
}

\element{cpo-mc}{
\begin{question}{cpo-ch14-q11}
    Which of the following is \emph{true} about voltage in a series circuit?
    The total of all voltage drops must:
    \begin{choices}
        \wrongchoice{equal to the total current.}
        \wrongchoice{add up to zero.}
      \correctchoice{add up to the total voltage supplied by the battery.}
        \wrongchoice{never exceed the total circuit resistance.}
    \end{choices}
\end{question}
}

\element{cpo-mc}{
\begin{question}{cpo-ch14-q12}
    A student connects six \SI{3}{\ohm} light bulbs in series to a \SI{9}{\volt} battery.
    The total circuit current is:
    \begin{multicols}{2}
    \begin{choices}
      \correctchoice{\SI{0.5}{\ampere}}
        \wrongchoice{\SI{2}{\ampere}}
        \wrongchoice{\SI{4.5}{\ampere}}
        \wrongchoice{\SI{18}{\ampere}}
    \end{choices}
    \end{multicols}
\end{question}
}

%% NOTE: circuit diagram used for Q13, Q14
\newcommand{\circuitMultiOne}{
    \ctikzset{bipoles/length=0.75cm}
    \begin{circuitikz}[scale=1.00]
        \draw (0,0) to [battery,l=\SI{3.0}{\volt}] (0,2)
                    to (1,2)
                    to [R] (1,1)
                    to [R] (1,0)
                    to (0,0);
        \draw (1,2) to (2,2)
                    to [R] (2,0)
                    to (1,0);
        \draw (2,2) to (3,2)
                    to [R] (3,0)
                    to (2,0);
    \end{circuitikz}
}

\newcommand{\circuitMultiTwo}{
    \ctikzset{bipoles/length=0.75cm}
    \begin{circuitikz}[scale=1.00]
        \draw (0,0) to [battery,l=\SI{1.5}{\volt}] (0,1)
                    to (1,1)
                    to [R] (1,0)
                    to (0,0);
        \draw (1,1) to (2,1)
                    to [R] (2,0)
                    to (1,0);
        \draw (2,1) to (3,1)
                    to [R] (3,0)
                    to (2,0);
        \draw (3,1) to (4,1)
                    to [R] (4,0)
                    to (3,0);
        \draw (4,1) to (5,1)
                    to [R] (5,0)
                    to (4,0);
    \end{circuitikz}
}

\newcommand{\circuitMultiThree}{
    \ctikzset{bipoles/length=0.75cm}
    \begin{circuitikz}[scale=1.00]
        \draw (0,0) to [battery,l=\SI{9.0}{\volt}] (0,1)
                    to [R] (1.5,1)
                    to [R] (3,1)
                    to [R] (3,0)
                    to [R] (1.5,0)
                    to [R] (0,0);
    \end{circuitikz}
}

\element{cpo-mc}{
\begin{questionmult}{ch14-Q13}
    Which of the circuit diagrams shown is a series circuit?
    \begin{choices}
        \AMCboxDimensions{down=-1.0em}
        \wrongchoice{\circuitMultiOne}
        \wrongchoice{\circuitMultiTwo}
      \correctchoice{\circuitMultiThree}
    \end{choices}
\end{questionmult}
}

\element{cpo-mc}{
\begin{questionmult}{ch14-Q14}
    Which of the circuit diagrams shown is a parallel circuit?
    \begin{choices}
        \AMCboxDimensions{down=-1.0em}
      \correctchoice{\circuitMultiOne}
      \correctchoice{\circuitMultiTwo}
        \wrongchoice{\circuitMultiThree}
    \end{choices}
\end{questionmult}
}

\element{cpo-mc}{
\begin{question}{cpo-ch14-q15}
    When a new branch containing a resistor is added to a parallel circuit,
        the total circuit resistance:
    \begin{choices}
      \correctchoice{decreases.}
        \wrongchoice{increases.}
        \wrongchoice{stays the same.}
        \wrongchoice{may increase or decrease, depending on the device.}
    \end{choices}
\end{question}
}

\element{cpo-mc}{
\begin{question}{cpo-ch14-q16}
    What is the basic difference between series and parallel circuits?
    \begin{choices}
        \wrongchoice{Simple series circuits are not used in electrical devices; parallel circuits are used in all electrical devices.}
        \wrongchoice{In a series circuit, there are multiple paths for the flow of charge; in a parallel circuit, there are only one paths.}
      \correctchoice{A series circuit contains one path for the flow of charge; a parallel circuit contains more than one path.}
        \wrongchoice{A series circuit obeys Ohm's law; a parallel circuit does not obey Ohm's law.}
    \end{choices}
\end{question}
}

\element{cpo-mc}{
\begin{question}{cpo-ch14-q17}
    The wiring in your home uses:
    \begin{multicols}{2}
    \begin{choices}
      \correctchoice{parallel circuits}
        \wrongchoice{series circuits}
        \wrongchoice{two-way circuits}
        \wrongchoice{three-way circuits}
    \end{choices}
    \end{multicols}
\end{question}
}

\element{cpo-mc}{
\begin{question}{cpo-ch14-q18}
    Which of the following statements best describes the difference between series and parallel circuits?
    \begin{choices}
        \wrongchoice{Series circuits are battery circuits, and parallel circuits are generator circuits.}
      \correctchoice{Series circuits have a single path, and parallel circuits have two or more paths.}
        \wrongchoice{Series circuits are used in computers, and parallel circuits are used in homes.}
        \wrongchoice{Series circuits have one switch in them, and parallel circuits have two switches in them.}
    \end{choices}
\end{question}
}

\element{cpo-mc}{
\begin{question}{cpo-ch14-q19}
    The voltage drop across each resistor in the circuit below is:
    \begin{center}
    \ctikzset{bipoles/length=0.75cm}
    \begin{circuitikz}[scale=1.33]
        \draw (0,0) to [battery,l=$\SI{6.0}{\volt}$] (0,1) to (1,1) to [R,l=$\SI{10}{\ohm}$] (1,0) to (0,0);
        \draw (1,1) to (2,1) to [R,l=$\SI{10}{\ohm}$] (2,0) to (1,0);
        \draw (2,1) to (3,1) to [R,l=$\SI{10}{\ohm}$] (3,0) to (2,0);
    \end{circuitikz}
    \end{center}
    \begin{multicols}{2}
    \begin{choices}
        \wrongchoice{\SI{0.2}{\volt}}
        \wrongchoice{\SI{5}{\volt}}
      \correctchoice{\SI{6}{\volt}}
        \wrongchoice{\SI{12}{\volt}}
    \end{choices}
    \end{multicols}
\end{question}
}

\element{cpo-mc}{
\begin{question}{cpo-ch14-q20}
    The current in each resistor in the circuit is:
    \begin{center}
    \ctikzset{bipoles/length=0.75cm}
    \begin{circuitikz}[scale=1.33]
        \draw (0,0) to [battery,l=$\SI{9.0}{\volt}$] (0,1) to (1,1) to [R,l=$\SI{3}{\ohm}$] (1,0) to (0,0);
        \draw (1,1) to (2,1) to [R,l=$\SI{3}{\ohm}$] (2,0) to (1,0);
        \draw (2,1) to (3,1) to [R,l=$\SI{3}{\ohm}$] (3,0) to (2,0);
    \end{circuitikz}
    \end{center}
    \begin{multicols}{2}
    \begin{choices}
        \wrongchoice{\SI{1/3}{\ampere}}
        \wrongchoice{\SI{1}{\ampere}}
      \correctchoice{\SI{3}{\ampere}}
        \wrongchoice{\SI{9}{\ampere}}
    \end{choices}
    \end{multicols}
\end{question}
}

\element{cpo-mc}{
\begin{questionmult}{ch14-Q21}
    A lamp has a \emph{short circuit}.
    %%Which of the following statements best describes the lamp's circuit?
    Which of the following statements correctly describes the lamp's circuit?
    \begin{choices}
      \correctchoice{The circuit is drawing a large amount of current.}
      \correctchoice{The circuit has a path of very low resistance.}
      \correctchoice{The lamp will not turn on.}
      %\correctchoice{All of the above.}
    \end{choices}
\end{questionmult}
}

\element{cpo-mc}{
\begin{question}{cpo-ch14-q22}
    The voltage across the \SI{1}{\ohm} resistor pictured below is:
    \begin{center}
    \ctikzset{bipoles/length=0.75cm}
    \begin{circuitikz}[scale=1.33]
        \draw (0,0) to [battery,l=\SI{3}{\volt}] (0,1) to (1,1) to [R,l=\SI{1}{\ohm}] (1,0) to (0,0);
        \draw (1,1) to (2,1) to [R,l=\SI{2}{\ohm}] (2,0) to (1,0);
    \end{circuitikz}
    \end{center}
    \begin{multicols}{2}
    \begin{choices}
        \wrongchoice{\SI{1/3}{\volt}}
        \wrongchoice{\SI{2/3}{\volt}}
        \wrongchoice{\SI{1}{\volt}}
      \correctchoice{\SI{3}{\volt}}
    \end{choices}
    \end{multicols}
\end{question}
}

\element{cpo-mc}{
\begin{question}{cpo-ch14-q23}
    Kirchoff's current law applies to a parallel circuit because it states that:
    \begin{choices}
        \wrongchoice{current is the same everywhere throughout one entire circuit.}
        \wrongchoice{current is the same everywhere throughout branching circuit paths.}
        \wrongchoice{current stays the same as more branches are added to a parallel circuit.}
      \correctchoice{All the current flowing into a branch point must flow out again.}
    \end{choices}
\end{question}
}

\element{cpo-mc}{
\begin{question}{cpo-ch14-q24}
    A student connects three \SI{1}{\ohm} light bulbs to a \SI{9}{\volt} battery in parallel.
    The total circuit current is:
    \begin{multicols}{2}
    \begin{choices}
        \wrongchoice{\SI{1}{\ampere}}
        \wrongchoice{\SI{3}{\ampere}}
        \wrongchoice{\SI{9}{\ampere}}
      \correctchoice{\SI{27}{\ampere}}
    \end{choices}
    \end{multicols}
\end{question}
}

\element{cpo-mc}{
\begin{question}{cpo-ch14-q25}
    Which of the following would create a total resistance of \SI{0.5}{\ohm}?
    \begin{choices}
      \correctchoice{Four \SI{2}{\ohm} resistors connected in parallel.}
        \wrongchoice{Four \SI{2}{\ohm} resistors connected in series.}
        \wrongchoice{Eight \SI{2}{\ohm} resistors connected in parallel.}
        \wrongchoice{Two \SI{2}{\ohm} resistors connected in series.}
    \end{choices}
\end{question}
}

\element{cpo-mc}{
\begin{question}{cpo-ch14-q26}
    A circuit containing two identical branches has:
    \begin{choices}
      \correctchoice{one-half the resistance it would have if it only contained one of the branches.}
        \wrongchoice{twice the resistance it would have if it only contained one of the branches.}
        \wrongchoice{the same resistance it would have if it only contained one of the branches.}
        \wrongchoice{no resistance at all.}
    \end{choices}
\end{question}
}

%% NOTE: circuit diagram used for Q27, Q28
\newcommand{\circuitParallelTwo}{
    \ctikzset{bipoles/length=0.75cm}
    \begin{circuitikz}[scale=1.25]
        \draw (0,0) to [battery,v] (0,2)
                    to (1,2)
                    to [R,l=$A$] (1,0)
                    to (0,0);
        \draw (1,2) to (2,2)
                    to [R,l=$B$] (2,1)
                    to [R,l=$C$] (2,0)
                    to (1,0);
    \end{circuitikz}
}

\element{cpo-mc}{
\begin{question}{cpo-ch14-q27}
    In the circuit shown below, three identical flashlight bulbs are screwed
        into sockets and are lighted when the circuit is closed:
    \begin{center}
        \circuitParallelTwo
    \end{center}
    Which bulb draws the least amount of current?
    \begin{choices}
        \wrongchoice{Bulb $A$}
        \wrongchoice{Bulb $B$}
        \wrongchoice{Bulb $C$}
      \correctchoice{Both $B$ and $C$ draw an equal amount of current but draw less than bulb $A$.}
    \end{choices}
\end{question}
}

\element{cpo-mc}{
\begin{question}{cpo-ch14-q28}
    In the circuit shown below, three identical flashlight bulbs are screwed
        into sockets and are lighted when the circuit is closed:
    \begin{center}
        \circuitParallelTwo
    \end{center}
    If you unscrewed bulb $A$:
    \begin{choices}
        \wrongchoice{only bulb $B$ would go out.}
        \wrongchoice{only bulb $C$ would go out.}
        \wrongchoice{bulbs $B$ and $C$ would go out.}
      \correctchoice{bulbs $B$ and $C$ would remain lit.}
    \end{choices}
\end{question}
}

\element{cpo-mc}{
\begin{question}{cpo-ch14-q29}
    In the circuit shown below, three identical flashlight bulbs are screwed
        into sockets and are lighted when the circuit is closed:
    \begin{center}
        \circuitParallelTwo
    \end{center}
    If you unscrewed bulb $C$:
    \begin{choices}
        \wrongchoice{only bulb $A$ would go out.}
        \wrongchoice{only bulb $B$ would go out.}
      \correctchoice{bulb $B$ and $C$ would go out.}
        \wrongchoice{bulb $B$ and $A$ would remain lit.}
    \end{choices}
\end{question}
}

\element{cpo-mc}{
\begin{question}{cpo-ch14-q30}
    A \SI{75}{\watt} bulb uses:
    \begin{choices}
        \wrongchoice{\SI{75}{\joule} of energy until it burns out.}
      \correctchoice{\SI{75}{\joule} of energy every second.}
        \wrongchoice{\SI{75}{\joule} of energy every hour.}
        \wrongchoice{\SI{75}{\watt} of power every hour.}
    \end{choices}
\end{question}
}

\element{cpo-mc}{
\begin{question}{cpo-ch14-q31}
    The power used by a circuit can be found by:
    \begin{choices}
        \wrongchoice{using Ohm's law.}
        \wrongchoice{dividing voltage by current.}
        \wrongchoice{adding all resistance.}
      \correctchoice{multiplying voltage and current.}
    \end{choices}
\end{question}
}

\element{cpo-mc}{
\begin{question}{cpo-ch14-q32}
    What do we buy from electric utility company?
    \begin{choices}
        \wrongchoice{Power, in watts (\si{\watt})}
      \correctchoice{Energy, in kilowatt-hours (\si{\kilo\watt\hour})}
        \wrongchoice{Current, in amperes (\si{\ampere})}
        \wrongchoice{Voltage, in volts (\si{\volt})}
    \end{choices}
\end{question}
}

\element{cpo-mc}{
\begin{question}{cpo-ch14-q33}
    Which of the following is \emph{true} about alternating current?
    \begin{choices}
        \wrongchoice{Alternating current flows in one direction only.}
        \wrongchoice{Batteries run on alternating current.}
      \correctchoice{The electricity in your house uses alternating current.}
        \wrongchoice{Alternating current is easy to recharge.}
    \end{choices}
\end{question}
}

\element{cpo-mc}{
\begin{question}{cpo-ch14-q34}
    The watt is a unit that represents the:
    \begin{choices}
        \wrongchoice{amount of energy consumed by electrical appliances.}
        \wrongchoice{flow of charge in a circuit.}
      \correctchoice{rate at which energy is changed from one form to another.}
        \wrongchoice{potential energy difference between two places in a circuit.}
    \end{choices}
\end{question}
}

\element{cpo-mc}{
\begin{question}{cpo-ch14-q35}
    How much would it cost to operate a \SI{100}{\watt} light bulb for \SI{10}{\hour} if the cost of electric energy is \num{10} cents per kilowatt-hour?
    %\SI{0.10}{USD\per\kilo\watt\per\hour}
    \begin{multicols}{2}
    \begin{choices}
        \wrongchoice{1 dollar}
        \wrongchoice{10 dollar}
        \wrongchoice{10 cents}
      \correctchoice{1 cent}
    \end{choices}
    \end{multicols}
\end{question}
}

\element{cpo-mc}{
\begin{question}{cpo-ch14-q36}
    How much does it cost to operate a \SI{1 500}{\watt} space heater for \SI{8}{\hour} if the cost of electric energy is \num{12} cents per kilowatt-hour?
    %\SI{0.12}{USD\per\kilo\watt\per\hour}
    \begin{multicols}{2}
    \begin{choices}
      \correctchoice{\$1.44}
        \wrongchoice{\$9.60}
        \wrongchoice{\$1.50}
        \wrongchoice{\$0.18}
    \end{choices}
    \end{multicols}
\end{question}
}

\element{cpo-mc}{
\begin{question}{cpo-ch14-q37}
    Which of the following devices draws the most current when operating on a \SI{120}{\volt} circuit?
    \begin{choices}
        \wrongchoice{\SI{10}{\watt} clock radio}
        \wrongchoice{\SI{40}{\watt} light bulb}
      \correctchoice{\SI{50}{\watt} fan}
        \wrongchoice{All devices would draw the same current.}
    \end{choices}
\end{question}
}

\endinput


