
%%--------------------------------------------------
%% CPO: AMC Open Free Response Questions
%%--------------------------------------------------


%% Chapter 5: Forces in Equililbrium
%%--------------------------------------------------


%% CPO Short Answer Questions
%%--------------------------------------------------
\element{cpo-short}{
\begin{question}{ch05-short-q01}
    List three ways to reduce friction.
    \AMCOpen{lines=3}{
        \wrongchoice[W]{w}\scoring{0}
        \wrongchoice[P]{p}\scoring{1}
        \correctchoice[C]{c}\scoring{2}
    }
\end{question}
}

\element{cpo-short}{
\begin{question}{ch05-short-q02}
    Identify each of the following as a scalar or vector quantity.
    \begin{enumerate*}
        \item speed
        \item mass
        \item velocity
        \item length
        \item temperature
        \item time
        \item weight
    \end{enumerate*}
    \AMCOpen{lines=3}{
        \wrongchoice[W]{w}\scoring{0}
        \wrongchoice[P]{p}\scoring{1}
        \correctchoice[C]{c}\scoring{2}
    }
\end{question}
}


%% CPO Problem Questions
%%--------------------------------------------------
\element{cpo-problem}{
\begin{question}{ch05-problem-q01}
    Using a scale of $\SI{1}{\centi\meter}=\SI{1}{\newton}$,
        represent the following displacement vectors:
    \begin{enumerate*}
        \item \SI{5}{\newton} west
        \item \SI{3}{\newton}, \ang{270}
        \item \SI{4}{\newton}, \ang{0}
    \end{enumerate*}
    \AMCOpen{lines=3}{
        \wrongchoice[W]{w}\scoring{0}
        \wrongchoice[P]{p}\scoring{1}
        \correctchoice[C]{c}\scoring{2}
    }
\end{question}
}

\element{cpo-problem}{
\begin{question}{ch05-problem-q02}
    Alexis pulls with \SI{3}{\newton} on a toy doll toward east.
    Brianna pulls on the same toy toward south with \SI{2}{\newton}.
    Calculate the magnitude of the resultant force on the toy.
    \AMCOpen{lines=3}{
        \wrongchoice[W]{w}\scoring{0}
        \wrongchoice[P]{p}\scoring{1}
        \correctchoice[C]{c}\scoring{2}
    }
    %% ANS: \SI{3.6}{\newton}
\end{question}
}

\element{cpo-problem}{
\begin{question}{ch05-problem-q03}
    A scale of $\SI{1}{\centi\meter}=\SI{1}{\newton}$ was used to
        draw the vector represented.
    Using a centimeter ruler, draw and identify the magnitude of the
        horizontal and vertical components of the \SI{4.5}{\newton}
        force illustrated in the vector drawing below:
    \begin{center}
        %% NOTE: add diagrams
    \end{center}
    \AMCOpen{lines=3}{
        \wrongchoice[W]{w}\scoring{0}
        \wrongchoice[P]{p}\scoring{1}
        \correctchoice[C]{c}\scoring{2}
    }
\end{question}
}

\element{cpo-problem}{
\begin{question}{ch05-problem-q04}
    A can of soda sits motionless on a table.
    Make a free-body diagram of the forces acting upon the can.
    \begin{center}
        %% NOTE: add diagrams
    \end{center}
    \AMCOpen{lines=3}{
        \wrongchoice[W]{w}\scoring{0}
        \wrongchoice[P]{p}\scoring{1}
        \correctchoice[C]{c}\scoring{2}
    }
    %% ANS: mechanical advantage, efficience, power
\end{question}
}


\element{cpo-problem}{
\begin{question}{ch05-problem-q05}
    When a \SI{20}{\newton} weight is hung on a spring, it
        stretches \SI{16}{\centi\meter}.
    How far does the spring stretch if a \SI{5.0}{\newton} weight
        is hung from the spring?
    \AMCOpen{lines=3}{
        \wrongchoice[W]{w}\scoring{0}
        \wrongchoice[P]{p}\scoring{1}
        \correctchoice[C]{c}\scoring{2}
    }
    %% ANS: \SI{4}{\centi\meter}
\end{question}
}

\element{cpo-problem}{
\begin{question}{ch05-problem-q06}
    The diagram below represents a lever with three forces applied to
        it at different positions.
    A \SI{4}{\newton}, clockwise force is applied \SI{1}{\meter} from
        the right end of the lever.
    A \SI{10}{\newton}, clockwise force is applied \SI{5}{\meter} to the
        left of the \SI{4}{\newton} force.
    \begin{center}
        %% NOTE: add diagrams
    \end{center}
    For the lever to be in equilibrium, what is the size of the counterclockwise
        force applied \SI{4}{\meter} from the right end of the lever?
    \AMCOpen{lines=3}{
        \wrongchoice[W]{w}\scoring{0}
        \wrongchoice[P]{p}\scoring{1}
        \correctchoice[C]{c}\scoring{2}
    }
    %% ANS: \SI{16}{\newton}
\end{question}
}


%% CPO Essay Questions
%%--------------------------------------------------
\element{cpo-essay}{
\begin{question}{ch05-essay-q01}
    Friction can be useful, but is is often desirable to reduce the amount
        between surfaces.
    List \num{3} example of useful friction and \num{3} example of friction
        that should be reduced.
    \AMCOpen{lines=3}{
        \wrongchoice[W]{w}\scoring{0}
        \wrongchoice[P]{p}\scoring{1}
        \correctchoice[C]{c}\scoring{2}
    }
\end{question}
}

\element{cpo-essay}{
\begin{question}{ch05-essay-q02}
    Explain why is is easier to open a door when pushing on
        the edge of the door close to the doorknob than pushing
        on the edge closet to the hinges.
    \AMCOpen{lines=3}{
        \wrongchoice[W]{w}\scoring{0}
        \wrongchoice[P]{p}\scoring{1}
        \correctchoice[C]{c}\scoring{2}
    }
\end{question}
}

\endinput


