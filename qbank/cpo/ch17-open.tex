
%%--------------------------------------------------
%% CPO: AMC Open Free Response Questions
%%--------------------------------------------------


%% Chapter 17: Electromagnets and Induction
%%--------------------------------------------------


%% CPO Short Answer Questions
%%--------------------------------------------------
\element{cpo-short}{
\begin{question}{ch17-short-q01}
    If a wire carrying electric current is looped into a coil,
        why is the magnetic field stronger inside the coil than outside
        the coil?
    \AMCOpen{lines=3}{
        \wrongchoice[W]{w}\scoring{0}
        \wrongchoice[P]{p}\scoring{1}
        \correctchoice[C]{c}\scoring{2}
    }
\end{question}
}

\element{cpo-short}{
\begin{question}{ch17-short-q02}
    What happens when you place a compass near a current-carrying wire?
    \AMCOpen{lines=3}{
        \wrongchoice[W]{w}\scoring{0}
        \wrongchoice[P]{p}\scoring{1}
        \correctchoice[C]{c}\scoring{2}
    }
\end{question}
}

\element{cpo-short}{
\begin{question}{ch17-short-q03}
    Describe the shape and strength of the magnetic field of a straight,
        current-carrying wire.
    \AMCOpen{lines=3}{
        \wrongchoice[W]{w}\scoring{0}
        \wrongchoice[P]{p}\scoring{1}
        \correctchoice[C]{c}\scoring{2}
    }
\end{question}
}

\element{cpo-short}{
\begin{question}{ch17-short-q04}
    What are two ways the magnetic field from current-carrying wires
        can be strengthened?
    \AMCOpen{lines=3}{
        \wrongchoice[W]{w}\scoring{0}
        \wrongchoice[P]{p}\scoring{1}
        \correctchoice[C]{c}\scoring{2}
    }
\end{question}
}

\element{cpo-short}{
\begin{question}{ch17-short-q05}
    A bar magnet is suspended so that it is free to rotate.
    When you hold a second bar magnet near the suspended magnet,
        the suspended magnet begins to rotate.
    Explain what must be done to keep the magnet rotating.
    \AMCOpen{lines=3}{
        \wrongchoice[W]{w}\scoring{0}
        \wrongchoice[P]{p}\scoring{1}
        \correctchoice[C]{c}\scoring{2}
    }
\end{question}
}

\element{cpo-short}{
\begin{question}{ch17-short-q06}
    What is the purpose of a commutator in an electric motor?
    \AMCOpen{lines=3}{
        \wrongchoice[W]{w}\scoring{0}
        \wrongchoice[P]{p}\scoring{1}
        \correctchoice[C]{c}\scoring{2}
    }
    %% ANS: an electric field
\end{question}
}

\element{cpo-short}{
\begin{question}{ch17-short-q07}
    How can you induce current in a wire using a permanent magnet?
    \AMCOpen{lines=3}{
        \wrongchoice[W]{w}\scoring{0}
        \wrongchoice[P]{p}\scoring{1}
        \correctchoice[C]{c}\scoring{2}
    }
\end{question}
}

\element{cpo-short}{
\begin{question}{ch17-short-q08}
    what is the main difference between a motor and a generator?
    \AMCOpen{lines=3}{
        \wrongchoice[W]{w}\scoring{0}
        \wrongchoice[P]{p}\scoring{1}
        \correctchoice[C]{c}\scoring{2}
    }
\end{question}
}


%% CPO Problem Questions
%%--------------------------------------------------
%\element{cpo-problem}{
%\begin{question}{ch17-problem-q01}
%    Describe the magnetic force between the two current-carrying
%        wires shown in the diagram.
%    \begin{center}
%        %% NOTE: add diagram
%    \end{center}
%    \AMCOpen{lines=3}{
%        \wrongchoice[W]{w}\scoring{0}
%        \wrongchoice[P]{p}\scoring{1}
%        \correctchoice[C]{c}\scoring{2}
%    }
%\end{question}
%}

%\element{cpo-problem}{
%\begin{question}{ch17-problem-q02}
%    In which direction does the rotating disk spin?
%    \begin{center}
%        %% NOTE: add diagram
%    \end{center}
%    \AMCOpen{lines=3}{
%        \wrongchoice[W]{w}\scoring{0}
%        \wrongchoice[P]{p}\scoring{1}
%        \correctchoice[C]{c}\scoring{2}
%    }
%\end{question}
%}

\element{cpo-problem}{
\begin{question}{ch17-problem-q03}
    On a European trip, you discover that the electric
        outlets have \SI{240}{\volt}.
    You realize that you need a transformer, so you quickly wrap
        \num{20} turns around the primary coil.
    If you need \SI{120}{\volt} to run your hairdryer,
        how many turns do you need to wrap around the secondary?
    \AMCOpen{lines=3}{
        \wrongchoice[W]{w}\scoring{0}
        \wrongchoice[P]{p}\scoring{1}
        \correctchoice[C]{c}\scoring{2}
    }
    %% ANS: \SI{40}{turns}
\end{question}
}

\element{cpo-problem}{
\begin{question}{ch17-problem-q04}
    A transformer has a primary voltage of \SI{2}{\volt} and a
        secondary of \SI{120}{\volt}.
    If the secondary coil has \num{100} turns, how many
        turns must the primary have?
    \AMCOpen{lines=3}{
        \wrongchoice[W]{w}\scoring{0}
        \wrongchoice[P]{p}\scoring{1}
        \correctchoice[C]{c}\scoring{2}
    }
    %% ANS: \num{10} turns
\end{question}
}

\element{cpo-problem}{
\begin{question}{ch17-problem-q05}
    A transformer has a secondary voltage of \SI{9}{\volt} when
        connected to a \SI{120}{\volt} outlet.
    What is the ratio to the number of turns on the primary
        coil to the number of turns on the secondary coil?
    \AMCOpen{lines=3}{
        \wrongchoice[W]{w}\scoring{0}
        \wrongchoice[P]{p}\scoring{1}
        \correctchoice[C]{c}\scoring{2}
    }
    %% ANS: 13.3
\end{question}
}

%% CPO Essay Questions
%%--------------------------------------------------
\element{cpo-essay}{
\begin{question}{ch17-essay-q01}
    Name three advantages en electromagnet has over a permanent magnet.
    \AMCOpen{lines=3}{
        \wrongchoice[W]{w}\scoring{0}
        \wrongchoice[P]{p}\scoring{1}
        \correctchoice[C]{c}\scoring{2}
    }
\end{question}
}

\element{cpo-essay}{
\begin{question}{ch17-essay-q02}
    List the 3 key components from which all electric motors are made.
    \AMCOpen{lines=3}{
        \wrongchoice[W]{w}\scoring{0}
        \wrongchoice[P]{p}\scoring{1}
        \correctchoice[C]{c}\scoring{2}
    }
    %% ANS: induction, friction, and contact
\end{question}
}

\element{cpo-essay}{
\begin{question}{ch17-essay-q03}
    Explain how an electric motor works.
    Be sure to use the words \emph{rotor}, \emph{commutator},
        \emph{permanent magnets} and \emph{electromagnets}.
    \AMCOpen{lines=3}{
        \wrongchoice[W]{w}\scoring{0}
        \wrongchoice[P]{p}\scoring{1}
        \correctchoice[C]{c}\scoring{2}
    }
    %% ANS: electrons leave the plates until no charge is left
\end{question}
}

\element{cpo-essay}{
\begin{question}{ch17-essay-q04}
    List the 3 ways to increase the induced voltage in a generator.
    \AMCOpen{lines=3}{
        \wrongchoice[W]{w}\scoring{0}
        \wrongchoice[P]{p}\scoring{1}
        \correctchoice[C]{c}\scoring{2}
    }
    %% ANS: number of turns, speed or field
\end{question}
}

\element{cpo-essay}{
\begin{question}{ch17-essay-q04}
    What did Michael Faraday discover about the relationship between
        electricity and magnetism?
    \AMCOpen{lines=3}{
        \wrongchoice[W]{w}\scoring{0}
        \wrongchoice[P]{p}\scoring{1}
        \correctchoice[C]{c}\scoring{2}
    }
\end{question}
}

\endinput


