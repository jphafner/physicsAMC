
%%--------------------------------------------------
%% CPO: Multiple Choice Questions
%%--------------------------------------------------


%% Chapter 10: Energy Flow and Systems
%%--------------------------------------------------


%% Learning Objectives
%%--------------------------------------------------

%% Name and describe the different forms of energy. 
%% Describe how energy is “lost” to a system. 
%% Make an energy flow diagram. 
%% Calculate power given force, time, height, mass, or other variables. 
%% Determine the efficiency of energy conversion in a system. 
%% Explain why heat engines can never be 100 % efficient. 
%% Draw energy flow diagrams for systems. 
%% Recognize the role of energy and power in technology, nature, and living things.


%% CPO Multiple Choice Questions
%%--------------------------------------------------
\element{cpo-mc}{
\begin{question}{cpo-ch10-q01}
    The efficiency of most machines is less than \SI{100}{\percent}.
    Due to friction, energy seems to be lost.
    While the energy is not \emph{truly} lost,
        it does work that is not useful or is converted to:
    \begin{multicols}{2}
    \begin{choices}
      \correctchoice{thermal energy}
        \wrongchoice{radiant energy}
        \wrongchoice{potential energy}
        \wrongchoice{nuclear energy}
    \end{choices}
    \end{multicols}
\end{question}
}

\element{cpo-mc}{
\begin{question}{cpo-ch10-q02}
    Batteries are devices that change chemical energy to:
    \begin{multicols}{2}
    \begin{choices}
        \wrongchoice{mechanical energy}
        \wrongchoice{radiant energy}
      \correctchoice{electrical energy}
        \wrongchoice{nuclear energy}
    \end{choices}
    \end{multicols}
\end{question}
}

\element{cpo-mc}{
\begin{questionmult}{ch10-Q03}
    Which of the following statements are \emph{true} about nuclear energy?
    \begin{choices}
        \wrongchoice{Nuclear energy can result from splitting large atoms.}
        \wrongchoice{Nuclear energy can result from combining large atoms.}
        \wrongchoice{Nuclear energy is how the sun and stars make energy.}
    \end{choices}
\end{questionmult}
}

\element{cpo-mc}{
\begin{question}{cpo-ch10-q04}
    Light is a form of:
    \begin{multicols}{2}
    \begin{choices}
        \wrongchoice{chemical energy}
        \wrongchoice{nuclear energy}
      \correctchoice{radiant energy}
        \wrongchoice{thermal energy}
    \end{choices}
    \end{multicols}
\end{question}
}

%\element{cpo-mc}{
%\begin{question}{cpo-ch10-q05}
%    Talia is pedaling down a steep hill on her bicycle and wants to be able
%        to coast up the next hill, which is \SI{25.0}{\meter} high,
%        without pedaling up the hill at all.
%    \begin{center}
%    \begin{tikzpicture}
%        %% NOTE: TODO: draw tikz
%    \end{tikzpicture}
%    \end{center}
%    Assuming her bicycle is \SI{100}{\percent} efficient,
%        her speed would have to be at least:
%    \begin{multicols}{2}
%    \begin{choices}
%        \wrongchoice{\SI{7.1}{\meter\per\second}.}
%      \correctchoice{\SI{22.1}{\meter\per\second}.}
%        \wrongchoice{\SI{71}{\meter\per\second}.}
%        \wrongchoice{\SI{240}{\meter\per\second}.}
%    \end{choices}
%    \end{multicols}
%\end{question}
%}

\element{cpo-mc}{
\begin{question}{cpo-ch10-q06}
    The rate at which work is done is called:
    \begin{multicols}{2}
    \begin{choices}
        \wrongchoice{force}
        \wrongchoice{efficiency}
      \correctchoice{power}
        \wrongchoice{energy}
    \end{choices}
    \end{multicols}
\end{question}
}

\element{cpo-mc}{
\begin{question}{cpo-ch10-q07}
    One watt is equivalent to one:
    \begin{choices}
        \wrongchoice{newton per second (\si{\newton\per\second})}
        \wrongchoice{kilogram meter per second (\si{\kilo\gram\meter\per\second})}
        \wrongchoice{newton meter (\si{\newton\meter})}
      \correctchoice{joule per second (\si{\joule\per\second})}
    \end{choices}
\end{question}
}

\element{cpo-mc}{
\begin{question}{cpo-ch10-q08}
    The efficiency of a machine is reduce by 
    \begin{choices}
        \wrongchoice{effort}
        \wrongchoice{work output}
        \wrongchoice{mechanical advantage}
      \correctchoice{friction}
    \end{choices}
\end{question}
}

\element{cpo-mc}{
\begin{question}{cpo-ch10-q09}
    Efficiency can be defined as the ratio of:
    \begin{choices}
        \wrongchoice{work input to the work output.}
      \correctchoice{work output to work input.}
        \wrongchoice{output force to input force.}
        \wrongchoice{input force to output force.}
    \end{choices}
\end{question}
}

\element{cpo-mc}{
\begin{question}{cpo-ch10-q10}
    Anastas eats a chocolate chip cookie and then climbs a set of stairs.
    In doing so, the chemical energy from the cookie is transformed into \rule[-0.1pt]{4em}{0.1pt} energy.
    \begin{choices}
        \wrongchoice{potential}
        \wrongchoice{kinetic}
      \correctchoice{both kinetic and potential}
        \wrongchoice{neither kinetic nor potential}
    \end{choices}
\end{question}
}

\element{cpo-mc}{
\begin{question}{cpo-ch10-q11}
    Hiroko uses \SI{1 200}{\joule} of energy to lift a \SI{200}{\newton} crate \SI{5}{\meter} using a pulley system.
    The efficiency of the system is:
    \begin{multicols}{4}
    \begin{choices}
        \wrongchoice{\SI{12}{\percent}}
        \wrongchoice{\SI{17}{\percent}}
        \wrongchoice{\SI{50}{\percent}}
      \correctchoice{\SI{83}{\percent}}
    \end{choices}
    \end{multicols}
\end{question}
}

\element{cpo-mc}{
\begin{question}{cpo-ch10-q12}
    Lola applies force to the handle of a jack so that she can change the flat tire on her car.
    If she uses \SI{1 250}{\joule} of energy on a jack that is \SI{40}{\percent}  efficient,
        how much energy is available to raise the car?
    \begin{multicols}{2}
    \begin{choices}
        \wrongchoice{\SI{1 200}{\joule}}
        \wrongchoice{\SI{750}{\joule}}
      \correctchoice{\SI{500}{\joule}}
        \wrongchoice{\SI{400}{\joule}}
    \end{choices}
    \end{multicols}
\end{question}
}

\element{cpo-mc}{
\begin{question}{cpo-ch10-q13}
    The brakes on an automobile become hot when they are used continuously because the:
    \begin{choices}
      \correctchoice{work input is being converted to heat.}
        \wrongchoice{work output is being converted to heat.}
        \wrongchoice{efficiency of the brakes is nearly \SI{100}{\percent}.}
        \wrongchoice{friction on the brakes causes an increase in the efficiency.}
    \end{choices}
\end{question}
}

\element{cpo-mc}{
\begin{question}{cpo-ch10-q14}
    An automobile is about \SI{15}{\percent} efficient.
    For every \SI{10}{\gallon} of fuel used by the car,
        the number of gallons actually used to move the vehicle is:
    \begin{multicols}{4}
    \begin{choices}
      \correctchoice{1.5}
        \wrongchoice{5.0}
        \wrongchoice{6.7}
        \wrongchoice{10}
    \end{choices}
    \end{multicols}
\end{question}
}

\element{cpo-mc}{
\begin{question}{cpo-ch10-q15}
    The work output is \SI{200}{\joule} for a machine that is \SI{80}{\percent} efficient.
    The work input is:
    \begin{multicols}{2}
    \begin{choices}
      \correctchoice{\SI{250}{\joule}}
        \wrongchoice{\SI{200}{\joule}}
        \wrongchoice{\SI{160}{\joule}}
        \wrongchoice{\SI{40}{\joule}}
    \end{choices}
    \end{multicols}
\end{question}
}

\element{cpo-mc}{
\begin{question}{cpo-ch10-q16}
    The efficiency of a modern bicycle is \SI{95}{\percent}.
    If you exert \SI{300}{\joule} in pedaling a bicycle on level ground, what is the work output?
    \begin{multicols}{2}
    \begin{choices}
        \wrongchoice{\SI{95}{\joule}}
        \wrongchoice{\SI{105}{\joule}}
        \wrongchoice{\SI{190}{\joule}}
      \correctchoice{\SI{285}{\joule}}
    \end{choices}
    \end{multicols}
\end{question}
}

\element{cpo-mc}{
\begin{question}{cpo-ch10-q17}
    If \SI{40}{\watt} of power are consumed in \SI{20}{\second},
        the maximum work done is:
    \begin{multicols}{2}
    \begin{choices}
        \wrongchoice{\SI{0.5}{\joule}}
        \wrongchoice{\SI{2}{\joule}}
        \wrongchoice{\SI{5}{\joule}}
      \correctchoice{\SI{800}{\joule}}
    \end{choices}
    \end{multicols}
\end{question}
}

\element{cpo-mc}{
\begin{question}{cpo-ch10-q18}
    The power used in doing \SI{100}{\joule} of work in \SI{10}{\second} is:
    \begin{multicols}{2}
    \begin{choices}
        \wrongchoice{\SI{1 000}{\watt}}
        \wrongchoice{\SI{100}{\watt}}
      \correctchoice{\SI{10}{\watt}}
        \wrongchoice{\SI{0.1}{\watt}}
    \end{choices}
    \end{multicols}
\end{question}
}

\element{cpo-mc}{
\begin{question}{cpo-ch10-q19}
    How much energy is needed to run a \SI{60}{\watt} light bulb for five seconds?
    \begin{multicols}{2}
    \begin{choices}
        \wrongchoice{\SI{12}{\joule}}
        \wrongchoice{\SI{30}{\joule}}
        \wrongchoice{\SI{60}{\joule}}
      \correctchoice{\SI{300}{\joule}}
    \end{choices}
    \end{multicols}
\end{question}
}

\element{cpo-mc}{
\begin{question}{cpo-ch10-q20}
    A bowler lifts her bowling ball a distance of \SI{0.5}{\meter} using \SI{3.5}{\joule} of energy.
    Her bowling ball has a mass of about:
    \begin{multicols}{2}
    \begin{choices}
        \wrongchoice{\SI{3}{\kilo\gram}}
      \correctchoice{\SI{7}{\kilo\gram}}
        \wrongchoice{\SI{3.5}{\kilo\gram}}
        \wrongchoice{Not enough information is given.}
    \end{choices}
    \end{multicols}
\end{question}
}

\element{cpo-mc}{
\begin{question}{cpo-ch10-q21}
    A gallon of gasoline contains chemical energy.
    If you pour a gallon of gasoline into your car,
        you could drive for about \SI{20}{\mile} on the highway (about \SI{20}{\minute}).
    You could choose instead to run a lawn mower for an hour with one gallon of gasoline.
    If the gasoline in either situation were to come in contact with a fire,
        the entire gallon would burn in less than a minute.
    Which reaction has the most power?
    \begin{choices}
        \wrongchoice{burning the gasoline in the car.}
      \correctchoice{bung the gasoline in a fire.}
        \wrongchoice{bung the gasoline in the lawn mower.}
        \wrongchoice{They are equal because the gasoline has the same chemical potential energy in each case.}
    \end{choices}
\end{question}
}

\element{cpo-mc}{
\begin{question}{cpo-ch10-q22}
    The ratio representing power includes all of the following  \emph{except}:
    \begin{choices}
        \wrongchoice{work $/$ time}
        \wrongchoice{force $\times$ velocity}
        \wrongchoice{(force $\times$ distance) $/$ time}
      \correctchoice{force $\times$ distance}
    \end{choices}
\end{question}
}

\element{cpo-mc}{
\begin{question}{cpo-ch10-q23}
    Appliances are described by the rate at which they use energy.
    These manufacturer ratings are called \rule[-0.1pt]{4em}{0.1pt} ratings.
    \begin{multicols}{2}
    \begin{choices}
        \wrongchoice{force.}
        \wrongchoice{work.}
      \correctchoice{power.}
        \wrongchoice{efficiency.}
    \end{choices}
    \end{multicols}
\end{question}
}

\element{cpo-mc}{
\begin{question}{cpo-ch10-q24}
    At each level in the energy pyramid, the amount of usable energy:
    \begin{choices}
        \wrongchoice{increases.}
      \correctchoice{decreases.}
        \wrongchoice{can increase or decrease.}
        \wrongchoice{remains the same.}
    \end{choices}
\end{question}
}

\element{cpo-mc}{
\begin{question}{cpo-ch10-q25}
    The organisms at the bottom of the food chain are:
    \begin{multicols}{2}
    \begin{choices}
        \wrongchoice{herbivores.}
        \wrongchoice{decomposers.}
        \wrongchoice{carnivores.}
      \correctchoice{producers.}
    \end{choices}
    \end{multicols}
\end{question}
}

\element{cpo-mc}{
\begin{question}{cpo-ch10-q26}
    Which of the following is an example of a decomposer?
    \begin{multicols}{2}
    \begin{choices}
        \wrongchoice{Snail}
        \wrongchoice{Hawk}
        \wrongchoice{Mouse}
      \correctchoice{Earthworm}
    \end{choices}
    \end{multicols}
\end{question}
}

\element{cpo-mc}{
\begin{question}{cpo-ch10-q27}
    Where does the energy used to drive the water cycle come from?
    \begin{multicols}{2}
    \begin{choices}
        \wrongchoice{Geothermal heat}
      \correctchoice{The sun}
        \wrongchoice{Air}
        \wrongchoice{Wind}
    \end{choices}
    \end{multicols}
\end{question}
}

\element{cpo-mc}{
\begin{question}{cpo-ch10-q28}
    When the total energy of a system remains the same over time,
        the system is in:
    \begin{multicols}{2}
    \begin{choices}
      \correctchoice{steady state.}
        \wrongchoice{cycle.}
        \wrongchoice{conversion.}
        \wrongchoice{transmission.}
    \end{choices}
    \end{multicols}
\end{question}
}

\element{cpo-mc}{
\begin{question}{cpo-ch10-q29}
    If a \SI{900}{\watt} toaster were rated in horsepower,
        its rating would be about \rule[-0.1pt]{4em}{0.1pt} horsepower.
    \begin{multicols}{2}
    \begin{choices}
        \wrongchoice{\num{0.830}}
      \correctchoice{\num{1.21}}
        \wrongchoice{\num{828}}
        \wrongchoice{\num{1 210}}
    \end{choices}
    \end{multicols}
\end{question}
}

\element{cpo-mc}{
\begin{question}{cpo-ch10-q30}
    An incandescent light bulb uses \SI{100}{\joule} of electrical energy every second.
    Due to heat loos, the energy available to light the room is reduced to \SI{6}{\joule\per\second}.
    The efficiency of the light bulb is:
    \begin{multicols}{2}
    \begin{choices}
      \correctchoice{\SI{6}{\percent}}
        \wrongchoice{\SI{16}{\percent}}
        \wrongchoice{\SI{60}{\percent}}
        \wrongchoice{\SI{106}{\percent}}
    \end{choices}
    \end{multicols}
\end{question}
}

\element{cpo-mc}{
\begin{question}{cpo-ch10-q31}
    Calculate the overall efficiency of a device that uses two different processes with efficiencies of \SI{80}{\percent} and \SI{50}{\percent}.
    \begin{multicols}{2}
    \begin{choices}
        \wrongchoice{\SI{1.6}{\percent}}
      \correctchoice{\SI{40}{\percent}}
        \wrongchoice{\SI{63}{\percent}}
        \wrongchoice{\SI{130}{\percent}}
    \end{choices}
    \end{multicols}
\end{question}
}

\endinput

