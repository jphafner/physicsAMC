
%%--------------------------------------------------
%% CPO: AMC Open Free Response Questions
%%--------------------------------------------------


%% Chapter 4: Machines, Work and Energy
%%--------------------------------------------------


%% CPO Short Answer Questions
%%--------------------------------------------------
\element{cpo-short}{
\begin{question}{ch04-short-q01}
    On Monday, Bik runs upstairs carrying a heavy book.
    The next day, she walks upstairs carrying the same book.
    Compare Bik's work and power on Monday and Tuesday.
    \AMCOpen{lines=3}{
        \wrongchoice[W]{w}\scoring{0}
        \wrongchoice[P]{p}\scoring{1}
        \correctchoice[C]{c}\scoring{2}
    }
\end{question}
}

\element{cpo-short}{
\begin{question}{ch04-short-q02}
    List four types of simple machines.
    \AMCOpen{lines=3}{
        \wrongchoice[W]{w}\scoring{0}
        \wrongchoice[P]{p}\scoring{1}
        \correctchoice[C]{c}\scoring{2}
    }
    %% ANS: wheel and axle, rope and pulley, ramp, gears, lever, screw
\end{question}
}

\element{cpo-short}{
\begin{question}{ch04-short-q03}
    Name the three basic parts of a lever.
    \AMCOpen{lines=3}{
        \wrongchoice[W]{w}\scoring{0}
        \wrongchoice[P]{p}\scoring{1}
        \correctchoice[C]{c}\scoring{2}
    }
    %% ANS: input arm, output arm and fulcrum
\end{question}
}

\element{cpo-short}{
\begin{question}{ch04-short-q04}
    State the relationship between the mechanical advantage of a pulley system
        and the number of strands supporting the pulley.
    \AMCOpen{lines=3}{
        \wrongchoice[W]{w}\scoring{0}
        \wrongchoice[P]{p}\scoring{1}
        \correctchoice[C]{c}\scoring{2}
    }
    %% ANS: mechanical advantage = number of supporting strands
\end{question}
}

\element{cpo-short}{
\begin{question}{ch04-short-q05}
    Compare the output work to the input work for a machine.
    \AMCOpen{lines=3}{
        \wrongchoice[W]{w}\scoring{0}
        \wrongchoice[P]{p}\scoring{1}
        \correctchoice[C]{c}\scoring{2}
    }
\end{question}
}

\element{cpo-short}{
\begin{question}{ch04-short-q05}
    Name the quantity represented by each of the following ratios below. \\
    \begin{itemize}
        \item[a.] \rule[-0.1pt]{5em}{0.1pt} $=\frac{output force}{input force}$ 
        \item[b.] \rule[-0.1pt]{5em}{0.1pt} $=\frac{work output}{work input}$ 
        \item[c.] \rule[-0.1pt]{5em}{0.1pt} $=\frac{work}{work time}$
    \end{itemize}
    \AMCOpen{lines=3}{
        \wrongchoice[W]{w}\scoring{0}
        \wrongchoice[P]{p}\scoring{1}
        \correctchoice[C]{c}\scoring{2}
    }
    %% ANS: mechanical advantage, efficience, power
\end{question}
}


%% CPO Problem Questions
%%--------------------------------------------------
\element{cpo-problem}{
\begin{question}{ch04-problem-q01}
    If one horsepower is equal to \SI{746}{\watt}, how much horsepower
        does a highly trained athlete generate by doing
        \SI{350}{\joule\per\second} for an hour?
    \AMCOpen{lines=3}{
        \wrongchoice[W]{w}\scoring{0}
        \wrongchoice[P]{p}\scoring{1}
        \correctchoice[C]{c}\scoring{2}
    }
    %% ANS: \SI{0.468}{hp}
\end{question}
}

\element{cpo-problem}{
\begin{question}{ch04-problem-q02}
    Calculate the power required to move a \SI{2000}{\kilo\gram}
        automobile to the top of a \SI{100}{\meter} hill in
        \SI{15.0}{\second}.
    Express the power both in units of watts and horsepower.
    ($\SI{1}{hp}=\SI{746}{\watt}$)
    \AMCOpen{lines=3}{
        \wrongchoice[W]{w}\scoring{0}
        \wrongchoice[P]{p}\scoring{1}
        \correctchoice[C]{c}\scoring{2}
    }
    %% ANS: \SI{130 667}{\watt}
    %% ANS: \SI{175}{hp}
\end{question}
}

\element{cpo-problem}{
\begin{question}{ch04-problem-q03}
    Carlos accelerates his \SI{3}{\kilo\gram} skateboard to a speed
        of \SI{3.96}{\meter\per\second} in \SI{3}{\second} in a
        distance of \SI{4}{\meter}
    How much work does he do?
    \AMCOpen{lines=3}{
        \wrongchoice[W]{w}\scoring{0}
        \wrongchoice[P]{p}\scoring{1}
        \correctchoice[C]{c}\scoring{2}
    }
    %% ANS: \SI{15.84}{\joule}
\end{question}
}

\element{cpo-problem}{
\begin{question}{ch04-problem-q04}
    Calculate the theoretical mechanical advantage of a ramp
        that is \SI{2}{\meter}  high and \SI{10}{\meter} long.
    \AMCOpen{lines=3}{
        \wrongchoice[W]{w}\scoring{0}
        \wrongchoice[P]{p}\scoring{1}
        \correctchoice[C]{c}\scoring{2}
    }
    %% ANS: \num{5}
\end{question}
}

\element{cpo-problem}{
\begin{question}{ch04-problem-q05}
    How much force would be required to lift a \SI{350}{\newton}
        weight by using a pulley system with a mechanical advantage
        of \num{5}?
    \AMCOpen{lines=3}{
        \wrongchoice[W]{w}\scoring{0}
        \wrongchoice[P]{p}\scoring{1}
        \correctchoice[C]{c}\scoring{2}
    }
    %% ANS: \SI{70}{\newton}
\end{question}
}

\element{cpo-problem}{
\begin{question}{ch04-problem-q06}
    A \SI{70}{\kilo\gram} bicycle racer climbs a \SI{500}{\meter}
        hill by doing \SI{400 000}{\joule} of work.
    Calculate the efficiency of his bicycle.
    \AMCOpen{lines=3}{
        \wrongchoice[W]{w}\scoring{0}
        \wrongchoice[P]{p}\scoring{1}
        \correctchoice[C]{c}\scoring{2}
    }
    %% ANS: \SI{70}{\newton}
\end{question}
}


%% CPO Essay Questions
%%--------------------------------------------------
\element{cpo-essay}{
\begin{question}{ch04-essay-q01}
    Explain the following statement: ``Applying a force to an object
        does not necessarily give the object energy.''
    \AMCOpen{lines=3}{
        \wrongchoice[W]{w}\scoring{0}
        \wrongchoice[P]{p}\scoring{1}
        \correctchoice[C]{c}\scoring{2}
    }
\end{question}
}

\element{cpo-essay}{
\begin{question}{ch04-essay-q02}
    Isiah and Ben have a race to carry identical objects from
        a point at the bottom of a hill to the same point at the top.
    Isaiah runs straight up a very steep part of the hill and finishes first.
    Ben chooses a more gradual ascent and, walking, arrives after Isiah.
    Who has done more work? Explain your answer.
    \AMCOpen{lines=3}{
        \wrongchoice[W]{w}\scoring{0}
        \wrongchoice[P]{p}\scoring{1}
        \correctchoice[C]{c}\scoring{2}
    }
    %% ANS: they do the same amount of work
\end{question}
}

\element{cpo-essay}{
\begin{question}{ch04-essay-q03}
    Describe each type (class) of lever, and give an example of each class.
    \AMCOpen{lines=3}{
        \wrongchoice[W]{w}\scoring{0}
        \wrongchoice[P]{p}\scoring{1}
        \correctchoice[C]{c}\scoring{2}
    }
\end{question}
}

\element{cpo-essay}{
\begin{question}{ch04-essay-q04}
    Explain why the efficiency of machines is always less
        than \SI{100}{\percent}.
    In the discussion use the words \emph{efficiency},
        \emph{work input}, \emph{work output},
        \emph{heat}, and \emph{friction}.
    \begin{center}
        %% NOTE: Add diagram
    \end{center}
    \AMCOpen{lines=3}{
        \wrongchoice[W]{w}\scoring{0}
        \wrongchoice[P]{p}\scoring{1}
        \correctchoice[C]{c}\scoring{2}
    }
\end{question}
}

\endinput


