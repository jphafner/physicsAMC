
%%--------------------------------------------------
%% CPO: Multiple Choice Questions
%%--------------------------------------------------


%% Chapter 17: Electromagnets and Induction
%%--------------------------------------------------


%% Learning Objectives
%%--------------------------------------------------

%% Describe the effect an electric current in a wire has on a compass.
%% Explain how to change the strength and direction of a wire’s magnetic field. 
%% Determine whether two wires or coils will attract or repel. 
%% Describe the role of magnets in electric motors.
%% Explain how electric motors operate. 
%% Learn the main parts of an electric motor.
%% Explain how a battery-powered motor works. 
%% Explain how a magnet can be used to produce current in a coil. 
%% Describe the design of a simple generator. 
%% Calculate the number of turns or voltage of a coil in a transformer.


%% CPO Multiple Choice Questions
%%--------------------------------------------------
\element{cpo-mc}{
\begin{question}{cpo-ch17-q01}
    When current flows through a coil of wire, you have a(n):
    \begin{multicols}{2}
    \begin{choices}
        \wrongchoice{superconductor}
      \correctchoice{electromagnet}
        \wrongchoice{ceramic magnet}
        \wrongchoice{semiconductor}
    \end{choices}
    \end{multicols}
\end{question}
}

\element{cpo-mc}{
\begin{question}{cpo-ch17-q02}
    Hans Christian {\O}rsted discovered that magnetism was related to electricity when he placed a compass needle in the presence of a current carrying wire and the:
    \begin{choices}
        \wrongchoice{current immediately turned off}
        \wrongchoice{current reversed direction}
      \correctchoice{compass needle moved as if the wire were a magnet}
        \wrongchoice{compass needle always pointed north}
    \end{choices}
\end{question}
}

\element{cpo-mc}{
\begin{question}{cpo-ch17-q03}
    The magnetic field is \emph{strongest} for a straight wire carrying electric current:
    \begin{choices}
        \wrongchoice{far from the wire}
      \correctchoice{close to the wire}
        \wrongchoice{at one end of the wire}
        \wrongchoice{There is no magnetic field around the wire}
    \end{choices}
\end{question}
}

\element{cpo-mc}{
\begin{question}{cpo-ch17-q04}
    The magnetic field is \emph{strongest} for a wire carrying electric current that has been looped into a coil:
    \begin{choices}
      \correctchoice{inside the coil}
        \wrongchoice{outside the coil}
        \wrongchoice{far away from the coil}
        \wrongchoice{There is no magnetic field around the wire}
    \end{choices}
\end{question}
}

\element{cpo-mc}{
\begin{question}{cpo-ch17-q05}
    A simple electromagnetic device consisting of a coil with many turns is known as a:
    \begin{multicols}{2}
    \begin{choices}
      \correctchoice{solenoid}
        \wrongchoice{electric motor}
        \wrongchoice{semiconductor}
        \wrongchoice{commutator}
    \end{choices}
    \end{multicols}
\end{question}
}

\element{cpo-mc}{
\begin{question}{cpo-ch17-q06}
    An electric charge creates magnetism by:
    \begin{multicols}{2}
    \begin{choices}
        \wrongchoice{static}
        \wrongchoice{isolation}
      \correctchoice{moving}
        \wrongchoice{domain}
    \end{choices}
    \end{multicols}
\end{question}
}

\element{cpo-mc}{
\begin{question}{cpo-ch17-q07}
    If you increase the current in an electromagnet:
    \begin{choices}
        \wrongchoice{the north and south poles will be switched}
      \correctchoice{the magnetic field will be stronger}
        \wrongchoice{the magnetic field will disappear}
        \wrongchoice{a short circuit will be created}
    \end{choices}
\end{question}
}

\element{cpo-mc}{
\begin{question}{cpo-ch17-q08}
    Solenoids and other coil electromagnets are used in all the following devices \emph{except}:
    \begin{multicols}{2}
    \begin{choices}
        \wrongchoice{speakers}
        \wrongchoice{electric motors}
        \wrongchoice{doorbells}
      \correctchoice{batteries}
    \end{choices}
    \end{multicols}
\end{question}
}

%\element{cpo-mc}{
%\begin{question}{cpo-ch17-q09}
%    Two parallel current-carrying wires next to each other
%        on a table are shown in the diagram:
%    \begin{center}
%        %% NOTE: add diagram
%    \end{center}
%    If their currents are in opposite direction:
%    \begin{multicols}{2}
%    \begin{choices}
%        \wrongchoice{the wires will lift off the table.}
%      \correctchoice{the wires will move away from each other.}
%        \wrongchoice{the wires will move toward each other.}
%        \wrongchoice{nothing will happen.}
%    \end{choices}
%    \end{multicols}
%\end{question}
%}

%\element{cpo-mc}{
%\begin{question}{cpo-ch17-q10}
%    An electric current moving upward through a straight wire creates
%        a magnetic field.
%    The diagram that correctly represents this magnetic field is:
%    \begin{multicols}{2}
%    \begin{choices}
%        %% NOTE: TODO: draw tikz, field lines
%    \end{choices}
%    \end{multicols}
%\end{question}
%}

\element{cpo-mc}{
\begin{question}{cpo-ch17-q11}
    The purpose of a commutator in an electric motor is to:
    \begin{choices}
        \wrongchoice{spin around so that the motor can do useful work.}
        \wrongchoice{create a voltage drop so that current flows.}
      \correctchoice{switch the electromagnets from north to south and back again.}
        \wrongchoice{attract and repel the magnets in the rotor.}
    \end{choices}
\end{question}
}

\element{cpo-mc}{
\begin{question}{cpo-ch17-q12}
    The device that switches the polarity of the electromagnets giving motion to the rotor in an electric motor is the:
    \begin{multicols}{2}
    \begin{choices}
        \wrongchoice{armature}
      \correctchoice{commutator}
        \wrongchoice{brush}
        \wrongchoice{knife switch}
    \end{choices}
    \end{multicols}
\end{question}
}

\element{cpo-mc}{
\begin{question}{cpo-ch17-q13}
    The rotating element of a motor is often referred to as the:
    \begin{multicols}{2}
    \begin{choices}
        \wrongchoice{field coil}
        \wrongchoice{brush}
        \wrongchoice{commutator}
      \correctchoice{armature}
    \end{choices}
    \end{multicols}
\end{question}
}

\element{cpo-mc}{
\begin{question}{cpo-ch17-q14}
    The main parts of a motor include all of the following \emph{except} a:
    \begin{choices}
        \wrongchoice{rotating element with magnets}
      \correctchoice{double-throw knife switch}
        \wrongchoice{stationary magnet surrounding the rotor}
        \wrongchoice{commutator}
    \end{choices}
\end{question}
}

\element{cpo-mc}{
\begin{question}{cpo-ch17-q15}
    An electric motor spins because:
    \begin{choices}
        \wrongchoice{the voltage pushes the motor around.}
      \correctchoice{an electromagnet attracts and then repels magnets in the rotor.}
        \wrongchoice{it is converting mechanical energy to electrical energy.}
        \wrongchoice{the electric current is always flowing in the same direction.}
    \end{choices}
\end{question}
}

\element{cpo-mc}{
\begin{question}{cpo-ch17-q16}
    A motor's basic function is to convert:
    \begin{choices}
      \correctchoice{electrical energy to mechanical energy.}
        \wrongchoice{alternating current to direct current.}
        \wrongchoice{mechanical energy to electrical energy.}
        \wrongchoice{high voltage to low voltage.}
    \end{choices}
\end{question}
}

%\element{cpo-mc}{
%\begin{question}{cpo-ch17-q17}
%    The diagram below represents the rotor of an electric motor.
%    \begin{center}
%        %% NOTE: add diagram
%    \end{center}
%    To cause the rotor to turn in a counter-clockwise direction, the
%        north pole of a magnet should be placed at position.
%    \begin{multicols}{2}
%    \begin{choices}
%        %% NOTE: change ABCD to IJKL
%      \correctchoice{$A$}
%        \wrongchoice{$B$}
%        \wrongchoice{$C$}
%        \wrongchoice{$D$}
%    \end{choices}
%    \end{multicols}
%\end{question}
%}

\element{cpo-mc}{
\begin{question}{cpo-ch17-q18}
    The device that uses electromagnetic induction to produce electricity is called a:
    \begin{multicols}{2}
    \begin{choices}
        \wrongchoice{motor}
        \wrongchoice{rotor}
        \wrongchoice{turbine}
      \correctchoice{generator}
    \end{choices}
    \end{multicols}
\end{question}
}

\element{cpo-mc}{
\begin{question}{cpo-ch17-q19}
    When a magnet moves into a coil of wire,
        electric current is caused to flow by:
    \begin{multicols}{2}
    \begin{choices}
        \wrongchoice{conduction}
        \wrongchoice{reduction}
      \correctchoice{induction}
        \wrongchoice{deduction}
    \end{choices}
    \end{multicols}
\end{question}
}

\element{cpo-mc}{
\begin{question}{cpo-ch17-q20}
    An electric generator produces:
    \begin{choices}
      \correctchoice{electrical energy}
        \wrongchoice{chemical energy}
        \wrongchoice{mechanical energy}
        \wrongchoice{variable force}
    \end{choices}
\end{question}
}

\element{cpo-mc}{
\begin{question}{cpo-ch17-q21}
    According to Faraday's law of induction,
        the faster a magnet is moved in and out of a coil:
    \begin{choices}
      \correctchoice{the greater the current produced.}
        \wrongchoice{the less current is produced.}
        \wrongchoice{the greater the capacitance produced.}
        \wrongchoice{the less capacitance is produced.}
    \end{choices}
\end{question}
}

\element{cpo-mc}{
\begin{questionmult}{ch17-Q22}
    A \emph{transformer} efficiently changes current and voltage:
    \begin{choices}
      \correctchoice{using electromagnetic induction}
      \correctchoice{using alternating current}
      \correctchoice{without changing power}
    \end{choices}
\end{questionmult}
}

\element{cpo-mc}{
\begin{question}{cpo-ch17-q23}
    Electromagnetic induction occurs when:
    \begin{choices}
        \wrongchoice{electromagnets are induced in a wire.}
        \wrongchoice{electrons are induced in a magnet by a moving wire.}
      \correctchoice{current is induced in a wire by a moving magnet.}
        \wrongchoice{a magnetic field is induced into a coil of wire by a current.}
    \end{choices}
\end{question}
}

\element{cpo-mc}{
\begin{question}{cpo-ch17-q24}
    A current can be induced to flow in a coil in all of the following situation \emph{except}:
    \begin{choices}
      \correctchoice{holding a powerful magnet in the middle of a coil of wire.}
        \wrongchoice{sliding a bar magnet into a coil of wire.}
        \wrongchoice{moving a coil of wire toward an electromagnet.}
        \wrongchoice{dropping a coil of wire over a bar magnet.}
    \end{choices}
\end{question}
}

\element{cpo-mc}{
\begin{question}{cpo-ch17-q25}
    A generator's basic function is to convert:
    \begin{choices}
        \wrongchoice{electrical energy to mechanical energy.}
        \wrongchoice{alternating current to direct current.}
      \correctchoice{mechanical energy to electrical energy.}
        \wrongchoice{high voltage to low voltage.}
    \end{choices}
\end{question}
}

\element{cpo-mc}{
\begin{question}{cpo-ch17-q26}
    the motion of a magnet that induces the greatest alternating current to flow in a coil is:
    \begin{choices}
        \wrongchoice{rapid movement in one direction through the coil.}
      \correctchoice{rapid back and force movement of the magnet through the coil.}
        \wrongchoice{slow movement in one direction through the coil.}
        \wrongchoice{no motion of the magnet.}
    \end{choices}
\end{question}
}

\element{cpo-mc}{
\begin{question}{cpo-ch17-q27}
    The battery charger for a rechargeable drill contains a transformer that converts \SI{120}{\volt} household voltage to the \SI{6}{\volt} required by the drill battery.
    If the primary coil of the transformer has \num{20} turns,
        how many turns are in the secondary coil?
    \begin{multicols}{2}
    \begin{choices}
        \wrongchoice{\num{20}}
        \wrongchoice{\num{6}}
        \wrongchoice{\num{2}}
      \correctchoice{\num{1}}
    \end{choices}
    \end{multicols}
\end{question}
}

\element{cpo-mc}{
\begin{question}{cpo-ch17-q28}
    A step down transformer has a primary voltage of \SI{120}{\volt}.
    The primary coil has \num{30} turns,
        and the secondary coil has \num{3} turns.
    The secondary voltage is:
    \begin{multicols}{2}
    \begin{choices}
        \wrongchoice{\SI{360}{\volt}}
        \wrongchoice{\SI{30}{\volt}}
      \correctchoice{\SI{12}{\volt}}
        \wrongchoice{\SI{9}{\volt}}
    \end{choices}
    \end{multicols}
\end{question}
}

\element{cpo-mc}{
\begin{question}{cpo-ch17-q29}
    An alternating current is produced by an electric generator because the:
    \begin{choices}
        \wrongchoice{chemical energy increases and decreases each turn.}
        \wrongchoice{nuclear energy increases and decreases each turn.}
      \correctchoice{magnetic field increases and decreases each turn.}
        \wrongchoice{magnetic field remain the same each turn.}
    \end{choices}
\end{question}
}

\element{cpo-mc}{
\begin{question}{cpo-ch17-q30}
    You want to build a transformer that steps up voltage.
    You will need:
    \begin{choices}
        \wrongchoice{more turns in the primary coil than the secondary coil.}
      \correctchoice{more turns in the secondary coil than the primary coil.}
        \wrongchoice{the same number of turns in the primary and secondary coil.}
        \wrongchoice{Not enough information given.}
    \end{choices}
\end{question}
}

\endinput

