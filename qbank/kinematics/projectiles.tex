

%% projectile motion 1
%%----------------------------------------
\element{projectile}{
\begin{question}{projectile-Q01}
    A boy throws a ball at an initial velocity of \SI{26}{\meter\per\second} at an angle of \ang{20} above the horizontal.
    How high above the projection point is the ball after \SI{1.4}{\second}?
    \begin{multicols}{2}
    \begin{choices}
        \wrongchoice{\SI{8.2}{\meter}}
        \wrongchoice{\SI{24}{\meter}}
        \wrongchoice{\SI{23}{\meter}}
      \correctchoice{\SI{2.8}{\meter}}
    \end{choices}
    \end{multicols}
\end{question}
}

\element{projectile}{
\begin{question}{projectile-Q02}
    The horizontal and vertical components of the initial velocity of a football are \SI{16}{\meter\per\second} and \SI{20}{\meter\per\second} respectively.
    How long does it take for the football to rise to the highest point of its trajectory?
    \begin{multicols}{2}
    \begin{choices}
        \wrongchoice{\SI{1.0}{\second}}
      \correctchoice{\SI{2.0}{\second}}
        \wrongchoice{\SI{3.0}{\second}}
        \wrongchoice{\SI{4.0}{\second}}
    \end{choices}
    \end{multicols}
\end{question}
}

\element{projectile}{
\begin{question}{projectile-Q03}
    A ball rolls over the edge of a table with a horizontal velocity of \SI{1.6}{\meter\per\second}.
    The height of the table is \SI{1.6}{\meter} and the horizontal range of the ball from the base of the table is \SI{20}{\meter}.
    What is the magnitude and direction of the ball's acceleration right after it leaves the table?
    \begin{multicols}{2}
    \begin{choices}
        \wrongchoice{\SI{4.9}{\second}}
        \wrongchoice{\SI{0.0}{\second}}
        \wrongchoice{\SI{19.6}{\second}}
      \correctchoice{\SI{9.8}{\second}}
    \end{choices}
    \end{multicols}
\end{question}
}

\element{projectile}{
\begin{question}{projectile-Q04}
    A hockey puck slides off the edge of a table with an initial velocity of \SI{20}{\meter\per\second}.
    The height of the table above the ground is \SI{2.0}{\meter}.
    What is the acceleration of the puck just before it touches the ground?
    \begin{multicols}{2}
    \begin{choices}
        \wrongchoice{\SI{20}{\meter\per\second\squared}}
        \wrongchoice{\SI{4.9}{\meter\per\second\squared}}
        \wrongchoice{\SI{19.6}{\meter\per\second\squared}}
      \correctchoice{\SI{9.8}{\meter\per\second\squared}}
    \end{choices}
    \end{multicols}
\end{question}
}

\element{projectile}{
\begin{question}{projectile-Q05}
    A ball rolls off the edge of a table.
    The horizontal component of the ball's velocity remains constant during its entire trajectory because
    \begin{choices}
        \wrongchoice{the ball is not acted upon by any force.}
        \wrongchoice{the net force acting on the ball is zero.}
      \correctchoice{the ball is not acted upon by a force in the horizontal direction.}
        \wrongchoice{None of the other choices is correct.}
    \end{choices}
\end{question}
}

\element{projectile}{
\begin{question}{projectile-Q06}
    A monkey is sitting at the top of a tree \SI{20}{\meter} high from the ground level.
    A person standing on the ground wants to feed the monkey.
    He uses a bow and arrow to launch the food to the monkey.
    If the person knows that the monkey is going to drop from the tree at the same instant that the person launches the food, how should the person aim the arrow containing the food?
    \begin{choices}
      \correctchoice{He should aim it at the monkey.}
        \wrongchoice{He should aim it below the monkey.}
        \wrongchoice{He should aim it above the monkey.}
        \wrongchoice{None of the other choices is correct.}
    \end{choices}
\end{question}
}


\element{projectile}{
\begin{question}{projectile-Q07}
    An athlete throws a ball with a velocity of \SI{40}{\meter\per\second} at an angle of \ang{20} above the horizontal.
    Which of the following statements is true in this case?
    \begin{choices}
        \wrongchoice{The vertical component of the velocity remains constant.}
        \wrongchoice{The horizontal component of the velocity changes.}
      \correctchoice{The vertical component of the velocity changes sign after the ball attains its maximum height.}
        \wrongchoice{The horizontal component of the velocity changes sign after the ball attains its maximum height.}
    \end{choices}
\end{question}
}


\element{projectile}{
\begin{question}{projectile-Q08}
    A bullet is fired from ground level with a speed of \SI{150}{\meter\per\second} at an angle \ang{30.0} above the horizontal.
    What is the vertical component of its velocity when it is at the highest point of its trajectory?
    [Assume gravitational acceleration is \SI{10.0}{\meter\per\second\squared}]
    \begin{multicols}{2}
    \begin{choices}
      \correctchoice{\SI{0}{\meter\per\second}}
        \wrongchoice{\SI{75}{\meter\per\second}}
        \wrongchoice{\SI{130}{\meter\per\second}}
        \wrongchoice{\SI{150}{\meter\per\second}}
    \end{choices}
    \end{multicols}
\end{question}
}


\element{projectile}{
\begin{question}{projectile-Q09}
    For general projectile motion, which statement is true when the projectile is at the initial and final points of the parabolic path?
    [Note: $x$ refers to the horizontal component, $y$ refers to the vertical component.]
    \begin{choices}
      \correctchoice{The magnitude of the $x$ and $y$ components of its velocity are same at both points.}
        \wrongchoice{The magnitude of its $x$ component of velocity at the initial point is smaller than its magnitude of the $y$ component of velocity at the final point.}
        \wrongchoice{The magnitude of its $x$ component of velocity at the initial point is bigger than its magnitude of the $y$ component of velocity at the final point.}
        \wrongchoice{The velocity components are zero at both points.}
    \end{choices}
\end{question}
}


\element{projectile}{
\begin{question}{projectile-Q10}
    James and John dive from an overhang into the lake below.
    James simply drops straight down from the edge.
    John takes a running start and jumps with an initial horizontal velocity of \SI{25}{\meter\per\second}.
    When they reach the lake below,
    \begin{choices}
        \wrongchoice{the splashdown speed of James is larger than that of John.}
      \correctchoice{the splashdown speed of John is larger than that of James.}
        \wrongchoice{they will both have the same splashdown speed.}
        \wrongchoice{the splashdown speed of James will always be \SI{9.8}{\meter\per\second} larger than that of John.}
    \end{choices}
\end{question}
}


\element{projectile}{
\begin{question}{projectile-Q11}
    Mary and Debra stand on a snow-covered roof.
    They both throw snowballs with the same initial speed,
        but in different directions.
    Mary throws her snowball downward, at \ang{30} below the horizontal;
        Debra throws her snowball upward, at \ang{30} above the horizontal. 
    When the snowballs reach the ground below,
    \begin{choices}
      \correctchoice{Debra's snowball will stay in the air longer than Mary's.}
        \wrongchoice{Mary's snowball will stay in the air longer than Debra's.}
        \wrongchoice{Both snowballs will take the same amount of time to hit the ground.}
        \wrongchoice{Debra's snowball never hits the ground since it is thrown upwards.}
    \end{choices}
\end{question}
}


\element{projectile}{
\begin{question}{projectile-Q12}
    If the initial speed of a projectile is doubled.
    \begin{choices}
        \wrongchoice{Its range will double.}
        \wrongchoice{Its range will be decreased by a factor of two.}
      \correctchoice{Its range will quadruple.}
        \wrongchoice{Its range will decrease by a factor of four.}
    \end{choices}
\end{question}
}


%% projectile motion 2
%%----------------------------------------
\element{projectile}{
\begin{question}{projectile-Q21}
    What is the value of the acceleration due to gravity near the surface of earth?
    \begin{multicols}{2}
    \begin{choices}
        \wrongchoice{\SI{-9.8}{ft\per\second\squared}}
        \wrongchoice{\SI{-9.8}{\meter\per\second}}
      \correctchoice{\SI{9.8}{\meter\per\second\squared}}
        \wrongchoice{\SI{32.2}{ft\per\second}}
    \end{choices}
    \end{multicols}
\end{question}
}

\element{projectile}{
\begin{question}{projectile-Q22}
    The acceleration due to gravity acts:
    \begin{choices}
        \wrongchoice{upward}
      \correctchoice{downward}
        \wrongchoice{depending on the motion of the object}
        \wrongchoice{toward outer sp}
    \end{choices}
\end{question}
}

\element{projectile}{
\begin{question}{projectile-Q23}
    A bullet is fired at an angle of \ang{45} from the horizontal.
    Neglecting air resistance,
        what is the direction of acceleration during the flight of the bullet?
    \begin{choices}
        \wrongchoice{upward}
      \correctchoice{downward}
        \wrongchoice{dependent on the initial velocity}
        \wrongchoice{\ang{45} from the horizontal}
    \end{choices}
\end{question}
}

\element{projectile}{
\begin{question}{projectile-Q24}
    A rocket is fired at \ang{30}.
    If the initial horizontal velocity ($v_{xi}$) is \SI{326}{\meter\per\second},
        what is the initial vertical velocity ($v_{yi}$)?
    \begin{multicols}{2}
    \begin{choices}
      \correctchoice{\SI{188}{\meter\per\second}}
        \wrongchoice{\SI{330}{\meter\per\second}}
        \wrongchoice{\SI{380}{\meter\per\second}}
        \wrongchoice{\SI{250}{\meter\per\second}}
    \end{choices}
    \end{multicols}
\end{question}
}

\element{projectile}{
\begin{question}{projectile-Q25}
    A golfer drives her golf ball from the tee down the fairway in a high arcing shot.
    When the ball is at the highest point of its flight:
    \begin{choices}
        \wrongchoice{the velocity and acceleration are both zero.}
        \wrongchoice{the horizontal velocity is zero and the vertical velocity is zero.}
      \correctchoice{the horizontal velocity is non-zero and the vertical velocity is zero.}
        \wrongchoice{the velocity is non-zero and the acceleration is zero.}
    \end{choices}
\end{question}
}

%\element{projectile}{
%\begin{question}{projectile-Q26}
%    In the ``Bull's Eye'' lab the ball rolled off the table.
%    What was the initial vertical ($v_{yi}$) velocity of the ball when it left the table?
%    \begin{choices}
%        \wrongchoice{depends on the initial x-velocity}
%        \wrongchoice{unknown; it must be measured}
%        \wrongchoice{depends on the size of the ball}
%        \wrongchoice{\SI{0}{\meter\per\second}} 
%    \end{choices}
%\end{question}
%}

\element{projectile}{
\begin{question}{projectile-Q27}
    A soccer ball is kicked at a \ang{55} to the horizontal with an initial velocity of \SI{8}{\meter\per\second}.
    What is the vertical component of velocity ($v_{yf}$) of the ball as it hits the ground?
    %\begin{multicols}{2}
    \begin{choices}
        \wrongchoice{\SI{8}{\meter\per\second}}
      \correctchoice{\SI{-6.6}{\meter\per\second}}
        \wrongchoice{\SI{-8}{\meter\per\second}}
        \wrongchoice{cannot be determined from given information}
    \end{choices}
    %\end{multicols}
\end{question}
}


%% projectile motion 3
%%----------------------------------------
\element{projectile}{
\begin{question}{projectile-Q31}
    What is the vertical component of a \SI{33}{\meter} vector that is at a \ang{76} angle with the horizontal?
    \begin{multicols}{3}
    \begin{choices}
        \wrongchoice{\SI{25}{\meter}}
        \wrongchoice{\SI{28}{\meter}}
        \wrongchoice{\SI{30}{\meter}}
      \correctchoice{\SI{32}{\meter}}
        \wrongchoice{\SI{33}{\meter}}
    \end{choices}
    \end{multicols}
\end{question}
}

\element{projectile}{
\begin{question}{projectile-Q32}
    A possible angle that a vector with a magnitude of \SI{17}{\meter} could make with the $x$ axis so that its horizontal component is \SI{12}{\meter} is:
    \begin{multicols}{3}
    \begin{choices}
        \wrongchoice{\ang{15}}
        \wrongchoice{\ang{30}}
      \correctchoice{\ang{45}}
        \wrongchoice{\ang{60}}
        \wrongchoice{\ang{75}}
    \end{choices}
    \end{multicols}
\end{question}
}

\element{projectile}{
\begin{question}{projectile-Q33}
    A ball is thrown horizontally out the window of a building with a velocity of \SI{8.0}{\meter\per\second} from a height of \SI{25}{\meter}.
    How far from the base of the building will the ball land?
    \begin{multicols}{3}
    \begin{choices}
        \wrongchoice{\SI{5.4}{\meter}}
        \wrongchoice{\SI{6.0}{\meter}}
        \wrongchoice{\SI{9.0}{\meter}}
        \wrongchoice{\SI{12}{\meter}}
      \correctchoice{\SI{18}{\meter}}
    \end{choices}
    \end{multicols}
\end{question}
}

\element{projectile}{
\begin{question}{projectile-Q34}
    Which vector best represents the displacement of a person who walks \SI{15}{\kilo\meter} at \ang{45} south of east, then \SI{30}{\kilo\meter} due west?
    \begin{multicols}{2}
    \begin{choices}
        \wrongchoice{\ang{17} west of south}
      \correctchoice{\ang{29} south of west}
        \wrongchoice{\ang{45} west of north}
        \wrongchoice{\ang{61} east of south}
        \wrongchoice{\ang{71} south of west}
    \end{choices}
    \end{multicols}
\end{question}
}

\element{projectile}{
\begin{question}{projectile-Q35}
    At its maximum height, the vertical component of velocity for an object projected at \SI{4.9}{\meter\per\second} at a \ang{60} to the horizontal is:
    \begin{multicols}{2}
    \begin{choices}
        \wrongchoice{\SI{9.8}{\meter\per\second}}
        \wrongchoice{\SI{4.9}{\meter\per\second}}
      \correctchoice{\SI{0}{\meter\per\second}}
        \wrongchoice{\SI{-4.9}{\meter\per\second}}
        \wrongchoice{\SI{-9.8}{\meter\per\second}}
    \end{choices}
    \end{multicols}
\end{question}
}

%\element{projectile}{
%\begin{question}{projectile-Q36}
%    A plane is heading to a destination
%        \SI{1750}{\kilo\meter} due north
%        at \SI{175}{\kilo\meter\per\hour} in 
%        a westward wind blowing \SI{25}{\kilo\meter\per\hour}.
%    At what angle from north should the plane be
%        oriented so that it reaches its destination?
%    \begin{multicols}{2}
%    \begin{choices}
%      \correctchoice{\ang{8}}
%        \wrongchoice{\ang{15}}
%        \wrongchoice{\ang{21}}
%        \wrongchoice{\ang{35}}
%        \wrongchoice{\ang{41}}
%    \end{choices}
%    \end{multicols}
%\end{question}
%}

%\element{projectile}{
%\begin{question}{projectile-Q37}
%    A plane is heading to a destination
%        \SI{1750}{\kilo\meter} due north
%        at \SI{175}{\kilo\meter\per\hour} in 
%        a westward wind blowing \SI{25}{\kilo\meter\per\hour}.
%    Without the compensatory angle described in question 36,
%        how far off course will the plane be after it has
%        traveled \SI{1750}{\kilo\meter} north?
%    \begin{multicols}{2}
%    \begin{choices}
%        \wrongchoice{\SI{150}{\kilo\meter}}
%        \wrongchoice{\SI{175}{\kilo\meter}}
%      \correctchoice{\SI{250}{\kilo\meter}}
%        \wrongchoice{\SI{375}{\kilo\meter}}
%        \wrongchoice{\SI{675}{\kilo\meter}}
%    \end{choices}
%    \end{multicols}
%\end{question}
%}

%\element{projectile}{
%\begin{question}{projectile-Q38}
%    A plane is heading to a destination
%        \SI{1750}{\kilo\meter} due north
%        at \SI{175}{\kilo\meter\per\hour} in 
%        a westward wind blowing \SI{25}{\kilo\meter\per\hour}.
%    Without the compensatory angle described in question 36,
%        What will be the magnitude of the velocity vector?
%    \begin{multicols}{2}
%    \begin{choices}
%        \wrongchoice{\SI{151}{\kilo\meter\per\hour}}
%        \wrongchoice{\SI{164}{\kilo\meter\per\hour}}
%        \wrongchoice{\SI{170}{\kilo\meter\per\hour}}
%        \wrongchoice{\SI{175}{\kilo\meter\per\hour}}
%      \correctchoice{\SI{177}{\kilo\meter\per\hour}}
%    \end{choices}
%    \end{multicols}
%\end{question}
%}

\element{projectile}{
\begin{question}{projectile-Q39}
    At what velocity should a ball be thrown at a \ang{45} so that it hits a target that is \SI{20}{\meter} away and placed at the same height?
    \begin{multicols}{2}
    \begin{choices}
        \wrongchoice{\SI{3}{\meter\per\second}}
        \wrongchoice{\SI{7}{\meter\per\second}}
        \wrongchoice{\SI{11}{\meter\per\second}}
      \correctchoice{\SI{14}{\meter\per\second}}
        \wrongchoice{\SI{18}{\meter\per\second}}
    \end{choices}
    \end{multicols}
\end{question}
}

\element{projectile}{
\begin{question}{projectile-Q40}
    A bullet fired horizontally from a height of three meters with a velocity of \SI{120}{\meter\per\second} will hit the ground after how many seconds?
    \begin{multicols}{2}
    \begin{choices}
        \wrongchoice{\SI{0.55}{\second}}
      \correctchoice{\SI{0.78}{\second}}
        \wrongchoice{\SI{0.96}{\second}}
        \wrongchoice{\SI{1.6}{\second}}
        \wrongchoice{\SI{2.8}{\second}}
    \end{choices}
    \end{multicols}
\end{question}
}


%% projectile motion 4
%%----------------------------------------
\element{projectile}{
\begin{question}{projectile-Q41}
    A ball is projected horizontally off the roof of a building with a speed of \SI{14.0}{\meter\per\second}.
    If the height of the roof is \SI{80}{\meter} and air resistance is neglible, what is the approximate time the ball is airborne?
    \begin{multicols}{2}
    \begin{choices}
        \wrongchoice{\SI{16.0}{\second}}
        \wrongchoice{\SI{3.0}{\second}}
        \wrongchoice{\SI{9.0}{\second}}
        \wrongchoice{\SI{81.0}{\second}}
      \correctchoice{\SI{4.0}{\second}}
    \end{choices}
    \end{multicols}
\end{question}
}

\element{projectile}{
\begin{question}{projectile-Q42}
    An object moving horizontally with speed $V$ falls off the edge of a vertical cliff and lands a distance $D$ from the base of the cliff.
    If it instead lands a distance $2D$ from the base of the cliff, how fast was it moving?
    Assume air resistance is neglible.
    %\begin{multicols}{2}
    \begin{choices}
        \wrongchoice{$V$}
        \wrongchoice{$\sqrt{2}V$}
      \correctchoice{$2V$}
        \wrongchoice{$4V$}
        \wrongchoice{It cannot be determined unless the height of the cliff is known.}
    \end{choices}
    %\end{multicols}
\end{question}
}



\element{projectile}{
\begin{question}{projectile-Q01}
    A diver initially moving horizontally with speed $v$
        dives off the edge of a vertical cliff and lands in the
        water a distance $d$ from the base of the cliff.
    How far from the base of the cliff would the diver have
        landed if the diver initially had been moving
        horizontally with speed $2v$?
    \begin{choices}
        \wrongchoice{$d$}
        \wrongchoice{$\sqrt{2}d$}
      \correctchoice{$2d$}
        \wrongchoice{$4d$}
        \wrongchoice{cannot be determined without knowing the height of the cliff}
    \end{choices}
\end{question}
}

\element{projectile}{
\begin{question}{projectile-Q01}
    A projectile is fired with initial velocity $v_i$
        at angle $\theta_i$ with the horizontal
        and follows the trajectory shown below.
        
    Which of the following pairs of graphs best represents
        the vertical components of the velocity and
        acceleration, $v$ and $a$, respectively,
        of the projectile as functions of time t?
    \begin{choices}
        \AMCboxDimensions{down=-1.5em}
        \correctchoice{
            \begin{tikzpicture}
                \begin{groupplot}[
                        mygroup,
                        group style={group size=2 by 1},
                    ]
                    \nextgroupplot[
                        xlabel={$t$},
                        ylabel={$x$},
                        xmin=0,xmax=10,
                        ymin=0,ymax=10,
                    ] \addplot[mystyle,domain=0:10] {0.1*x*x};
                    \nextgroupplot[
                        xlabel={$t$},
                        ylabel={$a$},
                        xmin=0,xmax=10,
                        ymin=0,ymax=10,
                    ] \addplot[mystyle,domain=0:10] {5};
                \end{groupplot}
            \end{tikzpicture}
        }
    \end{choices}
\end{question}
}

\element{projectile}{
\begin{question}{projectile-Q01}
    Two balls, projected at different times so they don't collide,
        have trajectories $A$ and $B$, as shown below.
    \begin{tikzpicture} 
        \draw [->] (0,0) -- (0,4.2); 
        \draw [domain=-2:2, samples=50] plot (\x, {1+cos(pi*\x r}));
        \draw [domain=-2:2, samples=50] plot (\x, {1+cos(pi*\x r}));
    \end{tikzpicture}
    Which statement is correct?
    \begin{choices}
        \wrongchoice{$v_{0B}$ must be greater than $v_{0A}$}
        \wrongchoice{$v_{0A}$ must be greater than $v_{0B}$}
        \wrongchoice{Ball $A$ is in the air longer than ball $B$}
        \wrongchoice{Ball $B$ is in the air longer than ball $A$}
        \wrongchoice{Ball $A$ has a greater acceleration than ball $B$}
        \wrongchoice{Ball $B$ has a greater acceleration than ball $A$}
    \end{choices}
\end{question}
}


%\element{projectile}{
%\begin{question}{projectile-Q01}
%    A rock is thrown from the edge of the top of a \SI{100}{ft}
%        tall building at some unknown angle above the horizontal.
%    The rock strikes the ground a horizontal distance of \SI{160}{ft}
%        from the base of the building \SI{5.0}{\second} after being thrown.
%    Assume that the ground is level and that the side of the building is vertical.
%    Determine the speed with which the rock was thrown
%    \begin{choices}
%        \wrongchoice{\SI{72}{ft\per\second}}
%        \wrongchoice{\SI{77}{ft\per\second}}
%      \correctchoice{\SI{68}{ft\per\second}}
%        \wrongchoice{\SI{82}{ft\per\second}}
%        \wrongchoice{\SI{87}{ft\per\second}}
%    \end{choices}
%\end{question}
%}

\element{projectile}{
\begin{question}{projectile-Q01}
    When a ball is kicked at an angle to the horizontal,
        what component of the ball's initial velocity determines the amount of time the ball will spend in the air?
    \begin{choices}
      \correctchoice{The vertical component}
        \wrongchoice{The horizontal component}
        \wrongchoice{Neither component, it is the angle that determines the time of flight.}
        \wrongchoice{Both components are needed.}
    \end{choices}
\end{question}
}


\element{projectile}{
\begin{question}{projectile-Q01}
    As the angle of launch with the horizontal increases
        for a projected object (i.e. a golf ball being hit),
        what happens to the horizontal component of the object's velocity?
    \begin{choices}
        \wrongchoice{It increases.}
        \wrongchoice{It decreases.}
      \correctchoice{It increases until it is a maximum at a \ang{45} angle, then decreases again.}
        \wrongchoice{It is unaffected by the launch angle.}
    \end{choices}
\end{question}
}


\element{projectile}{
\begin{question}{projectile-Q01}
    A ball is thrown and follows the parabolic path shown above.
    Air friction is negligible.
    Point $Q$ is the highest point on the path.
    Points $P$ and $R$ are the same height above the ground.
    How do the speeds of the ball at the three points compare?
    \begin{choices}
    (A) v P < v Q < v R
    (B) v R < v Q < v P
    (C) v Q < v R < v P
    (D) v Q < v P = v R
    (E) v P = v R < v Q
    \end{choices}
\end{question}
}

%% Part2: Which of the following diagrams best shows the
%%        direction of the acceleration of the ball at point P


\element{projectile}{
\begin{question}{projectile-Q01}
    A rock of mass $m$ is thrown horizontally off a building from
        a height $h$, as shown above.
    The speed of the rock as it leaves the thrower's hand at the
        edge of the building is $v_0$.
    How much time does it take the rock to travel from the edge
        of the building to the ground?
    \begin{choices}
    (A)
    (B) h/v 0
    (C) hv 0 / g
    (D) 2 h / g
    (E)
    \end{choices}
\end{question}
}


\element{projectile}{
\begin{question}{projectile-Q01}
    %% Inlude tikz diagram
    An object slides off a roof \SI{10}{\meter} above
        the ground with an initial horizontal speed of
        \SI{5}{\meter\per\second} as shown above.
    The time between the object's leaving the roof and
        hitting the ground is most nearly
    \begin{choices}
    \end{choices}
\end{question}
}


\element{projectile}{
\begin{question}{projectile-Q01}
    A projectile is fired from the surface of the Earth with
        a speed of \SI{200}{\meter\per\second} at an angle
        of \ang{30} above the horizontal.
    If the ground is level, what is the maximum height reached by the projectile?
    \begin{choices}
    (A) 5 m (B) 10 m (C) 500 m (D) 1,000 m (E) 2,000 m
    \end{choices}
\end{question}
}


\element{projectile}{
\begin{question}{projectile-Q01}
    A rock is dropped from the top of a \SI{45}{\meter} tower,
        and at the same time a ball is thrown from the top of
        the tower in a horizontal direction.
    Air resistance is negligible.
    The ball and the rock hit the level ground a distance of \SI{30}{\meter} apart.
    The horizontal velocity of the ball thrown was most nearly:
    \begin{multicols}{2}
    \begin{choices}
        \wrongchoice{\SI{5}{\meter\per\second}}
      \correctchoice{\SI{10}{\meter\per\second}}
        \wrongchoice{\SI{14.1}{\meter\per\second}}
        \wrongchoice{\SI{20}{\meter\per\second}}
        \wrongchoice{\SI{28.3}{\meter\per\second}}
    \end{choices}
    \end{multicols}
\end{question}
}


\element{projectile}{
\begin{question}{projectile-Q01}
    A ball is thrown and follows a parabolic path, as shown below.
    \begin{tikzpicture}
        \draw[smooth,domain=0:6.5] plot function{sin(2*x)*exp(-x/4)};
    \end{tikzpicture}
    Air friction is negligible.
    Point $Q$ is the highest point on the path.
    Which of the following best indicates the direction
        of the acceleration, if any, of the ball at point $Q$?
    \begin{choices}
        %% show a bunch of math arrows
    \end{choices}
\end{question}
}


\endinput


