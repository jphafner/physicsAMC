
%% NOTE: Take ideas from physicsHW/qbank

%% Algebra type problems
%%------------------------------

%% QUestion 3
\begin{comment}
 Question: In a speed trap, two pressure-activated strips are placed $120 {\rm m}$ apart on a highway on which the speed limit is $85 {\rm km/h}$. A driver going $110 {\rm km/h}$ notices a police car just as he/she activates the first strip, and immediately slows down. What deceleration is needed so that the car's average speed is within the speed limit when the car crosses the second strip?
 
Answer: Let $v_1= 110 {\rm km/h}$ be the speed of the car at the first strip. Let ${\mit\Delta} x = 120 {\rm m}$ be the distance between the two strips, and let ${\mit\Delta} t$ be the time taken by the car to travel from one strip to the other. The average velocity of the car is

\begin{displaymath} \bar{v} = \frac{{\mit\Delta}x}{{\mit\Delta} t}. \end{displaymath}

We need this velocity to be $85 {\rm km/h}$. Hence, we require

\begin{displaymath} {\mit\Delta}t = \frac{{\mit\Delta}x}{\bar{v}} = \frac{120}{85\times (1000/3600)} = 5.082 {\rm s}. \end{displaymath}

Here, we have changed units from ${\rm km/h}$ to ${\rm m/s}$. Now, assuming that the acceleration $a$ of the car is uniform, we have

\begin{displaymath} {\mit\Delta} x = v_1 {\mit\Delta} t + \frac{1}{2} a ({\mit\Delta t})^2, \end{displaymath}

which can be rearranged to give

\begin{displaymath} a = \frac{2 ({\mit\Delta}x - v_1 {\mit\Delta} t)}{({\mit\D... ...1000/3600)\times 5.082)}{(5.082)^2} = -2.73 {\rm m s^{-2}}. \end{displaymath}

Hence, the required deceleration is $2.73 {\rm m s^{-2}}$. 

\end{comment}


%% QUestion 3
\begin{comment}
 Question: In 1886, Steve Brodie achieved notoriety by allegedly jumping off the recently completed Brooklyn bridge, for a bet, and surviving. Given that the bridge rises 135ft over the East River, how long would Mr. Brodie have been in the air, and with what speed would he have struck the water? Give all answers in mks units. You may neglect air resistance.
  
  Answer: Mr. Brodie's net vertical displacement was $h = -135\times 0.3048 = -41.15 {\rm m}$. Assuming that his initial velocity was zero,

  \begin{displaymath} h = -\frac{1}{2} g t^2, \end{displaymath}

  where $t$ was his time of flight. Hence,

  \begin{displaymath} t = \sqrt{\frac{-2 h}{g}} = \sqrt{\frac{2\times 41.15}{9.81}} = 2.896 {\rm s}. \end{displaymath}

  His final velocity was

  \begin{displaymath} v =- g t = -9.81\times 2.896 = -28.41 {\rm m s}^{-1}. \end{displaymath}

  Thus, the speed with which he plunged into the East River was $28.41 {\rm m s}^{-1}$, or $63.6 {\rm mi/h}$! Clearly, Mr. Brodie's story should be taken with a pinch of salt. 

\end{comment}

