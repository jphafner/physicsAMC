

%% http://ctan.mirrorcatalogs.com/macros/latex/contrib/physics/physics.pdf
%%--------------------------------------------------------------------------------


%% GRE Physics 0177 Practice Exam
%%----------------------------------------

%% Page 12
\element{GREphysics}{
\begin{question}{GRE0177-Q01}
    Which of the following best illustrates the acceleration
        of a pendulum bob at points $a$ through $e$?
    \begin{choices}
        \wrongchoice{\includegraphics[keepaspectratio,scale=0.95]{GRE0177-Q01-A}}
        \wrongchoice{\includegraphics[keepaspectratio,scale=0.95]{GRE0177-Q01-B}}
      \correctchoice{\includegraphics[keepaspectratio,scale=0.95]{GRE0177-Q01-C}}
        \wrongchoice{\includegraphics[keepaspectratio,scale=0.95]{GRE0177-Q01-D}}
        \wrongchoice{\includegraphics[keepaspectratio,scale=0.95]{GRE0177-Q01-E}}
    \end{choices}
\end{question}
}

%% F = m \omega^2 r = m \mu g
\element{GREphysics}{
\begin{question}{GRE0177-Q02}
    The coefficient of static friction between a small coin and
        the surface of a turntable is \num{0.30}.
    The turntable rotates at \num{33.3} revolutions per minute.
    What is the maximum distance from the center of the
        turntable at which the coin will not slide?
    \begin{multicols}{2}
    \begin{choices}
        \wrongchoice{\SI{0.024}{\meter}}
        \wrongchoice{\SI{0.048}{\meter}}
        \wrongchoice{\SI{0.121}{\meter}}
      \correctchoice{\SI{0.242}{\meter}}
        \wrongchoice{\SI{0.484}{\meter}}
    \end{choices}
    \end{multicols}
\end{question}
}

%% F = v^2/t = GMm/r^2
%% V = 2 \pi r / T
\element{GREphysics}{
\begin{question}{GRE0177-Q03}
    A satellite of mass $m$ orbits a planet of mass $M$
        in a circular orbit of radius $R$.
    The time required for one revolution is
    \begin{choices}
        \wrongchoice{independent of $M$}
        \wrongchoice{proportional to  $\sqrt{M}$}
        \wrongchoice{linear in $R$}
      \correctchoice{proportional to $R^{3/2}$}
        \wrongchoice{proportional to $R^2$}
    \end{choices}
\end{question}
}

%% KE_i = mv_0^2, KE_f = 2/3 m v_0^2
\element{GREphysics}{
\begin{question}{GRE0177-Q04}
    In a nonrelativity, one-dimensional collision,
        a particle of mass $2m$ collides with a particle of
        mass $m$ at rest.
    If the particles stick together after the collision,
        what fraction of the initial kinetic energy is lost
        in the collision?
    \begin{multicols}{2}
    \begin{choices}
        \wrongchoice{\num{0}}
        \wrongchoice{\num{1/4}}
      \correctchoice{\num{1/3}}
        \wrongchoice{\num{1/2}}
        \wrongchoice{\num{2/3}}
    \end{choices}
    \end{multicols}
\end{question}
}

%% Page 14

%% There are six degrees of freedom
%% Three axial and three rotational
\element{GREphysics}{
\begin{question}{GRE0177-Q05}
    A three-dimensional harmonic oscillator is in thermal
        equilibrium with a temperature reservoir at
        temperature $T$.
    The average total energy of the oscillator is
    \begin{multicols}{2}
    \begin{choices}
        \wrongchoice{$\dfrac{kT}{2}$}
        \wrongchoice{$kT$}
        \wrongchoice{$\dfrac{3kT}{2}$}
      \correctchoice{$3kT$}
        \wrongchoice{$6kT$}
    \end{choices}
    \end{multicols}
\end{question}
}

%% in general -> dW = P dV
%% adiabatic  -> PV^\gamma = constant
%% monatomic  -> \gamma = 5/2
%% isothermal -> P = nRT / V
\element{GREphysics}{
\begin{question}{GRE0177-Q06}
    An ideal monatomic gas expands quasi-statically to twice its volume.
    If the process is isothermal,
        the work done by the gas is $W_i$.
    If the process is adiabatic,
        the work done by the gas is $W_a$.
    Which of the following is true?
    \begin{multicols}{2}
    \begin{choices}
        \wrongchoice{$W_i = W_a$}
        \wrongchoice{$0 = W_i < W_a$}
        \wrongchoice{$0 < W_i < W_a$}
        \wrongchoice{$0 = W_a < W_i$}
      \correctchoice{$0 < W_a < W_i$}
    \end{choices}
    \end{multicols}
\end{question}
}

\element{GREphysics}{
\begin{question}{GRE0177-Q07}
    Two long,
        identical bar magnets are placed under a horizontal piece of paper,
        as shown in the figure below.
    \begin{center}
        \includegraphics[keepaspectratio,scale=0.95]{GREphysics-GRE0177-Q07}
    \end{center}
    The paper is covered with iron filings.
    When the two north poles are small distance apart and touching the paper,
        the iron filings move into a patter that shows the magnetic field line.
    Which of the following best illustrates the patter that results?
    \begin{multicols}{2}
    \begin{choices}
        \wrongchoice{\includegraphics[keepaspectratio,scale=0.95]{GRE0177-Q07-A}}
      \correctchoice{\includegraphics[keepaspectratio,scale=0.95]{GRE0177-Q07-B}}
        \wrongchoice{\includegraphics[keepaspectratio,scale=0.95]{GRE0177-Q07-C}}
        \wrongchoice{\includegraphics[keepaspectratio,scale=0.95]{GRE0177-Q07-D}}
        \wrongchoice{\includegraphics[keepaspectratio,scale=0.95]{GRE0177-Q07-E}}
    \end{choices}
    \end{multicols}
\end{question}
}

%% Page 16
\element{GREphysics}{
\begin{question}{GRE0177-Q08}
    A positive charge $Q$ is located at a distance $L$,
        above an infinite grounded conducting plane,
        as shown in the figure above.
    What is the total charge induced on the plane?
    \begin{multicols}{2}
    \begin{choices}
        \wrongchoice{$2Q$}
        \wrongchoice{$Q$}
        \wrongchoice{$0$}
      \correctchoice{$-Q$}
        \wrongchoice{$-2Q$}
    \end{choices}
    \end{multicols}{2}
\end{question}
}

\element{GREphysics}{
\begin{question}{GRE0177-Q09}
    Five positive charges of magnitude $q$ are arranged symmetrically
        around the circumference of a circle of radius $r$.
    What is the magnitude of the electric fiedl at the center of the circle?
    ($k=\tfrac{1}{4\pi \epsilon_0}$)
    \begin{multicols}{2}
    \begin{choices}
      \correctchoice{$0$}
        \wrongchoice{$\dfrac{kq}{r^2}$}
        \wrongchoice{$\dfrac{5kq}{r^2}$}
        \wrongchoice{$\dfrac{kq}{r^2}\cos{\dfrac{2\pi}{5}}$}
        \wrongchoice{$\dfrac{5kq}{r^2}\cos{\dfrac{2\pi}{5}}$}
    \end{choices}
    \end{multicols}
\end{question}
}

%% 1/C_eq = 1/C_1 + 1/C_2
%% dW = V dq = q/C dq
%% W = 1/2 C V^2
\element{GREphysics}{
\begin{question}{GRE0177-Q10}
    A \SI{3}{\micro\farad} capacitor is connected in series
        with a \SI{6}{\micro\farad} capacitor.
    When a \SI{300}{\volt} potential difference is applied across this combination,
        the total energy stored in the two capacitors is
    \begin{choices}
      \correctchoice{\SI{0.09}{\joule}}
        \wrongchoice{\SI{0.18}{\joule}}
        \wrongchoice{\SI{0.27}{\joule}}
        \wrongchoice{\SI{0.41}{\joule}}
        \wrongchoice{\SI{0.81}{\joule}}
    \end{choices}
\end{question}
}

%% Use first image as second object
\element{GREphysics}{
\begin{question}{GRE0177-Q11}
    An object is located \SI{40}{\centi\meter} from the first
        two thin converging lenses of focal lengths
        \SI{20}{\centi\meter} and \SI{10}{\centi\meter},
        respecively, as shown in the figure below.
    \begin{center}
        \includegraphics[keepaspectratio,width=0.95\linewidth]{GREphysics-GRE0177-Q11}
    \end{center}
    The lenses are separated by \SI{30}{\centi\meter}.
    The final image formed by the two-lens system is located
    \begin{choices}
      \correctchoice{\SI{5.0}{\centi\meter} to the right of the second lens}
        \wrongchoice{\SI{13.3}{\centi\meter} to the right of the second lens}
        \wrongchoice{infinitely far to the right of the second lens}
        \wrongchoice{\SI{13.3}{\centi\meter} to the left of the second lens}
        \wrongchoice{\SI{100}{\centi\meter} to the left of the second lens}
    \end{choices}
\end{question}
}

%% only one option is virtual
\element{GREphysics}{
\begin{question}{GRE0177-Q12}
    A spherical, concave mirror is shown in the figure above.
    The focal point $F$ and the location of the object $O$
        are indicated.
    At what point will the image be located?
    \begin{multicols}{2}
    \begin{choices}
        \wrongchoice{I}
        \wrongchoice{II}
        \wrongchoice{III}
        \wrongchoice{IV}
      \correctchoice{V}
    \end{choices}
    \end{multicols}
\end{question}
}

%% Page 18
\element{GREphysics}{
\begin{question}{GRE0177-Q13}
    Two stars are separated by an angle of \SI3e-5}{\radian}.
    What is the diameter of the smallest telescope that can
        resolve the two stars using visible light
        ($\lambda = \SI{600}{\nano\meter}$)?
    [Ignore any effects due to Earth's atmosphere.]
    \begin{multicols}{2}
    \begin{choices}
        \wrongchoice{\SI{1}{\milli\meter}}
      \correctchoice{\SI{2.5}{\centi\meter}}
        \wrongchoice{\SI{10}{\centi\meter}}
        \wrongchoice{\SI{2.5}{\meter}}
        \wrongchoice{\SI{10}{\meter}}
    \end{choices}
    \end{multicols}
\end{question}
}

%% A_detector = \pi D^2 / 4
%% A_sphere   = 4 \pi R^2
%% ansser = A_detector / A_sphere
\element{GREphysics}{
\begin{question}{GRE0177-Q14}
    An \SI{8}{\centi\meter} diameter by \SI{8}{\centi\meter} long
        NaI(TI) detector detects gamma rays of a specific energy
        from a point source of radioactivity.
    What the source is placed just next to the detector at the
        center of the circular face, \SI{50}{\percent} of all
        emitted gamma rays at that energy are detected.
    If the detector is moved to \SI{1}{\meter} away,
        the fraction of detected gamma rays drops to
    \begin{multicols}{2}
    \begin{choices}
        \wrongchoice{\num{e-4}}
        \wrongchoice{\num{2e-4}}
      \correctchoice{\num{4e-4}}
        \wrongchoice{\num{8\pi e-4}}
        \wrongchoice{\num{16\pi e-4}}
    \end{choices}
    \end{multicols}
\end{question}
}

%% Testing discrimination of precision and accuracy
\element{GREphysics}{
\begin{question}{GRE0177-Q15}
    Five classes of students measure the height of a building.
    Each class uses a different method and each measures the height
        many difference times.
    The data for each class are plotted below.
    Which class made the most precise measurement?
    \begin{choices}
      \correctchoice{\includegraphics[keepaspectratio]{GRE0177-Q15-A}}
        \wrongchoice{\includegraphics[keepaspectratio]{GRE0177-Q15-B}}
        \wrongchoice{\includegraphics[keepaspectratio]{GRE0177-Q15-C}}
        \wrongchoice{\includegraphics[keepaspectratio]{GRE0177-Q15-D}}
        \wrongchoice{\includegraphics[keepaspectratio]{GRE0177-Q15-E}}
    \end{choices}
\end{question}
}

%% Page 20

%% Assume poisson distribution
%% Noise to data ratio is \sqrt{N}
\element{GREphysics}{
\begin{question}{GRE0177-Q16}
    A student makes \num{10} \num{1}-second measurements
        of the disintigration of a sample of a long-lived
        radioactive isotope and obtains the following values.
    \begin{equation*}
        3, 0, 2, 1, 2, 4, 0, 1, 2, 5
    \end{equation*}
    How long should the student count to establish the rate to
        an uncertainty of \SI{1}{\percent}
    \begin{multicols}{2}
    \begin{choices}
        \wrongchoice{\SI{80}{\second}}
        \wrongchoice{\SI{160}{\second}}
        \wrongchoice{\SI{2 000}{\second}}
      \correctchoice{\SI{5 000}{\second}}
        \wrongchoice{\SI{6 400}{\second}}
    \end{choices}
    \end{multicols}
\end{question}
}

\element{GREphysics}{
\begin{question}{GRE0177-Q17}
    The ground state electron configuration for phosphorus,
        which has \num{15} electrons, is
    \begin{multicols}{2}
    \begin{choices}
        \wrongchoice{$1s^2 2s^2 2p^6 3s^1 3p^4$}
        \wrongchoice{$1s^2 2s^2 2p^6 3s^2 3p^3$}
      \correctchoice{$1s^2 2s^2 2p^6 3s^2 3d^3$}
        \wrongchoice{$1s^2 2s^2 2p^6 3s^1 3d^4$}
        \wrongchoice{$1s^2 2s^2 2p^6 3p^2 3d^3$}
    \end{choices}
    \end{multicols}
\end{question}
}

\element{GREphysics}{
\begin{question}{GRE0177-Q18}
    The energy required to remove both electrons
        from the helium atom in its ground state
        is \SI{79.0}{\eV}.
    How much energy is required to ionize helium
        (i.e. to remove one electron)?
    \begin{choices}
      \correctchoice{\SI{24.6}{\eV}}
        \wrongchoice{\SI{39.5}{\eV}}
        \wrongchoice{\SI{51.8}{\eV}}
        \wrongchoice{\SI{54.4}{\eV}}
        \wrongchoice{\SI{65.4}{\eV}}
    \end{choices}
\end{question}
}

\element{GREphysics}{
\begin{question}{GRE0177-Q19}
    The primary source of the Sun's energy is a series
        of thermonuclear reactions in which the energy
        produced is $c^2$ times the mass difference
        between
    \begin{multicols}{2}
    \begin{choices}
        \wrongchoice{two hydrogen atoms and one helium atom}
      \correctchoice{four hydrogen atoms and one helium atom}
        \wrongchoice{six hydrogen atoms and two helium atom}
        \wrongchoice{three hydrogen atoms and one carbon atom}
        \wrongchoice{two hydrogen atoms plus two heliumn atom and one carbon atom}
    \end{choices}
    \end{multicols}
\end{question}
}

\element{GREphysics}{
\begin{question}{GRE0177-Q20}
    In the production of x-rays, the term ``bremsstrahlung'' refers
        to which of the following?
    \begin{choices}
        \wrongchoice{The cut-off wavelength, $\lambda_{min}$, of the
            x-ray tube} 
        \wrongchoice{The discrete x-ray lines emitted when an
            electron in an outer orbit fills a vacancy in
            an inner orbit of the atoms in the target
            metal of the x-ray tube}
        \wrongchoice{The discrete x-ray lines absorbed when an
            electron in an inner orbit fills a vacancy in
            an outer orbit of the atoms in the target
            metal of the x-ray tube}
        \wrongchoice{The smooth, continuous x-ray spectra
            produced by high-energy blackbody
            radiation from the x-ray tube}
      \correctchoice{The smooth, continuous x-ray spectra
            produced by rapidly decelerating electrons
            in the target metal of the x-ray tube}
    \end{choices}
\end{question}
}

%% E \prop 1/n^2
%% \lambda = c / f
\element{GREphysics}{
\begin{question}{GRE0177-Q21}
    In the hydrogen spectrum, the ratio of the wavelength for
        Lyman-$\alpha$ radiation ($n=2$ to $n=1$) to 
        Balmer-$\alpha$ radiation ($n=3$ to $n=2$) is
    \begin{multicols}{2}
    \begin{choices}
        \wrongchoice{\num{5/48}}
      \correctchoice{\num{5/27}}
        \wrongchoice{\num{1/3}}
        \wrongchoice{\num{3}}
        \wrongchoice{\num{27/5}}
    \end{choices}
    \end{multicols}
\end{question}
}

%% Page 22
\element{GREphysics}{
\begin{question}{GRE0177-Q22}
    An astronomer observes a very small moon orbiting a planet
        and measure the moon's minimum and maximum distances
        from the planet's center and the moon's maximum
        orbital speed.
    Which of the following CANNOT be calculated from these
        measurements?
    \begin{multicols}{2}
    \begin{choices}
      \correctchoice{Mass of the moon}
        \wrongchoice{Mass of the planet}
        \wrongchoice{Minimum speed of the moon}
        \wrongchoice{Period of the orbit}
        \wrongchoice{Semimajor axis of the orbit}
    \end{choices}
    \end{multicols}
\end{question}
}

\element{GREphysics}{
\begin{question}{GRE0177-Q23}
    A particel is constrained to move in a circle with a
        \SI{10}{\meter} radius.
    At one instance, the particle's speed is \SI{10}{\meter\per\second}
        and is increasing at a rate of \SI{10}{\meter\per\second\squared}.
    The angle between the particle's velocity and acceleration
        vector is
    \begin{multicols}{2}
    \begin{choices}
        \wrongchoice{\ang{0}}
        \wrongchoice{\ang{30}}
      \correctchoice{\ang{45}}
        \wrongchoice{\ang{60}}
        \wrongchoice{\ang{90}}
    \end{choices}
    \end{multicols}
\end{question}
}

\element{GREphysics}{
\begin{question}{GRE0177-Q24}
    A stone is thrown at an angle of \ang{45} above the horizontal
        $x$-axis in the $+x$-direction.
    If air resistance is ignored, which of the velocity
        versus time graphs shown above best represents
        $v_x$ verses $t$ and $v_y$ verse $t$, respecively?
    \includegraphics[keepaspectratio]{GRE0177-Q24-I}
    \includegraphics[keepaspectratio]{GRE0177-Q24-II}
    \includegraphics[keepaspectratio]{GRE0177-Q24-III}
    \includegraphics[keepaspectratio]{GRE0177-Q24-IV}
    \includegraphics[keepaspectratio]{GRE0177-Q24-V}
    \begin{choices}
        \wrongghoice{$v_x vs. t$: I, $v_y vs t$: IV}
        \wrongchoice{$v_x vs. t$: II, $v_y vs t$: I}
        \wrongchoice{$v_x vs. t$: II, $v_y vs t$: III}
      \correctchoice{$v_x vs. t$: II, $v_y vs t$: V}
        \wrongchoice{$v_x vs. t$: IV, $v_y vs t$: V}
    \end{choices}
\end{question}
}

%% Page 26
\element{GREphysics}{
\begin{question}{GRE0177-Q25}
    Seven pennies are arranged in a hexagonal, planar
        pattern so as to touch each neighbor, as shown in
        the figure below.
    \begin{center}
        \includegraphics[keepaspectratio,width=0.95\linewidth]{GREphysics-GRE0177-Q25}
    \end{center}
    Each penny is a uniform disk of mass $m$ and radius $r$.
    What is the moment of inertia of the system of seven
        about an axis that passes through the center of
        the central penny and is normal to the plane of
        the pennies?
    %% Parallel axis rule -> I = I_cm + MR^2
    %% moment of disk -> I = 1/2 m r^2
    \begin{multicols}{2}
    \begin{choices}
        \wrongchoice{$\dfrac{7 m r^2}{2}$}
        \wrongchoice{$\dfrac{13 m r^2}{2}$}
        \wrongchoice{$\dfrac{29 m r^2}{2}$}
        \wrongchoice{$\dfrac{49 m r^2}{2}$}
      \correctchoice{$\dfrac{55 m r^2}{2}$}
    \end{choices}
    \end{multicols}
\end{question}
}

\element{GREphysics}{
\begin{question}{GRE0177-Q26}
    A thin uniform rod of mass $M$ and length $L$ is
        positioned vertically above an anchored frictionless
        pivot point, as shown below, and then allowed to
        fall to the ground.
    \begin{center}
        \includegraphics[keepaspectratio]{GREphysics-GRE0177-Q26}
    \end{center}
    With what speed does the free end of the rod strike the
        ground?
    %% Think conservation of energy
    %% E = mgh + 1/2 mv^2 + 1/2 I \omega^2
    \begin{choices}
        \wrongchoice{$\sqrt{\frac{1}{3}gL}$}
        \wrongchoice{$\sqrt{gL}$}
      \correctchoice{$\sqrt{3gL}$}
        \wrongchoice{$\sqrt{12 gL}$}
        \wrongchoice{$12 \sqrt{gL}$}
    \end{choices}
\end{question}
}

\element{GREphysics}{
\begin{question}{GRE0177-Q27}
    The eigenvalues of a Hermitian operator are always
    \begin{choices}
      \correctchoice{real}
        \wrongchoice{imaginary}
        \wrongchoice{degenerate}
        \wrongchoice{linear}
        \wrongchoice{positive}
    \end{choices}
\end{question}
}

\element{GREphysics}{
\begin{question}{GRE0177-Q28}
    The states \ket{1}, \ket{2}, and \ket{3} are orthonormal.
    \begin{align*}
        \ket{\lambda_1} &= 5 \ket{1} - 3 \ket{2} + 2 \ket{3} \\
        \ket{\lambda_2} &=   \ket{1} - 5 \ket{2} + x \ket{3}
    \end{align*}
    For what values of $x$ are the states \ket{$\lambda_1$}
        and \ket{$\lambda_2$} given above orthogonal?
    \begin{multicols}{2}
    \begin{choices}
        \wrongchoice{10}
        \wrongchoice{5}
        \wrongchoice{0}
        \wrongchoice{-5}
      \correctchoice{-10}
    \end{choices}
    \end{multicols}
\end{question}
}

\element{GREphysics}{
\begin{question}{GRE0177-Q29}
    The state
    $\lambda = \frac{1}{\sqrt{6}} \lambda_{-1} + \frac{1}{\sqrt{2} + \frac{1 $
    is a linear combination of three orthonormal eigenstates of the
        operator $\hat{O}$ corresponding to eigenstates $-1$, $1$, and $2$.
    What is the expectation value of $\hat{O}$ for this state?
    %% <\phi|O|\phi> = -1/6 + 1/2 + 2/3
    \begin{multicols}{2}
    \begin{choices}
        \wrongchoice{$\dfrac{2}{3}$}
        \wrongchoice{$\sqrt{\dfrac{7}{6}}$}
      \correctchoice{$1$}
        \wrongchoice{$\dfrac{4}{3}$}
        \wrongchoice{$\dfrac{\sqrt{3} + 2\sqrt{2} - 1}{\sqrt{6}}$}
    \end{choices}
    \end{multicols}
\end{question}
}

%% Page 28
\element{GREphysics}{
\begin{question}{GRE0177-Q30}
    Which of the following functions could represent the radial
        wave function for an electron in an atom?
    ($r$ is the distance of the electron from the nucleus: $A$ and $b$ are constants.)
    \begin{itemize}
        \item[I]    $A e^{-br}$
        \item[II]   $A \sin{br}$
        \item[I*I]  $\frac{A}{r}$
    \end{itemize}
    %% Radial function must be exponentially decreasing
    \begin{multicols}{2}
    \begin{choices}
      \correctchoice{I only}
        \wrongchoice{II only}
        \wrongchoice{I and II only}
        \wrongchoice{I and III only}
        \wrongchoice{I, II, and III}
    \end{choices}
    \end{multicols}
\end{question}
}

%% Page 5
\element{GREphysics}{
\begin{question}{GRE0177-Q31}
    Positronium is an atom formed by an electron and a positron (antielectron).
    It is similar to the hydrogen atom, with the positron replacing the proton.
    If a positronium atom makes a transition from the state with
        $n=3$ to a state with $n=1$, the energy of the photon emitted in this
        transition is closest to
    %% Energy level of positronium is half those of Hydrogen
    %% 6.8eV instead of 13.6 eV.
    \begin{choices}
      \correctchoice{\SI{6.0}{\eV}}
        \wrongchoice{\SI{6.8}{\eV}}
        \wrongchoice{\SI{12.2}{\eV}}
        \wrongchoice{\SI{13.6}{\eV}}
        \wrongchoice{\SI{24.2}{\eV}}
    \end{choices}
\end{question}
}

\element{GREphysics}{
\begin{question}{GRE0177-Q32}
    If the total energy of a particle of mass $m$ is equal to twice its
        rest energy, then the magnitude of the particle's relativisitic momentum is
    %% E^2 = m^2 + p^2 = 4
    \begin{multicols}{2}
    \begin{choices}
        \wrontchoice{$\frac{mc}{2}$}
        \wrongchoice{$\frac{mc}{\sqrt{2}}$}
        \wrongchoice{$mc$}
      \correctchoice{$\sqrt{3}mc$}
        \wrongchoice{$2mc$}
    \end{choices}
    \end{multicols}
\end{question}
}

\element{GREphysics}{
\begin{question}{GRE0177-Q33}
    If a charged pion that decays in \SI{1e-8}{\second} in its own rest
        frame is to travel \SI{30}{\meter} in the laboratory before decaying,
        the pion's speed must be most nearly
    %% proper time of 1e-8: t = \gamma t_0
    %% lab frame of 30:  L = 30 = vt 
    \begin{choices}
        \wrongchoice{\SI{0.44e8}{\meter\per\second}}
        \wrongchoice{\SI{2.84e8}{\meter\per\second}}
        \wrongchoice{\SI{2.90e8}{\meter\per\second}}
      \correctchoice{\SI{2.98e8}{\meter\per\second}}
        \wrongchoice{\SI{3.00e8}{\meter\per\second}}
    \end{choices}
\end{question}
}

\element{GREphysics}{
\begin{question}{GRE0177-Q34}
    In an inertial reference frame $S$, two events occur on the $x$-axis
        separated in time by $\delta{}t$ and in space by $\delta{}x$.
    In another inertial reference frame $S'$, moving in the $x$-direction
        relative to $S$, the two events could occur at the same time under
        which, if any, of the following conditions?
    \begin{multicols}{2}
    \begin{choices}
        \wrongchoice{For any values of $\delta{}x$ and $\delta{}t$}
        \wrongchoice{Only if $\left| \frac{\delta{}x}{\delta{}t} \right| < c$}
      \correctchoice{Only if $\left| \frac{\delta{}x}{\delta{}t} \right| > c$}
        \wrongchoice{Only if $\left| \frac{\delta{}x}{\delta{}t} \right| = c$}
        \wrongchoice{Under no condition}
    \end{choices}
    \end{multicols}
\end{question}
}

\element{GREphysics}{
\begin{question}{GRE0177-Q35}
    If the absolute temperature of a blackbody is increased by a factor of $3$,
        the energy radiated per unit area does which of the following?
    \begin{multicols}{2}
    \begin{choices}
        \wrongchoice{Decreases by a factor of $81$}
        \wrongchoice{Decreases by a factor of $9$}
        \wrongchoice{Increases by a factor of $9$}
        \wrongchoice{Increases by a factor of $27$}
      \correctchoice{Increases by a factor of $81$}
    \end{choices}
    \end{multicols}
\end{question}
}

%% Page 30
\element{GREphysics}{
\begin{question}{GRE0177-Q36}
    Consider a quasi-static adiabatic expansion of an
        ideal gas from an initial state $i$ to a final
        state $f$.
    Which of the following statements is \emph{not} true?
    \begin{choices}
        \wrongchoice{No heat flows into or out of the gas}
        \wrongchoice{The entropy of state $i$ equals the entropy of state $f$.}
        \wrongchoice{The change of internal energy of the gas is $-\int{}P\,\mathrm{d}V$}
        \wrongchoice{The mechanical work done by the gas is $\int{}P\,\mathrm{d}V$}
      \correctchoice{The temperature of the gas remains constant.}
    \end{choices}
    \end{choices}
\end{question}
}

\element{GREphysics}{
\begin{question}{GRE0177-Q37}
    \begin{center}
        \includegraphics[keepaspectratio]{GREphysics-GRE0177-Q37}
    \end{center}
    A constant amount of an ideal gas undergoes the cyclic process $ABCA$
        in the $PV$ diagram shown above.
    The path $BC$ is isothermal.
    The work done by the gas during one complete cycle,
        beginning and ending at $A$, is most nearly
    \begin{multicols}{3}
    \begin{choices}
        \wrongchoice{\SI{600}{\kilo\joule}}
        \wrongchoice{\SI{300}{\kilo\joule}}
        \wrongchoice{\num{0}}
        \wrongchoice{\SI{-300}{\kilo\joule}}
        \wrongchoice{\SI{-600}{\kilo\joule}}
    \end{choices}
    \end{multicols}
\end{question}
}

\element{GREphysics}{
\begin{question}{GRE0177-Q38}
    \begin{center}
        \includegraphics[keepaspectratio]{GREphysics-GRE0177-Q37}
    \end{center}
    An $AC$ circuit consists of the elements shown above,
        with $R=\SI{10000}{\ohm}$, $L=\SI{25}{\milli\henry}$,
        and C an adjustable capacitance.
    The $AC$ voltage generator supplies a signal with an
        amplitude of \SI{40}{\volt} and angular frequency
        of \SI{1000}{\radian\per\second}.
    For what value of $C$ is the ampitude of the current maximized?
    \begin{multicols}{2}
    \begin{choices}
        \wrongchoice{\SI{4}{\nano\farad}}
        \wrongchoice{\SI{40}{\nano\farad}}
        \wrongchoice{\SI{4}{\micro\farad}}
        \wrongchoice{\SI{40}{\micro\farad}}
        \wrongchoice{\SI{400}{\micro\farad}}
    \end{choices}
    \end{multicols}
\end{question}
}

\element{GREphysics}{
\begin{question}{GRE0177-Q39}
    Which two of the following circuits are high-pass filters?
    \begin{multicols}{2}
        %% Enumerate with numbers and change roman to arabic
        \includegraphics[width=linewidth,keepaspectratio]{GRE0177-Q39-I}
        \includegraphics[width=linewidth,keepaspectratio]{GRE0177-Q39-II}
        \includegraphics[width=linewidth,keepaspectratio]{GRE0177-Q39-III}
        \includegraphics[width=linewidth,keepaspectratio]{GRE0177-Q39-IV}
    \end{multicols}{2}
    \begin{multicols}{2}
    \begin{choices}
        \wrongchoice{I and II}
        \wrongchoice{I and III}
        \wrongchoice{I and IV}
        \wrongchoice{II and III}
        \wrongchoice{II and IV}
    \end{choices}
    \end{multicols}
\end{question}
}

%% TODO: Start writing here
\element{GREphysics}{
\begin{question}{GRE0177-Q40}
    In the circuit shown above, the switch $S$ is closed at $t=0$.
    Which of the following best rerepsents the voltage across the inductor,
        as seen on an oscilloscope?
    \begin{multicols}{2}
    \begin{choices}
        \wrongchoice{
            \begin{tikzpicture}
            \end{tikzpicture}
        }
    \end{choices}
    \end{multicols}
\end{question}
}

%% 40: D

%% 41: E
%% 42: C
%% 43: D
%% 44: D
%% 45: B

%% 46: E
%% 47: B
%% 48: C
%% 49: E
%% 50: B

%% 51: B
%% 52: C
%% 53: B
%% 54: C
%% 55: E

%% 56: D
%% 57: A
%% 58: B
%% 59: B
%% 60: D

\element{GREphysics}{
\begin{question}{GRE0177-Q41}
    Which graph best represents the relationship between the
        absolute index of refraction and the speed of light
        ($f=\SI{5.09e14}{\hertz}$ in various media?
    \begin{multicols}{2}
    \begin{choices}
      \correctchoice{\includegraphics[keepaspectratio,width=\linewidth]{GRE0177-Q41-B}}
        \wrongchoice{\includegraphics[keepaspectratio,width=\linewidth]{GRE0177-Q41-C}}
        \wrongchoice{\includegraphics[keepaspectratio,width=\linewidth]{GRE0177-Q41-D}}
        \wrongchoice{\includegraphics[keepaspectratio,width=\linewidth]{GRE0177-Q41-A}}
    \end{choices}
    \end{multicols}
\end{question}
}

\element{GREphysics}{
\begin{question}{GRE0177-Q42}
    A \SI{25}{\gram} paper cup falls from rest off the edge of
        a tabletop \SI{0.90}{\meter} above the floor.
    If the cup has \SI{0.20}{\joule} of kinetic energy when
        it hits the floor, what is the total amount of energy
        converted into internal thermal energy during the
        cups fall?
    \begin{multicols}{2}
    \begin{choices}
      \correctchoice{\SI{0.02}{\joule}}
        \wrongchoice{\SI{2.2}{\joule}}
        \wrongchoice{\SI{0.22}{\joule}}
        \wrongchoice{\SI{2.20}{\joule}}
    \end{choices}
    \end{multicols}
\end{question}
}

\element{GREphysics}{
\begin{question}{GRE0177-Q43}
    Which electron transition between energy levels of hydrogen
        causes the emission of a photon of visible light?
    \begin{multicols}{2}
    \begin{choices}
      \correctchoice{$n=6$ to $n=5$}
        \wrongchoice{$n=5$ to $n=6$}
        \wrongchoice{$n=5$ to $n=2$}
        \wrongchoice{$n=2$ to $n=5$}
    \end{choices}
    \end{multicols}
\end{question}
}

%% Page 8
\element{GREphysics}{
\begin{question}{GRE0177-Q44}
    Which graph best represents an object in equilibrium moving in a straight line?
    \begin{multicols}{2}
    \begin{choices}
      \correctchoice{\includegraphics[keepaspectratio,width=\linewidth]{GRE0177-Q44-D}}
        \wrongchoice{\includegraphics[keepaspectratio,width=\linewidth]{GRE0177-Q44-C}}
        \wrongchoice{\includegraphics[keepaspectratio,width=\linewidth]{GRE0177-Q44-B}}
        \wrongchoice{\includegraphics[keepaspectratio,width=\linewidth]{GRE0177-Q44-A}}
    \end{choices}
    \end{multicols}
\end{question}
}

\element{GREphysics}{
\begin{question}{GRE0177-Q45}
    A body, $B$, is moving at constant speed in a horizontal circular path
        around point $P$.
    Which diagram shows the direction of the velocity ($v$) and the direction
        of the centripetal force ($F_c$) acting on the body?
    \begin{multicols}{2}
    \begin{choices}
      \correctchoice{\includegraphics[keepaspectratio,width=\linewidth]{GRE0177-Q45-C}}
        \wrongchoice{\includegraphics[keepaspectratio,width=\linewidth]{GRE0177-Q45-D}}
        \wrongchoice{\includegraphics[keepaspectratio,width=\linewidth]{GRE0177-Q45-A}}
        \wrongchoice{\includegraphics[keepaspectratio,width=\linewidth]{GRE0177-Q45-B}}
    \end{choices}
    \end{multicols}
\end{question}
}

\element{GREphysics}{
\begin{question}{GRE0177-Q46}
    Which graph best represents the relationship between photon energy
        and photon wavelength?
    \begin{multicols}{2}
    \begin{choices}
      \correctchoice{\includegraphics[keepaspectratio,width=\linewidth]{GRE0177-Q46-A}}
        \wrongchoice{\includegraphics[keepaspectratio,width=\linewidth]{GRE0177-Q46-B}}
        \wrongchoice{\includegraphics[keepaspectratio,width=\linewidth]{GRE0177-Q46-C}}
        \wrongchoice{\includegraphics[keepaspectratio,width=\linewidth]{GRE0177-Q46-D}}
    \end{choices}
    \end{multicols}
\end{question}
}

\element{GREphysics}{
\begin{question}{GRE0177-Q47}
    Which combination of initial horizontal velocity, ($v_H$), and
        initial vertical velocity, ($v_V$), results in the greatest
        horizontal range for a projectile over level ground?
        [Neglect friction.]
    \begin{multicols}{2}
    \begin{choices}
      \correctchoice{\includegraphics[keepaspectratio,width=\linewidth]{GRE0177-Q47-C}}
        \wrongchoice{\includegraphics[keepaspectratio,width=\linewidth]{GRE0177-Q47-D}}
        \wrongchoice{\includegraphics[keepaspectratio,width=\linewidth]{GRE0177-Q47-A}}
        \wrongchoice{\includegraphics[keepaspectratio,width=\linewidth]{GRE0177-Q47-B}}
    \end{choices}
    \end{multicols}
\end{question}
}

\element{GREphysics}{
\begin{question}{GRE0177-Q48}
    Which graph best represents the greatest amount of work?
    \begin{multicols}{2}
    \begin{choices}
      \correctchoice{\includegraphics[keepaspectratio,width=\linewidth]{GRE0177-Q48-B}}
        \wrongchoice{\includegraphics[keepaspectratio,width=\linewidth]{GRE0177-Q48-C}}
        \wrongchoice{\includegraphics[keepaspectratio,width=\linewidth]{GRE0177-Q48-D}}
        \wrongchoice{\includegraphics[keepaspectratio,width=\linewidth]{GRE0177-Q48-A}}
    \end{choices}
    \end{multicols}
\end{question}
}

\element{GREphysics}{
\begin{question}{GRE0177-Q49}
    When a ray of light traveling in water reaches a boundary with air,
        part of the light ray is reflected and part is refracted.
    Which ray diagram best represents the paths of the reflected and
        refracted light rays?
    \begin{multicols}{2}
    \begin{choices}
      \correctchoice{\includegraphics[keepaspectratio,width=\linewidth]{GRE0177-Q49-A}}
        \wrongchoice{\includegraphics[keepaspectratio,width=\linewidth]{GRE0177-Q49-B}}
        \wrongchoice{\includegraphics[keepaspectratio,width=\linewidth]{GRE0177-Q49-C}}
        \wrongchoice{\includegraphics[keepaspectratio,width=\linewidth]{GRE0177-Q49-D}}
    \end{choices}
    \end{multicols}
\end{question}
}

\element{GREphysics}{
\begin{question}{GRE0177-Q50}
    The graph below represents the work done against gravity by a
        student as she walks up a flight of stairs at constant speed.
    \begin{center}
        \includegraphics[keepaspectratio,width=0.95\linewidth]{GREphysics-GRE0177-Q50}
    \end{center}
    Compared to the power generated by the student after \SI{2.0}{\second}
        the power generated by the student after \SI{4.0}{\second} is
    \begin{choices}
      \correctchoice{twice as great}
        \wrongchoice{the same}
        \wrongchoice{half as great}
        \wrongchoice{four times as great}
    \end{choices}
\end{question}
}

\element{GREphysics}{
\begin{question}{GRE0177-Q51}
    The graph below represents the work done against gravity by a
        student as she walks up a flight of stairs at constant speed.
    \begin{center}
        \includegraphics[keepaspectratio,width=0.95\linewidth]{GREphysics-GRE0177-Q50}
    \end{center}
    Compared to the power generated by the student after \SI{2.0}{\second}
        the power generated by the student after \SI{4.0}{\second} is
    \begin{choices}
      \correctchoice{twice as great}
        \wrongchoice{the same}
        \wrongchoice{half as great}
        \wrongchoice{four times as great}
    \end{choices}
\end{question}
}


\endinput


