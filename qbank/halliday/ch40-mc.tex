
%%--------------------------------------------------
%% Halliday: Fundamentals of Physics
%%--------------------------------------------------


%% Chapter 40: All About Atoms
%%--------------------------------------------------


%% Learning Objectives
%%--------------------------------------------------

%% 40.01: Discuss the pattern that is seen in a plot of ionization energies versus atomic number $Z$.
%% 40.02: Identify that atoms have angular momentum and magnetism.
%% 40.03: Explain the Einstein--de Haas experiment.
%% 40.04: Identify the five quantum numbers of an electron in an atom and the allowed values of each.
%% 40.05: Determine the number of electron states allowed in a given shell and subshell.
%% 40.06: Identify that an electron in an atom has an orbital angular momentum $\vec{L}$ and an orbital magnetic dipole moment $\vec{\mu}_{orb}$.
%% 40.07: Calculate magnitudes for orbital angular momentum $\vec{L}$ and orbital magnetic dipole moment $\vec{\mu}_{orb}$ in terms of the orbital quantum number $l$.
%% 40.08: Apply the relationship between orbital angular momentum $\vec{L}$ and orbital magnetic dipole moment $\vec{\mu}_{orb}$.
%% 40.09: Identify that $\vec{L}$ and $\vec{\mu}_{orb}$ cannot be observed (measured) but a component on a measurement axis (usually called the $z$ axis) can.
%% 40.10: Calculate the $z$ components $L_z$ of an orbital angular momentum $L$ using the orbital magnetic quantum number m / .
%% 40.11: Calculate the $z$ components $\vec{\mu}_{orb,z}$ of an orbital magnetic dipole moment $\vec{\mu}_{orb}$ using the orbital magnetic quantum number $m_l$ and the Bohr magneton $m_B$.
%% 40.12: For a given orbital state or spin state, calculate the semiclassical angle $\theta$.
%% 40.13: Identify that a spin angular momentum $\vec{S}$ (usually simply called spin) and a spin magnetic dipole moment $\vec{\mu}_s$ are intrinsic properties of electrons (and also protons and neutrons).
%% 40.14: Calculate magnitudes for spin angular momentum $\vec{S}$ and spin magnetic dipole moment m s in terms of the spin quantum number $s$.
%% 40.15: Apply the relationship between the spin angular momentum $\vec{S}$ and the spin magnetic dipole moment $\vec{\mu}_s$.
%% 40.16: Identify that $\vec{S}$ and $\vec{\mu}_s$ cannot be observed (measured) but a component on a measurement axis can.
%% 40.17: Calculate the $z$ components $S_z$ of the spin angular momentum $\vec{S}$ using the spin magnetic quantum number $\vec{\mu}_s$.
%% 40.18: Calculate the $z$ components $\mu_{s,z}$ of the spin magnetic dipole moment $\vec{\mu}_s$ using the spin magnetic quantum number $\vec{\mu}_s$ and the Bohr magneton $m_B$.
%% 40.19: Identify the effective magnetic dipole moment of an atom.


%% Halliday Multiple Choice Questions
%%--------------------------------------------------
\element{halliday-mc}{
\begin{question}{halliday-ch40-q01}
    The magnitude of the orbital angular momentum of an electron in an atom is what multiple of $\hbar$?  ($l$ is a positive integer.)
    \begin{multicols}{3}
    \begin{choices}
        \wrongchoice{$1$}
        \wrongchoice{$\dfrac{1}{2}$}
      \correctchoice{$\sqrt{l(l+1)}$}
        \wrongchoice{$2l + 1$}
        \wrongchoice{$l^2$}
    \end{choices}
    \end{multicols}
\end{question}
}

\element{halliday-mc}{
\begin{question}{halliday-ch40-q02}
    The magnetic quantum number $m_l$ is most closely associated with what property of an electron in an atom?
    \begin{choices}
        \wrongchoice{Magnitude of the orbital angular momentum}
        \wrongchoice{Energy}
        \wrongchoice{$z$ component of the spin angular momentum}
      \correctchoice{$z$ component of the orbital angular momentum}
        \wrongchoice{Radius of the orbit}
    \end{choices}
\end{question}
}

\element{halliday-mc}{
\begin{question}{halliday-ch40-q03}
    The quantum number $m_s$ is most closely associated with what property of the electron in an atom?
    \begin{choices}
        \wrongchoice{Magnitude of the orbital angular momentum}
        \wrongchoice{Energy}
      \correctchoice{$z$ component of the spin angular momentum}
        \wrongchoice{$z$ component of the orbital angular momentum}
        \wrongchoice{Radius of the orbit}
    \end{choices}
\end{question}
}

\element{halliday-mc}{
\begin{question}{halliday-ch40-q04}
    Possible values of the principal quantum number $n$ for an electron in an atom are:
    \begin{choices}
        \wrongchoice{only $0$ and $1$}
        \wrongchoice{only $0$, $1$, $2$, \ldots , $\infty$}
        \wrongchoice{only $0$, $1$, \ldots , $-1$}
        \wrongchoice{only $\dfrac{1}{2}$ and $-\dfrac{1}{2}$}
      \correctchoice{only $1$, $2$, $3$, \ldots , $\infty$}
    \end{choices}
\end{question}
}

\element{halliday-mc}{
\begin{question}{halliday-ch40-q05}
    The number of values of the orbital quantum number number $n=3$ is:
    \begin{multicols}{3}
    \begin{choices}
        \wrongchoice{\num{1}}
        \wrongchoice{\num{2}}
      \correctchoice{\num{3}}
        \wrongchoice{\num{4}}
        \wrongchoice{\num{7}}
    \end{choices}
    \end{multicols}
\end{question}
}

\element{halliday-mc}{
\begin{question}{halliday-ch40-q06}
    The number of possible values of the magnetic quantum number $m_l$ associated with a given value of the orbital quantum number $l$ is:
    \begin{multicols}{3}
    \begin{choices}
        \wrongchoice{$1$}
        \wrongchoice{$2$}
        \wrongchoice{$l$}
        \wrongchoice{$2l$}
        \wrongchoice{$2l+1$}
    \end{choices}
    \end{multicols}
\end{question}
}

\element{halliday-mc}{
\begin{question}{halliday-ch40-q07}
    An atom is in a state with orbital quantum number $l=2$.
    Possible values of the magnetic quantum number $m_l$ are:
    \begin{multicols}{2}
    \begin{choices}
        \wrongchoice{$1$, $2$}
        \wrongchoice{$0$, $1$, $2$}
        \wrongchoice{$0$, $1$}
        \wrongchoice{$-1$, $0$, $1$}
      \correctchoice{$-2$, $-1$, $0$, $1$, $2$}
    \end{choices}
    \end{multicols}
\end{question}
}

\element{halliday-mc}{
\begin{question}{halliday-ch40-q08}
    An electron is in a quantum state for which the magnitude of the orbital angular momentum is $6\sqrt{2}\hbar$.
    How many allowed values of the $z$ component of the angular momentum are there?
    \begin{multicols}{3}
    \begin{choices}
        \wrongchoice{\num{4}}
        \wrongchoice{\num{5}}
        \wrongchoice{\num{7}}
      \correctchoice{\num{8}}
        \wrongchoice{\num{9}}
    \end{choices}
    \end{multicols}
\end{question}
}

\element{halliday-mc}{
\begin{question}{halliday-ch40-q09}
    An electron is in a quantum state for which there are seven allowed values of the $z$ component of the angular momentum. 
    The magnitude of the angular momentum is:
    \begin{multicols}{3}
    \begin{choices}
        \wrongchoice{$\sqrt{3}\hbar$}
        \wrongchoice{$\sqrt{7}\hbar$}
        \wrongchoice{$\sqrt{9}\hbar$}
      \correctchoice{$\sqrt{12}\hbar$}
        \wrongchoice{$\sqrt{14}\hbar$}
    \end{choices}
    \end{multicols}
\end{question}
}

\element{halliday-mc}{
\begin{question}{halliday-ch40-q10}
    The number of states in a subshell with orbital quantum number $l=3$ is
    \begin{multicols}{3}
    \begin{choices}
        \wrongchoice{\num{2}}
        \wrongchoice{\num{3}}
        \wrongchoice{\num{7}}
        \wrongchoice{\num{9}}
      \correctchoice{\num{14}}
    \end{choices}
    \end{multicols}
\end{question}
}

\element{halliday-mc}{
\begin{question}{halliday-ch40-q11}
    The number of states in a shell with principal quantum number $n=3$ is:
    \begin{multicols}{3}
    \begin{choices}
        \wrongchoice{\num{3}}
        \wrongchoice{\num{9}}
        \wrongchoice{\num{15}}
      \correctchoice{\num{18}}
        \wrongchoice{\num{25}}
    \end{choices}
    \end{multicols}
\end{question}
}

\element{halliday-mc}{
\begin{question}{halliday-ch40-q12}
    An electron in an atom is in a state with principal quantum number $n=4$. 
    The possible values of the orbital quantum number are:
    \begin{choices}
        \wrongchoice{$1$, $2$, $3$}
        \wrongchoice{$1$, $2$, $3$, $4$}
        \wrongchoice{$-3$, $-2$, $-1$, $0$, $1$, $2$, $3$}
      \correctchoice{$0$, $1$, $2$, $3$}
        \wrongchoice{$0$, $1$, $2$}
    \end{choices}
\end{question}
}

\element{halliday-mc}{
\begin{question}{halliday-ch40-q13}
    Space quantization means that:
    \begin{choices}
        \wrongchoice{space is quantized}
      \correctchoice{$L_z$ can have only certain discrete values}
        \wrongchoice{$\vec{L}$ and $-\vec{μ}$ are in the same direction}
        \wrongchoice{$\vec{L}$ and $-\vec{μ}$ are in opposite directions}
        \wrongchoice{an electron has a magnetic dipole moment}
    \end{choices}
\end{question}
}

\element{halliday-mc}{
\begin{question}{halliday-ch40-q14}
    An electron in an atom is in a state with $l=3$ and $m_l=2$.
    The angle between $\vec{L}$ and the $z$ axis is:
    \begin{multicols}{3}
    \begin{choices}
        \wrongchoice{\ang{48.2}}
        \wrongchoice{\ang{60}}
        \wrongchoice{\ang{30}}
        \wrongchoice{\ang{35.3}}
      \correctchoice{\ang{54.7}}
    \end{choices}
    \end{multicols}
\end{question}
}

\element{halliday-mc}{
\begin{question}{halliday-ch40-q15}
    An electron in an atom is in a state with $l=5$.
    The minimum angle between $\vec{L}$ and the $z$ axis is:
    \begin{multicols}{3}
    \begin{choices}
        \wrongchoice{\ang{0}}
        \wrongchoice{\ang{18.0}}
      \correctchoice{\ang{24.1}}
        \wrongchoice{\ang{36.7}}
        \wrongchoice{\ang{33.6}}
    \end{choices}
    \end{multicols}
\end{question}
}

\element{halliday-mc}{
\begin{question}{halliday-ch40-q16}
    In the relation $\mu_z=−m_l \mu_B$,
        the quantity $\mu_B$ is:
    \begin{choices}
      \correctchoice{the Bohr magneton}
        \wrongchoice{the component of the dipole moment along the magnetic field}
        \wrongchoice{the permeability of the material}
        \wrongchoice{a friction coefficient}
        \wrongchoice{none of the provided}
    \end{choices}
\end{question}
}

\element{halliday-mc}{
\begin{question}{halliday-ch40-q17}
    The electron states that constitute a single shell for an atom all have:
    \begin{choices}
      \correctchoice{the same value of $n$}
        \wrongchoice{the same value of $l$}
        \wrongchoice{the same value of $n$ and the same value of $l$}
        \wrongchoice{the same value of $l$ and the same value of $m_l$}
        \wrongchoice{the same set of all four quantum numbers}
    \end{choices}
\end{question}
}

\element{halliday-mc}{
\begin{question}{halliday-ch40-q18}
    The electron states that constitute a single subshell for an atom all have:
    \begin{choices}
        \wrongchoice{the same value of $n$}
        \wrongchoice{the same value of $l$}
      \correctchoice{the same value of $n$ and the same value of $l$}
        \wrongchoice{the same value of $l$ and the same value of $m_l$}
        \wrongchoice{the same set of all four quantum numbers}
    \end{choices}
\end{question}
}

\element{halliday-mc}{
\begin{question}{halliday-ch40-q19}
    The total number of electron states with $n=2$ and $l=1$ for an atom is:
    \begin{multicols}{3}
    \begin{choices}
        \wrongchoice{two}
        \wrongchoice{four}
      \correctchoice{six}
        \wrongchoice{eight}
        \wrongchoice{ten}
    \end{choices}
    \end{multicols}
\end{question}
}

\element{halliday-mc}{
\begin{question}{halliday-ch40-q20}
    The possible values for the magnetic quantum number $m_s$ of an electron in an atom:
    \begin{choices}
        \wrongchoice{depend on $n$}
        \wrongchoice{depend on $l$}
        \wrongchoice{depend on both $n$ and $l$}
        \wrongchoice{depend on whether there is an external magnetic field present}
      \correctchoice{are $\pm\dfrac{1}{2}$}
    \end{choices}
\end{question}
}

\element{halliday-mc}{
\begin{question}{halliday-ch40-q21}
    The Stern-Gerlach experiment makes use of:
    \begin{choices}
        \wrongchoice{a strong uniform magnetic field}
      \correctchoice{a strong non-uniform magnetic field}
        \wrongchoice{a strong uniform electric field}
        \wrongchoice{a strong non-uniform electric field}
        \wrongchoice{strong perpendicular electric and magnetic fields}
    \end{choices}
\end{question}
}

\element{halliday-mc}{
\begin{question}{halliday-ch40-q22}
    The magnetic field $\vec{B}$ is along the $z$ axis in a Stern-Gerlach experiment. 
    The force it exerts on a magnetic dipole with dipole moment $\vec{\mu}$ is proportional to:
    \begin{multicols}{3}
    \begin{choices}
        \wrongchoice{$\mu_z^2$}
        \wrongchoice{$B^2$}
      \correctchoice{$\dfrac{\mathrm{d}B}{\mathrm{d}z}$}
        \wrongchoice{$\dfrac{\mathrm{d}^2B}{\mathrm{d}z^2}$}
        \wrongchoice{$\int\,B\,\mathrm{d}z$}
    \end{choices}
    \end{multicols}
\end{question}
}

\element{halliday-mc}{
\begin{question}{halliday-ch40-q23}
    A magnetic dipole $\vec{\mu}$ is placed in a strong uniform magnetic field $\vec{B}$. 
    The associated force exerted on the dipole is:
    \begin{multicols}{2}
    \begin{choices}
        \wrongchoice{along $\vec{\mu}$}
        \wrongchoice{along $-\vec{\mu}$}
        \wrongchoice{along $\vec{B}$}
        \wrongchoice{along $\vec{\mu}\times\vec{B}$}
      \correctchoice{zero}
    \end{choices}
    \end{multicols}
\end{question}
}

\element{halliday-mc}{
\begin{question}{halliday-ch40-q24}
    The force exerted on a magnetic dipole as it moves with velocity $v$ through a Stern-Gerlach apparatus is:
    \begin{multicols}{2}
    \begin{choices}
        \wrongchoice{proportional to $v$}
        \wrongchoice{proportional to $\dfrac{1}{v}$}
        \wrongchoice{zero}
        \wrongchoice{proportional to $v^2$}
      \correctchoice{independent of $v$}
    \end{choices}
    \end{multicols}
\end{question}
}

\element{halliday-mc}{
\begin{question}{halliday-ch40-q25}
    A magnetic dipole is placed between the poles of a magnet as shown. 
    The direction of the associated force exerted on the dipole is:
    \begin{choices}
        \wrongchoice{positive $x$}
        \wrongchoice{positive $y$}
      \correctchoice{negative $x$}
        \wrongchoice{negative $y$}
        \wrongchoice{into or out of the page}
    \end{choices}
\end{question}
}

\element{halliday-mc}{
\begin{question}{halliday-ch40-q26}
    To observe the Zeeman effect one uses:
    \begin{choices}
        \wrongchoice{a strong uniform magnetic field}
      \correctchoice{a strong non-uniform magnetic field}
        \wrongchoice{a strong uniform electric field}
        \wrongchoice{a strong non-uniform electric field}
        \wrongchoice{mutually perpendicular electric and magnetic fields}
    \end{choices}
\end{question}
}

\element{halliday-mc}{
\begin{question}{halliday-ch40-q27}
    An electron in a K shell of an atom has the principal quantum number:
    \begin{multicols}{3}
    \begin{choices}
        \wrongchoice{$n=0$}
      \correctchoice{$n=1$}
        \wrongchoice{$n=2$}
        \wrongchoice{$n=3$}
        \wrongchoice{$n=\infty$}
    \end{choices}
    \end{multicols}
\end{question}
}

\element{halliday-mc}{
\begin{question}{halliday-ch40-q28}
    An electron in an L shell of an atom has the principal quantum number:
    \begin{multicols}{3}
    \begin{choices}
        \wrongchoice{$n=0$}
        \wrongchoice{$n=1$}
      \correctchoice{$n=2$}
        \wrongchoice{$n=3$}
        \wrongchoice{$n=\infty$}
    \end{choices}
    \end{multicols}
\end{question}
}

\element{halliday-mc}{
\begin{question}{halliday-ch40-q29}
    The most energetic photon in a continuous x-ray spectrum has an energy approximately equal to:
    \begin{choices}
        \wrongchoice{the energy of all the electrons in a target atom}
      \correctchoice{the kinetic energy of an incident-beam electron}
        \wrongchoice{the rest energy, $mc^2$ , of an electron}
        \wrongchoice{the total energy of a K-electron in the target atom}
        \wrongchoice{the kinetic energy of a K-electron in the target atom}
    \end{choices}
\end{question}
}

\element{halliday-mc}{
\begin{question}{halliday-ch40-q30}
    Two different electron beams are incident on two different targets and both produce x rays.
    The cutoff wavelength for target 1 is shorter than the cutoff wavelength for target 2. 
    We can conclude that:
    \begin{choices}
        \wrongchoice{target 2 has a higher atomic number than target 1}
        \wrongchoice{target 2 has a lower atomic number than target 1}
      \correctchoice{the electrons in beam 1 have greater kinetic energy than those in beam 2}
        \wrongchoice{the electrons in beam 1 have less kinetic energy than those in beam 2}
        \wrongchoice{target 1 is thicker than target 2}
    \end{choices}
\end{question}
}

\element{halliday-mc}{
\begin{question}{halliday-ch40-q31}
    A photon with the smallest wavelength in the continuous z-ray spectrum is emitted when:
    \begin{choices}
        \wrongchoice{an electron is knocked from a K shell}
        \wrongchoice{a valence electron is knocked from the atom}
        \wrongchoice{the incident electron becomes bound to the atom}
        \wrongchoice{the atom has the greatest recoil energy}
      \correctchoice{the incident electron loses all its energy in a single decelerating event}
    \end{choices}
\end{question}
}

\element{halliday-mc}{
\begin{question}{halliday-ch40-q32}
    Radiation with the minimum wavelength as well as the K x-ray lines are detected for a certain target. 
    The energy of the incident electrons is then doubled,
        with the result that:
    \begin{choices}
        \wrongchoice{the minimum wavelength increases and the wavelengths of the K lines remain the same}
      \correctchoice{the minimum wavelength decreases and the wavelengths of the K lines remain the same}
        \wrongchoice{the minimum wavelength and the wavelengths of the K lines all increase}
        \wrongchoice{the minimum wavelength and the wavelengths of the K lines all decrease}
        \wrongchoice{the minimum wavelength increases and the wavelengths of the K lines all decrease}
    \end{choices}
\end{question}
}

\element{halliday-mc}{
\begin{question}{halliday-ch40-q33}
    Characteristic K x-radiation of an element occurs when:
    \begin{choices}
        \wrongchoice{the incident electron is absorbed by a target nucleus}
        \wrongchoice{the incident electron is scattered by a target atom without an energy loss}
        \wrongchoice{an electron is ejected from an outer shell of a target atom}
      \correctchoice{an electron in a target atom makes a transition to the lowest energy state}
        \wrongchoice{the incident electron goes into the lowest energy state}
    \end{choices}
\end{question}
}

\element{halliday-mc}{
\begin{question}{halliday-ch40-q34}
    The K$_{\alpha}$ x rays arising from a cobalt ($Z=27$) target have a wavelength of about \SI{179}{\pico\meter}. 
    The atomic number of a target that gives rise to K$_{\alpha}$ x rays with a wavelength one-third as great ($\approx\SI{60}{\pico\meter}$) is:
    \begin{multicols}{3}
    \begin{choices}
        \wrongchoice{$Z = 9$}
        \wrongchoice{$Z = 10$}
        \wrongchoice{$Z = 12$}
        \wrongchoice{$Z = 16$}
      \correctchoice{$Z = 46$}
    \end{choices}
    \end{multicols}
\end{question}
}

\element{halliday-mc}{
\begin{question}{halliday-ch40-q35}
    In connection with x-ray emission the symbol K$_{\alpha}$ refers to:
    \begin{choices}
        \wrongchoice{an alpha particle radiation}
        \wrongchoice{an effect of the dielectric constant on energy levels}
        \wrongchoice{x-ray radiation from potassium}
        \wrongchoice{x-ray radiation associated with an electron going from $n=\infty$ to $n=1$}
      \correctchoice{x-ray radiation associated with an electron going from $n=2$ to $n=1$}
    \end{choices}
\end{question}
}

\element{halliday-mc}{
\begin{question}{halliday-ch40-q36}
    In connection with x-ray emission the symbol L$_{\beta}$ refers to:
    \begin{choices}
        \wrongchoice{a beta particle radiation}
        \wrongchoice{an atomic state of angular momentum $\dfrac{h}{2\pi}$}
        \wrongchoice{the inductance associated with an orbiting electron}
      \correctchoice{x-radiation associated with an electron going from $n=4$ to $n=2$}
        \wrongchoice{none of the above}
    \end{choices}
\end{question}
}

\element{halliday-mc}{
\begin{question}{halliday-ch40-q37}
    The transition shown gives rise to an x-ray. 
    \begin{center}
    \begin{tikzpicture}
        %% NOTE:
    \end{tikzpicture}
    \end{center}
    The correct label for this is:
    \begin{multicols}{3}
    \begin{choices}
        \wrongchoice{K$_{\alpha}$}
        \wrongchoice{K$_{\beta}$}
        \wrongchoice{L$_{\alpha}$}
        \wrongchoice{L$_{\beta}$}
        \wrongchoice{KL}
    \end{choices}
    \end{multicols}
\end{question}
}

\element{halliday-mc}{
\begin{question}{halliday-ch40-q38}
    In a Moseley graph:
    \begin{choices}
        \wrongchoice{the x-ray frequency is plotted as a function of atomic number}
        \wrongchoice{the square of the x-ray frequency is plotted as a function of atomic number}
      \correctchoice{the square root of the x-ray frequency is plotted as a function of atomic number}
        \wrongchoice{the x-ray frequency is plotted as a function of the square root of atomic number}
        \wrongchoice{the square root of the x-ray frequency is plotted as a function of atomic mass}
    \end{choices}
\end{question}
}

\element{halliday-mc}{
\begin{question}{halliday-ch40-q39}
    In calculating the x-ray energy levels the effective charge of the nucleus is taken to be $Z-b$,
        where $Z$ is the atomic number. 
    The parameter $b$ enters because:
    \begin{choices}
        \wrongchoice{an electron is removed from the inner shell}
        \wrongchoice{a proton is removed from the nucleus}
        \wrongchoice{the quantum mechanical force between two charges is less than the classical force}
      \correctchoice{the nucleus is screened by electrons}
        \wrongchoice{the Pauli exclusion principle must be obeyed}
    \end{choices}
\end{question}
}

\element{halliday-mc}{
\begin{question}{halliday-ch40-q40}
    The ratio of the wavelength of the K$_{\alpha}$ x-ray line for Nb (Z = 41) to that of Ga (Z = 31) is:
    \begin{multicols}{3}
    \begin{choices}
      \correctchoice{$\dfrac{9}{16}$}
        \wrongchoice{$\dfrac{16}{9}$}
        \wrongchoice{$\dfrac{3}{4}$}
        \wrongchoice{$\dfrac{4}{3}$}
        \wrongchoice{$1.15$}
    \end{choices}
    \end{multicols}
\end{question}
}

\element{halliday-mc}{
\begin{question}{halliday-ch40-q41}
    The Pauli exclusion principle is obeyed by:
    \begin{choices}
        \wrongchoice{all particles}
        \wrongchoice{all charged particles}
      \correctchoice{all particles with spin quantum numbers of $\dfrac{1}{2}$}
        \wrongchoice{all particles with spin quantum numbers of $1$}
        \wrongchoice{all particles with mass}
    \end{choices}
\end{question}
}

\element{halliday-mc}{
\begin{question}{halliday-ch40-q42}
    No state in an atom can be occupied by more than one electron. 
    This is most closely related to:
    \begin{choices}
        \wrongchoice{the wave nature of matter}
        \wrongchoice{the finite value for the speed of light}
        \wrongchoice{the Bohr magneton}
      \correctchoice{the Pauli exclusion principle}
        \wrongchoice{the Einstein-de Haas effect}
    \end{choices}
\end{question}
}

\element{halliday-mc}{
\begin{question}{halliday-ch40-q43}
    Electrons are in a two-dimensional square potential energy well with sides of length $L$. 
    The potential energy is infinite at the sides and zero inside. 
    The single-particle energies are given by $\left(\dfrac{h^2}{8mL^2}\right)\left(n_x^2+n_y^2\right)$,
        where $n_x$ and $n_y$ are integers. 
    At most the number of electrons that can have energy $8\left(\dfrac{h^2}{8mL^2}\right)$ is:
    \begin{multicols}{2}
    \begin{choices}
        \wrongchoice{1}
      \correctchoice{2}
        \wrongchoice{3}
        \wrongchoice{4}
        \wrongchoice{any number}
    \end{choices}
    \end{multicols}
\end{question}
}

\element{halliday-mc}{
\begin{question}{halliday-ch40-q44}
    Five electrons are in a two-dimensional square potential energy well with sides of length $L$.
    The potential energy is infinite at the sides and zero inside. 
    The single-particle energies are given by $\left(\dfrac{h^2}{8mL^2}\right)\left(n_x^2+n_y^2\right)$,
        where $n_x$ and $n_y$ are integers. 
    In units of $\left(\dfrac{h^2}{8mL^2}\right)$ the energy of the ground state of the system is:
    \begin{multicols}{3}
    \begin{choices}
        \wrongchoice{0}
        \wrongchoice{10}
      \correctchoice{19}
        \wrongchoice{24}
        \wrongchoice{48}
    \end{choices}
    \end{multicols}
\end{question}
}

\element{halliday-mc}{
\begin{question}{halliday-ch40-q45}
    Five electrons are in a two-dimensional square potential energy well with sides of length $L$.
    The potential energy is infinite at the sides and zero inside. 
    The single-particle energies are given by $\left(\dfrac{h^2}{8mL^2}\right)\left(n_x^2+n_y^2\right)$,
        where $n_x$ and $n_y$ are integers. 
    In units of $\left(\dfrac{h^2}{8mL^2}\right)$ the energy of the first excited state of the system is:
    \begin{multicols}{3}
    \begin{choices}
        \wrongchoice{$13$}
      \correctchoice{$22$}
        \wrongchoice{$24$}
        \wrongchoice{$25$}
        \wrongchoice{$27$}
    \end{choices}
    \end{multicols}
\end{question}
}

\element{halliday-mc}{
\begin{question}{halliday-ch40-q46}
    Five electrons are in a two-dimensional square potential energy well with sides of length $L$.
    The potential energy is infinite at the sides and zero inside. 
    The single-particle energies are given by $\left(\dfrac{h^2}{8mL^2}\right)\left(n_x^2+n_y^2\right)$,
        where $n_x$ and $n_y$ are integers. 
    The number of single-particle states with energy $5\left(\dfrac{h^2}{8mL^2}\right)$ is:
    \begin{multicols}{3}
    \begin{choices}
        \wrongchoice{$1$}
      \correctchoice{$2$}
        \wrongchoice{$3$}
        \wrongchoice{$4$}
        \wrongchoice{$5$}
    \end{choices}
    \end{multicols}
\end{question}
}

\element{halliday-mc}{
\begin{question}{halliday-ch40-q47}
    Six electrons are in a two-dimensional square potential energy well with sides of length $L$. 
    The potential energy is infinite at the sides and zero inside. 
    The single-particle energies are given by $\left(\dfrac{h^2}{8mL^2}\right)\left(n_x^2+n_y^2\right)$,
        where $n_x$ and $n_y$ are integers. 
    If a seventh electron is added to the system when it is in its ground state the least energy the additional electron can have is:
    \begin{multicols}{2}
    \begin{choices}
        \wrongchoice{$2\left(\dfrac{h^2}{8mL^2}\right)$}
        \wrongchoice{$5\left(\dfrac{h^2}{8mL^2}\right)$}
      \correctchoice{$10\left(\dfrac{h^2}{8mL^2}\right)$}
        \wrongchoice{$13\left(\dfrac{h^2}{8mL^2}\right)$}
        \wrongchoice{$18\left(\dfrac{h^2}{8mL^2}\right)$}
    \end{choices}
    \end{multicols}
\end{question}
}

\element{halliday-mc}{
\begin{question}{halliday-ch40-q48}
    When a lithium atom is made from a helium atom by adding a proton (and neutron) to the nucleus and an electron outside,
        the electron goes into an $n=2$, $l=0$ state rather than an $n=1$, $l=0$ state. 
    This is an indication that electrons:
    \begin{choices}
      \correctchoice{obey the Pauli exclusion principle}
        \wrongchoice{obey the minimum energy principle}
        \wrongchoice{undergo the Zeeman effect}
        \wrongchoice{are diffracted}
        \wrongchoice{and protons are interchangeable}
    \end{choices}
\end{question}
}

\element{halliday-mc}{
\begin{question}{halliday-ch40-q49}
    When a lithium atom in its ground state is made from a helium atom by adding a proton (and neutron) to the nucleus and an electron outside,
        the electron goes into an $n=2$, $l=0$ state rather than an $n=3$, $l=0$ state. 
    This is an indication that electrons:
    \begin{choices}
        \wrongchoice{obey the Pauli exclusion principle}
      \correctchoice{obey the minimum energy principle}
        \wrongchoice{undergo the Zeeman effect}
        \wrongchoice{are diffracted}
        \wrongchoice{and protons are interchangeable}
    \end{choices}
\end{question}
}

\element{halliday-mc}{
\begin{question}{halliday-ch40-q50}
    If electrons did not have intrinsic angular momentum (spin) but still obeyed the Pauli exclusion principle,
        the states occupied by electrons in the ground state of helium would be:
    \begin{choices}
        \wrongchoice{$\left(n=1, l=0\right); \left(n=1, l=0\right)$}
        \wrongchoice{$\left(n=1, l=0\right); \left(n=1, l=1\right)$}
      \correctchoice{$\left(n=1, l=0\right); \left(n=2, l=0\right)$}
        \wrongchoice{$\left(n=2, l=0\right); \left(n=2, l=1\right)$}
        \wrongchoice{$\left(n=2, l=1\right); \left(n=2, l=1\right)$}
    \end{choices}
\end{question}
}

\element{halliday-mc}{
\begin{question}{halliday-ch40-q51}
    The minimum energy principle tells us that:
    \begin{choices}
        \wrongchoice{the energy of an atom with a high atomic number is less than the energy of an atom with a low atomic number}
        \wrongchoice{the energy of an atom with a low atomic number is less than the energy of an atom with a high atomic number}
        \wrongchoice{when an atom makes an upward transition the energy of the absorbed photon is the least possible}
      \correctchoice{the ground state configuration of any atom is the one with the least energy}
        \wrongchoice{the ground state configuration of any atom is the one with the least ionization energy}
    \end{choices}
\end{question}
}

\element{halliday-mc}{
\begin{question}{halliday-ch40-q52}
    Which of the following ($n$, $l$, $m_l$, $m_s$) combinations is impossible for an electron in an atom?
    \begin{multicols}{2}
    \begin{choices}
        \wrongchoice{$3$, $1$, $1$, $-1/2$}
        \wrongchoice{$6$, $2$, $0$, $1/2$}
        \wrongchoice{$3$, $2$, $-2$, $-1/2$}
      \correctchoice{$3$, $1$, $-2$, $1/2$}
        \wrongchoice{$1$, $0$, $0$, $-1/2$}
    \end{choices}
    \end{multicols}
\end{question}
}

\element{halliday-mc}{
\begin{question}{halliday-ch40-q53}
    Which of the following subshells cannot exist?
    \begin{multicols}{3}
    \begin{choices}
        \wrongchoice{3p}
        \wrongchoice{2p}
        \wrongchoice{4d}
        \wrongchoice{3d}
      \correctchoice{2d}
    \end{choices}
    \end{multicols}
\end{question}
}

\element{halliday-mc}{
\begin{question}{halliday-ch40-q54}
    For any atom other than hydrogen and helium all electrons in the same shell have:
    \begin{choices}
        \wrongchoice{the same energy}
        \wrongchoice{the same magnitude of angular momentum}
        \wrongchoice{the same magnetic quantum number}
        \wrongchoice{the same spin quantum number}
      \correctchoice{none of the provided}
    \end{choices}
\end{question}
}

\element{halliday-mc}{
\begin{question}{halliday-ch40-q55}
    The states being filled from the beginning to end of the lanthanide series of atoms are:
    \begin{choices}
        \wrongchoice{$n=3$, $l=2$ states}
        \wrongchoice{$n=4$, $l=1$ states}
        \wrongchoice{$n=4$, $l=2$ states}
      \correctchoice{$n=4$, $l=3$ states}
        \wrongchoice{$n=5$, $l=2$ states}
    \end{choices}
\end{question}
}

\element{halliday-mc}{
\begin{question}{halliday-ch40-q56}
    The most energetic electron in any atom at the beginning of a period of the periodic table is in:
    \begin{choices}
      \correctchoice{an $l=0$ state}
        \wrongchoice{an $l=1$ state}
        \wrongchoice{an $l=2$ state}
        \wrongchoice{an $n=0$ state with unspecified angular momentum}
        \wrongchoice{an $n=1$ state with unspecified angular momentum}
    \end{choices}
\end{question}
}

\element{halliday-mc}{
\begin{question}{halliday-ch40-q57}
    The most energetic electron in any atom at the end of a period of the periodic table is in:
    \begin{choices}
        \wrongchoice{an $l=0$ state}
      \correctchoice{an $l=1$ state}
        \wrongchoice{an $l=2$ state}
        \wrongchoice{an $n=0$ state with unspecified angular momentum}
        \wrongchoice{an $n=1$ state with unspecified angular momentum}
    \end{choices}
\end{question}
}

\element{halliday-mc}{
\begin{question}{halliday-ch40-q58}
    The group of atoms at the ends of periods of the periodic table are called:
    \begin{choices}
        \wrongchoice{alkali metals}
        \wrongchoice{rare earths}
        \wrongchoice{transition metal atoms}
        \wrongchoice{alkaline atoms}
      \correctchoice{inert gas atoms}
    \end{choices}
\end{question}
}

\element{halliday-mc}{
\begin{question}{halliday-ch40-q59}
    The group of atoms at the beginning of periods of the periodic table are called:
    \begin{choices}
      \correctchoice{alkali metal atoms}
        \wrongchoice{rare earth atoms}
        \wrongchoice{transition metal atoms}
        \wrongchoice{alkaline atoms}
        \wrongchoice{inert gas atoms}
    \end{choices}
\end{question}
}

\element{halliday-mc}{
\begin{question}{halliday-ch40-q60}
    Suppose the energy required to ionize an argon atom is $i$,
        the energy to excite it is $e$,
        and its thermal energy at room temperature is $t$. 
    In increasing order, these three energies are:
    \begin{multicols}{3}
    \begin{choices}
        \wrongchoice{$i$, $e$, $t$}
        \wrongchoice{$t$, $i$, $e$}
      \correctchoice{$e$, $t$, $i$}
        \wrongchoice{$i$, $t$, $e$}
        \wrongchoice{$t$, $e$, $i$}
    \end{choices}
    \end{multicols}
\end{question}
}

\element{halliday-mc}{
\begin{question}{halliday-ch40-q61}
    The ionization energy of an atom in its ground state is:
    \begin{choices}
        \wrongchoice{the energy required to remove the least energetic electron}
      \correctchoice{the energy required to remove the most energetic electron}
        \wrongchoice{the energy difference between the most energetic electron and the least energetic electron}
        \wrongchoice{the same as the energy of a K$_{\alpha}$ photon}
        \wrongchoice{the same as the excitation energy of the most energetic electron}
    \end{choices}
\end{question}
}

\element{halliday-mc}{
\begin{question}{halliday-ch40-q62}
    The effective charge acting on a single valence electron outside a closed shell is about $N_e$,
        where $N$ is:
    \begin{choices}
        \wrongchoice{the atomic number of the nucleus}
        \wrongchoice{the atomic mass of the atom}
      \correctchoice{usually between 1 and 3}
        \wrongchoice{half the atomic number}
        \wrongchoice{less than 1}
    \end{choices}
\end{question}
}

\element{halliday-mc}{
\begin{question}{halliday-ch40-q63}
    In a laser:
    \begin{choices}
        \wrongchoice{excited atoms are stimulated to emit photons by radiation external to the laser}
        \wrongchoice{the transitions for laser emission are directly to the ground state}
        \wrongchoice{the states which give rise to laser emission are usually very unstable states that decay rapidly}
      \correctchoice{the state in which an atom is initially excited is never between two states that are involved in the stimulated emission}
        \wrongchoice{a minimum of two energy levels are required.}
    \end{choices}
\end{question}
}

\element{halliday-mc}{
\begin{question}{halliday-ch40-q64}
    Photons in a laser beam have the same energy,
        wavelength, polarization direction, and phase because:
    \begin{choices}
      \correctchoice{each is produced in an emission that is stimulated by another}
        \wrongchoice{all come from the same atom}
        \wrongchoice{the lasing material has only two quantum states}
        \wrongchoice{all photons are alike, no matter what their source}
        \wrongchoice{none of the provided}
    \end{choices}
\end{question}
}

\element{halliday-mc}{
\begin{question}{halliday-ch40-q65}
    A laser must be pumped to achieve:
    \begin{choices}
        \wrongchoice{a metastable state}
      \correctchoice{fast response}
        \wrongchoice{stimulated emission}
        \wrongchoice{population inversion}
        \wrongchoice{the same wavelength for all photons}
    \end{choices}
\end{question}
}

\element{halliday-mc}{
\begin{question}{halliday-ch40-q66}
    Photons in a laser beam are produced by:
    \begin{choices}
      \correctchoice{transitions from a metastable state}
        \wrongchoice{transitions to a metastable state}
        \wrongchoice{transitions from a state that decays rapidly}
        \wrongchoice{splitting of other photons}
        \wrongchoice{pumping}
    \end{choices}
\end{question}
}

\element{halliday-mc}{
\begin{question}{halliday-ch40-q67}
     Which of the following is essential for laser action to occur between two energy levels of an atom?
    \begin{choices}
        \wrongchoice{the lower level is metastable}
      \correctchoice{the upper level is metastable}
        \wrongchoice{the lower level is the ground state}
        \wrongchoice{there are more atoms in the lower level than in the upper level}
        \wrongchoice{the lasing material is a gas}
    \end{choices}
\end{question}
}

\element{halliday-mc}{
\begin{question}{halliday-ch40-q68}
    Which of the following is essential for laser action to occur between two energy levels of an atom?
    \begin{choices}
        \wrongchoice{the lower level is metastable}
      \correctchoice{there are more atoms the upper level than in the lower level}
        \wrongchoice{there are more atoms in the lower level than in the upper level}
        \wrongchoice{the lower level is the ground state}
        \wrongchoice{the lasing material is a gas}
    \end{choices}
\end{question}
}

\element{halliday-mc}{
\begin{question}{halliday-ch40-q69}
    Population inversion is important for the generation of a laser beam because it assures that:
    \begin{choices}
        \wrongchoice{spontaneous emission does not occur more often than stimulated emission}
        \wrongchoice{photons do not split too rapidly}
      \correctchoice{more photons are emitted than are absorbed}
        \wrongchoice{photons do not collide with each other}
        \wrongchoice{photons do not make upward transitions}
    \end{choices}
\end{question}
}

\element{halliday-mc}{
\begin{question}{halliday-ch40-q70}
    A metastable state is important for the generation of a laser beam because it assures that:
    \begin{choices}
      \correctchoice{spontaneous emission does not occur more often than stimulated emission}
        \wrongchoice{photons do not split too rapidly}
        \wrongchoice{more photons are emitted than are absorbed}
        \wrongchoice{photons do not collide with each other}
        \wrongchoice{photons do not make upward transitions}
    \end{choices}
\end{question}
}

\element{halliday-mc}{
\begin{question}{halliday-ch40-q71}
    Electrons in a certain laser make transitions from a metastable state to the ground state.
    Initially there are \num{6e20} atoms in the metastable state and \num{2e20} atoms in the ground state. 
    The number of photons that can be produced in a single burst is about:
    \begin{multicols}{3}
    \begin{choices}
        \wrongchoice{\num{2e20}}
        \wrongchoice{\num{3e20}}
      \correctchoice{\num{4e20}}
        \wrongchoice{\num{6e20}}
        \wrongchoice{\num{8e20}}
    \end{choices}
    \end{multicols}
\end{question}
}

\element{halliday-mc}{
\begin{question}{halliday-ch40-q72}
    In a helium-neon laser,
        the laser light arises from a transition from a \rule[-0.1pt]{4em}{0.1pt} state to a \rule[-0.1pt]{4em}{0.1pt} state.
    \begin{multicols}{3}
    \begin{choices}
        \wrongchoice{He, He}
      \correctchoice{Ne, Ne}
        \wrongchoice{He, Ne}
        \wrongchoice{Ne, He}
        \wrongchoice{N, He}
    \end{choices}
    \end{multicols}
\end{question}
}

\element{halliday-mc}{
\begin{question}{halliday-ch40-q73}
    The purpose of the mirrors at the ends of a helium-neon laser is:
    \begin{choices}
        \wrongchoice{to assure that no laser light leaks out}
      \correctchoice{to increase the number of stimulated emissions}
        \wrongchoice{to absorb some of the photons}
        \wrongchoice{to keep the light used for pumping inside the laser}
        \wrongchoice{to double the effective length of the laser}
    \end{choices}
\end{question}
}

\element{halliday-mc}{
\begin{question}{halliday-ch40-q74}
    A group of electromagnetic waves might
    \begin{itemize}
        \item[I.]   be monochromatic
        \item[II.]  be coherent
        \item[III.] have the same polarization direction
    \end{itemize}
    %% NOTE: questionmult??
    Which of these describe the waves from a laser?
    \begin{multicols}{2}
    \begin{choices}
        \wrongchoice{I only}
        \wrongchoice{II only}
        \wrongchoice{III only}
        \wrongchoice{I and II only}
      \correctchoice{I, II, and III}
    \end{choices}
    \end{multicols}
\end{question}
}

\element{halliday-mc}{
\begin{question}{halliday-ch40-q75}
    A laser beam can be sharply focused because it is:
    \begin{choices}
        \wrongchoice{highly coherent}
        \wrongchoice{plane polarized}
        \wrongchoice{intense}
        \wrongchoice{circularly polarized}
      \correctchoice{highly directional}
    \end{choices}
\end{question}
}

\element{halliday-mc}{
\begin{question}{halliday-ch40-q76}
    The ``e'' in laser stands for:
    \begin{multicols}{2}
    \begin{choices}
        \wrongchoice{electric}
        \wrongchoice{emf}
        \wrongchoice{energy}
      \correctchoice{emission}
        \wrongchoice{entropy}
    \end{choices}
    \end{multicols}
\end{question}
}


\endinput


