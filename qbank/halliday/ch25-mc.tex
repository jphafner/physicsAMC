
%%--------------------------------------------------
%% Halliday: Fundamentals of Physics
%%--------------------------------------------------


%% Chapter 25: Capacitance
%%--------------------------------------------------


%% Learning Objectives
%%--------------------------------------------------

%% 25.01: Sketch a schematic diagram of a circuit with a parallel-plate capacitor, a battery, and an open or closed switch.
%% 25.02: In a circuit with a battery, an open switch, and an uncharged capacitor, explain what happens to the conduction electrons when the switch is closed.
%% 25.03: For a capacitor, apply the relationship between the magnitude of charge $q$ on either plate (``the charge on the capacitor''), the potential difference $V$ between the plates (``the potential across the capacitor''), and the capacitance $C$ of the capacitor.


%% Halliday Multiple Choice Questions
%%--------------------------------------------------
\element{halliday-mc}{
\begin{question}{halliday-ch25-q01}
    The units of capacitance are equivalent to:
    \begin{choices}
        \wrongchoice{joule per coulomb (\si{\joule\per\coulomb})}
        \wrongchoice{volt per coulomb (\si{\volt\per\coulomb})}
        \wrongchoice{joule squared per coulomb (\si{\joule\squared\per\coulomb})}
        \wrongchoice{coulomb per joule (\si{\coulomb\per\joule})}
      \correctchoice{coulomb squared per joule (\si{\coulomb\squared\per\joule})}
    \end{choices}
\end{question}
}

\element{halliday-mc}{
\begin{question}{halliday-ch25-q02}
    A farad (\si{\farad}) is the same as a:
    \begin{choices}
        \wrongchoice{joule per volt (\si{\joule\per\volt})}
        \wrongchoice{volt per joule (\si{\volt\per\joule})}
      \correctchoice{coulomb per volt (\si{\coulomb\per\volt})}
        \wrongchoice{volt per coulomb (\si{\volt\per\coulomb})}
        \wrongchoice{newton per coulomb (\si{\newton\per\coulomb})}
    \end{choices}
\end{question}
}

\element{halliday-mc}{
\begin{question}{halliday-ch25-q03}
    A capacitor $C$ ``has a charge $Q$''.
    The actual charges on its plates are:
    \begin{multicols}{3}
    \begin{choices}
        \wrongchoice{$Q$, $Q$}
        \wrongchoice{$\dfrac{Q}{2}$, $\dfrac{Q}{2}$}
      \correctchoice{$Q$, $-Q$}
        \wrongchoice{$\dfrac{Q}{2}$, $\dfrac{-Q}{2}$}
        \wrongchoice{$Q$, zero}
    \end{choices}
    \end{multicols}
\end{question}
}

\element{halliday-mc}{
\begin{question}{halliday-ch25-q04}
    Each plate of a capacitor stores a charge of magnitude \SI{1}{\milli\coulomb} when a \SI{100}{\volt} potential difference is applied. 
    The capacitance is:
    \begin{multicols}{2}
    \begin{choices}
        \wrongchoice{\SI{5}{\micro\farad}}
      \correctchoice{\SI{10}{\micro\farad}}
        \wrongchoice{\SI{50}{\micro\farad}}
        \wrongchoice{\SI{100}{\micro\farad}}
        \wrongchoice{none of the provided}
    \end{choices}
    \end{multicols}
\end{question}
}

\element{halliday-mc}{
\begin{question}{halliday-ch25-q05}
    To charge a \SI{1}{\farad} capacitor with \SI{2}{\coulomb} requires a potential difference of:
    \begin{multicols}{2}
    \begin{choices}
      \correctchoice{\SI{2}{\volt}}
        \wrongchoice{\SI{0.2}{\volt}}
        \wrongchoice{\SI{5}{\volt}}
        \wrongchoice{\SI{0.5}{\volt}}
        \wrongchoice{none of the provided}
    \end{choices}
    \end{multicols}
\end{question}
}

\element{halliday-mc}{
\begin{question}{halliday-ch25-q06}
    The capacitance of a parallel-plate capacitor with plate area $A$ and plate separation $d$ is given by:
    \begin{multicols}{3}
    \begin{choices}
        \wrongchoice{$\dfrac{\epsilon_0 d}{A}$}
        \wrongchoice{$\dfrac{\epsilon_0 d}{2A}$}
      \correctchoice{$\dfrac{\epsilon_0 A}{d}$}
        \wrongchoice{$\dfrac{\epsilon_0 A}{2d}$}
        \wrongchoice{$\dfrac{Ad}{\epsilon_0}$}
    \end{choices}
    \end{multicols}
\end{question}
}

\element{halliday-mc}{
\begin{question}{halliday-ch25-q07}
    The capacitance of a parallel-plate capacitor is:
    \begin{choices}
      \correctchoice{proportional to the plate area}
        \wrongchoice{proportional to the charge stored}
        \wrongchoice{independent of any material inserted between the plates}
        \wrongchoice{proportional to the potential difference of the plates}
        \wrongchoice{proportional to the plate separation}
    \end{choices}
\end{question}
}

\element{halliday-mc}{
\begin{question}{halliday-ch25-q08}
    The plate areas and plate separations of five parallel plate capacitors are
    \begin{description}
        \item[capacitor 1:] area $A_0$, separation $d_0$
        \item[capacitor 2:] area $2A_0$, separation $2d_0$
        \item[capacitor 3:] area $2A_0$, separation $d_0/2$
        \item[capacitor 4:] area $A_0/2$, separation $2d_0$
        \item[capacitor 5:] area $A_0$, separation $d_0/2$
    \end{description}
    Rank these according to their capacitances, least to greatest.
    \begin{choices}
        \wrongchoice{1, 2, 3, 4, 5}
        \wrongchoice{5, 4, 3, 2, 1}
        \wrongchoice{5, 3 and 4 tie, then 1, 2}
      \correctchoice{4, 1 and 2 tie, then 5, 3}
        \wrongchoice{3, 5, 1 and 2 tie, 1, 4}
    \end{choices}
\end{question}
}

\element{halliday-mc}{
\begin{question}{halliday-ch25-q09}
    The capacitance of a parallel-plate capacitor can be increased by:
    \begin{choices}
        \wrongchoice{increasing the charge}
        \wrongchoice{decreasing the charge}
        \wrongchoice{increasing the plate separation}
      \correctchoice{decreasing the plate separation}
        \wrongchoice{decreasing the plate area}
    \end{choices}
\end{question}
}

\element{halliday-mc}{
\begin{question}{halliday-ch25-q10}
    If both the plate area and the plate separation of a parallel-plate capacitor are doubled,
        the capacitance is:
    \begin{choices}
        \wrongchoice{doubled}
        \wrongchoice{halved}
      \correctchoice{unchanged}
        \wrongchoice{tripled}
        \wrongchoice{quadrupled}
    \end{choices}
\end{question}
}

\element{halliday-mc}{
\begin{question}{halliday-ch25-q11}
    If the plate area of an isolated charged parallel-plate capacitor is doubled:
    \begin{choices}
        \wrongchoice{the electric field is doubled}
      \correctchoice{the potential difference is halved}
        \wrongchoice{the charge on each plate is halved}
        \wrongchoice{the surface charge density on each plate is doubled}
        \wrongchoice{none of the provided}
    \end{choices}
\end{question}
}

\element{halliday-mc}{
\begin{question}{halliday-ch25-q12}
    If the plate separation of an isolated charged parallel-plate capacitor is doubled:
    \begin{choices}
        \wrongchoice{the electric field is doubled}
        \wrongchoice{the potential difference is halved}
        \wrongchoice{the charge on each plate is halved}
        \wrongchoice{the surface charge density on each plate is doubled}
      \correctchoice{none of the provided}
    \end{choices}
\end{question}
}

\element{halliday-mc}{
\begin{question}{halliday-ch25-q13}
    Pulling the plates of an isolated charged capacitor apart:
    \begin{choices}
        \wrongchoice{increases the capacitance}
      \correctchoice{increases the potential difference}
        \wrongchoice{does not affect the potential difference}
        \wrongchoice{decreases the potential difference}
        \wrongchoice{does not affect the capacitance}
    \end{choices}
\end{question}
}

\element{halliday-mc}{
\begin{question}{halliday-ch25-q14}
    If the charge on a parallel-plate capacitor is doubled:
    \begin{choices}
        \wrongchoice{the capacitance is halved}
        \wrongchoice{the capacitance is doubled}
        \wrongchoice{the electric field is halved}
      \correctchoice{the electric field is doubled}
        \wrongchoice{the surface charge density is not changed on either plate}
    \end{choices}
\end{question}
}

\element{halliday-mc}{
\begin{question}{halliday-ch25-q15}
    A parallel-plate capacitor has a plate area of \SI{0.2}{\meter\squared} and a plate separation of \SI{0.1}{\milli\meter}.
    To obtain an electric field of \SI{2.0e6}{\volt\per\meter} between the plates,
        the magnitude of the charge on each plate should be:
    \begin{multicols}{2}
    \begin{choices}
        \wrongchoice{\SI{8.9e-7}{\coulomb}}
        \wrongchoice{\SI{1.8e-6}{\coulomb}}
        \wrongchoice{\SI{3.5e-6}{\coulomb}}
      \correctchoice{\SI{7.1e-6}{\coulomb}}
        \wrongchoice{\SI{1.4e-5}{\coulomb}}
    \end{choices}
    \end{multicols}
\end{question}
}

\element{halliday-mc}{
\begin{question}{halliday-ch25-q16}
    A parallel-plate capacitor has a plate area of \SI{0.2}{\meter\squared} and a plate separation of \SI{0.1}{\milli\meter}.
    If the charge on each plate has a magnitude of \SI{4e-6}{\coulomb} the potential difference across the plates is approximately:
    \begin{multicols}{2}
    \begin{choices}
        \wrongchoice{zero}
        \wrongchoice{\SI{4e-2}{\volt}}
        \wrongchoice{\SI{1e2}{\volt}}
      \correctchoice{\SI{2e2}{\volt}}
        \wrongchoice{\SI{4e8}{\volt}}
    \end{choices}
    \end{multicols}
\end{question}
}

\element{halliday-mc}{
\begin{question}{halliday-ch25-q17}
    The capacitance of a spherical capacitor with inner radius $a$ and outer radius $b$ is proportional to:
    \begin{multicols}{3}
    \begin{choices}
        \wrongchoice{$\dfrac{a}{b}$}
        \wrongchoice{$b-a$}
        \wrongchoice{$b^2-a^2$}
      \correctchoice{$\dfrac{ab}{b-a}$}
        \wrongchoice{$\dfrac{ab}{b^2-a^2}$}
    \end{choices}
    \end{multicols}
\end{question}
}

\element{halliday-mc}{
\begin{question}{halliday-ch25-q18}
    The capacitance of a single isolated spherical conductor with radius $R$ is proportional to:
    \begin{multicols}{2}
    \begin{choices}
      \correctchoice{$R$}
        \wrongchoice{$R^2$}
        \wrongchoice{$\dfrac{1}{R}$}
        \wrongchoice{$\dfrac{1}{R^2}$}
        \wrongchoice{none of the provided}
    \end{choices}
    \end{multicols}
\end{question}
}

\element{halliday-mc}{
\begin{question}{halliday-ch25-q19}
    Two conducting spheres have radii of $R_1$ and $R_2$,
        with $R_1$ greater than $R_2$. 
    If they are far apart the capacitance is proportional to:
    \begin{multicols}{2}
    \begin{choices}
      \correctchoice{$\dfrac{R_1 R_2}{R_1-R_2}$}
        \wrongchoice{$R_1^2-R_2^2$}
        \wrongchoice{$\dfrac{R_1-R_2}{R_1 R_2}$}
        \wrongchoice{$R_1^2+R_2^2$}
        \wrongchoice{none of the provided}
    \end{choices}
    \end{multicols}
\end{question}
}

\element{halliday-mc}{
\begin{question}{halliday-ch25-q20}
    The capacitance of a cylindrical capacitor can be increased by:
    \begin{choices}
        \wrongchoice{decreasing both the radius of the inner cylinder and the length}
      \correctchoice{increasing both the radius of the inner cylinder and the length}
        \wrongchoice{increasing the radius of the outer cylindrical shell and decreasing the length}
        \wrongchoice{decreasing the radius of the inner cylinder and increasing the radius of the outer cylindrical shell}
        \wrongchoice{only by decreasing the length}
    \end{choices}
\end{question}
}

\element{halliday-mc}{
\begin{question}{halliday-ch25-q21}
    A battery is used to charge a series combination of two identical capacitors. 
    If the potential difference across the battery terminals is $V$ and total charge $Q$ flows through the battery during the charging process then the charge on the positive plate of each capacitor and the potential difference across each capacitor are:
    \begin{choices}
        \wrongchoice{$Q/2$ and $V/2$, respectively}
        \wrongchoice{$Q$ and $V$ , respectively}
        \wrongchoice{$Q/2$ and $V$ , respectively}
      \correctchoice{$Q$ and $V/2$, respectively}
        \wrongchoice{$Q$ and $2V$, respectively}
    \end{choices}
\end{question}
}

\element{halliday-mc}{
\begin{question}{halliday-ch25-q22}
    A battery is used to charge a parallel combination of two identical capacitors. 
    If the potential difference across the battery terminals is $V$ and total charge $Q$ flows through the battery during the charging process then the charge on the positive plate of each capacitor and the potential difference across each capacitor are:
    \begin{choices}
        \wrongchoice{$Q/2$ and $V/2$, respectively}
        \wrongchoice{$Q$ and $V$ , respectively}
      \correctchoice{$Q/2$ and $V$ , respectively}
        \wrongchoice{$Q$ and $V/2$, respectively}
        \wrongchoice{$Q$ and $2V$, respectively}
    \end{choices}
\end{question}
}

\element{halliday-mc}{
\begin{question}{halliday-ch25-q23}
    A \SI{2}{\micro\farad} and a \SI{1}{\micro\farad} capacitor are connected in series and a potential difference is applied across the combination. 
    The \SI{2}{\micro\farad} capacitor has:
    \begin{choices}
        \wrongchoice{twice the charge of the \SI{1}{\micro\farad} capacitor}
        \wrongchoice{half the charge of the \SI{1}{\micro\farad} capacitor}
        \wrongchoice{twice the potential difference of the \SI{1}{\micro\farad} capacitor}
      \correctchoice{half the potential difference of the \SI{1}{\micro\farad} capacitor}
        \wrongchoice{none of the provided}
    \end{choices}
\end{question}
}

\element{halliday-mc}{
\begin{question}{halliday-ch25-q24}
    A \SI{2}{\micro\farad} and a \SI{1}{\micro\farad} capacitor are connected in parallel and a potential difference is applied across the combination. 
    The \SI{2}{\micro\farad} capacitor has:
    \begin{choices}
      \correctchoice{twice the charge of the \SI{1}{\micro\farad} capacitor}
        \wrongchoice{half the charge of the \SI{1}{\micro\farad} capacitor}
        \wrongchoice{twice the potential difference of the \SI{1}{\micro\farad} capacitor}
        \wrongchoice{half the potential difference of the \SI{1}{\micro\farad} capacitor}
        \wrongchoice{none of the provided}
    \end{choices}
\end{question}
}

\element{halliday-mc}{
\begin{question}{halliday-ch25-q25}
    Let $Q$ denote charge, $V$ denote potential difference, and $U$ denote stored energy. 
    Of these quantities, capacitors in series must have the same:
    \begin{multicols}{2}
    \begin{choices}
      \correctchoice{$Q$ only}
        \wrongchoice{$V$ only}
        \wrongchoice{$U$ only}
        \wrongchoice{$Q$ and $U$ only}
        \wrongchoice{$V$ and $U$ only}
    \end{choices}
    \end{multicols}
\end{question}
}

\element{halliday-mc}{
\begin{question}{halliday-ch25-q26}
    Let $Q$ denote charge, $V$ denote potential difference, and $U$ denote stored energy. 
    Of these quantities, capacitors in parallel must have the same:
    \begin{multicols}{2}
    \begin{choices}
        \wrongchoice{$Q$ only}
      \correctchoice{$V$ only}
        \wrongchoice{$U$ only}
        \wrongchoice{$Q$ and $U$ only}
        \wrongchoice{$V$ and $U$ only}
    \end{choices}
    \end{multicols}
\end{question}
}

\element{halliday-mc}{
\begin{question}{halliday-ch25-q27}
    Capacitors $C_1$ and $C_2$ are connected in parallel. 
    The equivalent capacitance is given by:
    \begin{multicols}{3}
    \begin{choices}
        \wrongchoice{$\dfrac{C_1 C_2}{(C_1+C_2}$}
        \wrongchoice{$\dfrac{(C_1+C_2}{C_1 C_2}$}
        \wrongchoice{$\dfrac{1}{C_1+C_2}$}
        \wrongchoice{$\dfrac{C_1}{C_2}$}
      \correctchoice{$C_1+C_2$}
    \end{choices}
    \end{multicols}
\end{question}
}

\element{halliday-mc}{
\begin{question}{halliday-ch25-q28}
    Capacitors $C_1$ and $C_2$ are connected in series. 
    The equivalent capacitance is given by:
    \begin{multicols}{3}
    \begin{choices}
      \correctchoice{$\dfrac{C_1 C_2}{(C_1+C_2}$}
        \wrongchoice{$\dfrac{(C_1+C_2}{C_1 C_2}$}
        \wrongchoice{$\dfrac{1}{C_1+C_2}$}
        \wrongchoice{$\dfrac{C_1}{C_2}$}
        \wrongchoice{$C_1+C_2$}
    \end{choices}
    \end{multicols}
\end{question}
}

\element{halliday-mc}{
\begin{question}{halliday-ch25-q29}
    Capacitors $C_1$ and $C_2$ are connected in series and a potential difference is applied to the combination. 
    If the capacitor that is equivalent to the combination has the same potential difference,
        then the charge on the equivalent capacitor is the same as:
    \begin{choices}
      \correctchoice{the charge on $C_1$}
        \wrongchoice{the sum of the charges on $C_1$ and $C_2$}
        \wrongchoice{the difference of the charges on $C_1$ and $C_2$}
        \wrongchoice{the product of the charges on $C_1$ and $C_2$}
        \wrongchoice{none of the provided}
    \end{choices}
\end{question}
}

\element{halliday-mc}{
\begin{question}{halliday-ch25-q30}
    Capacitors $C_1$ and $C_2$ are connected in parallel and a potential difference is applied to the combination. 
    If the capacitor that is equivalent to the combination has the same potential difference,
        then the charge on the equivalent capacitor is the same as:
    \begin{choices}
        \wrongchoice{the charge on $C_1$}
      \correctchoice{the sum of the charges on $C_1$ and $C_2$}
        \wrongchoice{the difference of the charges on $C_1$ and $C_2$}
        \wrongchoice{the product of the charges on $C_1$ and $C_2$}
        \wrongchoice{none of the provided}
    \end{choices}
\end{question}
}

\element{halliday-mc}{
\begin{question}{halliday-ch25-q31}
    Two identical capacitors are connected in series and two,
        each identical to the first, are connected in parallel. 
    The equivalent capacitance of the series connection is \rule[-0.1pt]{4em}{0.1pt} the equivalent capacitance of parallel connection.
    \begin{multicols}{2}
    \begin{choices}
        \wrongchoice{twice}
        \wrongchoice{four times}
        \wrongchoice{half}
      \correctchoice{one-fourth}
        \wrongchoice{the same as}
    \end{choices}
    \end{multicols}
\end{question}
}

\element{halliday-mc}{
\begin{question}{halliday-ch25-q32}
    Two identical capacitors, each with capacitance $C$,
        are connected in parallel and the combination is connected in series to a third identical capacitor. 
    The equivalent capacitance of this arrangement is:
    \begin{multicols}{3}
    \begin{choices}
      \correctchoice{$\dfrac{2C}{3}$}
        \wrongchoice{$C$}
        \wrongchoice{$\dfrac{3C}{2}$}
        \wrongchoice{$2C$}
        \wrongchoice{$3C$}
    \end{choices}
    \end{multicols}
\end{question}
}

\element{halliday-mc}{
\begin{question}{halliday-ch25-q33}
    A \SI{2}{\micro\farad} and a \SI{1}{\micro\farad} capacitor are connected in series and charged from a battery. 
    They store charges $P$ and $Q$, respectively. 
    When disconnected and charged separately using the same battery,
        they have charges $R$ and $S$, respectively. 
    Then:
    \begin{multicols}{2}
    \begin{choices}
      \correctchoice{$R > S > Q = P$}
        \wrongchoice{$P > Q > R = S$}
        \wrongchoice{$R > P = Q > S$}
        \wrongchoice{$R = P > S = Q$}
        \wrongchoice{$R > P > S = Q$}
    \end{choices}
    \end{multicols}
\end{question}
}

\element{halliday-mc}{
\begin{question}{halliday-ch25-q34}
    Capacitor $C_1$ is connected alone to a battery and charged until the magnitude of the charge on each plate is \SI{4.0e-8}{\coulomb}.
    Then it is removed from the battery and connected to two other capacitors $C_2$ and $C_3$,
        as shown. 
    \begin{center}
    \begin{tikzpicture}
        %% NOTE: circuitikz
    \end{tikzpicture}
    \end{center}
    The charge on the positive plate of $C_1$ is then \SI{1.0e-8}{\coulomb}.
    The charges on the positive plates of $C_2$ and $C_3$ are:
    \begin{choices}
      \correctchoice{$q_2=\SI{3.0e-8}{\coulomb}$ and $q_3=\SI{3.0e-8}{\coulomb}$}
        \wrongchoice{$q_2=\SI{2.0e-8}{\coulomb}$ and $q_3=\SI{2.0e-8}{\coulomb}$}
        \wrongchoice{$q_2=\SI{5.0e-8}{\coulomb}$ and $q_3=\SI{1.0e-8}{\coulomb}$}
        \wrongchoice{$q_2=\SI{3.0e-8}{\coulomb}$ and $q_3=\SI{1.0e-8}{\coulomb}$}
        \wrongchoice{$q_2=\SI{1.0e-8}{\coulomb}$ and $q_3=\SI{3.0e-8}{\coulomb}$}
    \end{choices}
\end{question}
}

\element{halliday-mc}{
\begin{question}{halliday-ch25-q35}
    Each of the four capacitors shown is \SI{500}{\micro\farad}.
    The voltmeter reads \SI{1000}{\volt}.
    \begin{center}
    \begin{tikzpicture}
        %% NOTE: circuitikz
    \end{tikzpicture}
    \end{center}
    The magnitude of the charge on each capacitor plate is:
    \begin{multicols}{2}
    \begin{choices}
        \wrongchoice{\SI{0.2}{\coulomb}}
      \correctchoice{\SI{0.5}{\coulomb}}
        \wrongchoice{\SI{20}{\coulomb}}
        \wrongchoice{\SI{50}{\coulomb}}
        \wrongchoice{none of the provided}
    \end{choices}
    \end{multicols}
\end{question}
}

\element{halliday-mc}{
\begin{question}{halliday-ch25-q36}
    The diagram shows four \SI{6}{\micro\farad} capacitors. 
    \begin{center}
    \begin{tikzpicture}
        %% NOTE: circuitikz
    \end{tikzpicture}
    \end{center}
    The capacitance between points $a$ and $b$ is:
    \begin{multicols}{3}
    \begin{choices}
        \wrongchoice{\SI{3}{\micro\farad}}
        \wrongchoice{\SI{4}{\micro\farad}}
      \correctchoice{\SI{6}{\micro\farad}}
        \wrongchoice{\SI{9}{\micro\farad}}
        \wrongchoice{\SI{1}{\micro\farad}}
    \end{choices}
    \end{multicols}
\end{question}
}

\element{halliday-mc}{
\begin{question}{halliday-ch25-q37}
    Each of the two \SI{25}{\micro\farad} capacitors shown is initially uncharged. 
    \begin{center}
    \begin{tikzpicture}
        %% NOTE: circuitikz
    \end{tikzpicture}
    \end{center}
    How many coulombs of charge pass through the ammeter $A$ after the switch $S$ is closed?
    \begin{multicols}{2}
    \begin{choices}
        \wrongchoice{\SI{0.10}{\coulomb}}
      \correctchoice{\SI{0.20}{\coulomb}}
        \wrongchoice{\SI{10}{\coulomb}}
        \wrongchoice{\SI{0.05}{\coulomb}}
        \wrongchoice{none of the provided}
    \end{choices}
    \end{multicols}
\end{question}
}

\element{halliday-mc}{
\begin{question}{halliday-ch25-q38}
    A \SI{20}{\farad} capacitor is charged to \SI{200}{\volt}.
    Its stored energy is:
    \begin{multicols}{3}
    \begin{choices}
        \wrongchoice{\SI{4000}{\joule}}
        \wrongchoice{\SI{4}{\joule}}
      \correctchoice{\SI{0.4}{\joule}}
        \wrongchoice{\SI{2000}{\joule}}
        \wrongchoice{\SI{0.1}{\joule}}
    \end{choices}
    \end{multicols}
\end{question}
}

\element{halliday-mc}{
\begin{question}{halliday-ch25-q39}
    A charged capacitor stores \SI{10}{\coulomb} at \SI{40}{\volt}.
    Its stored energy is:
    \begin{multicols}{3}
    \begin{choices}
        \wrongchoice{\SI{400}{\joule}}
        \wrongchoice{\SI{4}{\joule}}
        \wrongchoice{\SI{0.2}{\joule}}
        \wrongchoice{\SI{2.5}{\joule}}
      \correctchoice{\SI{200}{\joule}}
    \end{choices}
    \end{multicols}
\end{question}
}

\element{halliday-mc}{
\begin{question}{halliday-ch25-q40}
    A \SI{2}{\micro\farad} and a \SI{1}{\micro\farad} capacitor are connected in series and charged by a battery. 
    They store energies $P$ and $Q$, respectively. 
    When disconnected and charged separately using the same battery,
        they store energies $R$ and $S$, respectively. Then:
    \begin{multicols}{2}
    \begin{choices}
        \wrongchoice{$R > P > S > Q$}
        \wrongchoice{$P > Q > R > S$}
        \wrongchoice{$R > P > Q > S$}
        \wrongchoice{$P > R > S > Q$}
      \correctchoice{$R > S > Q > P$}
    \end{choices}
    \end{multicols}
\end{question}
}

\element{halliday-mc}{
\begin{question}{halliday-ch25-q41}
    The quantity $\dfrac{1}{2}\epsilon_0 E^2$ has the significance of:
    \begin{choices}
        \wrongchoice{energy divided by capacitance}
        \wrongchoice{energy divided by charge}
        \wrongchoice{energy}
      \correctchoice{energy divided by volume}
        \wrongchoice{energy dividied by electric potential}
    \end{choices}
\end{question}
}

\element{halliday-mc}{
\begin{question}{halliday-ch25-q42}
    Capacitors $A$ and $B$ are identical. 
    Capacitor $A$ is charged so it stores \SI{4}{\joule} of energy and capacitor $B$ is uncharged. 
    The capacitors are then connected in parallel. 
    The total stored energy in the capacitors is now:
    \begin{multicols}{3}
    \begin{choices}
        \wrongchoice{\SI{16}{\joule}}
        \wrongchoice{\SI{8}{\joule}}
        \wrongchoice{\SI{4}{\joule}}
      \correctchoice{\SI{2}{\joule}}
        \wrongchoice{\SI{1}{\joule}}
    \end{choices}
    \end{multicols}
\end{question}
}

\element{halliday-mc}{
\begin{question}{halliday-ch25-q43}
    To store a total of \SI{0.040}{\joule} of energy in the two identical capacitors shown,
    \begin{center}
    \begin{tikzpicture}
        %% NOTE: circuitikz
    \end{tikzpicture}
    \end{center}
        each should have a capacitance of:
    \begin{multicols}{3}
    \begin{choices}
        %% NOTE: double check this
        \wrongchoice{\SI{0.10}{\micro\farad}}
        \wrongchoice{\SI{0.50}{\micro\farad}}
      \correctchoice{\SI{1.0}{\micro\farad}}
        \wrongchoice{\SI{1.5}{\micro\farad}}
        \wrongchoice{\SI{2.0}{\micro\farad}}
    \end{choices}
    \end{multicols}
\end{question}
}

\element{halliday-mc}{
\begin{question}{halliday-ch25-q44}
    A battery is used to charge a parallel-plate capacitor,
        after which it is disconnected. 
    Then the plates are pulled apart to twice their original separation. 
    This process will double the:
    \begin{choices}
        \wrongchoice{capacitance}
        \wrongchoice{surface charge density on each plate}
      \correctchoice{stored energy}
        \wrongchoice{electric field between the two places}
        \wrongchoice{charge on each plate}
    \end{choices}
\end{question}
}

\element{halliday-mc}{
\begin{question}{halliday-ch25-q45}
    A parallel-plate capacitor has a plate area of \SI{0.3}{\meter\squared} and a plate separation of \SI{0.1}{\milli\meter}.
    If the charge on each plate has a magnitude of \SI{5e-6}{\coulomb} then the force exerted by one plate on the other has a magnitude of about:
    \begin{multicols}{2}
    \begin{choices}
        \wrongchoice{zero}
      \correctchoice{\SI{5}{\newton}}
        \wrongchoice{\SI{9}{\newton}}
        \wrongchoice{\SI{1e4}{\newton}}
        \wrongchoice{\SI{9e5}{\newton}}
    \end{choices}
    \end{multicols}
\end{question}
}

\element{halliday-mc}{
\begin{question}{halliday-ch25-q46}
    A certain capacitor has a capacitance of \SI{5.0}{\micro\farad}.
    After it is charged to \SI{5.0}{\micro\coulomb} and isolated,
        the plates are brought closer together so its capacitance becomes \SI{10}{\micro\farad}.
    The work done by the agent is about:
    \begin{multicols}{2}
    \begin{choices}
        \wrongchoice{zero}
        \wrongchoice{\SI{1.25e-6}{\joule}}
      \correctchoice{\SI{-1.25e-6}{\joule}}
        \wrongchoice{\SI{8.3e-7}{\joule}}
        \wrongchoice{\SI{-8.3e-7}{\joule}}
    \end{choices}
    \end{multicols}
\end{question}
}

\element{halliday-mc}{
\begin{question}{halliday-ch25-q47}
    A dielectric slab is slowly inserted between the plates of a parallel plate capacitor,
        while the potential difference between the plates is held constant by a battery. 
    As it is being inserted:
    \begin{choices}
        \wrongchoice{the capacitance, the potential difference between the plates, and the charge on the positive plate all increase}
        \wrongchoice{the capacitance, the potential difference between the plates, and the charge on the positive plate all decrease}
        \wrongchoice{the potential difference between the plates increases, the charge on the positive plate decreases, and the capacitance remains the same}
        \wrongchoice{the capacitance and the charge on the positive plate decrease but the potential difference between the plates remains the same}
      \correctchoice{the capacitance and the charge on the positive plate increase but the potential difference between the plates remains the same}
    \end{choices}
\end{question}
}

\element{halliday-mc}{
\begin{question}{halliday-ch25-q48}
    An air-filled parallel-plate capacitor has a capacitance of \SI{1}{\pico\farad}.
    The plate separation is then doubled and a wax dielectric is inserted,
        completely filling the space between the plates. 
    As a result, the capacitance becomes \SI{2}{\pico\farad}.
    The dielectric constant of the wax is:
    \begin{multicols}{3}
    \begin{choices}
        %% NOTE: check units
        \wrongchoice{\num{0.25}}
        \wrongchoice{\num{0.5}}
        \wrongchoice{\num{2.0}}
      \correctchoice{\num{4.0}}
        \wrongchoice{\num{8.0}}
    \end{choices}
    \end{multicols}
\end{question}
}

\element{halliday-mc}{
\begin{question}{halliday-ch25-q49}
    One of materials listed below is to be placed between two identical metal sheets,
        with no, air gap, to form a parallel-plate capacitor. 
    Which produces the greatest capacitance?
    \begin{choices}
        \wrongchoice{material of thickness \SI{0.1}{\milli\meter} and dielectric constant 2}
        \wrongchoice{material of thickness \SI{0.2}{\milli\meter} and dielectric constant 3}
        \wrongchoice{material of thickness \SI{0.3}{\milli\meter} and dielectric constant 2}
        \wrongchoice{material of thickness \SI{0.4}{\milli\meter} and dielectric constant 8}
      \correctchoice{material of thickness \SI{0.5}{\milli\meter} and dielectric constant 11}
    \end{choices}
\end{question}
}

\element{halliday-mc}{
\begin{question}{halliday-ch25-q50}
    Two capacitors are identical except that one is filled with air and the other with oil. 
    Both capacitors carry the same charge. 
    The ratio of the electric fields $E_{\text{air}}/E_{\text{oil}}$ is:
    \begin{choices}
        \wrongchoice{between 0 and 1}
        \wrongchoice{0}
        \wrongchoice{1}
      \correctchoice{between 1 and infinity}
        \wrongchoice{infinite}
    \end{choices}
\end{question}
}

\element{halliday-mc}{
\begin{question}{halliday-ch25-q51}
    A parallel-plate capacitor, with air dielectric, is charged by a battery,
        after which the battery is disconnected. 
    A slab of glass dielectric is then slowly inserted between the plates. 
    As it is being inserted:
    \begin{choices}
        \wrongchoice{a force repels the glass out of the capacitor}
      \correctchoice{a force attracts the glass into the capacitor}
        \wrongchoice{no force acts on the glass}
        \wrongchoice{a net charge appears on the glass}
        \wrongchoice{the glass makes the plates repel each other}
    \end{choices}
\end{question}
}

\element{halliday-mc}{
\begin{question}{halliday-ch25-q52}
    Two parallel-plate capacitors with the same plate separation but different capacitance are connected in parallel to a battery. 
    Both capacitors are filled with air. 
    The quantity that is \emph{not} the same for both capacitors when they are fully charged is:
    \begin{choices}
        \wrongchoice{potential difference}
        \wrongchoice{energy density}
        \wrongchoice{electric field between the plates}
      \correctchoice{charge on the positive plate}
        \wrongchoice{dielectric constant}
    \end{choices}
\end{question}
}

\element{halliday-mc}{
\begin{question}{halliday-ch25-q53}
    Two parallel-plate capacitors with the same plate area but different capacitance are connected in parallel to a battery.
    Both capacitors are filled with air. 
    The quantity that is the same for both capacitors when they are fully charged is:
    \begin{choices}
      \correctchoice{potential difference}
        \wrongchoice{energy density}
        \wrongchoice{electric field between the plates}
        \wrongchoice{charge on the positive plate}
        \wrongchoice{plate separation}
    \end{choices}
\end{question}
}

\element{halliday-mc}{
\begin{question}{halliday-ch25-q54}
    Two parallel-plate capacitors with different plate separation but the same capacitance are connected in series to a battery. 
    Both capacitors are filled with air. 
    The quantity that is \emph{not} the same for both capacitors when they are fully charged is:
    \begin{choices}
        \wrongchoice{potential difference}
        \wrongchoice{stored energy}
      \correctchoice{electric field between the plates}
        \wrongchoice{charge on the positive plate}
        \wrongchoice{dielectric constant}
    \end{choices}
\end{question}
}

\element{halliday-mc}{
\begin{question}{halliday-ch25-q55}
    Two parallel-plate capacitors with different capacitance but the same plate separation are connected in series to a battery. 
    Both capacitors are filled with air. 
    The quantity that is the same for both capacitors when they are fully charged is:
    \begin{choices}
        \wrongchoice{potential difference}
        \wrongchoice{stored energy}
        \wrongchoice{energy density}
        \wrongchoice{electric field between the plates}
      \correctchoice{charge on the positive plate}
    \end{choices}
\end{question}
}


\endinput


