
%%--------------------------------------------------
%% Halliday: Fundamentals of Physics
%%--------------------------------------------------


%% Chapter 22: Electric Fields
%%--------------------------------------------------


%% Learning Objectives
%%--------------------------------------------------

%% 22.01: Identify that at every point in the space surrounding a charged particle, the particle sets up an electric field $\vec{E}$, which is a vector quantity and thus has both magnitude and direction.
%% 22.02: Identify how an electric field $\vec{E}$ can be used to explain how a charged particle can exert an electrostatic force $\vec{F}$ on a second charged particle even though there is no contact between the particles.
%% 22.03: Explain how a small positive test charge is used (in principle) to measure the electric field at any given point.
%% 22.04: Explain electric field lines, including where they originate and terminate and what their spacing represents.


%% Halliday Multiple Choice Questions
%%--------------------------------------------------
\element{halliday-mc}{
\begin{question}{halliday-ch22-q01}
    An electric field is most directly related to:
    \begin{choices}
        \wrongchoice{the momentum of a test charge}
        \wrongchoice{the kinetic energy of a test charge}
        \wrongchoice{the potential energy of a test charge}
      \correctchoice{the force acting on a test charge}
        \wrongchoice{the charge carried by a test charge}
    \end{choices}
\end{question}
}

\element{halliday-mc}{
\begin{question}{halliday-ch22-q02}
    As used in the definition of electric field, a ``test charge'':
    \begin{choices}
        \wrongchoice{has zero charge}
        \wrongchoice{has charge of magnitude \SI{1}{\coulomb}}
        \wrongchoice{has charge of magnitude \SI{1.6e-19}{\coulomb}}
        \wrongchoice{must be an electron}
      \correctchoice{none of the provided}
    \end{choices}
\end{question}
}

\element{halliday-mc}{
\begin{question}{halliday-ch22-q03}
    Experimenter $A$ uses a test charge $q_0$ and experimenter $B$ uses a test charge $-2q_0$ to measure an electric field produced by stationary charges. 
    A finds a field that is:
    \begin{choices}
      \correctchoice{the same in both magnitude and direction as the field found by $B$}
        \wrongchoice{greater in magnitude than the field found by $B$}
        \wrongchoice{less in magnitude than the field found by $B$}
        \wrongchoice{opposite in direction to the field found by $B$}
        \wrongchoice{either greater or less than the field found by $B$, depending on the accelerations of the test charges}
    \end{choices}
\end{question}
}

\element{halliday-mc}{
\begin{question}{halliday-ch22-q04}
    The units of the electric field are:
    \begin{choices}
        \wrongchoice{newton coulomb squared (\si{\newton\coulomb\squared})}
        \wrongchoice{coulomb per newton (\si{\coulomb\per\newton})}
        \wrongchoice{newton (\si{\newton})}
      \correctchoice{newton per coulomb squared (\si{\newton\per\coulomb\squared})}
        \wrongchoice{coulomb per meter (\si{\coulomb\per\meter})}
    \end{choices}
\end{question}
}

\element{halliday-mc}{
\begin{question}{halliday-ch22-q05}
    The units of the electric field are:
    \begin{choices}
      \correctchoice{joule per coulomb per meter (\si{\joule\per\coulomb\per\meter})}
        \wrongchoice{joule per coulomb (\si{\joule\per\coulomb})}
        \wrongchoice{joule coulomb (\si{\joule\coulomb})}
        \wrongchoice{joule per meter (\si{\joule\per\meter})}
        \wrongchoice{none of the provided}
    \end{choices}
\end{question}
}

\element{halliday-mc}{
\begin{question}{halliday-ch22-q06}
    Electric field lines:
    \begin{choices}
        \wrongchoice{are trajectories of a test charge}
        \wrongchoice{are vectors in the direction of the electric field}
        \wrongchoice{form closed loops}
        \wrongchoice{cross each other in the region between two point charges}
      \correctchoice{are none of the provided}
    \end{choices}
\end{question}
}

\element{halliday-mc}{
\begin{question}{halliday-ch22-q07}
    Two thin spherical shells, one with radius $R$ and the other with radius $2R$,
        surround an isolated charged point particle. 
    The ratio of the number of field lines through the larger sphere to the number through the smaller is:
    \begin{multicols}{3}
    \begin{choices}
      \correctchoice{\num{1}}
        \wrongchoice{\num{2}}
        \wrongchoice{\num{4}}
        \wrongchoice{\num{1/2}}
        \wrongchoice{\num{1/4}}
    \end{choices}
    \end{multicols}
\end{question}
}

\element{halliday-mc}{
\begin{question}{halliday-ch22-q08}
    A certain physics textbook shows a region of space in which two electric field lines cross each other.
    We conclude that:
    \begin{choices}
        \wrongchoice{at least two point charges are present}
        \wrongchoice{an electrical conductor is present}
        \wrongchoice{an insulator is present}
        \wrongchoice{the field points in two directions at the same place}
      \correctchoice{the author made a mistake}
    \end{choices}
\end{question}
}

\element{halliday-mc}{
\begin{question}{halliday-ch22-q09}
    Choose the correct statement concerning electric field lines:
    \begin{choices}
        \wrongchoice{field lines may cross}
      \correctchoice{field lines are close together where the field is large}
        \wrongchoice{field lines point away from a negatively charged particle}
        \wrongchoice{a charged point particle released from rest moves along a field line}
        \wrongchoice{none of these are correct}
    \end{choices}
\end{question}
}

\element{halliday-mc}{
\begin{question}{halliday-ch22-q10}
    The diagram shows the electric field lines due to two charged parallel metal plates. 
    \begin{center}
    \begin{tikzpicture}
        %% NOTE:
    \end{tikzpicture}
    \end{center}
    We conclude that:
    \begin{choices}
        \wrongchoice{the upper plate is positive and the lower plate is negative}
      \correctchoice{a proton at $X$ would experience the same force if it were placed at $Y$}
        \wrongchoice{a proton at $X$ experiences a greater force than if it were placed at $Z$}
        \wrongchoice{a proton at $X$ experiences less force than if it were placed at $Z$}
        \wrongchoice{an electron at $X$ could have its weight balanced by the electrical force}
    \end{choices}
\end{question}
}

\element{halliday-mc}{
\begin{question}{halliday-ch22-q11}
    The diagram shows the electric field lines due to two charged parallel metal plates. 
    \begin{center}
    \begin{tikzpicture}
        %% NOTE:
    \end{tikzpicture}
    \end{center}
    Let $k$ denote $\dfrac{1}{4\pi\epsilon_0}$.
    The magnitude of the electric field at a distance $r$ from an isolated point particle with charge $q$ is:
    \begin{multicols}{3}
    \begin{choices}
        \wrongchoice{$\dfrac{kq}{r}$}
        \wrongchoice{$\dfrac{kr}{q}$}
        \wrongchoice{$\dfrac{kq}{r^3}$}
      \correctchoice{$\dfrac{kq}{r^2}$}
        \wrongchoice{$\dfrac{kq^2}{r^2}$}
    \end{choices}
    \end{multicols}
\end{question}
}

\element{halliday-mc}{
\begin{question}{halliday-ch22-q12}
    The diagram shows the electric field lines in a region of space containing two small charged spheres ($Y$ and $Z$).
    \begin{center}
    \begin{tikzpicture}
        %% NOTE:
    \end{tikzpicture}
    \end{center}
    Then:
    \begin{choices}
        \wrongchoice{$Y$ is negative and $Z$ is positive}
        \wrongchoice{the magnitude of the electric field is the same everywhere}
        \wrongchoice{the electric field is strongest midway between $Y$ and $Z$}
      \correctchoice{the electric field is not zero anywhere (except infinitely far from the spheres)}
        \wrongchoice{$Y$ and $Z$ must have the same sign}
    \end{choices}
\end{question}
}

\element{halliday-mc}{
\begin{question}{halliday-ch22-q13}
    The diagram shows the electric field lines in a region of space containing two small charged spheres ($Y$ and $Z$).
    \begin{center}
    \begin{tikzpicture}
        %% NOTE:
    \end{tikzpicture}
    \end{center}
    Then:
    \begin{choices}
        \wrongchoice{$Y$ is negative and $Z$ is positive}
        \wrongchoice{the magnitude of the electric field is the same everywhere}
        \wrongchoice{the electric field is strongest midway between $Y$ and $Z$}
      \correctchoice{$Y$ is positive and $Z$ is negative}
        \wrongchoice{$Y$ and $Z$ must have the same sign}
    \end{choices}
\end{question}
}

\element{halliday-mc}{
\begin{question}{halliday-ch22-q14}
    The electric field at a distance of \SI{10}{\centi\meter} from an isolated point particle with a charge of \SI{2e-9}{\coulomb} is:
    \begin{multicols}{2}
    \begin{choices}
        \wrongchoice{\SI{1.8}{\newton\per\coulomb}}
        \wrongchoice{\SI{180}{\newton\per\coulomb}}
        \wrongchoice{\SI{18}{\newton\per\coulomb}}
      \correctchoice{\SI{1800}{\newton\per\coulomb}}
        \wrongchoice{none of these}
    \end{choices}
    \end{multicols}
\end{question}
}

\element{halliday-mc}{
\begin{question}{halliday-ch22-q15}
    An isolated charged point particle produces an electric field with magnitude $E$ at a point \SI{2}{\meter} away from the charge. 
    A point at which the field magnitude is $\dfrac{E}{4}$ is:
    \begin{choices}
        \wrongchoice{\SI{1}{\meter} away from the particle}
        \wrongchoice{\SI{0.5}{\meter} away from the particle}
        \wrongchoice{\SI{2}{\meter} away from the particle}
      \correctchoice{\SI{4}{\meter} away from the particle}
        \wrongchoice{\SI{8}{\meter} away from the particle}
    \end{choices}
\end{question}
}

\element{halliday-mc}{
\begin{question}{halliday-ch22-q16}
    An isolated charged point particle produces an electric field with magnitude $E$ at a point \SI{2}{\meter} away. 
    At a point \SI{1}{\meter} from the particle the magnitude of the field is:
    \begin{multicols}{3}
    \begin{choices}
        \wrongchoice{$E$}
        \wrongchoice{$2E$}
      \correctchoice{$4E$}
        \wrongchoice{$\dfrac{E}{2}$}
        \wrongchoice{$\dfrac{E}{4}$}
    \end{choices}
    \end{multicols}
\end{question}
}

\element{halliday-mc}{
\begin{question}{halliday-ch22-q17}
    Two protons ($p_1$ and $p_2$) are on the $x$ axis, as shown below. 
    \begin{center}
    \begin{tikzpicture}
        %% NOTE: diagram
    \end{tikzpicture}
    \end{center}
    The directions of the electric field at points 1, 2, and 3, respectively, are:
    \begin{multicols}{3}
    \begin{choices}
        \wrongchoice{
            \begin{tikzpicture}
                %% NOTE: vector options
            \end{tikzpicture}
        }
    \end{choices}
    \end{multicols}
\end{question}
}

\element{halliday-mc}{
\begin{question}{halliday-ch22-q18}
    Two point particles, with a charges of $q_1$ and $q_2$,
        are placed a distance $r$ apart. 
    The electric field is zero at a point $P$ between the particles on the line segment connecting them. 
    We conclude that:
    \begin{choices}
        \wrongchoice{$q_1$ and $q_2$ must have the same magnitude and sign}
        \wrongchoice{P must be midway between the particles}
      \correctchoice{$q_1$ and $q_2$ must have the same sign but may have different magnitudes}
        \wrongchoice{$q_1$ and $q_2$ must have equal magnitudes and opposite signs}
        \wrongchoice{$q_1$ and $q_2$ must have opposite signs and may have different magnitudes}
    \end{choices}
\end{question}
}

\element{halliday-mc}{
\begin{question}{halliday-ch22-q19}
    %% NOTE: This could be altered in many ways???
    The diagrams below depict four different charge distributions. 
    The charge particles are all the same distance from the origin. 
    \begin{center}
    \begin{tikzpicture}
        %% NOTE:
    \end{tikzpicture}
    \end{center}
    The electric field at the origin:
    \begin{choices}
        \wrongchoice{is greatest for situation 1}
        \wrongchoice{is greatest for situation 3}
      \correctchoice{is zero for situation 4}
        \wrongchoice{is downward for situation 1}
        \wrongchoice{is downward for situation 3}
    \end{choices}
\end{question}
}

\newcommand{\hallidayChTwentyTwoQTwenty}{
\begin{tikzpicture}
    %% NOTE:
\end{tikzpicture}
}

\element{halliday-mc}{
\begin{question}{halliday-ch22-q20}
    The diagram shows a particle with positive charge $Q$ and a particle with negative charge $-Q$.
    \begin{center}
        \hallidayChTwentyTwoQTwenty
    \end{center}
    The electric field at point $P$ on the perpendicular bisector of the line joining them is:
    \begin{multicols}{3}
    \begin{choices}
        %% ANS ia A
        \wrongchoice{
            \begin{tikzpicture}
            \end{tikzpicture}
        }
    \end{choices}
    \end{multicols}
\end{question}
}

\element{halliday-mc}{
\begin{question}{halliday-ch22-q21}
    The diagram shows two identical particles, each with positive charge $Q$. 
    \begin{center}
        \hallidayChTwentyTwoQTwenty
    \end{center}
    The electric field at point $P$ on the perpendicular bisector of the line joining them is:
    \begin{multicols}{3}
    \begin{choices}
        %% ANS ia C
        \wrongchoice{
            \begin{tikzpicture}
            \end{tikzpicture}
        }
    \end{choices}
    \end{multicols}
\end{question}
}

\element{halliday-mc}{
\begin{question}{halliday-ch22-q22}
    Two point particles, one with charge \SI{+8e-9}{\coulomb} and the other with charge \SI{-2e-9}{\coulomb},
        are separated by \SI{4}{\meter}.
    The electric field midway between them is:
    \begin{multicols}{2}
    \begin{choices}
        \wrongchoice{\SI{9e9}{\newton\per\coulomb}}
        \wrongchoice{\SI{13 500}{\newton\per\coulomb}}
        \wrongchoice{\SI{135 000}{\newton\per\coulomb}}
        \wrongchoice{\SI{36e-9}{\newton\per\coulomb}}
      \correctchoice{\SI{22.5}{\newton\per\coulomb}}
    \end{choices}
    \end{multicols}
\end{question}
}

\element{halliday-mc}{
\begin{question}{halliday-ch22-q23}
    Two charged point particles are located at two vertices of an equilateral triangle and the electric field is zero at the third vertex. 
    We conclude:
    \begin{choices}
        \wrongchoice{the two particles have charges with opposite signs and the same magnitude}
        \wrongchoice{the two particles have charges with opposite signs and different magnitudes}
        \wrongchoice{the two particles have identical charges}
        \wrongchoice{the two particles have charges with the same sign but different magnitudes}
      \correctchoice{at least one other charged particle is present}
    \end{choices}
\end{question}
}

\element{halliday-mc}{
\begin{question}{halliday-ch22-q24}
    Two point particles, with the same charge,
        are located at two vertices of an equilateral triangle.
    A third charged particle is placed so the electric field at the third vertex is zero. 
    The third particle must:
    \begin{choices}
      \correctchoice{be on the perpendicular bisector of the line joining the first two charges}
        \wrongchoice{be on the line joining the first two charges}
        \wrongchoice{have the same charge as the first two particles}
        \wrongchoice{have charge of the same magnitude as the first two charges but its charge may have a different sign}
        \wrongchoice{be at the center of the triangle}
    \end{choices}
\end{question}
}

\element{halliday-mc}{
\begin{question}{halliday-ch22-q25}
    Positive charge $Q$ is uniformly distributed on a semicircular rod. 
    \begin{center}
    \begin{tikzpicture}
        %% NOTE:
    \end{tikzpicture}
    \end{center}
    What is the direction of the electric field at point $P$,
        the center of the semicircle?
    \begin{multicols}{2}
    \begin{choices}
        %% ANS is D
        \wrongchoice{
            \begin{tikzpicture}
                %% NOTE: vector options
            \end{tikzpicture}
        }
    \end{choices}
    \end{multicols}
\end{question}
}

\element{halliday-mc}{
\begin{question}{halliday-ch22-q26}
    Positive charge $+Q$ is uniformly distributed on the upper half a semicircular rod and negative charge $-Q$ is uniformly distributed on the lower half.
    \begin{center}
    \begin{tikzpicture}
        %% NOTE:
    \end{tikzpicture}
    \end{center}
    What is the direction of the electric field at point $P$,
        the center of the semicircle?
    \begin{multicols}{2}
    \begin{choices}
        %% ANS is B
        \wrongchoice{
            \begin{tikzpicture}
                %% NOTE: vector options
            \end{tikzpicture}
        }
    \end{choices}
    \end{multicols}
\end{question}
}

\element{halliday-mc}{
\begin{question}{halliday-ch22-q27}
    Positive charge $+Q$ is uniformly distributed on the upper half a rod and negative charge $-Q$ is uniformly distributed on the lower half. 
    \begin{center}
    \begin{tikzpicture}
        %% NOTE:
    \end{tikzpicture}
    \end{center}
    What is the direction of the electric field at point $P$,
        on the perpendicular bisector of the rod?
    \begin{multicols}{2}
    \begin{choices}
        %% ANS is B
        \wrongchoice{
            \begin{tikzpicture}
                %% NOTE: vector options
            \end{tikzpicture}
        }
    \end{choices}
    \end{multicols}
\end{question}
}

\element{halliday-mc}{
\begin{question}{halliday-ch22-q28}
    The electric field due to a uniform distribution of charge on a spherical shell is zero:
    \begin{choices}
        \wrongchoice{everywhere}
        \wrongchoice{nowhere}
        \wrongchoice{only at the center of the shell}
      \correctchoice{only inside the shell}
        \wrongchoice{only outside the shell}
    \end{choices}
\end{question}
}

\element{halliday-mc}{
\begin{question}{halliday-ch22-q29}
    A charged particle is placed in an electric field that varies with location. 
    No force is exerted on this charge:
    \begin{choices}
      \correctchoice{at locations where the electric field is zero}
        \wrongchoice{at locations where the electric field strength is $\dfrac{1}{\num{1.6e-19}}\si{\newton\per\coulomb}$}
        \wrongchoice{if the particle is moving along a field line}
        \wrongchoice{if the particle is moving perpendicularly to a field line}
        \wrongchoice{if the field is caused by an equal amount of positive and negative charge}
    \end{choices}
\end{question}
}

\element{halliday-mc}{
\begin{question}{halliday-ch22-q30}
    The magnitude of the force of a \SI{400}{\newton\per\coulomb} electric field on a \SI{0.02}{\coulomb} point charge is:
    \begin{multicols}{2}
    \begin{choices}
      \correctchoice{\SI{8.0}{\newton}}
        \wrongchoice{\SI{8e-5}{\newton}}
        \wrongchoice{\SI{8e-3}{\newton}}
        \wrongchoice{\SI{0.08}{\newton}}
        \wrongchoice{\SI{2e11}{\newton}}
    \end{choices}
    \end{multicols}
\end{question}
}

\element{halliday-mc}{
\begin{question}{halliday-ch22-q31}
    A \SI{200}{\newton\per\coulomb} electric field is in the positive $x$ direction. 
    The force on an electron in this field is:
    \begin{choices}
        \wrongchoice{\SI{200}{\newton} in the positive $x$ direction}
        \wrongchoice{\SI{200}{\newton} in the negative $x$ direction}
        \wrongchoice{\SI{3.2e-17}{\newton} in the positive $x$ direction}
      \correctchoice{\SI{3.2e-17}{\newton} in the negative $x$ direction}
        \wrongchoice{zero}
    \end{choices}
\end{question}
}

\element{halliday-mc}{
\begin{question}{halliday-ch22-q32}
    An electron traveling north enters a region where the electric field is uniform and points north.
    The electron:
    \begin{choices}
        \wrongchoice{speeds up}
      \correctchoice{slows down}
        \wrongchoice{veers east}
        \wrongchoice{veers west}
        \wrongchoice{continues with the same speed in the same direction}
    \end{choices}
\end{question}
}

\element{halliday-mc}{
\begin{question}{halliday-ch22-q33}
    An electron traveling north enters a region where the electric field is uniform and points west.
    The electron:
    \begin{choices}
        \wrongchoice{speeds up}
        \wrongchoice{slows down}
      \correctchoice{veers east}
        \wrongchoice{veers west}
        \wrongchoice{continues with the same speed in the same direction}
    \end{choices}
\end{question}
}

\element{halliday-mc}{
\begin{question}{halliday-ch22-q34}
    Two charged particles are arranged as shown. 
    \begin{center}
    \begin{tikzpicture}
        %% NOTE:
    \end{tikzpicture}
    \end{center}
    In which region could a third particle,
        with charge \SI{+1}{\coulomb},
        be placed so that the net electrostatic force on it is zero?
    \begin{multicols}{2}
    \begin{choices}
      \correctchoice{I only}
        \wrongchoice{I and II only}
        \wrongchoice{lII only}
        \wrongchoice{I and III only}
        \wrongchoice{II only}
    \end{choices}
    \end{multicols}
\end{question}
}

\element{halliday-mc}{
\begin{question}{halliday-ch22-q35}
    An electric dipole consists of a particle with a charge of \SI{+6e-6}{\coulomb} at the origin and a particle with a charge of \SI{-6e-6}{\coulomb} on the $x$ axis at $x=\SI{3e-3}{\meter}$. 
    Its dipole moment is:
    \begin{choices}
        \wrongchoice{\SI{1.8e-8}{\coulomb\meter}, in the positive $x$ direction}
      \correctchoice{\SI{1.8e-8}{\coulomb\meter}, in the negative $x$ direction}
        \wrongchoice{zero because the net charge is zero}
        \wrongchoice{\SI{1.8e-8}{\coulomb\meter}, in the positive $y$ direction}
        \wrongchoice{\SI{1.8e-8}{\coulomb\meter}, in the negative $y$ direction}
    \end{choices}
\end{question}
}

\element{halliday-mc}{
\begin{question}{halliday-ch22-q36}
    The force exerted by a uniform electric field on a dipole is:
    \begin{choices}
        \wrongchoice{parallel to the dipole moment}
        \wrongchoice{perpendicular to the dipole moment}
        \wrongchoice{parallel to the electric field}
        \wrongchoice{perpendicular to the electric field}
      \correctchoice{none of the provided}
    \end{choices}
\end{question}
}

\element{halliday-mc}{
\begin{question}{halliday-ch22-q37}
    An electric field exerts a torque on a dipole only if:
    \begin{choices}
        \wrongchoice{the field is parallel to the dipole moment}
      \correctchoice{the field is not parallel to the dipole moment}
        \wrongchoice{the field is perpendicular to the dipole moment}
        \wrongchoice{the field is not perpendicular to the dipole moment}
        \wrongchoice{the field is uniform}
    \end{choices}
\end{question}
}

\element{halliday-mc}{
\begin{question}{halliday-ch22-q38}
    The torque exerted by an electric field on a dipole is:
    \begin{choices}
        \wrongchoice{parallel to the field and perpendicular to the dipole moment}
        \wrongchoice{parallel to both the field and dipole moment}
      \correctchoice{perpendicular to both the field and dipole moment}
        \wrongchoice{parallel to the dipole moment and perpendicular to the field}
        \wrongchoice{not related to the directions of the field and dipole moment}
    \end{choices}
\end{question}
}

\element{halliday-mc}{
\begin{question}{halliday-ch22-q39}
    The diagrams show four possible orientations of an electric dipole in a uniform electric field $E$.
    \begin{center}
    \begin{tikzpicture}
        %% NOTE:
    \end{tikzpicture}
    \end{center}
    Rank them according to the magnitude of the torque exerted on the dipole by the field,
        least to greatest.
    \begin{choices}
        \wrongchoice{1, 2, 3, 4}
        \wrongchoice{4, 3, 2, 1}
        \wrongchoice{1, 2, 4, 3}
        \wrongchoice{3, 2 and 4 tie, then 1}
      \correctchoice{1, 2 and 4 tie, then 3}
    \end{choices}
\end{question}
}

\element{halliday-mc}{
\begin{question}{halliday-ch22-q40}
    A uniform electric field of \SI{300}{\newton\per\coulomb} makes an angle of \ang{25} with the dipole moment of an electric dipole. 
    If the torque exerted by the field has a magnitude of \SI{2.5e-7}{\newton\meter},
        the dipole moment must be:
    \begin{multicols}{2}
    \begin{choices}
        \wrongchoice{\SI{8.3e-10}{\coulomb\meter}}
        \wrongchoice{\SI{9.2e-10}{\coulomb\meter}}
      \correctchoice{\SI{2.0e-9}{\coulomb\meter}}
        \wrongchoice{\SI{8.3e-5}{\coulomb\meter}}
        \wrongchoice{\SI{1.8e-4}{\coulomb\meter}}
    \end{choices}
    \end{multicols}
\end{question}
}

\element{halliday-mc}{
\begin{question}{halliday-ch22-q41}
    When the dipole moment of a dipole in a uniform electric field rotates to become more nearly aligned with the field:
    \begin{choices}
        \wrongchoice{the field does positive work and the potential energy increases}
      \correctchoice{the field does positive work and the potential energy decreases}
        \wrongchoice{the field does negative work and the potential energy increases}
        \wrongchoice{the field does negative work and the potential energy decreases}
        \wrongchoice{the field does no work}
    \end{choices}
\end{question}
}

\element{halliday-mc}{
\begin{question}{halliday-ch22-q42}
    The dipole moment of a dipole in a \SI{300}{\newton\per\coulomb} electric field is initially perpendicular to the field,
        but it rotates so it is in the same direction as the field. 
    If the moment has a magnitude of \SI{2e-9}{\coulomb\meter},
        the work done by the field is:
    \begin{multicols}{2}
    \begin{choices}
        \wrongchoice{\SI{-12e-7}{\joule}}
        \wrongchoice{\SI{-6e-7}{\joule}}
        \wrongchoice{zero}
      \correctchoice{\SI{6e-7}{\joule}}
        \wrongchoice{\SI{12e-7}{\joule}}
    \end{choices}
    \end{multicols}
\end{question}
}

\element{halliday-mc}{
\begin{question}{halliday-ch22-q43}
    An electric dipole is oriented parallel to a uniform electric field, as shown.
    \begin{center}
    \begin{tikzpicture}
        %% NOTE:
    \end{tikzpicture}
    \end{center}
    It is rotated to one of the five orientations shown below. 
    Rank the final orientations according to the change in the potential energy of the dipole-field system,
        most negative to most positive.
    \begin{multicols}{2}
    \begin{choices}
      \correctchoice{1, 2, 3, 4}
        \wrongchoice{4, 3, 2, 1}
        \wrongchoice{1, 2, 4, 3}
        \wrongchoice{3, 2 and 4 tie, then 1}
        \wrongchoice{1, 2 and 4 tie, then 3}
    \end{choices}
    \end{multicols}
\end{question}
}

\element{halliday-mc}{
\begin{question}{halliday-ch22-q44}
    The purpose of Milliken's oil drop experiment was to determine:
    \begin{choices}
        \wrongchoice{the mass of an electron}
      \correctchoice{the charge of an electron}
        \wrongchoice{the ratio of charge to mass for an electron}
        \wrongchoice{the sign of the charge on an electron}
        \wrongchoice{viscosity}
    \end{choices}
\end{question}
}

\element{halliday-mc}{
\begin{question}{halliday-ch22-q45}
    A charged oil drop with a mass of \SI{2e-4}{\kilo\gram} is held suspended by a downward electric field of \SI{300}{\newton\per\coulomb}. 
    The charge on the drop is:
    \begin{multicols}{2}
    \begin{choices}
        \wrongchoice{\SI{+1.5e-6}{\coulomb}}
        \wrongchoice{\SI{-1.5e-6}{\coulomb}}
        \wrongchoice{\SI{+6.5e-6}{\coulomb}}
      \correctchoice{\SI{-6.5e-6}{\coulomb}}
        \wrongchoice{zero}
    \end{choices}
    \end{multicols}
\end{question}
}


\endinput


