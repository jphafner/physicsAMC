/Chapter: Chapter 1
Learning Objectives:

LO 1.1.0 Solve problems related to measuring things, including length.
LO 1.1.1 Identify the base quantities in the SI system.
LO 1.1.2 Name the most frequently used prefixes for SI units.
LO 1.1.3 Change units (here for length, area, and volume) by using chain-link conversions.
LO 1.1.4 Explain that the meter is defined in terms of the speed of light in a vacuum.

LO 1.2.0 Solve problems related to time.
LO 1.2.1 Change units for time by using chain-link conversions.
LO 1.2.2 Use various measures of time, such as for motion or as determined on different clocks.

LO 1.3.0 Solve problems related to mass.
LO 1.3.1 Change units for mass by using chain-link conversions.
LO 1.3.2 Relate density to mass and volume when the mass is uniformly distributed

Multiple Choice

1.  (5.0  104)  (3.0  106) = 
A)  1.5  109
B)  1.5  1010
C)  1.5  1011
D)  1.5  1012
E)  1.5  1013

Ans:  C
Difficulty:  Easy
Section:  1-1
Learning Objective 1.1.0


2.  (5.0  104)  (3.0  10–6) = 
A)  1.5  10–3
B)  1.5  10–1
C)  1.5  101
D)  1.5  103
E)  1.5  105

Ans:  B
Difficulty:  Easy
Section:  1-1
Learning Objective 1.1.0



3.  5.0  105 + 3.0  106 = 
A)  8.0  105
B)  8.0  106
C)  5.3  105
D)  3.5  105
E)  3.5  106

Ans:  E
Difficulty:  Easy
Section:  1-1
Learning Objective 1.1.0


4.  (7.0  106)/(2.0  10–6) = 
A)  3.5  10–12
B)  3.5  10–6 
C)  3.5
D)  3.5  106
E)  3.5  1012

Ans:  E
Difficulty:  Easy
Section:  1-1
Learning Objective 1.1.0


5. The SI base units have the dimensions of:
A) mass, weight, time
B) length, density, time
C) mass, length, time
D) weight, length, time
E) mass, length, speed

Ans: C
Difficulty: Easy
Section: 1-1
Learning Objective 1.1.1

6.  The SI base unit for mass is: 
A)  gram 
B)  pound 
C)  kilogram 
D)  ounce 
E)  kilopound 



Ans:  C
Difficulty:  Easy
Section:  1-1
Learning Objective 1.1.1

7.  A nanosecond is: 
A)  109 s 
B)  10–9 s 
C)  10-6 s 
D)  10-15 s 
E)  10–12 s

Ans:  B
Difficulty:  Easy
Section:  1-1
Learning Objective 1.1.2

8.  A gram is: 
A)  10–6 kg 
B)  10–3 kg 
C)  1 kg 
D)  103 kg 
E)  106 kg 

Ans:  B
Difficulty:  Easy
Section:  1-1
Learning Objective 1.1.2

9.  1 mi is equivalent to 1609 m so 55 mph is: 
A)  15 m/s 
B)  25 m/s 
C)  81 m/s 
D)  123 m/s 
E)  1500 m/s 

Ans:  B
Difficulty:  Medium
Section:  1-2
Learning Objective 1.2.1

10.  In 1866, the U. S. Congress defined the U. S. yard as exactly 3600/3937 international meter. This was done primarily because: 
A)  length can be measured more accurately in meters than in yards 
B)  the meter is more stable than the yard 



C)  this definition relates the common U. S. length units to a more widely used system
D)  there are more wavelengths in a yard than in a meter
E)  the members of this Congress were exceptionally intelligent

Ans:  C
Difficulty:  Easy
Section:  1-1
Learning Objective 1.1.0


11.  Which of the following is closest to a yard in length? 
A)  0.01 m 
B)  0.1 m 
C)  1 m 
D)  100 m 
E)  1000 m 

Ans:  C
Difficulty:  Easy
Section:  1-1
Learning Objective 1.1.0


12.  There is no SI base unit for area because: 
A)  an area has no thickness; hence no physical standard can be built
B)  we live in a three (not a two) dimensional world 
C)  it is impossible to express square feet in terms of meters
D)  area can be expressed in terms of square meters
E)  area is not an important physical quantity 

Ans:  D
Difficulty:  Easy
Section:  1-1
Learning Objective 1.1.0


13.  1 m is equivalent to 3.281 ft.  A cube with an edge of 1.5 ft has a volume of:
A)  1.2  102 m3
B)  9.6  10–2 m3
C)  10.5 m3
D)  9.5  10–2 m3
E)  0.21 m3

Ans:  B
Difficulty:  Medium
Section:  1-1



Learning Objective 1.1.3

14.  A sphere with a radius of 1.7 cm has a volume of: 
A)  2.1  10–5 m3
B)  9.1  10–4 m3
C)  3.6  10–3 m3
D)  0.11 m3
E)  21 m3

Ans:  A
Difficulty:  Medium
Section:  1-1
Learning Objective 1.1.3

15.  A sphere with a radius of 1.7 cm has a surface area of: 
A)  2.1  10–5 m2
B)  9.1  10–4 m2
C)  3.6  10–3 m2
D)  0.11 m2
E)  36 m2

Ans:  C
Difficulty:  Medium
Section:  1-1
Learning Objective 1.1.3

16.  A right circular cylinder with a radius of 2.3 cm and a height of 1.4 m has a volume of: 
A)  23 m3
B)  0.23 m3
C)  9.3  10–3 m3
D)  2.3  10–3 m3
E)  7.4  10–4 m3

Ans:  D
Difficulty:  Medium
Section:  1-1
Learning Objective 1.1.3

17.  A right circular cylinder with a radius of 2.3 cm and a height of 1.4 cm has a total surface area of: 
A)  1.7  10–3 m2
B)  3.2  10–3 m2
C)  2.0  10–3 m2
D)  5.3  10–3 m2
E)  7.4  10–3 m2



Ans:  D
Difficulty:  Medium
Section:  1-1
Learning Objective 1.1.3

18.  A cubic box with an edge of exactly 1 cm has a volume of:
A)  10–9 m3
B)  10–6 m3
C)  10–3 m3
D)  103 m3
E)  106 m3

Ans:  B
Difficulty:  Easy
Section:  1-1
Learning Objective 1.1.3

19.  A square with an edge of exactly 1 cm has an area of:
A)  10–6 m2
B)  10–4 m2
C)  10-2 m2 
D)  102 m2
E)  104 m2

Ans:  B
Difficulty:  Easy
Section:  1-1
Learning Objective 1.1.3

20.  The SI standard of length is based on: 
A)  the distance from the north pole to the equator along a meridian passing through Paris 
B)  wavelength of light emitted by Hg198 
C)  wavelength of light emitted by Kr86 
D)  a precision meter stick in Paris 
E)  the speed of light 

Ans:  E
Difficulty:  Easy
Section:  1-1
Learning Objective 1.1.4

21. Which of the SI standard units is NOT based on a constant of nature?
A) The meter, because the speed of light can vary from place to place.
B) The second, because cesium cannot be found everywhere.
C) The kilogram, because it is based on an arbitrary piece of metal.



D) The kilogram, because the gram is the standard unit.
E) The meter, because it is defined in terms of the speed of light in vacuum, and there is no perfect vacuum anywhere.

Ans: C.
Difficulty: Easy
Section: 1-1
Learning Objective 1.1.4

22.  The number of significant figures in 0.00150 is: 
A)  2 
B)  3 
C)  4 
D)  5 
E)  6 

Ans:  B
Difficulty:  Easy
Section:  1-1
Learning Objective 1.1.0


23.  The number of significant figures in 15.0 is: 
A)  1 
B)  2 
C)  3 
D)  4 
E)  5 

Ans:  C
Difficulty:  Easy
Section:  1-1
Learning Objective 1.1.0


24.  3.2  2.7 = 
A)  9 
B)  8 
C)  8.6 
D)  8.64 
E)  8.640 

Ans:  C
Difficulty:  Easy
Section:  1-1
Learning Objective 1.1.0



25.  1.513 + 27.3 =
A)  29
B)  28.8
C)  28.9
D)  28.81
E)  28.813

Ans:  B
Difficulty:  Easy
Section:  1-1
Learning Objective 1.1.0


26.  The SI standard of time is based on: 
A)  the daily rotation of the Earth 
B)  the frequency of light emitted by Kr86 
C)  the yearly revolution of the Earth about the sun 
D)  a precision pendulum clock 
E)  none of these 

Ans:  E
Difficulty:  Easy
Section:  1-2
Learning Objective 1.2.0


27. Six million seconds is approximately:
A)  One day
B)  Ten days
C)  Two months
D)  One year
E)  Ten years

Ans: C
Difficulty:  Easy
Section:  1-2
Learning Objective 1.2.1

28. Metric time is defined so that one day equals 10 hours; one hour equals 100 minutes; and one minute equals 100 seconds. One metric second equals how many normal seconds?
A) 0.60
B) 0.864
C) 1.00
D) 1.16
E) 1.67

Ans: B
Difficulty: Medium
Section: 1-2
Learning Objective 1.2.2

29. Which of the following weighs about a pound on Earth? 
A)  0.05 kg 
B)  0.5 kg 
C)  5 kg 
D)  50 kg 
E)  500 kg 

Ans:  B
Difficulty:  Easy
Section:  1-3
Learning Objective 1.3.0


30. If 1 u = 1.66 x 10-27 kg, 1 gram =
A)  1.66 x 10-24 u
B)  1.66 x 1024 u
C)  1.66 x 1027 u
D)  6.02 x 1026 u
E)  6.02 x 1023 u

Ans: E
Difficulty: Easy
Section: 1-3
Learning Objective 1.3.1

31. One tonne (metric ton) is 1000 kg. How many grams are there in one tonne?
A) 10-6
B) 10-3
C) 1
D) 103
E) 106

Ans: E
Difficulty: Easy
Section: 1-3
Learning Objective 1.3.1

32.  Two girders are made of the same material.  Girder A is twice as long as girder B and has a cross-sectional area that is twice as great.  The ratio of the mass density of girder A to the mass density of girder B is:
A)  4
B)  2
C)  1
D)  1/2
E)  1/4

Ans:  C
Difficulty:  Easy
Section:  1-3
Learning Objective 1.3.2

33.  The unit of mass density might be:
A)  pound per cubic foot
B)  gram per liter
C)  kilogram per meter
D)  cubic kilogram per meter
E)  cubic meter per kilogram

Ans:  B
Difficulty:  Easy
Section:  1-3
Learning Objective 1.3.2

34.  A sphere has a radius of 21 cm and a mass of 1.9 kg.  Its mass density is about:
A)  4.9  10–5 kg/m3
B)  2.1  10–4 kg/m3
C)  2.0  10–2 kg/m3
D)  16 kg/m3
E)  49 kg/m3

Ans:  E
Difficulty:  Medium
Section:  1-3
Learning Objective 1.3.2

35.  During a short interval of time the speed v in m/s of an automobile is given by v = at2 + bt3, where the time t is in seconds. The units of a and b are respectively: 
A)  ms2; ms4
B)  s3/m; s4/m 
C)  m/s2; m/s3
D)  m/s3; m/s4
E)  m/s4; m/s5

Ans:  D
Difficulty:  Medium
Section:  1-2
Learning Objective 1.2.0


36.  Suppose A = BC, where A has the dimensions L/M and C has the dimensions L/T. Then B has the dimension: 
A)  T/M 
B)  L2/TM 
C)  TM/L2
D)  L2T/M 
E)  M/L2T 

Ans:  A
Difficulty:  Medium
Section:  1-3
Learning Objective 1.3.0


37.  Suppose A = BnCm, where A has dimensions LT, B has dimensions L2T–1, and C has dimensions LT2. Then the exponents n and m have the values: 
A)  2/3; 1/3 
B)  2; 3 
C)  4/5; –1/5 
D)  1/5; 3/5 
E)  1/2; 1/2  

Ans:  D
Difficulty:  Medium
Section:  1-2
Learning Objective 1.2.0

