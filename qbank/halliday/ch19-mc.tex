
%%--------------------------------------------------
%% Halliday: Fundamentals of Physics
%%--------------------------------------------------


%% Chapter 19: The Kinetic Theory of Gases
%%--------------------------------------------------


%% Learning Objectives
%%--------------------------------------------------

%% 19.01: Identify Avogadro's number $N_A$.
%% 19.02: Apply the relationship between the number of moles $n$, the number of molecules $N$, and Avogadro’s number $N_A$.
%% 19.03: Apply the relationships between the mass $m$ of a sample, the molar mass $M$ of the molecules in the sample, the number of moles $n$ in the sample, and Avogadro's number $N_A$.
%% 18.01: Identify the lowest temperature as 0 on the Kelvin scale (absolute zero).


%% Halliday Multiple Choice Questions
%%--------------------------------------------------
\element{halliday-mc}{
\begin{question}{halliday-ch19-q01}
    Evidence that a gas consists mostly of empty space is the fact that:
    \begin{choices}
      \correctchoice{the density of a gas becomes much greater when it is liquefied}
        \wrongchoice{gases exert pressure on the walls of their containers}
        \wrongchoice{gases are transparent}
        \wrongchoice{heating a gas increases the molecular motion}
        \wrongchoice{nature abhors a vacuum}
    \end{choices}
\end{question}
}

\element{halliday-mc}{
\begin{question}{halliday-ch19-q02}
    Air enters a hot-air furnace at \SI{7}{\degreeCelsius} and leaves at \SI{77}{\degreeCelsius}.
    If the pressure does not change each entering cubic meter of air expands to:
    \begin{multicols}{3}
    \begin{choices}
        \wrongchoice{\SI{0.80}{\meter\cubed}}
      \correctchoice{\SI{1.25}{\meter\cubed}}
        \wrongchoice{\SI{1.9}{\meter\cubed}}
        \wrongchoice{\SI{7.0}{\meter\cubed}}
        \wrongchoice{\SI{11}{\meter\cubed}}
    \end{choices}
    \end{multicols}
\end{question}
}

\element{halliday-mc}{
\begin{question}{halliday-ch19-q03}
    \SI{273}{\centi\meter\cubed} of an ideal gas is at \SI{0}{\degreeCelsius}. 
    It is heated at constant pressure to \SI{10}{\degreeCelsius}. 
    It will now occupy:
    \begin{multicols}{3}
    \begin{choices}
        \wrongchoice{\SI{263}{\centi\meter\cubed}}
        \wrongchoice{\SI{273}{\centi\meter\cubed}}
      \correctchoice{\SI{283}{\centi\meter\cubed}}
        \wrongchoice{\SI{278}{\centi\meter\cubed}}
        \wrongchoice{\SI{293}{\centi\meter\cubed}}
    \end{choices}
    \end{multicols}
\end{question}
}

\element{halliday-mc}{
\begin{question}{halliday-ch19-q04}
    Two identical rooms in a house are connected by an open doorway. 
    The temperatures in the two rooms are maintained at different values. 
    Which room contains more air?
    \begin{choices}
        \wrongchoice{the room with higher temperature}
      \correctchoice{the room with lower temperature}
        \wrongchoice{the room with higher pressure}
        \wrongchoice{neither because both have the same pressure}
        \wrongchoice{neither because both have the same volume}
    \end{choices}
\end{question}
}

\element{halliday-mc}{
\begin{question}{halliday-ch19-q05}
    It is known that \SI{28}{\gram} of a certain ideal gas occupy \SI{22.4}{\liter} at standard conditions (\SI{0}{\degreeCelsius}, \SI{1}{\atm}). 
    The volume occupied by \SI{42}{\gram} of this gas at standard conditions is:
    \begin{multicols}{2}
    \begin{choices}
        \wrongchoice{\SI{14.9}{\liter}}
        \wrongchoice{\SI{22.4}{\liter}}
      \correctchoice{\SI{33.6}{\liter}}
        \wrongchoice{\SI{42}{\liter}}
        \wrongchoice{more data are needed}
    \end{choices}
    \end{multicols}
\end{question}
}

\element{halliday-mc}{
\begin{question}{halliday-ch19-q06}
    An automobile tire is pumped up to a gauge pressure of \SI{2.0e5}{\pascal} when the temperature is \SI{27}{\degreeCelsius}. 
    What is its gauge pressure after the car has been running on a hot day so that the tire temperature is \SI{77}{\degreeCelsius}?
    Assume that the volume remains fixed and take atmospheric pressure to be \SI{1.013e5}{\pascal}.
    \begin{multicols}{2}
    \begin{choices}
      \correctchoice{\SI{1.6e5}{\pascal}}
        \wrongchoice{\SI{2.6e5}{\pascal}}
        \wrongchoice{\SI{3.6e5}{\pascal}}
        \wrongchoice{\SI{5.9e5}{\pascal}}
        \wrongchoice{\SI{7.9e5}{\pascal}}
    \end{choices}
    \end{multicols}
\end{question}
}

\element{halliday-mc}{
\begin{question}{halliday-ch19-q07}
    A sample of an ideal gas is compressed by a piston from \SI{10}{\meter\cubed} to \SI{5}{\meter\cubed} and simultaneously cooled from \SI{273}{\degreeCelsius} to \SI{0}{\degreeCelsius}.
    As a result there is:
    \begin{choices}
        \wrongchoice{an increase in pressure}
        \wrongchoice{a decrease in pressure}
        \wrongchoice{a decrease in density}
        \wrongchoice{no change in density}
      \correctchoice{an increase in density}
    \end{choices}
\end{question}
}

\element{halliday-mc}{
\begin{question}{halliday-ch19-q08}
    A \SI{2}{\meter\cubed} weather balloon is loosely filled with helium at \SI{1}{\atm} (\SI{76}{\centi\meter} \ce{Hg}) and at \SI{27}{\degreeCelsius}.
    At an elevation of \SI{20 000}{\foot},
        the atmospheric pressure is down to \SI{38}{\centi\meter} \ce{Hg} and the helium has expanded,
        being under no constraint from the confining bag. 
    If the temperature at this elevation is \SI{-48}{\degreeCelsius},
        the gas volume is:
    \begin{multicols}{3}
    \begin{choices}
      \correctchoice{\SI{3}{\meter\cubed}}
        \wrongchoice{\SI{4}{\meter\cubed}}
        \wrongchoice{\SI{2}{\meter\cubed}}
        \wrongchoice{\SI{2.5}{\meter\cubed}}
        \wrongchoice{\SI{5.3}{\meter\cubed}}
    \end{choices}
    \end{multicols}
\end{question}
}

\element{halliday-mc}{
\begin{question}{halliday-ch19-q09}
    Oxygen (molar mass = \SI{32}{\gram}) occupies a volume of \SI{12}{\liter} when its temperature is \SI{20}{\degreeCelsius} and its pressure is \SI{1}{\atm}.
    Using $R = \SI{0.082}{\liter\atm\per\mole\per\kelvin}$,
        calculate the mass of the oxygen:
    \begin{multicols}{3}
    \begin{choices}
        \wrongchoice{\SI{6.4}{\gram}}
        \wrongchoice{\SI{10.7}{\gram}}
      \correctchoice{\SI{16}{\gram}}
        \wrongchoice{\SI{32}{\gram}}
        \wrongchoice{\SI{64}{\gram}}
    \end{choices}
    \end{multicols}
\end{question}
}

\element{halliday-mc}{
\begin{question}{halliday-ch19-q10}
    An ideal gas occupies \SI{12}{\liter} at \SI{293}{\kelvin} and \SI{1}{\atm} (\SI{76}{\centi\meter} \ce{Hg}).
    Its temperature is now raised to \SI{373}{\kelvin} and its pressure increased to \SI{215}{\centi\meter} \ce{Hg}. 
    The new volume is:
    \begin{multicols}{2}
    \begin{choices}
        \wrongchoice{\SI{0.2}{\liter}}
      \correctchoice{\SI{5.4}{\liter}}
        \wrongchoice{\SI{13.6}{\liter}}
        \wrongchoice{\SI{20.8}{\liter}}
        \wrongchoice{none of the provided}
    \end{choices}
    \end{multicols}
\end{question}
}

\element{halliday-mc}{
\begin{question}{halliday-ch19-q11}
    Use $R=\SI{8.2e-5}{\meter\cubed\atm\per\mole\per\kelvin}$ and $N_A=\SI{6.02e23}{\per\mole}$.
    The approximate number of air molecules in a \SI{1}{\meter\cubed} volume at room temperature (\SI{300}{\kelvin}) and atmospheric pressure is:
    \begin{multicols}{2}
    \begin{choices}
        \wrongchoice{\num{41}}
        \wrongchoice{\num{450}}
      \correctchoice{\num{2.5e25}}
        \wrongchoice{\num{2.7e26}}
        \wrongchoice{\num{5.4e26}}
    \end{choices}
    \end{multicols}
\end{question}
}

\element{halliday-mc}{
\begin{question}{halliday-ch19-q12}
    An air bubble doubles in volume as it rises from the bottom of a lake (\SI{1000}{\kilo\gram\per\meter\cubed}).
    Ignoring any temperature changes,
        the depth of the lake is:
    \begin{multicols}{3}
    \begin{choices}
        \wrongchoice{\SI{21}{\meter}}
        \wrongchoice{\SI{0.76}{\meter}}
        \wrongchoice{\SI{4.9}{\meter}}
      \correctchoice{\SI{10}{\meter}}
        \wrongchoice{\SI{0.99}{\meter}}
    \end{choices}
    \end{multicols}
\end{question}
}

\element{halliday-mc}{
\begin{question}{halliday-ch19-q13}
    An isothermal process for an ideal gas is represented on a $p$-$V$ diagram by:
    \begin{choices}
        \wrongchoice{a horizontal line}
        \wrongchoice{a vertical line}
        \wrongchoice{a portion of an ellipse}
        \wrongchoice{a portion of a parabola}
      \correctchoice{a portion of a hyperbola}
    \end{choices}
\end{question}
}

\element{halliday-mc}{
\begin{question}{halliday-ch19-q14}
    An ideal gas undergoes an isothermal process starting with a pressure of \SI{2e5}{\pascal} and a volume of \SI{6}{\centi\meter\cubed}. 
    Which of the following might be the pressure and volume of the final state?
    \begin{choices}
        \wrongchoice{\SI{1e5}{\pascal} and \SI{10}{\centi\meter\cubed}}
        \wrongchoice{\SI{3e5}{\pascal} and \SI{6}{\centi\meter\cubed}}
        \wrongchoice{\SI{4e5}{\pascal} and \SI{4}{\centi\meter\cubed}}
      \correctchoice{\SI{6e5}{\pascal} and \SI{2}{\centi\meter\cubed}}
        \wrongchoice{\SI{8e5}{\pascal} and \SI{2}{\centi\meter\cubed}}
    \end{choices}
\end{question}
}

\element{halliday-mc}{
\begin{question}{halliday-ch19-q15}
    The pressures $p$ and volumes $V$ of five ideal gases,
        with the same number of molecules, are given below. 
    Which has the highest temperature?
    \begin{choices}
        \wrongchoice{$p = \SI{1e5}{\pascal}$ and $V = \SI{10}{\centi\meter\cubed}$}
      \correctchoice{$p = \SI{3e5}{\pascal}$ and $V = \SI{6}{\centi\meter\cubed}$}
        \wrongchoice{$p = \SI{4e5}{\pascal}$ and $V = \SI{4}{\centi\meter\cubed}$}
        \wrongchoice{$p = \SI{6e5}{\pascal}$ and $V = \SI{2}{\centi\meter\cubed}$}
        \wrongchoice{$p = \SI{8e5}{\pascal}$ and $V = \SI{2}{\centi\meter\cubed}$}
    \end{choices}
\end{question}
}

\element{halliday-mc}{
\begin{question}{halliday-ch19-q16}
    During a slow adiabatic expansion of a gas:
    \begin{choices}
        \wrongchoice{the pressure remains constant}
        \wrongchoice{energy is added as heat}
        \wrongchoice{work is done on the gas}
      \correctchoice{no energy enters or leaves as heat}
        \wrongchoice{the temperature is constant}
    \end{choices}
\end{question}
}

\element{halliday-mc}{
\begin{question}{halliday-ch19-q17}
    An adiabatic process for an ideal gas is represented on a $p$-$V$ diagram by:
    \begin{choices}
        \wrongchoice{a horizontal line}
        \wrongchoice{a vertical line}
        \wrongchoice{a hyperbola}
        \wrongchoice{a circle}
      \correctchoice{none of the provided}
    \end{choices}
\end{question}
}

\element{halliday-mc}{
\begin{question}{halliday-ch19-q18}
    A real gas undergoes a process that can be represented as a curve on a $p$-$V$ diagram.
    The work done by the gas during this process is:
    \begin{multicols}{2}
    \begin{choices}
        \wrongchoice{$pV$}
        \wrongchoice{$p\left(V_2-V_1\right)$}
        \wrongchoice{$\left(p_2-p-1\right)V$}
      \correctchoice{$\int p\,\mathrm{d}V$}
        \wrongchoice{$V\,\mathrm{d}p$}
    \end{choices}
    \end{multicols}
\end{question}
}

\element{halliday-mc}{
\begin{question}{halliday-ch19-q19}
    A real gas is changed slowly from state 1 to state 2. 
    During this process no work is done on or by the gas. 
    This process must be:
    \begin{multicols}{2}
    \begin{choices}
        \wrongchoice{isothermal}
        \wrongchoice{adiabatic}
      \correctchoice{isovolumic}
        \wrongchoice{isobaric}
        \wrongchoice{a closed cycle with state 1 coinciding with state 2}
    \end{choices}
    \end{multicols}
\end{question}
}

\element{halliday-mc}{
\begin{question}{halliday-ch19-q20}
    A given mass of gas is enclosed in a suitable container so that it may be maintained at constant volume.
    Under these conditions,
        there can be no change in what property of the gas?
    \begin{choices}
        \wrongchoice{Pressure}
      \correctchoice{Density}
        \wrongchoice{Molecular kinetic energy}
        \wrongchoice{Internal energy}
        \wrongchoice{Temperature}
    \end{choices}
\end{question}
}

\element{halliday-mc}{
\begin{question}{halliday-ch19-q21}
    A quantity of an ideal gas is compressed to half its initial volume. 
    The process may be adiabatic,
        isothermal, or isobaric. 
    Rank those three processes in order of the work required of an external agent, least to greatest.
    \begin{choices}
        \wrongchoice{adiabatic, isothermal, isobaric}
        \wrongchoice{adiabatic, isobaric, isothermal}
        \wrongchoice{isothermal, adiabatic, isobaric}
        \wrongchoice{isobaric, adiabatic, isothermal}
      \correctchoice{isobaric, isothermal, adiabatic}
    \end{choices}
\end{question}
}

\element{halliday-mc}{
\begin{question}{halliday-ch19-q22}
    During a reversible adiabatic expansion of an ideal gas,
        which of the following is \emph{not} true?
    \begin{choices}
        \wrongchoice{$pV^{\gamma}=\text{constant}$}
        \wrongchoice{$pV = nRT$}
        \wrongchoice{$T V^{\gamma-1}=\text{constant}$}
        \wrongchoice{$|W| = \int p\,\mathrm{d}V$}
        %% D conflicts with A, only A or D are rational options
      \correctchoice{$pV = \text{constant}$}
    \end{choices}
\end{question}
}

\element{halliday-mc}{
\begin{question}{halliday-ch19-q23}
    In order that a single process be both isothermal and isobaric:
    \begin{choices}
        \wrongchoice{one must use an ideal gas}
        \wrongchoice{such a process is impossible}
      \correctchoice{a change of phase is essential}
        \wrongchoice{one may use any real gas such as \ce{N2}}
        \wrongchoice{one must use a solid}
    \end{choices}
\end{question}
}

\element{halliday-mc}{
\begin{question}{halliday-ch19-q24}
    Over 1 cycle of a cyclic process in which a system does net work on its environment:
    \begin{choices}
        \wrongchoice{the change in the pressure of the system cannot be zero}
        \wrongchoice{the change in the volume of the system cannot be zero}
        \wrongchoice{the change in the temperature of the system cannot be zero}
        \wrongchoice{the change in the internal energy of the system cannot be zero}
        \wrongchoice{none of the provided}
    \end{choices}
\end{question}
}

\element{halliday-mc}{
\begin{question}{halliday-ch19-q25}
    Evidence that molecules of a gas are in constant motion is:
    \begin{choices}
        \wrongchoice{winds exert pressure}
      \correctchoice{two gases interdiffuse quickly}
        \wrongchoice{warm air rises}
        \wrongchoice{energy as heat is needed to vaporize a liquid}
        \wrongchoice{gases are easily compressed}
    \end{choices}
\end{question}
}

\element{halliday-mc}{
\begin{question}{halliday-ch19-q26}
    According to the kinetic theory of gases,
        the pressure of a gas is due to:
    \begin{choices}
        \wrongchoice{change of kinetic energy of molecules as they strike the wall}
      \correctchoice{change of momentum of molecules as the strike the wall}
        \wrongchoice{average kinetic energy of the molecules}
        \wrongchoice{force of repulsion between the molecules}
        \wrongchoice{rms speed of the molecules}
    \end{choices}
\end{question}
}

\element{halliday-mc}{
\begin{question}{halliday-ch19-q27}
    The force on the walls of a vessel of a contained gas is due to:
    \begin{choices}
        \wrongchoice{the repulsive force between gas molecules}
        \wrongchoice{a slight loss in the speed of a gas molecule during a collision with the wall}
      \correctchoice{a change in momentum of a gas molecule during a collision with the wall}
        \wrongchoice{elastic collisions between gas molecules}
        \wrongchoice{inelastic collisions between gas molecules}
    \end{choices}
\end{question}
}

\element{halliday-mc}{
\begin{question}{halliday-ch19-q28}
    A gas is confined to a cylindrical container of radius \SI{1}{\centi\meter} and length \SI{1}{\meter}. 
    The pressure exerted on an end face,
        compared with the pressure exerted on the long curved face, is:
    \begin{choices}
        \wrongchoice{smaller because its area is smaller}
        \wrongchoice{smaller because most molecules cannot traverse the length of the cylinder without under- going collisions}
        \wrongchoice{larger because the face is flat}
        \wrongchoice{larger because the molecules have a greater distance in which to accelerate before they strike the face}
      \correctchoice{none of the provided}
    \end{choices}
\end{question}
}

\element{halliday-mc}{
\begin{question}{halliday-ch19-q29}
    Air is pumped into a bicycle tire at constant temperature. 
    The pressure increases because:
    \begin{choices}
      \correctchoice{more molecules strike the tire wall per second}
        \wrongchoice{the molecules are larger}
        \wrongchoice{the molecules are farther apart}
        \wrongchoice{each molecule is moving faster}
        \wrongchoice{each molecule has more kinetic energy}
    \end{choices}
\end{question}
}

\element{halliday-mc}{
\begin{question}{halliday-ch19-q30}
    The temperature of a gas is most closely related to:
    \begin{choices}
      \correctchoice{the kinetic energy of translation of its molecules}
        \wrongchoice{its total molecular kinetic energy}
        \wrongchoice{the sizes of its molecules}
        \wrongchoice{the potential energy of its molecules}
        \wrongchoice{the total energy of its molecules}
    \end{choices}
\end{question}
}

\element{halliday-mc}{
\begin{question}{halliday-ch19-q31}
    The temperature of low pressure hydrogen is reduced from \SI{100}{\degreeCelsius} to \SI{20}{\degreeCelsius}.
    The rms speed of its molecules decreases by approximately:
    \begin{multicols}{3}
    \begin{choices}
        \wrongchoice{\SI{80}{\percent}}
        \wrongchoice{\SI{89}{\percent}}
        \wrongchoice{\SI{46}{\percent}}
        \wrongchoice{\SI{21}{\percent}}
      \correctchoice{\SI{11}{\percent}}
    \end{choices}
    \end{multicols}
\end{question}
}

\element{halliday-mc}{
\begin{question}{halliday-ch19-q32}
    The mass of an oxygen molecule is 16 times that of a hydrogen molecule. 
    At room temperature,
        the ratio of the rms speed of an oxygen molecule to that of a hydrogen molecule is:
    \begin{multicols}{3}
    \begin{choices}
        \wrongchoice{\num{16}}
        \wrongchoice{\num{4}}
        \wrongchoice{\num{1}}
      \correctchoice{\num{1/4}}
        \wrongchoice{\num{1/16}}
    \end{choices}
    \end{multicols}
\end{question}
}

\element{halliday-mc}{
\begin{question}{halliday-ch19-q33}
    The rms speed of an oxygen molecule at \SI{0}{\degreeCelsius} is \SI{460}{\meter\per\second}.
    If the molar mass of oxygen is \SI{32}{\gram} and that of helium is \SI{4}{\gram},
        then the rms speed of a helium molecule at \SI{0}{\degreeCelsius} is:
    \begin{multicols}{3}
    \begin{choices}
        \wrongchoice{\SI{230}{\meter\per\second}}
        \wrongchoice{\SI{326}{\meter\per\second}}
        \wrongchoice{\SI{650}{\meter\per\second}}
        \wrongchoice{\SI{920}{\meter\per\second}}
      \correctchoice{\SI{1300}{\meter\per\second}}
    \end{choices}
    \end{multicols}
\end{question}
}

\element{halliday-mc}{
\begin{question}{halliday-ch19-q34}
    A sample of argon gas (molar mass \SI{40}{\gram}) is at four times the absolute temperature of a sample of hydrogen gas (molar mass \SI{2}{\gram}). 
    The ratio of the rms speed of the argon molecules to that of the hydrogen is:
    \begin{multicols}{3}
    \begin{choices}
        \wrongchoice{$1$}
        \wrongchoice{$5$}
        \wrongchoice{$\dfrac{1}{5}$}
      \correctchoice{$\sqrt{5}$}
        \wrongchoice{$\dfrac{1}{\sqrt{5}}$}
    \end{choices}
    \end{multicols}
\end{question}
}

\element{halliday-mc}{
\begin{question}{halliday-ch19-q35}
    If the molecules in a tank of hydrogen have the same rms speed as the molecules in a tank of oxygen,
        we may be sure that:
    \begin{choices}
        \wrongchoice{the pressures are the same}
        \wrongchoice{the hydrogen is at the higher temperature}
        \wrongchoice{the hydrogen is at the greater pressure}
        \wrongchoice{the temperatures are the same}
      \correctchoice{the oxygen is at the higher temperature}
    \end{choices}
\end{question}
}

\element{halliday-mc}{
\begin{question}{halliday-ch19-q36}
    The principle of equipartition of energy states that the internal energy of a gas is shared equally:
    \begin{choices}
        \wrongchoice{among the molecules}
        \wrongchoice{between kinetic and potential energy}
      \correctchoice{among the relevant degrees of freedom}
        \wrongchoice{between translational and vibrational kinetic energy}
        \wrongchoice{between temperature and pressure}
    \end{choices}
\end{question}
}

\element{halliday-mc}{
\begin{question}{halliday-ch19-q37}
    The number of degrees of freedom of a rigid diatomic molecule is:
    \begin{multicols}{3}
    \begin{choices}
        \wrongchoice{\num{2}}
        \wrongchoice{\num{3}}
        \wrongchoice{\num{4}}
      \correctchoice{\num{5}}
        \wrongchoice{\num{6}}
    \end{choices}
    \end{multicols}
\end{question}
}

\element{halliday-mc}{
\begin{question}{halliday-ch19-q38}
    The number of degrees of freedom of a triatomic molecule is:
    \begin{multicols}{3}
    \begin{choices}
        \wrongchoice{\num{1}}
        \wrongchoice{\num{3}}
        \wrongchoice{\num{6}}
        \wrongchoice{\num{8}}
      \correctchoice{\num{9}}
    \end{choices}
    \end{multicols}
\end{question}
}

\element{halliday-mc}{
\begin{question}{halliday-ch19-q39}
    Five molecules have speeds of \SI{2.8}{\meter\per\second},
        \SI{3.2}{\meter\per\second},
        \SI{5.8}{\meter\per\second},
        \SI{7.3}{\meter\per\second}, and \SI{7.4}{\meter\per\second}.
    Their root-mean-square speed is closest to:
    \begin{multicols}{3}
    \begin{choices}
        \wrongchoice{\SI{5.3}{\meter\per\second}}
      \correctchoice{\SI{5.7}{\meter\per\second}}
        \wrongchoice{\SI{7.3}{\meter\per\second}}
        \wrongchoice{\SI{28}{\meter\per\second}}
        \wrongchoice{\SI{32}{\meter\per\second}}
    \end{choices}
    \end{multicols}
\end{question}
}

\element{halliday-mc}{
\begin{question}{halliday-ch19-q40}
    The speeds of \num{25} molecules are distributed as follows: \num{5} in the range from \SIrange{2}{3}{\meter\per\second},
        \num{10} in the range from \SIrange{3}{4}{\meter\per\second},
        \num{5} in the range from \SIrange{4}{5}{\meter\per\second},
        \num{3} in the range from \SIrange{5}{6}{\meter\per\second},
        \num{1} in the range from \SIrange{6}{7}{\meter\per\second},
        and \num{1} in the range from \SIrange{7}{8}{\meter\per\second}.
    Their average speed is about:
    \begin{multicols}{3}
    \begin{choices}
        \wrongchoice{\SI{2}{\meter\per\second}}
        \wrongchoice{\SI{3}{\meter\per\second}}
      \correctchoice{\SI{4}{\meter\per\second}}
        \wrongchoice{\SI{5}{\meter\per\second}}
        \wrongchoice{\SI{6}{\meter\per\second}}
    \end{choices}
    \end{multicols}
\end{question}
}

\element{halliday-mc}{
\begin{question}{halliday-ch19-q41}
    In a system of $N$ gas molecules,
        the individual speeds are $v_1$, $v_2$, \ldots, $v_N$.
    The rms speed of these molecules is:
    \begin{choices}
        \wrongchoice{$\dfrac{1}{N}\sqrt{v_1+v_2+\ldots+v_n}$}
        \wrongchoice{$\dfrac{1}{N}\sqrt{v_1^2+v_2^2+\ldots+v_n^2}$}
      \correctchoice{$\sqrt{\dfrac{v_1^2+v_2^2+\ldots+v_n^2}{N}}$}
        \wrongchoice{$\sqrt{\left[\dfrac{v_1+v_2+\ldots+v_n}{N}\right]^2}$}
        \wrongchoice{$\sqrt{\dfrac{\left(v_1+v_2+\ldots+v_n\right)^2}{N}}$}
    \end{choices}
\end{question}
}

\element{halliday-mc}{
\begin{question}{halliday-ch19-q42}
    A system consists of $N$ gas molecules,
        each with mass $m$. 
    Their rms speed is $v_{rms}$. 
    Their total translational kinetic energy is:
    \begin{multicols}{2}
    \begin{choices}
        \wrongchoice{$\dfrac{m}{2}\left(Nv_{rms}\right)^2$}
        \wrongchoice{$\dfrac{N}{2}\left(mv_{rms}\right)^2$}
        \wrongchoice{$\dfrac{1}{2}\left(mv_{rms}\right)^2$}
      \correctchoice{$\dfrac{1}{2}Nm v_{rms}^2$}
        \wrongchoice{$N\left[\dfrac{mv_{rms}}{2}\right]^2$}
    \end{choices}
    \end{multicols}
\end{question}
}

\element{halliday-mc}{
\begin{question}{halliday-ch19-q43}
    The average speeds $v$ and molecular diameters $d$ of five ideal gases are given below.
    The number of molecules per unit volume is the same for all of them. 
    For which is the collision rate the greatest?
    \begin{choices}
        \wrongchoice{$v = v_0$  and $d = d_0$}
        \wrongchoice{$v = 2v_0$ and $d = \dfrac{d_0}{2}$}
        \wrongchoice{$v = 3v_0$ and $d = d_0$}
      \correctchoice{$v = v_0$  and $d = 2d_0$}
        \wrongchoice{$v = 4v_0$ and $d = \dfrac{d_0}{2}$}
    \end{choices}
\end{question}
}

\element{halliday-mc}{
\begin{question}{halliday-ch19-q44}
    The internal energy of an ideal gas depends on:
   \begin{choices}
     \correctchoice{the temperature only}
       \wrongchoice{the pressure only}
       \wrongchoice{the volume only}
       \wrongchoice{the temperature and pressure only}
       \wrongchoice{temperature, pressure, and volume}
    \end{choices}
\end{question}
}

\element{halliday-mc}{
\begin{question}{halliday-ch19-q45}
    The diagram shows three isotherms for an ideal gas,
        with $T_3-T_2$ the same as $T_2-T_1$.
    It also shows five thermodynamic processes carried out on the gas. 
    \begin{center}
    \begin{tikzpicture}
        %% NOTE:
    \end{tikzpicture}
    \end{center}
    Rank the processes in order of the change in the internal energy of the gas,
        least to greatest.
   \begin{choices}
       \wrongchoice{I, II, III, IV, V}
     \correctchoice{V, then I, III, and IV tied, then II}
       \wrongchoice{V, I, then III and IV tied, then II}
       \wrongchoice{IV, V, III, I, II}
       \wrongchoice{II, I, then III, IV, and V tied}
    \end{choices}
\end{question}
}

\element{halliday-mc}{
\begin{question}{halliday-ch19-q46}
    An ideal gas of $N$ monatomic molecules is in thermal equilibrium with an ideal gas of the same number of diatomic molecules and equilibrium is maintained as the temperature is increased.
    The ratio of the changes in the internal energies $\Delta E_{dia}/\Delta E_{mon}$ is:
    \begin{multicols}{3}
   \begin{choices}
        \wrongchoice{\num{1/2}}
        \wrongchoice{\num{3/5}}
        \wrongchoice{\num{1}}
      \correctchoice{\num{5/3}}
        \wrongchoice{\num{2}}
    \end{choices}
    \end{multicols}
\end{question}
}

\element{halliday-mc}{
\begin{question}{halliday-ch19-q47}
    Two ideal gases, each consisting of $N$ monatomic molecules,
        are in thermal equilibrium with each other and equilibrium is maintained as the temperature is increased. 
    A molecule of the first gas has mass $m$ and a molecule of the second has mass $4m$. 
    The ratio of the changes in the internal energies $\Delta E_{4m}/\Delta E_{m}$ is:
    \begin{multicols}{3}
    \begin{choices}
        \wrongchoice{\num{1/4}}
        \wrongchoice{\num{1/2}}
      \correctchoice{\num{1}}
        \wrongchoice{\num{2}}
        \wrongchoice{\num{4}}
    \end{choices}
    \end{multicols}
\end{question}
}

\element{halliday-mc}{
\begin{question}{halliday-ch19-q48}
    Three gases, one consisting of monatomic molecules,
        one consisting of diatomic molecules, and one consisting of polyatomic molecules,
        are in thermal equilibrium with each other and remain in thermal equilibrium as the temperature is raised. 
    All have the same number of molecules.
    The gases with the least and greatest change in internal energy are respectively:
    \begin{choices}
        \wrongchoice{polyatomic, monatomic}
      \correctchoice{monatomic, polyatomic}
        \wrongchoice{diatomic, monatomic}
        \wrongchoice{polyatomic, diatomic}
        \wrongchoice{monatomic, diatomic}
    \end{choices}
\end{question}
}

\element{halliday-mc}{
\begin{question}{halliday-ch19-q49}
    An ideal gas of $N$ diatomic molecules has temperature $T$. 
    If the number of molecules is doubled without changing the temperature,
        the internal energy increases by:
    \begin{multicols}{2}
    \begin{choices}
        \wrongchoice{zero}
        \wrongchoice{$\dfrac{1}{2}NkT$}
        \wrongchoice{$\dfrac{3}{2}NkT$}
      \correctchoice{$\dfrac{5}{2}NkT$}
        \wrongchoice{$3N kT$}
    \end{choices}
    \end{multicols}
\end{question}
}

\element{halliday-mc}{
\begin{question}{halliday-ch19-q50}
    Both the pressure and volume of an ideal gas of diatomic molecules are doubled. 
    The ratio of the new internal energy to the old,
        both measured relative to the internal energy at \SI{0}{\kelvin}, is
    \begin{multicols}{3}
    \begin{choices}
        \wrongchoice{\num{1/4}}
        \wrongchoice{\num{1/2}}
        \wrongchoice{\num{1}}
        \wrongchoice{\num{2}}
      \correctchoice{\num{4}}
    \end{choices}
    \end{multicols}
\end{question}
}

\element{halliday-mc}{
\begin{question}{halliday-ch19-q51}
    The pressure of an ideal gas of diatomic molecules is doubled by halving the volume. 
    The ratio of the new internal energy to the old,
        both measured relative to the internal energy at \SI{0}{\kelvin}, is:
    \begin{multicols}{3}
    \begin{choices}
        \wrongchoice{\num{1/4}}
        \wrongchoice{\num{1/2}}
      \correctchoice{\num{1}}
        \wrongchoice{\num{2}}
        \wrongchoice{\num{4}}
    \end{choices}
    \end{multicols}
\end{question}
}

\element{halliday-mc}{
\begin{question}{halliday-ch19-q52}
    When work $W$ is done on an ideal gas of $N$ diatomic molecules in thermal isolation the temperature increases by:
    \begin{multicols}{3}
    \begin{choices}
        \wrongchoice{$\dfrac{W}{2Nk}$}
        \wrongchoice{$\dfrac{W}{3Nk}$}
        \wrongchoice{$\dfrac{2W}{3Nk}$}
      \correctchoice{$\dfrac{2W}{5Nk}$}
        \wrongchoice{$\dfrac{W}{Nk}$}
    \end{choices}
    \end{multicols}
\end{question}
}

\element{halliday-mc}{
\begin{question}{halliday-ch19-q53}
    When work $W$ is done on an ideal gas of diatomic molecules in thermal isolation the increase in the total rotational energy of the molecules is:
    \begin{multicols}{3}
    \begin{choices}
        \wrongchoice{zero}
        \wrongchoice{$\dfrac{W}{3}$}
        \wrongchoice{$\dfrac{2W}{3}$}
      \correctchoice{$\dfrac{2W}{5}$}
        \wrongchoice{$W$}
    \end{choices}
    \end{multicols}
\end{question}
}

\element{halliday-mc}{
\begin{question}{halliday-ch19-q54}
    When work $W$ is done on an ideal gas of diatomic molecules in thermal isolation the increase in the total translational kinetic energy of the molecules is:
    \begin{multicols}{3}
    \begin{choices}
        \wrongchoice{zero}
        \wrongchoice{$\dfrac{2W}{3}$}
        \wrongchoice{$\dfrac{2W}{5}$}
      \correctchoice{$\dfrac{3W}{5}$}
        \wrongchoice{$W$}
    \end{choices}
    \end{multicols}
\end{question}
}

\element{halliday-mc}{
\begin{question}{halliday-ch19-q55}
    The pressure of an ideal gas is doubled in an isothermal process. 
    The root-mean-square speed of the molecules:
    \begin{choices}
      \correctchoice{does not change}
        \wrongchoice{increases by a factor of $\sqrt{2}$}
        \wrongchoice{decreases by a factor of $\dfrac{1}{\sqrt{2}}$}
        \wrongchoice{increases by a factor of $2$}
        \wrongchoice{decreases by a factor of $\dfrac{1}{2}$}
    \end{choices}
\end{question}
}

\element{halliday-mc}{
\begin{question}{halliday-ch19-q56}
    The Maxwellian speed distribution provides a direct explanation of:
    \begin{choices}
        \wrongchoice{thermal expansion}
        \wrongchoice{the ideal gas law}
        \wrongchoice{heat}
      \correctchoice{evaporation}
        \wrongchoice{boiling}
    \end{choices}
\end{question}
}

\element{halliday-mc}{
\begin{question}{halliday-ch19-q57}
    For a gas at thermal equilibrium the average speed $lv$,
        the most probable speed $v_p$,
        and the root-mean-square speed $v_{rms}$ are in the order:
    \begin{choices}
        \wrongchoice{$v_p < v_{rms} < v$}
        \wrongchoice{$v_{rms} < v_p < v$}
        \wrongchoice{$v < v_{rms} < v_p$}
      \correctchoice{$v_p < v < v_{rms}$}
        \wrongchoice{$v < v_p < v_{rms}$}
    \end{choices}
\end{question}
}

\element{halliday-mc}{
\begin{question}{halliday-ch19-q58}
    The average speed of air molecules at room temperature is about:
    \begin{choices}
        \wrongchoice{zero}
        \wrongchoice{\SI{2}{\meter\per\second} (walking speed)}
        \wrongchoice{\SI{30}{\meter\per\second} (fast car)}
      \correctchoice{\SI{500}{\meter\per\second} (supersonic airplane)}
        \wrongchoice{\SI{3e8}{\meter\per\second} (speed of light)}
    \end{choices}
\end{question}
}

\element{halliday-mc}{
\begin{question}{halliday-ch19-q59}
    The root-mean-square sped of molecules in a gas is:
    \begin{choices}
        \wrongchoice{the most probable speed}
        \wrongchoice{that speed such that half the molecules are moving faster than v rms and the other half are moving slower}
        \wrongchoice{the average speed of the molecules}
        \wrongchoice{the square root of the square of the average speed}
        \correctchoice{none of the provided}
    \end{choices}
\end{question}
}

\element{halliday-mc}{
\begin{question}{halliday-ch19-q60}
    According to the Maxwellian speed distribution,
        as the temperature increases the number of molecules with speeds within a small interval near the most probable speed:
    \begin{choices}
        \wrongchoice{increases}
      \correctchoice{decreases}
        \wrongchoice{increases at high temperatures and decreases at low}
        \wrongchoice{decreases at high temperatures and increases at low}
        \wrongchoice{stays the same}
    \end{choices}
\end{question}
}

\element{halliday-mc}{
\begin{question}{halliday-ch19-q61}
    According to the Maxwellian speed distribution,
        as the temperature increases the most probable speed:
    \begin{choices}
      \correctchoice{increases}
        \wrongchoice{decreases}
        \wrongchoice{increases at high temperatures and decreases at low}
        \wrongchoice{decreases at high temperatures and increases at low}
        \wrongchoice{stays the same}
    \end{choices}
\end{question}
}

\element{halliday-mc}{
\begin{question}{halliday-ch19-q62}
    According to the Maxwellian speed distribution,
        as the temperature increases the average speed:
    \begin{choices}
      \correctchoice{increases}
        \wrongchoice{decreases}
        \wrongchoice{increases at high temperatures and decreases at low}
        \wrongchoice{decreases at high temperatures and increases at low}
        \wrongchoice{stays the same}
    \end{choices}
\end{question}
}

\element{halliday-mc}{
\begin{question}{halliday-ch19-q63}
    As the pressure in an ideal gas is increased isothermally the average molecular speed:
    \begin{choices}
        \wrongchoice{increases}
        \wrongchoice{decreases}
        \wrongchoice{increases at high temperature, decreases at low}
        \wrongchoice{decreases at high temperature, increases at low}
      \correctchoice{stays the same}
    \end{choices}
\end{question}
}

\element{halliday-mc}{
\begin{question}{halliday-ch19-q64}
    As the volume of an ideal gas is increased at constant pressure the average molecular speed:
    \begin{choices}
        \wrongchoice{increases}
        \wrongchoice{decreases}
        \wrongchoice{increases at high temperature, decreases at low}
        \wrongchoice{decreases at high temperature, increases at low}
        \wrongchoice{stays the same}
    \end{choices}
\end{question}
}

\element{halliday-mc}{
\begin{question}{halliday-ch19-q65}
    Two ideal monatomic gases are in thermal equilibrium with each other. 
    Gas $A$ is composed of molecules with mass $m$ while gas $B$ is composed of molecules with mass $4m$. 
    The ratio of the average molecular speeds $v_A/v_B$ is:
    \begin{multicols}{3}
    \begin{choices}
        \wrongchoice{\num{1/4}}
        \wrongchoice{\num{1/2}}
        \wrongchoice{\num{1}}
      \correctchoice{\num{2}}
        \wrongchoice{\num{4}}
    \end{choices}
    \end{multicols}
\end{question}
}

\element{halliday-mc}{
\begin{question}{halliday-ch19-q66}
    Ideal monatomic gas $A$ is composed of molecules with mass $m$ while ideal monatomic gas $B$ is composed of molecules with mass $4m$. 
    The average molecular speeds are the same if the ratio of the temperatures $T_A/T_B$ is:
    \begin{multicols}{3}
    \begin{choices}
      \correctchoice{\num{1/4}}
        \wrongchoice{\num{1/2}}
        \wrongchoice{\num{1}}
        \wrongchoice{\num{2}}
        \wrongchoice{\num{4}}
    \end{choices}
    \end{multicols}
\end{question}
}

\element{halliday-mc}{
\begin{question}{halliday-ch19-q67}
    Two monatomic ideal gases are in thermal equilibrium with each other. 
    Gas $A$ is composed of molecules with mass $m$ while gas $B$ is composed of molecules with mass $4m$. 
    The ratio of the average translational kinetic energies $K_A/K_B$ is:
    \begin{multicols}{3}
    \begin{choices}
        \wrongchoice{\num{1/4}}
        \wrongchoice{\num{1/2}}
      \correctchoice{\num{1}}
        \wrongchoice{\num{2}}
        \wrongchoice{\num{4}}
    \end{choices}
    \end{multicols}
\end{question}
}

\element{halliday-mc}{
\begin{question}{halliday-ch19-q68}
    Ideal monatomic gas $A$ is composed of molecules with mass $m$ while ideal monatomic gas $B$ is composed of molecules with mass $4m$. 
    The average translational kinetic energies are the same if the ratio of the temperatures $T_A/T_B$ is:
    \begin{multicols}{3}
    \begin{choices}
        \wrongchoice{\num{1/4}}
        \wrongchoice{\num{1/2}}
      \correctchoice{\num{1}}
        \wrongchoice{\num{2}}
        \wrongchoice{\num{4}}
    \end{choices}
    \end{multicols}
\end{question}
}

\element{halliday-mc}{
\begin{question}{halliday-ch19-q69}
    Which of the following change when the pressure of an ideal gas is changed isothermally?
    \begin{choices}
      \correctchoice{Mean free path}
        \wrongchoice{Root-mean-square molecular speed}
        \wrongchoice{Internal energy}
        \wrongchoice{Most probable kinetic energy}
        \wrongchoice{Average speed}
    \end{choices}
\end{question}
}

\element{halliday-mc}{
\begin{question}{halliday-ch19-q70}
    When an ideal gas undergoes a slow isothermal expansion:
    \begin{choices}
      \correctchoice{the work done by the gas is the same as the energy absorbed as heat}
        \wrongchoice{the work done by the environment is the same as the energy absorbed as heat}
        \wrongchoice{the increase in internal energy is the same as the energy absorbed as heat}
        \wrongchoice{the increase in internal energy is the same as the work done by the gas}
        \wrongchoice{the increase in internal energy is the same as the work done by the environment}
    \end{choices}
\end{question}
}

\element{halliday-mc}{
\begin{question}{halliday-ch19-q71}
    The pressure of an ideal gas is doubled during a process in which the energy given up as heat by the gas equals the work done on the gas. 
    As a result, the volume is:
    \begin{choices}
        \wrongchoice{doubled}
      \correctchoice{halved}
        \wrongchoice{unchanged}
        \wrongchoice{need more information to answer}
        \wrongchoice{nonsense; the process is impossible}
    \end{choices}
\end{question}
}

\element{halliday-mc}{
\begin{question}{halliday-ch19-q72}
    The energy absorbed as heat by an ideal gas for an isothermal process equals:
    \begin{choices}
      \correctchoice{the work done by the gas}
        \wrongchoice{the work done on the gas}
        \wrongchoice{the change in the internal energy of the gas}
        \wrongchoice{the negative of the change in internal energy of the gas}
        \wrongchoice{zero since the process is isothermal}
    \end{choices}
\end{question}
}

\element{halliday-mc}{
\begin{question}{halliday-ch19-q73}
    An ideal gas has molar specific heat $C_p$ at constant pressure. 
    When the temperature of $n$ moles is increased by $\Delta T$ the increase in the internal energy is:
    \begin{multicols}{2}
    \begin{choices}
        \wrongchoice{$nC_p \Delta T$}
        \wrongchoice{$n \left(C_p+R\right) \Delta T$}
      \correctchoice{$n \left(C_p-R\right) \Delta T$}
        \wrongchoice{$n \left(2C_p+R\right) \Delta T$}
        \wrongchoice{$n \left(2C_p-R\right) \Delta T$}
    \end{choices}
    \end{multicols}
\end{question}
}

\element{halliday-mc}{
\begin{question}{halliday-ch19-q74}
    The temperature of $n$ moles of an ideal monatomic gas is increased by $\Delta T$ at constant pressure.
    The energy $Q$ absorbed as heat,
        change $\Delta E$ int in internal energy,
        and work $W$ done by the environment are given by:
    \begin{choices}
        \wrongchoice{$Q=\dfrac{5}{2}nR\Delta T$, $\Delta E_{int}=\text{zero}$,            $W=-nR\Delta T$}
        \wrongchoice{$Q=\dfrac{3}{2}nR\Delta T$, $\Delta E_{int}=\dfrac{5}{2}nR\Delta T$, $W=-\dfrac{3}{2}nR \Delta T$}
        \wrongchoice{$Q=\dfrac{5}{2}nR\Delta T$, $\Delta E_{int}=\dfrac{5}{2}nR\Delta T$, $W=\text{zero}$}
        \wrongchoice{$Q=\dfrac{3}{2}nR\Delta T$, $\Delta E_{int}=\text{zero}$,            $W=-nR\Delta T$}
      \correctchoice{$Q=\dfrac{5}{2}nR\Delta T$, $\Delta E_{int}=\dfrac{3}{2}nR\Delta T$, $W=-nR\Delta T$}
    \end{choices}
\end{question}
}

\element{halliday-mc}{
\begin{question}{halliday-ch19-q75}
    The temperature of $n$ moles of an ideal monatomic gas is increased by $\Delta T$ at constant volume.
    The energy $Q$ absorbed as heat,
        change $\Delta E$ int in internal energy,
        and work $W$ done by the environment are given by:
    \begin{choices}
        \wrongchoice{$Q=\dfrac{5}{2}nR\Delta T$, $\Delta E_{int}=\text{zero}$,            $W=-nR\Delta T$}
      \correctchoice{$Q=\dfrac{3}{2}nR\Delta T$, $\Delta E_{int}=\dfrac{5}{2}nR\Delta T$, $W=-\dfrac{3}{2}nR \Delta T$}
        \wrongchoice{$Q=\dfrac{5}{2}nR\Delta T$, $\Delta E_{int}=\dfrac{5}{2}nR\Delta T$, $W=\text{zero}$}
        \wrongchoice{$Q=\dfrac{3}{2}nR\Delta T$, $\Delta E_{int}=\text{zero}$,            $W=-nR\Delta T$}
        \wrongchoice{$Q=\dfrac{5}{2}nR\Delta T$, $\Delta E_{int}=\dfrac{3}{2}nR\Delta T$, $W=-nR\Delta T$}
    \end{choices}
\end{question}
}

\element{halliday-mc}{
\begin{question}{halliday-ch19-q76}
    The heat capacity at constant volume of an ideal gas depends on:
    \begin{choices}
        \wrongchoice{the temperature}
        \wrongchoice{the pressure}
        \wrongchoice{the volume}
      \correctchoice{the number of molecules}
        \wrongchoice{none of the provided}
    \end{choices}
\end{question}
}

\element{halliday-mc}{
\begin{question}{halliday-ch19-q77}
    The specific heat at constant volume of an ideal gas depends on:
    \begin{choices}
        \wrongchoice{the temperature}
        \wrongchoice{the pressure}
        \wrongchoice{the volume}
        \wrongchoice{the number of molecules}
      \correctchoice{none of the provided}
    \end{choices}
\end{question}
}

\element{halliday-mc}{
\begin{question}{halliday-ch19-q78}
    The difference between the molar specific heat at constant pressure and the molar specific heat at constant volume for an ideal gas is:
    \begin{choices}
        \wrongchoice{the Boltzmann constant $k_B$}
      \correctchoice{the universal gas constant $R$}
        \wrongchoice{the Avogadro constant $N_A$}
        \wrongchoice{$k_B T$}
        \wrongchoice{$RT$}
    \end{choices}
\end{question}
}

\element{halliday-mc}{
\begin{question}{halliday-ch19-q79}
    An ideal monatomic gas has a molar specific heat $C_v$ at constant volume of:
    \begin{multicols}{3}
    \begin{choices}
        \wrongchoice{$R$}
        \wrongchoice{$\dfrac{3R}{2}$}
        \wrongchoice{$\dfrac{5R}{2}$}
        \wrongchoice{$\dfrac{7R}{2}$}
        \wrongchoice{$\dfrac{9R}{2}$}
    \end{choices}
    \end{multicols}
\end{question}
}

\element{halliday-mc}{
\begin{question}{halliday-ch19-q80}
    The specific heat $C_v$ at constant volume of a monatomic gas at low pressure is proportional to $T_n$ where the exponent $n$ is:
    \begin{multicols}{3}
    \begin{choices}
        \wrongchoice{$-1$}
      \correctchoice{$0$}
        \wrongchoice{$1$}
        \wrongchoice{$\dfrac{1}{2}$}
        \wrongchoice{$2$}
    \end{choices}
    \end{multicols}
\end{question}
}

\element{halliday-mc}{
\begin{question}{halliday-ch19-q81}
    An ideal diatomic gas has a molar specific heat at constant pressure $C_p$ of:
    \begin{multicols}{3}
    \begin{choices}
        \wrongchoice{$R$}
        \wrongchoice{$\dfrac{3R}{2}$}
        \wrongchoice{$\dfrac{5R}{2}$}
      \correctchoice{$\dfrac{7R}{2}$}
        \wrongchoice{$\dfrac{9R}{2}$}
    \end{choices}
    \end{multicols}
\end{question}
}

\element{halliday-mc}{
\begin{question}{halliday-ch19-q82}
    The specific heat of a polyatomic gas is greater than the specific heat of a monatomic gas because:
    \begin{choices}
        \wrongchoice{the polyatomic gas does more positive work when energy is absorbed as heat}
        \wrongchoice{the monatomic gas does more positive work when energy is absorbed as heat}
      \correctchoice{the energy absorbed by the polyatomic gas is split among more degrees of freedom}
        \wrongchoice{the pressure is greater in the polyatomic gas}
        \wrongchoice{a monatomic gas cannot hold as much heat}
    \end{choices}
\end{question}
}

\element{halliday-mc}{
\begin{question}{halliday-ch19-q83}
    The ratio of the specific heat of a gas at constant volume to its specific heat at constant pressure is:
    \begin{choices}
        \wrongchoice{1}
      \correctchoice{less than 1}
        \wrongchoice{more than 1}
        \wrongchoice{has units of pressure/volume}
        \wrongchoice{has units of volume/pressure}
    \end{choices}
\end{question}
}

\element{halliday-mc}{
\begin{question}{halliday-ch19-q84}
    The ratio of the specific heat of an ideal gas at constant volume to its specific heat at constant pressure is:
    \begin{choices}
        \wrongchoice{$R$}
        \wrongchoice{$\dfrac{1}{R}$}
        \wrongchoice{dependent on the temperature}
        \wrongchoice{dependent on the pressure}
      \correctchoice{different for monatomic, diatomic, and polyatomic gases}
    \end{choices}
\end{question}
}

\element{halliday-mc}{
\begin{question}{halliday-ch19-q85}
    Consider the ratios of the heat capacities $\gamma = C_p/C_v$ for the three types of ideal gases:
        monatomic, diatomic, and polyatomic.
    \begin{choices}
      \correctchoice{$gamma$ is the greatest for monatomic gases}
        \wrongchoice{$gamma$ is the greatest for polyatomic gases}
        \wrongchoice{$gamma$ is the same only for diatomic and polyatomic gases}
        \wrongchoice{$gamma$ is the same only for monatomic and diatomic gases}
        \wrongchoice{$gamma$ is the same for all three}
    \end{choices}
\end{question}
}

\element{halliday-mc}{
\begin{question}{halliday-ch19-q86}
    $T V^{\gamma-1}$ is constant for an ideal gas undergoing an adiabatic process,
        where γ is the ratio of heat capacities $C_p/C_v$. 
    This is a direct consequence of:
    \begin{choices}
        \wrongchoice{the zeroth law of thermodynamics alone}
        \wrongchoice{the zeroth law and the ideal gas equation of state}
        \wrongchoice{the first law of thermodynamics alone}
        \wrongchoice{the ideal gas equation of state alone}
      \correctchoice{the first law and the equation of state}
    \end{choices}
\end{question}
}

\element{halliday-mc}{
\begin{question}{halliday-ch19-q87}
    Monatomic, diatomic, and polyatomic ideal gases each undergo slow adiabatic expansions from the same initial volume and the same initial pressure to the same final volume. 
    The magnitude of the work done by the environment on the gas:
    \begin{choices}
      \correctchoice{is greatest for the polyatomic gas}
        \wrongchoice{is greatest for the diatomic gas}
        \wrongchoice{is greatest for the monatomic gas}
        \wrongchoice{is the same only for the diatomic and polyatomic gases}
        \wrongchoice{is the same for all three gases}
    \end{choices}
\end{question}
}

\element{halliday-mc}{
\begin{question}{halliday-ch19-q88}
    The mean free path of a gas molecule is:
    \begin{choices}
        \wrongchoice{the shortest dimension of the containing vessel}
        \wrongchoice{the cube root of the volume of the containing vessel}
        \wrongchoice{approximately the diameter of a molecule}
        \wrongchoice{average distance between adjacent molecules}
      \correctchoice{average distance a molecule travels between intermolecular collisions}
    \end{choices}
\end{question}
}

\element{halliday-mc}{
\begin{question}{halliday-ch19-q89}
    The mean free path of molecules in a gas is:
    \begin{choices}
        \wrongchoice{the average distance a molecule travels before escaping}
      \correctchoice{the average distance a molecule travels between collisions}
        \wrongchoice{the greatest distance a molecule travels between collisions}
        \wrongchoice{the shortest distance a molecule travels between collisions}
        \wrongchoice{the average distance a molecule travels before splitting apart}
    \end{choices}
\end{question}
}

\element{halliday-mc}{
\begin{question}{halliday-ch19-q90}
    The mean free path of air molecules at room temperature and atmospheric pressure is about:
    \begin{multicols}{3}
    \begin{choices}
        \wrongchoice{\SI{e-3}{\meter}}
        \wrongchoice{\SI{e-5}{\meter}}
      \correctchoice{\SI{e-7}{\meter}}
        \wrongchoice{\SI{e-9}{\meter}}
        \wrongchoice{\SI{e-11}{\meter}}
    \end{choices}
    \end{multicols}
\end{question}
}

\element{halliday-mc}{
\begin{question}{halliday-ch19-q91}
    The mean free path of molecules in a gas is proportional to:
    \begin{choices}
        \wrongchoice{the molecular cross-sectional area}
      \correctchoice{the reciprocal of the molecular cross-sectional area}
        \wrongchoice{the root-mean-square molecular speed}
        \wrongchoice{the square of the average molecular speed}
        \wrongchoice{the molar mass}
    \end{choices}
\end{question}
}

\element{halliday-mc}{
\begin{question}{halliday-ch19-q92}
    The mean free path of molecules in a gas is proportional to:
    \begin{choices}
        \wrongchoice{the molecular diameter}
        \wrongchoice{the reciprocal of the molecular diameter}
        \wrongchoice{the molecular concentration}
      \correctchoice{the reciprocal of the molecular concentration}
        \wrongchoice{the average molecular speed}
    \end{choices}
\end{question}
}

\element{halliday-mc}{
\begin{question}{halliday-ch19-q93}
    In a certain gas the molecules are \SI{5.0e-9}{\meter} apart on average,
        have a mean free path of \SI{5.0e-6}{\meter},
        and have an average speed of \SI{500}{\meter\per\second}.
    The rate at which a molecule has collisions with other molecules is about:
    \begin{multicols}{2}
    \begin{choices}
        \wrongchoice{\SI{e-11}{\per\second}}
        \wrongchoice{\SI{e-8}{\per\second}}
        \wrongchoice{\SI{1}{\per\second}}
      \correctchoice{\SI{e8}{\per\second}}
        \wrongchoice{\SI{e11}{\per\second}}
    \end{choices}
    \end{multicols}
\end{question}
}

\element{halliday-mc}{
\begin{question}{halliday-ch19-q94}
    If the temperature $T$ of an ideal gas is increased at constant pressure the mean free path:
    \begin{choices}
        \wrongchoice{decreases in proportion to $\dfrac{1}{T}$}
        \wrongchoice{decreases in proportion to $\dfrac{1}{T^2}$}
      \correctchoice{increases in proportion to $T$}
        \wrongchoice{increases in proportion to $T^2$}
        \wrongchoice{does not change}
    \end{choices}
\end{question}
}

\element{halliday-mc}{
\begin{question}{halliday-ch19-q95}
    A certain ideal gas has a temperature \SI{300}{\kelvin} and a pressure \SI{5.0e4}{\pascal}.
    The molecules have a mean free path of \SI{4.0e-7}{\meter}.
    If the temperature is raised to \SI{350}{\kelvin} and the pressure is reduced to \SI{1.0e4}{\pascal} the mean free path is then:
    \begin{multicols}{2}
    \begin{choices}
        \wrongchoice{\SI{6.9e-8}{\meter}}
        \wrongchoice{\SI{9.3e-8}{\meter}}
        \wrongchoice{\SI{3.3e-7}{\meter}}
        \wrongchoice{\SI{1.7e-6}{\meter}}
      \correctchoice{\SI{2.3e-6}{\meter}}
    \end{choices}
    \end{multicols}
\end{question}
}


\endinput


