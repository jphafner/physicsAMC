
%%--------------------------------------------------
%% Halliday: Fundamentals of Physics
%%--------------------------------------------------


%% Chapter 43: Energy from the Nucleus
%%--------------------------------------------------


%% Learning Objectives
%%--------------------------------------------------

%% 43.01: Distinguish atomic and nuclear burning, noting that in both processes energy is produced because of a reduction of mass.
%% 43.02: Define the fission process.
%% 43.03: Describe the process of a thermal neutron causing a \ce{^{235}U} nucleus to undergo fission, and explain the role of the intermediate compound nucleus.
%% 43.04: For the absorption of a thermal neutron, calculate the change in the system's mass and the energy put into the resulting oscillation of the intermediate compound nucleus.
%% 43.05: For a given fission process, calculate the $Q$ value in terms of the binding energy per nucleon.
%% 43.06: Explain the Bohr--Wheeler model for nuclear fission, including the energy barrier.
%% 43.07: Explain why thermal neutrons cannot cause \ce{^{238}U}to undergo fission.
%% 43.08: Identify the approximate amount of energy (MeV) in the fission of any high-mass nucleus to two middle-mass nuclei.
%% 43.09: Relate the rate at which nuclei fission and the rate at which energy is released.


%% Halliday Multiple Choice Questions
%%--------------------------------------------------
\element{halliday-mc}{
\begin{question}{halliday-ch43-q01}
    If the nucleus of a lead atom were broken into two identical nuclei,
        the total mass of the resultant nuclei would be:
    \begin{choices}
        \wrongchoice{the same as before}
        \wrongchoice{greater than before}
      \correctchoice{less than before}
        \wrongchoice{converted into radiation}
        \wrongchoice{converted into kinetic energy}
    \end{choices}
\end{question}
}

\element{halliday-mc}{
\begin{question}{halliday-ch43-q02}
    Consider the following energies:
    \begin{enumerate}
        \item minimum energy needed to excite a hydrogen atom
        \item energy needed to ionize a hydrogen atom
        \item energy released in 235 U fission
        \item energy needed to remove a neutron from a 12 C nucleus
    \end{enumerate}
    Rank them in order of increasing value.
    \begin{multicols}{2}
    \begin{choices}
        \wrongchoice{1, 2, 3, 4}
        \wrongchoice{1, 3, 2, 4}
      \correctchoice{1, 2, 4, 3}
        \wrongchoice{2, 1, 4, 3}
        \wrongchoice{2, 4, 1, 3}
    \end{choices}
    \end{multicols}
\end{question}
}

\element{halliday-mc}{
\begin{question}{halliday-ch43-q03}
    The binding energy per nucleon:
    \begin{choices}
      \correctchoice{increases for all fission events}
        \wrongchoice{increases for some, but not all, fission events}
        \wrongchoice{decreases for all fission events}
        \wrongchoice{decreases for some, but not all, fission events}
        \wrongchoice{remains the same for all fission events}
    \end{choices}
\end{question}
}

\element{halliday-mc}{
\begin{question}{halliday-ch43-q04}
    When uranium undergoes fission as a result of neutron bombardment,
        the energy released is due to:
    \begin{choices}
        \wrongchoice{oxidation of the uranium}
        \wrongchoice{kinetic energy of the bombarding neutrons}
        \wrongchoice{radioactivity of the uranium nucleus}
        \wrongchoice{radioactivity of the fission products}
      \correctchoice{a reduction in binding energy}
    \end{choices}
\end{question}
}

\element{halliday-mc}{
\begin{question}{halliday-ch43-q05}
    The energy supplied by a thermal neutron in a fission event is essentially its:
    \begin{choices}
        \wrongchoice{excitation energy}
      \correctchoice{binding energy}
        \wrongchoice{kinetic energy}
        \wrongchoice{rest energy}
        \wrongchoice{electric potential energy}
    \end{choices}
\end{question}
}

\element{halliday-mc}{
\begin{question}{halliday-ch43-q06}
    he barrier to fission comes about because the fragments:
    \begin{choices}
      \correctchoice{attract each other via the strong nuclear force}
        \wrongchoice{repel each other electrically}
        \wrongchoice{produce magnetic fields}
        \wrongchoice{have large masses}
        \wrongchoice{attract electrons electrically}
    \end{choices}
\end{question}
}

\element{halliday-mc}{
\begin{question}{halliday-ch43-q07}
    \ce{^{235}U} is readily made fissionable by a thermal neutron but \ce{^{238}U} is not because:
    \begin{choices}
        \wrongchoice{the neutron has a smaller binding energy in \ce{^{236}U}}
        \wrongchoice{the neutron has a smaller excitation energy in \ce{^{236}U}}
        \wrongchoice{the potential barrier for the fragments is less in \ce{^{239}U}}
      \correctchoice{the neutron binding energy is greater than the barrier height for \ce{^{236}U} and less than the barrier height for \ce{^{239}U}}
        \wrongchoice{the neutron binding energy is less than the barrier height for \ce{^{236}U} and greater than the barrier height for \ce{^{239}U}}
    \end{choices}
\end{question}
}

\element{halliday-mc}{
\begin{question}{halliday-ch43-q08}
    An explosion does not result from a small piece of \ce{^{235}U} because:
    \begin{choices}
        \wrongchoice{it does not fission}
        \wrongchoice{the neutrons released move too fast}
        \wrongchoice{\ce{^{238}U} is required}
      \correctchoice{too many neutrons escape, preventing a chain reaction from starting}
        \wrongchoice{a few neutrons must be injected to start the chain reaction}
    \end{choices}
\end{question}
}

\element{halliday-mc}{
\begin{question}{halliday-ch43-q09}
    When \ce{^{236}U} fissions the fragments are:
    \begin{choices}
        \wrongchoice{always \ce{^{140}Xe} and \ce{^{94}Sr}}
        \wrongchoice{always identical}
        \wrongchoice{never \ce{^{140}Xe} and \ce{^{94}Sr}}
        \wrongchoice{never identical}
      \correctchoice{none of the provided}
    \end{choices}
\end{question}
}

\element{halliday-mc}{
\begin{question}{halliday-ch43-q10}
    Fission fragments usually decay by emitting:
    \begin{choices}
        \wrongchoice{alpha particles}
      \correctchoice{electrons and neutrinos}
        \wrongchoice{positrons and neutrinos}
        \wrongchoice{only neutrons}
        \wrongchoice{only electrons}
    \end{choices}
\end{question}
}

\element{halliday-mc}{
\begin{question}{halliday-ch43-q11}
    When \ce{^{236}U} fissions,
        the products might be:
    \begin{choices}
        \wrongchoice{\ce{^{146}Ba}, \ce{^{89}Kr}, and a proton}
      \correctchoice{\ce{^{146}Ba}, \ce{^{89}Kr}, and a neutron}
        \wrongchoice{\ce{^{148}Cs} and \ce{^{85}Br}}
        \wrongchoice{\ce{^{133}I}, \ce{^{92}Sr}, and an alpha particle}
        \wrongchoice{two uranium nuclei}
    \end{choices}
\end{question}
}

\element{halliday-mc}{
\begin{question}{halliday-ch43-q12}
    Consider all possible fission events. 
    Which of the following statements is true?
    \begin{choices}
        \wrongchoice{Light initial fragments have more protons than neutrons and heavy initial fragments have fewer protons than neutrons}
        \wrongchoice{Heavy initial fragments have more protons than neutrons and light initial fragments have fewer protons than neutrons}
        \wrongchoice{All initial fragments have more protons than neutrons}
        \wrongchoice{All initial fragments have about the same number of protons and neutrons}
      \correctchoice{All initial fragments have more neutrons than protons}
    \end{choices}
\end{question}
}

\element{halliday-mc}{
\begin{question}{halliday-ch43-q13}
    Which one of the following represents a fission reaction that can be activated by slow neutrons?
    \begin{choices}
        \wrongchoice{\ce{^{238}U_{92} + ^{1}n_{0} -> ^{90}Kr_{36} + ^{146}Cs_{55} + ^{2}H_{1} + ^{1}n_{0}}}
      \correctchoice{\ce{^{239}Pu_{94} + ^{1}n_{0} -> ^{96}Sr_{38} + ^{141}Ba_{56} + 3^{1}n_{0}}}
        \wrongchoice{\ce{^{238}U_{92} -> ^{234}Th_{90} + ^{4}He_{2}}}
        \wrongchoice{\ce{^{3}H_{1} + ^{2}H_{1} -> ^{4}He_{2} + ^{1}n_{0}}}
        \wrongchoice{\ce{^{107}Ag_{47} + ^{1}n_{0} -> ^{108}Ag_{47} -> ^{108}Cd_{48} + ^{0}e_{-1}}}
    \end{choices}
\end{question}
}

\element{halliday-mc}{
\begin{question}{halliday-ch43-q14}
    In the uranium disintegration series:
    \begin{choices}
        \wrongchoice{the emission of a $\beta^-$ particle increases the mass number $A$ by one and decreases the atomic number $Z$ by one}
        \wrongchoice{the disintegrating element merely ejects atomic electrons}
      \correctchoice{the emission of an $\alpha$ particle decreases the mass number $A$ by four and decreases the atomic number $Z$ by two}
        \wrongchoice{the nucleus always remains unaffected}
        \wrongchoice{the series of disintegrations continues until an element having eight outermost orbital electrons is obtained}
    \end{choices}
\end{question}
}

\element{halliday-mc}{
\begin{question}{halliday-ch43-q15}
    Separation of the isotopes of uranium requires a physical, rather than chemical, method because:
    \begin{choices}
        \wrongchoice{mixing other chemicals with uranium is too dangerous}
      \correctchoice{the isotopes are chemically the same}
        \wrongchoice{the isotopes have exactly the same number of neutrons per nucleus}
        \wrongchoice{natural uranium contains only \SI{0.7}{\percent} \ce{^{235}U}}
        \wrongchoice{uranium is the heaviest element in nature}
    \end{choices}
\end{question}
}

\element{halliday-mc}{
\begin{question}{halliday-ch43-q16}
    Which one of the following is \emph{not} needed in a nuclear fission reactor?
    \begin{choices}
        \wrongchoice{Moderator}
        \wrongchoice{Fuel}
        \wrongchoice{Coolant}
        \wrongchoice{Control device}
      \correctchoice{Accelerator}
    \end{choices}
\end{question}
}

\element{halliday-mc}{
\begin{question}{halliday-ch43-q17}
    The function of the control rods in a nuclear reactor is to:
    \begin{choices}
        \wrongchoice{increase fission by slowing down the neutrons}
        \wrongchoice{decrease the energy of the neutrons without absorbing them}
        \wrongchoice{increase the ability of the neutrons to cause fission}
      \correctchoice{decrease fission by absorbing neutrons}
        \wrongchoice{provide the critical mass for the fission reaction}
    \end{choices}
\end{question}
}

\element{halliday-mc}{
\begin{question}{halliday-ch43-q18}
    A nuclear reactor is operating at a certain power level,
        with its multiplication factor adjusted to unity. 
    The control rods are now used to reduce the power output to one-half its former value. 
    After the reduction in power the multiplication factor is maintained at:
    \begin{multicols}{3}
    \begin{choices}
        \wrongchoice{\num{1/2}}
        \wrongchoice{\num{1/4}}
        \wrongchoice{\num{2}}
        \wrongchoice{\num{4}}
      \correctchoice{\num{1}}
    \end{choices}
    \end{multicols}
\end{question}
}

\element{halliday-mc}{
\begin{question}{halliday-ch43-q19}
    The purpose of a moderator in a nuclear reactor is to:
    \begin{choices}
        \wrongchoice{provide neutrons for the fission process}
      \correctchoice{slow down fast neutrons to increase the probability of capture by uranium}
        \wrongchoice{absorb dangerous gamma radiation}
        \wrongchoice{shield the reactor operator from dangerous radiation}
        \wrongchoice{none of the provided}
    \end{choices}
\end{question}
}

\element{halliday-mc}{
\begin{question}{halliday-ch43-q20}
    In a neutron-induced fission process, delayed neutrons come from:
    \begin{choices}
      \correctchoice{the fission products}
        \wrongchoice{the original nucleus just before it absorbs the neutron}
        \wrongchoice{the original nucleus just after it absorbs the neutron}
        \wrongchoice{the moderator material}
        \wrongchoice{the control rods}
    \end{choices}
\end{question}
}

\element{halliday-mc}{
\begin{question}{halliday-ch43-q21}
    In a nuclear reactor the fissionable fuel is formed into pellets rather than finely ground and the pellets are mixed with the moderator. 
    This reduces the probability of:
    \begin{choices}
      \correctchoice{non-fissioning absorption of neutrons}
        \wrongchoice{loss of neutrons through the reactor container}
        \wrongchoice{absorption of two neutrons by single fissionable nucleus}
        \wrongchoice{loss of neutrons in the control rods}
        \wrongchoice{none of the provided}
    \end{choices}
\end{question}
}

\element{halliday-mc}{
\begin{question}{halliday-ch43-q22}
    In a subcritical nuclear reactor:
    \begin{choices}
      \correctchoice{the number of fission events per unit time decreases with time}
        \wrongchoice{the number of fission events per unit time increases with time}
        \wrongchoice{each fission event produces fewer neutrons than when the reactor is critical}
        \wrongchoice{each fission event produces more neutrons than when the reactor is critical}
        \wrongchoice{none of the provided}
    \end{choices}
\end{question}
}

\element{halliday-mc}{
\begin{question}{halliday-ch43-q23}
    In the normal operation of a nuclear reactor:
    \begin{choices}
        \wrongchoice{control rods are adjusted so the reactor is subcritical}
      \correctchoice{control rods are adjusted so the reactor is critical}
        \wrongchoice{the moderating fluid is drained}
        \wrongchoice{the moderating fluid is continually recycled}
        \wrongchoice{none of the provided}
    \end{choices}
\end{question}
}

\element{halliday-mc}{
\begin{question}{halliday-ch43-q24}
    In a nuclear power plant,
        the power discharged to the environment:
    \begin{choices}
        \wrongchoice{can be made zero by proper design}
        \wrongchoice{must be less than the electrical power generated}
        \wrongchoice{must be greater than the electrical power generated}
        \wrongchoice{can be entirely recycled to produce an equal amount of electrical power}
      \correctchoice{is not any of the provided}
    \end{choices}
\end{question}
}

\element{halliday-mc}{
\begin{question}{halliday-ch43-q25}
    The binding energy per nucleon:
    \begin{choices}
      \correctchoice{increases for all fusion events}
        \wrongchoice{increases for some, but not all, fusion events}
        \wrongchoice{remains the same for some fusion events}
        \wrongchoice{decreases for all fusion events}
        \wrongchoice{decreases for some, but not all, fusion events}
    \end{choices}
\end{question}
}

\element{halliday-mc}{
\begin{question}{halliday-ch43-q26}
    To produce energy by fusion of two nuclei, the nuclei must:
    \begin{choices}
      \correctchoice{have at least several thousand electron volts of kinetic energy}
        \wrongchoice{both be above iron in mass number}
        \wrongchoice{have more neutrons than protons}
        \wrongchoice{be unstable}
        \wrongchoice{be magic number nuclei}
    \end{choices}
\end{question}
}

\element{halliday-mc}{
\begin{question}{halliday-ch43-q27}
    Which one of the following represents a fusion reaction that yields large amounts of energy?
    \begin{choices}
        \wrongchoice{\ce{^{238}U_{92} + ^{1}n_{0} -> ^{90}Kr_{36} + ^{146}Cs_{55} + ^{2}H_{1} + ^{1}n_{0}}}
        \wrongchoice{\ce{^{239}Pu_{92} + ^{1}n_{0} -> ^{96}Sr_{38} + ^{141}Ba_{56} + 3^{1}n_{0}}}
        \wrongchoice{\ce{^{238}U_{92} → ^{234}Th_{90} + ^{4}He_{2}}}
      \correctchoice{\ce{^{3}H_{1} + ^{2}H_{1} -> ^{4}He_{2} + ^{1}n_{0}}}
        \wrongchoice{\ce{^{107}Ag_{47} + ^{1}n_{0} -> ^{108}Ag_{47} -> ^{108}Cd_{48} + ^{0}e_{-1}}}
    \end{choices}
\end{question}
}

\element{halliday-mc}{
\begin{question}{halliday-ch43-q28}
    The barrier to fusion comes about because protons:
    \begin{choices}
        \wrongchoice{attract each other via the strong nuclear force}
      \correctchoice{repel each other electrically}
        \wrongchoice{produce magnetic fields}
        \wrongchoice{attract neutrons via the strong nuclear force}
        \wrongchoice{attract electrons electrically}
    \end{choices}
\end{question}
}

\element{halliday-mc}{
\begin{question}{halliday-ch43-q29}
    High temperatures are required in thermonuclear fusion so that:
    \begin{choices}
        \wrongchoice{some nuclei are moving fast enough to overcome the barrier to fusion}
        \wrongchoice{there is a high probability some nuclei will strike each other head on}
        \wrongchoice{the atoms are ionized}
        \wrongchoice{thermal expansion gives the nuclei more room}
        \wrongchoice{the uncertainty principle can be circumvented}
    \end{choices}
\end{question}
}

\element{halliday-mc}{
\begin{question}{halliday-ch43-q30}
    For a controlled nuclear fusion reaction, one needs:
    \begin{choices}
      \correctchoice{high number density $n$ and high temperature $T$}
        \wrongchoice{high number density $n$ and low temperature $T$}
        \wrongchoice{low number density $n$ and high temperature $T$}
        \wrongchoice{low number density $n$ and low temperature $T$}
        \wrongchoice{high number density $n$ and temperature $T=\SI{0}{\kelvin}$}
    \end{choices}
\end{question}
}

\element{halliday-mc}{
\begin{question}{halliday-ch43-q31}
    Most of the energy produced by the Sun is due to:
    \begin{choices}
        \wrongchoice{nuclear fission}
      \correctchoice{nuclear fusion}
        \wrongchoice{chemical reaction}
        \wrongchoice{gravitational collapse}
        \wrongchoice{induced emfs associated with the Sun's magnetic field}
    \end{choices}
\end{question}
}

\element{halliday-mc}{
\begin{question}{halliday-ch43-q32}
    Nuclear fusion in stars produces all the chemical elements with mass numbers less than:
    \begin{multicols}{3}
    \begin{choices}
        \wrongchoice{56}
        \wrongchoice{66}
        \wrongchoice{70}
        \wrongchoice{82}
        \wrongchoice{92}
    \end{choices}
    \end{multicols}
\end{question}
}

\element{halliday-mc}{
\begin{question}{halliday-ch43-q33}
    Nuclear fusion in the Sun is increasing its supply of:
    \begin{multicols}{2}
    \begin{choices}
        \wrongchoice{hydrogen}
      \correctchoice{helium}
        \wrongchoice{nucleons}
        \wrongchoice{positrons}
        \wrongchoice{neutrons}
    \end{choices}
    \end{multicols}
\end{question}
}

\element{halliday-mc}{
\begin{question}{halliday-ch43-q34}
    Which of the following chemical elements is not produced by thermonuclear fusion in stars?
    \begin{choices}
        \wrongchoice{Carbon ($Z=6$, $A\approx 12$)}
        \wrongchoice{Silicon ($Z=14$, $A\approx 28$)}
        \wrongchoice{Oxygen ($Z=8$, $A\approx 16$)}
      \correctchoice{Mercury ($Z=80$, $A\approx 200$)}
        \wrongchoice{Chromium ($Z=24$, $A\approx 52$)}
    \end{choices}
\end{question}
}

\element{halliday-mc}{
\begin{question}{halliday-ch43-q35}
    The first step of the proton-proton cycle is:
    \begin{choices}
        \wrongchoice{\ce{^{1}H + ^{1}H -> ^{2}H}}
      \correctchoice{\ce{^{1}H + ^{1}H -> ^{2}H + e + + \nu}}
        \wrongchoice{\ce{^{1}H + ^{1}H -> ^{2}H + e^{-} + \nu}}
        \wrongchoice{\ce{^{1}H + ^{1}H -> ^{2}H + \gamma}}
        \wrongchoice{\ce{^{1}H + ^{1}H -> ^{2}H + e^{-} + \nu}}
    \end{choices}
\end{question}
}

\element{halliday-mc}{
\begin{question}{halliday-ch43-q36}
    The overall proton-proton cycle is equivalent to:
    \begin{choices}
        \wrongchoice{\ce{2^{1}H -> ^{2}H}}
        \wrongchoice{\ce{4^{1}H -> ^{4}H}}
        \wrongchoice{\ce{4^{1}H -> ^{4}H + 4n}}
      \correctchoice{\ce{4^{1}H + 2e^{-} -> ^{4}He + 2\nu + 6\gamma}}
        \wrongchoice{\ce{4^{1}H + 2e^{+} -> ^{4}He + 2\nu + 3\gamma}}
    \end{choices}
\end{question}
}

\element{halliday-mc}{
\begin{question}{halliday-ch43-q37}
    The energy released in a complete proton-proton cycle is about:
    \begin{multicols}{3}
    \begin{choices}
        \wrongchoice{\SI{3}{\kilo\eV}}
        \wrongchoice{\SI{30}{\kilo\eV}}
        \wrongchoice{\SI{3}{\mega\eV}}
      \correctchoice{\SI{30}{\mega\eV}}
        \wrongchoice{\SI{300}{\mega\eV}}
    \end{choices}
    \end{multicols}
\end{question}
}

\element{halliday-mc}{
\begin{question}{halliday-ch43-q38}
    For purposes of a practical (energy producing) reaction one wants a disintegration energy $Q$ that is:
    \begin{choices}
        \wrongchoice{positive for fusion reactions and negative for fission reactions}
        \wrongchoice{negative for fusion reactions and positive for fission reactions}
        \wrongchoice{negative for both fusion and fission reactions}
      \correctchoice{positive for both fusion and fission reactions}
        \wrongchoice{as close to zero as possible for both fusion and fission reactions}
    \end{choices}
\end{question}
}

\element{halliday-mc}{
\begin{question}{halliday-ch43-q39}
    Lawson's number is \SI{e20}{\second\per\meter\cubed}.
    If the density of deuteron nuclei is \SI{2e21}{\per\meter\cubed} what should the confinement time be to achieve sustained fusion?
    \begin{multicols}{3}
    \begin{choices}
        \wrongchoice{\SI{16}{\milli\second}}
      \correctchoice{\SI{50}{\milli\second}}
        \wrongchoice{\SI{160}{\milli\second}}
        \wrongchoice{\SI{250}{\milli\second}}
        \wrongchoice{\SI{500}{\milli\second}}
    \end{choices}
    \end{multicols}
\end{question}
}

\element{halliday-mc}{
\begin{question}{halliday-ch43-q40}
    Tokamaks confine deuteron plasmas using:
    \begin{choices}
        \wrongchoice{thick steel walls}
      \correctchoice{magnetic fields}
        \wrongchoice{laser beams}
        \wrongchoice{vacuum tubes}
        \wrongchoice{electric fields}
    \end{choices}
\end{question}
}

\element{halliday-mc}{
\begin{question}{halliday-ch43-q41}
    Most magnetic confinement projects attempt:
    \begin{choices}
        \wrongchoice{proton-proton fusion}
        \wrongchoice{proton-deuteron fusion}
      \correctchoice{deuteron-deuteron fusion}
        \wrongchoice{deuteron-triton fusion}
        \wrongchoice{triton-triton fusion}
    \end{choices}
\end{question}
}

\element{halliday-mc}{
\begin{question}{halliday-ch43-q42}
    Compared to fusion in a tokamak,
        laser fusion makes use of:
    \begin{choices}
        \wrongchoice{smaller particle number densities}
      \correctchoice{greater particle number densities}
        \wrongchoice{longer confinement times}
        \wrongchoice{higher temperatures}
        \wrongchoice{lower temperatures}
    \end{choices}
\end{question}
}

\element{halliday-mc}{
\begin{question}{halliday-ch43-q43}
    Most laser fusion projects attempt:
    \begin{choices}
        \wrongchoice{proton-proton fusion}
        \wrongchoice{proton-deuteron fusion}
        \wrongchoice{deuteron-deuteron fusion}
      \correctchoice{deuteron-triton fusion}
        \wrongchoice{triton-triton fusion}
    \end{choices}
\end{question}
}

\element{halliday-mc}{
\begin{question}{halliday-ch43-q44}
    In laser fusion, the laser light is:
    \begin{choices}
        \wrongchoice{emitted by the reacting nuclei}
        \wrongchoice{used to cause transitions between nuclear energy levels}
        \wrongchoice{used to cause transitions between atomic energy levels}
        \wrongchoice{used to replace the emitted gamma rays}
      \correctchoice{used to heat the fuel pellet}
    \end{choices}
\end{question}
}


\endinput


