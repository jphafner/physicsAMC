
%%--------------------------------------------------
%% Halliday: Fundamentals of Physics
%%--------------------------------------------------


%% Chapter 20: Entropy and the
%%      second law of thermodynamics
%%--------------------------------------------------


%% Learning Objectives
%%--------------------------------------------------

%% 20.01: Identify the second law of thermodynamics: If a process occurs in a closed system, the entropy of the system increases for irreversible processes and remains constant for reversible processes; it never decreases.
%% 20.02: Identify that entropy is a state function (the value for a particular state of the system does not depend on how that state is reached).
%% 20.03: Calculate the change in entropy for a process by integrating the inverse of the temperature (in kelvins) with respect to the heat $Q$ transferred during the process.
%% 20.04: For a phase change with a constant temperature process, apply the relationship between the entropy change $\Delta S$, the total transferred heat $Q$, and the temperature $T$ (in kelvins).
%% 20.05: For a temperature change $\Delta T$ that is small relative to the temperature $T$, apply the relationship between the entropy change $\Delta S$, the transferred heat $Q$, and the average temperature $T_{avg}$ (in kelvins).
%% 20.06: For an ideal gas, apply the relationship between the entropy change $\Delta S$ and the initial and final values of the pressure and volume.
%% 20.07: Identify that if a process is an irreversible one, the integration for the entropy change must be done for a reversible process that takes the system between the same initial and final states as the irreversible process.
%% 20.08: For stretched rubber, relate the elastic force to the rate at which the rubber's entropy changes with the change in the stretching distance.


%% Halliday Multiple Choice Questions
%%--------------------------------------------------
\element{halliday-mc}{
\begin{question}{halliday-ch20-q01}
    In a reversible process the system:
    \begin{choices}
      \correctchoice{is always close to equilibrium states}
        \wrongchoice{is close to equilibrium states only at the beginning and end}
        \wrongchoice{might never be close to any equilibrium state}
        \wrongchoice{is close to equilibrium states throughout, except at the beginning and end}
        \wrongchoice{is none of the provided}
    \end{choices}
\end{question}
}

\element{halliday-mc}{
\begin{question}{halliday-ch20-q02}
    A slow (quasi-static) process is \emph{not} reversible if:
    \begin{choices}
        \wrongchoice{the temperature changes}
        \wrongchoice{energy is absorbed or emitted as heat}
        \wrongchoice{work is done on the system}
      \correctchoice{friction is present}
        \wrongchoice{the pressure changes}
    \end{choices}
\end{question}
}

\element{halliday-mc}{
\begin{question}{halliday-ch20-q03}
    The difference in entropy $\Delta S = S_B - S_A$ for two states $A$ and $B$ of a system can be computed as the integral $\mathrm{d}Q/T$ provided:
    \begin{choices}
        \wrongchoice{$A$ and $B$ are on the same adiabat}
        \wrongchoice{$A$ and $B$ have the same temperature}
      \correctchoice{a reversible path is used for the integral}
        \wrongchoice{the change in internal energy is first computed}
        \wrongchoice{the energy absorbed as heat by the system is first computed}
    \end{choices}
\end{question}
}

\element{halliday-mc}{
\begin{question}{halliday-ch20-q04}
    Possible units of entropy are:
    \begin{choices}
        \wrongchoice{joule (\si{\joule})}
      \correctchoice{joule per kelvin (\si{\joule\per\kelvin})}
        \wrongchoice{per joule (\si{\per\joule})}
        \wrongchoice{liter atmosphere (\si{\liter\atm})}
        \wrongchoice{calorie per mole (\si{\calorie\per\mole})}
    \end{choices}
\end{question}
}

\element{halliday-mc}{
\begin{question}{halliday-ch20-q05}
    Which of the following is \emph{not} a state variable?
    \begin{choices}
      \correctchoice{Work}
        \wrongchoice{Internal energy}
        \wrongchoice{Entropy}
        \wrongchoice{Temperature}
        \wrongchoice{Pressure}
    \end{choices}
\end{question}
}

\element{halliday-mc}{
\begin{question}{halliday-ch20-q06}
    The change in entropy is zero for:
    \begin{choices}
      \correctchoice{reversible adiabatic processes}
        \wrongchoice{reversible isothermal processes}
        \wrongchoice{reversible processes during which no work is done}
        \wrongchoice{reversible isobaric processes}
        \wrongchoice{all adiabatic processes}
    \end{choices}
\end{question}
}

\element{halliday-mc}{
\begin{question}{halliday-ch20-q07}
    Which of the following processes leads to a change in entropy of zero for the system undergoing the process?
    \begin{choices}
        \wrongchoice{Non-cyclic isobaric (constant pressure)}
        \wrongchoice{Non-cyclic isochoric (constant volume)}
        \wrongchoice{Non-cyclic isothermal (constant temperature)}
      \correctchoice{Any closed cycle}
        \wrongchoice{None of the provided}
    \end{choices}
\end{question}
}

\element{halliday-mc}{
\begin{question}{halliday-ch20-q08}
    Rank, from smallest to largest,
        the changes in entropy of a pan of water on a hot plate,
        as the temperature of the water:
    \begin{enumerate}
        \item goes from \SI{20}{\degreeCelsius} to \SI{30}{\degreeCelsius}
        \item goes from \SI{30}{\degreeCelsius} to \SI{40}{\degreeCelsius}
        \item goes from \SI{40}{\degreeCelsius} to \SI{45}{\degreeCelsius}
        \item goes from \SI{80}{\degreeCelsius} to \SI{85}{\degreeCelsius}
    \end{enumerate}
    \begin{choices}
        \wrongchoice{1, 2, 3, 4}
        \wrongchoice{4, 3, 2, 1}
        \wrongchoice{1 and 2 tie, then 3 and 4 tie}
        \wrongchoice{3 and 4 tie, then 1 and 2 tie}
      \correctchoice{4, 3, 2, 1}
    \end{choices}
\end{question}
}

\element{halliday-mc}{
\begin{question}{halliday-ch20-q09}
    An ideal gas expands into a vacuum in a rigid vessel. 
    As a result there is:
    \begin{choices}
      \correctchoice{a change in entropy}
        \wrongchoice{an increase of pressure}
        \wrongchoice{a change in temperature}
        \wrongchoice{a decrease of internal energy}
        \wrongchoice{a change in phase}
    \end{choices}
\end{question}
}

\element{halliday-mc}{
\begin{question}{halliday-ch20-q10}
    Consider all possible isothermal contractions of an ideal gas. 
    The change in entropy of the gas:
    \begin{choices}
        \wrongchoice{is zero for all of them}
        \wrongchoice{does not decrease for any of them}
        \wrongchoice{does not increase for any of them}
        \wrongchoice{increases for all of them}
      \correctchoice{decreases for all of them}
    \end{choices}
\end{question}
}

\element{halliday-mc}{
\begin{question}{halliday-ch20-q11}
    An ideal gas is to taken reversibly from state $i$, at temperature $T_1$,
        to any of the other states labeled I, II, III, IV, and V on the $p$-$V$ diagram below.
    All are at the same temperature $T_2$.
    \begin{center}
    \begin{tikzpicture}
        %% NOTE:
    \end{tikzpicture}
    \end{center}
    Rank the five processes according to the change in entropy of the gas,
        least to greatest.
    \begin{choices}
      \correctchoice{I, II, III, IV, V}
        \wrongchoice{V, IV, III, II, I}
        \wrongchoice{I, then II, III, IV, and V tied}
        \wrongchoice{I, II, III, and IV tied, then V}
        \wrongchoice{I and V tied, then II, III, IV}
    \end{choices}
\end{question}
}

\element{halliday-mc}{
\begin{question}{halliday-ch20-q12}
    An ideal gas, consisting of $n$ moles, undergoes a reversible isothermal process during which the volume changes from $V_i$ to $V_f$.
    The change in entropy of the thermal reservoir in contact with the gas is given by:
    \begin{choices}
        \wrongchoice{$nR\left(V_f-V_i\right)$}
        \wrongchoice{$nR\ln\left(V_f-V_i\right)$}
      \correctchoice{$nR\ln\left(\dfrac{V_i}{V_f}\right)$}
        \wrongchoice{$nR\ln\left(\dfrac{V_f}{V_i}\right)$}
        \wrongchoice{none of the provided (entropy can't be calculated for a reversible process)}
    \end{choices}
\end{question}
}

\element{halliday-mc}{
\begin{question}{halliday-ch20-q13}
    One mole of an ideal gas expands reversibly and isothermally at temperature $T$ until its volume is doubled.
    The change of entropy of this gas for this process is:
    \begin{multicols}{3}
    \begin{choices}
      \correctchoice{$R\ln 2$}
        \wrongchoice{$\dfrac{\ln 2}{T}$}
        \wrongchoice{zero}
        \wrongchoice{$RT\ln 2$}
        \wrongchoice{$2R$}
    \end{choices}
    \end{multicols}
\end{question}
}

\element{halliday-mc}{
\begin{question}{halliday-ch20-q14}
    An ideal gas, consisting of $n$ moles,
        undergoes an irreversible process in which the temperature has the same value at the beginning and end.
    If the volume changes from $V_i$ to $V_f$,
        the change in entropy of the gas is given by:
    \begin{choices}
        \wrongchoice{$nR\left(V_f-V_i\right)$}
        \wrongchoice{$nR\ln\left(V_f-V_i\right)$}
        \wrongchoice{$nR\ln\left(\dfrac{V_i}{V_f}\right)$}
      \correctchoice{$nR\ln\left(\dfrac{V_f}{V_i}\right)$}
        \wrongchoice{none of the provided (entropy can't be calculated for a reversible process)}
    \end{choices}
\end{question}
}

\element{halliday-mc}{
\begin{question}{halliday-ch20-q15}
    The temperature of $n$ moles of a gas is increased from $T_i$ to $T_f$ at constant volume.
    If the molar specific heat at constant volume is $C_V$ and is independent of temperature,
        then change in the entropy of the gas is:
    \begin{multicols}{2}
    \begin{choices}
      \correctchoice{$nC_V\ln\left(\dfrac{T_f}{T_i}\right)$}
        \wrongchoice{$nC_V\ln\left(\dfrac{T_i}{T_f}\right)$}
        \wrongchoice{$nC_V\ln\left(T_f-T_i\right)$}
        \wrongchoice{$nC_V\ln\left(1-\dfrac{T_i}{T_f}\right)$}
        \wrongchoice{$nC_V\left(T_f-T_i\right)$}
    \end{choices}
    \end{multicols}
\end{question}
}

\element{halliday-mc}{
\begin{question}{halliday-ch20-q16}
    Consider the following processes:
        The temperature of two identical gases are increased from the same initial temperature to the same final temperature.
    Reversible processes are used.
    For gas $A$ the process is carried out at constant volume while for gas $B$ it is carried out at constant pressure.
    The change in entropy:
    \begin{choices}
        \wrongchoice{is the same for $A$ and $B$}
        \wrongchoice{is greater for $A$}
      \correctchoice{is greater for $B$}
        \wrongchoice{is greater for $A$ only if the initial temperature is low}
        \wrongchoice{is greater for $A$ only if the initial temperature is high}
    \end{choices}
\end{question}
}

\element{halliday-mc}{
\begin{question}{halliday-ch20-q17}
    A hot object and a cold object are placed in thermal contact and the combination is isolated.
    They transfer energy until they reach a common temperature.
    The change $\Delta S_h$ in the entropy of the hot object,
        the change $\Delta S_c$ in the entropy of the cold object,
        and the change $\Delta S_{\text{total}}$ in the entropy of the combination are:
    \begin{choices}
        \wrongchoice{$\Delta S_h > 0$, $\Delta S_c > 0$, $\Delta S_{\text{total}} > 0$}
      \correctchoice{$\Delta S_h < 0$, $\Delta S_c > 0$, $\Delta S_{\text{total}} > 0$}
        \wrongchoice{$\Delta S_h < 0$, $\Delta S_c > 0$, $\Delta S_{\text{total}} < 0$}
        \wrongchoice{$\Delta S_h > 0$, $\Delta S_c < 0$, $\Delta S_{\text{total}} > 0$}
        \wrongchoice{$\Delta S_h > 0$, $\Delta S_c < 0$, $\Delta S_{\text{total}} < 0$}
    \end{choices}
\end{question}
}

\element{halliday-mc}{
\begin{question}{halliday-ch20-q18}
    Let $S_I$ denote the change in entropy of a sample for an irreversible process from state $A$ to state $B$.
    Let $S_R$ denote the change in entropy of the same sample for a reversible process from state $A$ to state $B$.
    Then:
    \begin{multicols}{3}
    \begin{choices}
        \wrongchoice{$S_I > S_R$}
      \correctchoice{$S_I = S_R$}
        \wrongchoice{$S_I < S_R$}
        \wrongchoice{$S_I = 0$}
        \wrongchoice{$S_R = 0$}
    \end{choices}
    \end{multicols}
\end{question}
}

\element{halliday-mc}{
\begin{question}{halliday-ch20-q19}
    For all adiabatic processes:
    \begin{choices}
        \wrongchoice{the entropy of the system does not change}
        \wrongchoice{the entropy of the system increases}
        \wrongchoice{the entropy of the system decreases}
        \wrongchoice{the entropy of the system does not increase}
      \correctchoice{the entropy of the system does not decrease}
    \end{choices}
\end{question}
}

\element{halliday-mc}{
\begin{question}{halliday-ch20-q20}
    For all reversible processes involving a system and its environment:
    \begin{choices}
        \wrongchoice{the entropy of the system does not change}
        \wrongchoice{the entropy of the system increases}
      \correctchoice{the total entropy of the system and its environment does not change}
        \wrongchoice{the total entropy of the system and its environment increases}
        \wrongchoice{none of the provided}
    \end{choices}
\end{question}
}

\element{halliday-mc}{
\begin{question}{halliday-ch20-q21}
    For all irreversible processes involving a system and its environment:
    \begin{choices}
        \wrongchoice{the entropy of the system does not change}
        \wrongchoice{the entropy of the system increases}
        \wrongchoice{the total entropy of the system and its environment does not change}
      \correctchoice{the total entropy of the system and its environment increases}
        \wrongchoice{none of the provided}
    \end{choices}
\end{question}
}

\element{halliday-mc}{
\begin{question}{halliday-ch20-q22}
    According to the second law of thermodynamics:
    \begin{choices}
      \correctchoice{heat energy cannot be completely converted to work}
        \wrongchoice{work cannot be completely converted to heat energy}
        \wrongchoice{for all cyclic processes we have $\mathrm{d}Q/T<0$}
        \wrongchoice{the reason all heat engine efficiencies are less than \SI{100}{\percent} is friction, which is unavoidable}
        \wrongchoice{all of the provided are true}
    \end{choices}
\end{question}
}

\element{halliday-mc}{
\begin{question}{halliday-ch20-q23}
    Consider the following processes:
    \begin{itemize}
        \item[I.] Energy flows as heat from a hot object to a colder object
        \item[II.] Work is done on a system and an equivalent amount of energy is rejected as heat by the system
        \item[III.] Energy is absorbed as heat by a system and an equivalent amount of work is done by the system
    \end{itemize}
    Which are never found to occur?
    %% NOTE: questionmult
    \begin{choices}
        \wrongchoice{Only I}
        \wrongchoice{Only II}
      \correctchoice{Only III}
        \wrongchoice{Only II and III}
        \wrongchoice{I, II, and III}
    \end{choices}
\end{question}
}

\element{halliday-mc}{
\begin{question}{halliday-ch20-q24}
    An inventor suggests that a house might be heated by using a refrigerator to draw energy as heat from the ground and reject energy as heat into the house.
    He claims that the energy supplied to the house as heat can exceed the work required to run the refrigerator.
    This:
    \begin{choices}
        \wrongchoice{is impossible by first law}
        \wrongchoice{is impossible by second law}
        \wrongchoice{would only work if the ground and the house were at the same temperature}
        \wrongchoice{is impossible since heat energy flows from the (hot) house to the (cold) ground}
      \correctchoice{is possible}
    \end{choices}
\end{question}
}

\element{halliday-mc}{
\begin{question}{halliday-ch20-q25}
    In a thermally insulated kitchen,
        an ordinary refrigerator is turned on and its door is left open.
    The temperature of the room:
    \begin{choices}
        \wrongchoice{remains constant according to the first law of thermodynamics}
        \wrongchoice{increases according to the first law of thermodynamics}
        \wrongchoice{decreases according to the first law of thermodynamics}
        \wrongchoice{remains constant according to the second law of thermodynamics}
        \wrongchoice{increases according to the second law of thermodynamics}
    \end{choices}
\end{question}
}

\element{halliday-mc}{
\begin{question}{halliday-ch20-q26}
    A heat engine:
    \begin{choices}
        \wrongchoice{converts heat input to an equivalent amount of work}
        \wrongchoice{converts work to an equivalent amount of heat}
      \correctchoice{takes heat in, does work, and loses energy as heat}
        \wrongchoice{uses positive work done on the system to transfer heat from a low temperature reservoir to a high temperature reservoir}
        \wrongchoice{uses positive work done on the system to transfer heat from a high temperature reservoir to a low temperature reservoir.}
    \end{choices}
\end{question}
}

\element{halliday-mc}{
\begin{question}{halliday-ch20-q27}
    A heat engine absorbs energy of magnitude $|Q_H|$ as heat from a high temperature reservoir,
        does work of magnitude $|W|$, and transfers energy of magnitude $|Q_L|$ as heat to a low temperature reservoir.
    Its efficiency is:
    \begin{multicols}{3}
    \begin{choices}
        \wrongchoice{$\dfrac{|Q_H|}{|W|}$}
        \wrongchoice{$\dfrac{|Q_L|}{|W|}$}
        \wrongchoice{$\dfrac{|Q_H|}{|Q_L|}$}
      \correctchoice{$\dfrac{|W|}{|Q_H|}$}
        \wrongchoice{$\dfrac{|W|}{|Q_L|}$}
    \end{choices}
    \end{multicols}
\end{question}
}

\element{halliday-mc}{
\begin{question}{halliday-ch20-q28}
    The temperatures $T_C$ of the cold reservoirs and the temperatures $T_H$ of the hot reservoirs for four Carnot heat engines are
    \begin{description}
        \item[engine 1:] $T_C=\SI{400}{\kelvin}$ and $T_H=\SI{500}{\kelvin}$
        \item[engine 2:] $T_C=\SI{500}{\kelvin}$ and $T_H=\SI{600}{\kelvin}$
        \item[engine 3:] $T_C=\SI{400}{\kelvin}$ and $T_H=\SI{600}{\kelvin}$
        \item[engine 4:] $T_C=\SI{600}{\kelvin}$ and $T_H=\SI{800}{\kelvin}$
    \end{description}
    Rank these engines according to their efficiencies,
        least to greatest:
    \begin{choices}
        \wrongchoice{1, 2, 3, 4}
        \wrongchoice{1 and 2 tie, then 3 and 4 tie}
        \wrongchoice{2, 1, 3, 4}
        \wrongchoice{1, 2, 4, 3}
      \correctchoice{2, 1, 4, 3}
    \end{choices}
\end{question}
}

\element{halliday-mc}{
\begin{question}{halliday-ch20-q29}
    A Carnot heat engine runs between a cold reservoir at temperature $T_C$ and a hot reservoir at temperature $T_H$.
    You want to increase its efficiency.
    Of the following, which change results in the greatest increase in efficiency?
    The value of $\Delta T$ is the same for all changes.
    \begin{choices}
        \wrongchoice{Raise the temperature of the hot reservoir by $\Delta T$}
        \wrongchoice{Raise the temperature of the cold reservoir by $\Delta T$}
        \wrongchoice{Lower the temperature of the hot reservoir by $\Delta T$}
      \correctchoice{Lower the temperature of the cold reservoir by $\Delta T$}
        \wrongchoice{Lower the temperature of the hot reservoir by $\dfrac{1}{2}\Delta T$ and raise the temperature of the cold reservoir by $\dfrac{1}{2}\Delta T$}
    \end{choices}
\end{question}
}

\element{halliday-mc}{
\begin{question}{halliday-ch20-q30}
    A certain heat engine draws \SI{500}{\calorie\per\second} from a water bath at \SI{27}{\degreeCelsius} and transfers \SI{400}{\calorie\per\second} to a reservoir at a lower temperature.
    The efficiency of this engine is:
    \begin{multicols}{3}
    \begin{choices}
        \wrongchoice{\SI{80}{\percent}}
        \wrongchoice{\SI{75}{\percent}}
        \wrongchoice{\SI{55}{\percent}}
        \wrongchoice{\SI{25}{\percent}}
      \correctchoice{\SI{20}{\percent}}
    \end{choices}
    \end{multicols}
\end{question}
}

%% NOTE: halliday-ch20-q31 is missing

\element{halliday-mc}{
\begin{question}{halliday-ch20-q32}
    A heat engine that in each cycle does positive work and loses energy as heat,
        with no heat energy input, would violate:
    \begin{choices}
        \wrongchoice{the zeroth law of thermodynamics}
      \correctchoice{the first law of thermodynamics}
        \wrongchoice{the second law of thermodynamics}
        \wrongchoice{the third law of thermodynamics}
        \wrongchoice{Newton's second law}
    \end{choices}
\end{question}
}

\element{halliday-mc}{
\begin{question}{halliday-ch20-q33}
    A cyclical process that transfers energy as heat from a high temperature reservoir to a low temperature reservoir with no other change would violate:
    \begin{choices}
        \wrongchoice{the zeroth law of thermodynamics}
        \wrongchoice{the first law of thermodynamics}
        \wrongchoice{the second law of thermodynamics}
        \wrongchoice{the third law of thermodynamics}
        \wrongchoice{none of the provided}
    \end{choices}
\end{question}
}

\element{halliday-mc}{
\begin{question}{halliday-ch20-q33}
    A cyclical process that transfers energy as heat from a high temperature reservoir to a low temperature reservoir with no other change would violate:
    \begin{choices}
        \wrongchoice{the zeroth law of thermodynamics}
        \wrongchoice{the first law of thermodynamics}
        \wrongchoice{the second law of thermodynamics}
        \wrongchoice{the third law of thermodynamics}
        \wrongchoice{none of the provided}
    \end{choices}
\end{question}
}

\element{halliday-mc}{
\begin{question}{halliday-ch20-q34}
    On a warm day a pool of water transfers energy to the air as heat and freezes.
    This is a direct violation of:
    \begin{choices}
        \wrongchoice{the zeroth law of thermodynamics}
        \wrongchoice{the first law of thermodynamics}
      \correctchoice{the second law of thermodynamics}
        \wrongchoice{the third law of thermodynamics}
        \wrongchoice{none of the provided}
    \end{choices}
\end{question}
}

\element{halliday-mc}{
\begin{question}{halliday-ch20-q35}
    A heat engine in each cycle absorbs energy of magnitude $|Q_H|$ as heat from a high temperature reservoir,
        does work of magnitude $|W|$, and then absorbs energy of magnitude $|Q_L|$ as heat from a low temperature reservoir.
    If $|W|=|Q_H|+|Q_L|$ this engine violates:
    \begin{choices}
        \wrongchoice{the zeroth law of thermodynamics}
        \wrongchoice{the first law of thermodynamics}
      \correctchoice{the second law of thermodynamics}
        \wrongchoice{the third law of thermodynamics}
        \wrongchoice{none of the provided}
    \end{choices}
\end{question}
}

\element{halliday-mc}{
\begin{question}{halliday-ch20-q36}
    A heat engine in each cycle absorbs energy from a reservoir as heat and does an equivalent amount of work,
        with no other changes.
    This engine violates:
    \begin{choices}
        \wrongchoice{the zeroth law of thermodynamics}
        \wrongchoice{the first law of thermodynamics}
      \correctchoice{the second law of thermodynamics}
        \wrongchoice{the third law of thermodynamics}
        \wrongchoice{none of the provided}
    \end{choices}
\end{question}
}

\element{halliday-mc}{
\begin{question}{halliday-ch20-q37}
    A Carnot cycle:
    \begin{choices}
      \correctchoice{is bounded by two isotherms and two adiabats on a $p$-$V$ graph}
        \wrongchoice{consists of two isothermal and two constant volume processes}
        \wrongchoice{is any four-sided process on a $p$-$V$ graph}
        \wrongchoice{only exists for an ideal gas}
        \wrongchoice{has an efficiency equal to the enclosed area on a $p$-$V$ diagram}
    \end{choices}
\end{question}
}

\element{halliday-mc}{
\begin{question}{halliday-ch20-q38}
    According to the second law of thermodynamics:
    \begin{choices}
        \wrongchoice{all heat engines have the same efficiency}
        \wrongchoice{all reversible heat engines have the same efficiency}
        \wrongchoice{the efficiency of any heat engine is independent of its working substance}
      \correctchoice{the efficiency of a Carnot engine depends only on the temperatures of the two reservoirs}
        \wrongchoice{all Carnot engines theoretically have \SI{100}{\percent} efficiency}
    \end{choices}
\end{question}
}

\element{halliday-mc}{
\begin{question}{halliday-ch20-q39}
    A Carnot heat engine operates between \SI{400}{\kelvin} and \SI{500}{\kelvin}.
    Its efficiency is:
    \begin{multicols}{3}
    \begin{choices}
      \correctchoice{\SI{20}{\percent}}
        \wrongchoice{\SI{25}{\percent}}
        \wrongchoice{\SI{44}{\percent}}
        \wrongchoice{\SI{79}{\percent}}
        \wrongchoice{\SI{100}{\percent}}
    \end{choices}
    \end{multicols}
\end{question}
}

\element{halliday-mc}{
\begin{question}{halliday-ch20-q40}
    A Carnot heat engine operates between a hot reservoir at absolute temperature $T_H$ and a cold reservoir at absolute temperature $T_C$.
    Its efficiency is:
    \begin{multicols}{3}
    \begin{choices}
        \wrongchoice{$\dfrac{T_H}{T_C}$}
        \wrongchoice{$\dfrac{T_C}{T_H}$}
        \wrongchoice{$1-\dfrac{T_H}{T_C}$}
      \correctchoice{$1-\dfrac{T_C}{T_H}$}
        \wrongchoice{\SI{100}{\percent}}
    \end{choices}
    \end{multicols}
\end{question}
}

\element{halliday-mc}{
\begin{question}{halliday-ch20-q41}
    A heat engine operates between a high temperature reservoir at $T_H$ and a low temperature reservoir at $T_L$.
    Its efficiency is given by $1-\dfrac{T_L}{T_H}$:
    \begin{choices}
        \wrongchoice{only if the working substance is an ideal gas}
      \correctchoice{only if the engine is reversible}
        \wrongchoice{only if the engine is quasi-static}
        \wrongchoice{only if the engine operates on a Stirling cycle}
        \wrongchoice{no matter what characteristics the engine has}
    \end{choices}
\end{question}
}

\element{halliday-mc}{
\begin{question}{halliday-ch20-q42}
    The maximum theoretical efficiency of a Carnot heat engine operating between reservoirs at the steam point and at room temperature is about:
    \begin{multicols}{3}
    \begin{choices}
        \wrongchoice{\SI{10}{\percent}}
      \correctchoice{\SI{20}{\percent}}
        \wrongchoice{\SI{50}{\percent}}
        \wrongchoice{\SI{80}{\percent}}
        \wrongchoice{\SI{99}{\percent}}
    \end{choices}
    \end{multicols}
\end{question}
}

\element{halliday-mc}{
\begin{question}{halliday-ch20-q43}
    An inventor claims to have a heat engine that has an efficiency of \SI{40}{\percent} when it operates between a high temperature reservoir of 150 ◦ C and a low temperature reservoir of \SI{30}{\degreeCelsius}.
    This engine:
    \begin{choices}
        \wrongchoice{must violate the zeroth law of thermodynamics}
        \wrongchoice{must violate the first law of thermodynamics}
      \correctchoice{must violate the second law of thermodynamics}
        \wrongchoice{must violate the third law of thermodynamics}
        \wrongchoice{does not necessarily violate any of the laws of thermodynamics}
    \end{choices}
\end{question}
}

\element{halliday-mc}{
\begin{question}{halliday-ch20-q44}
    A Carnot heat engine and an irreversible heat engine both operate between the same high temperature and low temperature reservoirs.
    They absorb the same energy from the high temperature reservoir as heat.
    The irreversible engine:
    \begin{choices}
        \wrongchoice{does more work}
      \correctchoice{transfers more energy to the low temperature reservoir as heat}
        \wrongchoice{has the greater efficiency}
        \wrongchoice{has the same efficiency as the reversible engine}
        \wrongchoice{cannot absorb the same energy from the high temperature reservoir as heat without violating the second law of thermodynamics}
    \end{choices}
\end{question}
}

\element{halliday-mc}{
\begin{question}{halliday-ch20-q45}
    A perfectly reversible heat pump with a coefficient of performance of 14 supplies energy to a building as heat to maintain its temperature at \SI{27}{\degreeCelsius}.
    If the pump motor does work at the rate of \SI{1}{\kilo\watt},
        at what rate does the pump supply energy to the building as heat?
    \begin{multicols}{3}
    \begin{choices}
      \correctchoice{\SI{15}{\kilo\watt}}
        \wrongchoice{\SI{3.85}{\kilo\watt}}
        \wrongchoice{\SI{1.35}{\kilo\watt}}
        \wrongchoice{\SI{1.07}{\kilo\watt}}
        \wrongchoice{\SI{1.02}{\kilo\watt}}
    \end{choices}
    \end{multicols}
\end{question}
}

\element{halliday-mc}{
\begin{question}{halliday-ch20-q46}
    A heat engine operates between \SI{200}{\kelvin} and \SI{100}{\kelvin}.
    In each cycle it takes \SI{100}{\joule} from the hot reservoir,
        loses \SI{25}{\joule} to the cold reservoir,
        and does \SI{75}{\joule} of work.
    This heat engine violates:
    \begin{choices}
        \wrongchoice{both the first and second laws of thermodynamics}
        \wrongchoice{the first law but not the second law of thermodynamics}
      \correctchoice{the second law but not the first law of thermodynamics}
        \wrongchoice{neither the first law nor the second law of thermodynamics}
        \wrongchoice{cannot answer without knowing the mechanical equivalent of heat}
    \end{choices}
\end{question}
}

\element{halliday-mc}{
\begin{question}{halliday-ch20-q47}
    A refrigerator absorbs energy of magnitude $|Q_C|$ as heat from a low temperature reservoir and transfers energy of magnitude $|Q_H|$ as heat to a high temperature reservoir.
    Work $W$ is done on the working substance.
    The coefficient of performance is given by:
    \begin{multicols}{2}
    \begin{choices}
      \correctchoice{$\dfrac{\left|Q_C\right|}{W}$}
        \wrongchoice{$\dfrac{\left|Q_H\right|}{W}$}
        \wrongchoice{$\dfrac{\left|Q_C\right|+|Q_H|}{W}$}
        \wrongchoice{$\dfrac{W}{|Q_C|}$}
        \wrongchoice{$\dfrac{W}{|Q_H|}$}
    \end{choices}
    \end{multicols}
\end{question}
}

\element{halliday-mc}{
\begin{question}{halliday-ch20-q48}
    A reversible refrigerator operates between a low temperature reservoir at $T_C$ and a high temperature reservoir at $T_H$.
    Its coefficient of performance is given by:
    \begin{multicols}{3}
    \begin{choices}
        \wrongchoice{$\dfrac{T_H-T_C}{T_C}$}
      \correctchoice{$\dfrac{T_C}{T_H-T_C}$}
        \wrongchoice{$\dfrac{T_H-T_C}{T_H}$}
        \wrongchoice{$\dfrac{T_H}{T_H-T_C}$}
        \wrongchoice{$\dfrac{T_H}{T_H+T_C}$}
    \end{choices}
    \end{multicols}
\end{question}
}

\element{halliday-mc}{
\begin{question}{halliday-ch20-q49}
    An Carnot refrigerator runs between a cold reservoir at temperature $T_C$ and a hot reservoir at temperature $T_H$. 
    You want to increase its coefficient of performance. 
    Of the following,
        which change results in the greatest increase in the coefficient? 
    The value of $\Delta T$ is the same for all changes.
    \begin{choices}
        \wrongchoice{Raise the temperature of the hot reservoir by $\Delta T$}
      \correctchoice{Raise the temperature of the cold reservoir by $\Delta T$}
        \wrongchoice{Lower the temperature of the hot reservoir by $\Delta T$}
        \wrongchoice{Lower the temperature of the cold reservoir by $\Delta T$}
        \wrongchoice{Lower the temperature of the hot reservoir by $\dfrac{1}{2}\Delta T$ and raise the temperature of the cold reservoir by $\dfrac{1}{2}\Delta T$}
    \end{choices}
\end{question}
}

\element{halliday-mc}{
\begin{question}{halliday-ch20-q50}
    For one complete cycle of a reversible heat engine,
        which of the following quantities is \emph{not} zero?
    \begin{choices}
        \wrongchoice{the change in the entropy of the working gas}
        \wrongchoice{the change in the pressure of the working gas}
        \wrongchoice{the change in the internal energy of the working gas}
      \correctchoice{the work done by the working gas}
        \wrongchoice{the change in the temperature of the working gas}
    \end{choices}
\end{question}
}

\element{halliday-mc}{
\begin{question}{halliday-ch20-q51}
    Twenty-five identical molecules are in a box. 
    Microstates are designated by identifying the molecules in the left and right halves of the box. 
    The multiplicity of the configuration with 15 molecules in the right half and 10 molecules in the left half is:
    \begin{multicols}{2}
    \begin{choices}
        \wrongchoice{\num{1.03e23}}
      \correctchoice{\num{3.27e6}}
        \wrongchoice{\num{150}}
        \wrongchoice{\num{25}}
        \wrongchoice{\num{5}}
    \end{choices}
    \end{multicols}
\end{question}
}

\element{halliday-mc}{
\begin{question}{halliday-ch20-q52}
    Twenty-five identical molecules are in a box. 
    Microstates are designated by identifying the molecules in the left and right halves of the box. 
    The Boltzmann constant is \SI{1.38e-23}{\joule\per\kelvin}.
    The entropy associated with the configuration for which 15 molecules are in the left half and 10 molecules are in the right half is:
    \begin{multicols}{2}
    \begin{choices}
      \correctchoice{\SI{2.07e-22}{\joule\per\kelvin}}
        \wrongchoice{\SI{7.31e-22}{\joule\per\kelvin}}
        \wrongchoice{\SI{4.44e-23}{\joule\per\kelvin}}
        \wrongchoice{\SI{6.91e-23}{\joule\per\kelvin}}
        \wrongchoice{\SI{2.22e-23}{\joule\per\kelvin}}
    \end{choices}
    \end{multicols}
\end{question}
}

\element{halliday-mc}{
\begin{question}{halliday-ch20-q53}
    The thermodynamic state of a gas changes from one with \num{3.8e18} microstates to one with \num{7.9e19} microstates. 
    The Boltzmann constant is \SI{1.38e-23}{\joule\per\kelvin}. 
    The change in entropy is:
    \begin{choices}
        \wrongchoice{$\Delta S = \text{zero}$}
        \wrongchoice{$\Delta S = \SI{1.04e-23}{\joule\per\kelvin}$}
        \wrongchoice{$\Delta S = \SI{-1.04e-23}{\joule\per\kelvin}$}
      \correctchoice{$\Delta S = \SI{4.19e-23}{\joule\per\kelvin}$}
        \wrongchoice{$\Delta S = \SI{-4.19e-23}{\joule\per\kelvin}$}
    \end{choices}
\end{question}
}

\element{halliday-mc}{
\begin{question}{halliday-ch20-q54}
    Let $k$ be the Boltzmann constant. 
    If the configuration of the molecules in a gas changes so that the multiplicity is reduced to one-third its previous value,
        the entropy of the gas changes by:
    \begin{multicols}{2}
    \begin{choices}
        \wrongchoice{$\Delta S = \text{zero}$}
        \wrongchoice{$\Delta S = 3k\ln 2$}
        \wrongchoice{$\Delta S = -3k\ln 2$}
      \correctchoice{$\Delta S = -k\ln 3$}
        \wrongchoice{$\Delta S = k\ln 3$}
    \end{choices}
    \end{multicols}
\end{question}
}

\element{halliday-mc}{
\begin{question}{halliday-ch20-q55}
    Let $k$ be the Boltzmann constant. 
    If the configuration of molecules in a gas changes from one with a multiplicity of $M_1$ to one with a multiplicity of $M_2$,
        then entropy changes by:
    \begin{multicols}{2}
    \begin{choices}
        \wrongchoice{$\Delta S = \text{zero}$}
        \wrongchoice{$\Delta S = k\left(M_2-M_1\right)$}
        \wrongchoice{$\Delta S = k\dfrac{M_2}{M_1}$}
        \wrongchoice{$\Delta S = k\ln\left(M_2 M_1\right)$}
      \correctchoice{$\Delta S = k\ln\left(\dfrac{M_2}{M_1}\right)$}
    \end{choices}
    \end{multicols}
\end{question}
}

\element{halliday-mc}{
\begin{question}{halliday-ch20-q56}
    Let $k$ be the Boltzmann constant. 
    If the thermodynamic state of a gas at temperature $T$ changes isothermally and reversibly to a state with three times the number of microstates as initially,
        the energy input to the gas as heat is:
    \begin{multicols}{2}
    \begin{choices}
        \wrongchoice{$Q = \text{zero}$}
        \wrongchoice{$Q = 3kT$}
        \wrongchoice{$Q = -3kT$}
      \correctchoice{$kT\ln 3$}
        \wrongchoice{$-kT\ln 3$}
    \end{choices}
    \end{multicols}
\end{question}
}


\endinput


