
%%--------------------------------------------------
%% Halliday: Fundamentals of Physics
%%--------------------------------------------------


%% Chapter 17: Waves II
%%--------------------------------------------------


%% Learning Objectives
%%--------------------------------------------------

%% 17.01: Distinguish between a longitudinal wave and a transverse wave.
%% 17.02: Explain wavefronts and rays.
%% 17.03: Apply the relationship between the speed of sound through a material, the material's bulk modulus, and the material's density.
%% 17.04: Apply the relationship between the speed of sound, the distance traveled by a sound wave, and the time required to travel that distance.


%% Halliday Multiple Choice Questions
%%--------------------------------------------------
\element{halliday-mc}{
\begin{question}{halliday-ch17-q01}
    The speed of a sound wave is determined by:
    \begin{choices}
        \wrongchoice{its amplitude}
        \wrongchoice{its intensity}
        \wrongchoice{its pitch}
        \wrongchoice{number of harmonics present}
      \correctchoice{the transmitting medium}
    \end{choices}
\end{question}
}

\element{halliday-mc}{
\begin{question}{halliday-ch17-q02}
    Take the speed of sound to be \SI{340}{\meter\per\second}. 
    A thunder clap is heard about \SI{3}{\second} after the lightning is seen. 
    The source of both light and sound is:
    \begin{choices}
        \wrongchoice{moving overhead faster than the speed of sound}
        \wrongchoice{emitting a much higher frequency than is heard}
        \wrongchoice{emitting a much lower frequency than is heard}
      \correctchoice{about \SI{1000}{\meter} away}
        \wrongchoice{much more than \SI{1000}{\meter} away}
    \end{choices}
\end{question}
}

\element{halliday-mc}{
\begin{question}{halliday-ch17-q03}
    A sound wave has a wavelength of \SI{3.0}{\meter}. 
    The distance from a compression center to the adjacent rarefaction center is:
    \begin{choices}
        \wrongchoice{\SI{0.75}{\meter}}
      \correctchoice{\SI{1.5}{\meter}}
        \wrongchoice{\SI{3.0}{\meter}}
        \wrongchoice{need to know wave speed}
        \wrongchoice{need to know frequency}
    \end{choices}
\end{question}
}

\element{halliday-mc}{
\begin{question}{halliday-ch17-q04}
    A fire whistle emits a tone of \SI{170}{\hertz}. 
    Take the speed of sound in air to be \SI{340}{\meter\per\second}. 
    The wavelength of this sound is about:
    \begin{multicols}{3}
    \begin{choices}
        \wrongchoice{\SI{0.5}{\meter}}
        \wrongchoice{\SI{1.0}{\meter}}
        \wrongchoice{\SI{2.0}{\meter}}
        \wrongchoice{\SI{3.0}{\meter}}
        \wrongchoice{\SI{340}{\meter}}
    \end{choices}
    \end{multicols}
\end{question}
}

\element{halliday-mc}{
\begin{question}{halliday-ch17-q05}
    During a time interval of exactly one period of vibration of a tuning fork,
        the emitted sound travels a distance:
    \begin{choices}
        \wrongchoice{equal to the length of the tuning fork}
        \wrongchoice{equal to twice the length of the tuning fork}
        \wrongchoice{of about \SI{330}{\meter}}
        \wrongchoice{which decreases with time}
      \correctchoice{of one wavelength in air}
    \end{choices}
\end{question}
}

\element{halliday-mc}{
\begin{question}{halliday-ch17-q06}
    At points in a sound wave where the gas is maximally compressed,
        the pressure:
    \begin{choices}
      \correctchoice{is a maximum}
        \wrongchoice{is a minimum}
        \wrongchoice{is equal to the ambient value}
        \wrongchoice{is greater than the ambient value but less than the maximum}
        \wrongchoice{is less than the ambient value but greater than the minimum}
    \end{choices}
\end{question}
}

\element{halliday-mc}{
\begin{question}{halliday-ch17-q07}
    You are listening to an ``A'' note played on a violin string. 
    Let the subscript ``s'' refer to the violin string and ``a'' refer to the air. 
    Then:
    \begin{choices}
      \correctchoice{$f_s = f_a$ but $\lambda_s = \lambda_a$}
        \wrongchoice{$f_s = f_a$ and $\lambda_s = \lambda_a$}
        \wrongchoice{$\lambda_s = \lambda_a$ but $f_s = f_a$}
        \wrongchoice{$\lambda_s = \lambda_a$ and $f_s = f_a$}
        \wrongchoice{linear density of string = volume density of air}
    \end{choices}
\end{question}
}

\element{halliday-mc}{
\begin{question}{halliday-ch17-q08}
    ``Beats'' in sound refer to:
    \begin{choices}
        \wrongchoice{interference of two waves of the same frequency}
      \correctchoice{combination of two waves of slightly different frequency}
        \wrongchoice{reversal of phase of reflected wave relative to incident wave}
        \wrongchoice{two media having slightly different sound velocities}
        \wrongchoice{effect of relative motion of source and observer}
    \end{choices}
\end{question}
}

\element{halliday-mc}{
\begin{question}{halliday-ch17-q09}
    To produce beats it is necessary to use two waves:
    \begin{choices}
        \wrongchoice{traveling in opposite directions}
      \correctchoice{of slightly different frequencies}
        \wrongchoice{of equal wavelengths}
        \wrongchoice{of equal amplitudes}
        \wrongchoice{whose ratio of frequencies is an integer}
    \end{choices}
\end{question}
}

\element{halliday-mc}{
\begin{question}{halliday-ch17-q10}
    In order for two sound waves to produce audible beats,
        it is essential that the two waves have:
    \begin{choices}
        \wrongchoice{the same amplitude}
        \wrongchoice{the same frequency}
        \wrongchoice{the same number of harmonics}
        \wrongchoice{slightly different amplitudes}
      \correctchoice{slightly different frequencies}
    \end{choices}
\end{question}
}

\element{halliday-mc}{
\begin{question}{halliday-ch17-q11}
    The largest number of beats per second will be heard from which pair of tuning forks?
    \begin{choices}
        \wrongchoice{\SI{200}{\hertz} and \SI{201}{\hertz}}
        \wrongchoice{\SI{256}{\hertz} and \SI{260}{\hertz}}
        \wrongchoice{\SI{534}{\hertz} and \SI{540}{\hertz}}
      \correctchoice{\SI{763}{\hertz} and \SI{774}{\hertz}}
        \wrongchoice{\SI{8420}{\hertz} and \SI{8422}{\hertz}}
    \end{choices}
\end{question}
}

\element{halliday-mc}{
\begin{question}{halliday-ch17-q12}
    Two stationary tuning forks (\SI{350}{\hertz} and \SI{352}{\hertz}) are struck simultaneously. 
    The resulting sound is observed to:
    \begin{choices}
      \correctchoice{beat with a frequency of \SI{2}{beats\per\second}}
        \wrongchoice{beat with a frequency of \SI{351}{beats\per\second}}
        \wrongchoice{be loud but not beat}
        \wrongchoice{be Doppler shifted by \SI{2}{\hertz}}
        \wrongchoice{have a frequency of \SI{702}{\hertz}}
    \end{choices}
\end{question}
}

\element{halliday-mc}{
\begin{question}{halliday-ch17-q13}
    When listening to tuning forks of frequency \SI{256}{\hertz} and \SI{260}{\hertz},
        one hears the following number of beats per second:
    \begin{multicols}{2}
    \begin{choices}
         \wrongchoice{zero}
         \wrongchoice{2}
       \correctchoice{4}
         \wrongchoice{8}
         \wrongchoice{258}
    \end{choices}
    \end{multicols}
\end{question}
}

\element{halliday-mc}{
\begin{question}{halliday-ch17-q14}
    Two identical tuning forks vibrate at \SI{256}{\hertz}.
    One of them is then loaded with a drop of wax,
        after which \SI{6}{beats\per\second} are heard. 
    The period of the loaded tuning fork is:
    \begin{multicols}{2}
    \begin{choices}
        \wrongchoice{\SI{0.006}{\second}}
        \wrongchoice{\SI{0.005}{\second}}
      \correctchoice{\SI{0.004}{\second}}
        \wrongchoice{\SI{0.003}{\second}}
        \wrongchoice{none of the provided}
    \end{choices}
    \end{multicols}
\end{question}
}

\element{halliday-mc}{
\begin{question}{halliday-ch17-q15}
    Which of the following properties of a sound wave determine its ``pitch''?
    \begin{choices}
        \wrongchoice{Amplitude}
        \wrongchoice{Distance from source to detector}
      \correctchoice{Frequency}
        \wrongchoice{Phase}
        \wrongchoice{Speed}
    \end{choices}
\end{question}
}

\element{halliday-mc}{
\begin{question}{halliday-ch17-q16}
    Two notes are an ``octave'' apart. 
    The ratio of their frequencies is:
    \begin{multicols}{2}
    \begin{choices}
        \wrongchoice{$8$}
        \wrongchoice{$10$}
        \wrongchoice{$\sqrt{8}$}
      \correctchoice{$2$}
        \wrongchoice{$\sqrt{2}$}
    \end{choices}
    \end{multicols}
\end{question}
}

\element{halliday-mc}{
\begin{question}{halliday-ch17-q17}
    Consider two imaginary spherical surfaces with different radii,
        each centered on a point sound source emitting spherical waves. 
    The power transmitted across the larger sphere is the power
        transmitted across the smaller and the intensity at a point
        on the larger sphere is the intensity at a point on the smaller.
    \begin{choices}
        \wrongchoice{greater than, the same as}
        \wrongchoice{greater than, greater than}
        \wrongchoice{greater than, less than}
      \correctchoice{the same as, less than}
        \wrongchoice{the same as, the same as}
    \end{choices}
\end{question}
}

\element{halliday-mc}{
\begin{question}{halliday-ch17-q18}
    The sound intensity \SI{5.0}{\meter} from a point source is \SI{0.50}{\watt\per\meter}. 
    The power output of the source is:
    \begin{multicols}{2}
    \begin{choices}
        \wrongchoice{\SI{39}{\watt}}
      \correctchoice{\SI{160}{\watt}}
        \wrongchoice{\SI{266}{\watt}}
        \wrongchoice{\SI{320}{\watt}}
        \wrongchoice{\SI{390}{\watt}}
    \end{choices}
    \end{multicols}
\end{question}
}

\element{halliday-mc}{
\begin{question}{halliday-ch17-q19}
    The standard reference sound level is about:
    \begin{choices}
      \correctchoice{the threshold of human hearing at \SI{1000}{\hertz}}
        \wrongchoice{the threshold of pain for human hearing at \SI{1000}{\hertz}}
        \wrongchoice{the level of sound produced when the \SI{1}{\kilo\gram} standard mass is dropped \SI{1}{\meter} onto a concrete floor}
        \wrongchoice{the level of normal conversation}
        \wrongchoice{the level of sound emitted by a standard \SI{60}{\hertz} tuning fork}
    \end{choices}
\end{question}
}

\element{halliday-mc}{
\begin{question}{halliday-ch17-q20}
    The intensity of sound wave $A$ is 100 times that of sound wave $B$.
    Relative to wave $B$ the sound level of wave $A$ is:
    \begin{multicols}{2}
    \begin{choices}
        \wrongchoice{\SI{-2}{\decibel}}
        \wrongchoice{\SI{+2}{\decibel}}
        \wrongchoice{\SI{+10}{\decibel}}
      \correctchoice{\SI{+20}{\decibel}}
        \wrongchoice{\SI{+100}{\decibel}}
    \end{choices}
    \end{multicols}
\end{question}
}

\element{halliday-mc}{
\begin{question}{halliday-ch17-q21}
    The intensity of a certain sound wave is \SI{6}{\micro\watt\per\centi\meter\squared}. 
    If its intensity is raised by \SI{10}{\deci\bel},
        the new intensity is:
    \begin{multicols}{2}
    \begin{choices}
      \correctchoice{\SI{60}{\micro\watt\per\centi\meter\squared}}
        \wrongchoice{\SI{6.6}{\micro\watt\per\centi\meter\squared}}
        \wrongchoice{\SI{6.06}{\micro\watt\per\centi\meter\squared}}
        \wrongchoice{\SI{600}{\micro\watt\per\centi\meter\squared}}
        \wrongchoice{\SI{12}{\micro\watt\per\centi\meter\squared}}
    \end{choices}
    \end{multicols}
\end{question}
}

\element{halliday-mc}{
\begin{question}{halliday-ch17-q22}
    If the sound level is increased by \SI{10}{\deci\bel} the intensity increases by a factor of:
    \begin{multicols}{3}
    \begin{choices}
        \wrongchoice{\num{2}}
        \wrongchoice{\num{5}}
      \correctchoice{\num{10}}
        \wrongchoice{\num{20}}
        \wrongchoice{\num{100}}
    \end{choices}
    \end{multicols}
\end{question}
}

\element{halliday-mc}{
\begin{question}{halliday-ch17-q23}
    The sound level at a point $P$ is \SI{14}{\deci\bel} below the sound level at a point \SI{1.0}{\meter} from a point source.
    The distance from the source to point $P$ is:
    \begin{multicols}{2}
    \begin{choices}
        \wrongchoice{\SI{4.0}{\centi\meter}}
        \wrongchoice{\SI{202}{\meter}}
        \wrongchoice{\SI{2.0}{\meter}}
      \correctchoice{\SI{5.0}{\meter}}
        \wrongchoice{\SI{25}{\meter}}
    \end{choices}
    \end{multicols}
\end{question}
}

\element{halliday-mc}{
\begin{question}{halliday-ch17-q24}
    To raise the pitch of a certain piano string,
        the piano tuner:
    \begin{multicols}{2}
    \begin{choices}
        \wrongchoice{loosens the string}
      \correctchoice{tightens the string}
        \wrongchoice{shortens the string}
        \wrongchoice{lengthens the string}
        \wrongchoice{removes some mass}
    \end{choices}
    \end{multicols}
\end{question}
}

\element{halliday-mc}{
\begin{question}{halliday-ch17-q25}
    A piano wire has length $L$ and mass $M$. 
    If its fundamental frequency is $f$,
        its tension is:
    \begin{multicols}{2}
    \begin{choices}
        \wrongchoice{$\dfrac{2Lf}{m}$}
        \wrongchoice{$4M Lf$}
        \wrongchoice{$\dfrac{2M f^2}{L}$}
        \wrongchoice{$\dfrac{4f^2 L^3}{M}$}
      \correctchoice{$4LM f^2$}
    \end{choices}
    \end{multicols}
\end{question}
}

\element{halliday-mc}{
\begin{question}{halliday-ch17-q26}
    If the length of a piano wire (of given density) is increased by \SI{5}{\percent},
        what approximate change in tension is necessary to keep its fundamental frequency unchanged?
    \begin{choices}
        \wrongchoice{Decrease of \SI{10}{\percent}}
        \wrongchoice{Decrease of \SI{5}{\percent}}
      \correctchoice{Increase of \SI{5}{\percent}}
        \wrongchoice{Increase of \SI{10}{\percent}}
        \wrongchoice{Increase of \SI{20}{\percent}}
    \end{choices}
\end{question}
}

\element{halliday-mc}{
\begin{question}{halliday-ch17-q27}
    A piano wire has a length of \SI{81}{\centi\meter} and a mass of \SI{2.0}{\gram}.
    If its fundamental frequency is to be \SI{394}{\hertz},
        its tension must be:
    \begin{multicols}{2}
    \begin{choices}
        \wrongchoice{\SI{0.32}{\newton}}
      \correctchoice{\SI{63}{\newton}}
        \wrongchoice{\SI{130}{\newton}}
        \wrongchoice{\SI{250}{\newton}}
        \wrongchoice{none of the provided}
    \end{choices}
    \end{multicols}
\end{question}
}

\element{halliday-mc}{
\begin{question}{halliday-ch17-q28}
    A stretched wire of length \SI{1.0}{\meter} is clamped at both ends. 
    It is plucked at its center as shown.
    \begin{center}
    \begin{tikzpicture}
        %% NOTE:
    \end{tikzpicture}
    \end{center}
    The three longest wavelengths in the wire are (in meters):
    \begin{multicols}{2}
    \begin{choices}
        \wrongchoice{4, 2, 1}
        \wrongchoice{2, 1, 0.5}
      \correctchoice{2, 0.67, 0.4}
        \wrongchoice{1, 0.5, 0.33}
        \wrongchoice{1, 0.67, 0.5}
    \end{choices}
    \end{multicols}
\end{question}
}

\element{halliday-mc}{
\begin{question}{halliday-ch17-q29}
    Two identical strings, $A$ and $B$, have nearly the same tension. 
    When they both vibrate in their fundamental resonant modes,
        there is a beat frequency of \SI{3}{\hertz}. 
    When string $B$ is tightened slightly,
        to increase the tension, the beat frequency becomes \SI{6}{\hertz}. 
    This means:
    \begin{choices}
        \wrongchoice{that before tightening $A$ had a higher frequency than $B$, but after tightening, $B$ has a higher frequency than $A$}
        \wrongchoice{that before tightening $B$ had a higher frequency than $A$, but after tightening, $A$ has a higher frequency than $B$}
        \wrongchoice{that before and after tightening $A$ has a higher frequency than $B$}
      \correctchoice{that before and after tightening $B$ has a higher frequency than $A$}
        \wrongchoice{none of the provided}
    \end{choices}
\end{question}
}

\element{halliday-mc}{
\begin{question}{halliday-ch17-q30}
    Two pipes are each open at one end and closed at the other. 
    Pipe $A$ has length $L$ and pipe $B$ has length $2L$. 
    Which harmonic of pipe $B$ matches in frequency the fundamental of pipe $A$?
    \begin{choices}
        \wrongchoice{The fundamental}
        \wrongchoice{The second}
        \wrongchoice{The third}
        \wrongchoice{The fourth}
      \correctchoice{There are none}
    \end{choices}
\end{question}
}

\element{halliday-mc}{
\begin{question}{halliday-ch17-q31}
    A column of argon is open at one end and closed at the other. 
    The shortest length of such a column that will resonate with a \SI{200}{\hertz} tuning fork is \SI{42.5}{\centi\meter}. 
    The speed of sound in argon must be:
    \begin{multicols}{2}
    \begin{choices}
        \wrongchoice{\SI{85.0}{\meter\per\second}}
        \wrongchoice{\SI{170}{\meter\per\second}}
      \correctchoice{\SI{340}{\meter\per\second}}
        \wrongchoice{\SI{470}{\meter\per\second}}
        \wrongchoice{\SI{940}{\meter\per\second}}
    \end{choices}
    \end{multicols}
\end{question}
}

\element{halliday-mc}{
\begin{question}{halliday-ch17-q32}
    A tuning fork produces sound waves of wavelength $\lambda$ in air. 
    This sound is used to cause resonance in an air column,
        closed at one end and open at the other. 
    The length of this column \emph{cannot} be:
    \begin{multicols}{2}
    \begin{choices}
        \wrongchoice{$\dfrac{\lambda}{4}$}
      \correctchoice{$\dfrac{2\lambda}{4}$}
        \wrongchoice{$\dfrac{3\lambda}{4}$}
        \wrongchoice{$\dfrac{5\lambda}{4}$}
        \wrongchoice{$\dfrac{7\lambda}{4}$}
    \end{choices}
    \end{multicols}
\end{question}
}

\element{halliday-mc}{
\begin{question}{halliday-ch17-q33}
    A \SI{1024}{\hertz} tuning fork is used to obtain a series of resonance levels in a gas column of variable length,
        with one end closed and the other open.
    The length of the column changes by \SI{20}{\centi\meter} from resonance to resonance. 
    From this data, the speed of sound in this gas is:
    \begin{multicols}{2}
    \begin{choices}
        \wrongchoice{\SI{20}{\centi\meter\per\second}}
        \wrongchoice{\SI{51}{\centi\meter\per\second}}
        \wrongchoice{\SI{102}{\centi\meter\per\second}}
        \wrongchoice{\SI{205}{\meter\per\second}}
      \correctchoice{\SI{410}{\meter\per\second}}
    \end{choices}
    \end{multicols}
\end{question}
}

\element{halliday-mc}{
\begin{question}{halliday-ch17-q34}
    A vibrating tuning fork is held over a water column with one end closed and the other open.
    As the water level is allowed to fall,
        a loud sound is heard for water levels separated by \SI{17}{\centi\meter}.
    If the speed of sound in air is \SI{340}{\meter\per\second},
        the frequency of the tuning fork is:
    \begin{multicols}{2}
    \begin{choices}
        \wrongchoice{\SI{500}{\hertz}}
      \correctchoice{\SI{1000}{\hertz}}
        \wrongchoice{\SI{2000}{\hertz}}
        \wrongchoice{\SI{5780}{\hertz}}
        \wrongchoice{\SI{578 000}{\hertz}}
    \end{choices}
    \end{multicols}
\end{question}
}

\element{halliday-mc}{
\begin{question}{halliday-ch17-q35}
    An organ pipe with one end open and the other closed is operating at one of its resonant frequencies. 
    The open and closed ends are respectively:
    \begin{choices}
        \wrongchoice{pressure node, pressure node}
      \correctchoice{pressure node, displacement node}
        \wrongchoice{displacement antinode, pressure node}
        \wrongchoice{displacement node, displacement node}
        \wrongchoice{pressure antinode, pressure node}
    \end{choices}
\end{question}
}

\element{halliday-mc}{
\begin{question}{halliday-ch17-q36}
    An organ pipe with one end closed and the other open has length $L$. 
    Its fundamental frequency is proportional to:
    \begin{multicols}{3}
    \begin{choices}
        \wrongchoice{$L$}
      \correctchoice{$\dfrac{1}{L}$}
        \wrongchoice{$\dfrac{1}{L^2}$}
        \wrongchoice{$L^2$}
        \wrongchoice{$\sqrt{L}$}
    \end{choices}
    \end{multicols}
\end{question}
}

\element{halliday-mc}{
\begin{question}{halliday-ch17-q37}
    Five organ pipes are described below. 
    Which one has the highest frequency fundamental?
    \begin{choices}
        \wrongchoice{A \SI{2.3}{\meter} pipe with one end open and the other closed}
        \wrongchoice{A \SI{3.3}{\meter} pipe with one end open and the other closed}
      \correctchoice{A \SI{1.6}{\meter} pipe with both ends open}
        \wrongchoice{A \SI{3.0}{\meter} pipe with both ends open}
        \wrongchoice{A pipe in which the displacement nodes are \SI{5}{\meter} apart}
    \end{choices}
\end{question}
}

\element{halliday-mc}{
\begin{question}{halliday-ch17-q38}
    If the speed of sound is \SI{340}{\meter\per\second},
        the length of the shortest closed pipe that resonates at \SI{218}{\hertz} is:
    \begin{multicols}{3}
    \begin{choices}
        \wrongchoice{\SI{23}{\centi\meter}}
        \wrongchoice{\SI{17}{\centi\meter}}
        \wrongchoice{\SI{39}{\centi\meter}}
        \wrongchoice{\SI{78}{\centi\meter}}
        \wrongchoice{\SI{1.56}{\centi\meter}}
    \end{choices}
    \end{multicols}
\end{question}
}

\element{halliday-mc}{
\begin{question}{halliday-ch17-q39}
    The lowest tone produced by a certain organ comes from a \SI{3.0}{\meter} pipe with both ends open. 
    If the speed of sound is \SI{340}{\meter\per\second},
        the frequency of this tone is approximately:
    \begin{multicols}{3}
    \begin{choices}
        \wrongchoice{\SI{7}{\hertz}}
        \wrongchoice{\SI{14}{\hertz}}
        \wrongchoice{\SI{28}{\hertz}}
      \correctchoice{\SI{57}{\hertz}}
        \wrongchoice{\SI{70}{\hertz}}
    \end{choices}
    \end{multicols}
\end{question}
}

\element{halliday-mc}{
\begin{question}{halliday-ch17-q40}
    The speed of sound in air is \SI{340}{\meter\per\second}. 
    The length of the shortest pipe, closed at one end,
        that will respond to a \SI{512}{\hertz} tuning fork is approximately:
    \begin{multicols}{3}
    \begin{choices}
        \wrongchoice{\SI{4.2}{\centi\meter}}
        \wrongchoice{\SI{9.4}{\centi\meter}}
      \correctchoice{\SI{17}{\centi\meter}}
        \wrongchoice{\SI{33}{\centi\meter}}
        \wrongchoice{\SI{66}{\centi\meter}}
    \end{choices}
    \end{multicols}
\end{question}
}

\element{halliday-mc}{
\begin{question}{halliday-ch17-q41}
    If the speed of sound is \SI{340}{\meter\per\second},
        the two lowest frequencies of an \SI{0.5}{\meter} organ pipe,
        closed at one end, are approximately:
    \begin{choices}
        \wrongchoice{\SI{170}{\hertz} and \SI{340}{\hertz}}
      \correctchoice{\SI{170}{\hertz} and \SI{510}{\hertz}}
        \wrongchoice{\SI{340}{\hertz} and \SI{680}{\hertz}}
        \wrongchoice{\SI{340}{\hertz} and \SI{1020}{\hertz}}
        \wrongchoice{\SI{57}{\hertz}  and \SI{170}{\hertz}}
    \end{choices}
\end{question}
}

\element{halliday-mc}{
\begin{question}{halliday-ch17-q42}
    Organ pipe $Y$ (open at both ends) is half as long as organ pipe $X$ (open at one end) as shown.
    \begin{center}
    \begin{tikzpicture}
        %% NOTE:
    \end{tikzpicture}
    \end{center}
    The ratio of their fundamental frequencies $f_X:f_Y$ is:
    \begin{multicols}{3}
    \begin{choices}
      \correctchoice{$1:1$}
        \wrongchoice{$1:2$}
        \wrongchoice{$2:1$}
        \wrongchoice{$1:4$}
        \wrongchoice{$4:1$}
    \end{choices}
    \end{multicols}
\end{question}
}

\element{halliday-mc}{
\begin{question}{halliday-ch17-q43}
    A \SI{200}{\centi\meter} organ pipe with one end open is in resonance with a sound wave of wavelength \SI{270}{\centi\meter}. 
    The pipe is operating in its:
    \begin{choices}
        \wrongchoice{fundamental frequency}
      \correctchoice{second harmonic}
        \wrongchoice{third harmonic}
        \wrongchoice{fourth harmonic}
        \wrongchoice{fifth harmonic}
    \end{choices}
\end{question}
}

\element{halliday-mc}{
\begin{question}{halliday-ch17-q44}
    An organ pipe with both ends open is \SI{0.85}{\meter} long. 
    Assuming that the speed of sound is \SI{340}{\meter\per\second},
        the frequency of the third harmonic of this pipe is:
    \begin{multicols}{2}
    \begin{choices}
        \wrongchoice{\SI{200}{\hertz}}
        \wrongchoice{\SI{300}{\hertz}}
        \wrongchoice{\SI{400}{\hertz}}
      \correctchoice{\SI{600}{\hertz}}
        \wrongchoice{none of the provided}
    \end{choices}
    \end{multicols}
\end{question}
}

\element{halliday-mc}{
\begin{question}{halliday-ch17-q45}
    The ``A'' on a trumpet and a clarinet have the same pitch,
        but the two are clearly distinguishable. 
    Which property is most important in enabling one to distinguish between these two instruments?
    \begin{choices}
        \wrongchoice{Intensity}
        \wrongchoice{Fundamental frequency}
        \wrongchoice{Displacement amplitude}
        \wrongchoice{Pressure amplitude}
      \correctchoice{Harmonic content}
    \end{choices}
\end{question}
}

\element{halliday-mc}{
\begin{question}{halliday-ch17-q46}
    The valves of a trumpet and the slide of a trombone are for the purpose of:
    \begin{choices}
        \wrongchoice{playing short (staccato) notes}
        \wrongchoice{tuning the instruments}
        \wrongchoice{changing the harmonic content}
      \correctchoice{changing the length of the air column}
        \wrongchoice{producing gradations in loudness}
    \end{choices}
\end{question}
}

\element{halliday-mc}{
\begin{question}{halliday-ch17-q47}
    Two small identical speakers are connected (in phase) to the same source. 
    The speakers are \SI{3}{\meter} apart and at ear level. 
    An observer stands at $X$, \SI{4}{\meter} in front of one speaker as shown. 
    \begin{center}
    \begin{tikzpicture}
        %% NOTE:
    \end{tikzpicture}
    \end{center}
    If the amplitudes are not changed,
        the sound he hears will be most intense if the wavelength is:
    \begin{multicols}{2}
    \begin{choices}
      \correctchoice{\SI{1}{\meter}}
        \wrongchoice{\SI{2}{\meter}}
        \wrongchoice{\SI{3}{\meter}}
        \wrongchoice{\SI{4}{\meter}}
        \wrongchoice{\SI{5}{\meter}}
    \end{choices}
    \end{multicols}
\end{question}
}

%% NOTE: is this really different?
\element{halliday-mc}{
\begin{question}{halliday-ch17-q48}
    Two small identical speakers are connected (in phase) to the same source. 
    The speakers are \SI{3}{\meter} apart and at ear level. 
    An observer stands at $X$, \SI{4}{\meter} in front of one speaker as shown. 
    \begin{center}
    \begin{tikzpicture}
        %% NOTE:
    \end{tikzpicture}
    \end{center}
    The sound she hears will be most intense if the wavelength is:
    \begin{multicols}{2}
    \begin{choices}
        \wrongchoice{\SI{5}{\meter}}
        \wrongchoice{\SI{4}{\meter}}
        \wrongchoice{\SI{3}{\meter}}
        \wrongchoice{\SI{2}{\meter}}
      \correctchoice{\SI{1}{\meter}}
    \end{choices}
    \end{multicols}
\end{question}
}

\element{halliday-mc}{
\begin{question}{halliday-ch17-q49}
    The rise in pitch of an approaching siren is an apparent increase in its:
    \begin{choices}
        \wrongchoice{speed}
        \wrongchoice{amplitude}
      \correctchoice{frequency}
        \wrongchoice{wavelength}
        \wrongchoice{number of harmonics}
    \end{choices}
\end{question}
}

\element{halliday-mc}{
\begin{question}{halliday-ch17-q50}
    The diagram shows four situations in which a source of sound $S$ with variable frequency and a detector $D$ are either moving or stationary. 
    The arrows indicate the directions of motion. 
    \begin{center}
    \begin{tikzpicture}
        %% NOTE: 2x2 tikz
    \end{tikzpicture}
    \end{center}
    The speeds are all the same. 
    Detector 3 is stationary. 
    The frequency detected is the same. 
    Rank the situations according to the frequency of the source,
        lowest to highest.
    \begin{multicols}{2}
    \begin{choices}
        \wrongchoice{1, 2, 3, 4}
      \correctchoice{4, 3, 2, 1}
        \wrongchoice{1, 3, 4, 2}
        \wrongchoice{2, 1, 2, 3}
        \wrongchoice{None of the provided}
    \end{choices}
    \end{multicols}
\end{question}
}

\element{halliday-mc}{
\begin{question}{halliday-ch17-q51}
    A stationary source generates \SI{5.0}{\hertz} water waves whose speed is \SI{2.0}{\meter\per\second}.
    A boat is approaching the source at \SI{10}{\meter\per\second}.
    The frequency of these waves,
        as observed by a person in the boat, is:
    \begin{multicols}{2}
    \begin{choices}
        \wrongchoice{\SI{5.0}{\hertz}}
        \wrongchoice{\SI{15}{\hertz}}
        \wrongchoice{\SI{20}{\hertz}}
        \wrongchoice{\SI{25}{\hertz}}
      \correctchoice{\SI{30}{\hertz}}
    \end{choices}
    \end{multicols}
\end{question}
}

\element{halliday-mc}{
\begin{question}{halliday-ch17-q52}
    A stationary source $S$ generates circular outgoing waves on a lake. 
    The wave speed is \SI{5.0}{\meter\per\second} and the crest-to-crest distance is \SI{2.0}{\meter}.
    A person in a motor boat heads directly toward $S$ at \SI{3.0}{\meter\per\second}.
    To this person, the frequency of these waves is:
    \begin{multicols}{2}
    \begin{choices}
        \wrongchoice{\SI{1.0}{\hertz}}
        \wrongchoice{\SI{1.5}{\hertz}}
        \wrongchoice{\SI{2.0}{\hertz}}
      \correctchoice{\SI{4.0}{\hertz}}
        \wrongchoice{\SI{8.0}{\hertz}}
    \end{choices}
    \end{multicols}
\end{question}
}

\element{halliday-mc}{
\begin{question}{halliday-ch17-q53}
    A stationary source emits a sound wave of frequency $f$. 
    If it were possible for a man to travel toward the source at the speed of sound,
        he would observe the emitted sound to have a frequency of:
    \begin{multicols}{2}
    \begin{choices}
        \wrongchoice{zero}
        \wrongchoice{$\dfrac{f}{2}$}
        \wrongchoice{$\dfrac{2f}{3}$}
      \correctchoice{$2f$}
        \wrongchoice{infinity ($\infty$)}
    \end{choices}
    \end{multicols}
\end{question}
}

\element{halliday-mc}{
\begin{question}{halliday-ch17-q54}
    A source emits sound with a frequency of \SI{1000}{\hertz}.
    It and an observer are moving in the same direction with the same speed, \SI{100}{\meter\per\second}. 
    If the speed of sound is \SI{340}{\meter\per\second},
        the observer hears sound with a frequency of:
    \begin{multicols}{2}
    \begin{choices}
        \wrongchoice{\SI{294}{\hertz}}
        \wrongchoice{\SI{545}{\hertz}}
      \correctchoice{\SI{1000}{\hertz}}
        \wrongchoice{\SI{1830}{\hertz}}
        \wrongchoice{\SI{3400}{\hertz}}
    \end{choices}
    \end{multicols}
\end{question}
}

\element{halliday-mc}{
\begin{question}{halliday-ch17-q55}
    A source emits sound with a frequency of \SI{1000}{\hertz}.
    It and an observer are moving toward each other,
        each with a speed of \SI{100}{\meter\per\second}.
    If the speed of sound is \SI{340}{\meter\per\second},
        the observer hears sound with a frequency of:
    \begin{multicols}{2}
    \begin{choices}
        \wrongchoice{\SI{294}{\hertz}}
        \wrongchoice{\SI{545}{\hertz}}
        \wrongchoice{\SI{1000}{\hertz}}
      \correctchoice{\SI{1830}{\hertz}}
        \wrongchoice{\SI{3400}{\hertz}}
    \end{choices}
    \end{multicols}
\end{question}
}

\element{halliday-mc}{
\begin{question}{halliday-ch17-q56}
    A source emits sound with a frequency of \SI{1000}{\hertz}. 
    It is moving at \SI{20}{\meter\per\second} toward a stationary reflecting wall. 
    If the speed of sound is \SI{340}{\meter\per\second} an observer at rest directly behind the source hears a beat frequency of:
    \begin{multicols}{2}
    \begin{choices}
        \wrongchoice{\SI{11}{\hertz}}
        \wrongchoice{\SI{86}{\hertz}}
        \wrongchoice{\SI{97}{\hertz}}
      \correctchoice{\SI{118}{\hertz}}
        \wrongchoice{\SI{183}{\hertz}}
    \end{choices}
    \end{multicols}
\end{question}
}

\element{halliday-mc}{
\begin{question}{halliday-ch17-q57}
    In each of the following two situations a source emits sound with a frequency of \SI{1000}{\hertz}. 
    In situation I the source is moving at \SI{100}{\meter\per\second} toward an observer at rest. 
    In situation II the observer is moving at \SI{100}{\meter\per\second} toward the source,
        which is stationary. 
    The speed of sound is \SI{340}{\meter\per\second}. 
    The frequencies heard by the observers in the two situations are:
    \begin{choices}
      \correctchoice{I: \SI{1417}{\hertz}; II: \SI{1294}{\hertz}}
        \wrongchoice{I: \SI{1417}{\hertz}; II: \SI{1417}{\hertz}}
        \wrongchoice{I: \SI{1294}{\hertz}; II: \SI{1294}{\hertz}}
        \wrongchoice{I: \SI{773}{\hertz}; II: \SI{706}{\hertz}}
        \wrongchoice{I: \SI{773}{\hertz}; II: \SI{773}{\hertz}}
    \end{choices}
\end{question}
}

\element{halliday-mc}{
\begin{question}{halliday-ch17-q58}
    The Doppler shift formula for the frequency detected is
    \begin{equation*}
        f = f^{\prime} \dfrac{v \pm v_D}{v \pm v_S}
    \end{equation*}
    where $f$ is the frequency emitted, $v$ is the speed of sound,
    $v_D$ is the speed of the detector, and $v_S$ is the speed of the source. 
    Suppose the source is traveling at \SI{5}{\meter\per\second} away from the detector,
        the detector is traveling at \SI{7}{\meter\per\second} toward the source,
        and there is a \SI{3}{\meter\per\second} wind blowing from the source toward the detector. 
    The values that should be substituted into the Doppler shift equation are:
    \begin{choices}
        \wrongchoice{$v_D = \SI{7}{\meter\per\second}$ and $v_S  = \SI{5}{\meter\per\second}$}
      \correctchoice{$v_D = \SI{10}{\meter\per\second}$ and $v_S = \SI{8}{\meter\per\second}$}
        \wrongchoice{$v_D = \SI{4}{\meter\per\second}$ and $v_S  = \SI{2}{\meter\per\second}$}
        \wrongchoice{$v_D = \SI{10}{\meter\per\second}$ and $v_S = \SI{2}{\meter\per\second}$}
        \wrongchoice{$v_D = \SI{4}{\meter\per\second}$ and $v_S  = \SI{8}{\meter\per\second}$}
    \end{choices}
\end{question}
}

\element{halliday-mc}{
\begin{question}{halliday-ch17-q59}
    A plane produces a sonic boom only when:
    \begin{choices}
        \wrongchoice{it emits sound waves of very long wavelength}
        \wrongchoice{it emits sound waves of high frequency}
        \wrongchoice{it flys at high altitudes}
        \wrongchoice{it flys on a curved path}
      \correctchoice{it flys faster than the speed of sound}
    \end{choices}
\end{question}
}

\element{halliday-mc}{
\begin{question}{halliday-ch17-q60}
    If the speed of sound is \SI{340}{\meter\per\second} a plane flying at 
        \SI{400}{\meter\per\second} creates a conical shock wave with an apex half angle of:
    \begin{multicols}{2}
    \begin{choices}
        \wrongchoice{\ang{0} (no shock wave)}
        \wrongchoice{\ang{32}}
        \wrongchoice{\ang{40}}
        \wrongchoice{\ang{50}}
      \correctchoice{\ang{58}}
    \end{choices}
    \end{multicols}
\end{question}
}

\element{halliday-mc}{
\begin{question}{halliday-ch17-q61}
    The speed of sound is \SI{340}{\meter\per\second}. 
    A plane flys horizontally at an altitude of \SI{10 000}{\meter} and a speed of \SI{400}{\meter\per\second}.
    When an observer on the ground hears the sonic boom the horizontal distance from the point on its path directly above the observer to the plane is:
    \begin{multicols}{2}
    \begin{choices}
        \wrongchoice{\SI{5800}{\meter}}
      \correctchoice{\SI{6200}{\meter}}
        \wrongchoice{\SI{8400}{\meter}}
        \wrongchoice{\SI{12 000}{\meter}}
        \wrongchoice{\SI{16 000}{\meter}}
    \end{choices}
    \end{multicols}
\end{question}
}


\endinput


