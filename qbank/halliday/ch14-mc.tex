
%%--------------------------------------------------
%% Halliday: Fundamentals of Physics
%%--------------------------------------------------


%% Chapter 14: Fluids
%%--------------------------------------------------


%% Learning Objectives
%%--------------------------------------------------

%% 14.01: Distinguish fluids from solids.
%% 14.02: When mass is uniformly distributed, relate density to mass and volume.
%% 14.03: Apply the relationship between hydrostatic pressure, force, and the surface area over which that force acts.


%% Halliday Multiple Choice Questions
%%--------------------------------------------------
\element{halliday-mc}{
\begin{question}{halliday-ch14-q01}
    All fluids are:
    \begin{multicols}{2}
    \begin{choices}
        \wrongchoice{gases}
        \wrongchoice{liquids}
      \correctchoice{gases or liquids}
        \wrongchoice{non-metallic}
        \wrongchoice{transparent}
    \end{choices}
    \end{multicols}
\end{question}
}

\element{halliday-mc}{
\begin{question}{halliday-ch14-q02}
    Gases may be distinguished from other forms of matter by their:
    \begin{choices}
        \wrongchoice{lack of color}
        \wrongchoice{small atomic weights}
      \correctchoice{inability to form free surfaces}
        \wrongchoice{ability to flow}
        \wrongchoice{ability to exert a buoyant force}
    \end{choices}
\end{question}
}

\element{halliday-mc}{
\begin{question}{halliday-ch14-q03}
    One pascal (\SI{1}{\pascal}) is:
    \begin{choices}
        \wrongchoice{One newton per meter (\SI{1}{\newton\per\meter})}
        \wrongchoice{One meter per newton (\SI{1}{\meter\per\newton})}
        \wrongchoice{One kilogram per meter per second (\SI{1}{\kilo\gram\per\meter\per\second})}
      \correctchoice{One kilogram per meter per second (\SI{1}{\kilo\gram\per\meter\per\second\squared})}
        \wrongchoice{One newton per meter per second (\SI{1}{\newton\per\meter\per\second})}
    \end{choices}
\end{question}
}

\element{halliday-mc}{
\begin{question}{halliday-ch14-q04}
    Mercury is a convenient liquid to use in a barometer because:
    \begin{choices}
        \wrongchoice{it is a metal}
        \wrongchoice{it has a high boiling point}
        \wrongchoice{it expands little with temperature}
      \correctchoice{it has a high density}
        \wrongchoice{it looks silvery}
    \end{choices}
\end{question}
}

\element{halliday-mc}{
\begin{question}{halliday-ch14-q05}
    To obtain the absolute pressure from the gauge pressure:
    \begin{choices}
        \wrongchoice{subtract atmospheric pressure}
      \correctchoice{add atmospheric pressure}
        \wrongchoice{subtract 273}
        \wrongchoice{add 273}
        \wrongchoice{convert to \si{\newton\per\meter\squared}}
    \end{choices}
\end{question}
}

\element{halliday-mc}{
\begin{question}{halliday-ch14-q06}
    Barometers and open-tube manometers are two instruments that are used to measure pressure.
    \begin{choices}
        \wrongchoice{Both measure gauge pressure}
        \wrongchoice{Both measure absolute pressure}
        \wrongchoice{Barometers measure gauge pressure and manometers measure absolute pressure}
      \correctchoice{Barometers measure absolute pressure and manometers measure gauge pressure}
        \wrongchoice{Both measure an average of the absolute and gauge pressures}
    \end{choices}
\end{question}
}

\element{halliday-mc}{
\begin{question}{halliday-ch14-q07}
    To measure moderately low pressures oil with a density of \SI{8.5e2}{\kilo\gram\per\meter} is used in place of mercury in a barometer.
    A change in the height of the oil column of \SI{1.0}{\milli\meter} indicates a change in pressure of about:
    \begin{multicols}{2}
    \begin{choices}
        \wrongchoice{\SI{1.2e-7}{\pascal}}
        \wrongchoice{\SI{1.2e-5}{\pascal}}
        \wrongchoice{\SI{0.85}{\pascal}}
        \wrongchoice{\SI{1.2}{\pascal}}
      \correctchoice{\SI{8.3}{\pascal}}
    \end{choices}
    \end{multicols}
\end{question}
}

\element{halliday-mc}{
\begin{question}{halliday-ch14-q08}
    The pressure exerted on the ground by a man is greatest when:
    \begin{choices}
        \wrongchoice{he stands with both feet flat on the ground}
        \wrongchoice{he stands flat on one foot}
      \correctchoice{he stands on the toes of one foot}
        \wrongchoice{he lies down on the ground}
        \wrongchoice{all of the above yield the same pressure}
    \end{choices}
\end{question}
}

\element{halliday-mc}{
\begin{question}{halliday-ch14-q09}
    The vessels shown below all contain water to the same height. 
    \begin{center}
    \begin{tikzpicture}
        %% NOTE: 2 x 2 tikz
    \end{tikzpicture}
    \end{center}
    Rank them according to the pressure exerted by the water on the vessel bottoms,
        least to greatest
    \begin{multicols}{2}
    \begin{choices}
        \wrongchoice{1, 2, 3, 4}
        \wrongchoice{3, 4, 2, 1}
        \wrongchoice{4, 3, 2, 1}
        \wrongchoice{2, 3, 4, 1}
      \correctchoice{All pressures are the same}
    \end{choices}
    \end{multicols}
\end{question}
}

\element{halliday-mc}{
\begin{question}{halliday-ch14-q10}
    In a stationary homogeneous liquid:
    \begin{choices}
        \wrongchoice{pressure is the same at all points}
        \wrongchoice{pressure depends on the direction}
        \wrongchoice{pressure is independent of any atmospheric pressure on the upper surface of the liquid}
      \correctchoice{pressure is the same at all points at the same level}
        \wrongchoice{none of the above}
    \end{choices}
\end{question}
}

\element{halliday-mc}{
\begin{question}{halliday-ch14-q11}
    Which of the following five statements,
        concerning the upper surface pressure of a liquid, is \emph{false}?
    \begin{choices}
        \wrongchoice{It is independent of the surface area}
        \wrongchoice{It is the same for all points on that surface}
        \wrongchoice{It would not increase if the liquid depth were increased}
      \correctchoice{It would increase if the liquid density were increased}
        \wrongchoice{It would increase if the atmospheric pressure increased}
    \end{choices}
\end{question}
}

\element{halliday-mc}{
\begin{question}{halliday-ch14-q12}
    Several cans of different sizes and shapes are all filled with the same liquid to the same depth.  
    Then:
    \begin{choices}
        \wrongchoice{the weight of the liquid is the same for all cans}
        \wrongchoice{the force of the liquid on the bottom of each can is the same}
        \wrongchoice{the least pressure is at the bottom of the can with the largest bottom area}
        \wrongchoice{the greatest pressure is at the bottom of the can with the largest bottom area}
      \correctchoice{the pressure on the bottom of each can is the same}
    \end{choices}
\end{question}
}

\element{halliday-mc}{
\begin{question}{halliday-ch14-q13}
    An airtight box, having a lid of area \SI{80}{\centi\meter\squared},
        is partially evacuated. 
    Atmospheric pressure is \SI{1.01e5}{\pascal}. 
    A force of \SI{600}{\newton} is required to pull the lid off the box. 
    The pressure in the box was:
    \begin{multicols}{2}
    \begin{choices}
      \correctchoice{\SI{2.60e4}{\pascal}}
        \wrongchoice{\SI{6.35e4}{\pascal}}
        \wrongchoice{\SI{7.50e4}{\pascal}}
        \wrongchoice{\SI{1.38e5}{\pascal}}
        \wrongchoice{\SI{1.76e5}{\pascal}}
    \end{choices}
    \end{multicols}
\end{question}
}

\element{halliday-mc}{
\begin{question}{halliday-ch14-q14}
    A closed hemispherical shell of radius $R$ is filled with fluid at uniform pressure $p$. 
    The net force of the fluid on the curved portion of the shell is given by:
    \begin{multicols}{2}
    \begin{choices}
        \wrongchoice{$2\pi R^2 p$}
      \correctchoice{$\pi R^2 p$}
        \wrongchoice{$4\pi R^2 p$}
        \wrongchoice{$\dfrac{4}{3} \pi R^2 p$}
        \wrongchoice{$\dfrac{4}{3} \pi R^3 p$}
    \end{choices}
    \end{multicols}
\end{question}
}

\element{halliday-mc}{
\begin{question}{halliday-ch14-q15}
    The diagram shows a $U$-tube with cross-sectional area $A$ and partially filled with oil of density $\rho$.
    A solid cylinder, which fits the tube tightly but can slide without friction,
        is placed in the right arm. 
    \begin{center}
    \begin{tikzpicture}
        %% NOTE: tikz
    \end{tikzpicture}
    \end{center}
    The system is in equilibrium. 
    The weight of the cylinder is:
    \begin{multicols}{2}
    \begin{choices}
      \correctchoice{$AL \rho g$}
        \wrongchoice{$L^3 ρg$}
        \wrongchoice{$A\rho \left(L + h\right) g$}
        \wrongchoice{$A\rho \left(L - h\right) g$}
        \wrongchoice{none of the provided}
    \end{choices}
    \end{multicols}
\end{question}
}

\element{halliday-mc}{
\begin{question}{halliday-ch14-q16}
    The density of water is \SI{1.0}{\gram\per\centi\meter\cubed}. 
    \begin{center}
    \begin{tikzpicture}
        %% NOTE: tikz
    \end{tikzpicture}
    \end{center}
    The density of the oil in the left column of the $U$-tube shown below is:
    \begin{multicols}{2}
    \begin{choices}
        \wrongchoice{\SI{0.20}{\gram\per\centi\meter\cubed}}
      \correctchoice{\SI{0.80}{\gram\per\centi\meter\cubed}}
        \wrongchoice{\SI{1.0}{\gram\per\centi\meter\cubed}}
        \wrongchoice{\SI{1.3}{\gram\per\centi\meter\cubed}}
        \wrongchoice{\SI{5.0}{\gram\per\centi\meter\cubed}}
    \end{choices}
    \end{multicols}
\end{question}
}

\element{halliday-mc}{
\begin{question}{halliday-ch14-q17}
    A uniform $U$-tube is partially filled with water. 
    Oil, of density \SI{0.75}{\gram\per\centi\meter\cubed},
        is poured into the right arm until the water level in the left arm rises \SI{3}{\centi\meter}.
    The length of the oil column is then:
    \begin{multicols}{2}
    \begin{choices}
        \wrongchoice{\SI{2.25}{\centi\meter}}
      \correctchoice{\SI{8}{\centi\meter}}
        \wrongchoice{\SI{6}{\centi\meter}}
        \wrongchoice{\SI{4}{\centi\meter}}
        \wrongchoice{need to know the cross-sectional area of the $U$-tube}
    \end{choices}
    \end{multicols}
\end{question}
}

\element{halliday-mc}{
\begin{question}{halliday-ch14-q18}
    A long $U$-tube contains mercury (density = \SI{14e3}{\kilo\gram\per\meter\cubed}).
    When \SI{10}{\centi\meter} of water (density = \SI{1.0e3}{\kilo\gram\per\meter\cubed}) is poured into the left arm,
        the mercury in the right arm rises above its original level by:
    \begin{multicols}{2}
    \begin{choices}
      \correctchoice{\SI{0.36}{\centi\meter}}
        \wrongchoice{\SI{0.72}{\centi\meter}}
        \wrongchoice{\SI{14}{\centi\meter}}
        \wrongchoice{\SI{35}{\centi\meter}}
        \wrongchoice{\SI{70}{\centi\meter}}
    \end{choices}
    \end{multicols}
\end{question}
}

\element{halliday-mc}{
\begin{question}{halliday-ch14-q19}
    A bucket of water is pushed from left to right with increasing speed across a horizontal surface.
    Consider the pressure at two points at the same level in the water.
    \begin{choices}
        \wrongchoice{It is the same}
      \correctchoice{It is higher at the point on the left}
        \wrongchoice{It is higher at the point on the right}
        \wrongchoice{At first it is higher at the point on the left but as the bucket speeds up it is lower there}
        \wrongchoice{At first it is higher at the point on the right but as the bucket speeds up it is lower there}
    \end{choices}
\end{question}
}

\element{halliday-mc}{
\begin{question}{halliday-ch14-q20}
    A bucket resting on the floor of an elevator contains an incompressible fluid of density $\rho$. 
    When the elevator has an upward acceleration of magnitude $a$ the pressure difference between two points in a fluid separated by a vertical distance $\Delta h$,
        is given by:
    \begin{multicols}{2}
    \begin{choices}
        \wrongchoice{$\rho a\Delta h$}
        \wrongchoice{$\rho g\Delta h$}
      \correctchoice{$\rho\left(g + a\right)\Delta h$}
        \wrongchoice{$\rho\left(g - a\right)\Delta h$}
        \wrongchoice{$\rho ga\Delta h$}
    \end{choices}
    \end{multicols}
\end{question}
}

\element{halliday-mc}{
\begin{question}{halliday-ch14-q21}
    A bucket resting on the floor of an elevator contains an incompressible fluid of density $\rho$. 
    When the elevator has an downward acceleration of magnitude $a$ the pressure difference between two points in a fluid separated by a vertical distance $\Delta h$,
        is given by:
    \begin{multicols}{2}
    \begin{choices}
        \wrongchoice{$\rho a\Delta h$}
        \wrongchoice{$\rho g\Delta h$}
        \wrongchoice{$\rho\left(g + a\right)\Delta h$}
      \correctchoice{$\rho\left(g - a\right)\Delta h$}
        \wrongchoice{$\rho ga\Delta h$}
    \end{choices}
    \end{multicols}
\end{question}
}

\element{halliday-mc}{
\begin{question}{halliday-ch14-q22}
    ``An object completely submerged in a fluid displaces its own volume of fluid''.
        This is:
    \begin{choices}
        \wrongchoice{Pascal's paradox}
        \wrongchoice{Archimedes' principle}
        \wrongchoice{Pascal's principle}
      \correctchoice{true, but none of the provided}
        \wrongchoice{false}
    \end{choices}
\end{question}
}

\element{halliday-mc}{
\begin{question}{halliday-ch14-q23}
    A certain object floats in fluids of density
    \begin{enumerate}
        \item $0.9 \rho_0$
        \item $\rho_0$
        \item $1.1 \rho_0$
    \end{enumerate}
    Which of the following statements is true?
    \begin{choices}
        \wrongchoice{the buoyant force of fluid 1 is greater than the buoyant forces of the other two fluids}
        \wrongchoice{the buoyant force of fluid 3 is greater than the buoyant forces of the other two fluids}
      \correctchoice{the three fluids exert the same buoyant force}
        \wrongchoice{the object displace the same volume of all three fluids}
        \wrongchoice{none of the provided are true}
    \end{choices}
\end{question}
}

\element{halliday-mc}{
\begin{question}{halliday-ch14-q24}
    A certain object floats in fluids of density
    \begin{enumerate}
        \item $0.9 \rho_0$
        \item $\rho_0$
        \item $1.1 \rho_0$
    \end{enumerate}
    Rank these fluids according to the volume displaced by the object,
        least to greatest.
    \begin{choices}
        \wrongchoice{1, 2, 3}
      \correctchoice{3, 2, 1}
        \wrongchoice{2, 3, 1}
        \wrongchoice{3, 1, 2}
        \wrongchoice{All are the same}
    \end{choices}
\end{question}
}

\element{halliday-mc}{
\begin{question}{halliday-ch14-q25}
    Two identical blocks of ice float in water as shown. 
    \begin{center}
    \begin{tikzpicture}
        %% NOTE: 2 x 2 tikz
    \end{tikzpicture}
    \end{center}
    Then:
    \begin{choices}
        \wrongchoice{block $A$ displaces a greater volume of water since the pressure acts on a smaller bottom area}
        \wrongchoice{block $B$ displaces a greater volume of water since the pressure is less on its bottom}
      \correctchoice{the two blocks displace equal volumes of water since they have the same weight}
        \wrongchoice{block $A$ displaces a greater volume of water since its submerged end is lower in the water}
        \wrongchoice{block $B$ displaces a greater volume of water since its submerged end has a greater area}
    \end{choices}
\end{question}
}

\element{halliday-mc}{
\begin{question}{halliday-ch14-q26}
    A block of ice at \SI{0}{\degreeCelsius} is floating on the surface of ice water in a beaker. 
    The surface of the water just comes to the top of the beaker. 
    When the ice melts the water level will:
    \begin{choices}
        \wrongchoice{rise and overflow will occur}
      \correctchoice{remain the same}
        \wrongchoice{fall}
        \wrongchoice{depend on the initial ratio of water to ice}
        \wrongchoice{depend on the shape of the block of ice}
    \end{choices}
\end{question}
}

\element{halliday-mc}{
\begin{question}{halliday-ch14-q27}
    A block of ice at \SI{0}{\degreeCelsius} containing a piece of cork is floating on the surface of ice water in a beaker.
    When the ice has melted the water level:
    \begin{choices}
        \wrongchoice{is higher}
        \wrongchoice{is lower}
      \correctchoice{is the same}
        \wrongchoice{depends on the initial ratio of water to ice}
        \wrongchoice{depends on the shape of the ice block}
    \end{choices}
\end{question}
}

\element{halliday-mc}{
\begin{question}{halliday-ch14-q28}
    A pirate chest rests at the bottom of an ocean. 
    If the water is still, the net force it exerts on the chest:
    \begin{choices}
        \wrongchoice{is upward}
      \correctchoice{is downward}
        \wrongchoice{is zero}
        \wrongchoice{depends on the mass of the chest}
        \wrongchoice{depends on the contents of the chest}
    \end{choices}
\end{question}
}

\element{halliday-mc}{
\begin{question}{halliday-ch14-q29}
    A small steel ball floats in a half-full container of mercury. 
    When water is added:
    \begin{choices}
        \wrongchoice{the ball will float on the water}
      \correctchoice{the ball will rise slightly}
        \wrongchoice{the mercury will float on the water}
        \wrongchoice{the ball will sink to the bottom of the container}
        \wrongchoice{the ball will lower slightly more into the mercury}
    \end{choices}
\end{question}
}

\element{halliday-mc}{
\begin{question}{halliday-ch14-q30}
    A cork floats on the surface of an incompressible liquid in a container exposed to atmospheric pressure. 
    The container is then sealed and the air above the liquid is evacuated. 
    The cork:
    \begin{choices}
        \wrongchoice{sinks slightly}
        \wrongchoice{rises slightly}
      \correctchoice{floats at the same height}
        \wrongchoice{bobs up and down about its old position}
        \wrongchoice{behaves erratically}
    \end{choices}
\end{question}
}

\element{halliday-mc}{
\begin{question}{halliday-ch14-q31}
    An object hangs from a spring balance. 
    The balance indicates \SI{30}{\newton} in air and \SI{20}{\newton} when the object is submerged in water. 
    What does the balance indicate when the object is submersed in a liquid with a density that is half that of water?
    \begin{multicols}{3}
    \begin{choices}
        \wrongchoice{\SI{20}{\newton}}
        \wrongchoice{\SI{25}{\newton}}
        \wrongchoice{\SI{30}{\newton}}
        \wrongchoice{\SI{35}{\newton}}
        \wrongchoice{\SI{40}{\newton}}
    \end{choices}
    \end{multicols}
\end{question}
}

\element{halliday-mc}{
\begin{question}{halliday-ch14-q32}
    A fir wood board floats in fresh water with \SI{60}{\percent} of its volume under water. 
    The density of the wood in \si{\gram\per\centi\meter\cubed} is:
    %\begin{multicols}{2}
    \begin{choices}
        \wrongchoice{\num{0.4}}
        \wrongchoice{\num{0.5}}
      \correctchoice{\num{0.6}}
        \wrongchoice{less than \num{0.4}}
        \wrongchoice{more than \num{0.6}}
    \end{choices}
    %\end{multicols}
\end{question}
}

\element{halliday-mc}{
\begin{question}{halliday-ch14-q33}
    A boat floating in fresh water displaces \SI{16 000}{\newton} of water. 
    How many newtons of saltwater would it displace if it floats in saltwater of specific gravity \num{1.17}?
    \begin{multicols}{2}
    \begin{choices}
        \wrongchoice{\num{14 500}}
        \wrongchoice{\num{17 600}}
      \correctchoice{\num{16 000}}
        \wrongchoice{\num{284}}
        \wrongchoice{\num{234}}
    \end{choices}
    \end{multicols}
\end{question}
}

\element{halliday-mc}{
\begin{question}{halliday-ch14-q34}
    A rock, which weighs \SI{1400}{\newton} in air,
    has an apparent weight of \SI{900}{\newton} when submerged in fresh water (\SI{998}{\kilo\gram\per\meter\cubed}).
    The volume of the rock is:
    \begin{multicols}{2}
    \begin{choices}
        \wrongchoice{\SI{0.14}{\meter\cubed}}
        \wrongchoice{\SI{0.60}{\meter\cubed}}
        \wrongchoice{\SI{0.90}{\meter\cubed}}
      \correctchoice{\SI{5.1e-2}{\meter\cubed}}
        \wrongchoice{\SI{9.2e-2}{\meter\cubed}}
    \end{choices}
    \end{multicols}
\end{question}
}

\element{halliday-mc}{
\begin{question}{halliday-ch14-q35}
    A loaded ship passes from a lake (fresh water) to the ocean (saltwater). 
    Saltwater is more dense than fresh water and as a result the ship will:
    \begin{choices}
      \correctchoice{ride higher in the water}
        \wrongchoice{settle lower in the water}
        \wrongchoice{ride at the same level in the water}
        \wrongchoice{experience an increase in buoyant force}
        \wrongchoice{experience a decrease in buoyant force}
    \end{choices}
\end{question}
}

\element{halliday-mc}{
\begin{question}{halliday-ch14-q36}
    The dimensions of a wooden raft (density = \SI{150}{\kilo\gram\per\meter\cubed}) are $\SI{3.0}{\meter}\times\SI{3.0}{\meter}\times\SI{1.0}{\meter}$.
    What maximum load can it carry in seawater (density = \SI{1020}{\kilo\gram\per\meter\cubed})?
    \begin{multicols}{2}
    \begin{choices}
        \wrongchoice{\SI{1350}{\kilo\gram}}
      \correctchoice{\SI{7800}{\kilo\gram}}
        \wrongchoice{\SI{9200}{\kilo\gram}}
        \wrongchoice{\SI{19 500}{\kilo\gram}}
        \wrongchoice{\SI{24 300}{\kilo\gram}}
    \end{choices}
    \end{multicols}
\end{question}
}

\element{halliday-mc}{
\begin{question}{halliday-ch14-q37}
    A tin can has a volume of \SI{1000}{\centi\meter\cubed} and a mass of \SI{100}{\gram}.
    Approximately what mass of lead shot can it carry without sinking in water?
    \begin{multicols}{2}
    \begin{choices}
      \correctchoice{\SI{900}{\gram}}
        \wrongchoice{\SI{100}{\gram}}
        \wrongchoice{\SI{1000}{\gram}}
        \wrongchoice{\SI{1100}{\gram}}
        \wrongchoice{\SI{980}{\gram}}
    \end{choices}
    \end{multicols}
\end{question}
}

\element{halliday-mc}{
\begin{question}{halliday-ch14-q38}
    A block of wood weighs \SI{160}{\newton} and has a specific gravity of \num{0.60}. 
    To sink it in fresh water requires an additional downward force of:
    \begin{multicols}{2}
    \begin{choices}
        \wrongchoice{\SI{54}{\newton}}
        \wrongchoice{\SI{64}{\newton}}
        \wrongchoice{\SI{96}{\newton}}
      \correctchoice{\SI{110}{\newton}}
        \wrongchoice{\SI{240}{\newton}}
    \end{choices}
    \end{multicols}
\end{question}
}

\element{halliday-mc}{
\begin{question}{halliday-ch14-q39}
    A student standardizes the concentration of a saltwater solution by slowly adding salt until an egg will just float. 
    The procedure is based on the assumption that:
    \begin{choices}
        \wrongchoice{all eggs have the same volume}
        \wrongchoice{all eggs have the same weight}
      \correctchoice{all eggs have the same density}
        \wrongchoice{all eggs have the same shape}
        \wrongchoice{the salt tends to neutralize the cholesterol in the egg}
    \end{choices}
\end{question}
}

\element{halliday-mc}{
\begin{question}{halliday-ch14-q40}
    A solid has a volume of \SI{8}{\centi\meter\cubed}.
    When weighed on a spring scale calibrated in grams,
        the scale indicates \SI{20}{\gram}. 
    What does the scale indicate if the object is weighed while immersed in a liquid of density \SI{2}{\gram\per\centi\meter\cubed}?
    \begin{multicols}{2}
    \begin{choices}
        \wrongchoice{\SI{4}{\gram}}
        \wrongchoice{\SI{10}{\gram}}
        \wrongchoice{\SI{12}{\gram}}
        \wrongchoice{\SI{16}{\gram}}
        \wrongchoice{Zero, since the object will float}
    \end{choices}
    \end{multicols}
\end{question}
}

\element{halliday-mc}{
\begin{question}{halliday-ch14-q41}
    A \SI{210}{\gram} object apparently loses \SI{30}{\gram} when suspended in a liquid of density \SI{2.0}{\gram\per\centi\meter\cubed}. 
    The density of the object is:
    \begin{multicols}{2}
    \begin{choices}
        \wrongchoice{\SI{7.0}{\gram\per\centi\meter\cubed}}
        \wrongchoice{\SI{3.5}{\gram\per\centi\meter\cubed}}
        \wrongchoice{\SI{1.4}{\gram\per\centi\meter\cubed}}
      \correctchoice{\SI{14}{\gram\per\centi\meter\cubed}}
        \wrongchoice{none of the provided}
    \end{choices}
    \end{multicols}
\end{question}
}

\element{halliday-mc}{
\begin{question}{halliday-ch14-q42}
    A steel ax and an aluminum piston have the same apparent weight in water. 
    When they are weighed in air:
    \begin{choices}
        \wrongchoice{they weigh the same}
        \wrongchoice{the ax is heavier}
      \correctchoice{the piston is heavier}
        \wrongchoice{both weigh less than they did in water}
        \wrongchoice{depends on their shapes}
    \end{choices}
\end{question}
}

\element{halliday-mc}{
\begin{question}{halliday-ch14-q43}
    The apparent weight of a steel sphere immersed in various liquids is measured using a spring scale. 
    The greatest reading is obtained for that liquid:
    \begin{choices}
      \correctchoice{having the smallest density}
        \wrongchoice{having the largest density}
        \wrongchoice{subject to the greatest atmospheric pressure}
        \wrongchoice{having the greatest volume}
        \wrongchoice{in which the sphere was submerged deepest}
    \end{choices}
\end{question}
}

\element{halliday-mc}{
\begin{question}{halliday-ch14-q44}
    A \SI{0.50}{\newton} metal sinker appears (as measured using a spring scale) to have a weight of \SI{0.45}{\newton} when submerged in water. 
    The specific gravity of the metal is:
    \begin{multicols}{2}
    \begin{choices}
        \wrongchoice{\num{6}}
        \wrongchoice{\num{8}}
        \wrongchoice{\num{9}}
      \correctchoice{\num{10}}
        \wrongchoice{\num{12}}
    \end{choices}
    \end{multicols}
\end{question}
}

\element{halliday-mc}{
\begin{question}{halliday-ch14-q45}
    An object floats on the surface of a fluid. 
    For purposes of calculating the torque on it,
        the buoyant force is taken to act at:
    \begin{choices}
        \wrongchoice{the center of the bottom surface of the object}
        \wrongchoice{the center of gravity of the object}
      \correctchoice{the center of gravity of the fluid that the object replaced}
        \wrongchoice{the geometric center of the object}
        \wrongchoice{none of the above}
    \end{choices}
\end{question}
}

\element{halliday-mc}{
\begin{question}{halliday-ch14-q46}
    A blast of wind tips a sailboat in the clockwise direction when viewed from the stern. 
    When the wind ceases the boat rotates back toward the upright position if,
        when it is tilted, the center of buoyancy:
    \begin{choices}
        \wrongchoice{is above the center of gravity}
        \wrongchoice{is below the center of gravity}
      \correctchoice{is to the right of the center of gravity}
        \wrongchoice{is to the left of the center of gravity}
        \wrongchoice{coincides with the center of gravity}
    \end{choices}
\end{question}
}

\element{halliday-mc}{
\begin{question}{halliday-ch14-q47}
    A cork floats in water in a bucket resting on the floor of an elevator. 
    The elevator then accelerates upward. 
    During the acceleration:
    \begin{choices}
        \wrongchoice{the cork is immersed more}
        \wrongchoice{the cork is immersed less}
        \wrongchoice{the cork is immersed the same amount}
        \wrongchoice{at first the cork is immersed less but as the elevator speeds up it is immersed more}
        \wrongchoice{at first the cork is immersed more but as the elevator speeds up it is immersed less}
    \end{choices}
\end{question}
}

\element{halliday-mc}{
\begin{question}{halliday-ch14-q48}
    Two balls have the same shape and size but one is denser than the other. 
    If frictional forces are negligible when they are dropped in air,
        which has the greater acceleration?
    \begin{choices}
      \correctchoice{The heavier ball}
        \wrongchoice{The lighter ball}
        \wrongchoice{They have the same acceleration}
        \wrongchoice{The heavier ball if atmospheric pressure is high, they lighter ball if it is low}
        \wrongchoice{The lighter ball if atmospheric pressure is high, the heavier ball if it is low}
    \end{choices}
\end{question}
}

\element{halliday-mc}{
\begin{question}{halliday-ch14-q49}
    The principle of fluid pressure that is used in hydraulic brakes or lifts is that:
    \begin{choices}
        \wrongchoice{pressure is the same at all levels in a fluid}
      \correctchoice{increases of pressure are transmitted equally to all parts of a fluid}
        \wrongchoice{the pressure at a point in a fluid is due to the weight of the fluid above it}
        \wrongchoice{increases of pressure can only be transmitted through fluids}
        \wrongchoice{the pressure at a given depth is proportional to the depth in the fluid}
    \end{choices}
\end{question}
}

\element{halliday-mc}{
\begin{question}{halliday-ch14-q50}
    Which of the following statements about Pascal's principle is true?
    \begin{choices}
        \wrongchoice{It is valid only for incompressible fluids}
        \wrongchoice{It explains why light objects float}
        \wrongchoice{It explains why the pressure is greater at the bottom of a lake than at the surface}
        \wrongchoice{It is valid only for objects that are less dense than water}
      \correctchoice{None of the provided are true}
    \end{choices}
\end{question}
}

\element{halliday-mc}{
\begin{question}{halliday-ch14-q51}
    The hydraulic automobile jack illustrates:
    \begin{choices}
        \wrongchoice{Archimedes' principle}
      \correctchoice{Pascal's principle}
        \wrongchoice{Hooke's law}
        \wrongchoice{Newton's third law}
        \wrongchoice{Newton's second law}
    \end{choices}
\end{question}
}

\element{halliday-mc}{
\begin{question}{halliday-ch14-q52}
    One piston in a hydraulic lift has an area that is twice the area of the other. 
    When the pressure at the smaller piston is increased by $\Delta p$ the pressure at the larger piston:
    \begin{choices}
        \wrongchoice{increases by $2\Delta p$}
        \wrongchoice{increases by $\dfrac{\Delta p}{2}$}
      \correctchoice{increases by $\Delta p$}
        \wrongchoice{increases by $4 \Delta p$}
        \wrongchoice{does not change}
    \end{choices}
\end{question}
}

\element{halliday-mc}{
\begin{question}{halliday-ch14-q53}
    A hydraulic press has one piston of diameter \SI{2.0}{\centi\meter} and the other piston of diameter \SI{8.0}{\centi\meter}.
    What force must be applied to the smaller piston to obtain a force of \SI{1600}{\newton} at the larger piston?
    \begin{multicols}{2}
    \begin{choices}
        \wrongchoice{\SI{6.25}{\newton}}
        \wrongchoice{\SI{25}{\newton}}
        \wrongchoice{\SI{100}{\newton}}
        \wrongchoice{\SI{400}{\newton}}
        \wrongchoice{\SI{1600}{\newton}}
    \end{choices}
    \end{multicols}
\end{question}
}

\element{halliday-mc}{
\begin{question}{halliday-ch14-q54}
    The two arms of a $U$-tube are not identical,
        one having twice the diameter of the other. 
    A cork in the narrow arm requires a force of \SI{16}{\newton} to remove it. 
    The tube is filled with water and the wide arm is fitted with a piston. 
    The minimum force that must be applied to the piston to push the cork out is:
    \begin{multicols}{3}
    \begin{choices}
        \wrongchoice{\SI{4}{\newton}}
        \wrongchoice{\SI{8}{\newton}}
        \wrongchoice{\SI{16}{\newton}}
        \wrongchoice{\SI{32}{\newton}}
      \correctchoice{\SI{64}{\newton}}
    \end{choices}
    \end{multicols}
\end{question}
}

\element{halliday-mc}{
\begin{question}{halliday-ch14-q55}
    A $U$-tube has dissimilar arms,
        one having twice the diameter of the other. 
    It contains an incompressible fluid and is fitted with a sliding piston in each arm, with each piston in contact with the fluid.
    When the piston in the narrow arm is pushed down a distance $d$,
        the piston in the wide arm rises a distance:
    \begin{multicols}{3}
    \begin{choices}
        \wrongchoice{$d$}
        \wrongchoice{$2d$}
        \wrongchoice{$\dfrac{d}{2}$}
        \wrongchoice{$4d$}
      \correctchoice{$\dfrac{d}{4}$}
    \end{choices}
    \end{multicols}
\end{question}
}

\element{halliday-mc}{
\begin{question}{halliday-ch14-q56}
    A $U$-tube has dissimilar arms, one having twice the diameter of the other. 
    It contains an incompressible fluid and is fitted with a sliding piston in each arm, with each piston in contact with the fluid.
    When an applied force does work $W$ in pushing the piston in the narrow arm down, the fluid does work \rule[-0.1pt]{4em}{0.1pt} on the piston in the wide arm
    \begin{multicols}{3}
    \begin{choices}
      \correctchoice{$W$}
        \wrongchoice{$2W$}
        \wrongchoice{$\dfrac{W}{2}$}
        \wrongchoice{$4W$}
        \wrongchoice{$\dfrac{W}{4}$}
    \end{choices}
    \end{multicols}
\end{question}
}

\element{halliday-mc}{
\begin{question}{halliday-ch14-q57}
    A fluid is undergoing ``incompressible'' flow. 
    This means that:
    \begin{choices}
        \wrongchoice{the pressure at a given point cannot change with time}
        \wrongchoice{the velocity at a given point cannot change with time}
        \wrongchoice{the velocity must be the same everywhere}
        \wrongchoice{the pressure must be the same everywhere}
        \wrongchoice{the density cannot change with time or location}
    \end{choices}
\end{question}
}

\element{halliday-mc}{
\begin{question}{halliday-ch14-q58}
    A fluid is undergoing steady flow. 
    Therefore:
    \begin{choices}
        \wrongchoice{the velocity of any given molecule of fluid does not change}
        \wrongchoice{the pressure does not vary from point to point}
      \correctchoice{the velocity at any given point does not vary with time}
        \wrongchoice{the density does not vary from point to point}
        \wrongchoice{the flow is not uphill or downhill}
    \end{choices}
\end{question}
}

\element{halliday-mc}{
\begin{question}{halliday-ch14-q59}
    If $p$ is a pressure and $\rho$ is a density then $\rho/p$ has units of:
    \begin{choices}
        \wrongchoice{meter squared (\si{\meter\squared})}
      \correctchoice{meter squared per second squared (\si{\meter\squared\per\second\squared})}
        \wrongchoice{newton per meter squared (\si{\newton\per\meter\squared})}
        \wrongchoice{kilogram per meter squared (\si{\kilo\gram\per\meter\squared})}
        \wrongchoice{meter cubed per kilogram (\si{\meter\cubed\per\kilo\gram})}
    \end{choices}
\end{question}
}

\element{halliday-mc}{
\begin{question}{halliday-ch14-q60}
    One end of a cylindrical pipe has a radius of \SI{1.5}{\centi\meter}.
    Water (density = \SI{1.0e3}{\kilo\gram\per\meter\cubed}) streams steadily out at 7.0 m/s. The rate at which mass is leaving the pipe is:
    \begin{multicols}{2}
    \begin{choices}
        \wrongchoice{\SI{2.5}{\kilo\gram\per\second}}
      \correctchoice{\SI{4.9}{\kilo\gram\per\second}}
        \wrongchoice{\SI{7.0}{\kilo\gram\per\second}}
        \wrongchoice{\SI{48}{\kilo\gram\per\second}}
        \wrongchoice{\SI{7.0e3}{\kilo\gram\per\second}}
    \end{choices}
    \end{multicols}
\end{question}
}

\element{halliday-mc}{
\begin{question}{halliday-ch14-q61}
    One end of a cylindrical pipe has a radius of \SI{1.5}{\centi\meter}.
    Water (density = \SI{1.0e3}{\kilo\gram\per\meter\cubed}) streams steadily out at \SI{7.0}{\meter\per\second}.
    The volume flow rate is:
    \begin{multicols}{2}
    \begin{choices}
      \correctchoice{\SI{4.9e-3}{\meter\cubed\per\second\squared}}
        \wrongchoice{\SI{2.5}{\meter\cubed\per\second\squared}}
        \wrongchoice{\SI{4.9}{\meter\cubed\per\second\squared}}
        \wrongchoice{\SI{7.0}{\meter\cubed\per\second\squared}}
        \wrongchoice{\SI{48}{\meter\cubed\per\second\squared}}
    \end{choices}
    \end{multicols}
\end{question}
}

\element{halliday-mc}{
\begin{question}{halliday-ch14-q62}
    The equation of continuity for fluid flow can be derived from the conservation of:
    \begin{multicols}{2}
    \begin{choices}
        \wrongchoice{energy}
      \correctchoice{mass}
        \wrongchoice{angular momentum}
        \wrongchoice{volume}
        \wrongchoice{pressure}
    \end{choices}
    \end{multicols}
\end{question}
}

\element{halliday-mc}{
\begin{question}{halliday-ch14-q63}
    The diagram shows a pipe of uniform cross section in which water is flowing. 
    The directions of flow and the volume flow rates (in \si{\centi\meter\cubed\per\second}) are shown for various portions of the pipe.
    \begin{center}
    \begin{tikzpicture}
        %% NOTE: 2 x 2 tikz
    \end{tikzpicture}
    \end{center}
    The direction of flow and the volume flow rate in the portion marked $A$ are:
    \begin{multicols}{2}
    \begin{choices}
        \wrongchoice{$\downarrow$ and \SI{3}{\centi\meter\cubed\per\second}}
        \wrongchoice{$\uparrow$ and \SI{7}{\centi\meter\cubed\per\second}}
        \wrongchoice{$\downarrow$ and \SI{9}{\centi\meter\cubed\per\second}}
        \wrongchoice{$\uparrow$ and \SI{11}{\centi\meter\cubed\per\second}}
      \correctchoice{$\downarrow↓$ and \SI{15}{\centi\meter\cubed\per\second}}
    \end{choices}
    \end{multicols}
\end{question}
}

\element{halliday-mc}{
\begin{question}{halliday-ch14-q64}
    An incompressible liquid flows along the pipe as shown. 
    \begin{center}
    \begin{tikzpicture}
        %% NOTE: 2 x 2 tikz
    \end{tikzpicture}
    \end{center}
    The ratio of the speeds $v_2/v_1$ is:
    \begin{multicols}{2}
    \begin{choices}
      \correctchoice{$\dfrac{A_1}{A_2}$}
        \wrongchoice{$\dfrac{A_2}{A_1}$}
        \wrongchoice{$\sqrt{\dfrac{A_1}{A_2}}$}
        \wrongchoice{$\sqrt{\dfrac{A_2}{A_1}}$}
        \wrongchoice{$\dfrac{v_1}{v_2}$}
    \end{choices}
    \end{multicols}
\end{question}
}

\element{halliday-mc}{
\begin{question}{halliday-ch14-q65}
    Water flows through a cylindrical pipe of varying cross section. 
    The velocity is \SI{3.0}{\meter\per\second} at a point where the pipe diameter is \SI{1.0}{\centi\meter}.
    At a point where the pipe diameter is \SI{3.0}{\centi\meter},
        the velocity is:
    \begin{multicols}{2}
    \begin{choices}
        \wrongchoice{\SI{9}{\meter\per\second}}
        \wrongchoice{\SI{3}{\meter\per\second}}
        \wrongchoice{\SI{1}{\meter\per\second}}
      \correctchoice{\SI{0.33}{\meter\per\second}}
        \wrongchoice{\SI{0.11}{\meter\per\second}}
    \end{choices}
    \end{multicols}
\end{question}
}

\element{halliday-mc}{
\begin{question}{halliday-ch14-q66}
    A constriction in a pipe reduces its diameter from \SI{4.0}{\centi\meter} to \SI{2.0}{\centi\meter}.
    Where the pipe is narrow the water speed is \SI{8.0}{\meter\per\second}.
    Where it is wide the water speed is:
    \begin{multicols}{2}
    \begin{choices}
        \wrongchoice{\SI{2.0}{\meter\per\second}}
        \wrongchoice{\SI{4.0}{\meter\per\second}}
      \correctchoice{\SI{8.0}{\meter\per\second}}
        \wrongchoice{\SI{16}{\meter\per\second}}
        \wrongchoice{\SI{32}{\meter\per\second}}
    \end{choices}
    \end{multicols}
\end{question}
}

\element{halliday-mc}{
\begin{question}{halliday-ch14-q67}
    Water flows from a \SI{6.0}{\centi\meter} diameter pipe into an \SI{8.0}{\centi\meter} diameter pipe.
    The speed in the \SI{6.0}{\centi\meter} pipe is \SI{5.0}{\meter\per\second}.
    The speed in the \SI{8.0}{\centi\meter} pipe is:
    \begin{multicols}{2}
    \begin{choices}
      \correctchoice{\SI{2.8}{\meter\per\second}}
        \wrongchoice{\SI{3.7}{\meter\per\second}}
        \wrongchoice{\SI{6.6}{\meter\per\second}}
        \wrongchoice{\SI{8.8}{\meter\per\second}}
        \wrongchoice{\SI{9.9}{\meter\per\second}}
    \end{choices}
    \end{multicols}
\end{question}
}

\element{halliday-mc}{
\begin{question}{halliday-ch14-q68}
    A lawn sprinkler is made of a \SI{1.0}{\centi\meter} diameter garden hose with one end closed and 25 holes,
        each with a diameter of \SI{0.050}{\centi\meter}, cut near the closed end.
    If water flows at \SI{2.0}{\meter\per\second} in the hose,
        the speed of the water leaving a hole is:
    \begin{multicols}{2}
    \begin{choices}
        \wrongchoice{\SI{2.0}{\meter\per\second}}
      \correctchoice{\SI{32}{\meter\per\second}}
        \wrongchoice{\SI{40}{\meter\per\second}}
        \wrongchoice{\SI{600}{\meter\per\second}}
        \wrongchoice{\SI{800}{\meter\per\second}}
    \end{choices}
    \end{multicols}
\end{question}
}

\element{halliday-mc}{
\begin{question}{halliday-ch14-q69}
    Bernoulli's equation can be derived from the conservation of:
    \begin{choices}
      \correctchoice{energy}
        \wrongchoice{mass}
        \wrongchoice{angular momentum}
        \wrongchoice{volume}
        \wrongchoice{pressure}
    \end{choices}
\end{question}
}

\element{halliday-mc}{
\begin{question}{halliday-ch14-q70}
    Which of the following assumptions is \emph{not} made in the derivation of Bernoulli's equation?
    \begin{choices}
        \wrongchoice{Assume streamline flow}
        \wrongchoice{Neglect viscosity}
        \wrongchoice{Neglect friction}
      \correctchoice{Neglect gravity}
        \wrongchoice{Neglect turbulence}
    \end{choices}
\end{question}
}

\element{halliday-mc}{
\begin{question}{halliday-ch14-q71}
    The quantity $y$ appearing in Bernoulli's equation \emph{must} be measured:
    \begin{choices}
        \wrongchoice{upward from the center of Earth}
        \wrongchoice{upward from the surface of Earth}
        \wrongchoice{upward from the lowest point in the flow}
        \wrongchoice{downward from the highest point in the flow}
      \correctchoice{upward from any convenient level}
    \end{choices}
\end{question}
}

\element{halliday-mc}{
\begin{question}{halliday-ch14-q72}
    Water flows through a constriction in a horizontal pipe. 
    As it enters the constriction,
        the water's:
    \begin{choices}
      \correctchoice{speed increases and pressure decreases}
        \wrongchoice{speed increases and pressure remains constant}
        \wrongchoice{speed increases and pressure increases}
        \wrongchoice{speed decreases and pressure increases}
        \wrongchoice{speed decreases and pressure decreases}
    \end{choices}
\end{question}
}

\element{halliday-mc}{
\begin{question}{halliday-ch14-q73}
    Water is pumped through the hose shown below,
        from a lower level to an upper level. 
    \begin{center}
    \begin{tikzpicture}
        %% NOTE: 2 x 2 tikz
    \end{tikzpicture}
    \end{center}
    Compared to the water at point 1,
        the water at point 2:
    \begin{choices}
        \wrongchoice{has greater speed and greater pressure}
      \correctchoice{has greater speed and less pressure}
        \wrongchoice{has less speed and less pressure}
        \wrongchoice{has less speed and greater pressure}
        \wrongchoice{has greater speed and the same pressure}
    \end{choices}
\end{question}
}

\element{halliday-mc}{
\begin{question}{halliday-ch14-q74}
    A non-viscous incompressible liquid is flowing through a horizontal pipe of constant cross section. 
    Bernoulli's equation and the equation of continuity predict that the drop in pressure along the pipe:
    \begin{choices}
      \correctchoice{is zero}
        \wrongchoice{depends on the length of the pipe}
        \wrongchoice{depends on the fluid velocity}
        \wrongchoice{depends on the cross-sectional area of the pipe}
        \wrongchoice{depends on the height of the pipe}
    \end{choices}
\end{question}
}

\element{halliday-mc}{
\begin{question}{halliday-ch14-q75}
    A non-viscous incompressible fluid is pumped steadily into the narrow end of a long tapered pipe and emerges from the wide end. 
    The pressure at the input is greater than at the output.
    A possible explanation is:
    \begin{choices}
        \wrongchoice{the fluid speed increases from input to output}
        \wrongchoice{the fluid speed is the same at the two ends}
        \wrongchoice{the fluid is flowing uphill}
        \wrongchoice{the fluid is flowing downhill}
        \wrongchoice{the fluid is flowing horizontally}
    \end{choices}
\end{question}
}

\element{halliday-mc}{
\begin{question}{halliday-ch14-q76}
    Water is pumped into one end of a long pipe at the rate of \SI{40}{\liter\per\minute}. 
    It emerges at the other end at \SI{24}{\liter\per\minute}. 
    A possible reason for this decrease in flow is:
    \begin{choices}
        \wrongchoice{the water is being pumped uphill}
        \wrongchoice{the water is being pumped downhill}
        \wrongchoice{the diameter of the pipe is not the same at the two ends}
        \wrongchoice{friction in the pipe}
      \correctchoice{a leak in the pipe}
    \end{choices}
\end{question}
}

\element{halliday-mc}{
\begin{question}{halliday-ch14-q77}
    Consider a pipe containing a fluid, with the fluid being at rest. 
    To apply Bernoulli's equation to this situation:
    \begin{choices}
      \correctchoice{set $v$ equal to zero because there is no motion}
        \wrongchoice{set $g$ equal to zero because there is no acceleration}
        \wrongchoice{set $v$ and $g$ both equal to zero}
        \wrongchoice{set $p$ equal to the atmospheric pressure}
        \wrongchoice{cannot be done,
            Bernoulli’s equation applies only to fluids in motion}
    \end{choices}
\end{question}
}

\element{halliday-mc}{
\begin{question}{halliday-ch14-q78}
    Water (density = \SI{1.0e3}{\kilo\gram\per\meter\cubed}) flows through a horizontal tapered pipe.
    At the wide end its speed is \SI{4.0}{\meter\per\second}. 
    The difference in pressure between the two ends is \SI{4.5e3}{\pascal}.
    The speed of the water at the narrow end is:
    \begin{multicols}{2}
    \begin{choices}
        \wrongchoice{\SI{2.6}{\meter\per\second}}
        \wrongchoice{\SI{3.4}{\meter\per\second}}
        \wrongchoice{\SI{4.0}{\meter\per\second}}
        \wrongchoice{\SI{4.5}{\meter\per\second}}
      \correctchoice{\SI{5.0}{\meter\per\second}}
    \end{choices}
    \end{multicols}
\end{question}
}

\element{halliday-mc}{
\begin{question}{halliday-ch14-q79}
    Water is streaming downward from a faucet opening with an area of \SI{3.0e-5}{\meter\squared}.
    It leaves the faucet with a speed of \SI{5.0}{\meter\per\second}.
    The cross-sectional area of the stream \SI{0.50}{\meter} below the faucet is:
    \begin{multicols}{2}
    \begin{choices}
        \wrongchoice{\SI{1.5e-5}{\meter\squared}}
        \wrongchoice{\SI{2.0e-5}{\meter\squared}}
      \correctchoice{\SI{2.5e-5}{\meter\squared}}
        \wrongchoice{\SI{3.0e-5}{\meter\squared}}
        \wrongchoice{\SI{3.5e-5}{\meter\squared}}
    \end{choices}
    \end{multicols}
\end{question}
}

\element{halliday-mc}{
\begin{question}{halliday-ch14-q80}
    A large water tank, open at the top,
        has a small hole in the bottom. 
    When the water level is \SI{30}{\meter} above the bottom of the tank,
        the speed of the water leaking from the hole:
    \begin{multicols}{2}
    \begin{choices}
        \wrongchoice{is \SI{2.5}{\meter\per\second}}
      \correctchoice{is \SI{24}{\meter\per\second}}
        \wrongchoice{is \SI{44}{\meter\per\second}}
        \wrongchoice{cannot be calculated unless the area of the hole is given}
        \wrongchoice{cannot be calculated unless the areas of the hole and tank are given}
    \end{choices}
    \end{multicols}
\end{question}
}

\element{halliday-mc}{
\begin{question}{halliday-ch14-q81}
    A large tank filled with water has two holes in the bottom,
        one with twice the radius of the the speed of the water other. 
    In steady flow the speed of water leaving the larger hole is leaving the smaller.
    \begin{multicols}{2}
    \begin{choices}
        \wrongchoice{twice}
        \wrongchoice{four times}
        \wrongchoice{half}
        \wrongchoice{one-fourth}
      \correctchoice{the same as}
    \end{choices}
    \end{multicols}
\end{question}
}

\element{halliday-mc}{
\begin{question}{halliday-ch14-q82}
    A non-viscous incompressible fluid is pumped steadily up a vertical pipe with uniform cross section. 
    The difference in pressure between points at the top and bottom:
    \begin{choices}
      \correctchoice{is the same as it would be if the fluid were motionless}
        \wrongchoice{is greater at higher flow rates than at lower flow rates}
        \wrongchoice{is less at higher flow rates than at lower flow rates}
        \wrongchoice{does not depend on the density of the fluid}
        \wrongchoice{is zero}
    \end{choices}
\end{question}
}

\element{halliday-mc}{
\begin{question}{halliday-ch14-q83}
    A water line enters a house \SI{2.0}{\meter} below ground. 
    A smaller diameter pipe carries water to a faucet \SI{5.0}{\meter} above ground,
        on the second floor. 
    Water flows at \SI{2.0}{\meter\per\second} in the main line and at \SI{7.0}{\meter\per\second} on the second floor. 
    Take the density of water to be \SI{1.0e3}{\kilo\gram\per\meter\cubed}.
    The difference in pressure between the main line and the second floor is:
    \begin{choices}
      \correctchoice{\SI{7.15e4}{\pascal} with the main line at the higher pressure}
        \wrongchoice{\SI{2.65e4}{\pascal} with the main line at the higher pressure}
        \wrongchoice{\SI{7.15e4}{\pascal} with the main line at the lower pressure}
        \wrongchoice{\SI{2.65e4}{\pascal} with the main line at the lower pressure}
        \wrongchoice{\SI{9.4e4}{\pascal} with the main line at the higher pressure}
    \end{choices}
\end{question}
}

\element{halliday-mc}{
\begin{question}{halliday-ch14-q84}
    A person blows across the top of one arm of a $U$-tube partially filled with water.
    The water in that arm:
    \begin{choices}
      \correctchoice{rises slightly}
        \wrongchoice{drops slightly}
        \wrongchoice{remains at the same height}
        \wrongchoice{rises if the blowing is soft but drops if it is hard}
        \wrongchoice{rises if the blowing is hard but drops if it is soft}
    \end{choices}
\end{question}
}


\endinput


