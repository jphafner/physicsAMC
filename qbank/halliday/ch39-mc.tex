
%%--------------------------------------------------
%% Halliday: Fundamentals of Physics
%%--------------------------------------------------


%% Chapter 39: More About Matter Waves
%%--------------------------------------------------


%% Learning Objectives
%%--------------------------------------------------

%% 39.01: Identify the confinement principle: Confinement of a wave (including a matter wave) leads to the quantization of wavelengths and energy values.
%% 39.02: Sketch a one-dimensional infinite potential well, indicating the length (or width) and the potential energy of the walls.
%% 39.03: For an electron, apply the relationship between the de Broglie wavelength l and the kinetic energy.
%% 39.04: For an electron in a one-dimensional infinite potential well, apply the relationship between the de Broglie wavelength $\lambda$, the well's length, and the quantum number $n$.
%% 39.05: For an electron in a one-dimensional infinite potential well, apply the relationship between the allowed energies $E_n$, the well length $L$, and the quantum number $n$.
%% 39.06: Sketch an energy-level diagram for an electron in a one-dimensional infinite potential well, indicating the ground state and several excited states.
%% 39.07: Identify that a trapped electron tends to be in its ground state, can be excited to a higher-energy state, and cannot exist between the allowed states.
%% 39.08: Calculate the energy change required for an electron to move between states: a quantum jump up or down an energy-level diagram.
%% 39.09: If a quantum jump involves light, identify that an upward jump requires the absorption of a photon (to increase the electron's energy) and a downward jump requires the emission of a photon (to reduce the electron’s energy).
%% 39.10: If a quantum jump involves light, apply the relationships between the energy change and the frequency and wavelength associated with the photon.
%% 39.11: Identify the emission and absorption spectra of an electron in a one-dimensional infinite potential well.


%% Halliday Multiple Choice Questions
%%--------------------------------------------------
\element{halliday-mc}{
\begin{question}{halliday-ch39-q01}
    If a wave function $\Psi$ for a particle moving along the $x$ axis is normalized,
        then:
    \begin{multicols}{2}
    \begin{choices}
        \wrongchoice{$\int\,|\Psi|^2\,\mathrm{d}t = 1$}
      \correctchoice{$\int\,|\Psi|^2\,\mathrm{d}x = 1$}
        \wrongchoice{$\dfrac{\partial\Psi}{\partial x} = 1$}
        \wrongchoice{$\dfrac{\partial\Psi}{\partial t} = 1$}
        \wrongchoice{$|\Psi|^2 = 1$}
    \end{choices}
    \end{multicols}
\end{question}
}

\element{halliday-mc}{
\begin{question}{halliday-ch39-q02}
    The energy of a particle in a one-dimensional trap with zero potential energy in the interior and infinite potential energy at the walls is proportional to ($n$ = quantum number):
    \begin{multicols}{3}
    \begin{choices}
        \wrongchoice{$n$}
        \wrongchoice{$\dfrac{1}{n}$}
        \wrongchoice{$\dfrac{1}{n^2}$}
        \wrongchoice{$\sqrt{n}$}
      \correctchoice{$n^2$}
    \end{choices}
    \end{multicols}
\end{question}
}

\element{halliday-mc}{
\begin{question}{halliday-ch39-q03}
    The ground state energy of an electron in a one-dimensional trap with zero potential energy in the interior and infinite potential energy at the walls is \SI{2.0}{\eV}. 
    If the width of the well is doubled,
        the ground state energy will be:
    \begin{multicols}{3}
    \begin{choices}
      \correctchoice{\SI{0.5}{\eV}}
        \wrongchoice{\SI{1.0}{\eV}}
        \wrongchoice{\SI{2.0}{\eV}}
        \wrongchoice{\SI{4.0}{\eV}}
        \wrongchoice{\SI{8.0}{\eV}}
    \end{choices}
    \end{multicols}
\end{question}
}

\element{halliday-mc}{
\begin{question}{halliday-ch39-q04}
    An electron is in a one-dimensional trap with zero potential energy in the interior and infinite potential energy at the walls. 
    The ratio $\dfrac{E_3}{E_1}$ of the energy for $n=3$ to that for $n=1$ is:
    \begin{multicols}{3}
    \begin{choices}
        \wrongchoice{\num{1/3}}
        \wrongchoice{\num{1/9}}
        \wrongchoice{\num{3/1}}
      \correctchoice{\num{9/1}}
        \wrongchoice{\num{1/1}}
    \end{choices}
    \end{multicols}
\end{question}
}

\element{halliday-mc}{
\begin{question}{halliday-ch39-q05}
    A particle is trapped in a one-dimensional well with infinite potential energy at the walls. 
    Three possible pairs of energy levels are:
    \begin{enumerate}
        \item $n=3$ and $n=1$
        \item $n=3$ and $n=2$
        \item $n=4$ and $n=3$
    \end{enumerate}
    Order these pairs according to the difference in energy,
        least to greatest.
    \begin{multicols}{3}
    \begin{choices}
        \wrongchoice{1, 2, 3}
        \wrongchoice{3, 2, 1}
      \correctchoice{2, 3, 1}
        \wrongchoice{1, 3, 2}
        \wrongchoice{3, 1, 2}
    \end{choices}
    \end{multicols}
\end{question}
}

\element{halliday-mc}{
\begin{question}{halliday-ch39-q06}
    Identical particles are trapped in one-dimensional wells with infinite potential energy at the walls. 
    The widths $L$ of the traps and the quantum numbers $n$ of the particles are:
    \begin{enumerate}
        \item $L=2L_0$, $n=2$
        \item $L=2L_0$, $n=4$
        \item $L=3L_0$, $n=3$
        \item $L=4L_0$, $n=2$
    \end{enumerate}
    Rank them according to the kinetic energies of the particles,
        least to greatest.
    \begin{choices}
        \wrongchoice{1, 2, 3, 4}
        \wrongchoice{4, 3, 2, 1}
        \wrongchoice{1 and 3 tied, then 2, 4}
      \correctchoice{4, 2, then 1 and 3 tied}
        \wrongchoice{1, 3, then 2 and 4 tied}
    \end{choices}
\end{question}
}

\element{halliday-mc}{
\begin{question}{halliday-ch39-q07}
    Four different particles are trapped in one-dimensional wells with infinite potential energy at their walls. 
    The masses of the particles and the width of the wells are:
    \begin{enumerate}
        \item mass = $4m_0$, width = $2L_0$
        \item mass = $2m_0$, width = $2L_0$
        \item mass = $4m_0$, width = $L_0$
        \item mass = $m_0$, width = $2L_0$
    \end{enumerate}
    Rank them according to the kinetic energies of the particles when they are in their ground states.
    \begin{choices}
        \wrongchoice{1, 2, 3, 4}
      \correctchoice{1, 2, 3 and 4 tied}
        \wrongchoice{1 and 2 tied, then 3, 4}
        \wrongchoice{4, 3, 2, 1}
        \wrongchoice{3, 1, 2, 4}
    \end{choices}
\end{question}
}

\element{halliday-mc}{
\begin{question}{halliday-ch39-q08}
    The ground state energy of an electron in a one-dimensional trap with zero potential energy in the interior and infinite potential energy at the walls:
    \begin{choices}
        \wrongchoice{is zero}
        \wrongchoice{decreases with temperature}
        \wrongchoice{increases with temperature}
      \correctchoice{is independent of temperature}
        \wrongchoice{oscillates with time}
    \end{choices}
\end{question}
}

\element{halliday-mc}{
\begin{question}{halliday-ch39-q09}
    An electron is in a one-dimensional trap with zero potential energy in the interior and infinite potential energy at the walls.
    A graph of its wave function $\Psi(x)$ versus $x$ is shown.
    \begin{center}
    \begin{tikzpicture}
        %% NOTE:
    \end{tikzpicture}
    \end{center}
    The value of quantum number $n$ is:
    \begin{multicols}{3}
    \begin{choices}
        \wrongchoice{\num{0}}
        \wrongchoice{\num{2}}
      \correctchoice{\num{4}}
        \wrongchoice{\num{6}}
        \wrongchoice{\num{8}}
    \end{choices}
    \end{multicols}
\end{question}
}

\element{halliday-mc}{
\begin{question}{halliday-ch39-q10}
    An electron is in a one-dimensional trap with zero potential energy in the interior and infinite potential energy at the walls.
    A graph of its probability density $P(x)$ versus $x$ is shown. 
    \begin{center}
    \begin{tikzpicture}
        %% NOTE:
    \end{tikzpicture}
    \end{center}
    The value of the quantum number $n$ is:
    \begin{multicols}{3}
    \begin{choices}
        \wrongchoice{\num{0}}
        \wrongchoice{\num{1}}
      \correctchoice{\num{2}}
        \wrongchoice{\num{3}}
        \wrongchoice{\num{4}}
    \end{choices}
    \end{multicols}
\end{question}
}

\element{halliday-mc}{
\begin{question}{halliday-ch39-q11}
    A particle is trapped in an infinite potential energy well. 
    It is in the state with quantum number $n=14$. 
    How many nodes does the probability density have
        (counting the nodes at the ends of the well)?
    \begin{multicols}{3}
    \begin{choices}
        \wrongchoice{none}
        \wrongchoice{7}
        \wrongchoice{13}
        \wrongchoice{14}
      \correctchoice{15}
    \end{choices}
    \end{multicols}
\end{question}
}

\element{halliday-mc}{
\begin{question}{halliday-ch39-q12}
    A particle is trapped in an infinite potential energy well. 
    It is in the state with quantum number $n=14$.
    How many maxima does the probability density have?
    \begin{multicols}{3}
    \begin{choices}
        \wrongchoice{none}
        \wrongchoice{7}
        \wrongchoice{13}
      \correctchoice{14}
        \wrongchoice{15}
    \end{choices}
    \end{multicols}
\end{question}
}

\element{halliday-mc}{
\begin{question}{halliday-ch39-q13}
    A particle is confined to a one-dimensional trap by infinite potential energy walls.
    Of the following states, designed by the quantum number $n$,
        for which one is the probability density greatest near the center of the well?
    \begin{multicols}{3}
    \begin{choices}
        \wrongchoice{$n = 2$}
      \correctchoice{$n = 3$}
        \wrongchoice{$n = 4$}
        \wrongchoice{$n = 5$}
        \wrongchoice{$n = 6$}
    \end{choices}
    \end{multicols}
\end{question}
}

\element{halliday-mc}{
\begin{question}{halliday-ch39-q14}
    Two one-dimensional traps have infinite potential energy at their walls Trap $A$ has width $L$ and trap $B$ has width $2L$. 
    For which value of the quantum number $n$ does a particle in trap $B$ have the same energy as a particle in the ground state of trap $A$?
    \begin{multicols}{3}
    \begin{choices}
        \wrongchoice{$n = 1$}
      \correctchoice{$n = 2$}
        \wrongchoice{$n = 3$}
        \wrongchoice{$n = 4$}
        \wrongchoice{$n = 5$}
    \end{choices}
    \end{multicols}
\end{question}
}

\element{halliday-mc}{
\begin{question}{halliday-ch39-q15}
    An electron is trapped in a deep well with a width of \SI{0.3}{\nano\meter}.
    If it is in the state with quantum number $n=3$ its kinetic energy is:
    \begin{multicols}{2}
    \begin{choices}
        \wrongchoice{\SI{6.0e-28}{\joule}}
        \wrongchoice{\SI{1.8e-27}{\joule}}
        \wrongchoice{\SI{6.7e-19}{\joule}}
        \wrongchoice{\SI{2.0e-18}{\joule}}
      \correctchoice{\SI{6.0e-18}{\joule}}
    \end{choices}
    \end{multicols}
\end{question}
}

\element{halliday-mc}{
\begin{question}{halliday-ch39-q16}
    An electron is in a one-dimensional well with finite potential energy barriers at the walls. 
    The matter wave:
    \begin{choices}
        \wrongchoice{is zero at the barriers}
        \wrongchoice{is zero everywhere within each barrier}
        \wrongchoice{is zero in the well}
      \correctchoice{extends into the barriers}
        \wrongchoice{is discontinuous at the barriers}
    \end{choices}
\end{question}
}

\element{halliday-mc}{
\begin{question}{halliday-ch39-q17}
    A particle is confined by finite potential energy walls to a one-dimensional trap from $x=0$ to $x=L$. 
    Its wave function in the region $x>L$ has the form:
    \begin{choices}
        \wrongchoice{$\Psi(x) = A\sin\left(kx\right)$}
        \wrongchoice{$\Psi(x) = A\mathrm{e}^{kx}$}
      \correctchoice{$\Psi(x) = A\mathrm{e}^{-kx}$}
        \wrongchoice{$\Psi(x) = A\mathrm{e}^{ikx}$}
        \wrongchoice{$\Psi(x) = \text{zero}$}
    \end{choices}
\end{question}
}

\element{halliday-mc}{
\begin{question}{halliday-ch39-q18}
    A particle is trapped in a finite potential energy well that is deep enough so that the electron can be in the state with $n=4$. 
    For this state how many nodes does the probability density have?
    \begin{multicols}{3}
    \begin{choices}
        \wrongchoice{none}
        \wrongchoice{1}
      \correctchoice{3}
        \wrongchoice{5}
        \wrongchoice{7}
    \end{choices}
    \end{multicols}
\end{question}
}

\element{halliday-mc}{
\begin{question}{halliday-ch39-q19}
    A particle in a certain finite potential energy well can have any of five quantized energy values and no more. 
    Which of the following would allow it to have any of six quantized energy levels?
    \begin{choices}
        \wrongchoice{Increase the momentum of the particle}
        \wrongchoice{Decrease the momentum of the particle}
        \wrongchoice{Decrease the well width}
      \correctchoice{Increase the well depth}
        \wrongchoice{Decrease the well depth}
    \end{choices}
\end{question}
}

\element{halliday-mc}{
\begin{question}{halliday-ch39-q20}
    A particle in a certain finite potential energy well can have any of five quantized energy values and no more. 
    Which of the following would allow it to have any of six quantized energy levels?
    \begin{choices}
        \wrongchoice{Increase the energy of the particle}
        \wrongchoice{Decrease the energy of the particle}
        \wrongchoice{Make the well shallower}
      \correctchoice{Make the well deeper}
        \wrongchoice{Make the well narrower}
    \end{choices}
\end{question}
}

\element{halliday-mc}{
\begin{question}{halliday-ch39-q21}
    An electron in an atom initially has an energy \SI{5.5}{\eV} above the ground state energy. 
    It drops to a state with energy \SI{3.2}{\eV} above the ground state energy and emits a photon in the process.
    The wave associated with the photon has a wavelength of:
    \begin{multicols}{2}
    \begin{choices}
      \correctchoice{\SI{5.4e-7}{\meter}}
        \wrongchoice{\SI{3.0e-7}{\meter}}
        \wrongchoice{\SI{1.7e-7}{\meter}}
        \wrongchoice{\SI{1.15e-7}{\meter}}
        \wrongchoice{\SI{1.0e-7}{\meter}}
    \end{choices}
    \end{multicols}
\end{question}
}

\element{halliday-mc}{
\begin{question}{halliday-ch39-q22}
    An electron in an atom drops from an energy level at \SI{-1.1e-18}{\joule} to an energy level at \SI{-2.4e-18}{\joule}.
    The wave associated with the emitted photon has a frequency of:
    \begin{multicols}{2}
    \begin{choices}
        \wrongchoice{\SI{2.0e17}{\hertz}}
      \correctchoice{\SI{2.0e15}{\hertz}}
        \wrongchoice{\SI{2.0e13}{\hertz}}
        \wrongchoice{\SI{2.0e11}{\hertz}}
        \wrongchoice{\SI{2.0e9}{\hertz}}
    \end{choices}
    \end{multicols}
\end{question}
}

\element{halliday-mc}{
\begin{question}{halliday-ch39-q23}
    An electron in an atom initially has an energy \SI{7.5}{\eV} above the ground state energy. 
    It drops to a state with an energy of \SI{3.2}{\eV} above the ground state energy and emits a photon in the process. 
    The momentum of the photon is:
    \begin{multicols}{2}
    \begin{choices}
        \wrongchoice{\SI{1.7e-27}{\kilo\gram\meter\per\second}}
      \correctchoice{\SI{2.3e-27}{\kilo\gram\meter\per\second}}
        \wrongchoice{\SI{4.0e-27}{\kilo\gram\meter\per\second}}
        \wrongchoice{\SI{5.7e-27}{\kilo\gram\meter\per\second}}
        \wrongchoice{\SI{8.0e-27}{\kilo\gram\meter\per\second}}
    \end{choices}
    \end{multicols}
\end{question}
}

\element{halliday-mc}{
\begin{question}{halliday-ch39-q24}
    The quantum number $n$ is most closely associated with what property of the electron in a hydrogen atom?
    \begin{choices}
        \wrongchoice{Energy}
        \wrongchoice{Orbital angular momentum}
        \wrongchoice{Spin angular momentum}
        \wrongchoice{Magnetic moment}
        \wrongchoice{$z$ component of angular momentum}
    \end{choices}
\end{question}
}

\element{halliday-mc}{
\begin{question}{halliday-ch39-q25}
    Take the potential energy of a hydrogen atom to be zero for infinite separation of the electron and proton. 
    Then, according to quantum theory the energy $E_n$ of a state with principal quantum number $n$ is proportional to:
    \begin{multicols}{2}
    \begin{choices}
        \wrongchoice{$n$}
        \wrongchoice{$n^2$}
        \wrongchoice{$\dfrac{1}{n}$}
      \correctchoice{$\dfrac{1}{n^2}$}
        \wrongchoice{none of the provided}
    \end{choices}
    \end{multicols}
\end{question}
}

\element{halliday-mc}{
\begin{question}{halliday-ch39-q26}
    The binding energy of an electron in the ground state in a hydrogen atom is about:
    \begin{multicols}{2}
    \begin{choices}
        \wrongchoice{\SI{13.6}{\eV}}
        \wrongchoice{\SI{3.4}{\eV}}
        \wrongchoice{\SI{10.2}{\eV}}
        \wrongchoice{\SI{1.0}{\eV}}
        \wrongchoice{\SI{27.2}{\eV}}
    \end{choices}
    \end{multicols}
\end{question}
}

\element{halliday-mc}{
\begin{question}{halliday-ch39-q27}
    Take the potential energy of a hydrogen atom to be zero for infinite separation of the electron and proton. 
    Then the ground state energy is \SI{-13.6}{\eV}.
    The energy of the first excited state is:
    \begin{multicols}{2}
    \begin{choices}
        \wrongchoice{zero}
      \correctchoice{\SI{-3.4}{\eV}}
        \wrongchoice{\SI{-6.8}{\eV}}
        \wrongchoice{\SI{-9.6}{\eV}}
        \wrongchoice{\SI{-27}{\eV}}
    \end{choices}
    \end{multicols}
\end{question}
}

\element{halliday-mc}{
\begin{question}{halliday-ch39-q28}
    Take the potential energy of a hydrogen atom to be zero for infinite separation of the electron and proton. 
    Then the ground state energy is \SI{-13.6}{\eV}.
    The negative sign indicates:
    \begin{choices}
        \wrongchoice{the kinetic energy is negative}
        \wrongchoice{the potential energy is positive}
        \wrongchoice{the electron might escape from the atom}
      \correctchoice{the electron and proton are bound together}
        \wrongchoice{none of the provided}
    \end{choices}
\end{question}
}

\element{halliday-mc}{
\begin{question}{halliday-ch39-q29}
    Take the potential energy of a hydrogen atom to be zero for infinite separation of the electron and proton.
    Then the ground state energy of a hydrogen atom is \SI{-13.6}{\eV}.
    When the electron is in the first excited state its excitation energy is:
    \begin{multicols}{3}
    \begin{choices}
        \wrongchoice{zero}
        \wrongchoice{\SI{3.4}{\eV}}
        \wrongchoice{\SI{6.8}{\eV}}
      \correctchoice{\SI{10.2}{\eV}}
        \wrongchoice{\SI{13.6}{\eV}}
    \end{choices}
    \end{multicols}
\end{question}
}

\element{halliday-mc}{
\begin{question}{halliday-ch39-q30}
    Take the potential energy of a hydrogen atom to be zero for infinite separation of the electron and proton. 
    Then the ground state energy is \SI{-13.6}{\eV}.
    When the electron is in the first excited state the ionization energy is:
    \begin{multicols}{3}
    \begin{choices}
        \wrongchoice{zero}
      \correctchoice{\SI{3.4}{\eV}}
        \wrongchoice{\SI{6.8}{\eV}}
        \wrongchoice{\SI{10.2}{\eV}}
        \wrongchoice{\SI{13.6}{\eV}}
    \end{choices}
    \end{multicols}
\end{question}
}

\element{halliday-mc}{
\begin{question}{halliday-ch39-q31}
    The diagram shows the energy levels for an electron in a certain atom. 
    \begin{center}
    \begin{tikzpicture}
        %% NOTE: evenly spaced?
    \end{tikzpicture}
    \end{center}
    Of the transitions shown,
        which represents the emission of a photon with the most energy?
    \begin{multicols}{3}
    \begin{choices}
        \wrongchoice{$I$}
        \wrongchoice{$J$}
      \correctchoice{$K$}
        \wrongchoice{$L$}
        \wrongchoice{$M$}
    \end{choices}
    \end{multicols}
\end{question}
}

\element{halliday-mc}{
\begin{question}{halliday-ch39-q32}
    When a hydrogen atom makes the transition from the second excited state to the ground state (at \SI{-13.6}{\eV}) the energy of the photon emitted is:
    \begin{multicols}{3}
    \begin{choices}
        \wrongchoice{zero}
        \wrongchoice{\SI{1.5}{\eV}}
        \wrongchoice{\SI{9.1}{\eV}}
      \correctchoice{\SI{12.1}{\eV}}
        \wrongchoice{\SI{13.6}{\eV}}
    \end{choices}
    \end{multicols}
\end{question}
}

\element{halliday-mc}{
\begin{question}{halliday-ch39-q33}
    The series limit for the Balmer series represents a transition $m\to n$,
        where $\left(m, n\right)$ is:
    \begin{multicols}{3}
    \begin{choices}
        \wrongchoice{$\left(2, 1\right)$}
        \wrongchoice{$\left(3, 2\right)$}
        \wrongchoice{$\left(\infty, 0\right)$}
        \wrongchoice{$\left(\infty, 1\right)$}
      \correctchoice{$\left(\infty, 2\right)$}
    \end{choices}
    \end{multicols}
\end{question}
}

\element{halliday-mc}{
\begin{question}{halliday-ch39-q34}
    The Balmer series of hydrogen is important because it:
    \begin{choices}
        \wrongchoice{is the only one for which the quantum theory can be used}
        \wrongchoice{is the only series that occurs for hydrogen}
      \correctchoice{is in the visible region}
        \wrongchoice{involves the lowest possible quantum number $n$}
        \wrongchoice{involves the highest possible quantum number $n$}
    \end{choices}
\end{question}
}

\element{halliday-mc}{
\begin{question}{halliday-ch39-q35}
    The principle of complementarity is due to:
    \begin{multicols}{2}
    \begin{choices}
        \wrongchoice{Einstein}
        \wrongchoice{Maxwell}
        \wrongchoice{Newton}
      \correctchoice{Bohr}
        \wrongchoice{Schr\"{o}dinger}
    \end{choices}
    \end{multicols}
\end{question}
}

\element{halliday-mc}{
\begin{question}{halliday-ch39-q36}
    Which of the following sets of quantum numbers is possible for an electron in a hydrogen atom?
    \begin{choices}
      \correctchoice{$n=4$, $l=3$,  $m_f=-3$}
        \wrongchoice{$n=4$, $l=4$,  $m_f=-2$}
        \wrongchoice{$n=5$, $l=-1$, $m_f=2$}
        \wrongchoice{$n=3$, $l=1$,  $m_f=-2$}
        \wrongchoice{$n=2$, $l=3$,  $m_f=-2$}
    \end{choices}
\end{question}
}

\element{halliday-mc}{
\begin{question}{halliday-ch39-q37}
    The wave function for an electron in a state with zero angular momentum:
    \begin{choices}
        \wrongchoice{is zero everywhere}
      \correctchoice{is spherically symmetric}
        \wrongchoice{depends on the angle from the $z$ axis}
        \wrongchoice{depends on the angle from the $x$ axis}
        \wrongchoice{is spherically symmetric for some shells and depends on the angle from the z axis for others}
    \end{choices}
\end{question}
}

\element{halliday-mc}{
\begin{question}{halliday-ch39-q38}
    Consider the following:
    \begin{enumerate}
        \item the probability density for an $l=0$ state
        \item the probability density for a state with $l=0$
        \item the average of the probability densities for all states in an $l=0$ subshell
    \end{enumerate}
    Of these which are spherically symmetric?
    \begin{multicols}{2}
    \begin{choices}
        \wrongchoice{only 1}
        \wrongchoice{only 2}
        \wrongchoice{only 1 and 2}
      \correctchoice{only 1 and 3}
        \wrongchoice{1, 2, and 3}
    \end{choices}
    \end{multicols}
\end{question}
}

\element{halliday-mc}{
\begin{question}{halliday-ch39-q39}
    If the wave function $\Psi$ is spherically symmetric then the radial probability density is given by:
    \begin{multicols}{3}
    \begin{choices}
        \wrongchoice{$4\pi r^2 \Psi$}
        \wrongchoice{$|\Psi|^2$}
      \correctchoice{$4\pi r^2 |\Psi|^2$}
        \wrongchoice{$4\pi |\Psi|^2$}
        \wrongchoice{$4\pi r |\Psi|^2$}
    \end{choices}
    \end{multicols}
\end{question}
}

\element{halliday-mc}{
\begin{question}{halliday-ch39-q40}
    If $P(r)$ is the radial probability density then the probability that the separation of the electron and proton is between $r$ and $r+\mathrm{d}r$ is:
    \begin{multicols}{2}
    \begin{choices}
      \correctchoice{$P\,\mathrm{d}r$}
        \wrongchoice{$|P|^2\,\mathrm{d}r$}
        \wrongchoice{$4\pi r^2P\,\mathrm{d}r$}
        \wrongchoice{$4\pi r^2 |P|\,\mathrm{d}r$}
        \wrongchoice{$4\pi |P|^2\,\mathrm{d}r$}
    \end{choices}
    \end{multicols}
\end{question}
}

\element{halliday-mc}{
\begin{question}{halliday-ch39-q41}
    The radial probability density for the electron in the ground state of a hydrogen atom has a peak at about:
    \begin{multicols}{2}
    \begin{choices}
        \wrongchoice{\SI{0.5}{\pico\meter}}
        \wrongchoice{\SI{5}{\pico\meter}}
      \correctchoice{\SI{50}{\pico\meter}}
        \wrongchoice{\SI{500}{\pico\meter}}
        \wrongchoice{\SI{5000}{\pico\meter}}
    \end{choices}
    \end{multicols}
\end{question}
}


\endinput


