
%%--------------------------------------------------
%% Halliday: Fundamentals of Physics
%%--------------------------------------------------


%% Chapter 31: Electromagnetic Oscillations
%%              and Alternating Current
%%--------------------------------------------------


%% Learning Objectives
%%--------------------------------------------------

%% 31.01: Sketch an $LC$ oscillator and explain which quantities oscillate and what constitutes one period of the oscillation.
%% 31.02: For an $LC$ oscillator, sketch graphs of the potential difference across the capacitor and the current through the inductor as functions of time, and indicate the period $T$ on each graph.
%% 31.03: Explain the analogy between a block--spring oscillator and an $LC$ oscillator.
%% 31.04: For an $LC$ oscillator, apply the relationships between the angular frequency $\omega$ (and the related frequency $f$ and period $T$) and the values of the inductance and capacitance.
%% 31.05: Starting with the energy of a block--spring system, explain the derivation of the differential equation for charge $q$ in an $LC$ oscillator and then identify the solution for $q(t)$.
%% 31.06: For an $LC$ oscillator, calculate the charge $q$ on the capacitor for any given time and identify the amplitude $Q$ of the charge oscillations.
%% 31.07: Starting from the equation giving the charge $q(t)$ on the capacitor in an $LC$ oscillator, find the current $i(t)$ in the inductor as a function of time.
%% 31.08: For an $LC$ oscillator, calculate the current $i$ in the inductor for any given time and identify the amplitude $I$ of the current oscillations.
%% 31.09: For an $LC$ oscillator, apply the relationship between the charge amplitude $Q$, the current amplitude $I$, and the angular frequency $\omega$.
%% 31.10: From the expressions for the charge q and the current $i$ in an $LC$ oscillator, find the magnetic field energy $U_B (t)$ and the electric field energy $U_E (t)$ and the total energy.
%% 31.11: For an $LC$ oscillator, sketch graphs of the magnetic field energy $U_B (t)$, the electric field energy $U_E (t)$, and the total energy, all as functions of time.
%% 31.12: Calculate the maximum values of the magnetic field energy $U_B$ and the electric field energy $U_E$ and also calculate the total energy.


%% Halliday Multiple Choice Questions
%%--------------------------------------------------
\element{halliday-mc}{
\begin{question}{halliday-ch31-q01}
    A charged capacitor and an inductor are connected in series. 
    At time $t=0$ the current is zero,
        but the capacitor is charged. 
    If $T$ is the period of the resulting oscillations,
        the next time after $t=0$ that the current is a maximum is:
    \begin{multicols}{3}
    \begin{choices}
        \wrongchoice{$T$}
      \correctchoice{$\dfrac{T}{4}$}
        \wrongchoice{$\dfrac{T}{2}$}
        \wrongchoice{$T$}
        \wrongchoice{$2T$}
    \end{choices}
    \end{multicols}
\end{question}
}

\element{halliday-mc}{
\begin{question}{halliday-ch31-q02}
    A charged capacitor and an inductor are connected in series. 
    At time $t=0$ the current is zero,
        but the capacitor is charged. 
    If $T$ is the period of the resulting oscillations,
        the next time after $t=0$ that the charge on the capacitor is a maximum is:
    \begin{multicols}{3}
    \begin{choices}
        \wrongchoice{$T$}
        \wrongchoice{$\dfrac{T}{4}$}
      \correctchoice{$\dfrac{T}{2}$}
        \wrongchoice{$T$}
        \wrongchoice{$2T$}
    \end{choices}
    \end{multicols}
\end{question}
}

\element{halliday-mc}{
\begin{question}{halliday-ch31-q03}
    A charged capacitor and an inductor are connected in series. 
    At time $t=0$ the current is zero,
        but the capacitor is charged. 
    If $T$ is the period of the resulting oscillations,
        the next time after $t=0$ that the voltage across the inductor is a maximum is:
    \begin{multicols}{3}
    \begin{choices}
        \wrongchoice{$T$}
        \wrongchoice{$\dfrac{T}{4}$}
      \correctchoice{$\dfrac{T}{2}$}
        \wrongchoice{$T$}
        \wrongchoice{$2T$}
    \end{choices}
    \end{multicols}
\end{question}
}

\element{halliday-mc}{
\begin{question}{halliday-ch31-q04}
    A charged capacitor and an inductor are connected in series. 
    At time $t=0$ the current is zero,
        but the capacitor is charged. 
    If $T$ is the period of the resulting oscillations,
        the next time after $t=0$ that the energy stored in the magnetic field of the inductor is a maximum is:
    \begin{multicols}{3}
    \begin{choices}
        \wrongchoice{$T$}
      \correctchoice{$\dfrac{T}{4}$}
        \wrongchoice{$\dfrac{T}{2}$}
        \wrongchoice{$T$}
        \wrongchoice{$2T$}
    \end{choices}
    \end{multicols}
\end{question}
}

\element{halliday-mc}{
\begin{question}{halliday-ch31-q05}
    A charged capacitor and an inductor are connected in series. 
    At time $t=0$ the current is zero,
        but the capacitor is charged. 
    If $T$ is the period of the resulting oscillations,
        the next time after $t=0$ that the energy stored in the electric field of the capacitor is a maximum is:
    \begin{multicols}{3}
    \begin{choices}
        \wrongchoice{$T$}
        \wrongchoice{$\dfrac{T}{4}$}
      \correctchoice{$\dfrac{T}{2}$}
        \wrongchoice{$T$}
        \wrongchoice{$2T$}
    \end{choices}
    \end{multicols}
\end{question}
}

\element{halliday-mc}{
\begin{question}{halliday-ch31-q06}
    A capacitor in an $LC$ oscillator has a maximum potential difference of \SI{15}{\volt} and a maximum energy of \SI{360}{\micro\joule}. 
    At a certain instant the energy in the capacitor is \SI{40}{\micro\joule}. 
    At that instant what is the potential difference across the capacitor?
    \begin{multicols}{3}
    \begin{choices}
        \wrongchoice{zero}
      \correctchoice{\SI{5}{\volt}}
        \wrongchoice{\SI{10}{\volt}}
        \wrongchoice{\SI{15}{\volt}}
        \wrongchoice{\SI{20}{\volt}}
    \end{choices}
    \end{multicols}
\end{question}
}

\element{halliday-mc}{
\begin{question}{halliday-ch31-q07}
    Which of the following has the greatest effect in decreasing the oscillation frequency of an $LC$ circuit? 
    Using instead:
    \begin{multicols}{2}
    \begin{choices}
        \wrongchoice{$\dfrac{L}{2}$ and $\dfrac{C}{2}$}
        \wrongchoice{$\dfrac{L}{2}$ and $\dfrac{2}{C}$}
        \wrongchoice{$2L$ and $\dfrac{C}{2}$}
      \correctchoice{$2L$ and $2C$}
        \wrongchoice{none of the provided}
    \end{choices}
    \end{multicols}
\end{question}
}

\element{halliday-mc}{
\begin{question}{halliday-ch31-q08}
    We desire to make an $LC$ circuit that oscillates at \SI{100}{\hertz} using an inductance of \SI{2.5}{\henry}. 
    We also need a capacitance of:
    \begin{multicols}{3}
    \begin{choices}
        \wrongchoice{\SI{1}{\farad}}
        \wrongchoice{\SI{1}{\milli\farad}}
      \correctchoice{\SI{1}{\micro\farad}}
        \wrongchoice{\SI{100}{\micro\farad}}
        \wrongchoice{\SI{1}{\pico\farad}}
    \end{choices}
    \end{multicols}
\end{question}
}

\element{halliday-mc}{
\begin{question}{halliday-ch31-q09}
    An $LC$ circuit consists of a \SI{1}{\micro\farad} capacitor and a \SI{4}{\milli\henry} inductor. 
    Its oscillation frequency is approximately:
    \begin{multicols}{2}
    \begin{choices}
        \wrongchoice{\SI{0.025}{\hertz}}
        \wrongchoice{\SI{25}{\hertz}}
        \wrongchoice{\SI{60}{\hertz}}
      \correctchoice{\SI{2500}{\hertz}}
        \wrongchoice{\SI{15 800}{\hertz}}
    \end{choices}
    \end{multicols}
\end{question}
}

\element{halliday-mc}{
\begin{question}{halliday-ch31-q10}
    An $LC$ circuit has an oscillation frequency of \SI{e5}{\hertz}. 
    If $C=\SI{0.1}{\micro\farad}$, then $L$ must be about:
    \begin{multicols}{2}
    \begin{choices}
        \wrongchoice{\SI{10}{\milli\henry}}
        \wrongchoice{\SI{1}{\milli\henry}}
      \correctchoice{\SI{25}{\micro\henry}}
        \wrongchoice{\SI{2.5}{\micro\henry}}
        \wrongchoice{\SI{1}{\pico\henry}}
    \end{choices}
    \end{multicols}
\end{question}
}

\element{halliday-mc}{
\begin{question}{halliday-ch31-q11}
    In the circuit shown, switch $S$ is first pushed up to charge the capacitor. 
    When $S$ is then pushed down,
    \begin{center}
    \begin{circuitikz}
        %% NOTE: circuitk
    \end{circuitikz}
    \end{center}
        the current in the circuit will oscillate at a frequency of:
    \begin{multicols}{2}
    \begin{choices}
      \correctchoice{\SI{318}{\hertz}}
        \wrongchoice{\SI{0.01}{\hertz}}
        \wrongchoice{\SI{12 500}{\hertz}}
        \wrongchoice{\SI{2000}{\hertz}}
        \wrongchoice{depends on $V_0$}
    \end{choices}
    \end{multicols}
\end{question}
}

\element{halliday-mc}{
\begin{question}{halliday-ch31-q12}
    Radio receivers are usually tuned by adjusting the capacitor of an $LC$ circuit. 
    If $C=C_1$ for a frequency of \SI{600}{\kilo\hertz},
        then for a frequency of \SI{1200}{\kilo\hertz} one must adjust $C$ to:
    \begin{multicols}{3}
    \begin{choices}
        \wrongchoice{$\dfrac{C_1}{2}$}
      \correctchoice{$\dfrac{C 1}{4}$}
        \wrongchoice{$2C_1$}
        \wrongchoice{$4C_1$}
        \wrongchoice{$\sqrt{2}C_1$}
    \end{choices}
    \end{multicols}
\end{question}
}

\element{halliday-mc}{
\begin{question}{halliday-ch31-q13}
    An $LC$ series circuit with an inductance $L$ and a capacitance $C$ has an oscillation frequency $f$.
    Two inductors, each with inductance $L$, and two capacitors,
        each with capacitance $C$, are all wired in series and the circuit is completed. 
    The oscillation frequency is:
    \begin{multicols}{3}
    \begin{choices}
        \wrongchoice{$\dfrac{f}{4}$}
        \wrongchoice{$\dfrac{f}{2}$}
      \correctchoice{$f$}
        \wrongchoice{$2f$}
        \wrongchoice{$4f$}
    \end{choices}
    \end{multicols}
\end{question}
}

\element{halliday-mc}{
\begin{question}{halliday-ch31-q14}
    The electrical analog of a spring constant $k$ is:
    \begin{multicols}{3}
    \begin{choices}
        \wrongchoice{$L$}
        \wrongchoice{$\dfrac{1}{L}$}
        \wrongchoice{$C$}
      \correctchoice{$\dfrac{1}{C}$}
        \wrongchoice{$R$}
    \end{choices}
    \end{multicols}
\end{question}
}

\element{halliday-mc}{
\begin{question}{halliday-ch31-q15}
    Consider the mechanical system consisting of two springs and a block,
        as shown.
    \begin{center}
    \begin{circuitikz}
        %% NOTE: tikz
    \end{circuitikz}
    \end{center}
    Which one of the five electrical circuits is the analog of the mechanical system?
    \begin{multicols}{2}
    \begin{choices}
        %% NOTE: ANS is A
        \wrongchoice{
            \begin{circuitikz}
                %% NOTE: tikz
            \end{circuitikz}
        }
    \end{choices}
    \end{multicols}
\end{question}
}

\element{halliday-mc}{
\begin{question}{halliday-ch31-q16}
    A \SI{150}{\gram} block on the end of a spring with a spring constant of \SI{35}{\newton\per\meter} is pulled aside \SI{25}{\centi\meter} and released from rest. 
    In the electrical analog the initial charge on the capacitor is:
    \begin{multicols}{3}
    \begin{choices}
        \wrongchoice{\SI{0.15}{\coulomb}}
        \wrongchoice{\SI{6.67}{\coulomb}}
      \correctchoice{\SI{0.025}{\coulomb}}
        \wrongchoice{\SI{40}{\coulomb}}
        \wrongchoice{\SI{35}{\coulomb}}
    \end{choices}
    \end{multicols}
\end{question}
}

\element{halliday-mc}{
\begin{question}{halliday-ch31-q17}
    A \SI{150}{\gram} block on the end of a spring with a spring constant of \SI{35}{\newton\per\meter} is pulled aside \SI{25}{\centi\meter} and released from rest. 
    In the electrical analog the initial charge on the capacitor is:
    \begin{multicols}{3}
    \begin{choices}
        \wrongchoice{\SI{0.15}{\coulomb}}
        \wrongchoice{\SI{6.67}{\coulomb}}
      \correctchoice{\SI{0.025}{\coulomb}}
        \wrongchoice{\SI{40}{\coulomb}}
        \wrongchoice{\SI{35}{\coulomb}}
    \end{choices}
    \end{multicols}
\end{question}
}

\element{halliday-mc}{
\begin{question}{halliday-ch31-q18}
    A capacitor in an $LC$ oscillator has a maximum potential difference of \SI{15}{\volt} and a maximum energy of \SI{360}{\micro\joule}. 
    At a certain instant the energy in the capacitor is \SI{40}{\micro\joule}. 
    At that instant what is the potential difference across the capacitor?
    \begin{multicols}{3}
    \begin{choices}
        \wrongchoice{zero}
      \correctchoice{\SI{5}{\volt}}
        \wrongchoice{\SI{10}{\volt}}
        \wrongchoice{\SI{15}{\volt}}
        \wrongchoice{\SI{20}{\volt}}
    \end{choices}
    \end{multicols}
\end{question}
}

\element{halliday-mc}{
\begin{question}{halliday-ch31-q19}
    A capacitor in an $LC$ oscillator has a maximum potential difference of \SI{15}{\volt} and a maximum energy of \SI{360}{\micro\joule}. 
    At a certain instant the energy in the capacitor is \SI{40}{\micro\joule}.
    At that instant what is the emf induced in the inductor?
    \begin{multicols}{3}
    \begin{choices}
        \wrongchoice{zero}
        \wrongchoice{\SI{5}{\volt}}
      \correctchoice{\SI{10}{\volt}}
        \wrongchoice{\SI{15}{\volt}}
        \wrongchoice{\SI{20}{\volt}}
    \end{choices}
    \end{multicols}
\end{question}
}

\element{halliday-mc}{
\begin{question}{halliday-ch31-q20}
    In an oscillating $LC$ circuit,
        the total stored energy is $U$.
    The maximum energy stored in the capacitor during one cycle is:
    \begin{multicols}{3}
    \begin{choices}
        \wrongchoice{$\dfrac{U}{2}$}
        \wrongchoice{$\dfrac{U}{\sqrt{2}}$}
      \correctchoice{$U$}
        \wrongchoice{$\dfrac{U}{2\pi}$}
        \wrongchoice{$\dfrac{U}{\pi}$}
    \end{choices}
    \end{multicols}
\end{question}
}

\element{halliday-mc}{
\begin{question}{halliday-ch31-q21}
    In an oscillating $LC$ circuit,
        the total stored energy is $U$ and the maximum charge on the capacitor is $Q$. 
    When the charge on the capacitor is $Q/2$,
        the energy stored in the inductor is:
    \begin{multicols}{3}
    \begin{choices}
        \wrongchoice{$\dfrac{U}{2}$}
        \wrongchoice{$\dfrac{U}{4}$}
        \wrongchoice{$\dfrac{4U}{3}$}
        \wrongchoice{$\dfrac{3U}{2}$}
      \correctchoice{$\dfrac{3U}{4}$}
    \end{choices}
    \end{multicols}
\end{question}
}

\element{halliday-mc}{
\begin{question}{halliday-ch31-q22}
    The total energy in an $LC$ circuit is \SI{5.0e-6}{\joule}.
    If $C=\SI{15}{\micro\farad}$ the charge on the capacitor is:
    \begin{multicols}{3}
    \begin{choices}
        \wrongchoice{\SI{0.82}{\micro\coulomb}}
        \wrongchoice{\SI{8.5}{\micro\coulomb}}
      \correctchoice{\SI{12}{\micro\coulomb}}
        \wrongchoice{\SI{17}{\micro\coulomb}}
        \wrongchoice{\SI{24}{\micro\coulomb}}
    \end{choices}
    \end{multicols}
\end{question}
}

\element{halliday-mc}{
\begin{question}{halliday-ch31-q23}
    The total energy in an $LC$ circuit is \SI{5.0e-6}{\joule}.
    If $L=\SI{25}{\milli\henry}$ the maximum current is:
    \begin{multicols}{3}
    \begin{choices}
        \wrongchoice{\SI{10}{\milli\ampere}}
        \wrongchoice{\SI{14}{\milli\ampere}}
      \correctchoice{\SI{20}{\milli\ampere}}
        \wrongchoice{\SI{28}{\milli\ampere}}
        \wrongchoice{\SI{40}{\milli\ampere}}
    \end{choices}
    \end{multicols}
\end{question}
}

\element{halliday-mc}{
\begin{question}{halliday-ch31-q24}
    At time $t=0$ the charge on the \SI{50}{\micro\farad} capacitor in an $LC$ circuit is \SI{15}{\micro\coulomb} and there is no current. 
    If the inductance is \SI{20}{\milli\henry} the maximum current is:
    \begin{multicols}{2}
    \begin{choices}
        \wrongchoice{\SI{15}{\nano\ampere}}
        \wrongchoice{\SI{15}{\micro\ampere}}
        \wrongchoice{\SI{6.7}{\milli\ampere}}
      \correctchoice{\SI{15}{\ampere}}
        \wrongchoice{\SI{15}{\ampere}}
    \end{choices}
    \end{multicols}
\end{question}
}

\element{halliday-mc}{
\begin{question}{halliday-ch31-q25}
    An $LC$ circuit has an inductance of \SI{20}{\milli\henry} and a capacitance of \SI{5.0}{\micro\farad}.
    At time $t=0$ the charge on the capacitor is \SI{3.0}{\micro\coulomb} and the current is \SI{7.0}{\milli\ampere}.
    The total energy is:
    \begin{multicols}{2}
    \begin{choices}
        \wrongchoice{\SI{4.1e-7}{\joule}}
        \wrongchoice{\SI{4.9e-7}{\joule}}
        \wrongchoice{\SI{9.0e-7}{\joule}}
      \correctchoice{\SI{1.4e-6}{\joule}}
        \wrongchoice{\SI{2.8e-6}{\joule}}
    \end{choices}
    \end{multicols}
\end{question}
}

\element{halliday-mc}{
\begin{question}{halliday-ch31-q26}
    An $LC$ circuit has a capacitance of \SI{30}{\micro\farad} and an inductance of \SI{15}{\milli\henry}. 
    At time $t=0$ the charge on the capacitor is \SI{10}{\micro\coulomb} and the current is \SI{20}{\milli\ampere}. 
    The maximum charge on the capacitor is:
    \begin{multicols}{3}
    \begin{choices}
        \wrongchoice{\SI{8.9}{\micro\coulomb}}
        \wrongchoice{\SI{10}{\micro\coulomb}}
        \wrongchoice{\SI{12}{\micro\coulomb}}
      \correctchoice{\SI{17}{\micro\coulomb}}
        \wrongchoice{\SI{24}{\micro\coulomb}}
    \end{choices}
    \end{multicols}
\end{question}
}

\element{halliday-mc}{
\begin{question}{halliday-ch31-q27}
    An $LC$ circuit has an inductance of \SI{15}{\milli\henry} and a capacitance of \SI{10}{\micro\farad}. 
    At one instant the charge on the capacitor is \SI{25}{\micro\coulomb}. 
    At that instant the current is changing at the rate of:
    \begin{multicols}{2}
    \begin{choices}
        \wrongchoice{zero}
        \wrongchoice{\SI{1.7e-8}{\ampere\per\second}}
        \wrongchoice{\SI{5.9e-3}{\ampere\per\second}}
        \wrongchoice{\SI{3.8e-2}{\ampere\per\second}}
      \correctchoice{\SI{170}{\ampere\per\second}}
    \end{choices}
    \end{multicols}
\end{question}
}

\element{halliday-mc}{
\begin{question}{halliday-ch31-q28}
    An $LC$ circuit has a capacitance of \SI{30}{\micro\farad} and an inductance of \SI{15}{\milli\henry}. 
    At time $t=0$ the charge on the capacitor is \SI{10}{\micro\coulomb} and the current is \SI{20}{\milli\ampere}. 
    The maximum current is:
    \begin{multicols}{3}
    \begin{choices}
        \wrongchoice{\SI{18}{\milli\ampere}}
        \wrongchoice{\SI{20}{\milli\ampere}}
      \correctchoice{\SI{25}{\milli\ampere}}
        \wrongchoice{\SI{35}{\milli\ampere}}
        \wrongchoice{\SI{42}{\milli\ampere}}
    \end{choices}
    \end{multicols}
\end{question}
}

\element{halliday-mc}{
\begin{question}{halliday-ch31-q29}
    The graphs show the total electromagnetic energy in two $RLC$ circuits as functions of time.
    \begin{center}
    \begin{tikzpicture}
        %% NOTE:
    \end{tikzpicture}
    \end{center}
    Which of the following statements might be true?
    \begin{choices}
      \correctchoice{Circuit 1 has a smaller resistance and a larger inductance}
        \wrongchoice{Circuit 1 has a larger resistance and a smaller inductance}
        \wrongchoice{Circuit 1 has the same resistance and a larger inductance}
        \wrongchoice{Circuit 1 has a larger resistance and a larger capacitance}
        \wrongchoice{Circuit 1 has the same resistance and a smaller capacitance}
    \end{choices}
\end{question}
}

\element{halliday-mc}{
\begin{question}{halliday-ch31-q30}
    An $RLC$ circuit has a resistance of \SI{200}{\ohm} and an inductance of \SI{15}{\milli\henry}.
    Its oscillation frequency is \SI{7000}{\hertz}.
    At time $t=0$ the current is \SI{25}{\milli\ampere} and there is no charge on the capacitor. 
    After five complete cycles the current is:
    \begin{multicols}{2}
    \begin{choices}
        \wrongchoice{zero}
        \wrongchoice{\SI{1.8e-6}{\ampere}}
      \correctchoice{\SI{2.1e-4}{\ampere}}
        \wrongchoice{\SI{2.3e-3}{\ampere}}
        \wrongchoice{\SI{2.5e-2}{\ampere}}
    \end{choices}
    \end{multicols}
\end{question}
}

\element{halliday-mc}{
\begin{question}{halliday-ch31-q31}
    An $RLC$ circuit has an inductance of \SI{25}{\milli\henry} and a capacitance of \SI{5.0}{\micro\farad}. 
    The charge on the capacitor does \emph{not} oscillate but rather decays exponentially to zero. 
    The resistance in the circuit must be:
    \begin{choices}
      \correctchoice{greater than or equal to \SI{20 000}{\ohm}}
        \wrongchoice{less than \SI{20 000}{\ohm} but greater than \SI{10 000}{\ohm}}
        \wrongchoice{less than \SI{10 000}{\ohm} but greater than \SI{5 000}{\ohm}}
        \wrongchoice{less than \SI{5 000}{\ohm} but greater than zero}
        \wrongchoice{zero}
    \end{choices}
\end{question}
}

\element{halliday-mc}{
\begin{question}{halliday-ch31-q32}
    A series circuit with an inductance of \SI{15}{\milli\henry},
        a capacitance of \SI{35}{\micro\farad},
        and a resistance of \SI{5.0}{\ohm} contains a sinusoidal source of emf with a frequency of \SI{500}{\hertz}. 
    The frequency with which the charge on the capacitor oscillates is:
    \begin{choices}
      \correctchoice{\SI{500}{\hertz}}
        \wrongchoice{\SI{1.4}{\kilo\hertz}}
        \wrongchoice{greater than \SI{1.4}{\kilo\hertz}}
        \wrongchoice{less than \SI{500}{\hertz}}
        \wrongchoice{between \SI{500}{\hertz} and \SI{1.4}{\kilo\hertz}}
    \end{choices}
\end{question}
}

\element{halliday-mc}{
\begin{question}{halliday-ch31-q33}
    The rapid exponential decay in just a few cycles of the charge on the plates of capacitor in an $RLC$ circuit might be due to:
    \begin{choices}
        \wrongchoice{a large inductance}
        \wrongchoice{a large capacitance}
        \wrongchoice{a small capacitance}
      \correctchoice{a large resistance}
        \wrongchoice{a small resistance}
    \end{choices}
\end{question}
}

\element{halliday-mc}{
\begin{question}{halliday-ch31-q34}
    An $RLC$ circuit has a capacitance of \SI{12}{\micro\farad},
        an inductance of \SI{25}{\milli\henry},
        and a resistance of \SI{60}{\ohm}.  
    The current oscillates with an angular frequency of:
    \begin{multicols}{2}
    \begin{choices}
        \wrongchoice{\SI{1.2e3}{\radian\per\second}}
      \correctchoice{\SI{1.4e3}{\radian\per\second}}
        \wrongchoice{\SI{1.8e3}{\radian\per\second}}
        \wrongchoice{\SI{2.2e3}{\radian\per\second}}
        \wrongchoice{\SI{2.6e3}{\radian\per\second}}
    \end{choices}
    \end{multicols}
\end{question}
}

\element{halliday-mc}{
\begin{question}{halliday-ch31-q35}
    The angular frequency of a certain $RLC$ series circuit is $\omega_0$. 
    A source of sinusoidal emf,
        with angular frequency $2\omega$,
        is inserted into the circuit. 
    After transients die out the angular frequency of the current oscillations is:
    \begin{multicols}{3}
    \begin{choices}
        \wrongchoice{$\dfrac{\omega_0}{2}$}
        \wrongchoice{$\omega_0$}
      \correctchoice{$2\omega_0$}
        \wrongchoice{$\dfrac{3\omega_0}{2}$}
        \wrongchoice{$3\omega_0$}
    \end{choices}
    \end{multicols}
\end{question}
}

\element{halliday-mc}{
\begin{question}{halliday-ch31-q36}
    The angular frequency of a certain RLC series circuit is $\omega_0$. 
    A source of sinusoidal emf,
        with angular frequency $\omega$,
        is inserted into the circuit and $\omega$ is varied while the amplitude of the source is held constant. 
    For which of the following values of $\omega$ is the amplitude of the current oscillations the greatest?
    \begin{multicols}{2}
    \begin{choices}
        \wrongchoice{$\dfrac{\omega_0}{5}$}
        \wrongchoice{$\dfrac{\omega_0}{2}$}
      \correctchoice{$\omega_0$}
        \wrongchoice{$2\omega_0$}
        \wrongchoice{None of them (they all produce the same current amplitude)}
    \end{choices}
    \end{multicols}
\end{question}
}

\element{halliday-mc}{
\begin{question}{halliday-ch31-q37}
    An $RLC$ circuit has a sinusoidal source of emf. 
    The average rate at which the source supplies energy is \SI{5}{\nano\watt}.
    This must also be:
    \begin{choices}
        \wrongchoice{the average rate at which energy is stored in the capacitor}
        \wrongchoice{the average rate at which energy is stored in the inductor}
      \correctchoice{the average rate at which energy is dissipated in the resistor}
        \wrongchoice{twice the average rate at which energy is stored in the capacitor}
        \wrongchoice{three times the average rate at which energy is stored in the inductor}
    \end{choices}
\end{question}
}

\element{halliday-mc}{
\begin{question}{halliday-ch31-q38}
    In a purely capacitive circuit the current:
    \begin{choices}
      \correctchoice{leads the voltage by one-fourth of a cycle}
        \wrongchoice{leads the voltage by one-half of a cycle}
        \wrongchoice{lags the voltage by one-fourth of a cycle}
        \wrongchoice{lags the voltage by one-half of a cycle}
        \wrongchoice{is in phase with the potential difference across the plates}
    \end{choices}
\end{question}
}

\element{halliday-mc}{
\begin{question}{halliday-ch31-q39}
    In a purely resistive circuit the current:
    \begin{choices}
        \wrongchoice{leads the voltage by one-fourth of a cycle}
        \wrongchoice{leads the voltage by one-half of a cycle}
        \wrongchoice{lags the voltage by one-fourth of a cycle}
        \wrongchoice{lags the voltage by one-half of a cycle}
      \correctchoice{is in phase with the voltage}
    \end{choices}
\end{question}
}

\element{halliday-mc}{
\begin{question}{halliday-ch31-q40}
    In a purely inductive circuit, the current lags the voltage by:
    \begin{choices}
        \wrongchoice{zero}
      \correctchoice{one-fourth of a cycle}
        \wrongchoice{one-half of a cycle}
        \wrongchoice{three-fourths of a cycle}
        \wrongchoice{one cycle}
    \end{choices}
\end{question}
}

\element{halliday-mc}{
\begin{question}{halliday-ch31-q41}
    A series $RL$ circuit is connected to an emf source of angular frequency $\omega$. 
    The current:
    \begin{choices}
        \wrongchoice{leads the applied emf by $\tan^{-1}\left(\dfrac{\omega L}{R}\right)$}
      \correctchoice{lags the applied emf by $\tan^{-1}\left(\dfrac{\omega L}{R}\right)$}
        \wrongchoice{lags the applied emf by $\tan^{-1}\left(\dfrac{\omega R}{L}\right)$}
        \wrongchoice{leads the applied emf by $\tan^{-1}\left(\dfrac{\omega R}{L}\right)$}
        \wrongchoice{is zero}
    \end{choices}
\end{question}
}

\element{halliday-mc}{
\begin{question}{halliday-ch31-q42}
    A series $RC$ circuit is connected to an emf source having angular frequency $\omega$. 
    The current:
    \begin{choices}
      \correctchoice{leads the source emf by $\tan^{-1}\left(\dfrac{1}{\omega CR}\right)$}
        \wrongchoice{lags the source emf by $\tan^{-1}\left(\dfrac{1}{\omega CR}\right)$}
        \wrongchoice{leads the source emf by $\tan^{-1}\left(\omega CR\right)$}
        \wrongchoice{lags the source emf by $\tan^{-1}\left(\omega CR\right)$}
        \wrongchoice{leads the source emf by $\dfrac{\pi}{4}$}
    \end{choices}
\end{question}
}

\element{halliday-mc}{
\begin{question}{halliday-ch31-q43}
    In an $RLC$ series circuit,
        which is connected to a source of emf $E_m\cos\left(\omega t\right)$,
        the current lags the voltage by \ang{45} if:
    \begin{multicols}{2}
    \begin{choices}
        \wrongchoice{$R = \dfrac{1}{\omega C - \omega L}$}
        \wrongchoice{$R = \dfrac{1}{\omega L - \omega C}$}
      \correctchoice{$R = \omega L - \dfrac{1}{\omega C}$}
        \wrongchoice{$R = \omega C - \dfrac{1}{\omega L}$}
        \wrongchoice{$\omega L = \dfrac{1}{\omega C}$}
    \end{choices}
    \end{multicols}
\end{question}
}

\element{halliday-mc}{
\begin{question}{halliday-ch31-q44}
    A coil has a resistance of \SI{60}{\ohm} and an impedance of \SI{100}{\ohm}.
    Its reactance is:
    \begin{multicols}{3}
    \begin{choices}
        \wrongchoice{\SI{40}{\ohm}}
        \wrongchoice{\SI{60}{\ohm}}
      \correctchoice{\SI{80}{\ohm}}
        \wrongchoice{\SI{117}{\ohm}}
        \wrongchoice{\SI{160}{\ohm}}
    \end{choices}
    \end{multicols}
\end{question}
}

\element{halliday-mc}{
\begin{question}{halliday-ch31-q45}
    The reactance of a \SI{35}{\micro\farad} capacitor connected to a \SI{400}{\hertz} generator is:
    \begin{multicols}{3}
    \begin{choices}
        \wrongchoice{zero}
        \wrongchoice{\SI{0.014}{\ohm}}
        \wrongchoice{\SI{0.088}{\ohm}}
      \correctchoice{\SI{11}{\ohm}}
        \wrongchoice{\SI{71}{\ohm}}
    \end{choices}
    \end{multicols}
\end{question}
}

\element{halliday-mc}{
\begin{question}{halliday-ch31-q46}
    A \SI{35}{\micro\farad} capacitor is connected to a source of sinusoidal emf with a frequency of \SI{400}{\hertz} and a maximum emf of \SI{20}{\volt}. 
    The maximum current is:
    \begin{multicols}{3}
    \begin{choices}
        \wrongchoice{zero}
        \wrongchoice{\SI{0.28}{\ampere}}
      \correctchoice{\SI{1.8}{\ampere}}
        \wrongchoice{\SI{230}{\ampere}}
        \wrongchoice{\SI{1400}{\ampere}}
    \end{choices}
    \end{multicols}
\end{question}
}

\element{halliday-mc}{
\begin{question}{halliday-ch31-q47}
    A \SI{45}{\milli\henry} inductor is connected to a source of sinusoidal emf with a frequency of \SI{400}{\hertz} and a maximum emf of \SI{20}{\volt}.
    The maximum current is:
    \begin{multicols}{2}
    \begin{choices}
        \wrongchoice{zero}
        \wrongchoice{\SI{0.18}{\ampere}}
        \wrongchoice{\SI{1.1}{\ampere}}
        \wrongchoice{\SI{360}{\ampere}}
        \wrongchoice{\SI{2300}{\ampere}}
    \end{choices}
    \end{multicols}
\end{question}
}

\element{halliday-mc}{
\begin{question}{halliday-ch31-q48}
    The impedance of an $RLC$ series circuit is definitely increased if:
    \begin{multicols}{2}
    \begin{choices}
        \wrongchoice{$C$ decreases}
        \wrongchoice{$L$ increases}
        \wrongchoice{$L$ decreases}
      \correctchoice{$R$ increases}
        \wrongchoice{$R$ decreases}
    \end{choices}
    \end{multicols}
\end{question}
}

\element{halliday-mc}{
\begin{question}{halliday-ch31-q49}
    An $RLC$ series circuit has $R=\SI{4}{\ohm}$, $X_C=\SI{3}{\ohm}$, and $X_L=\SI{6}{\ohm}$. 
    The impedance of this circuit is:
    \begin{multicols}{3}
    \begin{choices}
      \correctchoice{\SI{5}{\ohm}}
        \wrongchoice{\SI{7}{\ohm}}
        \wrongchoice{\SI{9.8}{\ohm}}
        \wrongchoice{\SI{13}{\ohm}}
        \wrongchoice{\SI{7.8}{\ohm}}
    \end{choices}
    \end{multicols}
\end{question}
}

\element{halliday-mc}{
\begin{question}{halliday-ch31-q50}
    The impedance of the circuit shown is:
    \begin{center}
    \begin{tikzpicture}
        %% NOTE:
    \end{tikzpicture}
    \end{center}
    \begin{multicols}{3}
    \begin{choices}
        \wrongchoice{\SI{21}{\ohm}}
        \wrongchoice{\SI{50}{\ohm}}
        \wrongchoice{\SI{63}{\ohm}}
      \correctchoice{\SI{65}{\ohm}}
        \wrongchoice{\SI{98}{\ohm}}
    \end{choices}
    \end{multicols}
\end{question}
}

\element{halliday-mc}{
\begin{question}{halliday-ch31-q51}
    An electric motor, under load, has an effective resistance of \SI{30}{\ohm} and an inductive reactance of \SI{40}{\ohm}. 
    When powered by a source with a maximum voltage of \SI{420}{\volt},
        the maximum current is:
    \begin{multicols}{3}
    \begin{choices}
        \wrongchoice{\SI{6.0}{\ampere}}
      \correctchoice{\SI{8.4}{\ampere}}
        \wrongchoice{\SI{10.5}{\ampere}}
        \wrongchoice{\SI{12.0}{\ampere}}
        \wrongchoice{\SI{14.0}{\ampere}}
    \end{choices}
    \end{multicols}
\end{question}
}

\element{halliday-mc}{
\begin{question}{halliday-ch31-q52}
    An $RL$ series circuit is connected to an ac generator with a maximum emf of \SI{20}{\volt}.
    If the maximum potential difference across the resistor is \SI{16}{\volt},
        then the maximum potential difference across the inductor is:
    \begin{multicols}{3}
    \begin{choices}
        \wrongchoice{\SI{2}{\volt}}
        \wrongchoice{\SI{4}{\volt}}
      \correctchoice{\SI{12}{\volt}}
        \wrongchoice{\SI{25.6}{\volt}}
        \wrongchoice{\SI{36}{\volt}}
    \end{choices}
    \end{multicols}
\end{question}
}

\element{halliday-mc}{
\begin{question}{halliday-ch31-q53}
    When the amplitude of the oscillator in a series $RLC$ circuit is doubled:
    \begin{choices}
        \wrongchoice{the impedance is doubled}
        \wrongchoice{the voltage across the capacitor is halved}
        \wrongchoice{the capacitive reactance is halved}
        \wrongchoice{the power factor is doubled}
      \correctchoice{the current amplitude is doubled}
    \end{choices}
\end{question}
}

\element{halliday-mc}{
\begin{question}{halliday-ch31-q54}
    When the frequency of the oscillator in a series $RLC$ circuit is doubled:
    \begin{choices}
        \wrongchoice{the capacitive reactance is doubled}
      \correctchoice{the capacitive reactance is halved}
        \wrongchoice{the impedance is doubled}
        \wrongchoice{the current amplitude is doubled}
        \wrongchoice{the current amplitude is halved}
    \end{choices}
\end{question}
}

\element{halliday-mc}{
\begin{question}{halliday-ch31-q55}
    In an $RLC$ series circuit,
        the source voltage is leading the current at a given frequency $f$. 
    If $f$ is lowered slightly, then the circuit impedance will:
    \begin{choices}
        \wrongchoice{increase}
      \correctchoice{decrease}
        \wrongchoice{remain the same}
        \wrongchoice{need to know the amplitude of the source voltage}
        \wrongchoice{need to know whether the phase angle is larger or smaller than \ang{45}}
    \end{choices}
\end{question}
}

\element{halliday-mc}{
\begin{question}{halliday-ch31-q56}
    In the diagram, the function $y(t)=y_m\sin\left(\omega t\right)$ is plotted as a solid curve.
    The other three curves have the form $y(t)=y_m\sin\left(\omega t + \phi\right)$,
        where $\phi$ is between $\pi/2$ and $+\pi/2$.
    \begin{center}
    \begin{tikzpicture}
        %% NOTE:
    \end{tikzpicture}
    \end{center}
    Rank the curves according to the value of $\phi$,
        from the most negative to the most positive.
    \begin{multicols}{2}
    \begin{choices}
        \wrongchoice{1, 2, 3}
        \wrongchoice{2, 3, 1}
        \wrongchoice{3, 2, 1}
      \correctchoice{1, 3, 2}
        \wrongchoice{2, 1, 3}
    \end{choices}
    \end{multicols}
\end{question}
}

\element{halliday-mc}{
\begin{question}{halliday-ch31-q57}
    An $RLC$ series circuit has $L=\SI{100}{\milli\henry}$ and $C=\SI{1}{\micro\farad}$.
    It is connected to a \SI{1000}{\hertz} source and the source emf is found to lead the current by \ang{75}.
    The value of $R$ is:
    \begin{multicols}{2}
    \begin{choices}
        \wrongchoice{\SI{12.6}{\ohm}}
      \correctchoice{\SI{126}{\ohm}}
        \wrongchoice{\SI{175}{\ohm}}
        \wrongchoice{\SI{1750}{\ohm}}
        \wrongchoice{\SI{1810}{\ohm}}
    \end{choices}
    \end{multicols}
\end{question}
}

\element{halliday-mc}{
\begin{question}{halliday-ch31-q58}
    An $RLC$ series circuit is driven by a sinusoidal emf with angular frequency $\omega_d$.
    If $\omega_d$ is increased without changing the amplitude of the emf the current amplitude increases. 
    If $L$ is the inductance, $C$ is the capacitance, and $R$ is the resistance,
        this means that:
    \begin{multicols}{2}
    \begin{choices}
        \wrongchoice{$\omega_d L > \dfrac{1}{\omega_d C}$}
      \correctchoice{$\omega_d L < \dfrac{1}{\omega_d C}$}
        \wrongchoice{$\omega_d L = \dfrac{1}{\omega_d C}$}
        \wrongchoice{$\omega_d L > R$}
        \wrongchoice{$\omega_d L < R$}
    \end{choices}
    \end{multicols}
\end{question}
}

\element{halliday-mc}{
\begin{question}{halliday-ch31-q59}
    In a sinusoidally driven series $RLC$ circuit,
        the inductive reactance is $X_L=\SI{200}{\ohm}$,
        the capacitive reactance is $X_C=\SI{100}{\ohm}$,
        and the resistance is $R=\SI{50}{\ohm}$.
    The current and applied emf would be in phase if:
    \begin{choices}
        \wrongchoice{the resistance is increased to \SI{100}{\ohm}, with no other changes}
        \wrongchoice{the resistance is increased to \SI{200}{\ohm}, with no other changes}
        \wrongchoice{the inductance is reduced to zero, with no other changes}
        \wrongchoice{the capacitance is doubled, with no other changes}
      \correctchoice{the capacitance is halved, with no other changes}
    \end{choices}
\end{question}
}

\element{halliday-mc}{
\begin{question}{halliday-ch31-q60}
    In a sinusoidally driven series $RLC$ circuit the current lags the applied emf. 
    The rate at which energy is dissipated in the resistor can be increased by:
    \begin{choices}
      \correctchoice{decreasing the capacitance and making no other changes}
        \wrongchoice{increasing the capacitance and making no other changes}
        \wrongchoice{increasing the inductance and making no other changes}
        \wrongchoice{increasing the driving frequency and making no other changes}
        \wrongchoice{decreasing the amplitude of the driving emf and making no other changes}
    \end{choices}
\end{question}
}

\element{halliday-mc}{
\begin{question}{halliday-ch31-q61}
    An $RLC$ series circuit, connected to a source $E$,
        is at resonance. 
    Then:
    \begin{choices}
        \wrongchoice{the voltage across $R$ is zero}
      \correctchoice{the voltage across $R$ equals the applied voltage}
        \wrongchoice{the voltage across $C$ is zero}
        \wrongchoice{the voltage across $L$ equals the applied voltage}
        \wrongchoice{the applied voltage and current differ in phase by \ang{90}}
    \end{choices}
\end{question}
}

\element{halliday-mc}{
\begin{question}{halliday-ch31-q62}
    An $RLC$ series circuit is connected to an oscillator with a maximum emf of \SI{100}{\volt}.
    If the voltage amplitudes $V_R$, $V_L$, and $V_C$ are all equal to each other,
        then $V_R$ must be:
    \begin{multicols}{3}
    \begin{choices}
        \wrongchoice{\SI{33}{\volt}}
        \wrongchoice{\SI{50}{\volt}}
        \wrongchoice{\SI{67}{\volt}}
        \wrongchoice{\SI{87}{\volt}}
      \correctchoice{\SI{100}{\volt}}
    \end{choices}
    \end{multicols}
\end{question}
}

\element{halliday-mc}{
\begin{question}{halliday-ch31-q63}
    A resistor, an inductor, and a capacitor are connected in parallel to a sinusoidal source of emf.
    Which of the following is true?
    \begin{choices}
        \wrongchoice{The currents in all branches are in phase.}
      \correctchoice{The potential differences across all branches are in phase.}
        \wrongchoice{The current in the capacitor branch leads the current in the inductor branch by one-fourth of a cycle.}
        \wrongchoice{The potential difference across the capacitor branch leads the potential difference across the inductor branch by one-fourth of a cycle.}
        \wrongchoice{The current in the capacitor branch lags the current in the inductor branch by one-fourth of a cycle.}
    \end{choices}
\end{question}
}

\element{halliday-mc}{
\begin{question}{halliday-ch31-q64}
    The rms value of an ac current is:
    \begin{choices}
        \wrongchoice{its peak value}
        \wrongchoice{its average value}
      \correctchoice{that steady current that produces the same rate of heating in a resistor as the actual current}
        \wrongchoice{that steady current that will charge a battery at the same rate as the actual current}
        \wrongchoice{zero}
    \end{choices}
\end{question}
}

\element{halliday-mc}{
\begin{question}{halliday-ch31-q65}
    The rms value of a sinusoidal voltage is $V_0/2$,
        where $V_0$ is the amplitude. 
    \begin{center}
    \begin{tikzpicture}
        %% NOTE:
    \end{tikzpicture}
    \end{center}
    What is the rms value of its fully rectified wave? 
    Recall that $V_{\text{rect}}(t)=|V(t)|$.
    \begin{multicols}{3}
    \begin{choices}
        \wrongchoice{$\dfrac{V_0^2}{\sqrt{2}}$}
        \wrongchoice{$\dfrac{V_0^2}{2}$}
        \wrongchoice{$\sqrt{2}V_0$}
      \correctchoice{$\dfrac{V_0}{\sqrt{2}}$}
        \wrongchoice{$\dfrac{V_0}{2\sqrt{2}}$}
    \end{choices}
    \end{multicols}
\end{question}
}

\element{halliday-mc}{
\begin{question}{halliday-ch31-q66}
    A sinusoidal voltage $V(t)$ has an rms value of \SI{100}{\volt}.
    Its maximum value is:
    \begin{multicols}{3}
    \begin{choices}
        \wrongchoice{\SI{100}{\volt}}
        \wrongchoice{\SI{707}{\volt}}
        \wrongchoice{\SI{70.7}{\volt}}
      \correctchoice{\SI{141}{\volt}}
        \wrongchoice{\SI{200}{\volt}}
    \end{choices}
    \end{multicols}
\end{question}
}

\element{halliday-mc}{
\begin{question}{halliday-ch31-q67}
    An ac generator produces \SI{10}{\volt} (rms) at \SI{400}{\radian\per\second}.
    It is connected to a series $RL$ circuit ($R=\SI{17.3}{\ohm}$, $L=\SI{0.025}{\henry}$).
    The rms current is:
    \begin{choices}
        \wrongchoice{\SI{0.50}{\ampere} and leads the emf by \ang{30}}
        \wrongchoice{\SI{0.71}{\ampere} and lags the emf by \ang{30}}
        \wrongchoice{\SI{1.40}{\ampere} and lags the emf by \ang{60}}
      \correctchoice{\SI{0.50}{\ampere} and lags the emf by \ang{30}}
        \wrongchoice{\SI{0.58}{\ampere} and leads the emf by \ang{90}}
    \end{choices}
\end{question}
}

\element{halliday-mc}{
\begin{question}{halliday-ch31-q68}
    An ac generator producing \SI{10}{\volt} (rms) at \SI{200}{\radian\per\second} is connected in series with a \SI{50}{\ohm} resistor, a \SI{400}{\milli\henry} inductor, and a \SI{200}{\micro\farad} capacitor.
    The rms current is:
    \begin{multicols}{3}
    \begin{choices}
        \wrongchoice{\SI{0.125}{\ampere}}
        \wrongchoice{\SI{0.135}{\ampere}}
        \wrongchoice{\SI{0.18}{\ampere}}
        \wrongchoice{\SI{0.20}{\ampere}}
        \wrongchoice{\SI{0.40}{\ampere}}
    \end{choices}
    \end{multicols}
\end{question}
}

\element{halliday-mc}{
\begin{question}{halliday-ch31-q69}
    An ac generator producing \SI{10}{\volt} (rms) at \SI{200}{\radian\per\second} is connected in series with a \SI{50}{\ohm} resistor,
        a \SI{400}{\milli\henry} inductor, and a \SI{200}{\micro\farad} capacitor. 
    The rms voltage across the resistor is:
    \begin{multicols}{3}
    \begin{choices}
        \wrongchoice{\SI{2.5}{\volt}}
        \wrongchoice{\SI{3.4}{\volt}}
      \correctchoice{\SI{6.7}{\volt}}
        \wrongchoice{\SI{10.0}{\volt}}
        \wrongchoice{\SI{10.8}{\volt}}
    \end{choices}
    \end{multicols}
\end{question}
}

\element{halliday-mc}{
\begin{question}{halliday-ch31-q70}
    An ac generator producing \SI{10}{\volt} (rms) at \SI{200}{\radian\per\second} is connected in series with a \SI{50}{\ohm} resistor, a \SI{400}{\milli\henry} inductor, and a \SI{200}{\micro\farad} capacitor. 
    The rms voltage across the capacitor is:
    \begin{multicols}{3}
    \begin{choices}
        \wrongchoice{\SI{2.5}{\volt}}
      \correctchoice{\SI{3.4}{\volt}}
        \wrongchoice{\SI{6.7}{\volt}}
        \wrongchoice{\SI{10.0}{\volt}}
        \wrongchoice{\SI{10.8}{\volt}}
    \end{choices}
    \end{multicols}
\end{question}
}

\element{halliday-mc}{
\begin{question}{halliday-ch31-q71}
    An ac generator producing \SI{10}{\volt} (rms) at \SI{200}{\radian\per\second} is connected in series with a \SI{50}{\ohm} resistor, a \SI{400}{\milli\henry} inductor, and a \SI{200}{\micro\farad} capacitor.
    The rms voltage across the inductor is:
    \begin{multicols}{3}
    \begin{choices}
        \wrongchoice{\SI{2.5}{\volt}}
        \wrongchoice{\SI{3.4}{\volt}}
        \wrongchoice{\SI{6.7}{\volt}}
        \wrongchoice{\SI{10.0}{\volt}}
      \correctchoice{\SI{10.8}{\volt}}
    \end{choices}
    \end{multicols}
\end{question}
}

\element{halliday-mc}{
\begin{question}{halliday-ch31-q72}
    The ideal meters shown read rms current and voltage. 
    %% nOTE: tikz
    The average power delivered to the load is:
    \begin{choices}
        \wrongchoice{definitely equal to $VI$}
        \wrongchoice{definitely more than $VI$}
      \correctchoice{possibly equal to $VI$ even if the load contains an inductor and a capacitor}
        \wrongchoice{definitely less than $VI$}
        \wrongchoice{zero, as is the average of any sine wave}
    \end{choices}
\end{question}
}

\element{halliday-mc}{
\begin{question}{halliday-ch31-q73}
    The average power supplied to the circuit shown passes through a maximum when which one of the following is increased continuously from a very low to a very high value?
    \begin{center}
    \begin{tikzpicture}
        %% NOTE:
    \end{tikzpicture}
    \end{center}
    \begin{choices}
        \wrongchoice{Source emf $\varepsilon$}
        \wrongchoice{$R$}
        \wrongchoice{$C$}
        \wrongchoice{Source frequency $f$}
        \wrongchoice{None of the provided}
    \end{choices}
\end{question}
}

\element{halliday-mc}{
\begin{question}{halliday-ch31-q74}
    In a series $RLC$ circuit the rms value of the generator emf is $E$ and the rms value of the current is $i$. 
    The current lags the emf by $\phi$. 
    The average power supplied by the generator is given by:
    \begin{multicols}{3}
    \begin{choices}
        \wrongchoice{$\dfrac{iE}{2}\cos\phi$}
        \wrongchoice{$iE$}
        \wrongchoice{$\dfrac{i^2}{Z}$}
        \wrongchoice{$\dfrac{i^2}{Z}$}
      \correctchoice{$i^2 R$}
    \end{choices}
    \end{multicols}
\end{question}
}

\element{halliday-mc}{
\begin{question}{halliday-ch31-q75}
    The units of the power factor are:
    \begin{multicols}{2}
    \begin{choices}
        \wrongchoice{ohm}
        \wrongchoice{watt}
        \wrongchoice{radian}
        \wrongchoice{ohm\textsuperscript{1/2}}
      \correctchoice{none of the provided}
    \end{choices}
    \end{multicols}
\end{question}
}

\element{halliday-mc}{
\begin{question}{halliday-ch31-q76}
    A series circuit consists of a \SI{15}{\ohm} resistor, a \SI{25}{\milli\henry} inductor, and a \SI{35}{\micro\farad} capacitor. 
    If the frequency is \SI{100}{\hertz} the power factor is:
    \begin{multicols}{3}
    \begin{choices}
        \wrongchoice{zero}
        \wrongchoice{0.20}
      \correctchoice{0.45}
        \wrongchoice{0.89}
        \wrongchoice{1.0}
    \end{choices}
    \end{multicols}
\end{question}
}

\element{halliday-mc}{
\begin{question}{halliday-ch31-q77}
    The main reason that alternating current replaced direct current for general use is:
    \begin{choices}
        \wrongchoice{ac generators do not need slip rings}
      \correctchoice{ac voltages may be conveniently transformed}
        \wrongchoice{electric clocks do not work on dc}
        \wrongchoice{a given ac current does not heat a power line as much as the same dc current}
        \wrongchoice{ac minimizes magnetic effects}
    \end{choices}
\end{question}
}

\element{halliday-mc}{
\begin{question}{halliday-ch31-q78}
    A step-down transformer is used to:
    \begin{choices}
        \wrongchoice{increase the power}
        \wrongchoice{decrease the power}
        \wrongchoice{increase the voltage}
      \correctchoice{decrease the voltage}
        \wrongchoice{change ac to dc}
    \end{choices}
\end{question}
}

\element{halliday-mc}{
\begin{question}{halliday-ch31-q79}
    Iron, rather than copper, is used in the core of transformers because iron:
    \begin{choices}
        \wrongchoice{can withstand a higher temperature}
        \wrongchoice{has a greater resistivity}
      \correctchoice{has a very high permeability}
        \wrongchoice{makes a good permanent magnet}
        \wrongchoice{insulates the primary from the secondary}
    \end{choices}
\end{question}
}

\element{halliday-mc}{
\begin{question}{halliday-ch31-q80}
    The core of a transformer is made in a laminated form to:
    \begin{choices}
        \wrongchoice{facilitate easy assembly}
        \wrongchoice{reduce $i^2 R$ losses in the coils}
        \wrongchoice{increase the magnetic flux}
        \wrongchoice{save weight}
      \correctchoice{prevent eddy currents}
    \end{choices}
\end{question}
}

\element{halliday-mc}{
\begin{question}{halliday-ch31-q81}
    A generator supplies \SI{100}{\volt} to the primary coil of a transformer. 
    The primary has 50 turns and the secondary has 500 turns. 
    The secondary voltage is:
    \begin{multicols}{3}
    \begin{choices}
      \correctchoice{\SI{1000}{\volt}}
        \wrongchoice{\SI{500}{\volt}}
        \wrongchoice{\SI{250}{\volt}}
        \wrongchoice{\SI{100}{\volt}}
        \wrongchoice{\SI{10}{\volt}}
    \end{choices}
    \end{multicols}
\end{question}
}

\element{halliday-mc}{
\begin{question}{halliday-ch31-q82}
    The resistance of the primary coil of a well-designed, $1:10$ step-down transformer is \SI{1}{\ohm}. 
    With the secondary circuit open,
        the primary is connected to a \SI{12}{\volt} ac generator. 
    The primary current is:
    \begin{choices}
      \correctchoice{essentially zero}
        \wrongchoice{about \SI{12}{\ampere}}
        \wrongchoice{about \SI{120}{\ampere}}
        \wrongchoice{depends on the actual number of turns in the primary coil}
        \wrongchoice{depends on the core material}
    \end{choices}
\end{question}
}

\element{halliday-mc}{
\begin{question}{halliday-ch31-q83}
    The primary of an ideal transformer has 100 turns and the secondary has 600 turns. 
    Then:
    \begin{choices}
        \wrongchoice{the power in the primary circuit is less than that in the secondary circuit}
        \wrongchoice{the currents in the two circuits are the same}
        \wrongchoice{the voltages in the two circuits are the same}
      \correctchoice{the primary current is six times the secondary current}
        \wrongchoice{the frequency in the secondary circuit is six times that in the primary circuit}
    \end{choices}
\end{question}
}

\element{halliday-mc}{
\begin{question}{halliday-ch31-q84}
    The primary of a $3:1$ step-up transformer is connected to a source and the secondary is connected to a resistor $R$. 
    The power dissipated by $R$ in this situation is $P$.
    If $R$ is connected directly to the source it will dissipate a power of:
    \begin{multicols}{3}
    \begin{choices}
      \correctchoice{$\dfrac{P}{9}$}
        \wrongchoice{$\dfrac{P}{3}$}
        \wrongchoice{$P$}
        \wrongchoice{$3P$}
        \wrongchoice{$9P$}
    \end{choices}
    \end{multicols}
\end{question}
}

\element{halliday-mc}{
\begin{question}{halliday-ch31-q85}
    In an ideal $1:8$ step-down transformer,
        the primary power is \SI{10}{\kilo\watt} and the secondary current is \SI{25}{\ampere}. 
    The primary voltage is:
    \begin{multicols}{3}
    \begin{choices}
        \wrongchoice{\SI{25 600}{\volt}}
      \correctchoice{\SI{3200}{\volt}}
        \wrongchoice{\SI{400}{\volt}}
        \wrongchoice{\SI{50}{\volt}}
        \wrongchoice{\SI{6.25}{\volt}}
    \end{choices}
    \end{multicols}
\end{question}
}

\element{halliday-mc}{
\begin{question}{halliday-ch31-q86}
    A source with an impedance of \SI{100}{\ohm} is connected to the primary coil of a transformer and a resistance $R$ is connected to the secondary coil. 
    If the transformer has 500 turns in its primary coil and 100 turns in its secondary coil the greatest power will be dissipated in the resistor if $R=$:
    \begin{multicols}{3}
    \begin{choices}
        \wrongchoice{zero}
        \wrongchoice{\SI{0.25}{\hertz}}
      \correctchoice{\SI{4.0}{\hertz}}
        \wrongchoice{\SI{50}{\hertz}}
        \wrongchoice{\SI{100}{\hertz}}
    \end{choices}
    \end{multicols}
\end{question}
}


\endinput


