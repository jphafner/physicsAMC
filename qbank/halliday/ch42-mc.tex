
%%--------------------------------------------------
%% Halliday: Fundamentals of Physics
%%--------------------------------------------------


%% Chapter 42: Nuclear Physics
%%--------------------------------------------------


%% Learning Objectives
%%--------------------------------------------------

%% 42.01: Explain the general arrangement for Rutherford scattering and what was learned from it.
%% 42.02: In a Rutherford scattering arrangement, apply the relationship between the projectile's initial kinetic energy and the distance of its closest approach to the target nucleus.


%% Halliday Multiple Choice Questions
%%--------------------------------------------------
\element{halliday-mc}{
\begin{question}{halliday-ch42-q01}
    The smallest particle of any chemical element that can exist by itself and yet retain the qualities that distinguish it as that element is:
    \begin{multicols}{2}
    \begin{choices}
        \wrongchoice{an electron}
        \wrongchoice{a proton}
        \wrongchoice{a neutron}
      \correctchoice{an atom}
        \wrongchoice{a molecule}
    \end{choices}
    \end{multicols}
\end{question}
}

\element{halliday-mc}{
\begin{question}{halliday-ch42-q02}
    Of the following,
        which has the smallest rest energy?
    \begin{multicols}{2}
    \begin{choices}
        \wrongchoice{A neutron}
      \correctchoice{An electron}
        \wrongchoice{An ion}
        \wrongchoice{A proton}
        \wrongchoice{An atom}
    \end{choices}
    \end{multicols}
\end{question}
}

\element{halliday-mc}{
\begin{question}{halliday-ch42-q03}
    The mass of an electron:
    \begin{choices}
        \wrongchoice{is almost the same as that of a neutron}
        \wrongchoice{is negative}
        \wrongchoice{equals that of a proton}
        \wrongchoice{is zero if the electron is at rest}
      \correctchoice{is much less than that of a proton}
    \end{choices}
\end{question}
}

\element{halliday-mc}{
\begin{question}{halliday-ch42-q04}
    The mass of a neutron:
    \begin{choices}
        \wrongchoice{equals that of an electron}
        \wrongchoice{equals that of a proton}
      \correctchoice{is a little more than that of a proton}
        \wrongchoice{is exactly that of a proton plus an electron}
        \wrongchoice{is as yet unmeasured}
    \end{choices}
\end{question}
}

\element{halliday-mc}{
\begin{question}{halliday-ch42-q05}
    The mass of a hydrogen atom is approximately:
    \begin{multicols}{3}
    \begin{choices}
      \correctchoice{\SI{e-27}{\kilo\gram}}
        \wrongchoice{\SI{e-31}{\kilo\gram}}
        \wrongchoice{\SI{e-24}{\kilo\gram}}
        \wrongchoice{\SI{e-13}{\kilo\gram}}
        \wrongchoice{\SI{e-8}{\kilo\gram}}
    \end{choices}
    \end{multicols}
\end{question}
}

\element{halliday-mc}{
\begin{question}{halliday-ch42-q06}
    One atomic mass unit is about:
    \begin{multicols}{2}
    \begin{choices}
        \wrongchoice{\SI{1.66e-31}{\kilo\gram}}
        \wrongchoice{\SI{9.11e-31}{\kilo\gram}}
      \correctchoice{\SI{1.66e-27}{\kilo\gram}}
        \wrongchoice{\SI{9.11e-27}{\kilo\gram}}
        \wrongchoice{\SI{1.66e-25}{\kilo\gram}}
    \end{choices}
    \end{multicols}
\end{question}
}

\element{halliday-mc}{
\begin{question}{halliday-ch42-q07}
    The atomic number of an element is:
    \begin{choices}
        \wrongchoice{the whole number nearest to its mass}
      \correctchoice{the number of protons in its nucleus}
        \wrongchoice{the nearest whole number of hydrogen atoms having the same mass as a single atom of the given element}
        \wrongchoice{the number of neutrons in its nucleus}
        \wrongchoice{its order of discovery}
    \end{choices}
\end{question}
}

\element{halliday-mc}{
\begin{question}{halliday-ch42-q08}
    Iron has atomic number 26. 
    Naturally mined iron contains isotopes of mass numbers 54, 56, 57, and 58. 
    Which of the following statements is \emph{false}?
    \begin{choices}
        \wrongchoice{Every atom of iron has 26 protons}
        \wrongchoice{Some iron atoms have 30 neutrons}
      \correctchoice{Some iron atoms have 54 neutrons}
        \wrongchoice{The isotopes may be separated in a mass spectrometer}
        \wrongchoice{There are four kinds of naturally occurring iron atoms with the same chemical properties}
    \end{choices}
\end{question}
}

\element{halliday-mc}{
\begin{question}{halliday-ch42-q09}
    Let $Z$ denote the atomic number and $A$ denote the mass number of a nucleus. 
    The number of neutrons in this nucleus is:
    \begin{multicols}{3}
    \begin{choices}
        \wrongchoice{$Z$}
      \correctchoice{$A-Z$}
        \wrongchoice{$A-2Z$}
        \wrongchoice{$A$}
        \wrongchoice{$2A-Z$}
    \end{choices}
    \end{multicols}
\end{question}
}

\element{halliday-mc}{
\begin{question}{halliday-ch42-q10}
    The isotopes of an element:
    \begin{choices}
        \wrongchoice{cannot be separated at all}
        \wrongchoice{occur well separated in nature}
      \correctchoice{have similar chemical behavior}
        \wrongchoice{cannot be separated by physical methods}
        \wrongchoice{have equal masses}
    \end{choices}
\end{question}
}

\element{halliday-mc}{
\begin{question}{halliday-ch42-q11}
    Bromine, with atomic mass \SI{79.942}{u},
        is composed of nearly equal amounts of two isotopes,
        one of which contains 79 nucleons per atom. 
    The mass number of the other isotope is:
    \begin{multicols}{3}
    \begin{choices}
        \wrongchoice{\num{78}}
        \wrongchoice{\num{79}}
        \wrongchoice{\num{80}}
      \correctchoice{\num{81}}
        \wrongchoice{\num{82}}
    \end{choices}
    \end{multicols}
\end{question}
}

\element{halliday-mc}{
\begin{question}{halliday-ch42-q12}
    The mass density of an atomic nucleus is:
    \begin{choices}
        \wrongchoice{about \SI{e15}{\kilo\gram\per\meter\cubed}}
        \wrongchoice{about \SI{e12}{\kilo\gram\per\meter\cubed}}
        \wrongchoice{increases with increasing nuclear mass}
        \wrongchoice{increases with decreasing nuclear radius}
      \correctchoice{about the same as that of all other nuclei}
    \end{choices}
\end{question}
}

\element{halliday-mc}{
\begin{question}{halliday-ch42-q13}
    Volumes of atomic nuclei are proportional to:
    \begin{choices}
      \correctchoice{the mass number}
        \wrongchoice{the atomic number}
        \wrongchoice{the total nuclear spin}
        \wrongchoice{the number of neutrons}
        \wrongchoice{none of the provided}
    \end{choices}
\end{question}
}

\element{halliday-mc}{
\begin{question}{halliday-ch42-q14}
    A femtometer is:
    \begin{multicols}{2}
    \begin{choices}
        \wrongchoice{larger than \SI{e-9}{\meter}}
        \wrongchoice{\SI{e-9}{\meter}}
        \wrongchoice{\SI{e-12}{\meter}}
      \correctchoice{\SI{e-15}{\meter}}
        \wrongchoice{\SI{e-18}{\meter}}
    \end{choices}
    \end{multicols}
\end{question}
}

\element{halliday-mc}{
\begin{question}{halliday-ch42-q15}
    A nucleus with a mass number of 64 has a mean radius of about:
    \begin{multicols}{2}
    \begin{choices}
      \correctchoice{\SI{4.8}{\femto\meter}}
        \wrongchoice{\SI{19}{\femto\meter}}
        \wrongchoice{\SI{77}{\femto\meter}}
        \wrongchoice{\SI{260}{\femto\meter}}
        \wrongchoice{\SI{2.6e5}{\femto\meter}}
    \end{choices}
    \end{multicols}
\end{question}
}

\element{halliday-mc}{
\begin{question}{halliday-ch42-q16}
    A proton in a large nucleus:
    \begin{choices}
        \wrongchoice{attracts all other protons}
        \wrongchoice{repels all other protons}
        \wrongchoice{repels all neutrons}
      \correctchoice{attracts some protons and repels others}
        \wrongchoice{attracts some neutrons and repels others}
    \end{choices}
\end{question}
}

\element{halliday-mc}{
\begin{question}{halliday-ch42-q17}
    Two protons are separated by \SI{e-16}{\meter}.
        The nuclear ($N$), electrostatic ($E$), and gravitational ($G$) forces between these protons,
        in order of increasing strength, are:
    \begin{multicols}{2}
    \begin{choices}
        \wrongchoice{$E$, $N$, $G$}
        \wrongchoice{$N$, $G$, $E$}
      \correctchoice{$G$, $E$, $N$}
        \wrongchoice{$G$, $N$, $E$}
        \wrongchoice{$E$, $G$, $N$}
    \end{choices}
    \end{multicols}
\end{question}
}

\element{halliday-mc}{
\begin{question}{halliday-ch42-q18}
    Two protons are about \SI{e-10}{\meter} apart. 
    Their relative motion is chiefly determined by:
    \begin{choices}
        \wrongchoice{gravitational forces}
      \correctchoice{electrical forces}
        \wrongchoice{nuclear forces}
        \wrongchoice{magnetic forces}
        \wrongchoice{torque due to electric dipole moments}
    \end{choices}
\end{question}
}

\element{halliday-mc}{
\begin{question}{halliday-ch42-q19}
    The binding energy of a nucleus is the energy that must be supplied to:
    \begin{choices}
        \wrongchoice{remove a nucleon}
        \wrongchoice{remove an alpha particle}
        \wrongchoice{remove a beta particle}
      \correctchoice{separate the nucleus into its constituent nucleons}
        \wrongchoice{separate the nucleus into a collection of alpha particles}
    \end{choices}
\end{question}
}

\element{halliday-mc}{
\begin{question}{halliday-ch42-q20}
    If a nucleus has mass $M$, $Z$ protons (mass $m_p$),
        and $N$ neutrons (mass $m_n$),
        its binding energy is equal to:
    \begin{multicols}{2}
    \begin{choices}
        \wrongchoice{$M c^2$}
        \wrongchoice{$\left(M - Zm_p - Nm_n \right)c^2$}
      \correctchoice{$\left(Z m_p + N m_n - M\right)c^2$}
        \wrongchoice{$\left(Z m_p + N m_n\right)c^2$}
        \wrongchoice{$\left(Z m_p - M\right)c^2$}
    \end{choices}
    \end{multicols}
\end{question}
}

\element{halliday-mc}{
\begin{question}{halliday-ch42-q21}
    Stable nuclei generally:
    \begin{choices}
        \wrongchoice{have a greater number of protons than neutrons}
        \wrongchoice{have low mass numbers}
        \wrongchoice{have high mass numbers}
        \wrongchoice{are beta emitters}
      \correctchoice{none of the provided}
    \end{choices}
\end{question}
}

\element{halliday-mc}{
\begin{question}{halliday-ch42-q22}
    Let $A$ be the mass number and $Z$ be the atomic number of a nucleus. 
    Which of the following is approximately correct for light nuclei?
    \begin{multicols}{3}
    \begin{choices}
        \wrongchoice{$Z = 2A$}
        \wrongchoice{$Z = A$}
      \correctchoice{$Z = \dfrac{A}{2}$}
        \wrongchoice{$Z = \sqrt{A}$}
        \wrongchoice{$Z = A^2$}
    \end{choices}
    \end{multicols}
\end{question}
}

\element{halliday-mc}{
\begin{question}{halliday-ch42-q23}
    The greatest binding energy per nucleon occurs for nuclides with masses near that of:
    \begin{multicols}{2}
    \begin{choices}
        \wrongchoice{helium}
        \wrongchoice{sodium}
      \correctchoice{iron}
        \wrongchoice{mercury}
        \wrongchoice{uranium}
    \end{choices}
    \end{multicols}
\end{question}
}

\element{halliday-mc}{
\begin{question}{halliday-ch42-q24}
    Which of the following nuclides is least likely to be detected?
    \begin{multicols}{2}
    \begin{choices}
        \wrongchoice{\ce{^{52}Fe} ($Z=26$)}
      \correctchoice{\ce{^{115}Nd} ($Z=60$)}
        \wrongchoice{\ce{^{175}Lu} ($Z=71$)}
        \wrongchoice{\ce{^{208}Pb} ($Z=82$)}
        \wrongchoice{\ce{^{238}U} ($Z=92$)}
    \end{choices}
    \end{multicols}
\end{question}
}

\element{halliday-mc}{
\begin{question}{halliday-ch42-q25}
    The half-life of a radioactive substance is:
    \begin{choices}
        \wrongchoice{half the time it takes for the entire substance to decay}
        \wrongchoice{usually about 50 years}
        \wrongchoice{the time for radium to change into lead}
        \wrongchoice{calculated from $E=mc^2$}
      \correctchoice{the time for half the substance to decay}
    \end{choices}
\end{question}
}

\element{halliday-mc}{
\begin{question}{halliday-ch42-q26}
    Which expression correctly describes the radioactive decay of a substance whose half-life is $T$?
    \begin{choices}
      \correctchoice{$N(t) = N_0\mathrm{e}^{-\left(\dfrac{t\ln 2}{T}\right)}$}
        \wrongchoice{$N(t) = N_0\mathrm{e}^{-\left(\dfrac{t}{T}\right)}$}
        \wrongchoice{$N(t) = N_0\mathrm{e}^{-tT}$}
        \wrongchoice{$N(t) = N_0\mathrm{e}^{-tT\ln 2}$}
        \wrongchoice{$N(t) = N_0\mathrm{e}^{-\left(\dfrac{t}{T}\right)\ln 2}$}
    \end{choices}
\end{question}
}

\element{halliday-mc}{
\begin{question}{halliday-ch42-q27}
    Radioactive element $A$ decays to the stable element $B$ with a half-life $T$.
    Starting with a sample of pure $A$ and no $B$,
        which graph below correctly shows the number of $A$ atoms,
        $N_A$, as a function of time $t$?
    \begin{multicols}{2}
    \begin{choices}
        %% NOTE: ANS is D
        \wrongchoice{
            \begin{tikzpicture}
                %% NOTE: pgplots options
            \end{tikzpicture}
        }
    \end{choices}
    \end{multicols}
\end{question}
}

\element{halliday-mc}{
\begin{question}{halliday-ch42-q28}
    A large collection of nuclei are undergoing alpha decay. 
    The rate of decay at any instant is proportional to:
    \begin{choices}
      \correctchoice{the number of undecayed nuclei present at that instant}
        \wrongchoice{the time since the decays started}
        \wrongchoice{the time remaining before all have decayed}
        \wrongchoice{the half-life of the decay}
        \wrongchoice{the average time between decays}
    \end{choices}
\end{question}
}

\element{halliday-mc}{
\begin{question}{halliday-ch42-q29}
    The relation between the disintegration constant $\lambda$ and the half-life $T$ of a radioactive substance is:
    \begin{multicols}{2}
    \begin{choices}
        \wrongchoice{$\lambda = 2T$}
        \wrongchoice{$\lambda = \dfrac{1}{T}$}
        \wrongchoice{$\lambda = \dfrac{2}{T}$}
      \correctchoice{$\lambda T = \ln 2$}
        \wrongchoice{$\lambda T = \ln\left(\dfrac{1}{2}\right)$}
    \end{choices}
    \end{multicols}
\end{question}
}

\element{halliday-mc}{
\begin{question}{halliday-ch42-q30}
    Possible units for the disintegration constant $\lambda$ are:
    \begin{choices}
        \wrongchoice{kilogram per second (\si{\kilo\gram\per\second})}
        \wrongchoice{second per kilogram (\si{\second\per\kilo\gram})}
        \wrongchoice{hour (\si{\hour})}
      \correctchoice{per day (\si{\per\day})}
        \wrongchoice{per centimeter (\si{\per\centi\meter})}
    \end{choices}
\end{question}
}

\element{halliday-mc}{
\begin{question}{halliday-ch42-q31}
    The half-life of a given nuclear disintegration $A\to B$:
    \begin{choices}
        \wrongchoice{depends on the initial number of $A$ atoms}
        \wrongchoice{depends on the initial number of $B$ atoms}
        \wrongchoice{is an exponentially increasing function of time}
        \wrongchoice{is an exponentially decreasing function of time}
      \correctchoice{none of the provided}
    \end{choices}
\end{question}
}

\element{halliday-mc}{
\begin{question}{halliday-ch42-q32}
    The graph shows the activity $R$ as a function of time $t$ for three radioactive samples. 
    \begin{center}
    \begin{tikzpicture}
        %% NOTE:
    \end{tikzpicture}
    \end{center}
    Rank the samples according to their half-lives,
        shortest to longest.
    \begin{multicols}{2}
    \begin{choices}
        \wrongchoice{1, 2, 3}
        \wrongchoice{1, 3, 2}
        \wrongchoice{2, 1, 3}
        \wrongchoice{2, 3, 1}
        \wrongchoice{3, 1, 2}
    \end{choices}
    \end{multicols}
\end{question}
}

\element{halliday-mc}{
\begin{question}{halliday-ch42-q33}
    The half-life of radium is about 1600 years. 
    If a rock initially contains \SI{1}{\gram} of radium,
        the amount left after 6400 years will be about:
    \begin{multicols}{2}
    \begin{choices}
        \wrongchoice{\SI{938}{\milli\gram}}
        \wrongchoice{\SI{62}{\milli\gram}}
      \correctchoice{\SI{31}{\milli\gram}}
        \wrongchoice{\SI{16}{\milli\gram}}
        \wrongchoice{less than \SI{16}{\milli\gram}}
    \end{choices}
    \end{multicols}
\end{question}
}

\element{halliday-mc}{
\begin{question}{halliday-ch42-q34}
    Starting with a sample of pure \ce{^{66}Cu},
        \num{7/8} of it decays into \ce{Zn} in 15 minutes. 
    The corresponding half-life is:
    \begin{multicols}{2}
    \begin{choices}
        \wrongchoice{15 minutes}
      \correctchoice{5 minutes}
        \wrongchoice{7 minutes}
        \wrongchoice{3.75 minutes}
        \wrongchoice{10 minutes}
    \end{choices}
    \end{multicols}
\end{question}
}

\element{halliday-mc}{
\begin{question}{halliday-ch42-q35}
    \ce{^{210}Bi} (an isotope of bismuth) has a half-life of 5.0 days. 
    The time for three-quarters of a sample of \ce{Bi} to decay is:
    \begin{multicols}{2}
    \begin{choices}
        \wrongchoice{2.5 days}
      \correctchoice{10 days}
        \wrongchoice{15 days}
        \wrongchoice{20 days}
        \wrongchoice{3.75 days}
    \end{choices}
    \end{multicols}
\end{question}
}

\element{halliday-mc}{
\begin{question}{halliday-ch42-q36}
    Radioactive \ce{^{210}Sr} has a half-life of 30 years. 
    What percent of a sample of \ce{^{90}Sr} will remain after 60 years?
    \begin{multicols}{3}
    \begin{choices}
        \wrongchoice{\SI{0}{\percent}}
      \correctchoice{\SI{25}{\percent}}
        \wrongchoice{\SI{50}{\percent}}
        \wrongchoice{\SI{75}{\percent}}
        \wrongchoice{\SI{14}{\percent}}
    \end{choices}
    \end{multicols}
\end{question}
}

\element{halliday-mc}{
\begin{question}{halliday-ch42-q37}
    The half-life of a radioactive isotope is \SI{6.5}{\hour}. 
    If there are initially \num{48e32} atoms of this isotope,
        the number of atoms of this isotope remaining after \SI{26}{\hour} is:
    \begin{multicols}{3}
    \begin{choices}
        \wrongchoice{\num{12e32}}
        \wrongchoice{\num{6e32}}
      \correctchoice{\num{3e32}}
        \wrongchoice{\num{6e4}}
        \wrongchoice{\num{3e2}}
    \end{choices}
    \end{multicols}
\end{question}
}

\element{halliday-mc}{
\begin{question}{halliday-ch42-q38}
    At the end of \SI{14}{\minute}, \num{1/16} of a sample of radioactive polonium remains. 
    The corresponding half-life is:
    \begin{multicols}{3}
    \begin{choices}
        \wrongchoice{\SI{7/8}{\minute}}
        \wrongchoice{\SI{8/7}{\minute}}
        \wrongchoice{\SI{7/4}{\minute}}
      \correctchoice{\SI{7/2}{\minute}}
        \wrongchoice{\SI{14/3}{\minute}}
    \end{choices}
    \end{multicols}
\end{question}
}

\element{halliday-mc}{
\begin{question}{halliday-ch42-q39}
    The half-life of a radioactive isotope is 140 days. 
    In how many days does the decay rate of a sample of this isotope decrease to one-fourth of its initial decay rate?
    \begin{multicols}{3}
    \begin{choices}
        \wrongchoice{35}
        \wrongchoice{105}
        \wrongchoice{187}
        \wrongchoice{210}
      \correctchoice{280}
    \end{choices}
    \end{multicols}
\end{question}
}

\element{halliday-mc}{
\begin{question}{halliday-ch42-q40}
    Of the three common types of radiation (alpha, beta, gamma) from radioactive sources,
        electric charge is carried by:
    \begin{choices}
        \wrongchoice{only beta and gamma}
        \wrongchoice{only beta}
        \wrongchoice{only alpha and gamma}
        \wrongchoice{only alpha}
      \correctchoice{only alpha and beta}
    \end{choices}
\end{question}
}

\element{halliday-mc}{
\begin{question}{halliday-ch42-q41}
    An alpha particle is:
    \begin{choices}
      \correctchoice{a helium atom with two electrons removed}
        \wrongchoice{an aggregate of two or more electrons}
        \wrongchoice{a hydrogen atom}
        \wrongchoice{the ultimate unit of positive charge}
        \wrongchoice{sometimes negatively charged}
    \end{choices}
\end{question}
}

\element{halliday-mc}{
\begin{question}{halliday-ch42-q42}
    A nucleus with mass number $A$ and atomic number $Z$ emits an alpha particle. 
    The mass number and atomic number, respectively,
        of the daughter nucleus are:
    \begin{multicols}{2}
    \begin{choices}
        \wrongchoice{$A$, $Z-2$}
        \wrongchoice{$A-2$, $Z−2$}
        \wrongchoice{$A-2$, $Z$}
        \wrongchoice{$A-4$, $Z$}
      \correctchoice{$A-4$, $Z-2$}
    \end{choices}
    \end{multicols}
\end{question}
}

\element{halliday-mc}{
\begin{question}{halliday-ch42-q43}
    Radioactive polonium, \ce{^{214}Po} ($Z=84$),
        decays by alpha emission to:
    \begin{multicols}{2}
    \begin{choices}
        \wrongchoice{\ce{^{214}Po} $(Z = 84)$}
      \correctchoice{\ce{^{210}Pb} $(Z = 82)$}
        \wrongchoice{\ce{^{214}At} $(Z = 85)$}
        \wrongchoice{\ce{^{218}Po} $(Z = 84)$}
        \wrongchoice{\ce{^{210}Bi} $(Z = 83)$}
    \end{choices}
    \end{multicols}
\end{question}
}

\element{halliday-mc}{
\begin{question}{halliday-ch42-q44}
    A radium atom, \ce{^{226}Ra} ($Z=86$) emits an alpha particle. 
    The number of protons in the resulting atom is:
    \begin{multicols}{2}
    \begin{choices}
      \correctchoice{84}
        \wrongchoice{85}
        \wrongchoice{86}
        \wrongchoice{88}
        \wrongchoice{some other number}
    \end{choices}
    \end{multicols}
\end{question}
}

\element{halliday-mc}{
\begin{question}{halliday-ch42-q45}
    Some alpha emitters have longer half-lives than others because:
    \begin{choices}
        \wrongchoice{their alpha particles have greater mass}
        \wrongchoice{their alpha particles have less mass}
      \correctchoice{their barriers to decay are higher and wider}
        \wrongchoice{their barriers to decay are lower and narrower}
        \wrongchoice{their decays include the emission of a photon}
    \end{choices}
\end{question}
}

\element{halliday-mc}{
\begin{question}{halliday-ch42-q46}
    In an alpha decay the disintegration energy appears chiefly as:
    \begin{choices}
        \wrongchoice{photon energies}
      \correctchoice{the kinetic energies of the alpha and the daughter nucleus}
        \wrongchoice{the excitation energy of the daughter nucleus}
        \wrongchoice{the excitation energy of the alpha particle}
        \wrongchoice{heat}
    \end{choices}
\end{question}
}

\element{halliday-mc}{
\begin{question}{halliday-ch42-q47}
    Rank the following collections of particles according to the total binding energy of all the particles in each collection, least to greatest.
    \begin{description}
        \item[collection 1:] \ce{^{244}Pu} ($Z=94$) nucleus alone
        \item[collection 2:] \ce{^{240}U}  ($Z=92$) nucleus, $\alpha$ particle
        \item[collection 3:] \ce{^{240}U}  ($Z=92$) nucleus, two separated protons, two separated neutrons
    \end{description}
    \begin{multicols}{2}
    \begin{choices}
        \wrongchoice{1, 2, 3}
        \wrongchoice{3, 2, 1}
        \wrongchoice{2, 1, 3}
      \correctchoice{1, 3, 2}
        \wrongchoice{2, 3, 1}
    \end{choices}
    \end{multicols}
\end{question}
}

\element{halliday-mc}{
\begin{question}{halliday-ch42-q48}
    A beta particle is:
    \begin{choices}
        \wrongchoice{a helium nucleus}
      \correctchoice{an electron or a positron}
        \wrongchoice{a radioactive element}
        \wrongchoice{any negative particle}
        \wrongchoice{a hydrogen atom}
    \end{choices}
\end{question}
}

\element{halliday-mc}{
\begin{question}{halliday-ch42-q49}
    Beta particles from various radioactive sources all have:
    \begin{choices}
      \correctchoice{the same mass}
        \wrongchoice{the same speed}
        \wrongchoice{the same charge}
        \wrongchoice{the same deflection}
        \wrongchoice{the same energy in a magnetic field}
    \end{choices}
\end{question}
}

\element{halliday-mc}{
\begin{question}{halliday-ch42-q50}
    A radioactive atom $X$ emits a $\beta^-$ particle. 
    The resulting atom:
    \begin{choices}
        \wrongchoice{must be very reactive chemically}
      \correctchoice{has an atomic number that is one more than that of $X$}
        \wrongchoice{has a mass number that is one less than that of $X$}
        \wrongchoice{must be radioactive}
        \wrongchoice{is the same chemical element as $X$}
    \end{choices}
\end{question}
}

\element{halliday-mc}{
\begin{question}{halliday-ch42-q51}
    A nucleus with mass number $A$ and atomic number $Z$ undergoes $\beta^-$ decay. 
    The mass number and atomic number,
        respectively, of the daughter nucleus are:
    \begin{multicols}{2}
    \begin{choices}
        \wrongchoice{$A$, $Z-1$}
        \wrongchoice{$A-1$, $Z$}
        \wrongchoice{$A+1$, $Z-1$}
      \correctchoice{$A$, $Z+1$}
        \wrongchoice{$A$, $Z-1$}
    \end{choices}
    \end{multicols}
\end{question}
}

\element{halliday-mc}{
\begin{question}{halliday-ch42-q52}
    A nucleus with mass number $A$ and atomic number $Z$ undergoes $\beta^+$ decay. 
    The mass number and atomic number,
        respectively, of the daughter nucleus are:
    \begin{multicols}{2}
    \begin{choices}
        \wrongchoice{$A-1$, $Z-1$}
        \wrongchoice{$A-1$, $Z+1$}
        \wrongchoice{$A+1$, $Z-1$}
        \wrongchoice{$A$, $Z+1$}
      \correctchoice{$A$, $Z-1$}
    \end{choices}
    \end{multicols}
\end{question}
}

\element{halliday-mc}{
\begin{question}{halliday-ch42-q53}
    In addition to the daughter nucleus and an electron or positron,
        the products of a beta decay include:
    \begin{choices}
        \wrongchoice{a neutron}
      \correctchoice{a neutrino}
        \wrongchoice{a proton}
        \wrongchoice{an alpha particle}
        \wrongchoice{no other particle}
    \end{choices}
\end{question}
}

\element{halliday-mc}{
\begin{question}{halliday-ch42-q54}
    The energies of electrons emitted in $\beta^-$ decays have a continuous spectrum because:
    \begin{choices}
        \wrongchoice{the original neutron has a continuous spectrum}
      \correctchoice{a neutrino can carry off energy}
        \wrongchoice{the emitted electron is free}
        \wrongchoice{energy is not conserved}
        \wrongchoice{the daughter nucleus may have any energy}
    \end{choices}
\end{question}
}

\element{halliday-mc}{
\begin{question}{halliday-ch42-q55}
    If \ce{^{204}Tl} ($Z=81$) emits a $\beta^-$ particle from its nucleus:
    \begin{choices}
        \wrongchoice{stable \ce{Tl} is formed}
        \wrongchoice{\ce{^{202}Hg}  ($Z=80$) is formed}
      \correctchoice{\ce{^{204}Pb} ($Z=82$) is formed}
        \wrongchoice{radioactive \ce{Tl} is formed}
        \wrongchoice{\ce{^{197}Au} ($Z=79$) is formed}
    \end{choices}
\end{question}
}

\element{halliday-mc}{
\begin{question}{halliday-ch42-q56}
    An atom of \ce{^{235}U} ($Z=92$) disintegrates to \ce{^{207}Pb} ($Z=82$) with a half-life of about a billion years by emitting seven alpha particles and \rule[-0.1pt]{4em}{0.1pt} $\beta^-$ particles.
    \begin{multicols}{3}
    \begin{choices}
        \wrongchoice{3}
      \correctchoice{4}
        \wrongchoice{5}
        \wrongchoice{6}
        \wrongchoice{7}
    \end{choices}
    \end{multicols}
\end{question}
}

\element{halliday-mc}{
\begin{question}{halliday-ch42-q57}
    When ordinary sodium (\ce{^{23}Na}, $Z=11$) is bombarded with deuterons,
        the products are a neutron and:
    \begin{multicols}{2}
    \begin{choices}
        \wrongchoice{\ce{^{27}Al}, $Z = 13$}
        \wrongchoice{\ce{^{24}Na}, $Z = 11$}
        \wrongchoice{\ce{^{24}Mg}, $Z = 12$}
      \correctchoice{\ce{^{25}Mg}, $Z = 12$}
        \wrongchoice{\ce{^{20}Ne}, $Z = 10$}
    \end{choices}
    \end{multicols}
\end{question}
}

\element{halliday-mc}{
\begin{question}{halliday-ch42-q58}
    \ce{^{65}Cu} can be turned into \ce{^{66}Cu},
        with no accompany produce except a gamma, if bombarded with:
    \begin{multicols}{2}
    \begin{choices}
        \wrongchoice{protons}
      \correctchoice{neutrons}
        \wrongchoice{deuterons}
        \wrongchoice{electrons}
        \wrongchoice{alpha particles}
    \end{choices}
    \end{multicols}
\end{question}
}

\element{halliday-mc}{
\begin{question}{halliday-ch42-q59}
    Magnesium has atomic number 12, hydrogen has atomic number 1,
        and helium has atomic number 2. 
    In the nuclear reaction \ce{^{24}Mg + 2 H -> X + ^{4}He}, the quantity X is:
    \begin{multicols}{2}
    \begin{choices}
        \wrongchoice{\ce{^{23}Na} ($Z = 11$)}
        \wrongchoice{\ce{^{22}Ne} ($Z = 10$)}
        \wrongchoice{\ce{^{21}Na} ($Z = 11$)}
        \wrongchoice{\ce{^{21}Ne} ($Z = 10$)}
      \correctchoice{\ce{^{22}Na} ($Z = 11$)}
    \end{choices}
    \end{multicols}
\end{question}
}

\element{halliday-mc}{
\begin{question}{halliday-ch42-q60}
    Aluminum has atomic number 13, helium has atomic number 2,
        and silicon has atomic number 14. 
    In the nuclear reaction \ce{^{27}Al + ^{4}He -> ^{30}Si + X}, the particle X is:
    \begin{multicols}{2}
    \begin{choices}
        \wrongchoice{an $\alpha$ particle}
        \wrongchoice{a positron}
        \wrongchoice{an electron}
      \correctchoice{a proton}
        \wrongchoice{a neutron}
    \end{choices}
    \end{multicols}
\end{question}
}

\element{halliday-mc}{
\begin{question}{halliday-ch42-q61}
    The \ce{^{66}Cu} ($Z=29$) produced in a nuclear bombardment is unstable,
        changing to by the emission of:
    \begin{multicols}{2}
    \begin{choices}
        \wrongchoice{a proton}
        \wrongchoice{a gamma ray photon}
        \wrongchoice{a positron}
      \correctchoice{an electron}
        \wrongchoice{an alpha particle}
    \end{choices}
    \end{multicols}
\end{question}
}

\element{halliday-mc}{
\begin{question}{halliday-ch42-q62}
    When ordinary sulfur, \ce{^{32}S} ($Z=16$),
        is bombarded with neutrons, the products are \ce{^{32}P} ($Z=15$) and:
    \begin{multicols}{2}
    \begin{choices}
        \wrongchoice{alpha particles}
      \correctchoice{protons}
        \wrongchoice{deuterons}
        \wrongchoice{gamma ray particles}
        \wrongchoice{electrons}
    \end{choices}
    \end{multicols}
\end{question}
}

\element{halliday-mc}{
\begin{question}{halliday-ch42-q63}
    A certain nucleus, after absorbing a neutron, emits a $\beta^-$ and then splits into two alpha particles.
    The ($A$, $Z$) of the original nucleus must have been:
    \begin{multicols}{3}
    \begin{choices}
        \wrongchoice{6, 2}
        \wrongchoice{6, 3}
        \wrongchoice{7, 2}
      \correctchoice{7, 3}
        \wrongchoice{8, 4}
    \end{choices}
    \end{multicols}
\end{question}
}

\element{halliday-mc}{
\begin{question}{halliday-ch42-q64}
    When \ce{^{23}Na} ($Z=11$) is bombarded with protons,
        the products are \ce{^{20}Ne} ($Z=10$) and:
    \begin{choices}
        \wrongchoice{a neutron}
      \correctchoice{an alpha particle}
        \wrongchoice{a deuteron}
        \wrongchoice{a gamma ray particle}
        \wrongchoice{two beta particles}
    \end{choices}
\end{question}
}

\element{halliday-mc}{
\begin{question}{halliday-ch42-q65}
    Bombardment of \ce{^{28}Si} ($Z=14$) with alpha particles may produce:
    \begin{choices}
        \wrongchoice{a proton and \ce{^{31}P} ($Z=15$)}
        \wrongchoice{hydrogen and \ce{^{32}S} ($Z=16$)}
        \wrongchoice{a deuteron and \ce{^{27}Al} ($Z=13$)}
        \wrongchoice{helium and \ce{^{31}P} ($Z=15$)}
        \wrongchoice{\ce{^{35}Cl} ($Z=17$)}
    \end{choices}
\end{question}
}

\element{halliday-mc}{
\begin{question}{halliday-ch42-q66}
    The becquerel (\si{\becquerel}) is the correct unit to use in reporting the measurement of:
    \begin{choices}
      \correctchoice{the rate of decay of a radioactive source}
        \wrongchoice{the ability of a beam of gamma ray photons to produce ions in a target}
        \wrongchoice{the energy delivered by radiation to a target}
        \wrongchoice{the biological effect of radiation}
        \wrongchoice{none of the provided}
    \end{choices}
\end{question}
}

\element{halliday-mc}{
\begin{question}{halliday-ch42-q67}
    The gray (\si{\gray}) is the correct unit to use in reporting the measurement of:
    \begin{choices}
        \wrongchoice{the rate of decay of a radioactive source}
        \wrongchoice{the ability of a beam of gamma ray photons to produce ions in a target}
      \correctchoice{the energy per unit mass of target delivered by radiation to a target}
        \wrongchoice{the biological effect of radiation}
        \wrongchoice{none of the provided}
    \end{choices}
\end{question}
}

\element{halliday-mc}{
\begin{question}{halliday-ch42-q68}
    The sievert (\si{\sievert}) is the correct unit to use in reporting the measurement of:
    \begin{choices}
        \wrongchoice{the rate of decay of a radioactive source}
        \wrongchoice{the ability of a beam of gamma ray photons to produce ions in a target}
        \wrongchoice{the energy delivered by radiation to a target}
      \correctchoice{the biological effect of radiation}
        \wrongchoice{none of the provided}
    \end{choices}
\end{question}
}


\endinput


