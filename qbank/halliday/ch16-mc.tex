
%%--------------------------------------------------
%% Halliday: Fundamentals of Physics
%%--------------------------------------------------


%% Chapter 16: Waves I
%%--------------------------------------------------


%% Learning Objectives
%%--------------------------------------------------

%% 16.01: Identify the three main types of waves
%% 16.02: Distinguish between transverse waves and longitudinal waves.
%% 16.03: given a displacement function for a traverse wave, determine amplitude $y_m$, angular wave number $k$, angular frequency $\omega$, phase constant $\phi$, and the direction of travel, anc calculate the phase $kx\pm \omega t + \phi$ and the displacement at any given time and position.
%% 16.04: Given a displacement function for a traverse wave, calculate the time between two given displacements.
%% 16.05: Sketch a graph of a transverse wave as a function of position, identifying amplitude $y_m$, wavelength $\lambda$, where the slope is greatest, where it is zero, and where the string elements have positive velocity, negative velocity, and zero velocity.
%% 16.06: Given a graph of displacement versus time for a transverse wave, determine amplitude $y_m$ and period $T$.
%% 16.07: Describe the effect on a transverse wave of changing phase constant $\phi$.
%% 16.08: Apply the relation between the wave speed $v$, the distance traveled by the wave, and the time required for that travel.
%% 16.09: Apply the relationships between wave speed $v$, angular frequency $\omega$, angular wave number $k$, wavelength $\lambda$, period $T$, and frequency $f$.
%% 16.10: Describe the motion of a string element as a transverse wave moves through its location, and identify when its transverse speed is zero and when it is maximum.
%% 16.11: Calculate the transverse velocity $u(t)$ of a string element as a transverse wave moves through its location.
%% 16.12: Calculate the transverse acceleration $a(t)$ of a string element as a transverse wave moves through its location.
%% 16.13: Given a graph of displacement, transverse velocity, or transverse acceleration, determine the phase constant $\phi$.


%% Halliday Multiple Choice Questions
%%--------------------------------------------------
\element{halliday-mc}{
\begin{question}{halliday-ch16-q01}
    For a transverse wave on a string the string displacement is described by $y(x, t) = f(x - at)$,
        where $f$ is a given function and $a$ is a positive constant. 
    Which of the following does \emph{not} necessarily follow from this statement?
    \begin{choices}
        \wrongchoice{The shape of the string at time $t=0$ is given by $f(x)$.}
        \wrongchoice{The shape of the waveform does not change as it moves along the string.}
        \wrongchoice{The waveform moves in the positive $x$ direction.}
        \wrongchoice{The speed of the waveform is $a$.}
      \correctchoice{The speed of the waveform is $x/t$.}
    \end{choices}
\end{question}
}

\element{halliday-mc}{
\begin{question}{halliday-ch16-q02}
    A sinusoidal wave is traveling toward the right as shown. 
    Which letter correctly labels the amplitude of the wave?
    \begin{center}
    \begin{tikzpicture}
        %% NOTE:
    \end{tikzpicture}
    \end{center}
    \begin{multicols}{5}
    \begin{choices}[o]
        %% ans: D
        \wrongchoice{$A$}
        \wrongchoice{$B$}
        \wrongchoice{$C$}
        \wrongchoice{$D$}
        \wrongchoice{$E$}
    \end{choices}
    \end{multicols}
\end{question}
}

\element{halliday-mc}{
\begin{question}{halliday-ch16-q03}
    A sinusoidal wave is traveling toward the right as shown. 
    Which letter correctly labels the wavelength of the wave?
    \begin{center}
    \begin{tikzpicture}
        %% NOTE:
    \end{tikzpicture}
    \end{center}
    \begin{multicols}{5}
    \begin{choices}[o]
        %% ans: A
        \wrongchoice{$A$}
        \wrongchoice{$B$}
        \wrongchoice{$C$}
        \wrongchoice{$D$}
        \wrongchoice{$E$}
    \end{choices}
    \end{multicols}
\end{question}
}

\element{halliday-mc}{
\begin{question}{halliday-ch16-q04}
    In the diagram below, the interval PQ represents:
    \begin{center}
    \begin{tikzpicture}
        %% NOTE:
    \end{tikzpicture}
    \end{center}
    \begin{multicols}{2}
    \begin{choices}
        \wrongchoice{wavelength$/2$}
        \wrongchoice{wavelength}
        \wrongchoice{$2\times$amplitude}
      \correctchoice{period$/2$}
        \wrongchoice{period}
    \end{choices}
    \end{multicols}
\end{question}
}

\element{halliday-mc}{
\begin{question}{halliday-ch16-q05}
    Let $f$ be the frequency, $v$ the speed, and $T$ the period of a sinusoidal traveling wave. 
    The correct relationship is:
    \begin{multicols}{2}
    \begin{choices}
      \correctchoice{$f = \dfrac{1}{T}$}
        \wrongchoice{$f = v + T$}
        \wrongchoice{$f = vT$}
        \wrongchoice{$f = \dfrac{v}{T}$}
        \wrongchoice{$f = \dfrac{T}{v}$}
    \end{choices}
    \end{multicols}
\end{question}
}

\element{halliday-mc}{
\begin{question}{halliday-ch16-q06}
    Let $f$ be the frequency, $v$ the speed, and $T$ the period of a sinusoidal traveling wave.
    The angular frequency is given by:
    \begin{multicols}{3}
    \begin{choices}
        \wrongchoice{$\dfrac{1}{T}$}
      \correctchoice{$\dfrac{2\pi}{T}$}
        \wrongchoice{$vT$}
        \wrongchoice{$\dfrac{f}{T}$}
        \wrongchoice{$\dfrac{T}{f}$}
    \end{choices}
    \end{multicols}
\end{question}
}

\element{halliday-mc}{
\begin{question}{halliday-ch16-q07}
    The displacement of a string is given by
    \begin{equation*}
        y(x, t) = y_m \sin\left( kx + \omega t \right) .
    \end{equation*}
    The wavelength of the wave is:
    \begin{multicols}{3}
    \begin{choices}
        \wrongchoice{$\dfrac{2\pi k}{\omega}$}
        \wrongchoice{$\dfrac{k}{\omega}$}
        \wrongchoice{$\omega k$}
      \correctchoice{$\dfrac{2\pi}{k}$}
        \wrongchoice{$\dfrac{k}{2\pi}$}
    \end{choices}
    \end{multicols}
\end{question}
}

\element{halliday-mc}{
\begin{question}{halliday-ch16-q08}
    Three traveling sinusoidal waves are on identical strings,
        with the same tension. 
    The mathematical forms of the waves are $y_1 (x, t) = y_m\sin(3x - 6t)$,
        $y_2 (x, t) = y_m\sin(4x - 8t)$,
        and $y_3 (x, t) = y_m\sin(6x - 12t)$,
        where $x$ is in meters and $t$ is in seconds. 
    Match each mathematical form to the appropriate graph below.
    \begin{center}
    \begin{tikzpicture}
        %% NOTE:
    \end{tikzpicture}
    \end{center}
    \begin{choices}
      \correctchoice{$y_1:i$,   $y_2:ii$,   $y_3:ii$}
        \wrongchoice{$y_1:iii$, $y_2:ii$,   $y_3:i$}
        \wrongchoice{$y_1:i$,   $y_2:iii$,  $y_3:ii$}
        \wrongchoice{$y_1:ii$,  $y_2:i$,    $y_3:iii$}
        \wrongchoice{$y_1:iii$, $y_2:i$,    $y_3:ii$}
    \end{choices}
\end{question}
}

\element{halliday-mc}{
\begin{question}{halliday-ch16-q09}
    The displacement of a string is given by
    \begin{equation*}
        y(x, t) = y m sin(kx + ωt) .
    \end{equation*}
    The speed of the wave is:
    \begin{multicols}{3}
    \begin{choices}
        \wrongchoice{$\dfrac{2\pi k}{\omega}$}
      \correctchoice{$\dfrac{\omega}{k}$}
        \wrongchoice{$\omega k$}
        \wrongchoice{$\dfrac{2\pi}{k}$}
        \wrongchoice{$\dfrac{k}{2\pi}$}
    \end{choices}
    \end{multicols}
\end{question}
}

\element{halliday-mc}{
\begin{question}{halliday-ch16-q10}
    A wave is described by $y(x, t) = 0.1 \sin\left(3x + 10t\right)$, where $x$ is in meters, $y$ is in centimeters, and $t$ is in seconds. 
    The angular wave number is:
    \begin{multicols}{2}
    \begin{choices}
        \wrongchoice{\SI{0.10}{\radian\per\meter}}
        \wrongchoice{\SI[parse-numbers=false]{3\pi}{\radian\per\meter}}
        \wrongchoice{\SI{10}{\radian\per\meter}}
        \wrongchoice{\SI[parse-numbers=false]{10\pi}{\radian\per\meter}}
      \correctchoice{\SI{3.0}{\radian\per\centi\meter}}
    \end{choices}
    \end{multicols}
\end{question}
}

\element{halliday-mc}{
\begin{question}{halliday-ch16-q11}
    A wave is described by $y(x, t) = 0.1 \sin\left(3x - 10t\right)$, where $x$ is in meters, $y$ is in centimeters, and $t$ is in seconds. 
    The angular frequency is:
    \begin{multicols}{2}
    \begin{choices}
        \wrongchoice{\SI{0.10}{\radian\per\second}}
        \wrongchoice{\SI[parse-numbers=false]{3.0\pi}{\radian\per\second}}
        \wrongchoice{\SI[parse-numbers=false]{10\pi}{\radian\per\second}}
        \wrongchoice{\SI[parse-numbers=false]{20\pi}{\radian\per\second}}
      \correctchoice{\SI{10}{\radian\per\second}}
    \end{choices}
    \end{multicols}
\end{question}
}

\element{halliday-mc}{
\begin{question}{halliday-ch16-q12}
    Water waves in the sea are observed to have a wavelength of \SI{300}{\meter} and a frequency of \SI{0.07}{\hertz}.
    The speed of these waves is:
    \begin{multicols}{2}
    \begin{choices}
        \wrongchoice{\SI{0.00021}{\meter\per\second}}
        \wrongchoice{\SI{2.1}{\meter\per\second}}
      \correctchoice{\SI{21}{\meter\per\second}}
        \wrongchoice{\SI{210}{\meter\per\second}}
        \wrongchoice{none of the provided}
    \end{choices}
    \end{multicols}
\end{question}
}

\element{halliday-mc}{
\begin{question}{halliday-ch16-q13}
    Sinusoidal water waves are generated in a large ripple tank. 
    The waves travel at \SI{20}{\centi\meter\per\second} and their adjacent crests are \SI{5.0}{\centi\meter} apart. 
    The time required for each new whole cycle to be generated is:
    \begin{multicols}{3}
    \begin{choices}
        \wrongchoice{\SI{100}{\second}}
        \wrongchoice{\SI{4.0}{\second}}
        \wrongchoice{\SI{2.0}{\second}}
        \wrongchoice{\SI{0.5}{\second}}
      \correctchoice{\SI{0.25}{\second}}
    \end{choices}
    \end{multicols}
\end{question}
}

\element{halliday-mc}{
\begin{question}{halliday-ch16-q14}
    A traveling sinusoidal wave is shown below. 
    \begin{center}
    \begin{tikzpicture}
        %% NOTE:
    \end{tikzpicture}
    \end{center}
    At which point is the motion \ang{180} out of phase with the motion at point $P$?
    \begin{multicols}{5}
    \begin{choices}[o]
        \wrongchoice{$A$}
        \wrongchoice{$B$}
      \correctchoice{$C$}
        \wrongchoice{$D$}
        \wrongchoice{$E$}
    \end{choices}
    \end{multicols}
\end{question}
}

\element{halliday-mc}{
\begin{question}{halliday-ch16-q15}
    The displacement of a string carrying a traveling sinusoidal wave is given by
    \begin{equation*}
        y(x, t) = y_m \sin\left(kx - \omega t - \phi\right) .
    \end{equation*}
    At time $t=0$ the point at $x=0$ has a displacement of zero and is moving in the positive $y$ direction. 
    The phase constant $\phi$ is:
    \begin{multicols}{3}
    \begin{choices}
        \wrongchoice{\ang{45}}
        \wrongchoice{\ang{90}}
        \wrongchoice{\ang{135}}
      \correctchoice{\ang{180}}
        \wrongchoice{\ang{270}}
    \end{choices}
    \end{multicols}
\end{question}
}

\element{halliday-mc}{
\begin{question}{halliday-ch16-q16}
    The displacement of a string carrying a traveling sinusoidal wave is given by
    \begin{equation*}
        y(x, t) = y_m \sin\left( kx - \omega t - \phi\right) .
    \end{equation*}
    At time $t=0$ the point at $x=0$ has a velocity of zero and a positive displacement. 
    The phase constant $\phi$ is:
    \begin{multicols}{3}
    \begin{choices}
        \wrongchoice{\ang{45}}
        \wrongchoice{\ang{90}}
        \wrongchoice{\ang{135}}
        \wrongchoice{\ang{180}}
      \correctchoice{\ang{270}}
    \end{choices}
    \end{multicols}
\end{question}
}

\element{halliday-mc}{
\begin{question}{halliday-ch16-q17}
    The displacement of a string carrying a traveling sinusoidal wave is given by
    \begin{equation*}
        y(x, t) = y_m \sin\left( kx - \omega t - \phi\right) .
    \end{equation*}
    At time $t=0$ the point at $x=0$ has velocity $v_0$ and displacement $y_0$. 
    The phase constant $\phi$ is given by $\tan\phi=$
    \begin{multicols}{3}
    \begin{choices}
        \wrongchoice{$\dfrac{v_0 }{\omega y_0}$}
      \correctchoice{$\dfrac{\omega y_0}{v_0}$}
        \wrongchoice{$\dfrac{\omega v_0}{y_0}$}
        \wrongchoice{$\dfrac{y_0}{\omega v_0}$}
        \wrongchoice{$\dfrac{\omega v_0}{y_0}$}
    \end{choices}
    \end{multicols}
\end{question}
}

\element{halliday-mc}{
\begin{question}{halliday-ch16-q18}
    A sinusoidal transverse wave is traveling on a string. 
    Any point on the string:
    \begin{choices}
        \wrongchoice{moves in the same direction as the wave}
        \wrongchoice{moves in simple harmonic motion with a different frequency than that of the wave}
      \correctchoice{moves in simple harmonic motion with the same angular frequency as the wave}
        \wrongchoice{moves in uniform circular motion with a different angular speed than the wave}
        \wrongchoice{moves in uniform circular motion with the same angular speed as the wave}
    \end{choices}
\end{question}
}

\element{halliday-mc}{
\begin{question}{halliday-ch16-q19}
    Here are the equations for three waves traveling on separate strings. 
    \begin{description}
        \item[wave 1:] $y(x, t) = \left(\SI{2.0}{\milli\meter}\right) \sin\left[\left(\SI{4.0}{\per\meter}\right)x - \left(\SI{3.0}{\per\second}\right)t \right]$
        \item[wave 2:] $y(x, t) = \left(\SI{1.0}{\milli\meter}\right) \sin\left[\left(\SI{8.0}{\per\meter}\right)x - \left(\SI{4.0}{\per\second}\right)t \right]$
        \item[wave 3:] $y(x, t) = \left(\SI{1.0}{\milli\meter}\right) \sin\left[\left(\SI{4.0}{\per\meter}\right)x - \left(\SI{8.0}{\per\second}\right)t \right]$
    \end{description}
    Rank them according to the maximum transverse speed,
        least to greatest.
    \begin{multicols}{2}
    \begin{choices}
        \wrongchoice{1, 2, 3}
        \wrongchoice{1, 3, 2}
      \correctchoice{2, 1, 3}
        \wrongchoice{2, 3, 1}
        \wrongchoice{3, 1, 2}
    \end{choices}
    \end{multicols}
\end{question}
}

\element{halliday-mc}{
\begin{question}{halliday-ch16-q20}
    The transverse wave shown is traveling from left to right in a medium. 
    \begin{center}
    \begin{tikzpicture}
        %% NOTE:
    \end{tikzpicture}
    \end{center}
    The direction of the instantaneous velocity of the medium at point $P$ is:
    \begin{multicols}{2}
    \begin{choices}
        %% ANS is A
        \wrongchoice{
            \begin{tikzpicture}
            \end{tikzpicture}
        }
    \end{choices}
    \end{multicols}
\end{question}
}

\element{halliday-mc}{
\begin{question}{halliday-ch16-q21}
    A wave traveling to the right on a stretched string is shown below. 
    \begin{center}
    \begin{tikzpicture}
        %% NOTE:
    \end{tikzpicture}
    \end{center}
    The direction of the instantaneous velocity of the point $P$ on the string is:
    \begin{multicols}{2}
    \begin{choices}
        %% ANS is B
        \wrongchoice{
            \begin{tikzpicture}
            \end{tikzpicture}
        }
    \end{choices}
    \end{multicols}
\end{question}
}

\element{halliday-mc}{
\begin{question}{halliday-ch16-q22}
    Sinusoidal waves travel on five different strings, all with the same tension. 
    Four of the strings have the same linear mass density,
        but the fifth has a different linear mass density. 
    Use the mathematical forms of the waves, given below,
        to identify the string with the different linear mass density. 
    In the expressions $x$ and $y$ are in centimeters and $t$ is in seconds.
    \begin{choices}
        \wrongchoice{$y\left(x, t\right) = \left(\SI{2}{\centi\meter}\right)\sin\left(2x -4t\right)$}
        \wrongchoice{$y\left(x, t\right) = \left(\SI{2}{\centi\meter}\right)\sin\left(4x -10t\right)$}
        \wrongchoice{$y\left(x, t\right) = \left(\SI{2}{\centi\meter}\right)\sin\left(6x -12t\right)$}
        \wrongchoice{$y\left(x, t\right) = \left(\SI{2}{\centi\meter}\right)\sin\left(8x -16t\right)$}
      \correctchoice{$y\left(x, t\right) = \left(\SI{2}{\centi\meter}\right)\sin\left(10x-20t\right)$}
    \end{choices}
\end{question}
}

\element{halliday-mc}{
\begin{question}{halliday-ch16-q23}
    Any point on a string carrying a sinusoidal wave is moving with its maximum speed when:
    \begin{choices}
        \wrongchoice{the magnitude of its acceleration is a maximum}
        \wrongchoice{the magnitude of its displacement is a maximum}
      \correctchoice{the magnitude of its displacement is a minimum}
        \wrongchoice{the magnitude of its displacement is half the amplitude}
        \wrongchoice{the magnitude of its displacement is one-fourth the amplitude}
    \end{choices}
\end{question}
}

\element{halliday-mc}{
\begin{question}{halliday-ch16-q24}
    The mathematical forms for three sinusoidal traveling waves are given by
    \begin{description}
        \item[wave 1:] $y\left(x,t\right) = \left(\SI{2}{\centi\meter}\right) sin\left(3x-6t\right)$
        \item[wave 2:] $y\left(x,t\right) = \left(\SI{3}{\centi\meter}\right) sin\left(4x-12t\right)$
        \item[wave 3:] $y\left(x,t\right) = \left(\SI{4}{\centi\meter}\right) sin\left(5x-11t\right)$
    \end{description}
    where $x$ is in meters and $t$ is in seconds. 
    Of these waves:
    \begin{choices}
        \wrongchoice{wave 1 has the greatest wave speed and the greatest maximum transverse string speed}
        \wrongchoice{wave 2 has the greatest wave speed and wave 1 has the greatest maximum transverse string speed}
        \wrongchoice{wave 3 has the greatest wave speed and the greatest maximum transverse string speed}
      \correctchoice{wave 2 has the greatest wave speed and wave 3 has the greatest maximum transverse string speed}
        \wrongchoice{wave 3 has the greatest wave speed and wave 2 has the greatest maximum transverse string speed}
    \end{choices}
\end{question}
}

\element{halliday-mc}{
\begin{question}{halliday-ch16-q25}
    Suppose the maximum speed of a string carrying a sinusoidal wave is $v_s$.
    When the displacement of a point on the string is half its maximum,
        the speed of the point is:
    \begin{multicols}{3}
    \begin{choices}
        \wrongchoice{$\dfrac{v_s}{2}$}
        \wrongchoice{$2v_s$}
        \wrongchoice{$\dfrac{v_s}{4}$}
        \wrongchoice{$\dfrac{3 v_s}{4}$}
      \correctchoice{$\dfrac{\sqrt{3} v_s}{2}$}
    \end{choices}
    \end{multicols}
\end{question}
}

\element{halliday-mc}{
\begin{question}{halliday-ch16-q26}
    A string carries a sinusoidal wave with an amplitude of \SI{2.0}{\centi\meter} and a frequency of \SI{100}{\hertz}.
    The maximum speed of any point on the string is:
    \begin{multicols}{2}
    \begin{choices}
        \wrongchoice{\SI{2.0}{\meter\per\second}}
        \wrongchoice{\SI{4.0}{\meter\per\second}}
        \wrongchoice{\SI{6.3}{\meter\per\second}}
      \correctchoice{\SI{13}{\meter\per\second}}
        \wrongchoice{unknown (not enough information is given)}
    \end{choices}
    \end{multicols}
\end{question}
}

\element{halliday-mc}{
\begin{question}{halliday-ch16-q27}
    A transverse traveling sinusoidal wave on a string has a frequency of \SI{100}{\hertz},
        a wavelength of \SI{0.040}{\meter}, and an amplitude of \SI{2.0}{\milli\meter}.
    %The maximum velocity in \si{\meter\per\second} of any point on the string is:
    The maximum velocity of any point on the string is:
    \begin{multicols}{3}
    \begin{choices}
        \wrongchoice{\SI{0.2}{\meter\per\second}}
      \correctchoice{\SI{1.3}{\meter\per\second}}
        \wrongchoice{\SI{4}{\meter\per\second}}
        \wrongchoice{\SI{15}{\meter\per\second}}
        \wrongchoice{\SI{25}{\meter\per\second}}
    \end{choices}
    \end{multicols}
\end{question}
}

\element{halliday-mc}{
\begin{question}{halliday-ch16-q28}
    A transverse traveling sinusoidal wave on a string has a frequency of \SI{100}{\hertz},
        a wavelength of \SI{0.040}{\meter}, and an amplitude of \SI{2.0}{\milli\meter}.
    %The maximum acceleration in m/s 2 of any point on the string is:
    The maximum acceleration of any point on the string is:
    \begin{multicols}{2}
    \begin{choices}
        %\wrongchoice{\SI{0}{\meter\per\second\squared}}
        \wrongchoice{zero}
        \wrongchoice{\SI{130}{\meter\per\second\squared}}
        \wrongchoice{\SI{395}{\meter\per\second\squared}}
      \correctchoice{\SI{790}{\meter\per\second\squared}}
        \wrongchoice{\SI{1600}{\meter\per\second\squared}}
    \end{choices}
    \end{multicols}
\end{question}
}

\element{halliday-mc}{
\begin{question}{halliday-ch16-q29}
    The speed of a sinusoidal wave on a string depends on:
    \begin{choices}
        \wrongchoice{the frequency of the wave}
        \wrongchoice{the wavelength of the wave}
        \wrongchoice{the length of the string}
        \wrongchoice{the tension in the string}
        \wrongchoice{the amplitude of the wave}
    \end{choices}
\end{question}
}

\element{halliday-mc}{
\begin{question}{halliday-ch16-q30}
    The time required for a small pulse to travel from $A$ to $B$ on a stretched cord shown is \emph{not} altered by changing:
    \begin{choices}
        \wrongchoice{the linear mass density of the cord}
        \wrongchoice{the length between $A$ and $B$}
      \correctchoice{the shape of the pulse}
        \wrongchoice{the tension in the cord}
        \wrongchoice{none of the above (changes in all alter the time)}
    \end{choices}
\end{question}
}

\element{halliday-mc}{
\begin{question}{halliday-ch16-q31}
    The diagrams show three identical strings that have been put under tension by suspending blocks of \SI{5}{\kilo\gram} each. 
    \begin{center}
    \begin{tikzpicture}
        %% NOTE: 1 x 3
    \end{tikzpicture}
    \end{center}
    For which is the wave speed the greatest?
    \begin{multicols}{3}
    \begin{choices}
        \wrongchoice{1}
        \wrongchoice{2}
        \wrongchoice{3}
      \correctchoice{1 and 3 tie}
        \wrongchoice{2 and 3 tie}
    \end{choices}
    \end{multicols}
\end{question}
}

\element{halliday-mc}{
\begin{question}{halliday-ch16-q32}
    For a given medium, the frequency of a wave is:
    \begin{choices}
        \wrongchoice{independent of wavelength}
        \wrongchoice{proportional to wavelength}
      \correctchoice{inversely proportional to wavelength}
        \wrongchoice{proportional to the amplitude}
        \wrongchoice{inversely proportional to the amplitude}
    \end{choices}
\end{question}
}

\element{halliday-mc}{
\begin{question}{halliday-ch16-q33}
    The tension in a string with a linear mass density of \SI{0.0010}{\kilo\gram\per\meter} is \SI{0.40}{\newton}. 
    A sinusoidal wave with a wavelength of \SI{20}{\centi\meter} on this string has a frequency of:
    \begin{multicols}{2}
    \begin{choices}
        \wrongchoice{\SI{0.0125}{\hertz}}
        \wrongchoice{\SI{0.25}{\hertz}}
      \correctchoice{\SI{100}{\hertz}}
        \wrongchoice{\SI{630}{\hertz}}
        \wrongchoice{\SI{2000}{\hertz}}
    \end{choices}
    \end{multicols}
\end{question}
}

\element{halliday-mc}{
\begin{question}{halliday-ch16-q34}
    When a \SI{100}{\hertz} oscillator is used to generate a sinusoidal wave on a certain string the wavelength is \SI{10}{\centi\meter}.
    When the tension in the string is doubled the generator produces a wave with a frequency and wavelength of:
    \begin{choices}
        \wrongchoice{\SI{200}{\hertz} and \SI{20}{\centi\meter}}
        \wrongchoice{\SI{141}{\hertz} and \SI{10}{\centi\meter}}
        \wrongchoice{\SI{100}{\hertz} and \SI{20}{\centi\meter}}
      \correctchoice{\SI{100}{\hertz} and \SI{14}{\centi\meter}}
        \wrongchoice{\SI{50}{\hertz} and \SI{14}{\centi\meter}}
    \end{choices}
\end{question}
}

\element{halliday-mc}{
\begin{question}{halliday-ch16-q35}
    A source of frequency $f$ sends waves of wavelength $\lambda$ traveling with speed $v$ in some medium.
    If the frequency is changed from $f$ to $2f$,
        then the new wavelength and new speed are (respectively):
    \begin{multicols}{3}
    \begin{choices}
        \wrongchoice{$2\lambda$, $v$}
      \correctchoice{$\dfrac{\lambda}{2}$, $v$}
        \wrongchoice{$\lambda$, $2v$}
        \wrongchoice{$\lambda$, $\dfrac{v}{2}$}
        \wrongchoice{$\dfrac{\lambda}{2}$, $2v$}
    \end{choices}
    \end{multicols}
\end{question}
}

\element{halliday-mc}{
\begin{question}{halliday-ch16-q36}
    A long string is constructed by joining the ends of two shorter strings. 
    The tension in the strings is the same but string I has 4 times the linear mass density of string II. 
    When a sinusoidal wave passes from string I to string II:
    \begin{choices}
        \wrongchoice{the frequency decreases by a factor of 4}
        \wrongchoice{the frequency decreases by a factor of 2}
        \wrongchoice{the wavelength decreases by a factor of 4}
      \correctchoice{the wavelength decreases by a factor of 2}
        \wrongchoice{the wavelength increases by a factor of 2}
    \end{choices}
\end{question}
}

\element{halliday-mc}{
\begin{question}{halliday-ch16-q37}
    Three separate strings are made of the same material. 
    String 1 has length $L$ and tension $\tau$,
        string 2 has length $2L$ and tension $2\tau$,
        and string 3 has length $3L$ and tension $3\tau$. 
    A pulse is started at one end of each string. 
    If the pulses start at the same time,
        the order in which they reach the other end is:
    \begin{choices}
      \correctchoice{1, 2, 3}
        \wrongchoice{3, 2, 1}
        \wrongchoice{2, 3, 1}
        \wrongchoice{3, 1, 2}
        \wrongchoice{they all take the same time}
    \end{choices}
\end{question}
}

\element{halliday-mc}{
\begin{question}{halliday-ch16-q38}
    A long string is constructed by joining the ends of two shorter strings. 
    The tension in the strings is the same but string I has 4 times the linear mass density of string II. 
    When a sinusoidal wave passes from string I to string II:
    \begin{choices}
        \wrongchoice{the frequency decreases by a factor of 4}
        \wrongchoice{the frequency decreases by a factor of 2}
        \wrongchoice{the wave speed decreases by a factor of 4}
        \wrongchoice{the wave speed decreases by a factor of 2}
      \correctchoice{the wave speed increases by a factor of 2}
    \end{choices}
\end{question}
}

\element{halliday-mc}{
\begin{question}{halliday-ch16-q39}
    Two identical but separate strings, with the same tension,
        carry sinusoidal waves with the same frequency. 
    Wave $A$ has a amplitude that is twice that of wave $B$ and transmits energy at a rate that is \rule[-0.1pt]{4em}{0.1pt} that of wave $B$.
    \begin{multicols}{2}
    \begin{choices}
        \wrongchoice{half}
        \wrongchoice{twice}
        \wrongchoice{one-fourth}
      \correctchoice{four times}
        \wrongchoice{eight times}
    \end{choices}
    \end{multicols}
\end{question}
}

%% NOTE: 40 is duplicate of 39
%% NOTE: reword to something else, maybe twice the frequency?
\element{halliday-mc}{
\begin{question}{halliday-ch16-q40}
    Two identical but separate strings, with the same tension,
        carry sinusoidal waves with the same frequency. 
    Wave $A$ has an amplitude that is twice that of wave $B$ and transmits energy at a rate that is \rule[-0.1pt]{4em}{0.1pt} that of wave $B$.
    \begin{multicols}{2}
    \begin{choices}
        \wrongchoice{half}
        \wrongchoice{twice}
        \wrongchoice{one-fourth}
      \correctchoice{four times}
        \wrongchoice{eight times}
    \end{choices}
    \end{multicols}
\end{question}
}

\element{halliday-mc}{
\begin{question}{halliday-ch16-q41}
    A sinusoidal wave is generated by moving the end of a string up and down periodically. 
    The generator must supply the greatest power when the end of the string
    \begin{choices}
        \wrongchoice{has its greatest acceleration}
        \wrongchoice{has its greatest displacement}
        \wrongchoice{has half its greatest displacement}
        \wrongchoice{has one-fourth its greatest displacement}
      \correctchoice{has its least displacement}
    \end{choices}
\end{question}
}

\element{halliday-mc}{
\begin{question}{halliday-ch16-q42}
    A sinusoidal wave is generated by moving the end of a string up and down periodically. 
    The generator does not supply any power when the end of the string
    \begin{choices}
        \wrongchoice{has its least acceleration}
      \correctchoice{has its greatest displacement}
        \wrongchoice{has half its greatest displacement}
        \wrongchoice{has one-fourth its greatest displacement}
        \wrongchoice{has its least displacement}
    \end{choices}
\end{question}
}

\element{halliday-mc}{
\begin{question}{halliday-ch16-q43}
    The sum of two sinusoidal traveling waves is a sinusoidal traveling wave only if:
    \begin{choices}
        \wrongchoice{their amplitudes are the same and they travel in the same direction.}
        \wrongchoice{their amplitudes are the same and they travel in opposite directions.}
      \correctchoice{their frequencies are the same and they travel in the same direction.}
        \wrongchoice{their frequencies are the same and they travel in opposite directions.}
        \wrongchoice{their frequencies are the same and their amplitudes are the same.}
    \end{choices}
\end{question}
}

\element{halliday-mc}{
\begin{question}{halliday-ch16-q44}
    Two traveling sinusoidal waves interfere to produce a wave with the mathematical form
    \begin{equation*}
        %% Orig has alpha and not phi here?
        y(x, t) = y_m\sin\left(kx + \omega t + \phi\right) .
    \end{equation*}
    If the value of $\phi$ is appropriately chosen,
        the two waves might be:
    \begin{choices}
        \wrongchoice{$y_1 \left(x,t\right) = \left(\dfrac{y_m}{3}\right)\sin\left(kx + \omega t\right)$ and \\
                     $y 2 \left(x, t\right) = \left(\dfrac{y_m}{3}\right)\sin\left(kx + \omega t + \phi\right)$}
        \wrongchoice{$y_1 \left(x,t\right) = 0.7y_m\sin\left(kx-\omega t\right)$ and \\
                     $y 2 \left(x, t\right) = 0.7y_m\sin\left(kx-\omega t + \phi\right)$}
        \wrongchoice{$y_1 \left(x,t\right) = 0.7y_m\sin\left(kx-\omega t\right)$ and \\
                     $y 2 \left(x, t\right) = 0.7y_m\sin\left(kx+\omega t + \phi\right)$}
        \wrongchoice{$y_1 \left(x,t\right) = 0.7y_m\sin\left[\left(\dfrac{kx}{2}\right) - \left(\dfrac{\omega t}{2}\right)\right]$ and \\
                     $y_2 \left(x, t\right) = 0.7y_m\sin\left[\left(\dfrac{kx}{2}\right) - \left(\dfrac{\omega t}{2}\right) +\phi\right]$}
      \correctchoice{$y_1 \left(x,t\right) = 0.7y_m\sin\left(kx+\omega t\right)$ and \\
                     $y_2 \left(x, t\right) = 0.7y_m\sin\left(kx+\omega t+\phi\right)$}
    \end{choices}
\end{question}
}

\element{halliday-mc}{
\begin{question}{halliday-ch16-q45}
    Fully constructive interference between two sinusoidal waves of the same frequency occurs only if they:
    \begin{choices}
        \wrongchoice{travel in opposite directions and are in phase}
        \wrongchoice{travel in opposite directions and are \ang{180} out of phase}
      \correctchoice{travel in the same direction and are in phase}
        \wrongchoice{travel in the same direction and are \ang{180} out of phase}
        \wrongchoice{travel in the same direction and are \ang{90} out of phase}
    \end{choices}
\end{question}
}

\element{halliday-mc}{
\begin{question}{halliday-ch16-q46}
    Fully destructive interference between two sinusoidal waves of the same frequency and amplitude occurs only if they:
    \begin{choices}
        \wrongchoice{travel in opposite directions and are in phase}
        \wrongchoice{travel in opposite directions and are \ang{180} out of phase}
        \wrongchoice{travel in the same direction and are in phase}
      \correctchoice{travel in the same direction and are \ang{180} out of phase}
        \wrongchoice{travel in the same direction and are \ang{90} out of phase}
    \end{choices}
\end{question}
}

\element{halliday-mc}{
\begin{question}{halliday-ch16-q47}
    Two sinusoidal waves travel in the same direction and have the same frequency. 
    Their amplitudes are $y_{1m}$ and $y_{2m}$.
    The smallest possible amplitude of the resultant wave is:
    \begin{choices}
        \wrongchoice{$y_{1m} + y_{2m}$ and occurs if they are \ang{180} out of phase}
      \correctchoice{$|y_{1m} - y_{2m}|$ and occurs if they are \ang{180} out of phase}
        \wrongchoice{$y_{1m} + y_{2m}$ and occurs if they are in phase}
        \wrongchoice{$|y_{1m} - y_{2m}|$ and occurs if they are in phase}
        \wrongchoice{$|y_{1m} - y_{2m}|$ and occurs if they are \ang{90} out of phase}
    \end{choices}
\end{question}
}

\element{halliday-mc}{
\begin{question}{halliday-ch16-q48}
    Two sinusoidal waves have the same angular frequency,
        the same amplitude $y_m$, and travel in the same direction in the same medium. 
    If they differ in phase by \ang{50},
        the amplitude of the resultant wave is given by:
    \begin{multicols}{3}
    \begin{choices}
        \wrongchoice{$0.64 y_m$}
        \wrongchoice{$1.3 y_m$}
        \wrongchoice{$0.91 y_m$}
      \correctchoice{$1.8 y_m$}
        \wrongchoice{$0.35 y_m$}
    \end{choices}
    \end{multicols}
\end{question}
}

\element{halliday-mc}{
\begin{question}{halliday-ch16-q49}
    Two separated sources emit sinusoidal traveling waves that have the same wavelength $\lambda$ and are in phase at their respective sources. 
    One travels a distance $l_1$ to get to the observation point while the other travels a distance $l_2$. 
    The amplitude is a minimum at the observation point if $l_1-l_2$ is:
    \begin{choices}
      \correctchoice{an odd multiple of $\dfrac{\lambda}{2}$}
        \wrongchoice{an odd multiple of $\dfrac{\lambda}{4}$}
        \wrongchoice{a multiple of $\lambda$}
        \wrongchoice{an odd multiple of $\dfrac{\pi}{2}$}
        \wrongchoice{a multiple of $\pi$}
    \end{choices}
\end{question}
}

\element{halliday-mc}{
\begin{question}{halliday-ch16-q50}
    Two separated sources emit sinusoidal traveling waves that have the same wavelength $\lambda$ and are in phase at their respective sources. 
    One travels a distance $l_1$ to get to the observation point while the other travels a distance $l_2$. 
    The amplitude is a maximum at the observation point if $l_1-l_2$ is:
    \begin{choices}
        \wrongchoice{an odd multiple of $\dfrac{\lambda}{2}$}
        \wrongchoice{an odd multiple of $\dfrac{\lambda}{4}$}
      \correctchoice{a multiple of $\lambda$}
        \wrongchoice{an odd multiple of $\dfrac{\pi}{2}$}
        \wrongchoice{a multiple of $\pi$}
    \end{choices}
\end{question}
}

\element{halliday-mc}{
\begin{question}{halliday-ch16-q51}
    Two sources, $S_1$ and $S_2$,
        each emit waves of wavelength $\lambda$ in the same medium. 
    \begin{center}
    \begin{tikzpicture}
        %% NOTE:
    \end{tikzpicture}
    \end{center}
    The phase difference between the two waves,
        at the point $P$ shown, is $\left(2\pi/\lambda\right)\left(l_2-l_1\right) + \epsilon$.
    The quantity is:
    \begin{choices}
        \wrongchoice{the distance $S_1 S_2$}
        \wrongchoice{the angle $S_1 PS_2$}
        \wrongchoice{$\dfrac{\pi}{2}$}
      \correctchoice{the phase difference between the two sources}
        \wrongchoice{zero for transverse waves, $\pi$ for longitudinal waves}
    \end{choices}
\end{question}
}

\element{halliday-mc}{
\begin{question}{halliday-ch16-q52}
    A wave on a stretched string is reflected from a fixed end P of the string. 
    The phase difference, at $P$,
        between the incident and reflected waves is:
    \begin{choices}
        \wrongchoice{zero}
      \correctchoice{$\pi\,\si{\radian}$}
        \wrongchoice{$\dfrac{\pi}{2}\,\si{\radian}$}
        \wrongchoice{depends on the velocity of the wave}
        \wrongchoice{depends on the frequency of the wave}
    \end{choices}
\end{question}
}

\element{halliday-mc}{
\begin{question}{halliday-ch16-q53}
    The sinusoidal wave
    \begin{equation*}
        y\left(x,t\right) = y_m\sin\left(kx - \omega t\right)
    \end{equation*}
    is incident on the fixed end of a string at $x=L$. 
    The reflected wave is given by:
    \begin{choices}
        \wrongchoice{$y_m \sin\left(kx + \omega t\right)$}
        \wrongchoice{$-y_m\sin\left(kx + \omega t\right)$}
        \wrongchoice{$y_m \sin\left(kx + \omega t - kL\right)$}
      \correctchoice{$y_m \sin\left(kx + \omega t - 2kL\right)$}
        \wrongchoice{$-y_m\sin\left(kx + \omega t + 2kL\right)$}
    \end{choices}
\end{question}
}

\element{halliday-mc}{
\begin{question}{halliday-ch16-q54}
    A wave on a string is reflected from a fixed end. 
    The reflected wave:
    \begin{choices}
        \wrongchoice{is in phase with the original wave at the end}
      \correctchoice{is \ang{180} out of phase with the original wave at the end}
        \wrongchoice{has a larger amplitude than the original wave}
        \wrongchoice{has a larger speed than the original wave}
        \wrongchoice{cannot be transverse}
    \end{choices}
\end{question}
}

\element{halliday-mc}{
\begin{question}{halliday-ch16-q55}
    A standing wave:
    \begin{choices}
      \correctchoice{can be constructed from two similar waves traveling in opposite directions}
        \wrongchoice{must be transverse}
        \wrongchoice{must be longitudinal}
        \wrongchoice{has motionless points that are closer than half a wavelength}
        \wrongchoice{has a wave velocity that differs by a factor of two from what it would be for a traveling wave}
    \end{choices}
\end{question}
}

\element{halliday-mc}{
\begin{question}{halliday-ch16-q56}
    Which of the following represents a standing wave?
    \begin{choices}
        %% NOTE: try formatting this better
        \wrongchoice{$y = \left(\SI{6.0}{\milli\meter}\right)\sin\left[\left(\SI{3.0}{\per\meter}\right)x + \\
                          \left(\SI{2.0}{\per\second}\right)t\right] -\left(\SI{6.0}{\milli\meter}\right) \cos\left[\left(\SI{3.0}{\per\meter} \right)x+\num{2.0}\right]$}
      \correctchoice{$y = \left(\SI{6.0}{\milli\meter}\right)\cos\left[\left(\SI{3.0}{\per\meter}\right)x - \\
                          \left(\SI{2.0}{\per\second}\right)t\right] +\left(\SI{6.0}{\milli\meter}\right) \cos\left[\left(\SI{2.0}{\per\second}\right)t+\left(\SI{3.0}{\per\meter}\right)x\right]$}
        \wrongchoice{$y = \left(\SI{6.0}{\milli\meter}\right)\cos\left[\left(\SI{3.0}{\per\meter}\right)x - \\
                          \left(\SI{2.0}{\per\second}\right)t\right] -\left(\SI{6.0}{\milli\meter}\right) \sin\left[\left(\SI{2.0}{\per\second}\right)t-\num{3.0}\right]$}
        \wrongchoice{$y = \left(\SI{6.0}{\milli\meter}\right)\sin\left[\left(\SI{3.0}{\per\meter}\right)x - \\
                          \left(\SI{2.0}{\per\second}\right)t\right] -\left(\SI{6.0}{\milli\meter}\right) \cos\left[\left(\SI{2.0}{\per\second}\right)t+\left(\SI{3.0}{\per\meter}\right)x\right]$}
        \wrongchoice{$y = \left(\SI{6.0}{\milli\meter}\right)\sin\left[\left(\SI{3.0}{\per\meter}\right)x\right] + \\
                          \left(\SI{6.0}{\milli\meter}\right)\cos\left[\left(\SI{2.0}{\per\second}\right)t\right]$}
    \end{choices}
\end{question}
}

\element{halliday-mc}{
\begin{question}{halliday-ch16-q57}
    When a certain string is clamped at both ends,
        the lowest four resonant frequencies are 50, 100, 150, and 200 Hz. 
    When the string is also clamped at its midpoint,
        the lowest four resonant frequencies are:
    \begin{choices}
        %% NOTE: format these
        \wrongchoice{50, 100, 150, and 200 Hz}
        \wrongchoice{50, 150, 250, and 300 Hz}
      \correctchoice{100, 200, 300, and 400 Hz}
        \wrongchoice{25, 50, 75, and 100 Hz}
        \wrongchoice{75, 150, 225, and 300 Hz}
    \end{choices}
\end{question}
}

\element{halliday-mc}{
\begin{question}{halliday-ch16-q58}
    When a certain string is clamped at both ends,
        the lowest four resonant frequencies are measured to be 100, 150, 200, and 250 Hz. 
    One of the resonant frequencies (below 200 Hz) is missing. 
    What is it?
    \begin{multicols}{3}
    \begin{choices}
        \wrongchoice{\SI{25}{\hertz}}
      \correctchoice{\SI{50}{\hertz}}
        \wrongchoice{\SI{75}{\hertz}}
        \wrongchoice{\SI{125}{\hertz}}
        \wrongchoice{\SI{225}{\hertz}}
    \end{choices}
    \end{multicols}
\end{question}
}

\element{halliday-mc}{
\begin{question}{halliday-ch16-q59}
    Two traveling waves $y_1 = A\sin\left[ k\left(x-vt\right)\right]$
        and $y_2 = A\sin\left[k\left(x+vt\right)\right]$ are superposed on the same string. 
    The distance between the adjacent nodes is:
    \begin{multicols}{3}
    \begin{choices}
        \wrongchoice{$\dfrac{vt}{\pi}$}
        \wrongchoice{$\dfrac{vt}{2\pi}$}
        \wrongchoice{$\dfrac{\pi}{2k}$}
      \correctchoice{$\dfrac{\pi}{k}$}
        \wrongchoice{$\dfrac{2\pi}{k}$}
    \end{choices}
    \end{multicols}
\end{question}
}

\element{halliday-mc}{
\begin{question}{halliday-ch16-q60}
    If $\lambda$ is the wavelength of each of the component sinusoidal traveling waves that form a standing wave,
        the distance between adjacent nodes in the standing wave is:
    \begin{multicols}{3}
    \begin{choices}
        \wrongchoice{$\dfrac{\lambda}{4}$}
      \correctchoice{$\dfrac{\lambda}{2}$}
        \wrongchoice{$\dfrac{3\lambda}{4}$}
        \wrongchoice{$\lambda$}
        \wrongchoice{$2\lambda$}
    \end{choices}
    \end{multicols}
\end{question}
}

\element{halliday-mc}{
\begin{question}{halliday-ch16-q61}
    A standing wave pattern is established in a string as shown. 
    \begin{center}
    \begin{tikzpicture}
        %% NOTE:
    \end{tikzpicture}
    \end{center}
    The wavelength of one of the component traveling waves is:
    \begin{multicols}{3}
    \begin{choices}
        \wrongchoice{\SI{0.25}{\meter}}
        \wrongchoice{\SI{0.5}{\meter}}
        \wrongchoice{\SI{1}{\meter}}
        \wrongchoice{\SI{2}{\meter}}
      \correctchoice{\SI{4}{\meter}}
    \end{choices}
    \end{multicols}
\end{question}
}

\element{halliday-mc}{
\begin{question}{halliday-ch16-q62}
    Standing waves are produced by the interference of two traveling sinusoidal waves,
        each of frequency \SI{100}{\hertz}. 
    The distance from the second node to the fifth node is \SI{60}{\centi\meter}. 
    The wavelength of each of the two original waves is:
    \begin{multicols}{3}
    \begin{choices}
        \wrongchoice{\SI{50}{\centi\meter}}
      \correctchoice{\SI{40}{\centi\meter}}
        \wrongchoice{\SI{30}{\centi\meter}}
        \wrongchoice{\SI{20}{\centi\meter}}
        \wrongchoice{\SI{15}{\centi\meter}}
    \end{choices}
    \end{multicols}
\end{question}
}

\element{halliday-mc}{
\begin{question}{halliday-ch16-q63}
    A string of length \SI{100}{\centi\meter} is held fixed at both ends and vibrates in a standing wave pattern.
    The wavelengths of the constituent traveling waves \emph{cannot} be:
    \begin{multicols}{3}
    \begin{choices}
      \correctchoice{\SI{400}{\centi\meter}}
        \wrongchoice{\SI{200}{\centi\meter}}
        \wrongchoice{\SI{100}{\centi\meter}}
        \wrongchoice{\SI{66.7}{\centi\meter}}
        \wrongchoice{\SI{50}{\centi\meter}}
    \end{choices}
    \end{multicols}
\end{question}
}

\element{halliday-mc}{
\begin{question}{halliday-ch16-q64}
    A string of length $L$ is clamped at each end and vibrates in a standing wave pattern. 
    The wavelengths of the constituent traveling waves \emph{cannot} be:
    \begin{multicols}{3}
    \begin{choices}
        \wrongchoice{$L$}
        \wrongchoice{$2L$}
        \wrongchoice{$\dfrac{L}{2}$}
        \wrongchoice{$\dfrac{2L}{3}$}
      \correctchoice{$4L$}
    \end{choices}
    \end{multicols}
\end{question}
}

\element{halliday-mc}{
\begin{question}{halliday-ch16-q65}
    Two sinusoidal waves, each of wavelength \SI{5}{\meter} and amplitude \SI{10}{\centi\meter},
        travel in opposite directions on a \SI{20}{\meter} long stretched string that is clamped at each end. 
    Excluding the nodes at the ends of the string,
        how many nodes appear in the resulting standing wave?
    \begin{multicols}{3}
    \begin{choices}
        \wrongchoice{\num{3}}
        \wrongchoice{\num{4}}
        \wrongchoice{\num{5}}
      \correctchoice{\num{7}}
        \wrongchoice{\num{8}}
    \end{choices}
    \end{multicols}
\end{question}
}

\element{halliday-mc}{
\begin{question}{halliday-ch16-q66}
    A string, clamped at its ends, vibrates in three segments. 
    The string is \SI{100}{\centi\meter} long. 
    The wavelength is:
    \begin{multicols}{2}
    \begin{choices}
        \wrongchoice{\SI{33.3}{\centi\meter}}
      \correctchoice{\SI{66.7}{\centi\meter}}
        \wrongchoice{\SI{150}{\centi\meter}}
        \wrongchoice{\SI{300}{\centi\meter}}
        \wrongchoice{need to know the frequency}
    \end{choices}
    \end{multicols}
\end{question}
}

\element{halliday-mc}{
\begin{question}{halliday-ch16-q67}
    A stretched string, clamped at its ends, vibrates in its fundamental frequency. 
    To double the fundamental frequency,
        one can change the string tension by a factor of:
    \begin{multicols}{3}
    \begin{choices}
        \wrongchoice{$2$}
      \correctchoice{$4$}
        \wrongchoice{$\sqrt{2}$}
        \wrongchoice{$\dfrac{1}{2}$}
        \wrongchoice{$\dfrac{1}{\sqrt{2}}$}
    \end{choices}
    \end{multicols}
\end{question}
}

\element{halliday-mc}{
\begin{question}{halliday-ch16-q68}
    When a string is vibrating in a standing wave pattern the power transmitted across an antinode,
    compared to the power transmitted across a node, is:
    \begin{choices}
        \wrongchoice{more}
        \wrongchoice{less}
      \correctchoice{the same (zero)}
        \wrongchoice{the same (non-zero)}
        \wrongchoice{sometimes more, sometimes less, and sometimes the same}
    \end{choices}
\end{question}
}

\element{halliday-mc}{
\begin{question}{halliday-ch16-q69}
    A \SI{40}{\centi\meter} long string,
        with one end clamped and the other free to move transversely,
        is vibrating in its fundamental standing wave mode. 
    The wavelength of the constituent traveling waves is:
    \begin{multicols}{3}
    \begin{choices}
        \wrongchoice{\SI{10}{\centi\meter}}
        \wrongchoice{\SI{20}{\centi\meter}}
        \wrongchoice{\SI{40}{\centi\meter}}
        \wrongchoice{\SI{80}{\centi\meter}}
        \wrongchoice{\SI{160}{\centi\meter}}
    \end{choices}
    \end{multicols}
\end{question}
}

\element{halliday-mc}{
\begin{question}{halliday-ch16-q70}
    A \SI{30}{\centi\meter} long string,
        with one end clamped and the other free to move transversely,
        is vibrating in its second harmonic. 
    The wavelength of the constituent traveling waves is:
    \begin{multicols}{3}
    \begin{choices}
        \wrongchoice{\SI{10}{\centi\meter}}
        \wrongchoice{\SI{30}{\centi\meter}}
      \correctchoice{\SI{40}{\centi\meter}}
        \wrongchoice{\SI{60}{\centi\meter}}
        \wrongchoice{\SI{120}{\centi\meter}}
    \end{choices}
    \end{multicols}
\end{question}
}

\element{halliday-mc}{
\begin{question}{halliday-ch16-q71}
    A \SI{40}{\centi\meter} long string,
        with one end clamped and the other free to move transversely,
        is vibrating in its fundamental standing wave mode. 
    If the wave speed is \SI{320}{\centi\meter\per\second} the frequency is:
    \begin{multicols}{3}
    \begin{choices}
        \wrongchoice{\SI{32}{\hertz}}
        \wrongchoice{\SI{16}{\hertz}}
        \wrongchoice{\SI{8}{\hertz}}
        \wrongchoice{\SI{4}{\hertz}}
      \correctchoice{\SI{2}{\hertz}}
    \end{choices}
    \end{multicols}
\end{question}
}


\endinput


