

%% This defines the lua function mycommand()
\directlua{require('./qbank/lua/common.lua')}

%%----------------------------------------
%% Error Analysis
%%----------------------------------------

%% Objectives
%%----------------------------------------

%% 1. Propogate error for calculated values
%% 2. Calculate the expectation and variance of any distribution


%% Propogation of Error
%%----------------------------------------
\element{error}{
\begin{question}{ErrorQ01}
\luaexec{
    %% Question
    local Q = [[
        A circle has a radius of
            \string\SI{\%.2f \string\pm \%.2f}{\string\centi\string\meter}.
        What is the absolute uncertainty in the area
            of the circle?
        [Area of circle is $A = \pi r^2$]
    ]]
    %% Random Values
    local radius = round(100 * math.random(),2)
    local delta = round(math.sqrt(math.random()),2)
    %% Random Permutations
    local tab = {}
    for i=1,7 do
        tab[i] = i
    end
    tab = permute(tab,7,7)
    %% 1 foot = .3048 meter
    local area = math.pi * math.pow(radius,2)
    local right = area * math.sqrt(2) *  (delta/radius)
    local wrong = {
        right * delta,
        right / delta,
        right * radius,
        right / radius,
        right * math.sqrt(2),
        right / math.sqrt(2),
        area,
    }
    %% Print Question
    tex.print( string.format(Q,radius,delta) )
    %% Formatting
    local SI = [[\string\SI[retain-zero-exponent]{\%.2e}{\string\centi\string\meter\string\squared}]]
    %% Print MC Options
    tex.print( BeginMulticolsTwo )
        tex.print( BeginChoices )
            tex.print( string.format( CorrectChoice,string.format(SI,right) ) )
            for i=1,nwrong do
                tex.print( string.format( WrongChoice,string.format(SI,wrong[tab[i]]) ) )
            end
        tex.print( EndChoices )
    tex.print( EndMulticols )
}
\end{question}
}

\element{error}{
\begin{question}{ErrorQ02}
\luaexec{
    %% Question
    local Q = [[
        A rectangular slab of metal has dimensions
            \string\SI{\%.2f \string\pm \%.2f}{\string\centi\string\meter}
            by \string\SI{\%.2f \string\pm \%.2f}{\string\centi\string\meter}.
        What is the absolute uncertainty in the area
            of the metal slab?
    ]]
    %% Random Values
    local w  = round(100 * math.random(),2)
    local dw = round(math.sqrt(math.random()),2)
    local l  = round(100 * math.random(),2)
    local dl = round(math.sqrt(math.random()),2)
    %% Random Permutations
    local tab = {}
    for i=1,9 do
        tab[i] = i
    end
    tab = permute(tab,9,9)
    %% 1 foot = .3048 meter
    local area = w * l
    local right = area * math.sqrt( math.pow((dw/w),2) + math.pow((dl/l),2) )
    local wrong = {
        area / math.sqrt( math.pow((dw/w),2) + math.pow((dl/l),2) ),
        area * math.sqrt( math.pow((w/dw),2) + math.pow((l/dl),2) ),
        area / math.sqrt( math.pow((w/dw),2) + math.pow((l/dl),2) ),
        right * area,
        right / area,
        area * math.sqrt( (w/dw) + (l/dl) ),
        area / math.sqrt( (w/dw) + (l/dl) ),
        area * math.sqrt( (dw/w) + (dl/l) ),
        area / math.sqrt( (dw/w) + (dl/l) ),
    }
    %% Print Question
    tex.print( string.format(Q,w,dw,l,dl) )
    %% Formatting
    local SI = [[\string\SI{\%.2e}{\string\centi\string\meter\string\squared}]]
    %% Print MC Options
    tex.print( BeginMulticolsTwo )
        tex.print( BeginChoices )
            tex.print( string.format( CorrectChoice,string.format(SI,right) ) )
            for i=1,nwrong do
                tex.print( string.format( WrongChoice,string.format(SI,wrong[tab[i]]) ) )
            end
        tex.print( EndChoices )
    tex.print( EndMulticols )
}
\end{question}
}


%% Marble Problems
%%----------------------------------------
\element{error}{
\begin{question}{ErrorQ03}
\luaexec{
    %% Question
    local Q = [[
        A single marble has a mass of
            \string\SI{\%.2f \string\pm \%.2f}{\string\kilo\string\gram}.
        What is the absolute uncertainty in the mass of
            \string\num{\%d} marbles?
    ]]
    %% Random Values
    local m  = round(100 * math.random(),2)
    local dm = round(math.sqrt(math.random()),2)
    local n  = math.random(15,100)
    %% Random Permutations
    local tab = {}
    for i=1,8 do
        tab[i] = i
    end
    tab = permute(tab,8,8)
    %% 1 foot = .3048 meter
    local right = dm * math.sqrt( n )
    local wrong = {
        right * n,
        right / n,
        right * dm,
        right / dm,
        right * m,
        right / m,
        right * math.sqrt(n),
        right / math.sqrt(n),
    }
    %% Print Question
    tex.print( string.format(Q,m,dm,n) )
    %% Formatting
    local SI = [[\string\SI[retain-zero-exponent]{\%.2e}{\string\kilo\string\gram}]]
    %% Print MC Options
    tex.print( BeginMulticolsTwo )
        tex.print( BeginChoices )
            tex.print( string.format( CorrectChoice,string.format(SI,right) ) )
            for i=1,nwrong do
                tex.print( string.format( WrongChoice,string.format(SI,wrong[tab[i]]) ) )
            end
        tex.print( EndChoices )
    tex.print( EndMulticols )
}
\end{question}
}


\element{error}{
\begin{question}{ErrorQ04}
\luaexec{
    %% Question
    local Q = [[
        The mass of \string\num{\%d} identical marbles are
            measured to have a mass of
            \string\SI{\%.2f \string\pm \%.2f}{\string\gram}.
        What is the absolute uncertainty in the mass of
            \string\num{1} marble?
    ]]
    %% Random Values
    local n  = math.random(15,100)
    local m  = round(100 * math.random(),2)
    local dm = round(math.sqrt(math.random()),2)
    %% Random Permutations
    local tab = {}
    for i=1,8 do
        tab[i] = i
    end
    tab = permute(tab,8,8)
    %% 1 foot = .3048 meter
    local right = dm / math.sqrt( n )
    local wrong = {
        right * math.sqrt(n),
        right / math.sqrt(n),
        right * dm,
        right / dm,
        right * n,
        right / n,
        right * m,
        right / m,
    }
    %% Print Question
    tex.print( string.format(Q,n,m,dm) )
    %% Formatting
    local SI = [[\string\SI[retain-zero-exponent]{\%.2e}{\string\gram}]]
    %% Print MC Options
    tex.print( BeginMulticolsTwo )
        tex.print( BeginChoices )
            tex.print( string.format( CorrectChoice,string.format(SI,right) ) )
            for i=1,nwrong do
                tex.print( string.format( WrongChoice,string.format(SI,wrong[tab[i]]) ) )
            end
        tex.print( EndChoices )
    tex.print( EndMulticols )
}
\end{question}
}


\element{error}{
\begin{question}{ErrorQ05}
\luaexec{
    %% Question
    local Q = [[
        A single marble has a volume of
            \string\SI{\%.2f \string\pm \%.2f}{\string\milli\string\liter}.
        What is the absolute uncertainty in the volume of
            \string\num{\%d} marbles?
    ]]
    %% Random Values
    local v  = round(100 * math.random(),2)
    local dv = round(math.sqrt(math.random()),2)
    local n  = math.random(15,100)
    %% Random Permutations
    local tab = {}
    for i=1,8 do
        tab[i] = i
    end
    tab = permute(tab,8,8)
    %% 1 foot = .3048 meter
    local right = math.sqrt(n) * dv
    local wrong = {
        right * dv,
        right / dv,
        right * n,
        right / n,
        right * v,
        right / v,
        right * math.sqrt(n),
        right / math.sqrt(n),
    }
    %% Print Question
    tex.print( string.format(Q,v,dv,n) )
    %% Formatting
    local SI = [[\string\SI[retain-zero-exponent]{\%.2e}{\string\milli\string\liter}]]
    %% Print MC Options
    tex.print( BeginMulticolsTwo )
        tex.print( BeginChoices )
            tex.print( string.format( CorrectChoice,string.format(SI,right) ) )
            for i=1,nwrong do
                tex.print( string.format( WrongChoice,string.format(SI,wrong[tab[i]]) ) )
            end
        tex.print( EndChoices )
    tex.print( EndMulticols )
}
\end{question}
}


\element{error}{
\begin{question}{ErrorQ06}
\luaexec{
    %% Question
    local Q = [[
        The volume of \string\num{\%d} identical marbles are
            measured to have a volume of
            \string\SI{\%.2f \string\pm \%.2f}{\string\milli\string\liter}.
        What is the absolute uncertainty in the volume of
            \string\SI{1}{marble}?
    ]]
    %% Random Values
    local v  = round(100 * math.random(),2)
    local dv = round(math.sqrt(math.random()),2)
    local n  = math.random(15,100)
    %% Random Permutations
    local tab = {}
    for i=1,8 do
        tab[i] = i
    end
    tab = permute(tab,8,8)
    %% 1 foot = .3048 meter
    local right = dv / math.sqrt( n )
    local wrong = {
        dv,
        dv / n,
        right * n,
        right / n,
        right * dv,
        right / dv,
        right * v,
        right / v,
    }
    %% Print Question
    tex.print( string.format(Q,n,v,dv) )
    %% Formatting
    local SI = [[\string\SI[retain-zero-exponent]{\%.2e}{\string\milli\string\liter}]]
    %% Print MC Options
    tex.print( BeginMulticolsTwo )
        tex.print( BeginChoices )
            tex.print( string.format( CorrectChoice,string.format(SI,right) ) )
            for i=1,nwrong do
                tex.print( string.format( WrongChoice,string.format(SI,wrong[tab[i]]) ) )
            end
        tex.print( EndChoices )
    tex.print( EndMulticols )
}
\end{question}
}


%% Uncertainty or Standard deviation
%%----------------------------------------
\element{error}{
\begin{question}{ErrorQ07}
\luaexec{
    %% Question
    local Q = [[
        You record the time it takes a pendulum
            to make \string\num{10} complete oscillations.
        You repeat the measurement another four times.
        They are recorded as follows:
            \string\SI{\%.2f}{\string\second},
            \string\SI{\%.2f}{\string\second},
            \string\SI{\%.2f}{\string\second},
            \string\SI{\%.2f}{\string\second},
            \string\SI{\%.2f}{\string\second}.
        What is the absolute uncertainty in the measurement
            for a single complete oscillation
            of this pendulum?
    ]]
    %% Random Values
    local time = round(math.random() * math.random(10,20),2)
    local value1 = {}
    local value2 = {}
    local sum1 = 0
    local sum2 = 0
    for i=1,5 do
        value1[i] = time + round((0.5*gaussian()),2)
        value2[i] = math.pow(value1[i],2)
        sum1 = sum1 + value1[i]
        sum2 = sum2 + value2[i] 
    end
    %% Random Permutations
    local tab = {}
    for i=1,9 do
        tab[i] = i
    end
    tab = permute(tab,9,9)
    %% Calculate values
    local right = math.sqrt( (sum2/5) - math.pow((sum1/5),2) ) / math.sqrt(10)
    local wrong = {
        right * math.sqrt(10),
        right * math.sqrt(10) / 10,
        right / math.sqrt(10),
        math.sqrt(math.abs((sum2/5) - (sum1/5))),
        math.sqrt(math.abs((sum2/5) - (sum1/5))) / 10,
        math.sqrt(math.abs((sum2/5) - (sum1/5))) / math.sqrt(10),
        math.sqrt(math.abs((sum2) - (sum1))),
        math.sqrt(math.abs((sum2) - (sum1))) / 10,
        math.sqrt(math.abs((sum2) - (sum1))) / math.sqrt(10),
    }
    % Print Question
    tex.print( string.format(Q,value1[1],value1[2],value1[3],value1[4],value1[5]) )
    %% Formatting
    local SI = [[ \string\SI[retain-zero-exponent]{\%.2e}{\string\second} ]]
    %% Print MC Options
    tex.print( BeginMulticolsTwo )
        tex.print( BeginChoices )
            tex.print( string.format( CorrectChoice, string.format(SI,right) ) )
            for i=1,nwrong do
                tex.print( string.format( WrongChoice, string.format(SI,wrong[tab[i]]) ) )
            end
        tex.print( EndChoices )
    tex.print( EndMulticols )
}
\end{question}
}


\element{error}{
\begin{question}{ErrorQ08}
\luaexec{
    %% Question
    local Q = [[
        You measure the time it takes a ball
            to roll down an incline plane.
        You repeat the measurement another four times.
        They are recorded as follows:
            \string\SI{\%.2f}{\string\second},
            \string\SI{\%.2f}{\string\second},
            \string\SI{\%.2f}{\string\second},
            \string\SI{\%.2f}{\string\second},
            \string\SI{\%.2f}{\string\second}.
        What is the absolute uncertainty in the time
            it takes the ball to roll down
            the incline plane?
    ]]
    %% Random Values
    local time = round(math.random() * math.random(10,20),2)
    local value1 = {}
    local value2 = {}
    local sum1 = 0
    local sum2 = 0
    for i=1,5 do
        value1[i] = time + round(0.5*gaussian(),2)
        value2[i] = math.pow(value1[i],2)
        sum1 = sum1 + value1[i]
        sum2 = sum2 + value2[i] 
    end
    %% Random Permutations
    local tab = {}
    for i=1,8 do
        tab[i] = i
    end
    tab = permute(tab,8,8)
    %% Calculate values
    local right = math.sqrt( (sum2/5) - math.pow((sum1/5),2) )
    local wrong = {
        right * math.sqrt(5),
        right / math.sqrt(5),
        math.sqrt( math.abs((sum2/5) - (sum1/5)) ),
        math.sqrt( math.abs(sum2 - math.pow(sum1/5,2)) ),
        (sum2/5) - (sum1/5),
        (sum2/5) - math.pow(sum1/5,2),
        sum1/5,
        sum2/5,
    }
    %% Print Question
    tex.print( string.format(Q,value1[1],value1[2],value1[3],value1[4],value1[5]) )
    %% Formatting
    local SI = [[ \string\SI[retain-zero-exponent]{\%.2e}{\string\second} ]]
    %% Print MC Options
    tex.print( BeginMulticolsTwo )
        tex.print( BeginChoices )
            tex.print( string.format( CorrectChoice, string.format(SI,right) ) )
            for i=1,nwrong do
                tex.print( string.format( WrongChoice, string.format(SI,wrong[tab[i]]) ) )
            end
        tex.print( EndChoices )
    tex.print( EndMulticols )
}
\end{question}
}


%% Experimental Measurements
%%----------------------------------------
\element{error}{
\begin{question}{ErrorQ09}
\luaexec{
    %% Question
    local Q = [[
        You record the time it takes a pendulum
            to make \string\num{10} complete oscillations.
        You repeat the measurement another four times.
        They are recorded as follows:
            \string\SI{\%.2f}{\string\second},
            \string\SI{\%.2f}{\string\second},
            \string\SI{\%.2f}{\string\second},
            \string\SI{\%.2f}{\string\second},
            \string\SI{\%.2f}{\string\second}.
        What is the absolute uncertainty in the measurement
            for a single complete oscillation
            of this pendulum?
    ]]
    %% Random Values
    local time = round(math.random() * math.random(10,20),2)
    local value1 = {}
    local value2 = {}
    local sum1 = 0
    local sum2 = 0
    for i=1,5 do
        value1[i] = time + round(gaussian(),2)
        value2[i] = math.pow(value1[i],2)
        sum1 = sum1 + value1[i]
        sum2 = sum2 + value2[i] 
    end
    %% Random Permutations
    local tab = {}
    for i=1,8 do
        tab[i] = i
    end
    tab = permute(tab,8,8)
    %% Calculate values
    local right = math.sqrt( (sum2/5) - math.pow((sum1/5),2) ) / math.sqrt(10)
    local wrong = {
        right * math.sqrt(10),
        right * math.sqrt(10) / 10,
        math.sqrt( (sum2/5) - (sum1/5) ),
        math.sqrt( (sum2/5) - (sum1/5) ) / 10,
        math.sqrt( (sum2/5) - (sum1/5) ) / math.sqrt(10),
        math.sqrt( (sum2) - (sum1) ),
        math.sqrt( (sum2) - (sum1) ) / 10,
        math.sqrt( (sum2) - (sum1) ) / math.sqrt(10),
    }
    % Print Question
    tex.print( string.format(Q,value1[1],value1[2],value1[3],value1[4],value1[5]) )
    %% Formatting
    local SI = [[ \string\SI{\%.2e}{\string\second} ]]
    %% Print MC Options
    tex.print( BeginMulticolsTwo )
        tex.print( BeginChoices )
            tex.print( string.format( CorrectChoice, string.format(SI,right) ) )
            for i=1,nwrong do
                tex.print( string.format( WrongChoice, string.format(SI,wrong[tab[i]]) ) )
            end
        tex.print( EndChoices )
    tex.print( EndMulticols )
}
\end{question}
}

