

%% This defines the lua function mycommand()
%\directlua{require('./qbank/lua/common.lua')}


%% Objectives
%%--------------------------------------------------

%% 1. Construct an empirical test to falsify a scientific claim


%% Falsification
%%--------------------------------------------------

%% Consider differentiating the letter O from the number 0

\element{wrong}{
\begin{questionmult}{WrongQ01}
\luaexec{
    local Q = [[
        Each of the four figures below represents a card.
        There is always a letter on one side of the card and a number on the other side.
        Your job is to verify whether or not the following rule is true:
            If there is a vowel on one side of the card,
            then there must be an odd number on the other side. 
        Which cards must you turn over to test the truth of the rule?
    ]]
    local vowel = {[[A]], [[E]], [[I]], [[O]], [[U]],}
    local consonant = {[[H]],[[J]], [[K]], [[M]], [[N]],}
    local odd = {1, 3, 5, 7, 9,}
    local even = {2, 4, 6, 8, 10,}
    local n = {
        math.random(1,5),
        math.random(1,5),
        math.random(1,5),
        math.random(1,5),
    }
    tex.print( Q )
    tex.print( BeginMulticolsFour )
        tex.print( BeginChoices )
            tex.print( string.format( CorrectChoice, vowel[n[1]]) )
            tex.print( string.format( WrongChoice, consonant[n[2]]) )
            tex.print( string.format( CorrectChoice, even[n[3]] ) )
            tex.print( string.format( WrongChoice, odd[n[4]] ) )
        tex.print( EndChoices )
    tex.print( EndMulticols )
}
\end{questionmult}
}

%% NOTE: Changed Q01 odd to even
\element{wrong}{
\begin{questionmult}{WrongQ02}
\luaexec{
    local Q = [[
        Each of the four figures below represents a card.
        There is always a letter on one side of the card and a number on the other side.
        Your job is to verify whether or not the following rule is true:
            If there is a vowel on one side of the card,
            then there must be an even number on the other side. 
        Which cards must you turn over to test the truth of the rule?
    ]]
    local vowel = {[[A]], [[E]], [[I]], [[O]], [[U]],}
    local consonant = {[[H]],[[J]], [[K]], [[M]], [[N]],}
    local odd = {1, 3, 5, 7, 9,}
    local even = {2, 4, 6, 8, 10,}
    local n = {
        math.random(1,5),
        math.random(1,5),
        math.random(1,5),
        math.random(1,5),
    }
    tex.print( Q )
    tex.print( BeginMulticolsFour )
        tex.print( BeginChoices )
            tex.print( string.format( CorrectChoice, vowel[n[1]]) )
            tex.print( string.format( WrongChoice, consonant[n[2]]) )
            tex.print( string.format( WrongChoice, even[n[3]] ) )
            tex.print( string.format( CorrectChoice, odd[n[4]] ) )
        tex.print( EndChoices )
    tex.print( EndMulticols )
}
\end{questionmult}
}

%% NOTE: Changed Q01 vowel to consonant
\element{wrong}{
\begin{questionmult}{WrongQ03}
\luaexec{
    local Q = [[
        Each of the four figures below represents a card.
        There is always a letter on one side of the card and a number on the other side.
        Your job is to verify whether or not the following rule is true:
            If there is a consonant on one side of the card,
            then there must be an odd number on the other side. 
        Which cards must you turn over to test the truth of the rule?
    ]]
    local vowel = {[[A]], [[E]], [[I]], [[O]], [[U]],}
    local consonant = {[[H]],[[J]], [[K]], [[M]], [[N]],}
    local odd = {1, 3, 5, 7, 9,}
    local even = {2, 4, 6, 8, 10,}
    local n = {
        math.random(1,5),
        math.random(1,5),
        math.random(1,5),
        math.random(1,5),
    }
    tex.print( Q )
    tex.print( BeginMulticolsFour )
        tex.print( BeginChoices )
            tex.print( string.format( WrongChoice, vowel[n[1]]) )
            tex.print( string.format( CorrectChoice, consonant[n[2]]) )
            tex.print( string.format( CorrectChoice, even[n[3]] ) )
            tex.print( string.format( WrongChoice, odd[n[4]] ) )
        tex.print( EndChoices )
    tex.print( EndMulticols )
}
\end{questionmult}
}

%% NOTE: Changed Q01 odd to even and vowel to consonant
\element{wrong}{
\begin{questionmult}{WrongQ04}
\luaexec{
    local Q = [[
        Each of the four figures below represents a card.
        There is always a letter on one side of the card and a number on the other side.
        Your job is to verify whether or not the following rule is true:
            If there is a consonant on one side of the card,
            then there must be an odd number on the other side. 
        Which cards must you turn over to test the truth of the rule?
    ]]
    local vowel = {[[A]], [[E]], [[I]], [[O]], [[U]],}
    local consonant = {[[H]],[[J]], [[K]], [[M]], [[N]],}
    local odd = {1, 3, 5, 7, 9,}
    local even = {2, 4, 6, 8, 10,}
    local n = {
        math.random(1,5),
        math.random(1,5),
        math.random(1,5),
        math.random(1,5),
    }
    tex.print( Q )
    tex.print( BeginMulticolsFour )
        tex.print( BeginChoices )
            tex.print( string.format( WrongChoice, vowel[n[1]]) )
            tex.print( string.format( CorrectChoice, consonant[n[2]]) )
            tex.print( string.format( CorrectChoice, odd[n[3]] ) )
            tex.print( string.format( WrongChoice, even[n[4]] ) )
        tex.print( EndChoices )
    tex.print( EndMulticols )
}
\end{questionmult}
}

\begin{comment}
\element{wrong}{
\begin{questionmult}{WrongQ05}
    You are serving as the chaperon and bouncer at a local student bar and coffee house.
    Rather than standing at the door checking IDs all the time,
        you have occupied a table so you can do some work.
    When patrons come in and give their order,
        the servers bring you cards with the patron's order
        on one side and their best guess of the patron's age on the other.
    You then decide whether to go and check IDs.
    (The servers can be assumed to be trustworthy and are pretty good guessers.)
    %% Need to word so you know what the q wants
    \begin{multicols}{2}
        \begin{choices}
          \correctchoice{16}
            \wrongchoice{Coke}
            \wrongchoice{52}
          \correctchoice{Gin}
        \end{choices}
    \end{multicols}
\end{questionmult}
}

\element{wrong}{
\begin{questionmult}{WrongQ06}
    %Now consider another version. 
    You are an officer at the border of a country. 
    Each of the four cards that follow represents a traveler. 
    One side of the card lists whether the person is entering the
        country or is in transit (just passing through). 
    The other side of the card shows what vaccinations the person has received.
    You must make sure that any person who is entering the country is
        vaccinated against cholera.
    %% Reword so question is more clear
    Which cards must you turn over to test the truth of the rule?
    %(Cheng & Holyoak, 1985)
    %% TODO: CHeck
    \begin{multicols}{2}
        \begin{choices}
          \correctchoice{\fbox{\LARGE Entering}}
            \wrongchoice{\fbox{\LARGE Transit}}
            \wrongchoice{\fbox{\LARGE Cholera Mumps Typhoid}}
          \correctchoice{\fbox{\LARGE Flu Mumps}}
        \end{choices}
    \end{multicols}
\end{questionmult}
}
\end{comment}

