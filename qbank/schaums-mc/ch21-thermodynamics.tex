
%%--------------------------------------------------
%% Schaum's Outline of Applied Physics
%%--------------------------------------------------


%% Chapter 21: Thermodynamics
%%--------------------------------------------------


%% Schaum's Multiple Choice Questions
%%--------------------------------------------------
\element{schaums-mc}{
\begin{question}{ch21-Q01}
    A heat engine takes in heat at one temperature and turns:
    \begin{choices}
        \wrongchoice{all of it into work}
      \correctchoice{some of it into work and rejects the rest at a lower temperature}
        \wrongchoice{some of it into work and rejects the rest at the same temperature}
        \wrongchoice{some of it into work and rejects the rest at a higher temperature}
    \end{choices}
\end{question}
}

\element{schaums-mc}{
\begin{question}{ch21-Q02}
    A process that can be reversed without energy input from an outside source is one that takes place at constant:
    \begin{choices}
        \wrongchoice{pressure}
        \wrongchoice{density}
        \wrongchoice{velocity}
      \correctchoice{temperature}
    \end{choices}
\end{question}
}

\element{schaums-mc}{
\begin{question}{ch21-Q03}
    In an adiabatic process in a system,
    \begin{choices}
        \wrongchoice{its temperature stays the same}
        \wrongchoice{its pressure stays the same}
      \correctchoice{no heat enters or leaves it}
        \wrongchoice{no work is done by or on it}
    \end{choices}
\end{question}
}

\element{schaums-mc}{
\begin{questionmult}{ch21-Q04}
    Without work being done on it,
        a gas cannot be:
    \begin{choices}
      \correctchoice{compressed isobarically}
      \correctchoice{compressed isothermally}
      \correctchoice{compressed adiabatically}
        \wrongchoice{expanded adiabatically}
    \end{choices}
\end{questionmult}
}

\element{schaums-mc}{
\begin{question}{ch21-Q05}
    To be completely efficient (which is impossible),
        the exhaust temperature of a frictionless heat engine would have to be:
    \begin{choices}
      \correctchoice{\SI{0}{\kelvin}}
        \wrongchoice{\SI{273}{\kelvin}}
        \wrongchoice{less than its intake temperature}
        \wrongchoice{the same as its intake temperature}
    \end{choices}
\end{question}
}

\element{schaums-mc}{
\begin{question}{ch21-Q06}
    The Carnot cycle does not include an:
    \begin{choices}
      \correctchoice{isobaric expansion}
        \wrongchoice{isothermal expansion}
        \wrongchoice{adiabatic expansion}
        \wrongchoice{adiabatic compression}
    \end{choices}
\end{question}
}

\element{schaums-mc}{
\begin{question}{ch21-Q07}
    The efficiency of a Carnot engine operating between the absolute temperatures $T_1$ and $T_2$ is:
    \begin{choices}
        \wrongchoice{equal to $T_2/T_1$}
        \wrongchoice{\SI{100}{\percent}}
      \correctchoice{the maximum possible between these temperatures}
        \wrongchoice{the same as that of an actual engine operating between these temperatures}
    \end{choices}
\end{question}
}

\element{schaums-mc}{
\begin{question}{ch21-Q08}
    Which of the following engines is the most efficient?
    \begin{choices}
        \wrongchoice{gasoline engine}
        \wrongchoice{diesel engine}
        \wrongchoice{steam engine}
      \correctchoice{Carnot engine}
    \end{choices}
\end{question}
}

\element{schaums-mc}{
\begin{question}{ch21-Q09}
    The fuel in a diesel engine is ignited by:
    \begin{choices}
        \wrongchoice{a spark plug}
      \correctchoice{the hot air into which it is injected}
        \wrongchoice{being compressed until it is hot enough}
        \wrongchoice{exhaust gases left over from the previous cycle}
    \end{choices}
\end{question}
}

\element{schaums-mc}{
\begin{question}{ch21-Q10}
    The heat a refrigerator absorbs from its contents is:
    \begin{choices}
      \correctchoice{less than it gives off}
        \wrongchoice{the same amount it gives off}
        \wrongchoice{more than it gives off}
        \wrongchoice{any of the above, depending on its design}
    \end{choices}
\end{question}
}

\element{schaums-mc}{
\begin{question}{ch21-Q11}
    Four kilojoules of heat is given off by a gas when it is compressed from \SI{0.08}{\meter\cubed} to \SI{0.05}{\meter\cubed} under a pressure of \SI{200}{\kilo\pascal}.
    The internal energy of the gas:
    \begin{choices}
        \wrongchoice{decreases}
        \wrongchoice{is unchanged}
      \correctchoice{increases}
        \wrongchoice{it depends on the initial and final temperatures}
    \end{choices}
\end{question}
}

\element{schaums-mc}{
\begin{question}{ch21-Q12}
    A Carnot engine that absorbs heat at \SI{300}{\degreeCelsius} and exhausts heat at \SI{100}{\degreeCelsius} has an efficiency of:
    \begin{multicols}{2}
    \begin{choices}
        \wrongchoice{\SI{33}{\percent}}
      \correctchoice{\SI{35}{\percent}}
        \wrongchoice{\SI{65}{\percent}}
        \wrongchoice{\SI{67}{\percent}}
    \end{choices}
    \end{multicols}
\end{question}
}

\element{schaums-mc}{
\begin{question}{ch21-Q13}
    If a Carnot engine absorbs \SI{10}{\kilo\joule} of heat per cycle when it operates between \SI{500}{\kelvin} and \SI{400}{\kelvin},
        the work it does per cycle is:
    \begin{multicols}{2}
    \begin{choices}
      \correctchoice{\SI{2}{\kilo\joule}}
        \wrongchoice{\SI{2.5}{\kilo\joule}}
        \wrongchoice{\SI{8}{\kilo\joule}}
        \wrongchoice{\SI{10}{\kilo\joule}}
    \end{choices}
    \end{multicols}
\end{question}
}

\element{schaums-mc}{
\begin{question}{ch21-Q14}
    To have an efficiency of 40 percent,
        a heat engine that exhausts heat at \SI{350}{\kelvin} must absorb heat at no less than:
    \begin{multicols}{2}
    \begin{choices}
        \wrongchoice{\SI{210}{\kelvin}}
      \correctchoice{\SI{583}{\kelvin}}
        \wrongchoice{\SI{875}{\kelvin}}
        \wrongchoice{\SI{1038}{\kelvin}}
    \end{choices}
    \end{multicols}
\end{question}
}

\element{schaums-mc}{
\begin{question}{ch21-Q15}
    The coefficient of performance of a refrigerator that gives off \SI{3}{\joule} of heat for every joule of mechanical energy input is:
    \begin{multicols}{2}
    \begin{choices}
        \wrongchoice{\num{1}}
      \correctchoice{\num{2}}
        \wrongchoice{\num{3}}
        \wrongchoice{\num{4}}
    \end{choices}
    \end{multicols}
\end{question}
}


\endinput


