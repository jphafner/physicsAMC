
%%--------------------------------------------------
%% Schaum's Outline of Applied Physics
%%--------------------------------------------------


%% Chapter 33: Physical and Quantum Optics
%%--------------------------------------------------


%% Schaum's Multiple Choice Questions
%%--------------------------------------------------
\element{schaums-mc}{
\begin{questionmult}{ch33-Q01}
    Which one or more of the following produce coherent electromagnetic waves?
    \begin{choices}
      \correctchoice{two lasers of the same frequency}
      \correctchoice{two antennas fed by the same radio transmitter}
      \correctchoice{a pinhole in a cover over a monochromatic light source and its reflection in a mirror}
        \wrongchoice{two pinholes in a cover over a monochromatic light source}
    \end{choices}
\end{questionmult}
}

\element{schaums-mc}{
\begin{question}{ch33-Q02}
    The resolving power of a lens can be improved by increasing which one or more of the following?
    \begin{choices}
      \correctchoice{the diameter of the lens}
        \wrongchoice{the object distance}
        \wrongchoice{the wavelength of the light}
        \wrongchoice{the brightness of the light}
    \end{choices}
\end{question}
}

\element{schaums-mc}{
\begin{question}{ch33-Q03}
    A beam of transverse waves whose vibrations occur in all directions perpendicular to their direction of motion is:
    \begin{multicols}{2}
    \begin{choices}
        \wrongchoice{polarized}
      \correctchoice{unpolarized}
        \wrongchoice{resolved}
        \wrongchoice{diffracted}
    \end{choices}
    \end{multicols}
\end{question}
}

\element{schaums-mc}{
\begin{question}{ch33-Q04}
    Which one or more of the following cannot be polarized?
    \begin{multicols}{2}
    \begin{choices}
      \correctchoice{sound waves}
        \wrongchoice{white light}
        \wrongchoice{radio waves}
        \wrongchoice{X-rays}
    \end{choices}
    \end{multicols}
\end{question}
}

\element{schaums-mc}{
\begin{question}{ch33-Q05}
    Photons in a vacuum have the same:
    \begin{multicols}{2}
    \begin{choices}
       \correctchoice{velocity}
         \wrongchoice{energy}
         \wrongchoice{frequency}
         \wrongchoice{wavelength}
    \end{choices}
    \end{multicols}
\end{question}
}

\element{schaums-mc}{
\begin{question}{ch33-Q06}
    When the voltage applied to an X-ray tube is increased,
        the X-rays have a greater:
    \begin{multicols}{2}
    \begin{choices}
        \wrongchoice{number per second}
        \wrongchoice{velocity}
      \correctchoice{energy}
        \wrongchoice{wavelength}
    \end{choices}
    \end{multicols}
\end{question}
}

\element{schaums-mc}{
\begin{question}{ch33-Q07}
    The photons in red light whose wavelength is \SI{650}{\nano\meter} have an energy of:
    \begin{multicols}{2}
    \begin{choices}
         \wrongchoice{\SI{4.3e-40}{\joule}}
         \wrongchoice{\SI{1.3e-31}{\joule}}
         \wrongchoice{\SI{1.0e-27}{\joule}}
       \correctchoice{\SI{3.1e-19}{\joule}}
    \end{choices}
    \end{multicols}
\end{question}
}

\element{schaums-mc}{
\begin{question}{ch33-Q08}
    An X-ray tube produces a \SI{0.50}{\watt} beam of \SI{1.0e19}{\hertz} X-rays. 
    The tube emits:
    \begin{multicols}{2}
    \begin{choices}
      \correctchoice{\SI{7.5e13}{photons\per\second}}
        \wrongchoice{\SI{3.0e14}{photons\per\second}}
        \wrongchoice{\SI{2.2e15}{photons\per\second}}
        \wrongchoice{\SI{3.0e18}{photons\per\second}}
    \end{choices}
    \end{multicols}
\end{question}
}

\element{schaums-mc}{
\begin{question}{ch33-Q09}
    The kinetic energy of an electron whose velocity is \SI{1.5e7}{\meter\per\second} is:
    \begin{multicols}{2}
    \begin{choices}
        \wrongchoice{\SI{1.64e-35}{\eV}}
        \wrongchoice{\SI{1.56e-3}{\eV}}
      \correctchoice{\SI{0.64}{\kilo\eV}}
        \wrongchoice{\SI{1.64}{\kilo\eV}}
    \end{choices}
    \end{multicols}
\end{question}
}

\element{schaums-mc}{
\begin{question}{ch33-Q10}
    The velocity of an electron whose kinetic energy is \SI{1.5}{\kilo\eV} is:
    \begin{multicols}{2}
    \begin{choices}
        \wrongchoice{\SI{7.3e5}{\meter\per\second}}
        \wrongchoice{\SI{5.1e6}{\meter\per\second}}
      \correctchoice{\SI{2.3e7}{\meter\per\second}}
        \wrongchoice{\SI{7.3e12}{\meter\per\second}}
    \end{choices}
    \end{multicols}
\end{question}
}

\endinput


