
%%--------------------------------------------------
%% Schaum's Outline of Applied Physics
%%--------------------------------------------------


%% Chapter 22: Heat Transfer
%%--------------------------------------------------


%% Schaum's Multiple Choice Questions
%%--------------------------------------------------
\element{schaums-mc}{
\begin{question}{ch22-Q01}
    Heat transfer by conduction occurs:
    \begin{choices}
        \wrongchoice{only in liquids}
        \wrongchoice{only in solids}
        \wrongchoice{only in liquids and solids}
      \correctchoice{in solids, liquids, and gases}
    \end{choices}
\end{question}
}

\element{schaums-mc}{
\begin{question}{ch22-Q02}
    Heat transfer by convection occurs:
    \begin{choices}
        \wrongchoice{only in liquids}
        \wrongchoice{only in gases}
      \correctchoice{only in liquids and gases}
        \wrongchoice{in solids, liquids, and gases}
    \end{choices}
\end{question}
}

\element{schaums-mc}{
\begin{question}{ch22-Q03}
    Radiation occurs:
    \begin{choices}
        \wrongchoice{only from liquids}
        \wrongchoice{only from solids}
        \wrongchoice{only from liquids and solids}
      \correctchoice{from solids, liquids, and gases}
    \end{choices}
\end{question}
}

\element{schaums-mc}{
\begin{questionmult}{ch22-Q04}
    Heat flows through a slab of some material at a rate that depends on which one or more of the following?
    \begin{choices}
      \correctchoice{the thickness of the slab}
      \correctchoice{the area of the slab}
        \wrongchoice{the specific heat capacity of the material}
      \correctchoice{the temperature difference between the faces of the slab}
    \end{choices}
\end{questionmult}
}

\element{schaums-mc}{
\begin{question}{ch22-Q05}
    The temperature of an object that emits electromagnetic radiation must be:
    \begin{choices}
        \wrongchoice{higher than \SI{0}{\degreeCelsius}}
      \correctchoice{higher than \SI{0}{\kelvin}}
        \wrongchoice{higher than that of its surroundings}
        \wrongchoice{high enough for it to glow}
    \end{choices}
\end{question}
}

\element{schaums-mc}{
\begin{question}{ch22-Q06}
    A concrete wall \SI{6.0}{\meter} long, \SI{3.5}{\meter} high, and \SI{25}{\centi\meter} thick has a conductivity of \SI{0.80}{\watt\per\meter\per\degreeCelsius}. 
    When the wall has an outside temperature of \SI{5}{\degreeCelsius} and an inside temperature of \SI{20}{\degreeCelsius},
        heat flows through it at:
    \begin{multicols}{2}
    \begin{choices}
        \wrongchoice{\SI{10}{\watt}}
      \correctchoice{\SI{1.0}{\kilo\watt}}
        \wrongchoice{\SI{1.6}{\kilo\watt}}
        \wrongchoice{\SI{1.7}{\kilo\watt}}
    \end{choices}
    \end{multicols}
\end{question}
}

\element{schaums-mc}{
\begin{question}{ch22-Q07}
    Heat flows through a wooden board \SI{30}{\milli\meter} thick at \SI{0.0086}{\watt\per\centi\meter\squared} when one of its sides is \SI{20}{\degreeCelsius} warmer than the other. 
    The thermal conductivity of the wood is:
    \begin{multicols}{2}
    \begin{choices}
        \wrongchoice{\SI{0.013}{\watt\per\meter\per\degreeCelsius}}
      \correctchoice{\SI{0.13}{\watt\per\meter\per\degreeCelsius}}
        \wrongchoice{\SI{0.52}{\watt\per\meter\per\degreeCelsius}}
        \wrongchoice{\SI{0.78}{\watt\per\meter\per\degreeCelsius}}
    \end{choices}
    \end{multicols}
\end{question}
}

\element{schaums-mc}{
\begin{question}{ch22-Q08}
    %% NOTE: double check value and R value wording
    Glass wool has a thermal conductivity of \SI{0.30}{Btu\per\foot\squared\per\hour\per\degree\Fahrenheit\per\inch}.
    An $R$---3.0 layer of glass wool has a thickness of:
    \begin{multicols}{2}
    \begin{choices}
        \wrongchoice{\SI{1}{\inch}}
      \correctchoice{\SI{0.9}{\inch}}
        \wrongchoice{\SI{3}{\inch}}
        \wrongchoice{\SI{10}{\inch}}
    \end{choices}
    \end{multicols}
\end{question}
}

\element{schaums-mc}{
\begin{question}{ch22-Q09}
    The temperature of the earth's surface averages \SI{15}{\degreeCelsius}. 
    If the earth were a blackbody,
        it would radiate energy at a rate of:
    \begin{multicols}{2}
    \begin{choices}
        \wrongchoice{\SI{0.0029}{\watt\per\meter\squared}}
        \wrongchoice{\SI{251}{\watt\per\meter\squared}}
        \wrongchoice{\SI{288}{\watt\per\meter\squared}}
      \correctchoice{\SI{390}{\watt\per\meter\squared}}
    \end{choices}
    \end{multicols}
\end{question}
}

\element{schaums-mc}{
\begin{question}{ch22-Q10}
    The bare skin of a certain person is at an average temperature of \SI{33}{\degreeCelsius} in a room whose temperature is \SI{20}{\degreeCelsius}.
    If $e=1$, the net rate at which the person loses energy by radiation is:
    \begin{multicols}{2}
    \begin{choices}
        \wrongchoice{\SI{0.16}{\micro\watt\per\centi\meter\squared}}
      \correctchoice{\SI{7.9}{\milli\watt\per\centi\meter\squared}}
        \wrongchoice{\SI{50}{\milli\watt\per\centi\meter\squared}}
        \wrongchoice{\SI{79}{\watt\per\meter\squared}}
    \end{choices}
    \end{multicols}
\end{question}
}

\element{schaums-mc}{
\begin{question}{ch22-Q11}
    An iron bar at \SI{200}{\degreeCelsius} radiates energy at \SI{50}{\watt}.
    At \SI{250}{\degreeCelsius} the same bar will radiate at:
    \begin{multicols}{2}
    \begin{choices}
        \wrongchoice{\SI{55}{\watt}}
        \wrongchoice{\SI{63}{\watt}}
      \correctchoice{\SI{75}{\watt}}
        \wrongchoice{\SI{122}{\watt}}
    \end{choices}
    \end{multicols}
\end{question}
}


\endinput


