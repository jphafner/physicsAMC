
%%--------------------------------------------------
%% Schaum's Outline of Applied Physics
%%--------------------------------------------------


%% Chapter 32: Lenses
%%--------------------------------------------------


%% Schaum's Multiple Choice Questions
%%--------------------------------------------------
\element{schaums-mc}{
\begin{question}{ch32-Q01}
    The image of a real object farther from a converging lens than $f$ is always which one or more of the following?
    \begin{choices}
        \wrongchoice{smaller than the object}
        \wrongchoice{the same size as the object}
        \wrongchoice{virtual}
      \correctchoice{inverted}
    \end{choices}
\end{question}
}

\element{schaums-mc}{
\begin{question}{ch32-Q02}
    The image of a real object closer to a converging lens than $f$ is always which one or more of the following?
    \begin{choices}
        \wrongchoice{smaller than the object}
        \wrongchoice{the same size as the object}
      \correctchoice{virtual}
        \wrongchoice{inverted}
    \end{choices}
\end{question}
}

\element{schaums-mc}{
\begin{question}{ch32-Q03}
    The image of a real object the distance $f$ from a converging lens:
    \begin{choices}
      \correctchoice{does not exist}
        \wrongchoice{is virtual, erect, and larger than the object}
        \wrongchoice{is real, inverted, and larger than the object}
        \wrongchoice{is real, inverted, and the same size as the object}
    \end{choices}
\end{question}
}

\element{schaums-mc}{
\begin{question}{ch32-Q04}
    A real image formed by a lens is always which one or more of the following?
    \begin{choices}
        \wrongchoice{smaller than the object}
        \wrongchoice{larger than the object}
        \wrongchoice{erect}
      \correctchoice{inverted}
    \end{choices}
\end{question}
}

\element{schaums-mc}{
\begin{question}{ch32-Q05}
    The image formed by a diverging lens of a real object is never which one or more of the following?
    \begin{choices}
      \correctchoice{real}
        \wrongchoice{virtual}
        \wrongchoice{erect}
        \wrongchoice{smaller than the object}
    \end{choices}
\end{question}
}

\element{schaums-mc}{
\begin{question}{ch32-Q06}
    A negative magnification corresponds to an image that is:
    \begin{choices}
        \wrongchoice{erect}
      \correctchoice{inverted}
        \wrongchoice{larger than the object}
        \wrongchoice{smaller than the object}
    \end{choices}
\end{question}
}

\element{schaums-mc}{
\begin{question}{ch32-Q07}
    A negative image distance corresponds to an image that is:
    \begin{choices}
        \wrongchoice{erect}
        \wrongchoice{inverted}
        \wrongchoice{real}
      \correctchoice{virtual}
    \end{choices}
\end{question}
}

\element{schaums-mc}{
\begin{question}{ch32-Q08}
    The object distance of a converging lens of focal length $f$ used as a magnifying glass must be:
    \begin{choices}
      \correctchoice{less than $f$}
        \wrongchoice{$f$}
        \wrongchoice{between $f$ and $2f$}
        \wrongchoice{more than $2f$}
    \end{choices}
\end{question}
}

\element{schaums-mc}{
\begin{question}{ch32-Q09}
    The image distance of an object located \SI{12}{\centi\meter} from a converging lens of focal length \SI{16}{\centi\meter} is:
    \begin{multicols}{2}
    \begin{choices}
        \wrongchoice{\SI[retain-explicit-plus]{-4}{\centi\meter}}
      \correctchoice{\SI[retain-explicit-plus]{-48}{\centi\meter}}
        \wrongchoice{\SI[retain-explicit-plus]{+4}{\centi\meter}}
        \wrongchoice{\SI[retain-explicit-plus]{+48}{\centi\meter}}
    \end{choices}
    \end{multicols}
\end{question}
}

\element{schaums-mc}{
\begin{question}{ch32-Q10}
    The image distance of an object located \SI{16}{\centi\meter} from a converging lens of focal length \SI{12}{\centi\meter} is:
    \begin{multicols}{2}
    \begin{choices}
        \wrongchoice{\SI[retain-explicit-plus]{-4}{\centi\meter}}
        \wrongchoice{\SI[retain-explicit-plus]{-48}{\centi\meter}}
        \wrongchoice{\SI[retain-explicit-plus]{+4}{\centi\meter}}
      \correctchoice{\SI[retain-explicit-plus]{+48}{\centi\meter}}
    \end{choices}
    \end{multicols}
\end{question}
}

\element{schaums-mc}{
\begin{question}{ch32-Q11}
    If the image of an object \SI{6}{\centi\meter} from a lens is \SI{6}{\centi\meter} behind the object,
        the lens has a focal length of:
    \begin{multicols}{2}
    \begin{choices}
        \wrongchoice{\SI[retain-explicit-plus]{-12}{\centi\meter}}
        \wrongchoice{\SI[retain-explicit-plus]{+3}{\centi\meter}}
        \wrongchoice{\SI[retain-explicit-plus]{+4}{\centi\meter}}
      \correctchoice{\SI[retain-explicit-plus]{+12}{\centi\meter}}
    \end{choices}
    \end{multicols}
\end{question}
}

\element{schaums-mc}{
\begin{question}{ch32-Q12}
    A candle \SI{6}{\centi\meter} high is \SI{80}{\centi\meter} in front of a lens whose focal length is \SI[retain-explicit-plus]{+60}{\centi\meter}.
    The image is:
    \begin{choices}
        \wrongchoice{\SI{2}{\centi\meter} high, erect}
        \wrongchoice{\SI{2}{\centi\meter} high, inverted}
        \wrongchoice{\SI{18}{\centi\meter} high, erect}
      \correctchoice{\SI{18}{\centi\meter} high, inverted}
    \end{choices}
\end{question}
}

\element{schaums-mc}{
\begin{question}{ch32-Q13}
    A candle \SI{6}{\centi\meter} high is \SI{120}{\centi\meter} in front of a lens whose focal length is \SI[retain-explicit-plus]{+60}{\centi\meter}.
    The image is:
    \begin{choices}
        \wrongchoice{\SI{3}{\centi\meter} high, erect}
        \wrongchoice{\SI{3}{\centi\meter} high, inverted}
        \wrongchoice{\SI{6}{\centi\meter} high, erect}
      \correctchoice{\SI{6}{\centi\meter} high, inverted}
    \end{choices}
\end{question}
}

\element{schaums-mc}{
\begin{question}{ch32-Q14}
    A candle \SI{6}{\centi\meter} high is \SI{150}{\centi\meter} in front of a lens whose focal length is \SI[retain-explicit-plus]{+60}{\centi\meter}.
    The image is:
    \begin{choices}
        \wrongchoice{\SI{4}{\centi\meter} long, erect}
      \correctchoice{\SI{4}{\centi\meter} long, inverted}
        \wrongchoice{\SI{9}{\centi\meter} long, erect}
        \wrongchoice{\SI{9}{\centi\meter} long, inverted}
    \end{choices}
\end{question}
}

\element{schaums-mc}{
\begin{question}{ch32-Q15}
    The focal length of a magnifying glass that produces an image six times larger than an object \SI{10}{\milli\meter} away is:
    \begin{multicols}{2}
    \begin{choices}
        \wrongchoice{\SI[retain-explicit-plus]{+1.4}{\milli\meter}}
        \wrongchoice{\SI[retain-explicit-plus]{+2.0}{\milli\meter}}
        \wrongchoice{\SI[retain-explicit-plus]{+8.6}{\milli\meter}}
      \correctchoice{\SI[retain-explicit-plus]{+12}{\milli\meter}}
    \end{choices}
    \end{multicols}
\end{question}
}

\element{schaums-mc}{
\begin{question}{ch32-Q16}
    A projector whose lens has a focal length of \SI[retain-explicit-plus]{+12}{\centi\meter} forms an image \SI{90}{\centi\meter} high of a slide whose picture area is \SI{30}{\milli\meter} high. 
    How far is the lens from the screen?
    \begin{multicols}{2}
    \begin{choices}
       \correctchoice{\SI{348}{\centi\meter}}
         \wrongchoice{\SI{360}{\centi\meter}}
         \wrongchoice{\SI{372}{\centi\meter}}
         \wrongchoice{\SI{384}{\centi\meter}}
    \end{choices}
    \end{multicols}
\end{question}
}


\endinput


