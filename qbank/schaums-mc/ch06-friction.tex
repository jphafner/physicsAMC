
%%--------------------------------------------------
%% Schaum's Outline of Applied Physics
%%--------------------------------------------------


%% Chapter 6: Friction
%%--------------------------------------------------


%% Schaum's Multiple Choice Questions
%%--------------------------------------------------
\element{schaums-mc}{
\begin{question}{ch06-Q01}
    Relative to the force needed to keep a box moving at constant velocity across a floor,
        to start the box moving usually needs:
    \begin{choices}
        \wrongchoice{less force}
        \wrongchoice{the same force}
      \correctchoice{more force}
        \wrongchoice{any of the above, depending on the natures of the surfaces in contact}
    \end{choices}
\end{question}
}

\element{schaums-mc}{
\begin{questionmult}{ch06-Q02}
    When two surfaces are in contact,
        the frictional force between them depends on which one or more of the following?
    \begin{choices}
      \correctchoice{the normal force pressing one surface against the other}
        \wrongchoice{the areas of the surfaces}
      \correctchoice{whether the surfaces are stationary or in relative motion}
      \correctchoice{whether a lubricant is used or not}
    \end{choices}
\end{questionmult}
}

\element{schaums-mc}{
\begin{question}{ch06-Q03}
    A brick has the dimensions \SI{8}{\centi\meter} $\times$ \SI{16}{\centi\meter} $\times$ \SI{32}{\centi\meter}.
    The force of starting friction between the brick and a wooden floor is:
    \begin{choices}
        \wrongchoice{a maximum when the brick rests on the \SI{8}{\centi\meter} $\times$ \SI{16}{\centi\meter} face}
        \wrongchoice{a maximum when the brick rest on the \SI{8}{\centi\meter} $\times$ \SI{32}{\centi\meter} face}
        \wrongchoice{a maximum when the brick rests on the \SI{16}{\centi\meter} $\times$ \SI{32}{\centi\meter} face}
      \correctchoice{the same regardless of which face it rests on}
    \end{choices}
\end{question}
}

\element{schaums-mc}{
\begin{question}{ch06-Q04}
    The coefficient of kinetic friction between two oiled steel surfaces is:
    \begin{multicols}{2}
    \begin{choices}
      \correctchoice{\num{0.03}}
        \wrongchoice{\SI{0.03}{\newton}}
        \wrongchoice{\SI{0.03}{\newton\per\kilo\gram}}
        \wrongchoice{\SI{0.03}{\kilo\gram\per\newton}}
    \end{choices}
    \end{multicols}
\end{question}
}

\element{schaums-mc}{
\begin{question}{ch06-Q05}
    A \SI{60}{\newton} force is needed to start a \SI{60}{\kilo\gram} skater moving across a frozen lake. The coefficient of static friction for steel on ice is approximately:
    \begin{multicols}{2}
    \begin{choices}
        \wrongchoice{\num{0.06}}
      \correctchoice{\num{0.1}}
        \wrongchoice{\num{0.6}}
        \wrongchoice{\num{1}}
    \end{choices}
    \end{multicols}
\end{question}
}

\element{schaums-mc}{
\begin{question}{ch06-Q06}
    The coefficient of static friction for wood on concrete is \num{0.6}. 
    The force needed to set a \SI{40}{\kilo\gram} crate in motion on a concrete floor is:
    \begin{multicols}{2}
    \begin{choices}
        \wrongchoice{\SI{2.4}{\newton}}
        \wrongchoice{\SI{24}{\newton}}
      \correctchoice{\SI{235}{\newton}}
        \wrongchoice{\SI{653}{\newton}}
    \end{choices}
    \end{multicols}
\end{question}
}

\element{schaums-mc}{
\begin{question}{ch06-Q07}
    A \SI{153}{\kilo\gram} engine on wooden skids is resting on a level floor. 
    The coefficients of static and kinetic friction are,
        respectively, \num{0.5} and \num{0.4}.
    When two men push on the engine with a total horizontal force of \SI{500}{\newton},
        the frictional force that acts on the skids is:
    \begin{multicols}{2}
    \begin{choices}
      \correctchoice{\SI{500}{\newton}}
        \wrongchoice{\SI{600}{\newton}}
        \wrongchoice{\SI{750}{\newton}}
        \wrongchoice{\SI{1500}{\newton}}
    \end{choices}
    \end{multicols}
\end{question}
}

\element{schaums-mc}{
\begin{question}{ch06-Q08}
    The coefficient of static friction between a car's tires and a level road is \num{0.80}. 
    If the car is to be stopped in a maximum time of \SI{3.0}{\second},
        its speed cannot exceed:
    \begin{multicols}{2}
    \begin{choices}
        \wrongchoice{\SI{2.4}{\meter\per\second}}
        \wrongchoice{\SI{2.6}{\meter\per\second}}
        \wrongchoice{\SI{7.8}{\meter\per\second}}
      \correctchoice{\SI{23.5}{\meter\per\second}}
    \end{choices}
    \end{multicols}
\end{question}
}

\element{schaums-mc}{
\begin{question}{ch06-Q09}
    A force applied to a \SI{50}{\kilo\gram} box on a level floor is just enough to start it moving. 
    The coefficients of static and kinetic friction are,
        respectively, \num{0.5} and \num{0.3}. 
    If the same force continues to be applied to the box,
        it will have an acceleration of approximately:
    \begin{multicols}{2}
    \begin{choices}
      \correctchoice{\SI{2}{\meter\per\second\squared}}
        \wrongchoice{\SI{3}{\meter\per\second\squared}}
        \wrongchoice{\SI{4}{\meter\per\second\squared}}
        \wrongchoice{\SI{5}{\meter\per\second\squared}}
    \end{choices}
    \end{multicols}
\end{question}
}

\element{schaums-mc}{
\begin{question}{ch06-Q10}
    After descending a slope,
        a skier coasts on level snow for \SI{20}{\meter} before coming to a stop. 
    If the coefficient of friction between skis and snow is \num{0.05},
        the skier's speed at the foot of the slope was:
    \begin{multicols}{2}
    \begin{choices}
        \wrongchoice{\SI{3.1}{\meter\per\second}}
      \correctchoice{\SI{4.4}{\meter\per\second}}
        \wrongchoice{\SI{6.3}{\meter\per\second}}
        \wrongchoice{\SI{19.6}{\meter\per\second}}
    \end{choices}
    \end{multicols}
\end{question}
}


\endinput


