
%%--------------------------------------------------
%% Schaum's Outline of Applied Physics
%%--------------------------------------------------


%% Chapter 20: Kinetic Theory of Matter
%%--------------------------------------------------


%% Schaum's Multiple Choice Questions
%%--------------------------------------------------
\element{schaums-mc}{
\begin{question}{ch20-Q01}
    Molecular motion in a gas stops:
    \begin{choices}
      \correctchoice{at absolute zero}
        \wrongchoice{when the gas becomes a liquid}
        \wrongchoice{when the gas becomes a solid}
        \wrongchoice{when the pressure on it exceeds a certain value}
    \end{choices}
\end{question}
}

\element{schaums-mc}{
\begin{question}{ch20-Q02}
    At a given temperature:
    \begin{choices}
        \wrongchoice{the molecules in a gas all have the same average velocity}
      \correctchoice{the molecules in a gas all have the same average energy}
        \wrongchoice{light gas molecules have lower average energies than heavy gas molecules}
        \wrongchoice{heavy gas molecules have lower average energies than light gas molecules}
    \end{choices}
\end{question}
}

\element{schaums-mc}{
\begin{question}{ch20-Q03}
    The temperature of a gas sample in a container of fixed volume is raised. 
    The gas exerts a higher pressure on the walls of its container because its molecules
    \begin{choices}
        \wrongchoice{lose more potential energy when they strike the walls.}
        \wrongchoice{lose more kinetic energy when they strike the walls}
        \wrongchoice{are in contact with the walls for a shorter time}
      \correctchoice{have higher average velocities and strike the walls more often}
    \end{choices}
\end{question}
}

\element{schaums-mc}{
\begin{question}{ch20-Q04}
    The volume of a gas sample is increased while its temperature is held constant. 
    The gas exerts a lower pressure on the walls of its container partly because its molecules strike the walls:
    \begin{choices}
      \correctchoice{less often}
        \wrongchoice{with lower velocities}
        \wrongchoice{with less energy}
        \wrongchoice{with less force}
    \end{choices}
\end{question}
}

\element{schaums-mc}{
\begin{question}{ch20-Q05}
    When evaporation occurs,
        the liquid that remains is cooler because:
    \begin{choices}
        \wrongchoice{the pressure on the liquid decreases}
        \wrongchoice{the volume of the liquid decreases}
      \correctchoice{the slowest molecules remain behind}
        \wrongchoice{the fastest molecules remain behind}
    \end{choices}
\end{question}
}

\element{schaums-mc}{
\begin{question}{ch20-Q06}
    When a volume of air is heated,
    \begin{choices}
        \wrongchoice{it can hold less water vapor}
      \correctchoice{it can hold more water vapor}
        \wrongchoice{the amount of water vapor it can hold does not change}
        \wrongchoice{its relative humidity increases}
    \end{choices}
\end{question}
}

\element{schaums-mc}{
\begin{question}{ch20-Q07}
    Cooling saturated air causes:
    \begin{choices}
        \wrongchoice{its relative humidity to decrease}
        \wrongchoice{its relative humidity to increase}
        \wrongchoice{its ability to take up water vapor to increase}
      \correctchoice{some of its water content to condense out}
    \end{choices}
\end{question}
}

\element{schaums-mc}{
\begin{questionmult}{ch20-Q08}
    Which one or more of the following quantities are the same for both a mole of oxygen molecules and a mole of nitrogen molecules at the same temperature and pressure?
    \begin{choices}
      \correctchoice{the number of molecules present}
        \wrongchoice{the average velocities of the molecules}
      \correctchoice{the volume of the gas}
        \wrongchoice{the density of the gas}
    \end{choices}
\end{questionmult}
}

\element{schaums-mc}{
\begin{question}{ch20-Q09}
    How many moles of \ce{H} atoms are present in \SI{1}{\mole} of water, \ce{H2O}?
    \begin{multicols}{2}
    \begin{choices}
        \wrongchoice{\num{2/3}}
        \wrongchoice{\num{1}}
      \correctchoice{\num{2}}
        \wrongchoice{\num{3}}
    \end{choices}
    \end{multicols}
\end{question}
}

\element{schaums-mc}{
\begin{question}{ch20-Q10}
    The mass of a nitrogen atom is \SI{14}{u}. 
    The number of moles of molecular nitrogen (\ce{N2}) in \SI{56}{\gram} of nitrogen is:
    \begin{multicols}{2}
    \begin{choices}
      \correctchoice{\num{2}}
        \wrongchoice{\num{4}}
        \wrongchoice{\num{28}}
        \wrongchoice{\num{64}}
    \end{choices}
    \end{multicols}
\end{question}
}

\element{schaums-mc}{
\begin{question}{ch20-Q11}
    A gas sample at \SI{200}{\kelvin} is heated until its temperature is \SI{400}{\kelvin}. 
    If the original average velocity of the gas molecules was $v$,
        their new average velocity is:
    \begin{multicols}{2}
    \begin{choices}
        \wrongchoice{$v$}
      \correctchoice{$\sqrt{2}v$}
        \wrongchoice{$2v$}
        \wrongchoice{$4v$}
    \end{choices}
    \end{multicols}
\end{question}
}

\element{schaums-mc}{
\begin{question}{ch20-Q12}
    The molecules of a gas at \SI{10}{\degreeCelsius} would have twice as much average kinetic energy at:
    \begin{multicols}{2}
    \begin{choices}
        \wrongchoice{\SI{20}{\degreeCelsius}}
        \wrongchoice{\SI{293}{\degreeCelsius}}
        \wrongchoice{\SI{566}{\degreeCelsius}}
        \wrongchoice{\SI{859}{\degreeCelsius}}
    \end{choices}
    \end{multicols}
\end{question}
}

\element{schaums-mc}{
\begin{question}{ch20-Q13}
    An oxygen molecule has 16 times the mass of a hydrogen molecule. 
    A sample of hydrogen gas whose molecules have the same average kinetic energy as the molecules in a sample of oxygen at \SI{400}{\kelvin} is at a temperature of:
    \begin{multicols}{2}
    \begin{choices}
        \wrongchoice{\SI{25}{\kelvin}}
      \correctchoice{\SI{400}{\kelvin}}
        \wrongchoice{\SI{1600}{\kelvin}}
        \wrongchoice{\SI{6400}{\kelvin}}
    \end{choices}
    \end{multicols}
\end{question}
}


\endinput


