
%%--------------------------------------------------
%% Schaum's Outline of Applied Physics
%%--------------------------------------------------


%% Chapter 25: Direct-Current Circuits
%%--------------------------------------------------


%% Schaum's Multiple Choice Questions
%%--------------------------------------------------
\element{schaums-mc}{
\begin{question}{ch25-Q01}
    A resistor $R$ connected to a battery dissipates energy at the rate $P$.
    If another resistor is connected in parallel with $R$,
        the power dissipated by $R$ is:
    \begin{choices}
        \wrongchoice{less than $P$}
      \correctchoice{$P$}
        \wrongchoice{more than $P$}
        \wrongchoice{it depends on the values of the resistances}
    \end{choices}
\end{question}
}

\element{schaums-mc}{
\begin{question}{ch25-Q02}
    A resistor $R$ connected to a battery dissipates energy at the rate $P$.
    If another resistor is connected in parallel with $R$,
        the power dissipated by $R$ is:
    %% NOTE: reword
    The second resistor of Question 25.1 is connected in series with $R$. 
    The power dissipated by $R$ is now:
    \begin{choices}
      \correctchoice{less than $P$}
        \wrongchoice{$P$}
        \wrongchoice{more than $P$}
        \wrongchoice{it depends on the values of the resistances}
    \end{choices}
\end{question}
}

\element{schaums-mc}{
\begin{question}{ch25-Q03}
    A network is being analyzed using Kirchhoff's rules. 
    If the wrong direction is assumed for one of the currents $I$,
        the calculated current will be:
    \begin{multicols}{2}
    \begin{choices}
        \wrongchoice{zero}
      \correctchoice{$-I$}
        \wrongchoice{$I$}
        \wrongchoice{incorrect}
    \end{choices}
    \end{multicols}
\end{question}
}

\element{schaums-mc}{
\begin{question}{ch25-Q04}
    A \SI{60}{\volt} potential difference is applied across a \SI{5}{\ohm} and a \SI{10}{\ohm} resistor in series.
    The current in the \SI{5}{\ohm}resistor is:
    \begin{multicols}{2}
    \begin{choices}
      \correctchoice{\SI{4}{\ampere}}
        \wrongchoice{\SI{6}{\ampere}}
        \wrongchoice{\SI{12}{\ampere}}
        \wrongchoice{\SI{18}{\ampere}}
    \end{choices}
    \end{multicols}
\end{question}
}

\element{schaums-mc}{
\begin{question}{ch25-Q05}
    The potential difference across the \SI{5}{\ohm}resistor in Question 25.4 is:
    \begin{multicols}{2}
    \begin{choices}
      \correctchoice{\SI{20}{\volt}}
        \wrongchoice{\SI{30}{\volt}}
        \wrongchoice{\SI{40}{\volt}}
        \wrongchoice{\SI{60}{\volt}}
    \end{choices}
    \end{multicols}
\end{question}
}

\element{schaums-mc}{
\begin{question}{ch25-Q06}
    A \SI{5}{\ohm} resistor and a \SI{10}{\ohm}resistor are connected in parallel. 
    Their equivalent resistance is:
    \begin{multicols}{2}
    \begin{choices}
        \wrongchoice{\SI{0.3}{\ohm}}
      \correctchoice{\SI{3.3}{\ohm}}
        \wrongchoice{\SI{7.5}{\ohm}}
        \wrongchoice{\SI{15}{\ohm}}
    \end{choices}
    \end{multicols}
\end{question}
}

\element{schaums-mc}{
\begin{question}{ch25-Q07}
    A \SI{60}{\volt} potential difference is applied across the resistors of Question 25.6. 
    The current in the \SI{5}{\ohm} resistor is:
    \begin{multicols}{2}
    \begin{choices}
        \wrongchoice{\SI{4}{\ampere}}
        \wrongchoice{\SI{6}{\ampere}}
      \correctchoice{\SI{12}{\ampere}}
        \wrongchoice{\SI{18}{\ampere}}
    \end{choices}
    \end{multicols}
\end{question}
}

\element{schaums-mc}{
\begin{question}{ch25-Q08}
    The equivalent resistance of two identical resistors in parallel is \SI{10}{\ohm}.
    The equivalent resistance of the same resistors in series would be:
    \begin{multicols}{2}
    \begin{choices}
        \wrongchoice{\SI{10}{\ohm}}
        \wrongchoice{\SI{20}{\ohm}}
      \correctchoice{\SI{40}{\ohm}}
        \wrongchoice{\SI{100}{\ohm}}
    \end{choices}
    \end{multicols}
\end{question}
}

\element{schaums-mc}{
\begin{question}{ch25-Q09}
    A network of three \SI{5}{\ohm}resistors cannot have an equivalent resistance of:
    \begin{multicols}{2}
    \begin{choices}
        \wrongchoice{\SI{1.67}{\ohm}}
      \correctchoice{\SI{2.5}{\ohm}}
        \wrongchoice{\SI{7.5}{\ohm}}
        \wrongchoice{\SI{15}{\ohm}}
    \end{choices}
    \end{multicols}
\end{question}
}

\element{schaums-mc}{
\begin{question}{ch25-Q10}
    A battery of emf \SI{12}{\volt} whose internal resistance is negligible is connected across a \SI{10}{\ohm} resistor in parallel with a resistor of resistance $R$. 
    If the battery current is \SI{2}{\ampere},
        the value of $R$ is:
    \begin{multicols}{2}
    \begin{choices}
        \wrongchoice{\SI{3.75}{\ohm}}
        \wrongchoice{\SI{6}{\ohm}}
      \correctchoice{\SI{15}{\ohm}}
        \wrongchoice{\SI{20}{\ohm}}
    \end{choices}
    \end{multicols}
\end{question}
}

\element{schaums-mc}{
\begin{question}{ch25-Q11}
    A current of \SI{9}{\ampere} flows when a \SI{120}{\volt} battery is connected across a \SI{12}{\ohm}resistor. 
    The battery has an internal resistance of:
    \begin{multicols}{2}
    \begin{choices}
      \correctchoice{\SI{1.3}{\ohm}}
        \wrongchoice{\SI{12}{\ohm}}
        \wrongchoice{\SI{13.3}{\ohm}}
        \wrongchoice{\SI{25.3}{\ohm}}
    \end{choices}
    \end{multicols}
\end{question}
}

\element{schaums-mc}{
\begin{question}{ch25-Q12}
    A battery of emf \SI{12}{\volt} and internal resistance \SI{1}{\ohm} is connected across a \SI{3}{\ohm} resistor.
        The potential difference across the resistor is:
    \begin{multicols}{2}
    \begin{choices}
        \wrongchoice{\SI{2.25}{\volt}}
        \wrongchoice{\SI{3}{\volt}}
      \correctchoice{\SI{9}{\volt}}
        \wrongchoice{\SI{12}{\volt}}
    \end{choices}
    \end{multicols}
\end{question}
}

\element{schaums-mc}{
\begin{question}{ch25-Q13}
    A \SI{5}{\ohm} resistor and a \SI{10}{\ohm} resistor are connected in series to a battery. 
    If heat is produced in the \SI{5}{\ohm}resistor at the rate $P$,
        the rate at which heat is produced in the \SI{10}{\ohm} resistor is:
    \begin{multicols}{2}
    \begin{choices}
        \wrongchoice{$P/4$}
        \wrongchoice{$P/2$}
        \wrongchoice{$P$}
      \correctchoice{$2P$}
    \end{choices}
    \end{multicols}
\end{question}
}

\element{schaums-mc}{
\begin{question}{ch25-Q14}
    A \SI{5}{\ohm} resistor and a \SI{10}{\ohm} resistor are connected in parallel to a battery. 
    If heat is produced in the \SI{5}{\ohm}resistor at the rate $P$,
        the rate at which heat is produced in the \SI{10}{\ohm} resistor is:
    \begin{multicols}{2}
    \begin{choices}
        \wrongchoice{$P/4$}
      \correctchoice{$P/2$}
        \wrongchoice{$P$}
        \wrongchoice{$2P$}
    \end{choices}
    \end{multicols}
\end{question}
}


\endinput


