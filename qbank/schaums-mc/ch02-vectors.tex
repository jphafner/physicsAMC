
%%--------------------------------------------------
%% Schaum's Outline of Applied Physics
%%--------------------------------------------------


%% Chapter 2: Vectors
%%--------------------------------------------------


%% Schaum's Multiple Choice Questions
%%--------------------------------------------------
\element{schaums-mc}{
\begin{question}{ch02-mc-Q01}
    A box suspended by a rope is pulled to one side by a horizontal force. 
    The tension in the rope:
    \begin{choices}
        \wrongchoice{is less than before}
        \wrongchoice{is unchanged}
      \correctchoice{is greater than before}
        \wrongchoice{may be any of the above, depending on how strong the force is}
    \end{choices}
\end{question}
}

\element{schaums-mc}{
\begin{question}{ch02-mc-Q02}
    The minimum number of unequal forces whose vector sum can equal zero is:
    \begin{multicols}{4}
    \begin{choices}
        \wrongchoice{\num{2}}
      \correctchoice{\num{3}}
        \wrongchoice{\num{4}}
        \wrongchoice{\num{6}}
    \end{choices}
    \end{multicols}
\end{question}
}

\element{schaums-mc}{
\begin{question}{ch02-mc-Q03}
    When two vectors are added together,
        their resultant is a minimum when the angle between them is:
    \begin{multicols}{4}
    \begin{choices}
        \wrongchoice{\ang{0}}
        \wrongchoice{\ang{45}}
        \wrongchoice{\ang{90}}
      \correctchoice{\ang{180}}
    \end{choices}
    \end{multicols}
\end{question}
}

\element{schaums-mc}{
\begin{question}{ch02-mc-Q04}
    Of the following pairs of forces,
        which one or more cannot be added to give a resultant force of \SI{2}{\newton}?
    \begin{multicols}{2}
    \begin{choices}
        \wrongchoice{\SI{1}{\newton} and \SI{1}{\newton}}
        \wrongchoice{\SI{1}{\newton} and \SI{2}{\newton}}
        \wrongchoice{\SI{1}{\newton} and \SI{3}{\newton}}
      \correctchoice{\SI{1}{\newton} and \SI{4}{\newton}}
    \end{choices}
    \end{multicols}
\end{question}
}

\element{schaums-mc}{
\begin{questionmult}{ch02-mc-Q05}
    Of the following sets of displacements,
        which one or more might be able to return a car to its starting point?
    \begin{choices}
        %% NOTE: change formatting
        \wrongchoice{3, 4, 12, and 20 km}
      \correctchoice{5, 10, 15, and 20 km}
        \wrongchoice{20, 60, 80, and 180 km}
      \correctchoice{100, 100, 100, and 100 km}
    \end{choices}
\end{questionmult}
}

\element{schaums-mc}{
\begin{question}{ch02-mc-Q06}
    A boat whose velocity through the water is \SI{14}{\kilo\meter\per\hour} is moving in a river whose current is \SI{6}{\kilo\meter\per\hour} relative to the riverbed. 
    The velocity of the boat relative to the riverbed must be between:
    \begin{multicols}{2}
    \begin{choices}
        \wrongchoice{\SIrange{6}{14}{\kilo\meter\per\hour}}
        \wrongchoice{\SIrange{6}{20}{\kilo\meter\per\hour}}
        \wrongchoice{\SIrange{8}{14}{\kilo\meter\per\hour}}
      \correctchoice{\SIrange{8}{20}{\kilo\meter\per\hour}}
    \end{choices}
    \end{multicols}
\end{question}
}

\element{schaums-mc}{
\begin{question}{ch02-mc-Q07}
    A ship travels \SI{200}{\kilo\meter} to the south and then \SI{400}{\kilo\meter} to the west. 
    The ship’s displacement from its starting point is:
    \begin{multicols}{2}
    \begin{choices}
        \wrongchoice{\SI{200}{\kilo\meter}}
        \wrongchoice{\SI{400}{\kilo\meter}}
      \correctchoice{\SI{450}{\kilo\meter}}
        \wrongchoice{\SI{600}{\kilo\meter}}
    \end{choices}
    \end{multicols}
\end{question}
}

\element{schaums-mc}{
\begin{question}{ch02-mc-Q08}
    A ship travels \SI{200}{\kilo\meter} to the south and then \SI{400}{\kilo\meter} to the west. 
    %% NOTE: changed wording
    At what angle west of south should the ship have headed to arrive at the same place in a straight path?
    \begin{multicols}{4}
    \begin{choices}
        \wrongchoice{\ang{22}}
        \wrongchoice{\ang{45}}
        \wrongchoice{\ang{50}}
      \correctchoice{\ang{63}}
    \end{choices}
    \end{multicols}
\end{question}
}

\element{schaums-mc}{
\begin{question}{ch02-mc-Q09}
    A conveyor belt has a velocity of \SI{4.00}{\meter\per\second} at an angle of \ang{40} above the horizontal.
    The vertical component of its velocity is
    \begin{multicols}{2}
    \begin{choices}
        \wrongchoice{\SI{2.00}{\meter\per\second}}
      \correctchoice{\SI{2.57}{\meter\per\second}}
        \wrongchoice{\SI{3.06}{\meter\per\second}}
        \wrongchoice{\SI{3.36}{\meter\per\second}}
    \end{choices}
    \end{multicols}
\end{question}
}

\element{schaums-mc}{
\begin{question}{ch02-mc-Q10}
    An object is acted on by two forces of \SI{20}{\newton} each. 
    The angle between the forces is \ang{120}. 
    The resultant force on the object has the magnitude:
    \begin{multicols}{2}
    \begin{choices}
      \correctchoice{\SI{20}{\newton}}
        \wrongchoice{\SI{28}{\newton}}
        \wrongchoice{\SI{34}{\newton}}
        \wrongchoice{\SI{40}{\newton}}
    \end{choices}
    \end{multicols}
\end{question}
}

\element{schaums-mc}{
\begin{question}{ch02-mc-Q11}
    A force $\mathbf{F}$ has the components $F_x$ and $F_y$.
    The magnitude $F_x$ of the force component in the $x$ direction is given by:
    \begin{multicols}{2}
    \begin{choices}
        \wrongchoice{$F - F_y$}
        \wrongchoice{$\sqrt{F} - \sqrt{F_y}$}
        \wrongchoice{$\sqrt{F − F_y}$}
      \correctchoice{$\sqrt{F^2 − F_y^2}$}
    \end{choices}
    \end{multicols}
\end{question}
}

\element{schaums-mc}{
\begin{question}{ch02-mc-Q12}
    Forces of \SI{20}{\newton} to the south,
        \SI{40}{\newton} to the northeast,
        and \SI{10}{\newton} to the east act on an object. 
    The magnitude of the resultant force on the object is:
    \begin{multicols}{2}
    \begin{choices}
        \wrongchoice{\SI{10}{\newton}}
        \wrongchoice{\SI{20}{\newton}}
      \correctchoice{\SI{39}{\newton}}
        \wrongchoice{\SI{46}{\newton}}
    \end{choices}
    \end{multicols}
\end{question}
}

\element{schaums-mc}{
\begin{question}{ch02-mc-Q13}
    Forces of \SI{20}{\newton} to the south,
        \SI{40}{\newton} to the northeast,
        and \SI{10}{\newton} to the east act on an object. 
    The resultant for the forces is:
    \begin{multicols}{2}
    \begin{choices}
        \wrongchoice{\ang{24} east of north}
        \wrongchoice{\ang{45} east of north}
        \wrongchoice{\ang{52} east of north}
      \correctchoice{\ang{78} east of north}
    \end{choices}
    \end{multicols}
\end{question}
}

\endinput


