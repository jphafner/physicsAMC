
%%--------------------------------------------------
%% Schaum's Outline of Applied Physics
%%--------------------------------------------------


%% Chapter 8: Momentum
%%--------------------------------------------------


%% Schaum's Multiple Choice Questions
%%--------------------------------------------------
\element{schaums-mc}{
\begin{question}{ch08-Q01}
    A \SI{300}{\gram} iron ball has the same diameter as a \SI{105}{\gram} aluminum ball. 
    The balls are dropped at the same time from a cliff. 
    Just before they reach the ground, they have the same:
    \begin{choices}
        \wrongchoice{acceleration}
        \wrongchoice{momentum}
        \wrongchoice{kinetic energy}
        \wrongchoice{potential energy}
    \end{choices}
\end{question}
}

\element{schaums-mc}{
\begin{question}{ch08-Q02}
    A ball with the momentum $\mathbf{p}$ strikes a wall and bounces off. 
    The change in the ball's momentum is ideally:
    \begin{multicols}{2}
    \begin{choices}
        \wrongchoice{$0$}
        \wrongchoice{$\mathbf{p}$}
        \wrongchoice{$2\mathbf{p}$}
        \wrongchoice{$-2\mathbf{p}$}
    \end{choices}
    \end{multicols}
\end{question}
}

\element{schaums-mc}{
\begin{question}{ch08-Q03}
    An elastic collision conserves:
    \begin{choices}
        \wrongchoice{momentum but not kinetic energy}
        \wrongchoice{kinetic energy but not momentum}
        \wrongchoice{both momentum and kinetic energy}
        \wrongchoice{neither momentum nor kinetic energy}
    \end{choices}
\end{question}
}

\element{schaums-mc}{
\begin{question}{ch08-Q04}
    An inelastic collision conserves:
    \begin{choices}
        \wrongchoice{momentum but not kinetic energy}
        \wrongchoice{kinetic energy but not momentum}
        \wrongchoice{both momentum and kinetic energy}
        \wrongchoice{neither momentum nor kinetic energy}
    \end{choices}
\end{question}
}

\element{schaums-mc}{
\begin{question}{ch08-Q05}
    A \SI{60}{\kilo\gram} skater pushes a \SI{50}{\kilo\gram} skater,
        who moves away at \SI{2.0}{\meter\per\second}. 
    As a result, the first skater moves backward at:
    \begin{multicols}{2}
    \begin{choices}
        \wrongchoice{\SI{0.6}{\meter\per\second}}
        \wrongchoice{\SI{1.7}{\meter\per\second}}
        \wrongchoice{\SI{2.0}{\meter\per\second}}
        \wrongchoice{\SI{2.4}{\meter\per\second}}
    \end{choices}
    \end{multicols}
\end{question}
}

\element{schaums-mc}{
\begin{question}{ch08-Q06}
    A \SI{60}{\gram} tennis ball moving at \SI{8.0}{\meter\per\second} strikes a stationary tennis racket perpendicularly and bounces off at \SI{6.0}{\meter\per\second}.
    The impulse given to the racket is:
    \begin{multicols}{2}
    \begin{choices}
        \wrongchoice{\SI{0.12}{\newton\second}}
        \wrongchoice{\SI{0.36}{\newton\second}}
        \wrongchoice{\SI{0.48}{\newton\second}}
        \wrongchoice{\SI{0.84}{\newton\second}}
    \end{choices}
    \end{multicols}
\end{question}
}

\element{schaums-mc}{
\begin{question}{ch08-Q07}
    During a serve,
        a tennis racket exerts an average force of \SI{250}{\newton} on a \SI{60}{\gram} tennis ball, initially at rest, for \SI{5.0}{\milli\second} (\SI{0.0050}{\second}). 
    The ball's kinetic energy afterward is:
    \begin{multicols}{2}
    \begin{choices}
        \wrongchoice{\SI{0.78}{\joule}}
        \wrongchoice{\SI{1.25}{\joule}}
        \wrongchoice{\SI{13}{\joule}}
        \wrongchoice{\SI{127}{\joule}}
    \end{choices}
    \end{multicols}
\end{question}
}

\element{schaums-mc}{
\begin{question}{ch08-Q08}
    A \SI{1500}{\kilo\gram} truck whose velocity is \SI{60}{\kilo\meter\per\hour} overtakes a 4000-kg truck moving in the same direction at \SI{35}{\kilo\meter\per\hour}. 
    The trucks collide and stick together,
        and the initial velocity of the wreckage is:
    \begin{multicols}{2}
    \begin{choices}
        \wrongchoice{\SI{9.1}{\kilo\meter\per\hour}}
        \wrongchoice{\SI{42}{\kilo\meter\per\hour}}
        \wrongchoice{\SI{48}{\kilo\meter\per\hour}}
        \wrongchoice{\SI{53}{\kilo\meter\per\hour}}
    \end{choices}
    \end{multicols}
\end{question}
}

\element{schaums-mc}{
\begin{question}{ch08-Q09}
    A \SI{1500}{\kilo\gram} truck whose velocity is \SI{60}{\kilo\meter\per\hour} overtakes a 4000-kg truck moving in the same direction at \SI{35}{\kilo\meter\per\hour}. 
    The trucks collide and stick together,
        and the initial velocity of the wreckage is:
    %% NOTE: change wording
    The trucks in Question 8.8 are headed in opposite directions when they collide. 
    The trucks again stick together,
        and the wreckage now has an initial velocity of:
    \begin{multicols}{2}
    \begin{choices}
        \wrongchoice{\SI{9.1}{\kilo\meter\per\hour}}
        \wrongchoice{\SI{42}{\kilo\meter\per\hour}}
        \wrongchoice{\SI{48}{\kilo\meter\per\hour}}
        \wrongchoice{\SI{53}{\kilo\meter\per\hour}}
    \end{choices}
    \end{multicols}
\end{question}
}

\element{schaums-mc}{
\begin{question}{ch08-Q10}
    A \SI{1500}{\kilo\gram} truck whose velocity is \SI{60}{\kilo\meter\per\hour} overtakes a 4000-kg truck moving in the same direction at \SI{35}{\kilo\meter\per\hour}. 
    The trucks collide and stick together,
        and the initial velocity of the wreckage is:
    %% NOTE: change wording
    The trucks in Question 8.8 lost a total KE that is
    \begin{choices}
        \wrongchoice{zero}
        %% NOTE: change wording
        \wrongchoice{less than the KE lost by the trucks in Question 8.9}
        \wrongchoice{the same as the KE lost by the trucks in Question 8.9}
        \wrongchoice{more than the KE lost by the trucks in Question 8.9}
    \end{choices}
\end{question}
}

\element{schaums-mc}{
\begin{question}{ch08-Q11}
    An \SI{800}{\kilo\gram} car headed south at \SI{40}{\kilo\meter\per\hour} strikes a \SI{1200}{\kilo\gram} car headed west at \SI{25}{\kilo\meter\per\hour}. 
    The cars stick together and the initial velocity of the wreckage is:
    \begin{multicols}{2}
    \begin{choices}
        \wrongchoice{\SI{22}{\kilo\meter\per\hour}}
        \wrongchoice{\SI{31}{\kilo\meter\per\hour}}
        \wrongchoice{\SI{33}{\kilo\meter\per\hour}}
        \wrongchoice{\SI{47}{\kilo\meter\per\hour}}
    \end{choices}
    \end{multicols}
\end{question}
}

\element{schaums-mc}{
\begin{question}{ch08-Q12}
    An \SI{800}{\kilo\gram} car headed south at \SI{40}{\kilo\meter\per\hour} strikes a \SI{1200}{\kilo\gram} car headed west at \SI{25}{\kilo\meter\per\hour}. 
    The cars stick together and the initial velocity of the wreckage is:
    %%  NOTE: change wording
    The wreckage of Question 8.11 moves off at:
    \begin{multicols}{2}
    \begin{choices}
        \wrongchoice{\ang{20} W of S}
        \wrongchoice{\ang{43} W of S}
        \wrongchoice{\ang{47} W of S}
        \wrongchoice{\ang{70} W of S}
    \end{choices}
    \end{multicols}
\end{question}
}

\element{schaums-mc}{
\begin{question}{ch08-Q13}
    A ball is dropped from a height of \SI{300}{\centi\meter}.
    If the coefficient of restitution is \num{0.600},
        the ball rebounds to a height of:
    \begin{multicols}{2}
    \begin{choices}
        \wrongchoice{\SI{104}{\centi\meter}}
        \wrongchoice{\SI{108}{\centi\meter}}
        \wrongchoice{\SI{134}{\centi\meter}}
        \wrongchoice{\SI{180}{\centi\meter}}
    \end{choices}
    \end{multicols}
\end{question}
}


\endinput

