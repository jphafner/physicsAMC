
%%--------------------------------------------------
%% Schaum's Outline of Applied Physics
%%--------------------------------------------------


%% Chapter 28: Electromagnetic Induction
%%--------------------------------------------------


%% Schaum's Multiple Choice Questions
%%--------------------------------------------------
\element{schaums-mc}{
\begin{questionmult}{ch28-Q01}
    Upon which one or more of the following does the magnetic flux through a wire loop in a magnetic field depend?
    \begin{choices}
      \correctchoice{the area of the loop}
        \wrongchoice{the shape of the loop}
      \correctchoice{the magnitude of the field}
      \correctchoice{the angle between the plane of the loop and the direction of the field}
    \end{choices}
\end{questionmult}
}

\element{schaums-mc}{
\begin{question}{ch28-Q02}
    According to Lenz's law an induced current:
    \begin{choices}
      \correctchoice{gives rise to a magnetic field of its own that opposes the flux change that created it}
        \wrongchoice{gives rise to a magnetic field of its own that reinforces the flux change that created it}
        \wrongchoice{appears when a wire moves parallel to a magnetic field}
        \wrongchoice{appears when a wire moves perpendicular to a magnetic field}
    \end{choices}
\end{question}
}

\element{schaums-mc}{
\begin{question}{ch28-Q03}
    The cause of the alternating current in the secondary coil of a transformer is an emf produced by:
    \begin{choices}
        \wrongchoice{the varying electric field of the primary coil}
      \correctchoice{the varying magnetic field of the primary coil}
        \wrongchoice{the varying magnetic field of the secondary coil}
        \wrongchoice{the voltage applied to the primary coil}
    \end{choices}
\end{question}
}

\element{schaums-mc}{
\begin{questionmult}{ch28-Q04}
    Upon which one or more of the following does the magnetic energy of a coil that carries a current $I$ depend?
    \begin{choices}
      \correctchoice{$I$}
      \correctchoice{the number of turns in the coil}
        \wrongchoice{the resistance of the coil}
      \correctchoice{the presence of an iron core in the coil}
    \end{choices}
\end{questionmult}
}

\element{schaums-mc}{
\begin{question}{ch28-Q05}
    The horizontal steel cargo boom of a freighter traveling at \SI{10}{\meter\per\second} is \SI{7.0}{\meter} long and is at an angle of \ang{75} relative to the direction of the ship's motion. 
    The magnetic field of the earth in that region has a vertical component of \SI{4.0e-5}{\tesla}.
    The potential difference between the ends of the boom is:
    \begin{multicols}{2}
    \begin{choices}
        \wrongchoice{\SI{0.06}{\milli\volt}}
        \wrongchoice{\SI{0.72}{\milli\volt}}
      \correctchoice{\SI{2.7}{\milli\volt}}
        \wrongchoice{\SI{2.8}{\milli\volt}}
    \end{choices}
    \end{multicols}
\end{question}
}

\element{schaums-mc}{
\begin{question}{ch28-Q06}
    A wire loop that encloses an area of \SI{15}{\centi\meter\squared} is perpendicular to a magnetic field of \SI{0.10}{\tesla}. 
    If the field drops to \SI{0.04}{\tesla} in \SI{0.2}{\second},
        the average emf induced in the loop is:
    \begin{multicols}{2}
    \begin{choices}
        \wrongchoice{\SI{0.3}{\milli\volt}}
      \correctchoice{\SI{0.45}{\milli\volt}}
        \wrongchoice{\SI{4}{\milli\volt}}
        \wrongchoice{\SI{4.5}{\volt}}
    \end{choices}
    \end{multicols}
\end{question}
}

\element{schaums-mc}{
\begin{question}{ch28-Q07}
    A \num{200} turn coil whose resistance is \SI{4}{\ohm}encloses an area of \SI{20}{\centi\meter\squared}. 
    A changing magnetic field parallel to the coil axis induces a current of \SI{1.2}{\ampere} in the coil.
    How rapidly is the magnetic field changing?
    \begin{multicols}{2}
    \begin{choices}
        \wrongchoice{\SI{0.75}{\tesla\per\second}}
      \correctchoice{\SI{12}{\tesla\per\second}}
        \wrongchoice{\SI{14.4}{\tesla\per\second}}
        \wrongchoice{\SI{30}{\tesla\per\second}}
    \end{choices}
    \end{multicols}
\end{question}
}

\element{schaums-mc}{
\begin{question}{ch28-Q08}
    A transformer has \num{300} turns in its primary coil and \num{75} turns in its secondary coil. 
    When the current in the secondary coil is \SI{20}{\ampere},
        the current in the primary coil is:
    \begin{multicols}{2}
    \begin{choices}
       \correctchoice{\SI{5}{\ampere}}
         \wrongchoice{\SI{25}{\ampere}}
         \wrongchoice{\SI{80}{\ampere}}
         \wrongchoice{\SI{6.4}{\kilo\ampere}}
    \end{choices}
    \end{multicols}
\end{question}
}

\element{schaums-mc}{
\begin{question}{ch28-Q09}
    If the power input to the transformer of Question 28.7 is \SI{40}{\watt},
        the power output is:
    \begin{multicols}{2}
    \begin{choices}
        \wrongchoice{\SI{2.5}{\watt}}
        \wrongchoice{\SI{10}{\watt}}
      \correctchoice{\SI{40}{\watt}}
        \wrongchoice{\SI{160}{\watt}}
    \end{choices}
    \end{multicols}
\end{question}
}

\element{schaums-mc}{
\begin{question}{ch28-Q10}
    The self-induced emf in a \SI{0.1}{\henry} coil in which the current is changing at \SI{200}{\ampere\per\second} is:
    \begin{multicols}{2}
    \begin{choices}
        \wrongchoice{\SI{10}{\volt}}
      \correctchoice{\SI{20}{\volt}}
        \wrongchoice{\SI{0.1}{\kilo\volt}}
        \wrongchoice{\SI{2}{\kilo\volt}}
    \end{choices}
    \end{multicols}
\end{question}
}

\element{schaums-mc}{
\begin{question}{ch28-Q11}
    While the current in a circuit is falling from \SI{8}{\ampere} to \SI{2}{\ampere} in \SI{20}{\milli\second},
        the average induced emf in the circuit is \SI{12}{\volt}. 
    The inductance of the circuit is:
    \begin{multicols}{2}
    \begin{choices}
      \correctchoice{\SI{0.04}{\henry}}
        \wrongchoice{\SI{0.12}{\henry}}
        \wrongchoice{\SI{100}{\henry}}
        \wrongchoice{\SI{3.6}{\kilo\henry}}
    \end{choices}
    \end{multicols}
\end{question}
}

\element{schaums-mc}{
\begin{question}{ch28-Q12}
    An \SI{80}{\milli\henry} coil whose resistance is \SI{4}{\ohm} is connected to a \SI{12}{\volt} battery of negligible internal resistance. 
    The current in the coil begins to increase at a rate of:
    \begin{multicols}{2}
    \begin{choices}
        \wrongchoice{\SI{0.15}{\ampere\per\second}}
        \wrongchoice{\SI{3}{\ampere\per\second}}
        \wrongchoice{\SI{7.5}{\ampere\per\second}}
      \correctchoice{\SI{150}{\ampere\per\second}}
    \end{choices}
    \end{multicols}
\end{question}
}

\element{schaums-mc}{
\begin{question}{ch28-Q13}
    The time needed for the current in the coil of Question 28.12 to reach 63 percent of its final value is:
    \begin{multicols}{2}
    \begin{choices}
      \correctchoice{\SI{0.02}{\second}}
        \wrongchoice{\SI{0.05}{\second}}
        \wrongchoice{\SI{20}{\second}}
        \wrongchoice{\SI{50}{\second}}
    \end{choices}
    \end{multicols}
\end{question}
}

\element{schaums-mc}{
\begin{question}{ch28-Q14}
    The energy stored in the magnetic field of a \SI{12}{\milli\henry} coil in which the current is \SI{5}{\ampere} is:
    \begin{multicols}{2}
    \begin{choices}
        \wrongchoice{\SI{1.8}{\milli\joule}}
        \wrongchoice{\SI{30}{\milli\joule}}
      \correctchoice{\SI{0.15}{\joule}}
        \wrongchoice{\SI{0.3}{\joule}}
    \end{choices}
    \end{multicols}
\end{question}
}


\endinput


