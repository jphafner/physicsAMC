
%%--------------------------------------------------
%% Schaum's Outline of Applied Physics
%%--------------------------------------------------


%% Chapter 36: Nuclear Physics
%%--------------------------------------------------


%% Schaum's Multiple Choice Questions
%%--------------------------------------------------
\element{schaums-mc}{
\begin{question}{ch36-Q01}
    Nearly all the volume occupied by matter consists of:
    \begin{multicols}{2}
    \begin{choices}
        \wrongchoice{electrons}
        \wrongchoice{protons}
        \wrongchoice{neutrons}
      \correctchoice{empty space}
    \end{choices}
    \end{multicols}
\end{question}
}

\element{schaums-mc}{
\begin{question}{ch36-Q02}
    The atomic number of an element is the number of:
    \begin{choices}
      \correctchoice{protons in its nucleus}
        \wrongchoice{neutrons in its nucleus}
        \wrongchoice{protons and neutrons in its nucleus}
        \wrongchoice{electrons in its nucleus}
    \end{choices}
\end{question}
}

\element{schaums-mc}{
\begin{question}{ch36-Q03}
    The weakest of the four fundamental interactions is the:
    \begin{multicols}{2}
    \begin{choices}
      \correctchoice{gravitational}
        \wrongchoice{electromagnetic}
        \wrongchoice{strong}
        \wrongchoice{weak}
    \end{choices}
    \end{multicols}
\end{question}
}

\element{schaums-mc}{
\begin{question}{ch36-Q04}
    The isotopes of an element all have the same:
    \begin{multicols}{2}
    \begin{choices}
      \correctchoice{atomic number}
        \wrongchoice{mass number}
        \wrongchoice{binding energy}
        \wrongchoice{half-life}
    \end{choices}
    \end{multicols}
\end{question}
}

\element{schaums-mc}{
\begin{question}{ch36-Q05}
    Each nucleus of the nitrogen isotope \ce{^{16}_{7}N} contains:
    \begin{multicols}{2}
    \begin{choices}
        \wrongchoice{7 neutrons}
      \correctchoice{9 neutrons}
        \wrongchoice{16 neutrons}
        \wrongchoice{23 neutrons}
    \end{choices}
    \end{multicols}
\end{question}
}

\element{schaums-mc}{
\begin{question}{ch36-Q06}
    Nuclear fusion and fission reactions give off energy because:
    \begin{choices}
        \wrongchoice{the binding energy per nucleon is least for nuclei of intermediate size}
      \correctchoice{the binding energy per nucleon is most for nuclei of intermediate size}
        \wrongchoice{they liberate neutrons}
        \wrongchoice{they liberate protons}
    \end{choices}
\end{question}
}

\element{schaums-mc}{
\begin{question}{ch36-Q07}
    In a chain reaction:
    \begin{choices}
        \wrongchoice{protons and neutrons join to form atomic nuclei}
        \wrongchoice{light nuclei join to form heavy ones}
      \correctchoice{neutrons emitted during the fission of heavy nuclei induce fissions in other nuclei}
        \wrongchoice{uranium is burned in a type of furnace called a reactor}
    \end{choices}
\end{question}
}

\element{schaums-mc}{
\begin{question}{ch36-Q08}
    The energy that heats the sun has its origin in:
    \begin{choices}
        \wrongchoice{radioactivity}
        \wrongchoice{nuclear fission}
      \correctchoice{the production of helium from hydrogen}
        \wrongchoice{the production of hydrogen from helium}
    \end{choices}
\end{question}
}

\element{schaums-mc}{
\begin{question}{ch36-Q09}
    An alpha particle consists of:
    \begin{choices}
        \wrongchoice{two protons}
        \wrongchoice{two protons and two electrons}
      \correctchoice{two protons and two neutrons}
        \wrongchoice{two protons, two neutrons, and two electrons}
    \end{choices}
\end{question}
}

\element{schaums-mc}{
\begin{question}{ch36-Q10}
    Gamma rays have the same basic nature as:
    \begin{choices}
        \wrongchoice{alpha particles}
        \wrongchoice{electrons}
        \wrongchoice{positrons}
      \correctchoice{X-rays}
    \end{choices}
\end{question}
}

\element{schaums-mc}{
\begin{question}{ch36-Q11}
    The most penetrating of the following radiations is:
    \begin{choices}
        \wrongchoice{an alpha particle}
        \wrongchoice{an electron}
        \wrongchoice{a positron}
      \correctchoice{a gamma ray}
    \end{choices}
\end{question}
}

\element{schaums-mc}{
\begin{question}{ch36-Q12}
    The half-life of a radionuclide equals:
    \begin{choices}
        \wrongchoice{half the time needed for a sample to completely decay}
        \wrongchoice{half the time a sample can be kept before it starts to decay}
      \correctchoice{the time needed for half a sample to decay}
        \wrongchoice{the time needed for the rest of a sample to decay once half of it has already decayed}
    \end{choices}
\end{question}
}

\element{schaums-mc}{
\begin{question}{ch36-Q13}
    During the decay of a radionuclide,
        its half-life:
    \begin{choices}
        \wrongchoice{decreases}
      \correctchoice{does not change}
        \wrongchoice{increases}
        \wrongchoice{any of the provided, depending on the nuclide}
    \end{choices}
\end{question}
}

\element{schaums-mc}{
\begin{question}{ch36-Q14}
    The sum of the masses of 10 protons and 10 neutrons is \SI{0.172}{u} more than the mass of a \ce{^{20}_{10}Ne} nucleus. 
    The binding energy per nucleon in this nucleus is:
    \begin{multicols}{2}
    \begin{choices}
        \wrongchoice{\SI{8.6e-3}{\eV}}
      \correctchoice{\SI{8.0}{\mega\eV}}
        \wrongchoice{\SI{16.0}{\mega\eV}}
        \wrongchoice{\SI{7.7e14}{\eV}}
    \end{choices}
    \end{multicols}
\end{question}
}

\element{schaums-mc}{
\begin{question}{ch36-Q15}
    When the uranium isotope \ce{^{234}_{92}U} undergoes alpha decay,
        the result is the nuclide:
    \begin{multicols}{2}
    \begin{choices}
      \correctchoice{\ce{^{230}_{90}Th}}
        \wrongchoice{\ce{^{230}_{92}U}}
        \wrongchoice{\ce{^{232}_{88}Ra}}
        \wrongchoice{\ce{^{230}_{88}RA}}
    \end{choices}
    \end{multicols}
\end{question}
}

\element{schaums-mc}{
\begin{question}{ch36-Q16}
    When the strontium isotope \ce{^{87}_{38}Sr} undergoes gamma decay,
        the result is the nuclide:
    \begin{multicols}{2}
    \begin{choices}
        \wrongchoice{\ce{^{87}_{37}Rb}}
      \correctchoice{\ce{^{87}_{38}Sr}}
        \wrongchoice{\ce{^{87}_{39}Y}}
        \wrongchoice{\ce{^{83}_{36}Kr}}
    \end{choices}
    \end{multicols}
\end{question}
}

\element{schaums-mc}{
\begin{question}{ch36-Q17}
    The copper isotope \ce{^{64}_{29}Cu} decays into the nickel isotope \ce{^{64}_{28}Ni} by emitting:
    \begin{choices}
        \wrongchoice{an electron}
      \correctchoice{a positron}
        \wrongchoice{an alpha particle}
        \wrongchoice{a gamma ray}
    \end{choices}
\end{question}
}

\element{schaums-mc}{
\begin{question}{ch36-Q18}
    A certain radionuclide has a half-life of \SI{12}{\hour}. 
    Starting from \SI{1.00}{\gram} of the nuclide,
        the amount left after \SI{2}{\day} will be:
    \begin{multicols}{2}
    \begin{choices}
        \wrongchoice{zero}
      \correctchoice{\SI{0.0625}{\gram}}
        \wrongchoice{\SI{0.16}{\gram}}
        \wrongchoice{\SI{0.25}{\gram}}
    \end{choices}
    \end{multicols}
\end{question}
}

\element{schaums-mc}{
\begin{question}{ch36-Q19}
    After \SI{2}{\hour} has elapsed,
        \num{1/16} of the original quantity of a certain radionuclide remains undecayed. 
    The half-life of this radionuclide is:
    \begin{multicols}{2}
    \begin{choices}
        \wrongchoice{\SI{15}{\minute}}
      \correctchoice{\SI{30}{\minute}}
        \wrongchoice{\SI{45}{\minute}}
        \wrongchoice{\SI{60}{\minute}}
    \end{choices}
    \end{multicols}
\end{question}
}


\endinput


