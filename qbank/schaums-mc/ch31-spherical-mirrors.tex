
%%--------------------------------------------------
%% Schaum's Outline of Applied Physics
%%--------------------------------------------------


%% Chapter 31: Spherical Mirros
%%--------------------------------------------------


%% Schaum's Multiple Choice Questions
%%--------------------------------------------------
\element{schaums-mc}{
\begin{question}{ch31-Q01}
    When a spherical mirror forms a real image,
        the image relative to its object is always:
    \begin{multicols}{2}
    \begin{choices}
        \wrongchoice{smaller}
        \wrongchoice{larger}
        \wrongchoice{erect}
      \correctchoice{inverted}
    \end{choices}
    \end{multicols}
\end{question}
}

\element{schaums-mc}{
\begin{question}{ch31-Q02}
    An image distance that is negative means that the image is:
    \begin{multicols}{2}
    \begin{choices}
        \wrongchoice{erect}
        \wrongchoice{inverted}
        \wrongchoice{real}
      \correctchoice{virtual}
    \end{choices}
    \end{multicols}
\end{question}
}

\element{schaums-mc}{
\begin{question}{ch31-Q03}
    A magnification that is negative means that the image is:
    \begin{choices}
        \wrongchoice{erect}
      \correctchoice{inverted}
        \wrongchoice{smaller than the object}
        \wrongchoice{larger than the object}
    \end{choices}
\end{question}
}

\element{schaums-mc}{
\begin{question}{ch31-Q04}
    An erect image is formed by a concave mirror, with focal length $f$, when the object distance is:
    \begin{choices}
        %% NOTE: replace f with focal distance
      \correctchoice{less than $f$}
        \wrongchoice{$f$}
        \wrongchoice{between $f$ and $2f$}
        \wrongchoice{greater than $2f$}
    \end{choices}
\end{question}
}

\element{schaums-mc}{
\begin{question}{ch31-Q05}
    A concave mirror, with focal length $f$, forms an enlarged image, $p$, of an object:
    \begin{choices}
        \wrongchoice{for no values of $p$}
      \correctchoice{when $p$ is less than $2f$}
        \wrongchoice{when $p$ is more than $2f$}
        \wrongchoice{for all values of $p$}
    \end{choices}
\end{question}
}

\element{schaums-mc}{
\begin{question}{ch31-Q06}
    A convex mirror forms an enlarged image of an object:
    \begin{choices}
      \correctchoice{for no values of $p$}
        \wrongchoice{when $p$ is less than $2f$}
        \wrongchoice{when $p$ is more than $2f$}
        \wrongchoice{for all values of $p$}
    \end{choices}
\end{question}
}

\element{schaums-mc}{
\begin{question}{ch31-Q07}
    The focal length of a convex mirror whose radius of curvature is \SI{24}{\centi\meter} is:
    \begin{multicols}{2}
    \begin{choices}
      \correctchoice{\SI[retain-explicit-plus]{-12}{\centi\meter}}
        \wrongchoice{\SI[retain-explicit-plus]{-48}{\centi\meter}}
        \wrongchoice{\SI[retain-explicit-plus]{+12}{\centi\meter}}
        \wrongchoice{\SI[retain-explicit-plus]{+48}{\centi\meter}}
    \end{choices}
    \end{multicols}
\end{question}
}

\element{schaums-mc}{
\begin{question}{ch31-Q08}
    A candle \SI{6}{\centi\meter} high is \SI{40}{\centi\meter} in front of a concave mirror whose focal length is \SI[retain-explicit-plus]{+60}{\centi\meter}. 
    The image is:
    \begin{choices}
        \wrongchoice{\SI{2}{\centi\meter} high, erect}
      \correctchoice{\SI{18}{\centi\meter} high, erect}
        \wrongchoice{\SI{2}{\centi\meter} high, inverted}
        \wrongchoice{\SI{18}{\centi\meter} high, inverted}
    \end{choices}
\end{question}
}

\element{schaums-mc}{
\begin{question}{ch31-Q09}
    The candle of Question 31.8 is \SI{100}{\centi\meter} in front of the same mirror. 
    The image is now:
    \begin{choices}
        \wrongchoice{\SI{4}{\centi\meter} long, erect}
        \wrongchoice{\SI{9}{\centi\meter} long, erect}
        \wrongchoice{\SI{4}{\centi\meter} long, inverted}
      \correctchoice{\SI{9}{\centi\meter} long, inverted}
    \end{choices}
\end{question}
}

\element{schaums-mc}{
\begin{question}{ch31-Q10}
    A candle \SI{6}{\centi\meter} high is \SI{40}{\centi\meter} in front of a convex mirror whose focal length is \SI{-60}{\centi\meter}. 
    The image is:
    \begin{choices}
      \correctchoice{\SI{3.6}{\centi\meter} long, erect}
        \wrongchoice{\SI{10}{\centi\meter} long, erect}
        \wrongchoice{\SI{3.6}{\centi\meter} long, inverted}
        \wrongchoice{\SI{10}{\centi\meter} long, inverted}
    \end{choices}
\end{question}
}


\endinput


