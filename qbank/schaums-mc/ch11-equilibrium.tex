
%%--------------------------------------------------
%% Schaum's Outline of Applied Physics
%%--------------------------------------------------


%% Chapter 11: Equibilibrium
%%--------------------------------------------------


%% Schaum's Multiple Choice Questions
%%--------------------------------------------------
\element{schaums-mc}{
\begin{questionmult}{ch11-Q01}
    Which one or more of the following sets of horizontal forces could leave an object in equilibrium?
    \begin{choices}
        %% NOTE: formatting?
        \wrongchoice{5, 10, and 20 N}
      \correctchoice{6, 12, and 18 N}
      \correctchoice{8, 8, and 8 N}
        \wrongchoice{2, 4, 8, and 16 N}
    \end{choices}
\end{questionmult}
}

\element{schaums-mc}{
\begin{question}{ch11-Q02}
    A box of weight $w$ is supported by two ropes. 
    The magnitude of:
    \begin{choices}
        \wrongchoice{the tension in each rope must be $w/2$}
        \wrongchoice{the tension in each rope must be $w$}
      \correctchoice{the vector sum of the tensions in both ropes must be $w$}
        \wrongchoice{the vector sum of the tensions in both ropes must be greater than $w$}
    \end{choices}
\end{question}
}

\element{schaums-mc}{
\begin{question}{ch11-Q03}
    The torques that act on an object in equilibrium have a vector sum of zero about:
    \begin{choices}
        \wrongchoice{one point only}
        \wrongchoice{one or more points}
      \correctchoice{all points}
        \wrongchoice{any of the above, depending on the situation}
    \end{choices}
\end{question}
}

\element{schaums-mc}{
\begin{question}{ch11-Q04}
    The point about which torques are calculated when studying the equilibrium of an object:
    \begin{choices}
        \wrongchoice{must pass through the center of gravity of the object}
        \wrongchoice{must pass through one end of the object}
        \wrongchoice{must be located inside of the object}
      \correctchoice{may be located anywhere}
    \end{choices}
\end{question}
}

\element{schaums-mc}{
\begin{question}{ch11-Q05}
    A \SI{250}{\newton} box hangs from a rope. 
    If the box is pushed with a horizontal force of \SI{145}{\newton},
        the angle between the rope and the vertical is:
    \begin{multicols}{2}
    \begin{choices}
      \correctchoice{\ang{30}◦}
        \wrongchoice{\ang{45}}
        \wrongchoice{\ang{60}}
        \wrongchoice{\ang{75}}
    \end{choices}
    \end{multicols}
\end{question}
}

\element{schaums-mc}{
\begin{question}{ch11-Q06}
    A box is suspended by a rope. 
    When a horizontal force of \SI{100}{\newton} acts on the box,
        it moves to the side until the rope is at an angle of \ang{20} with the vertical. 
    The weight of the box is:
    \begin{multicols}{2}
    \begin{choices}
        \wrongchoice{\SI{36}{\newton}}
        \wrongchoice{\SI{106}{\newton}}
      \correctchoice{\SI{275}{\newton}}
        \wrongchoice{\SI{292}{\newton}}
    \end{choices}
    \end{multicols}
\end{question}
}

\element{schaums-mc}{
\begin{question}{ch11-Q07}
    A \SI{5}{\pound} picture is held up by two strings that go from its upper corners to a hook on the wall. 
    Each string has a breaking strength of \SI{3}{\pound}. 
    The maximum angle between the strings if they are not to break is:
    \begin{multicols}{2}
    \begin{choices}
        \wrongchoice{\ang{34}}
      \correctchoice{\ang{67}}
        \wrongchoice{\ang{56}}
        \wrongchoice{\ang{80}}
    \end{choices}
    \end{multicols}
\end{question}
}

\element{schaums-mc}{
\begin{question}{ch11-Q08}
    A picture hangs from two wires that go from its upper corners to a nail in the wall. 
    If the picture weighs \SI{8.0}{\newton} and each wire makes a \ang{30} angle with the vertical,
        the tension in each wire is:
    \begin{multicols}{2}
    \begin{choices}
        \wrongchoice{\SI{4.0}{\newton}}
      \correctchoice{\SI{4.6}{\newton}}
        \wrongchoice{\SI{8.0}{\newton}}
        \wrongchoice{\SI{9.2}{\newton}}
    \end{choices}
    \end{multicols}
\end{question}
}

\element{schaums-mc}{
\begin{question}{ch11-Q09}
    An \SI{0.80}{\kilo\newton} load hangs from the end of a horizontal boom \SI{2.0}{\meter} long hinged to a vertical mast. 
    A rope \SI{2.5}{\meter} long joins the end of the boom with a point on the mast \SI{1.5}{\meter} above the hinge. 
    The tension in the rope is:
    \begin{multicols}{2}
    \begin{choices}
        \wrongchoice{\SI{0.48}{\kilo\newton}}
        \wrongchoice{\SI{0.80}{\kilo\newton}}
        \wrongchoice{\SI{1.00}{\kilo\newton}}
      \correctchoice{\SI{1.33}{\kilo\newton}}
    \end{choices}
    \end{multicols}
\end{question}
}

\element{schaums-mc}{
\begin{question}{ch11-Q10}
    An \SI{0.80}{\kilo\newton} load hangs from the end of a horizontal boom \SI{2.0}{\meter} long hinged to a vertical mast. 
    A rope \SI{2.5}{\meter} long joins the end of the boom with a point on the mast \SI{1.5}{\meter} above the hinge. 
    The tension in the rope is:
    %% NOTE: reword
    The boom of Question 11.9 exerts an inward force on the hinge of:
    \begin{multicols}{2}
    \begin{choices}
        \wrongchoice{\SI{0.60}{\kilo\newton}}
        \wrongchoice{\SI{0.80}{\kilo\newton}}
      \correctchoice{\SI{1.07}{\kilo\newton}}
        \wrongchoice{\SI{1.67}{\kilo\newton}}
    \end{choices}
    \end{multicols}
\end{question}
}

\element{schaums-mc}{
\begin{question}{ch11-Q11}
    A \SI{50}{\kilo\gram} barrel hangs from one end of a \SI{30}{\kilo\gram} beam \SI{3.0}{\meter} long. 
    The distance from the loaded end to the balance point is:
    \begin{multicols}{2}
    \begin{choices}
      \correctchoice{\SI{56}{\centi\meter}}
        \wrongchoice{\SI{75}{\centi\meter}}
        \wrongchoice{\SI{94}{\centi\meter}}
        \wrongchoice{\SI{113}{\centi\meter}}
    \end{choices}
    \end{multicols}
\end{question}
}

\element{schaums-mc}{
\begin{question}{ch11-Q12}
    A \SI{50}{\kilo\gram} steel pipe \SI{4.0}{\meter} long is supported by a rope attached \SI{1.7}{\meter} from one end. 
    The downward force that must be applied to the end of the pipe closest to the rope to keep the pipe horizontal is:
    \begin{multicols}{2}
    \begin{choices}
        \wrongchoice{\SI{8.8}{\newton}}
      \correctchoice{\SI{86}{\newton}}
        \wrongchoice{\SI{227}{\newton}}
        \wrongchoice{\SI{490}{\newton}}
    \end{choices}
    \end{multicols}
\end{question}
}



\endinput

