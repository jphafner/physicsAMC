
%%--------------------------------------------------
%% Schaum's Outline of Applied Physics
%%--------------------------------------------------


%% Chapter 10: Rotational Motion
%%--------------------------------------------------


%% Schaum's Multiple Choice Questions
%%--------------------------------------------------
\element{schaums-mc}{
\begin{question}{ch10-Q01}
    A particle the distance $R$ from the axis of rotation of a spinning body has:
    \begin{choices}
        \wrongchoice{an angular velocity proportional to $R$}
        \wrongchoice{an angular velocity proportional to $1/R$}
      \correctchoice{a linear velocity proportional to $R$}
        \wrongchoice{a linear velocity proportional to $1/R$}
    \end{choices}
\end{question}
}

\element{schaums-mc}{
\begin{question}{ch10-Q02}
    A hoop and a disk of the same mass and radius roll down an inclined plane. 
    At the bottom they have the same:
    \begin{choices}
        \wrongchoice{angular velocity}
        \wrongchoice{angular momentum}
        \wrongchoice{kinetic energy of rotation}
      \correctchoice{potential energy}
    \end{choices}
\end{question}
}

\element{schaums-mc}{
\begin{question}{ch10-Q03}
    A ball is swung in a circle at the end of a string. 
    A property of the ball that does not depend on the length of the string is its:
    \begin{choices}
      \correctchoice{angular velocity}
        \wrongchoice{angular momentum}
        \wrongchoice{linear velocity}
        \wrongchoice{centripetal acceleration}
    \end{choices}
\end{question}
}

\element{schaums-mc}{
\begin{question}{ch10-Q04}
    A ball is swung in a circle at the end of a string. 
    A property of the ball that does not depend on the length of the string is its:
    %% NOTE: reword
    The ball of Question 10.3 does not have to possess:
    \begin{choices}
        \wrongchoice{angular velocity}
        \wrongchoice{angular momentum}
      \correctchoice{angular acceleration}
        \wrongchoice{centripetal acceleration}
    \end{choices}
\end{question}
}

\element{schaums-mc}{
\begin{question}{ch10-Q05}
    A quarter of a circle contains:
    \begin{multicols}{2}
    \begin{choices}
        \wrongchoice{\SI{\pi/4}{\radian}}
      \correctchoice{\SI{\pi/2}{\radian}}
        \wrongchoice{\SI{\pi}{\radian}}
        \wrongchoice{\SI{2\pi}{\radian}}
    \end{choices}
    \end{multicols}
\end{question}
}

\element{schaums-mc}{
\begin{question}{ch10-Q06}
    An angle of \SI{\pi/18}{\radian} is equivalent to:
    \begin{multicols}{2}
    \begin{choices}
      \correctchoice{\ang{10}}
        \wrongchoice{\ang{18}}
        \wrongchoice{\ang{20}}
        \wrongchoice{\ang{36}}
    \end{choices}
    \end{multicols}
\end{question}
}

\element{schaums-mc}{
\begin{question}{ch10-Q07}
    The linear velocity of the rim of a wheel \SI{80}{\centi\meter} in diameter when the wheel turns at \SI{90}{\revolution\per\minute} is:
    \begin{multicols}{2}
    \begin{choices}
        \wrongchoice{\SI{0.6\pi}{\meter\per\second}}
      \correctchoice{\SI{1.2\pi}{\meter\per\second}}
        \wrongchoice{\SI{1.5\pi}{\meter\per\second}}
        \wrongchoice{\SI{2.4\pi}{\meter\per\second}}
    \end{choices}
    \end{multicols}
\end{question}
}

\element{schaums-mc}{
\begin{question}{ch10-Q08}
    In \SI{30}{\second} the crankshaft of a truck engine operating at \SI{2400}{\revolution\per\minute} turns through:
    \begin{multicols}{2}
    \begin{choices}
        \wrongchoice{\SI{382}{\radian}}
        \wrongchoice{\SI{1200}{\radian}}
        \wrongchoice{\SI{3770}{\radian}}
      \correctchoice{\SI{7540}{\radian}}
    \end{choices}
    \end{multicols}
\end{question}
}

\element{schaums-mc}{
\begin{question}{ch10-Q09}
    A pulley is uniformly accelerated from rest to an angular velocity of \SI{30}{\radian\per\second} in \SI{8.0}{\second}. 
    The total angle through which the pulley turned during the acceleration is:
    \begin{multicols}{2}
    \begin{choices}
        \wrongchoice{\SI{60}{\radian}}
      \correctchoice{\SI{120}{\radian}}
        \wrongchoice{\SI{240}{\radian}}
        \wrongchoice{\SI{3600}{\radian}}
    \end{choices}
    \end{multicols}
\end{question}
}

\element{schaums-mc}{
\begin{question}{ch10-Q10}
    A motor takes \SI{6.0}{\second} to go from \SI{150}{\radian\per\second} to \SI{50}{\radian\per\second} at constant angular acceleration. 
    The total angle through which the motor's shaft turned during the acceleration is:
    \begin{multicols}{2}
    \begin{choices}
        \wrongchoice{\SI{300}{\radian}}
      \correctchoice{\SI{600}{\radian}}
        \wrongchoice{\SI{1200}{\radian}}
        \wrongchoice{\SI{3600}{\radian}}
    \end{choices}
    \end{multicols}
\end{question}
}

\element{schaums-mc}{
\begin{question}{ch10-Q11}
    A solid iron cylinder A rolls down a ramp,
        and an identical iron cylinder $B$ slides down the same ramp without friction.
    \begin{choices}
        \wrongchoice{$A$ reaches the bottom first.}
      \correctchoice{$B$ reaches the bottom first.}
        \wrongchoice{$A$ and $B$ reach the bottom together.}
        \wrongchoice{Any of the above, depending on the angle of the ramp.}
    \end{choices}
\end{question}
}

\element{schaums-mc}{
\begin{question}{ch10-Q12}
    A solid iron cylinder A rolls down a ramp,
        and an identical iron cylinder $B$ slides down the same ramp without friction.
    %% NOTE: reword
    When the cylinders of Question 10.11 reach the bottom of the ramp,
    \begin{choices}
        \wrongchoice{the kinetic energy of $A$ is more than the kinetic energy of $B$}
        \wrongchoice{the kinetic energy of $B$ is more than the kinetic energy of $A$}
      \correctchoice{$A$ and $B$ have the same kinetic energy}
        \wrongchoice{any of the above, depending on the angle of the ramp}
    \end{choices}
\end{question}
}

\element{schaums-mc}{
\begin{question}{ch10-Q13}
    A solid wooden disk rolls down a ramp. 
    The center of the disk is then cut out,
        and the resulting doughnut rolls down the same ramp. 
    At the bottom of the ramp the doughnut's velocity is:
    \begin{choices}
      \correctchoice{less than that of the solid disk}
        \wrongchoice{the same as that of the solid disk}
        \wrongchoice{greater than that of the solid disk}
        \wrongchoice{any of the above, depending on the angle of the ramp}
    \end{choices}
\end{question}
}

\element{schaums-mc}{
\begin{question}{ch10-Q14}
    A flywheel whose moment of inertia is \SI{4.0}{\kilo\gram\meter\squared} is acted on by a torque of \SI{50}{\newton\meter}. 
    Six seconds after starting from an angular velocity of \SI{40}{\radian\per\second} the flywheel will have turned through:
    \begin{multicols}{2}
    \begin{choices}
        \wrongchoice{\SI{225}{\radian}}
        \wrongchoice{\SI{315}{\radian}}
      \correctchoice{\SI{465}{\radian}}
        \wrongchoice{\SI{3053}{\radian}}
    \end{choices}
    \end{multicols}
\end{question}
}

\element{schaums-mc}{
\begin{question}{ch10-Q15}
    A flywheel whose moment of inertia is \SI{4.0}{\kilo\gram\meter\squared} is acted on by a torque of \SI{50}{\newton\meter}. 
    Six seconds after starting from an angular velocity of \SI{40}{\radian\per\second} the flywheel will have turned through:
    %% NOTE: reword
    The kinetic energy of the flywheel of Question 10.14 will have increased by:
    \begin{multicols}{2}
    \begin{choices}
        \wrongchoice{\SI{8.05}{\kilo\joule}}
        \wrongchoice{\SI{11.25}{\kilo\joule}}
      \correctchoice{\SI{23.25}{\kilo\joule}}
        \wrongchoice{\SI{26.45}{\kilo\joule}}
    \end{choices}
    \end{multicols}
\end{question}
}

\element{schaums-mc}{
\begin{question}{ch10-Q16}
    The torque needed to bring a turbine whose moment of inertia is \SI{60}{\slug\foot\squared} to rest in \SI{12}{\second} from an initial angular velocity of \SI{80}{\radian\per\second} is:
    \begin{multicols}{2}
    \begin{choices}
        \wrongchoice{\SI{9}{\pound\foot}}
        \wrongchoice{\SI{16}{\pound\foot}}
        \wrongchoice{\SI{200}{\pound\foot}}
      \correctchoice{\SI{400}{\pound\foot}}
    \end{choices}
    \end{multicols}
\end{question}
}

\element{schaums-mc}{
\begin{question}{ch10-Q17}
    A \SI{5.0}{\newton} force acts tangentially on the rim of a wheel of radius \SI{36}{\centi\meter}.
    Starting from rest, the wheel makes \SI{10}{\revolution} in \SI{4.0}{\second}.
    The moment of inertia of the wheel is:
    \begin{multicols}{2}
    \begin{choices}
        \wrongchoice{\SI{0.11}{\kilo\gram\meter\squared}}
      \correctchoice{\SI{0.23}{\kilo\gram\meter\squared}}
        \wrongchoice{\SI{0.46}{\kilo\gram\meter\squared}}
        \wrongchoice{\SI{1.44}{\kilo\gram\meter\squared}}
    \end{choices}
    \end{multicols}
\end{question}
}

\element{schaums-mc}{
\begin{question}{ch10-Q18}
    An engine turning at \SI{3000}{\revolution\per\minute} develops \SI{75}{\kilo\watt}.
    The torque on the engine's shaft:
    \begin{choices}
        \wrongchoice{is \SI{2.6}{\newton\meter}}
        \wrongchoice{is \SI{25}{\newton\meter}}
      \correctchoice{is \SI{239}{\newton\meter}}
        \wrongchoice{depends on the shaft radius}
    \end{choices}
\end{question}
}


\endinput

