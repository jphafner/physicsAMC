
%%--------------------------------------------------
%% Schaum's Outline of Applied Physics
%%--------------------------------------------------


%% Chapter 14: Simple Harmonic Motion
%%--------------------------------------------------


%% Schaum's Multiple Choice Questions
%%--------------------------------------------------
\element{schaums-mc}{
\begin{questionmult}{ch14-Q01}
    The period of a simple harmonic oscillator does not depend on which one or more of the following?
    \begin{choices}
        \wrongchoice{its mass}
        \wrongchoice{its frequency}
        \wrongchoice{its force constant}
      \correctchoice{its amplitude}
    \end{choices}
\end{questionmult}
}

\element{schaums-mc}{
\begin{questionmult}{ch14-Q02}
    The period of a simple pendulum depends upon its:
    \begin{choices}
        \wrongchoice{mass}
      \correctchoice{length}
        \wrongchoice{maximum velocity}
        \wrongchoice{total energy}
    \end{choices}
\end{questionmult}
}

\element{schaums-mc}{
\begin{question}{ch14-Q03}
    When an object that is undergoing simple harmonic motion passes through its equilibrium position,
        its velocity is:
    \begin{choices}
        \wrongchoice{zero}
        \wrongchoice{half its maximum value}
      \correctchoice{its maximum value}
        \wrongchoice{none of the above}
    \end{choices}
\end{question}
}

\element{schaums-mc}{
\begin{question}{ch14-Q04}
    A spring is cut in three equal parts. 
    If its original force constant was $k$,
        each new spring has a force constant of:
    \begin{multicols}{2}
    \begin{choices}
        \wrongchoice{$k/3$}
        \wrongchoice{$k$}
      \correctchoice{$3k$}
        \wrongchoice{$9k$}
    \end{choices}
    \end{multicols}
\end{question}
}

\element{schaums-mc}{
\begin{question}{ch14-Q05}
    A vertical spring \SI{60}{\milli\meter} long resting on a table is compressed by \SI{5.0}{\milli\meter} when a \SI{200}{\gram} mass is placed on it. 
    The force constant of the spring is:
    \begin{multicols}{2}
    \begin{choices}
        \wrongchoice{\SI{1.6}{\newton\per\meter}}
        \wrongchoice{\SI{40}{\newton\per\meter}}
        \wrongchoice{\SI{196}{\newton\per\meter}}
      \correctchoice{\SI{392}{\newton\per\meter}}
    \end{choices}
    \end{multicols}
\end{question}
}

\element{schaums-mc}{
\begin{question}{ch14-Q06}
    A spring is stretched by \SI{30}{\milli\meter} when a force of \SI{0.40}{\newton} is applied to it. 
    The potential energy of the stretched spring is:
    \begin{multicols}{2}
    \begin{choices}
        \wrongchoice{\SI{2.8e-5}{\joule}}
        \wrongchoice{\SI{7.2e-5}{\joule}}
      \correctchoice{\SI{6.0e-3}{\joule}}
        \wrongchoice{\SI{1.2e-2}{\joule}}
    \end{choices}
    \end{multicols}
\end{question}
}

\element{schaums-mc}{
\begin{question}{ch14-Q07}
    A vertical spring \SI{60}{\milli\meter} long resting on a table is compressed by \SI{5.0}{\milli\meter} when a \SI{200}{\gram} mass is placed on it. 
    The force constant of the spring is:
    %% NOTE: reword
    If it is pressed down and released,
        the mass of Question 14.5 will oscillate up and down with a period of:
    \begin{multicols}{2}
    \begin{choices}
        \wrongchoice{\SI{0.0032}{\second}}
        \wrongchoice{\SI{0.057}{\second}}
      \correctchoice{\SI{0.14}{\second}}
        \wrongchoice{\SI{0.44}{\second}}
    \end{choices}
    \end{multicols}
\end{question}
}

\element{schaums-mc}{
\begin{question}{ch14-Q08}
    A particle that undergoes simple harmonic motion has a period of \SI{0.40}{\second} and an amplitude of \SI{12}{\milli\meter}. 
    The maximum velocity of the particle is:
    \begin{multicols}{2}
    \begin{choices}
        \wrongchoice{\SI{3}{\centi\meter\per\second}}
      \correctchoice{\SI{19}{\centi\meter\per\second}}
        \wrongchoice{\SI{38}{\centi\meter\per\second}}
        \wrongchoice{\SI{43}{\centi\meter\per\second}}
    \end{choices}
    \end{multicols}
\end{question}
}

\element{schaums-mc}{
\begin{question}{ch14-Q09}
    If a mass of \SI{15}{\gram} is to oscillate at \SI{12}{\hertz},
        it should hang from a spring whose force constant is:
    \begin{multicols}{2}
    \begin{choices}
        \wrongchoice{\SI{1.1}{\newton\per\meter}}
        \wrongchoice{\SI{1.3}{\newton\per\meter}}
        \wrongchoice{\SI{7.1}{\newton\per\meter}}
      \correctchoice{\SI{85}{\newton\per\meter}}
    \end{choices}
    \end{multicols}
\end{question}
}

\element{schaums-mc}{
\begin{question}{ch14-Q10}
    The end of a diving board moves down by \SI{30}{\centi\meter} when a girl stands on it. 
    If she then bounces up and down,
        the frequency of the oscillations:
    \begin{choices}
      \correctchoice{is \SI{0.91}{\hertz}}
        \wrongchoice{is \SI{1.1}{\hertz}}
        \wrongchoice{is \SI{2.3}{\hertz}}
        \wrongchoice{depends on her mass}
    \end{choices}
\end{question}
}

\element{schaums-mc}{
\begin{question}{ch14-Q11}
    To increase the period of a harmonic oscillator from \SI{3}{\second} to \SI{6}{\second},
        its original mass of \SI{20}{\gram} should be changed to:
    \begin{multicols}{2}
    \begin{choices}
        \wrongchoice{\SI{5}{\gram}}
        \wrongchoice{\SI{10}{\gram}}
        \wrongchoice{\SI{40}{\gram}}
      \correctchoice{\SI{80}{\gram}}
    \end{choices}
    \end{multicols}
\end{question}
}

\element{schaums-mc}{
\begin{question}{ch14-Q12}
    %%  NOTE: SI cycles
    The \SI{400}{\gram} piston in a compressor oscillates up and down through a total distance of \SI{80}{\milli\meter}. 
    The maximum force on the piston when it goes through 10 cycles/s is:
    \begin{multicols}{2}
    \begin{choices}
        \wrongchoice{\SI{1.0}{\newton}}
        \wrongchoice{\SI{6.3}{\newton}}
      \correctchoice{\SI{63}{\newton}}
        \wrongchoice{\SI{126}{\newton}}
    \end{choices}
    \end{multicols}
\end{question}
}

\element{schaums-mc}{
\begin{question}{ch14-Q13}
    A boy holding on to the end of a rope swings back and forth once every \SI{4.0}{\second}. 
    The length of the rope:
    \begin{choices}
        \wrongchoice{is \SI{2.5}{\meter}}
      \correctchoice{is \SI{4.0}{\meter}}
        \wrongchoice{is \SI{6.2}{\meter}}
        \wrongchoice{cannot be found without knowing the boy's mass}
    \end{choices}
\end{question}
}


\endinput

