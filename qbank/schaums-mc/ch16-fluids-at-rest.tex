
%%--------------------------------------------------
%% Schaum's Outline of Applied Physics
%%--------------------------------------------------


%% Chapter 16: Fluids at Rest
%%--------------------------------------------------


%% Schaum's Multiple Choice Questions
%%--------------------------------------------------
\element{schaums-mc}{
\begin{question}{ch16-Q01}
    The pressure on the bottom of a vessel filled with liquid does not depend upon the:
    \begin{choices}
        \wrongchoice{height of the liquid}
      \correctchoice{area of the liquid surface}
        \wrongchoice{liquid density}
        \wrongchoice{acceleration of gravity}
    \end{choices}
\end{question}
}

\element{schaums-mc}{
\begin{question}{ch16-Q02}
    Buoyancy occurs because,
        as the depth in a fluid increases, the:
    \begin{choices}
        \wrongchoice{density increases}
        \wrongchoice{density decreases}
      \correctchoice{pressure increases}
        \wrongchoice{pressure decreases}
    \end{choices}
\end{question}
}

\element{schaums-mc}{
\begin{questionmult}{ch16-Q03}
    The pressure of the atmosphere at sea level corresponds to which one or more of the following?
    \begin{multicols}{2}
    \begin{choices}
        \wrongchoice{\SI{1.013}{\pascal}}
        \wrongchoice{\SI{98}{\newton\per\meter\squared}}
      \correctchoice{\SI{1013}{\milli\bar}}
      \correctchoice{\SI{14.7}{\pound\per\inch\squared}}
    \end{choices}
    \end{multicols}
\end{questionmult}
}

\element{schaums-mc}{
\begin{question}{ch16-Q04}
    The output piston of a hydraulic press cannot exceed the input piston's:
    \begin{multicols}{2}
    \begin{choices}
        \wrongchoice{displacement}
        \wrongchoice{velocity}
      \correctchoice{work}
        \wrongchoice{force}
    \end{choices}
    \end{multicols}
\end{question}
}

\element{schaums-mc}{
\begin{question}{ch16-Q05}
    A \SI{2}{\newton} stone dropped into a lake sinks. 
    The buoyant force on the stone is:
    \begin{choices}
        \wrongchoice{zero}
      \correctchoice{less than \SI{2}{\newton}}
        \wrongchoice{\SI{2}{\newton}}
        \wrongchoice{more than \SI{2}{\newton}}
    \end{choices}
\end{question}
}

\element{schaums-mc}{
\begin{question}{ch16-Q06}
    Seawater has a density 3 percent greater than that of freshwater. 
    A boat will float:
    \begin{choices}
      \correctchoice{higher in seawater than in freshwater}
        \wrongchoice{at the same level in seawater and in freshwater}
        \wrongchoice{lower in seawater than in freshwater}
        \wrongchoice{any of the above, depending on the shape of the boat’s hull}
    \end{choices}
\end{question}
}

\element{schaums-mc}{
\begin{question}{ch16-Q07}
    The density of copper is \SI{8.9e3}{\kilo\gram\per\meter\cubed}.
    The volume of a \SI{15}{\gram} copper bracelet is:
    \begin{multicols}{2}
    \begin{choices}
        \wrongchoice{\SI{0.6}{\centi\meter\cubed}}
      \correctchoice{\SI{1.7}{\centi\meter\cubed}}
        \wrongchoice{\SI{17}{\centi\meter\cubed}}
        \wrongchoice{\SI{134}{\centi\meter\cubed}}
    \end{choices}
    \end{multicols}
\end{question}
}

\element{schaums-mc}{
\begin{question}{ch16-Q08}
    Ethanol has a specific gravity of \num{0.79}. 
    One liter of ethanol weighs:
    \begin{multicols}{2}
    \begin{choices}
        \wrongchoice{\SI{0.79}{\newton}}
        \wrongchoice{\SI{1.3}{\newton}}
      \correctchoice{\SI{7.7}{\newton}}
        \wrongchoice{\SI{12.4}{\newton}}
    \end{choices}
    \end{multicols}
\end{question}
}

\element{schaums-mc}{
\begin{question}{ch16-Q09}
    A gallon of water and a gallon of antifreeze solution weigh,
        respectively, \SI{8.4}{\pound} and \SI{9.2}{\pound}. 
    The antifreeze solution has a specific gravity of:
    \begin{multicols}{2}
    \begin{choices}
        \wrongchoice{\num{0.095}}
        \wrongchoice{\num{0.80}}
        \wrongchoice{\num{0.91}}
      \correctchoice{\num{1.1}}
    \end{choices}
    \end{multicols}
\end{question}
}

\element{schaums-mc}{
\begin{question}{ch16-Q10}
    The air in a vertical cylinder \SI{20}{\centi\meter} in diameter that is open at the top supports a 20-kg piston. 
    The absolute pressure on the air is:
    \begin{multicols}{2}
    \begin{choices}
        \wrongchoice{\SI{0.0064}{\bar}}
        \wrongchoice{\SI{0.062}{\bar}}
        \wrongchoice{\SI{1.019}{\bar}}
      \correctchoice{\SI{1.075}{\bar}}
    \end{choices}
    \end{multicols}
\end{question}
}

\element{schaums-mc}{
\begin{question}{ch16-Q11}
    A restaurant lobster tank is filled with seawater of density \SI{1.03}{\gram\per\centi\meter\cubed} to a depth of \SI{60}{\centi\meter}. 
    If there are no lobsters in the tank,
        the gauge pressure on its bottom is:
    \begin{multicols}{2}
    \begin{choices}
        \wrongchoice{\SI{6.1}{\pascal}}
        \wrongchoice{\SI{0.62}{\kilo\pascal}}
      \correctchoice{\SI{6.1}{\kilo\pascal}}
        \wrongchoice{\SI{107.4}{\kilo\pascal}}
    \end{choices}
    \end{multicols}
\end{question}
}

\element{schaums-mc}{
\begin{question}{ch16-Q12}
    A restaurant lobster tank is filled with seawater of density \SI{1.03}{\gram\per\centi\meter\cubed} to a depth of \SI{60}{\centi\meter}. 
    If there are no lobsters in the tank,
        the gauge pressure on its bottom is:
    %% NOTE: reword
    Six lobsters are put in the tank of Question 16.11 and enough water is removed to keep its depth at \SI{60}{\centi\meter}. 
    The gauge pressure on the tank bottom is now:
    \begin{choices}
        \wrongchoice{smaller}
      \correctchoice{the same}
        \wrongchoice{greater}
        \wrongchoice{it depends on the mass of the lobsters}
    \end{choices}
\end{question}
}

\element{schaums-mc}{
\begin{question}{ch16-Q13}
    A \SI{20}{\gram} spoon is put into a dish filled with water to the brim,
        and \SI{3.0}{\centi\meter\cubed} of water overflows. 
    The weight of the spoon in the water is:
    \begin{multicols}{2}
    \begin{choices}
        \wrongchoice{\SI{0.03}{\newton}}
      \correctchoice{\SI{0.17}{\newton}}
        \wrongchoice{\SI{0.20}{\newton}}
        \wrongchoice{\SI{0.23}{\newton}}
    \end{choices}
    \end{multicols}
\end{question}
}

\element{schaums-mc}{
\begin{question}{ch16-Q14}
    A wooden plank \SI{2.5}{\meter} long, \SI{40}{\centi\meter} wide, and \SI{60}{\milli\meter} thick floats in water with \SI{15}{\milli\meter} of its thickness above the surface. 
    The plank's mass is:
    \begin{multicols}{2}
    \begin{choices}
        \wrongchoice{\SI{15}{\kilo\gram}}
      \correctchoice{\SI{45}{\kilo\gram}}
        \wrongchoice{\SI{60}{\kilo\gram}}
        \wrongchoice{\SI{441}{\kilo\gram}}
    \end{choices}
    \end{multicols}
\end{question}
}

\element{schaums-mc}{
\begin{question}{ch16-Q15}
    A force of \SI{1000}{\pound} is needed to raise a \SI{12.9}{\foot\cubed} concrete block (weight density \SI{140}{\pound\per\foot\cubed}) to the surface of a freshwater lake. 
    The force needed to lift the block out of the water is:
    \begin{multicols}{2}
    \begin{choices}
        \wrongchoice{\SI{700}{\pound}}
        \wrongchoice{\SI{1062}{\pound}}
        \wrongchoice{\SI{1140}{\pound}}
      \correctchoice{\SI{1800}{\pound}}
    \end{choices}
    \end{multicols}
\end{question}
}

\element{schaums-mc}{
\begin{question}{ch16-Q16}
    A hydraulic press has an input piston \SI{10}{\milli\meter} in diameter and an output piston \SI{50}{\milli\meter} in diameter. 
    An input force of \SI{80}{\newton} gives an output force of:
    \begin{multicols}{2}
    \begin{choices}
        \wrongchoice{\SI{3.2}{\newton}}
        \wrongchoice{\SI{16}{\newton}}
        \wrongchoice{\SI{400}{\newton}}
      \correctchoice{\SI{2000}{\newton}}
    \end{choices}
    \end{multicols}
\end{question}
}

\element{schaums-mc}{
\begin{question}{ch16-Q17}
    A hydraulic press has an input piston \SI{10}{\milli\meter} in diameter and an output piston \SI{50}{\milli\meter} in diameter. 
    An input force of \SI{80}{\newton} gives an output force of:
    %% NOTE: reword
    Pushing the input piston of the press in Question 16.16 through \SI{40}{\milli\meter} causes the output piston to be pushed by:
    \begin{multicols}{2}
    \begin{choices}
      \correctchoice{\SI{1.6}{\milli\meter}}
        \wrongchoice{\SI{8}{\milli\meter}}
        \wrongchoice{\SI{40}{\milli\meter}}
        \wrongchoice{\SI{200}{\milli\meter}}
    \end{choices}
    \end{multicols}
\end{question}
}




\endinput

