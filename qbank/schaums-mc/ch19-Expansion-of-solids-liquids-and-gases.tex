
%%--------------------------------------------------
%% Schaum's Outline of Applied Physics
%%--------------------------------------------------


%% Chapter 19: Expansion of Solids, Liquids, and Gases
%%--------------------------------------------------


%% Schaum's Multiple Choice Questions
%%--------------------------------------------------
\element{schaums-mc}{
\begin{question}{ch19-Q01}
    At what temperature will an iron bar ($a=\SI{1.2e-5}{\per\degreeCelsius}$) be longer by \num{0.10} percent than it is at \SI{20}{\degreeCelsius}?
    \begin{multicols}{2}
    \begin{choices}
        \wrongchoice{\SI{63}{\degreeCelsius}}
        \wrongchoice{\SI{83}{\degreeCelsius}}
      \correctchoice{\SI{103}{\degreeCelsius}}
        \wrongchoice{\SI{120}{\degreeCelsius}}
    \end{choices}
    \end{multicols}
\end{question}
}

\element{schaums-mc}{
\begin{question}{ch19-Q02}
    An aluminum pot whose volume is \SI{1000}{\centi\meter\cubed} at \SI{20}{\degreeCelsius} has a volume of \SI{1006}{\centi\meter\cubed} at \SI{100}{\degreeCelsius}.
    The coefficient of linear expansion of aluminum is:
    \begin{multicols}{2}
    \begin{choices}
      \correctchoice{\SI{2.5e-5}{\per\degreeCelsius}}
        \wrongchoice{\SI{6.0e-5}{\per\degreeCelsius}}
        \wrongchoice{\SI{7.5e-5}{\per\degreeCelsius}}
        \wrongchoice{\SI{2.25e-4}{\per\degreeCelsius}}
    \end{choices}
    \end{multicols}
\end{question}
}

\element{schaums-mc}{
\begin{question}{ch19-Q03}
    An iron bar is heated from \SI{10}{\degreeCelsius} to \SI{100}{\degreeCelsius}.
    \begin{choices}
      \correctchoice{Its volume increases and its density decreases.}
        \wrongchoice{Its volume increases and its density is unchanged.}
        \wrongchoice{Its volume increases and its density increases.}
        \wrongchoice{Its volume is unchanged and its density increases.}
    \end{choices}
\end{question}
}

\element{schaums-mc}{
\begin{question}{ch19-Q04}
    Copper has a coefficient of linear expansion of \SI{1.7e-5}{\per\degreeCelsius}. 
    A copper bar \SI{1.000}{\meter} long at \SI{20}{\degreeCelsius} is cooled until it is \SI{1}{\milli\meter} shorter. 
    The new temperature of the bar is:
    \begin{multicols}{2}
    \begin{choices}
        \wrongchoice{\SI{-17}{\degreeCelsius}}
      \correctchoice{\SI{-35}{\degreeCelsius}}
        \wrongchoice{\SI{-59}{\degreeCelsius}}
        \wrongchoice{\SI{-79}{\degreeCelsius}}
    \end{choices}
    \end{multicols}
\end{question}
}

\element{schaums-mc}{
\begin{question}{ch19-Q05}
    The Celsius equivalent of \SI{200}{\kelvin} is:
    \begin{multicols}{2}
    \begin{choices}
      \correctchoice{\SI{-73}{\degreeCelsius}}
        \wrongchoice{\SI{73}{\degreeCelsius}}
        \wrongchoice{\SI{232}{\degreeCelsius}}
        \wrongchoice{\SI{473}{\degreeCelsius}}
    \end{choices}
    \end{multicols}
\end{question}
}

\element{schaums-mc}{
\begin{question}{ch19-Q06}
    The temperature of an object increases by \SI{50}{\degreeCelsius}. 
    The increase in its absolute temperature is:
    \begin{multicols}{2}
    \begin{choices}
        \wrongchoice{\SI{28}{\kelvin}}
      \correctchoice{\SI{50}{\kelvin}}
        \wrongchoice{\SI{90}{\kelvin}}
        \wrongchoice{\SI{323}{\kelvin}}
    \end{choices}
    \end{multicols}
\end{question}
}

\element{schaums-mc}{
\begin{question}{ch19-Q07}
    An object's temperature is raised by \SI{100}{\degreeCelsius}. 
    The resulting increase in its absolute temperature is:
    \begin{multicols}{2}
    \begin{choices}
        \wrongchoice{\SI{32}{\kelvin}}
      \correctchoice{\SI{100}{\kelvin}}
        \wrongchoice{\SI{180}{\kelvin}}
        \wrongchoice{\SI{373}{\kelvin}}
    \end{choices}
    \end{multicols}
\end{question}
}

\element{schaums-mc}{
\begin{question}{ch19-Q08}
    Nitrogen boils at \SI{-320}{\degree\Fahrenheit}. 
    On the Rankine scale this temperature is:
    \begin{multicols}{2}
    \begin{choices}
        \wrongchoice{\SI{-47}{\degree\Rankine}}
        \wrongchoice{\SI{108}{\degree\Rankine}}
      \correctchoice{\SI{140}{\degree\Rankine}}
        \wrongchoice{\SI{172}{\degree\Rankine}}
    \end{choices}
    \end{multicols}
\end{question}
}

\element{schaums-mc}{
\begin{question}{ch19-Q09}
    The pressure and absolute temperature of a gas sample whose volume is fixed are related by which one or more of the following formulas?
    \begin{multicols}{2}
    \begin{choices}
        \wrongchoice{$\dfrac{P_1}{T_2} = \dfrac{P_2}{T_1}$}
      \correctchoice{$\dfrac{P_1}{T_1} = \dfrac{P_2}{T_2}$}
        \wrongchoice{$\dfrac{P_1}{P_2} = \dfrac{T_2}{T_1}$}
        \wrongchoice{$\dfrac{P_1}{T_1} = \dfrac{T_2}{P_1}$}
    \end{choices}
    \end{multicols}
\end{question}
}

\element{schaums-mc}{
\begin{question}{ch19-Q10}
    At constant temperature the absolute pressure on \SI{10}{\foot\cubed} of air is increased from \SI{20}{\pound\per\inch\squared} to \SI{80}{\pound\per\inch\squared}. 
    The volume of the air is now:
    \begin{multicols}{2}
    \begin{choices}
      \correctchoice{\SI{2.5}{\foot\cubed}}
        \wrongchoice{\SI{5}{\foot\cubed}}
        \wrongchoice{\SI{40}{\foot\cubed}}
        \wrongchoice{\SI{80}{\foot\cubed}}
    \end{choices}
    \end{multicols}
\end{question}
}

\element{schaums-mc}{
\begin{question}{ch19-Q11}
    A sample of oxygen whose volume at \SI{0}{\degreeCelsius} and \SI{200}{\kilo\pascal} of pressure is \SI{4.00}{\liter} is compressed to \SI{1.00}{\liter} and its temperature is raised to \SI{273}{\degreeCelsius}.
    The pressure of the gas is now:
    \begin{multicols}{2}
    \begin{choices}
        \wrongchoice{\SI{100}{\kilo\pascal}}
        \wrongchoice{\SI{400}{\kilo\pascal}}
        \wrongchoice{\SI{800}{\kilo\pascal}}
      \correctchoice{\SI{1600}{\kilo\pascal}}
    \end{choices}
    \end{multicols}
\end{question}
}


\endinput


