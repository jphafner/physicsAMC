
%%--------------------------------------------------
%% Serway: Physics for Scientists and Engineers
%%--------------------------------------------------


%% Chapter 19: Superposition and Standing Waves
%%--------------------------------------------------


%% Table of Contents
%%--------------------------------------------------

%% 19.1 Temperature and the Zeroth Law of Thermodynamics
%% 19.2 Thermometers and the Celsius Temperature Scale
%% 19.3 The Constant-Volume Gas Thermometer and the Absolute Temperature Scale
%% 19.4 Thermal Expansion of Solids and Liquids
%% 19.5 Macroscopic Description of an Ideal Gas


%% Serway Multiple Choice Questions
%%--------------------------------------------------
\element{serway-mc}{
\begin{questionmult}{serway-ch19-q01}
    In order to understand the concept of temperature it is necessary to understand:
    \begin{choices}
      \correctchoice{the zeroth law of thermodynamics.}
        \wrongchoice{the first law of thermodynamics.}
        \wrongchoice{the second law of thermodynamics.}
        %\wrongchoice{all of the above.}
        %\wrongchoice{only (b) and (c) above.}
    \end{choices}
\end{questionmult}
}

\element{serway-mc}{
\begin{questionmult}{serway-ch19-q02}
    In order for two objects to have the same temperature,
        they must:
    \begin{choices}
      \correctchoice{be in thermal equilibrium.}
        \wrongchoice{be in thermal contact with each other.}
        \wrongchoice{have the same relative ``hotness'' or ``coldness'' when touched.}
        %\wrongchoice{have all of the properties listed above.}
        %\wrongchoice{have only properties (b) and (c) above.}
    \end{choices}
\end{questionmult}
}

\element{serway-mc}{
\begin{question}{serway-ch19-q03}
    A pressure of \SI{10}{\milli\meter} \ce{Hg} is measured using a constant-volume gas thermometer at a temperature of \SI{50}{\degreeCelsius}.
    What is the pressure at the zero-point temperature?
    \begin{multicols}{2}
    \begin{choices}
         \wrongchoice{\SI{68.3}{\milli\meter} \ce{Hg}}
         \wrongchoice{\SI{1.8}{\milli\meter} \ce{Hg}}
         \wrongchoice{\SI{31.8}{\milli\meter} \ce{Hg}}
       \correctchoice{\SI{11.8}{\milli\meter} \ce{Hg}}
         \wrongchoice{\SI{8.5}{\milli\meter} \ce{Hg}}
    \end{choices}
    \end{multicols}
\end{question}
}

\element{serway-mc}{
\begin{question}{serway-ch19-q04}
    A pressure of \SI{10}{\milli\meter} \ce{Hg} is measured at the triple-point of water using a constant-volume gas thermometer,
        what will the pressure be at \SI{50}{\degreeCelsius}?
    \begin{multicols}{2}
    \begin{choices}
         \wrongchoice{\SI{31.8}{\milli\meter} \ce{Hg}}
         \wrongchoice{\SI{11.8}{\milli\meter} \ce{Hg}}
       \correctchoice{\SI{8.5}{\milli\meter} \ce{Hg}}
         \wrongchoice{\SI{54.6}{\milli\meter} \ce{Hg}}
         \wrongchoice{\SI{68.3}{\milli\meter} \ce{Hg}}
    \end{choices}
    \end{multicols}
\end{question}
}

\element{serway-mc}{
\begin{question}{serway-ch19-q05}
    A temperature difference of \SI{5}{\kelvin} is equal to:
    \begin{choices}
        \wrongchoice{a difference of \ang{9} on the Celsius scale.}
      \correctchoice{a difference of \ang{9} on the Fahrenheit scale.}
        \wrongchoice{a difference of \ang{2.8} on the Rankine scale.}
        \wrongchoice{a difference of \ang{0.5} on the Fahrenheit scale.}
        \wrongchoice{a difference of \ang{2.8} on the Celsius scale}
    \end{choices}
\end{question}
}

\element{serway-mc}{
\begin{question}{serway-ch19-q06}
    A thermometer registers a change in temperature of \SI{100}{\degree\Fahrenheit}. 
    What change in temperature does this correspond to on the Kelvin Scale?
    \begin{multicols}{3}
    \begin{choices}
        \wrongchoice{\SI{453}{\kelvin}}
        \wrongchoice{\SI{328}{\kelvin}}
        \wrongchoice{\SI{180}{\kelvin}}
      \correctchoice{\SI{55.6}{\kelvin}}
        \wrongchoice{\SI{24.5}{\kelvin}}
    \end{choices}
    \end{multicols}
\end{question}
}

\element{serway-mc}{
\begin{question}{serway-ch19-q07}
    Helium condenses into the liquid phase at approximately \SI{4}{\kelvin}. 
    What temperature,
        in degrees Fahrenheit, does this correspond to?
    \begin{multicols}{3}
    \begin{choices}
        \wrongchoice{\SI{-182}{\degree\Fahrenheit}}
        \wrongchoice{\SI{-269}{\degree\Fahrenheit}}
        \wrongchoice{\SI{-118}{\degree\Fahrenheit}}
      \correctchoice{\SI{-452}{\degree\Fahrenheit}}
        \wrongchoice{\SI{-484}{\degree\Fahrenheit}}
    \end{choices}
    \end{multicols}
\end{question}
}

\element{serway-mc}{
\begin{question}{serway-ch19-q08}
    Two thermometers are calibrated,
        one in degrees Celsius and the other in degrees Fahrenheit. 
    At what temperature do their readings measure the same temperature?
    \begin{multicols}{2}
    \begin{choices}
        \wrongchoice{\SI{218.15}{\kelvin}}
      \correctchoice{\SI{233.15}{\kelvin}}
        \wrongchoice{\SI{273.15}{\kelvin}}
        \wrongchoice{\SI{40.15}{\kelvin}}
        \wrongchoice{\SI{0}{\kelvin}}
    \end{choices}
    \end{multicols}
\end{question}
}

\element{serway-mc}{
\begin{question}{serway-ch19-q09}
    A child has a temperature of \SI{104}{\degree\Fahrenheit}. 
    What is the temperature in degrees kelvin?
    \begin{multicols}{3}
    \begin{choices}
        \wrongchoice{\SI{40}{\kelvin}}
        \wrongchoice{\SI{406}{\kelvin}}
        \wrongchoice{\SI{401}{\kelvin}}
      \correctchoice{\SI{313}{\kelvin}}
        \wrongchoice{\SI{349}{\kelvin}}
    \end{choices}
    \end{multicols}
\end{question}
}

\element{serway-mc}{
\begin{question}{serway-ch19-q10}
    At what temperature is the Celsius scale reading equal to twice the Fahrenheit scale reading?
    \begin{multicols}{3}
    \begin{choices}
      \correctchoice{\SI{-12.3}{\degree\Fahrenheit}}
        \wrongchoice{\SI{-24.6}{\degree\Fahrenheit}}
        \wrongchoice{\SI{-12.3}{\degreeCelsius}}
        \wrongchoice{\SI{-6.1}{\degreeCelsius}}
        \wrongchoice{\SI{-20}{\degree\Fahrenheit}}
    \end{choices}
    \end{multicols}
\end{question}
}

\element{serway-mc}{
\begin{question}{serway-ch19-q11}
    A bridge is made with segments of concrete \SI{50}{\meter} long. 
    If the linear expansion coefficient is \SI{12e-6}{\per\degreeCelsius},
        how much spacing is needed to allow for expansion during an extreme temperature change of \SI{150}{\degree\Fahrenheit}?
    \begin{multicols}{3}
    \begin{choices}
        \wrongchoice{\SI{10}{\centi\meter}}
        \wrongchoice{\SI{2.5}{\centi\meter}}
        \wrongchoice{\SI{7.5}{\centi\meter}}
      \correctchoice{\SI{5.0}{\centi\meter}}
        \wrongchoice{\SI{9.5}{\centi\meter}}
    \end{choices}
    \end{multicols}
\end{question}
}

\element{serway-mc}{
\begin{question}{serway-ch19-q12}
    A building made with a steel structure is \SI{650}{\meter} high on a winter day when the temperature is \SI{0}{\degree\Fahrenheit}.
    How much taller is the building when it is \SI{100}{\degree\Fahrenheit}?
    (The linear expansion coefficient of steel is \SI{11e-6}{\per\degreeCelsius}.)
    \begin{multicols}{3}
    \begin{choices}
        \wrongchoice{\SI{71}{\centi\meter}}
        \wrongchoice{\SI{36}{\centi\meter}}
      \correctchoice{\SI{40}{\centi\meter}}
        \wrongchoice{\SI{46}{\centi\meter}}
        \wrongchoice{\SI{65}{\centi\meter}}
    \end{choices}
    \end{multicols}
\end{question}
}

\element{serway-mc}{
\begin{question}{serway-ch19-q13}
    A gallon container is filled with gasoline. 
    How many gallons are lost if the temperature increases by \SI{25}{\degree\Fahrenheit}?
    (The volume expansion of gasoline is \SI{9.6e-4}{\per\degreeCelsius}.)
    (Neglect the change in volume of the container.)
    \begin{multicols}{2}
    \begin{choices}
        \wrongchoice{\SI{2.4e-2}{\gallon}}
      \correctchoice{\SI{1.3e-2}{\gallon}}
        \wrongchoice{\SI{3.6e-2}{\gallon}}
        \wrongchoice{\SI{4.8e-2}{\gallon}}
        \wrongchoice{\SI{9.6e-2}{\gallon}}
    \end{choices}
    \end{multicols}
\end{question}
}

\element{serway-mc}{
\begin{question}{serway-ch19-q14}
    An auditorium has dimensions $\SI{10}{\meter}\times\SI{10}{\meter}\times\SI{60}{\meter}$.
    How many moles of air fill this volume at STP?
    \begin{multicols}{2}
    \begin{choices}
        \wrongchoice{\SI{2.7e2}{\mole}}
        \wrongchoice{\SI{2.7e4}{\mole}}
        \wrongchoice{\SI{2.7e3}{\mole}}
      \correctchoice{\SI{2.7e5}{\mole}}
        \wrongchoice{\SI{2.7e6}{\mole}}
    \end{choices}
    \end{multicols}
\end{question}
}

\element{serway-mc}{
\begin{question}{serway-ch19-q15}
    An auditorium has a volume of \SI{6e3}{\meter\cubed}. 
    How many molecules of air are needed to fill the auditorium at STP?
    \begin{multicols}{2}
    \begin{choices}
      \correctchoice{\num{1.6e29}}
        \wrongchoice{\num{1.6e27}}
        \wrongchoice{\num{1.6e25}}
        \wrongchoice{\num{1.6e23}}
        \wrongchoice{\num{1.6e20}}
    \end{choices}
    \end{multicols}
\end{question}
}

\element{serway-mc}{
\begin{question}{serway-ch19-q16}
    One mole of an ideal gas is held at a constant pressure of \SI{1}{\atm}. 
    Find the change in volume if the temperature changes by \SI{50}{\degreeCelsius}.
    \begin{multicols}{3}
    \begin{choices}
        \wrongchoice{\SI{1}{\liter}}
        \wrongchoice{\SI{2}{\liter}}
        \wrongchoice{\SI{3}{\liter}}
      \correctchoice{\SI{4}{\liter}}
        \wrongchoice{\SI{5}{\liter}}
    \end{choices}
    \end{multicols}
\end{question}
}

\element{serway-mc}{
\begin{question}{serway-ch19-q17}
    One mole of an ideal gas is held at a constant volume of 1 liter. 
    Find the change in pressure if the temperature increases by \SI{50}{\degreeCelsius}.
    \begin{multicols}{3}
    \begin{choices}
        \wrongchoice{\SI{3}{\atm}}
      \correctchoice{\SI{4}{\atm}}
        \wrongchoice{\SI{2}{\atm}}
        \wrongchoice{\SI{1}{\atm}}
        \wrongchoice{\SI{5}{\atm}}
    \end{choices}
    \end{multicols}
\end{question}
}

\element{serway-mc}{
\begin{question}{serway-ch19-q18}
    One mole of an ideal gas has a temperature of \SI{25}{\degreeCelsius}. 
    If the volume is held constant and the pressure is doubled,
        the final temperature will be
    \begin{multicols}{3}
    \begin{choices}
        \wrongchoice{\SI{174}{\degreeCelsius}}
        \wrongchoice{\SI{596}{\degreeCelsius}}
        \wrongchoice{\SI{50}{\degreeCelsius}}
      \correctchoice{\SI{323}{\degreeCelsius}}
        \wrongchoice{\SI{25}{\degreeCelsius}}
    \end{choices}
    \end{multicols}
\end{question}
}

\element{serway-mc}{
\begin{question}{serway-ch19-q19}
    A bicycle pump contains air at STP. 
    As the tire is pumped up,
        the volume of air decreases by \SI{50}{\percent} with each stroke. 
    What is the new pressure of air in the chamber after the first stroke,
        assuming no temperature change?
    \begin{multicols}{3}
    \begin{choices}
      \correctchoice{\SI{2}{\atm}}
        \wrongchoice{\SI{1}{\atm}}
        \wrongchoice{\SI{0.5}{\atm}}
        \wrongchoice{\SI{0.1}{\atm}}
        \wrongchoice{\SI{3}{\atm}}
    \end{choices}
    \end{multicols}
\end{question}
}

\element{serway-mc}{
\begin{question}{serway-ch19-q20}
    A helium-filled balloon has a volume of \SI{1}{\meter\cubed}. 
    As it rises in the earth's atmosphere,
        its volume expands. 
    What will its new volume be if its original temperature and pressure are \SI{20}{\degreeCelsius} and \SI{1}{\atm},
        and its final temperature and pressure are \SI{-40}{\degreeCelsius} and \SI{0.1}{\atm}?
    \begin{multicols}{3}
    \begin{choices}
        \wrongchoice{\SI{4}{\meter\cubed}}
        \wrongchoice{\SI{6}{\meter\cubed}}
      \correctchoice{\SI{8}{\meter\cubed}}
        \wrongchoice{\SI{10}{\meter\cubed}}
        \wrongchoice{\SI{1.5}{\meter\cubed}}
    \end{choices}
    \end{multicols}
\end{question}
}

\element{serway-mc}{
\begin{question}{serway-ch19-q21}
    A bubble having a diameter of \SI{1.00}{\centi\meter} is released from the bottom of a swimming pool where the depth is \SI{5.00}{\meter}. 
    What will the diameter of the bubble be when it reaches the surface? 
    The temperature of the water at the surface is \SI{20.0}{\degreeCelsius},
        whereas it is \SI{15.0}{\degreeCelsius} at the bottom. 
    (The density of water is \SI{1.00e3}{\kilo\gram\per\meter\cubed}.)
    \begin{multicols}{3}
    \begin{choices}
        \wrongchoice{\SI{1.05}{\centi\meter}}
      \correctchoice{\SI{1.15}{\centi\meter}}
        \wrongchoice{\SI{1.45}{\centi\meter}}
        \wrongchoice{\SI{1.65}{\centi\meter}}
        \wrongchoice{\SI{1.35}{\centi\meter}}
    \end{choices}
    \end{multicols}
\end{question}
}

\element{serway-mc}{
\begin{question}{serway-ch19-q22}
    A scuba diver has his lungs filled to half capacity (\SI{3}{\liter}) when \SI{10}{\meter} below the surface. 
    If the diver holds his breath while quietly rising to the surface,
        what will the volume of the lungs be at the surface? 
    Assume the temperature is the same at all depths. 
    (The density of water is \SI{1.0e3}{\kilo\gram\per\meter\cubed}.)
    \begin{multicols}{3}
    \begin{choices}
      \correctchoice{\SI{5.9}{\liter}}
        \wrongchoice{\SI{4.5}{\liter}}
        \wrongchoice{\SI{6.4}{\liter}}
        \wrongchoice{\SI{3.9}{\liter}}
        \wrongchoice{\SI{3.1}{\liter}}
    \end{choices}
    \end{multicols}
\end{question}
}

\element{serway-mc}{
\begin{question}{serway-ch19-q23}
    Two identical containers, $A$ and $B$,
        hold equal amounts of the same ideal gas at the same $P_o$, $V_o$ and $T_o$.
    The pressure of $A$ then decreases by a half while its volume doubles;
        the pressure of $B$ doubles while its volume decreases by a half.
    Which statement correctly describes the temperatures of the gases after the changes?
    \begin{choices}
        \wrongchoice{$T_A = 0.5 T_B = T_o$}
        \wrongchoice{$T_B = 0.5 T_A = T_o$}
      \correctchoice{$T_B = T_A = T_o$}
        \wrongchoice{$T_A = 2 T_B = T_o$}
        \wrongchoice{$T_B = 2 T_A = T_o$}
    \end{choices}
\end{question}
}

\element{serway-mc}{
\begin{question}{serway-ch19-q24}
    Which of the following is not a possible thermometric property of a body?
    \begin{choices}
        \wrongchoice{The change in length of a solid.}
        \wrongchoice{The change in volume of a gas at constant pressure.}
        \wrongchoice{The change in pressure of a gas at constant volume.}
      \correctchoice{The change in weight at constant pressure and volume.}
        \wrongchoice{The change in electrical resistance of a conductor.}
    \end{choices}
\end{question}
}

\element{serway-mc}{
\begin{question}{serway-ch19-q25}
    A pebble size object and a bowling ball size probe from a spaceship land on a large asteroid that is far from any star. 
    After a long period of time has passed,
        it is highly probable that the pebble and the probe:
    \begin{choices}
        \wrongchoice{have each had the same change in temperature.}
        \wrongchoice{have each had the same change in volume.}
      \correctchoice{are in thermal equilibrium with one another.}
        \wrongchoice{are not in thermal equilibrium with one another.}
        \wrongchoice{are in thermal equilibrium with one another, but are not at the same temperature.}
    \end{choices}
\end{question}
}

\element{serway-mc}{
\begin{question}{serway-ch19-q26}
    A temperature difference of 9 Celsius degrees is equal to a Fahrenheit temperature difference of:
    \begin{choices}
        %% NOTE: Reformat for multicols
        \wrongchoice{5 Fahrenheit degrees.}
        \wrongchoice{9 Fahrenheit degrees.}
      \correctchoice{16 Fahrenheit degrees.}
        \wrongchoice{37 Fahrenheit degrees.}
        \wrongchoice{41 Fahrenheit degrees.}
    \end{choices}
\end{question}
}

\element{serway-mc}{
\begin{question}{serway-ch19-q27}
    Death Valley in California receives many German tourists. 
    When you convert a summer temperature reading of \SI{130}{\degree\Fahrenheit} to the Celsius scale they use at home,
        you find that the Celsius temperature is:
    \begin{multicols}{3}
    \begin{choices}
        \wrongchoice{\SI{26}{\degreeCelsius}}
      \correctchoice{\SI{54}{\degreeCelsius}}
        \wrongchoice{\SI{72}{\degreeCelsius}}
        \wrongchoice{\SI{176}{\degreeCelsius}}
        \wrongchoice{\SI{327}{\degreeCelsius}}
    \end{choices}
    \end{multicols}
\end{question}
}

\element{serway-mc}{
\begin{question}{serway-ch19-q28}
    A beaker is filled to the \SI{500}{\milli\liter} mark with alcohol. 
    What increase in volume the beaker contain when the temperature changes from \SI{5}{\degreeCelsius} to \SI{30}{\degreeCelsius}?
    (Neglect the expansion of the beaker,
        evaporation of alcohol and absorption of water vapor by alcohol.) 
    $\beta_{\text{alcohol}} = \SI{1.12e-4}{\per\degreeCelsius}$
    \begin{multicols}{3}
    \begin{choices}
        \wrongchoice{\SI{0.47}{\milli\liter}}
        \wrongchoice{\SI{0.93}{\milli\liter}}
      \correctchoice{\SI{1.4}{\milli\liter}}
        \wrongchoice{\SI{1.7}{\milli\liter}}
        \wrongchoice{\SI{2.5}{\milli\liter}}
    \end{choices}
    \end{multicols}
\end{question}
}

\element{serway-mc}{
\begin{question}{serway-ch19-q29}
    What is the change in area of a \SI{60.0}{\centi\meter} by \SI{150}{\centi\meter} automobile windshield when the temperature changes from \SI{0}{\degreeCelsius} to \SI{36.0}{\degreeCelsius}. 
    The coefficient of linear expansion of glass is \SI{9e-6}{\per\degreeCelsius}.
    \begin{multicols}{3}
    \begin{choices}
        \wrongchoice{\SI{1.62}{\centi\meter\squared}}
        \wrongchoice{\SI{2.92}{\centi\meter\squared}}
        \wrongchoice{\SI{3.24}{\centi\meter\squared}}
        \wrongchoice{\SI{4.86}{\centi\meter\squared}}
      \correctchoice{\SI{5.83}{\centi\meter\squared}}
    \end{choices}
    \end{multicols}
\end{question}
}

\element{serway-mc}{
\begin{question}{serway-ch19-q30}
    A container with a one liter capacity at \SI{27}{\degreeCelsius} is filled with helium to a pressure of \SI{2}{\atm}. 
    ($\SI{1}{\atm}=\SI{1.0e6}{\newton\per\meter\squared}$.)
    How many moles of helium does it hold?
    \begin{multicols}{3}
    \begin{choices}
        \wrongchoice{\SI{0.040}{\mole}}
      \correctchoice{\SI{0.080}{\mole}}
        \wrongchoice{\SI{0.45}{\mole}}
        \wrongchoice{\SI{0.90}{\mole}}
        \wrongchoice{\SI{1.0}{\mole}}
    \end{choices}
    \end{multicols}
\end{question}
}

\element{serway-mc}{
\begin{question}{serway-ch19-q31}
    Two bodies can be in thermal equilibrium with one another when they are at the same temperature even if they
    \begin{choices}
        \wrongchoice{absorb different quantities of thermal energy from their surroundings in equal time intervals.}
        \wrongchoice{have different masses.}
        \wrongchoice{have different volumes.}
      \correctchoice{have any of the properties listed above.}
        \wrongchoice{have any of the properties listed above and one of them is contact with a third body at a different temperature.}
    \end{choices}
\end{question}
}

\element{serway-mc}{
\begin{question}{serway-ch19-q32}
    Angela claims that she wears a cylindrical-shaped hollow gold bracelet because it expands less than a solid one with a change in temperature. 
    Clarissa claims that a cylindrical-shaped solid gold bracelet expands less than a hollow one.
    Which one, if either, is correct?
    \begin{choices}
        \wrongchoice{Angela, because the bracelet expands outward on its outer surface and inward on its inner surface.}
        \wrongchoice{Clarissa, because the bracelet expands outward on its outer surface and inward on its inner surface.}
        \wrongchoice{Angela, because the inner circumference does not change, but the outer circumference expands.}
        \wrongchoice{Clarissa, because the inner circumference does not change, but the outer circumference expands.}
      \correctchoice{Neither, because both the inner and outer circumferences increase in length.}
    \end{choices}
\end{question}
}

\element{serway-mc}{
\begin{questionmult}{serway-ch19-q33}
    A student has written the equation below to convert a temperature in degrees Fahrenheit into Kelvins.
    \begin{equation*}
        T_K = \dfrac{9}{5} T_F + 32
    \end{equation*}
    What is wrong with this equation?
    \begin{choices}
      \correctchoice{The factor in front of $T_F$ should be $\dfrac{5}{9}$}
      \correctchoice{The numerical factor $\dfrac{5}{9}$ should multiply $\left(T_F - 32\right)$.}
      \correctchoice{An additional 273.15 Kelvins must be added to the right side of the equation.}
        %\correctchoice{All the corrections above are required.}
        %\wrongchoice{Only corrections (b) and (c) are required.}
    \end{choices}
\end{questionmult}
}

\element{serway-mc}{
\begin{questionmult}{serway-ch19-q34}
    Two moles of an ideal gas are placed in a container of adjustable volume. 
    When measurements are made:
    \begin{choices}
      \correctchoice{the pressure is inversely proportional to the volume at constant temperature.}
      \correctchoice{the temperature is directly proportional to the volume at constant pressure.}
      \correctchoice{the temperature is directly proportional to the pressure at constant volume.}
        %\correctchoice{all the statements above are found to be correct.}
        %\wrongchoice{only statements (a) and (b) are found to be correct.}
    \end{choices}
\end{questionmult}
}

\element{serway-mc}{
\begin{question}{serway-ch19-q35}
    When the coefficient of linear expansion, $\alpha$,
        and the temperature change, $T_f-T_i$,
        are large, a length $L_i$ of a solid substance expands in length to:
    \begin{choices}
        \wrongchoice{$L_f = \alpha L_i \left(T_f - T_i\right)$}
        \wrongchoice{$L_f = L_i \left[ 1 + \alpha \left(T_f - T_i\right) \right]$}
        \wrongchoice{$L_f = L_i \left[ 1 + \ln\left(\alpha\left(T_f - T_i\right)\right) \right]$}
      \correctchoice{$L_f = L_i \mathrm{e}^{\alpha\left(T_f - T_i\right)}$}
        \wrongchoice{$L_f = L_i \left[ 1 + \mathrm{e}^{\alpha\left(T_f-T_i\right)}\right]$}
    \end{choices}
\end{question}
}

\element{serway-mc}{
\begin{question}{serway-ch19-q36}
    A square plate has an area of \SI{29.00}{\centi\meter\squared} at \SI{20.0}{\degreeCelsius}.
    It will be used in a low temperature experiment at $T=\SI{10.0}{\kelvin}$ where it must have an area of \SI{28.00}{\centi\meter\squared}.
    What area must be removed form the plate at \SI{20.0}{\degreeCelsius} for it to have the correct area at \SI{10.0}{\kelvin}?
    (The coefficient of linear expansion is \SI{10e-6}{\per\degreeCelsius}.)
    \begin{multicols}{2}
    \begin{choices}
        \wrongchoice{\SI{0.0793}{\centi\meter\squared}}
        \wrongchoice{\SI{0.159}{\centi\meter\squared}}
        \wrongchoice{\SI{0.238}{\centi\meter\squared}}
      \correctchoice{\SI{0.841}{\centi\meter\squared}}
        \wrongchoice{\SI{0.921}{\centi\meter\squared}}
    \end{choices}
    \end{multicols}
\end{question}
}

\element{serway-mc}{
\begin{question}{serway-ch19-q37}
    Equal volumes of hydrogen and helium gas are at the same pressure. 
    The gram molecular mass of helium is four times that of hydrogen. 
    If the total mass of both gases is the same,
        the ratio of the temperature of helium (\ce{He}) to that of hydrogen (\ce{H2}) is:
    \begin{multicols}{3}
    \begin{choices}
        \wrongchoice{\num{1/4}}
      \correctchoice{\num{1/2}}
        \wrongchoice{\num{1}}
        \wrongchoice{\num{2}}
        \wrongchoice{\num{4}}
    \end{choices}
    \end{multicols}
\end{question}
}

\element{serway-mc}{
\begin{question}{serway-ch19-q38}
    Equal masses of hydrogen and helium gas are at the same temperature in vessels of equal volume. 
    The gram molecular mass of helium is four times that of hydrogen. 
    If the total mass of both gases is the same,
        the ratio of the pressure of helium (\ce{He}) to that of hydrogen (\ce{H2}) is:
    \begin{multicols}{3}
    \begin{choices}
        \wrongchoice{\num{1/4}}
      \correctchoice{\num{1/2}}
        \wrongchoice{\num{1}}
        \wrongchoice{\num{2}}
        \wrongchoice{\num{4}}
    \end{choices}
    \end{multicols}
\end{question}
}

\element{serway-mc}{
\begin{question}{serway-ch19-q39}
    Steel blocks A and B, which have equal masses,
        are at $T_A=\SI{300}{\degreeCelsius}$ and $T_B=\SI{400}{\degreeCelsius}$.
    Block $C$, with $m_C=2m_A$,
        is at $T_C=\SI{350}{\degreeCelsius}$. 
    Blocks $A$ and $B$ are placed in contact,
        isolated, and allowed to come into equilibrium. 
    Then they are placed in contact with block $C$.
    At that instant,
    \begin{multicols}{2}
    \begin{choices}
        \wrongchoice{$T_A = T_B < T_C$}
      \correctchoice{$T_A = T_B = T_C$}
        \wrongchoice{$T_A = T_B > T_C$}
        \wrongchoice{$T_A + T_B = T_C$}
        \wrongchoice{$T_A - T_B = T_C$}
    \end{choices}
    \end{multicols}
\end{question}
}


\endinput


