
%%--------------------------------------------------
%% Serway: Physics for Scientists and Engineers
%%--------------------------------------------------


%% NOTE: provide threshold of hearing in REF.tex
%%--------------------------------------------------


%% Chapter 17: Sound Waves
%%--------------------------------------------------


%% Table of Contents
%%--------------------------------------------------

%% 17.1 Speed of Sound Waves
%% 17.2 Periodic Sound Waves
%% 17.3 Intensity of Periodic Sound Waves
%% 17.4 The Doppler Effect
%% 17.5 Digital Sound Recording
%% 17.6 Motion Picture Sound


%% Serway Multiple Choice Questions
%%--------------------------------------------------
\element{serway-mc}{
\begin{question}{serway-ch17-q01}
    The velocity of sound in sea water is \SI{1533}{\meter\per\second}.
    Find the bulk modulus of sea water if its density is \SI{1.025e3}{\kilo\gram\per\meter\cubed}.
    \begin{multicols}{2}
    \begin{choices}
        \wrongchoice{\SI{2.6e9}{\newton\per\meter\squared}}
        \wrongchoice{\SI{2.2e9}{\newton\per\meter\squared}}
        \wrongchoice{\SI{2.0e9}{\newton\per\meter\squared}}
      \correctchoice{\SI{2.4e9}{\newton\per\meter\squared}}
        \wrongchoice{\SI{2.8e9}{\newton\per\meter\squared}}
    \end{choices}
    \end{multicols}
\end{question}
}

\element{serway-mc}{
\begin{question}{serway-ch17-q02}
    A sculptor strikes a piece of marble with a hammer.
    Find the speed of sound through the marble.
    (The Young's modulus is \SI{50e9}{\newton\per\meter\squared}
        and its density is \SI{2.7e3}{\kilo\gram\per\meter\cubed}.)
    \begin{multicols}{2}
    \begin{choices}
        \wrongchoice{\SI{5.1}{\kilo\meter\per\second}}
      \correctchoice{\SI{4.3}{\kilo\meter\per\second}}
        \wrongchoice{\SI{3.5}{\kilo\meter\per\second}}
        \wrongchoice{\SI{1.3}{\kilo\meter\per\second}}
        \wrongchoice{\SI{1.8}{\kilo\meter\per\second}}
    \end{choices}
    \end{multicols}
\end{question}
}

\element{serway-mc}{
\begin{question}{serway-ch17-q03}
    The Young's modulus for aluminum is \SI{7.02e10}{\newton\per\meter\squared}.
    If the speed of sound in aluminum is measured to be \SI{5.10}{\kilo\meter\per\second},
        find its density.
    \begin{multicols}{2}
    \begin{choices}
        \wrongchoice{\SI{11.3e3}{\kilo\gram\per\meter\cubed}}
        \wrongchoice{\SI{7.80e3}{\kilo\gram\per\meter\cubed}}
      \correctchoice{\SI{2.70e3}{\kilo\gram\per\meter\cubed}}
        \wrongchoice{\SI{29.3e3}{\kilo\gram\per\meter\cubed}}
        \wrongchoice{\SI{1.40e3}{\kilo\gram\per\meter\cubed}}
    \end{choices}
    \end{multicols}
\end{question}
}

\element{serway-mc}{
\begin{question}{serway-ch17-q04}
    It is possible to hear an approaching train before you can see it by listening to the sound wave through the track.
    If the elastic modulus is \SI{2.0e11}{\newton\per\meter\squared} and the density of steel is \SI{7.8e3}{\kilo\gram\per\meter\cubed},
        approximately how many times faster is the speed of sound in the track than in air?
    ($v_{\text{air}}\approx\SI{340}{\meter\per\second}$.)
    \begin{multicols}{3}
    \begin{choices}
        \wrongchoice{\num{20}}
        \wrongchoice{\num{5}}
        \wrongchoice{\num{10}}
      \correctchoice{\num{15}}
        \wrongchoice{\num{25}}
    \end{choices}
    \end{multicols}
\end{question}
}

\element{serway-mc}{
\begin{question}{serway-ch17-q05}
    A harmonic longitudinal wave propagating down a tube filled with a compressible gas has the form $s(x,t)= s_{\mathrm{m}}\cos\left(kx-\omega t\right)$.
    Its velocity can be obtained from
    \begin{multicols}{3}
    \begin{choices}
      \correctchoice{$\omega$}
        \wrongchoice{$\omega k$}
        \wrongchoice{$k$}
        \wrongchoice{$\dfrac{\omega}{k}$}
        \wrongchoice{$\dfrac{k}{\omega}$}
    \end{choices}
    \end{multicols}
\end{question}
}

\element{serway-mc}{
\begin{question}{serway-ch17-q06}
    Calculate the pressure amplitude of a \SI{500}{\hertz}
        sound wave in helium if the displacement amplitude is equal to \SI{5e-8}{\meter}.
    ($\rho=\SI{0.179}{\kilo\gram\per\meter\cubed}$, $v=\SI{972}{\meter\per\second}$.)
    \begin{multicols}{2}
    \begin{choices}
        \wrongchoice{\SI{3.5e-2}{\newton\per\meter\squared}}
        \wrongchoice{\SI{1.6e-2}{\newton\per\meter\squared}}
      \correctchoice{\SI{2.7e-2}{\newton\per\meter\squared}}
        \wrongchoice{\SI{4.2e-2}{\newton\per\meter\squared}}
        \wrongchoice{\SI{2.0e-2}{\newton\per\meter\squared}}
    \end{choices}
    \end{multicols}
\end{question}
}

\element{serway-mc}{
\begin{question}{serway-ch17-q07}
    Calculate the displacement amplitude of a \SI{20}{\kilo\hertz}
        sound wave in helium if it has a pressure amplitude of \SI{8e-3}{\newton\per\meter\squared}. ($\rho=\SI{0.179}{\kilo\gram\per\meter\cubed}$, $v=\SI{972}{\meter\per\second}$.)
    \begin{multicols}{2}
    \begin{choices}
        \wrongchoice{\SI{2.9e-10}{\meter}}
      \correctchoice{\SI{3.7e-10}{\meter}}
        \wrongchoice{\SI{7.8e-9}{\meter}}
        \wrongchoice{\SI{2.4e-9}{\meter}}
        \wrongchoice{\SI{1.9e-10}{\meter}}
    \end{choices}
    \end{multicols}
\end{question}
}

\element{serway-mc}{
\begin{question}{serway-ch17-q08}
    The variation in the pressure of helium gas,
        measured from its equilibrium value,
        is given by $\Delta P=\num{2.9e-5}\cos\left(6.2x-3000t\right)$ where $x$ and $t$ have units \si{\meter} and \si{\second},
        and $\Delta P$ is measured in \si{\newton\per\meter\squared}.
    Determine the frequency of the wave.
    \begin{multicols}{2}
    \begin{choices}
        \wrongchoice{\SI{1500}{\hertz}}
      \correctchoice{\SI{477}{\hertz}}
        \wrongchoice{\SI{1.01}{\hertz}}
        \wrongchoice{\SI{0.32}{\hertz}}
        \wrongchoice{\SI{239}{\hertz}}
    \end{choices}
    \end{multicols}
\end{question}
}

\element{serway-mc}{
\begin{question}{serway-ch17-q09}
    The variation in the pressure of helium gas,
        measured from its equilibrium value,
        is given by $\Delta P=\num{2.9e-5}\cos\left(6.2x-3000t\right)$ where $x$ and $t$ have units \si{\meter} and \si{\second},
        and $\Delta P$ is measured in \si{\newton\per\meter\squared}.
    Determine the wavelength of the wave.
    \begin{multicols}{3}
    \begin{choices}
        \wrongchoice{\SI{1500}{\meter}}
        \wrongchoice{\SI{0.32}{\meter}}
        \wrongchoice{\SI{477}{\meter}}
      \correctchoice{\SI{1.01}{\meter}}
        \wrongchoice{\SI{0.50}{\meter}}
    \end{choices}
    \end{multicols}
\end{question}
}

\element{serway-mc}{
\begin{question}{serway-ch17-q10}
    The variation in the pressure of helium gas,
        measured from its equilibrium value,
        is given by $\Delta P=\num{2.9e-5}\cos\left(6.2x-3000t\right)$ where $x$ and $t$ have units \si{\meter} and \si{\second}.
    Determine the speed of the wave.
    \begin{multicols}{3}
    \begin{choices}
        \wrongchoice{\SI{1515}{\meter\per\second}}
        \wrongchoice{\SI{153}{\meter\per\second}}
      \correctchoice{\SI{484}{\meter\per\second}}
        \wrongchoice{\SI{828}{\meter\per\second}}
        \wrongchoice{\SI{101}{\meter\per\second}}
    \end{choices}
    \end{multicols}
\end{question}
}

\element{serway-mc}{
\begin{question}{serway-ch17-q11}
    Determine the intensity of a harmonic longitudinal wave with a pressure amplitude of \SI{8e-3}{\newton\per\meter\squared} propagating down a tube filled with helium.
    ($\rho=\SI{0.179}{\kilo\gram\per\meter\cubed}$, $v=\SI{972}{\meter\per\second}$.)
    \begin{multicols}{2}
    \begin{choices}
        \wrongchoice{\SI{3.7e-7}{\watt\per\meter\squared}}
      \correctchoice{\SI{1.8e-7}{\watt\per\meter\squared}}
        \wrongchoice{\SI{9.2e-8}{\watt\per\meter\squared}}
        \wrongchoice{\SI{4.6e-8}{\watt\per\meter\squared}}
        \wrongchoice{\SI{1.5e-9}{\watt\per\meter\squared}}
    \end{choices}
    \end{multicols}
\end{question}
}

\element{serway-mc}{
\begin{question}{serway-ch17-q12}
    Calculate the intensity level in \si{\decibel} a sound wave that has an intensity of \SI{15e-4}{\watt\per\meter\squared}.
    \begin{multicols}{3}
    \begin{choices}
        \wrongchoice{\SI{20}{\decibel}}
        \wrongchoice{\SI{200}{\decibel}}
      \correctchoice{\SI{92}{\decibel}}
        \wrongchoice{\SI{9}{\decibel}}
        \wrongchoice{\SI{10}{\decibel}}
    \end{choices}
    \end{multicols}
\end{question}
}

\element{serway-mc}{
\begin{question}{serway-ch17-q13}
    A jet plane has a sound level of \SI{150}{\decibel}.
    What is the intensity?
    \begin{multicols}{2}
    \begin{choices}
        \wrongchoice{\SI{1}{\watt\per\meter\squared}}
        \wrongchoice{\SI{100}{\watt\per\meter\squared}}
        \wrongchoice{\SI{10}{\watt\per\meter\squared}}
      \correctchoice{\SI{1000}{\watt\per\meter\squared}}
        \wrongchoice{\SI{10000}{\watt\per\meter\squared}}
    \end{choices}
    \end{multicols}
\end{question}
}

\element{serway-mc}{
\begin{question}{serway-ch17-q14}
    By what factor will an intensity change when the corresponding sound level increases by \SI{3}{\decibel}?
    \begin{multicols}{3}
    \begin{choices}
        \wrongchoice{\num{3}}
        \wrongchoice{\num{0.5}}
      \correctchoice{\num{2}}
        \wrongchoice{\num{4}}
        \wrongchoice{\num{0.3}}
    \end{choices}
    \end{multicols}
\end{question}
}

\element{serway-mc}{
\begin{question}{serway-ch17-q15}
    By what factor is the intensity of sound at a rock concert louder than that
        of a whisper when the two intensity levels are \SI{120}{\decibel} and \SI{20}{\decibel} respectively?
    \begin{multicols}{3}
    \begin{choices}
        \wrongchoice{\num{e12}}
        \wrongchoice{\num{e8}}
        \wrongchoice{\num{e6}}
      \correctchoice{\num{e10}}
        \wrongchoice{\num{e11}}
    \end{choices}
    \end{multicols}
\end{question}
}

\element{serway-mc}{
\begin{question}{serway-ch17-q16}
    A point source emits sound with a power output of \SI{100}{\watt}.
    What is the intensity at a distance of \SI{10.0}{\meter} from the source?
    \begin{multicols}{2}
    \begin{choices}
      \correctchoice{\SI{7.96e-2}{\watt\per\meter\squared}}
        \wrongchoice{\SI{7.96e-1}{\watt\per\meter\squared}}
        \wrongchoice{\SI{7.96e0}{\watt\per\meter\squared}}
        \wrongchoice{\SI{7.96e1}{\watt\per\meter\squared}}
        \wrongchoice{\SI{7.96e-3}{\watt\per\meter\squared}}
    \end{choices}
    \end{multicols}
\end{question}
}

\element{serway-mc}{
\begin{question}{serway-ch17-q17}
    A point source emits sound waves with a power output of \SI{100}{\watt}.
    What is the sound level at a distance of \SI{10}{\meter}?
    \begin{multicols}{3}
    \begin{choices}
        \wrongchoice{\SI{139}{\decibel}}
        \wrongchoice{\SI{119}{\decibel}}
        \wrongchoice{\SI{129}{\decibel}}
      \correctchoice{\SI{109}{\decibel}}
        \wrongchoice{\SI{10}{\decibel}}
    \end{choices}
    \end{multicols}
\end{question}
}

\element{serway-mc}{
\begin{question}{serway-ch17-q18}
    A car approaches a stationary police car at \SI{36}{\meter\per\second}.
    The frequency of the siren (relative to the police car) is \SI{500}{\hertz}.
    What is the frequency heard by an observer in the moving car as he approaches the police car?
    (Assume the velocity of sound in air is \SI{343}{\meter\per\second}.)
    \begin{multicols}{3}
    \begin{choices}
        \wrongchoice{\SI{220}{\hertz}}
        \wrongchoice{\SI{448}{\hertz}}
        \wrongchoice{\SI{5264}{\hertz}}
      \correctchoice{\SI{552}{\hertz}}
        \wrongchoice{\SI{383}{\hertz}}
    \end{choices}
    \end{multicols}
\end{question}
}

\element{serway-mc}{
\begin{question}{serway-ch17-q19}
    A car moving at \SI{36}{\meter\per\second} passes a stationary police car whose siren has a frequency of \SI{500}{\hertz}.
    What is the change in the frequency heard by an observer in the moving car as he passes the police car?
    (The speed of sound in air is \SI{343}{\meter\per\second}.)
    \begin{multicols}{3}
    \begin{choices}
        \wrongchoice{\SI{416}{\hertz}}
        \wrongchoice{\SI{208}{\hertz}}
      \correctchoice{\SI{105}{\hertz}}
        \wrongchoice{\SI{52}{\hertz}}
        \wrongchoice{\SI{552}{\hertz}}
    \end{choices}
    \end{multicols}
\end{question}
}

\element{serway-mc}{
\begin{question}{serway-ch17-q20}
    A truck moving at \SI{36}{\meter\per\second} passes a police car moving at \SI{45}{\meter\per\second} in the opposite direction.
    If the frequency of the siren relative to the police car is \SI{500}{\hertz},
        what is the frequency heard by an observer in the truck as the police car approaches the truck?
    (The speed of sound in air is \SI{343}{\meter\per\second}.)
    \begin{multicols}{3}
    \begin{choices}
        \wrongchoice{\SI{396}{\hertz}}
      \correctchoice{\SI{636}{\hertz}}
        \wrongchoice{\SI{361}{\hertz}}
        \wrongchoice{\SI{393}{\hertz}}
        \wrongchoice{\SI{617}{\hertz}}
    \end{choices}
    \end{multicols}
\end{question}
}

\element{serway-mc}{
\begin{question}{serway-ch17-q21}
    A truck moving at \SI{36}{\meter\per\second} passes a police car moving at \SI{45}{\meter\per\second} in the opposite direction. 
    If the frequency of the siren is \SI{500}{\hertz} relative to the police car,
        what is the frequency heard by an observer in the truck after the police car passes the truck? 
    (The speed of sound in air is \SI{343}{\meter\per\second}.)
    \begin{multicols}{3}
    \begin{choices}
        \wrongchoice{\SI{361}{\hertz}}
        \wrongchoice{\SI{636}{\hertz}}
        \wrongchoice{\SI{393}{\hertz}}
      \correctchoice{\SI{396}{\hertz}}
        \wrongchoice{\SI{383}{\hertz}}
    \end{choices}
    \end{multicols}
\end{question}
}

\element{serway-mc}{
\begin{question}{serway-ch17-q22}
    A truck moving at \SI{36}{\meter\per\second} passes a police car moving at \SI{45}{\meter\per\second} in the opposite direction. 
    If the frequency of the siren is \SI{500}{\hertz} relative to the police car,
        what is the change in frequency heard by an observer in the truck as the two vehicles pass each other?
    (The speed of sound in air is \SI{343}{\meter\per\second}.)
    \begin{multicols}{3}
    \begin{choices}
        \wrongchoice{\SI{242}{\hertz}}
        \wrongchoice{\SI{238}{\hertz}}
      \correctchoice{\SI{240}{\hertz}}
        \wrongchoice{\SI{236}{\hertz}}
        \wrongchoice{\SI{234}{\hertz}}
    \end{choices}
    \end{multicols}
\end{question}
}

\element{serway-mc}{
\begin{question}{serway-ch17-q23}
    How fast is the Concorde moving if it reaches Mach 1.5? 
    (The speed of sound in air is \SI{344}{\meter\per\second}.)
    \begin{multicols}{3}
    \begin{choices}
        \wrongchoice{\SI{229}{\meter\per\second}}
      \correctchoice{\SI{516}{\meter\per\second}}
        \wrongchoice{\SI{416}{\meter\per\second}}
        \wrongchoice{\SI{728}{\meter\per\second}}
        \wrongchoice{\SI{858}{\meter\per\second}}
    \end{choices}
    \end{multicols}
\end{question}
}

\element{serway-mc}{
\begin{question}{serway-ch17-q24}
    A stone is thrown into a quiet pool of water. 
    With no fluid friction,
        the amplitude of the waves falls off with distance $r$ from the impact point as:
    \begin{multicols}{3}
    \begin{choices}
        \wrongchoice{$\dfrac{1}{r^3}$}
        \wrongchoice{$\dfrac{1}{r^2}$}
        \wrongchoice{$\dfrac{1}{r^{\frac{3}{2}}}$}
      \correctchoice{$\dfrac{1}{r^{\frac{1}{2}}}$}
        \wrongchoice{$\dfrac{1}{r}$}
    \end{choices}
    \end{multicols}
\end{question}
}

\element{serway-mc}{
\begin{question}{serway-ch17-q25}
    A wave generated in a medium is a longitudinal wave when:
    \begin{choices}
        \wrongchoice{there is a net transport of matter by the wave.}
        \wrongchoice{the molecules of the medium are unable to exert forces on each other.}
      \correctchoice{molecular displacements are parallel to the wave velocity.}
        \wrongchoice{molecular displacements are perpendicular to the wave velocity.}
        \wrongchoice{the density of the medium is less than the density of water.}
    \end{choices}
\end{question}
}

\element{serway-mc}{
\begin{question}{serway-ch17-q26}
    When you hear the horn of a car that is approaching you,
        the frequency that you hear is larger than that heard by a person in the car because:
    \begin{choices}
        \wrongchoice{wave crests are farther apart by the distance the car travels in one period.}
      \correctchoice{wave crests are closer together by the distance the car travels in one period.}
        \wrongchoice{the car gets ahead of each wave crest before it emits the next one.}
        \wrongchoice{the speed of sound in air is increased by the speed of the car.}
        \wrongchoice{a speeding car emits more wavecrests in each period.}
    \end{choices}
\end{question}
}

\element{serway-mc}{
\begin{question}{serway-ch17-q27}
    While you are sounding a tone on a toy whistle,
        you notice a friend running toward you. 
    If you want her to hear the same frequency that you hear even though she is approaching,
        you must:
    \begin{choices}
        \wrongchoice{stay put.}
        \wrongchoice{run towards her at the same speed.}
      \correctchoice{run away from her at the same speed.}
        \wrongchoice{stay put and play a note of higher frequency.}
        \wrongchoice{run towards her and play a note of higher frequency.}
    \end{choices}
\end{question}
}

\element{serway-mc}{
\begin{question}{serway-ch17-q28}
    To decrease the intensity of the sound you are hearing from your speaker system by a factor of \num{36},
        you can:
    \begin{choices}
        \wrongchoice{reduce the amplitude by a factor of \num{12} and increase your distance from the speaker by a factor of \num{3}.}
        \wrongchoice{reduce the amplitude by a factor of \num{4} and increase your distance from the speaker by a factor of \num{3}.}
      \correctchoice{reduce the amplitude by a factor of \num{2} and increase your distance from the speaker by a factor of \num{3}.}
        \wrongchoice{reduce the amplitude by a factor of \num{3} and increase your distance from the speaker by a factor of \num{4}.}
        \wrongchoice{reduce the amplitude by a factor of \num{3} and increase your distance from the speaker by a factor of \num{12}.}
    \end{choices}
\end{question}
}

\element{serway-mc}{
\begin{question}{serway-ch17-q29}
    Drummers like to have high-pitched cymbals that vibrate at high frequencies. 
    To obtain the highest frequencies,
        a cymbal of a fixed size should be made of a material:
    \begin{choices}
        \wrongchoice{with a low Young's modulus and a low density.}
        \wrongchoice{with a low Young's modulus and a high density.}
      \correctchoice{with a high Young's modulus and a low density.}
        \wrongchoice{with a high Young's modulus and a high density.}
        \wrongchoice{composed of a metal-plastic laminate.}
    \end{choices}
\end{question}
}

\element{serway-mc}{
\begin{question}{serway-ch17-q30}
    A person standing in the street is unaware of an bird dropping that is falling from a point directly above him with increasing velocity. 
    If the dropping were producing sound of a fixed frequency,
        as it approaches the person would hear the sound:
    \begin{choices}
        \wrongchoice{drop in frequency.}
        \wrongchoice{stay at the same frequency.}
      \correctchoice{increase in frequency.}
        \wrongchoice{decrease in loudness.}
        \wrongchoice{stay at the same loudness.}
    \end{choices}
\end{question}
}

\element{serway-mc}{
\begin{question}{serway-ch17-q31}
    (Do not try the following: it could kill you. 
    This question is only about a hypothetical possibility.) 
    If you were standing below an object falling at terminal velocity,
        as it approached you, you would hear the sound:
    \begin{choices}
        \wrongchoice{drop in frequency.}
      \correctchoice{stay at the same frequency.}
        \wrongchoice{increase in frequency.}
        \wrongchoice{decrease in loudness.}
        \wrongchoice{stay at the same loudness.}
    \end{choices}
\end{question}
}

\element{serway-mc}{
\begin{question}{serway-ch17-q32}
    A plane wave propagating along the $x$-axis has the form
    \begin{equation*}
        \Psi(r,t) = \left(\dfrac{\SI{0.002}{\centi\meter}}{r}\right) \sin\left(\dfrac{8\pi}{\si{\meter}}x-\dfrac{2720\pi}{\si{\second}}t\right).
    \end{equation*}
    The wavelength is:
    \begin{multicols}{3}
    \begin{choices}
        \wrongchoice{\SI{0.0398}{\meter}}
        \wrongchoice{\SI{0.125}{\meter}}
      \correctchoice{\SI{0.250}{\meter}}
        \wrongchoice{\SI{4.00}{\meter}}
        \wrongchoice{\SI{8.00}{\meter}}
    \end{choices}
    \end{multicols}
\end{question}
}

\element{serway-mc}{
\begin{question}{serway-ch17-q33}
    A plane wave propagating along the $x$-axis has the form
    \begin{equation*}
        \Psi(r,t) = \left(\dfrac{\SI{0.002}{\centi\meter}}{r}\right) \sin\left(\dfrac{8\pi}{\si{\meter}}x-\dfrac{2720\pi}{\si{\second}}t\right).
    \end{equation*}
    The frequency of the wave is:
    \begin{multicols}{2}
    \begin{choices}
        \wrongchoice{\SI{3.68e-2}{\hertz}}
        \wrongchoice{\SI{7.35e-2}{\hertz}}
      \correctchoice{\SI{1360}{\hertz}}
        \wrongchoice{\SI{2720}{\hertz}}
        \wrongchoice{\SI[parse-numbers=false]{2720\pi}{\hertz}}
    \end{choices}
    \end{multicols}
\end{question}
}

\element{serway-mc}{
\begin{question}{serway-ch17-q34}
    A spherical wave has the form
    \begin{equation*}
        \Psi(r,t) = \left(\dfrac{\SI{0.002}{\centi\meter}}{r}\right) \sin\left(\dfrac{8\pi}{\si{\meter}}x-\dfrac{2720\pi}{\si{\second}}t\right).
    \end{equation*}
    The amplitude of the wave a distance $r$ from the source is:
    \begin{multicols}{2}
    \begin{choices}
        \wrongchoice{\SI{0.002}{\centi\meter}}
        \wrongchoice{$\dfrac{\SI{0.002}{\centi\meter}}{\sqrt{r}}$}
      \correctchoice{$\dfrac{\SI{0.002}{\centi\meter}}{r}$}
        \wrongchoice{$\dfrac{\left(\SI{0.002}{\centi\meter}\right)^2}{r}$}
        \wrongchoice{$\dfrac{\SI{0.002}{\centi\meter}}{r^2}$}
    \end{choices}
    \end{multicols}
\end{question}
}

\element{serway-mc}{
\begin{question}{serway-ch17-q35}
    A spherical wave has the form
    \begin{equation*}
        \Psi(r,t) = \left(\dfrac{\SI{0.002}{\centi\meter}}{r}\right) \sin\left(\dfrac{8\pi}{\si{\meter}}x-\dfrac{2720\pi}{\si{\second}}t\right).
    \end{equation*}
    The wavelength of the wave is:
    \begin{multicols}{3}
    \begin{choices}
      \correctchoice{\SI{0.25}{\meter}}
        \wrongchoice{\SI{0.50}{\meter}}
        \wrongchoice{\SI{4}{\meter}}
        \wrongchoice{\SI{8}{\meter}}
        \wrongchoice{\SI[parse-numbers=false]{4\pi}{\meter}}
    \end{choices}
    \end{multicols}
\end{question}
}

\element{serway-mc}{
\begin{question}{serway-ch17-q36}
    A spherical wave has the form
    \begin{equation*}
        \Psi(r,t) = \left(\dfrac{\SI{0.002}{\centi\meter}}{r}\right) \sin\left(\dfrac{8\pi}{\si{\meter}}x-\dfrac{2720\pi}{\si{\second}}t\right).
    \end{equation*}
    The frequency of the wave is:
    \begin{multicols}{2}
    \begin{choices}
        \wrongchoice{\SI{3.68e-4}{\hertz}}
        \wrongchoice{\SI{7.35e-4}{\hertz}}
      \correctchoice{\SI{1360}{\hertz}}
        \wrongchoice{\SI{2720}{\hertz}}
        \wrongchoice{\SI[parse-numbers=false]{2720\pi}{\hertz}}
    \end{choices}
    \end{multicols}
\end{question}
}

\element{serway-mc}{
\begin{question}{serway-ch17-q37}
    A spherical wave has the form
    \begin{equation*}
        \Psi(r,t) = \left(\dfrac{\SI{0.002}{\centi\meter}}{r}\right) \sin\left(\dfrac{8\pi}{\si{\meter}}x-\dfrac{2720\pi}{\si{\second}}t\right).
    \end{equation*}
    The velocity of the wave is:
    \begin{multicols}{2}
    \begin{choices}
        \wrongchoice{\SI{0.00588}{\meter\per\second}}
        \wrongchoice{\SI{16}{\meter\per\second}}
      \correctchoice{\SI{340}{\meter\per\second}}
        \wrongchoice{\SI{1360}{\meter\per\second}}
        \wrongchoice{\SI{2720}{\meter\per\second}}
    \end{choices}
    \end{multicols}
\end{question}
}

\element{serway-mc}{
\begin{question}{serway-ch17-q38}
    A boy has climbed to the top of a \SI{6.00}{\meter} tall tree. 
    When he shouts, the sound waves have an intensity of \SI{0.250}{\watt\per\meter\squared} at a \SI{1.00}{\meter} distance from the top of the tree. 
    The ratio of intensity at the base of the tree to the intensity \SI{1.00}{\meter} from the top of the tree is:
    \begin{multicols}{3}
    \begin{choices}
      \correctchoice{$\dfrac{1}{36}$}
        \wrongchoice{$\dfrac{1}{25}$}
        \wrongchoice{$\dfrac{1}{6}$}
        \wrongchoice{$\dfrac{1}{5}$}
        \wrongchoice{$\dfrac{1}{\sqrt{6}}$}
    \end{choices}
    \end{multicols}
\end{question}
}

\element{serway-mc}{
\begin{question}{serway-ch17-q39}
    Which of the following does not have a precise definition in terms of the physical properties of sound waves?
    \begin{multicols}{2}
    \begin{choices}
        \wrongchoice{frequency}
        \wrongchoice{intensity}
      \correctchoice{loudness}
        \wrongchoice{sound level}
        \wrongchoice{wavelength}
    \end{choices}
    \end{multicols}
\end{question}
}

\element{serway-mc}{
\begin{question}{serway-ch17-q40}
    A friend hands you an equation sheet with the following equation for the Doppler effect:
    \begin{equation*}
        f^{\prime} = \dfrac{v-v_0}{v+v_S} f\, .
    \end{equation*}
    This version of the equation is correct with signs as given only if:
    \begin{choices}
        \wrongchoice{the observer and source are approaching each other.}
        \wrongchoice{the observer is approaching the source while the source is moving away from the observer.}
        \wrongchoice{the observer is moving away from the source while the source is approaching the observer.}
      \correctchoice{the observer and source are moving away from each other.}
        \wrongchoice{the observer and source are moving in perpendicular directions.}
    \end{choices}
\end{question}
}

\element{serway-mc}{
\begin{question}{serway-ch17-q41}
    A boat sounds a fog horn on a day when both the sea water and the air temperature are \SI{25.0}{\degreeCelsius}. 
    The speed of sound in sea water is \SI{1533}{\meter\per\second}. 
    How much earlier does a dolphin \SI{1000}{\meter} from the source hear the sound than a person in a boat that is also \SI{1000}{\meter} distant? 
    (Ignore the time it takes the sound to reach the water surface.)
    \begin{multicols}{3}
    \begin{choices}
        \wrongchoice{\SI{0.652}{\second}}
        \wrongchoice{\SI{2.12}{\second}}
      \correctchoice{\SI{2.24}{\second}}
        \wrongchoice{\SI{2.77}{\second}}
        \wrongchoice{\SI{2.90}{\second}}
    \end{choices}
    \end{multicols}
\end{question}
}

\element{serway-mc}{
\begin{question}{serway-ch17-q42}
    A source of sound waves is placed at the center of a very large sound-reflecting wall.
    The source emits \SI{0.500}{\watt} of power. 
    Two meters from the source, the intensity is:
    \begin{multicols}{2}
    \begin{choices}
        \wrongchoice{\SI{9.95e-3}{\watt\per\meter\squared}}
      \correctchoice{\SI{1.99e-2}{\watt\per\meter\squared}}
        \wrongchoice{\SI{0.313}{\watt\per\meter\squared}}
        \wrongchoice{\SI{0.399}{\watt\per\meter\squared}}
        \wrongchoice{\SI{0.625}{\watt\per\meter\squared}}
    \end{choices}
    \end{multicols}
\end{question}
}

\element{serway-mc}{
\begin{question}{serway-ch17-q43}
    On a day when the speed of sound in the upper air is \SI{320}{\meter\per\second},
        you fly coast to coast in the United States,
        a distance of about \SI{4850}{\kilo\meter},
        in about one hour,
        if the Mach number for the speed of your airplane is about
    \begin{multicols}{3}
    \begin{choices}
        \wrongchoice{\num{1}}
        \wrongchoice{\num{2}}
        \wrongchoice{\num{3}}
      \correctchoice{\num{4}}
        \wrongchoice{\num{5}}
    \end{choices}
    \end{multicols}
\end{question}
}

\element{serway-mc}{
\begin{question}{serway-ch17-q44}
    A fire engine approaches a wall at \SI{5}{\meter\per\second} while the siren emits a tone of \SI{500}{\hertz} frequency. 
    At the time, the speed of sound in air is \SI{340}{\meter\per\second}.
    How many beats per second do the people on the fire engine hear?
    \begin{multicols}{3}
    \begin{choices}
        \wrongchoice{zero}
      \correctchoice{\num{15}}
        \wrongchoice{\num{29}}
        \wrongchoice{\num{63}}
        \wrongchoice{\num{250}}
    \end{choices}
    \end{multicols}
\end{question}
}

\element{serway-mc}{
\begin{question}{serway-ch17-q45}
    The figure below shows the positions of particles in a longitudinal standing wave.
    \begin{center}
    \begin{tikzpicture}
        %% NOTE: TODO: draw tikz
    \end{tikzpicture}
    \end{center}
    One quarter period later the particle distribution is shown in:
    \begin{choices}
        %% NOTE: ANS is E
        \wrongchoice{
            \begin{tikzpicture}
                %% NOTE: TODO: draw tikz
            \end{tikzpicture}
        }
    \end{choices}
\end{question}
}


\endinput


