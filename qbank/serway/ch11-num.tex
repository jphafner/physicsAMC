
%%--------------------------------------------------
%% Serway: Physics for Scientists and Engineers
%%--------------------------------------------------


%% Chapter 11: Angular Momentum
%%--------------------------------------------------


%% Serway Numeric Questions
%%--------------------------------------------------
\element{serway-num}{
\begin{questionmultx}{serway-ch11-q43}
    Halley's comet moves about the sun in an elliptical orbit with its closest approach to the sun being 0.59 A.U. and its furthest distance being 35 A.U.
    [1 Astronomical Unit (A.U.) is the Earth-sun distance]. 
    If the comet's speed at closest approach is \SI{54}{\kilo\meter\per\second},
        what is its speed when it is farthest from the sun?
    %% ANSWER:  \SI{910}{\meter\per\second}
    \AMCnumericChoices{9.1}{
        vertical=false,
        digits=3,decimals=2,sign=true,Tsign=\hspace{1ex}\Large a,
        borderwidth=0pt,backgroundcol=white,approx=5
    }
    \AMCnumericChoices{3}{
        vertical=false,
        digits=2,decimals=0,sign=true,Tsign=\hspace{1ex}\Large b,
        borderwidth=0pt,backgroundcol=white,approx=5
    }
\end{questionmultx}
}

\element{serway-num}{
\begin{questionmultx}{serway-ch11-q44}
    What is the angular momentum of the moon about the Earth? 
    The mass of the moon is \SI{7.35e22}{\kilo\gram},
        the center-to-center separation of the Earth and the moon is \SI{3.84e5}{\kilo\meter},
        and the orbital period of the moon is 27.3 days.
    %% ANSWER:  \SI{2.89e34}{\meter\per\second}
    \AMCnumericChoices{2.89}{
        vertical=false,
        digits=3,decimals=2,sign=true,Tsign=\hspace{1ex}\Large a,
        borderwidth=0pt,backgroundcol=white,approx=5
    }
    \AMCnumericChoices{34}{
        vertical=false,
        digits=2,decimals=0,sign=true,Tsign=\hspace{1ex}\Large b,
        borderwidth=0pt,backgroundcol=white,approx=5
    }
\end{questionmultx}
}

\element{serway-num}{
\begin{questionmultx}{serway-ch11-q45}
    A regulation basketball has a \SI{25}{\centi\meter} diameter and a mass of \SI{0.56}{\kilo\gram}.
    It may be approximated as a thin spherical shell with a moment of inertia $\frac{2}{3}MR^2$.
    Starting from rest,
        how long will it take a basketball to roll without slipping \SI{4.0}{\meter} down an incline at \ang{30} to the horizontal?
    %% ANSWER:  \SI{1.65}{\second}
    \AMCnumericChoices{1.65}{
        vertical=false,
        digits=3,decimals=2,sign=true,Tsign=\hspace{1ex}\Large a,
        borderwidth=0pt,backgroundcol=white,approx=5
    }
    \AMCnumericChoices{0}{
        vertical=false,
        digits=2,decimals=0,sign=true,Tsign=\hspace{1ex}\Large b,
        borderwidth=0pt,backgroundcol=white,approx=5
    }
\end{questionmultx}
}

\element{serway-num}{
\begin{questionmultx}{serway-ch11-q46}
    A coin with a diameter \SI{3.0}{\centi\meter} rolls up a \ang{30} inclined plane. 
    The coin starts out with an initial angular speed of \SI{60.0}{\radian\per\second} and rolls in a straight line without slipping. 
    If the moment of inertia of the coin is $\frac{1}{2}MR^2$,
        how far will the coin roll up the inclined plane?
    %% ANSWER:  \SI{12.4}{\centi\meter}
    \AMCnumericChoices{1.24}{
        vertical=false,
        digits=3,decimals=2,sign=true,Tsign=\hspace{1ex}\Large a,
        borderwidth=0pt,backgroundcol=white,approx=5
    }
    \AMCnumericChoices{1}{
        vertical=false,
        digits=2,decimals=0,sign=true,Tsign=\hspace{1ex}\Large b,
        borderwidth=0pt,backgroundcol=white,approx=5
    }
\end{questionmultx}
}


\endinput


