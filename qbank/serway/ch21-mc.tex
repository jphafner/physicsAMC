
%%--------------------------------------------------
%% Serway: Physics for Scientists and Engineers
%%--------------------------------------------------


%% Chapter 21: The Kinetic Theory of Gases
%%--------------------------------------------------


%% Table of Contents
%%--------------------------------------------------

%% 21.1 Molecular Model of an Ideal Gas
%% 21.2 Molar Specific Heat of an Ideal Gas
%% 21.3 Adiabatic Processes for an Ideal Gas
%% 21.4 The Equipartition of Energy
%% 21.5 Distribution of Molecular Speeds


%% Serway Multiple Choice Questions
%%--------------------------------------------------
\element{serway-mc}{
\begin{question}{serway-ch21-q01}
    A container having a volume of \SI{1.0}{\meter\cubed} holds \SI{5.0}{\mole} of helium gas at \SI{50}{\degreeCelsius}. 
    If the helium behaves like an ideal gas,
        the total energy of the system is:
    \begin{multicols}{2}
    \begin{choices}
      \correctchoice{\SI{2.0e4}{\joule}}
        \wrongchoice{\SI{2.5e4}{\joule}}
        \wrongchoice{\SI{1.7e3}{\joule}}
        \wrongchoice{\SI{1.5e3}{\joule}}
        \wrongchoice{\SI{4.0e4}{\joule}}
    \end{choices}
    \end{multicols}
\end{question}
}

\element{serway-mc}{
\begin{question}{serway-ch21-q02}
    A container having a volume of \SI{1.0}{\meter\cubed} holds \SI{5.0}{\mole} of helium gas at \SI{50}{\degreeCelsius}. 
    If the helium behaves like an ideal gas,
        the average kinetic energy per molecule is:
    \begin{multicols}{2}
    \begin{choices}
        \wrongchoice{\SI{6.7e-20}{\joule}}
        \wrongchoice{\SI{1.0e-21}{\joule}}
        \wrongchoice{\SI{1.0e-20}{\joule}}
      \correctchoice{\SI{6.7e-21}{\joule}}
        \wrongchoice{\SI{1.3e-20}{\joule}}
    \end{choices}
    \end{multicols}
\end{question}
}

\element{serway-mc}{
\begin{question}{serway-ch21-q03}
    The average kinetic energy of a nitrogen molecule at room temperature (\SI{20}{\degreeCelsius}) is:
    \begin{multicols}{2}
    \begin{choices}
        \wrongchoice{\SI{2e-21}{\joule}}
        \wrongchoice{\SI{4e-21}{\joule}}
        \wrongchoice{\SI{6e-21}{\joule}}
        \wrongchoice{\SI{8e-21}{\joule}}
      \correctchoice{\SI{1e-20}{\joule}}
    \end{choices}
    \end{multicols}
\end{question}
}

\element{serway-mc}{
\begin{question}{serway-ch21-q04}
    The average translational speed of a nitrogen molecule at
        room temperature (\SI{20}{\degreeCelsius}) is approximately:
    \begin{multicols}{3}
    \begin{choices}
        \wrongchoice{\SI{100}{\meter\per\second}}
      \correctchoice{\SI{500}{\meter\per\second}}
        \wrongchoice{\SI{300}{\meter\per\second}}
        \wrongchoice{\SI{700}{\meter\per\second}}
        \wrongchoice{\SI{200}{\meter\per\second}}
    \end{choices}
    \end{multicols}
\end{question}
}

\element{serway-mc}{
\begin{question}{serway-ch21-q05}
    A box contains about \num{5.0e21} hydrogen atoms at room temperature (\SI{21}{\degreeCelsius}).
    Determine the thermal energy of these atoms.
    \begin{multicols}{3}
    \begin{choices}
        \wrongchoice{\SI{10}{\joule}}
        \wrongchoice{\SI{20}{\joule}}
      \correctchoice{\SI{30}{\joule}}
        \wrongchoice{\SI{5.0}{\joule}}
        \wrongchoice{\SI{1.0}{\joule}}
    \end{choices}
    \end{multicols}
\end{question}
}

\element{serway-mc}{
\begin{question}{serway-ch21-q06}
    Five gas molecules are found to have speeds of
        \SI{100}{\meter\per\second},
        \SI{200}{\meter\per\second},
        \SI{300}{\meter\per\second},
        \SI{400}{\meter\per\second},
        and \SI{500}{\meter\per\second},
    The rms speed is:
    \begin{multicols}{3}
    \begin{choices}
        \wrongchoice{\SI{390}{\meter\per\second}}
        \wrongchoice{\SI{300}{\meter\per\second}}
        \wrongchoice{\SI{360}{\meter\per\second}}
      \correctchoice{\SI{330}{\meter\per\second}}
        \wrongchoice{\SI{320}{\meter\per\second}}
    \end{choices}
    \end{multicols}
\end{question}
}

\element{serway-mc}{
\begin{question}{serway-ch21-q07}
    Find the specific heat of a gas kept at constant volume when it takes \SI{1.0e4}{\joule} of heat to raise the temperature of \SI{5.0}{\mole} of the gas \SI{200}{\kelvin} above the initial temperature.
    \begin{multicols}{2}
    \begin{choices}
        \wrongchoice{\SI{7.5}{\calorie\per\mole\per\kelvin}}
        \wrongchoice{\SI{5.0}{\calorie\per\mole\per\kelvin}}
      \correctchoice{\SI{2.4}{\calorie\per\mole\per\kelvin}}
        \wrongchoice{\SI{10}{\calorie\per\mole\per\kelvin}}
        \wrongchoice{\SI{20}{\calorie\per\mole\per\kelvin}}
    \end{choices}
    \end{multicols}
\end{question}
}

\element{serway-mc}{
\begin{question}{serway-ch21-q08}
    The air in an automobile engine at \SI{20}{\degreeCelsius} is compressed from an initial pressure of \SI{1.0}{\atm} and a volume of \SI{200}{\centi\meter\cubed} to a volume of \SI{20}{\centi\meter\cubed}.
    Find the temperature if the air behaves like an ideal gas ($\gamma=\num{1.4}$) and the compression is adiabatic.
    \begin{multicols}{3}
    \begin{choices}
        \wrongchoice{\SI{730}{\degreeCelsius}}
      \correctchoice{\SI{460}{\degreeCelsius}}
        \wrongchoice{\SI{25}{\degreeCelsius}}
        \wrongchoice{\SI{50}{\degreeCelsius}}
        \wrongchoice{\SI{20}{\degreeCelsius}}
    \end{choices}
    \end{multicols}
\end{question}
}

\element{serway-mc}{
\begin{question}{serway-ch21-q09}
    During an adiabatic compression,
        a volume of air decreases to \num{1/4} its original size. 
    Calculate its final pressure if its original pressure was \SI{1}{\atm}. 
    (Assume the air behaves like an ideal gas with $\gamma=\num{1.4}$.)
    \begin{multicols}{3}
    \begin{choices}
      \correctchoice{\SI{7.0}{\atm}}
        \wrongchoice{\SI{5.6}{\atm}}
        \wrongchoice{\SI{3.5}{\atm}}
        \wrongchoice{\SI{2.2}{\atm}}
        \wrongchoice{\SI{0.14}{\atm}}
    \end{choices}
    \end{multicols}
\end{question}
}

\element{serway-mc}{
\begin{question}{serway-ch21-q10}
    An ideal gas is allowed to expand adiabatically until its volume increases by \SI{50}{\percent}.
    By approximately what factor is the pressure reduced?
    ($\gamma=\num{5/3}$)
    \begin{multicols}{3}
    \begin{choices}
        \wrongchoice{\num{1.5}}
      \correctchoice{\num{2.0}}
        \wrongchoice{\num{2.5}}
        \wrongchoice{\num{3.0}}
        \wrongchoice{\num{3.5}}
    \end{choices}
    \end{multicols}
\end{question}
}

\element{serway-mc}{
\begin{question}{serway-ch21-q11}
    When we say that the speed of sound is measured under adiabatic conditions we assume that:
    \begin{choices}
      \correctchoice{the time associated with heat conduction is slow relative to the speed of the wave.}
        \wrongchoice{no heat can flow between the system and its surroundings.}
        \wrongchoice{the speed of the wave is directly proportional to the bulk modulus.}
        \wrongchoice{the speed of the wave is proportional to the square root of the bulk modulus.}
        \wrongchoice{air is an ideal gas.}
    \end{choices}
\end{question}
}

\element{serway-mc}{
\begin{question}{serway-ch21-q12}
    Assume \SI{3.0}{\mole} of a diatomic gas has an internal energy of \SI{10}{\kilo\joule}.
    Determine the temperature of the gas after it has reached equilibrium.
    \begin{multicols}{3}
    \begin{choices}
        \wrongchoice{\SI{270}{\kelvin}}
      \correctchoice{\SI{160}{\kelvin}}
        \wrongchoice{\SI{800}{\kelvin}}
        \wrongchoice{\SI{1550}{\kelvin}}
        \wrongchoice{\SI{400}{\kelvin}}
    \end{choices}
    \end{multicols}
\end{question}
}

\element{serway-mc}{
\begin{question}{serway-ch21-q13}
    Nitrogen gas is heated by a pulsed laser to \SI{50 000}{\kelvin}. 
    If the diameter of the nitrogen atoms is assumed to be \SI{1.0e-10}{\meter},
        and the pressure is \SI{1.0}{\atm}, what is the mean free path?
    \begin{multicols}{2}
    \begin{choices}
      \correctchoice{\SI{1.5e-4}{\meter}}
        \wrongchoice{\SI{1.5e-7}{\meter}}
        \wrongchoice{\SI{1.5e-10}{\meter}}
        \wrongchoice{\SI{1.5e-14}{\meter}}
        \wrongchoice{\SI{1.5e-2}{\meter}}
    \end{choices}
    \end{multicols}
\end{question}
}

\element{serway-mc}{
\begin{question}{serway-ch21-q14}
    Assume molecules have an average diameter of \SI{3.00e-10}{\meter}.
    How many times larger is the mean free path than the diameter at STP? 
    (Assume the pressure is \SI{1.01e5}{\newton\per\meter\squared}.)
    \begin{multicols}{3}
    \begin{choices}
        \wrongchoice{\num{500}}
      \correctchoice{\num{300}}
        \wrongchoice{\num{700}}
        \wrongchoice{\num{1000}}
        \wrongchoice{\num{2500}}
    \end{choices}
    \end{multicols}
\end{question}
}

\element{serway-mc}{
\begin{question}{serway-ch21-q15}
    The internal energy of $n$ moles of an ideal gas depends on:
    \begin{choices}
      \correctchoice{one state variable $T$.}
        \wrongchoice{two state variables $T$ and $V$.}
        \wrongchoice{two state vartiables $T$ and $P$.}
        \wrongchoice{three state variables $T$, $P$ and $V$.}
        \wrongchoice{four variables $R$, $T$, $P$ and $V$.}
    \end{choices}
\end{question}
}

\element{serway-mc}{
\begin{question}{serway-ch21-q16}
    A molecule in a uniform ideal gas can collide with other molecules when their centers are equal to or less than
    \begin{choices}
        \wrongchoice{one radius away from its center.}
      \correctchoice{one diameter away from its center.}
        \wrongchoice{two diameters away from its center.}
        \wrongchoice{twice the cube root of volume away from its center.}
        \wrongchoice{2 diameters away from its center.}
    \end{choices}
\end{question}
}

\element{serway-mc}{
\begin{question}{serway-ch21-q17}
    The average molecular translational kinetic energy of a molecule in an ideal gas is:
    %\begin{multicols}{2}
    \begin{choices}
      \correctchoice{$\dfrac{3}{2} k_B T$}
        \wrongchoice{$\dfrac{3}{2} R T$}
        \wrongchoice{$\dfrac{5}{2} k_B T$}
        \wrongchoice{$\dfrac{5}{2} R T$}
        \wrongchoice{$\dfrac{n+3}{2} k_B T$, where $n=$ number of internal degrees of freedom.}
    \end{choices}
    %\end{multicols}
\end{question}
}

\element{serway-mc}{
\begin{question}{serway-ch21-q18}
    The relation $PV=nRT$ holds for all ideal gases. 
    The additional relation $PV^{\gamma}$ holds for an adiabatic process. 
    The figure below shows two curves:
        one is an adiabat and one is an isotherm. 
    \begin{center}
    \begin{tikzpicture}
        %% NOTE:
    \end{tikzpicture}
    \end{center}
    Each starts at the same pressure and volume. 
    Which statement is correct? 
    (Note: ``$\propto$'' means ``is proportional to''.)
    \begin{choices}
        \wrongchoice{Isotherm: $P\propto\dfrac{1}{V}$;          Adiabat $P\propto\dfrac{1}{V}$;         $A$ is both an isotherm and an adiabat.}
        \wrongchoice{Isotherm: $P\propto\dfrac{1}{V^{\gamma}}$;  Adiabat $P\propto\dfrac{1}{V}$;         $B$ is an isotherm, $A$ is an adiabat.}
      \correctchoice{Isotherm: $P\propto\dfrac{1}{V}$;          Adiabat $P\propto\dfrac{1}{V^{\gamma}}$; $A$ is an isotherm, $B$ is an adiabat.}
        \wrongchoice{Isotherm: $P\propto\dfrac{1}{V^{\gamma}}$;  Adiabat $P\propto\dfrac{1}{V^{\gamma}}$; $B$ is both an isotherm andnan adiabat.}
        \wrongchoice{I cannot answer this without additional information about the starting temperature.}
    \end{choices}
\end{question}
}

\element{serway-mc}{
\begin{question}{serway-ch21-q19}
    Which statement below is \emph{not} an assumption made in the molecular model of an ideal gas?
    \begin{choices}
        \wrongchoice{The average separation between molecules is large compared with the dimensions of the molecules.}
      \correctchoice{The molecules undergo inelastic collisions with one another.}
        \wrongchoice{The forces between molecules are short range.}
        \wrongchoice{The molecules obey Newton’s laws of motion.}
        \wrongchoice{Any molecule can move in any direction with equal probability.}
    \end{choices}
\end{question}
}

\element{serway-mc}{
\begin{question}{serway-ch21-q20}
    The theorem of equipartition of energy states that the energy each degree of freedom contributes to each molecule in the system (an ideal gas) is:
    \begin{multicols}{3}
    \begin{choices}
        \wrongchoice{$\dfrac{1}{2} mv^2$}
        \wrongchoice{$\dfrac{1}{3} k_B T$}
      \correctchoice{$\dfrac{1}{2} k_B T$}
        \wrongchoice{$\dfrac{3}{2} m v^2$}
        \wrongchoice{$\dfrac{3}{2} k_B T$}
    \end{choices}
    \end{multicols}
\end{question}
}

\element{serway-mc}{
\begin{question}{serway-ch21-q21}
    The specific heat at constant volume at \SI{0}{\degreeCelsius} of one mole of an ideal monatomic gas is:
    \begin{multicols}{3}
    \begin{choices}
        \wrongchoice{$\dfrac{1}{2} R$}
        \wrongchoice{$R$}
      \correctchoice{$\dfrac{3}{2} R$}
        \wrongchoice{$2 R$}
        \wrongchoice{$\dfrac{5}{2} R$}
    \end{choices}
    \end{multicols}
\end{question}
}

\element{serway-mc}{
\begin{question}{serway-ch21-q22}
    The specific heat at constant volume at \SI{0}{\degreeCelsius} of one mole of an ideal diatomic gas is:
    \begin{multicols}{3}
    \begin{choices}
        \wrongchoice{$\dfrac{1}{2} R$}
        \wrongchoice{$R$}
        \wrongchoice{$\dfrac{3}{2} R$}
        \wrongchoice{$2 R$}
      \correctchoice{$\dfrac{5}{2} R$}
    \end{choices}
    \end{multicols}
\end{question}
}

\element{serway-mc}{
\begin{question}{serway-ch21-q23}
    The specific heat at constant pressure at \SI{0}{\degreeCelsius} of one mole of an ideal monatomic gas is:
    \begin{multicols}{3}
    \begin{choices}
        \wrongchoice{$\dfrac{1}{2} R$}
        \wrongchoice{$R$}
        \wrongchoice{$\dfrac{3}{2} R$}
        \wrongchoice{$2 R$}
      \correctchoice{$\dfrac{5}{2} R$}
    \end{choices}
    \end{multicols}
\end{question}
}

\element{serway-mc}{
\begin{question}{serway-ch21-q24}
    When we consider a thin horizontal layer of the atmosphere,
        of thickness $\mathrm{d}y$, of area $A$, with pressure $P$ on the bottom,
        with an average mass $m$ per molecule,
        and $n_V$ molecules per unit volume,
        the magnitude of the difference of the pressure at the top and bottom of the layer is given by $\mathrm{d}P=$
    \begin{multicols}{2}
    \begin{choices}
        \wrongchoice{$mg\,\mathrm{d}y$}
      \correctchoice{$mgn_V\,\mathrm{d}y$}
        \wrongchoice{$mgA\,\mathrm{d}y$}
        \wrongchoice{$mgn_VA\,\mathrm{d}y$}
        \wrongchoice{$mgn_VAP\,\mathrm{d}y$}
    \end{choices}
    \end{multicols}
\end{question}
}

\element{serway-mc}{
\begin{question}{serway-ch21-q25}
    Burt states that the molecular model of an ideal gas assumes that the molecules of the gas do not collide with one another. 
    Brooks states that it assumes that there is only one molecule moving back and forth between opposite walls in the container. 
    Which one, if either, is correct?
    \begin{choices}[o]
        %% NOTE: questionmult ??
        \wrongchoice{Burt, because the time interval between collisions with the same wall is $\Delta t = \dfrac{2d}{v}$, where $v$, the velocity, is perpendicular to two opposite walls.}
        \wrongchoice{Brooks, because $\Delta t$ will be greater if there is more than one molecule in the container.}
        \wrongchoice{Both, because (a) and (b) are both correct.}
        \wrongchoice{Both, because $\Delta t$, a time average over the components of velocity perpendicular to pairs of walls, is correct as long as the density is low and collisions are inelastic.}
      \correctchoice{Neither: $\Delta t$, a time average over the components of velocity perpendicular to pairs of walls, is correct as long as the density is low and collisions are elastic.}
    \end{choices}
\end{question}
}

\element{serway-mc}{
\begin{question}{serway-ch21-q26}
    The temperature of a quantity of an ideal gas is:
    \begin{choices}[o]
        %% NOTE: questionmult ??
        \wrongchoice{one measure of its ability to transfer thermal energy to another body.}
        \wrongchoice{proportional to the average molecular kinetic energy of the molecules.}
        \wrongchoice{proportional to the internal energy of the gas.}
      \correctchoice{correctly described by all the statements above.}
        \wrongchoice{correctly described only by (a) and (b) above.}
    \end{choices}
\end{question}
}

\element{serway-mc}{
\begin{question}{serway-ch21-q27}
    Two tanks of gas, one of hydrogen, \ce{H2}, and one of helium, \ce{He},
        contain equal numbers of moles of gas. 
    The gram-molecular mass of \ce{He} is twice that of \ce{H2}.
    Both tanks of gas are at the same temperature,
        \SI{293}{\kelvin}. 
    Which statement(s) below is(are) correct when we ignore vibrational motion?
    \begin{choices}
        \wrongchoice{The total internal energy of the hydrogen is the same as that of the helium.}
        \wrongchoice{The total internal energy of the hydrogen is 1.4 times that of the helium.}
        \wrongchoice{The total internal energy of the helium is 1.4 times that of the hydrogen.}
      \correctchoice{The total internal energy of the hydrogen is 1.67 times that of the helium.}
        \wrongchoice{The total internal energy of the helium is 1.67 times that of the hydrogen.}
    \end{choices}
\end{question}
}

\element{serway-mc}{
\begin{question}{serway-ch21-q28}
    Two tanks of gas, one of hydrogen, \ce{H2}, and one of helium, \ce{He},
        contain equal masses of gas. 
    The gram-molecular mass of \ce{He} is twice that of \ce{H2}. 
    Both tanks of gas are at the same temperature, \SI{293}{\kelvin}.
    Which statement(s) below is(are) correct when we ignore vibrational motion?
    \begin{choices}
        \wrongchoice{The total internal energy of the hydrogen is the same as that of the helium.}
        \wrongchoice{The total internal energy of the hydrogen is 167 times that of the helium.}
        \wrongchoice{The total internal energy of the helium is 1.67 times that of the hydrogen.}
      \correctchoice{The total internal energy of the hydrogen is 3.33 times that of the helium.}
        \wrongchoice{The total internal energy of the helium is 3.33 times that of the hydrogen.}
    \end{choices}
\end{question}
}

\element{serway-mc}{
\begin{question}{serway-ch21-q29}
    One mole of hydrogen,
        one mole of nitrogen and one mole of oxygen are held in a \SI{22.4e3}{\centi\meter\cubed} enclosed vessel at \SI{20}{\degreeCelsius}. 
    The pressure in the vessel is:
    \begin{multicols}{2}
    \begin{choices}
        \wrongchoice{\SI{109}{\newton\per\meter\squared}}
        \wrongchoice{\SI{304}{\newton\per\meter\squared}}
        \wrongchoice{\SI{326}{\newton\per\meter\squared}}
        \wrongchoice{\SI{1.09e5}{\newton\per\meter\squared}}
      \correctchoice{\SI{3.26e5}{\newton\per\meter\squared}}
    \end{choices}
    \end{multicols}
\end{question}
}

\element{serway-mc}{
\begin{question}{serway-ch21-q30}
    The root mean square speed of a gas molecule is greater than the average speed,
        because the former gives a greater weight to:
    \begin{choices}
        \wrongchoice{lighter molecules.}
        \wrongchoice{heavier molecules.}
        \wrongchoice{lower speeds.}
      \correctchoice{higher speeds.}
        \wrongchoice{more probable speeds.}
    \end{choices}
\end{question}
}


\endinput


