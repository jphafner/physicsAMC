
%%--------------------------------------------------
%% Serway: Physics for Scientists and Engineers
%%--------------------------------------------------


%% Chapter 15: Oscillatory Motion
%%--------------------------------------------------


%% Serway Numeric Questions
%%--------------------------------------------------
\element{serway-num}{
\begin{questionmultx}{serway-ch15-q35}
    An automobile ($m=\SI{1.00e3}{\kilo\gram}$) is driven into a brick wall in a safety test. 
    The bumper behaves like a spring ($k=\SI{5.00e6}{\newton\per\meter}$),  
        and is observed to compress a distance of \SI{3.16}{\centi\meter} as the car is brought to rest. 
    What was the initial speed of the automobile?
    %% ANSWER:  \SI{2.23}{\meter\per\second}
    \AMCnumericChoices{2.23}{
        vertical=false,
        digits=3,decimals=2,sign=true,Tsign=\hspace{1ex}\Large a,
        borderwidth=0pt,backgroundcol=white,approx=5
    }
    \AMCnumericChoices{0}{
        vertical=false,
        digits=2,decimals=0,sign=true,Tsign=\hspace{1ex}\Large b,
        borderwidth=0pt,backgroundcol=white,approx=5
    }
\end{questionmultx}
}

\element{serway-num}{
\begin{questionmultx}{serway-ch15-q36}
    The mat of a trampoline is held by 32 springs,
        each having a spring constant of \SI{5000}{\newton\per\meter}.
    A person with a mass of \SI{40.0}{\kilo\gram} jumps from a platform \SI{1.93}{\meter} high onto the trampoline. 
    Determine the stretch of each of the springs.
    %% ANSWER:  \SI{9.97}{\centi\meter}
    \AMCnumericChoices{9.97}{
        vertical=false,
        digits=3,decimals=2,sign=true,Tsign=\hspace{1ex}\Large a,
        borderwidth=0pt,backgroundcol=white,approx=5
    }
    \AMCnumericChoices{0}{
        vertical=false,
        digits=2,decimals=0,sign=true,Tsign=\hspace{1ex}\Large b,
        borderwidth=0pt,backgroundcol=white,approx=5
    }
\end{questionmultx}
}

\element{serway-num}{
\begin{questionmultx}{serway-ch15-q37}
    An archer pulls her bow string back \SI{0.4}{\meter} by exerting a force that increases uniformly from zero to \SI{240}{\newton}.
    What is the equivalent spring constant of the bow,
        and how much work is done in pulling the bow?
    %% ANSWER:  \SI{600}{\newton\per\meter}
    %% ANSWER:  \SI{48}{\joule}
    \AMCnumericChoices{9.97}{
        vertical=false,
        digits=3,decimals=2,sign=true,Tsign=\hspace{1ex}\Large a,
        borderwidth=0pt,backgroundcol=white,approx=5
    }
    \AMCnumericChoices{0}{
        vertical=false,
        digits=2,decimals=0,sign=true,Tsign=\hspace{1ex}\Large b,
        borderwidth=0pt,backgroundcol=white,approx=5
    }
\end{questionmultx}
}

\element{serway-num}{
\begin{questionmultx}{serway-ch15-q38}
    An ore car of mass \SI{4000}{\kilo\gram} starts from rest and rolls downhill on tracks from a mine. A spring with k = 400 000 N/m is located at the end of the tracks. 
    At the spring's maximum compression,
        the car is at an elevation \SI{10}{\meter} lower than its elevation at the starting point. 
    How much is the spring compressed in stopping the ore car? Ignore friction.
    %% ANSWER:  \SI{1.4}{\meter}
    \AMCnumericChoices{9.97}{
        vertical=false,
        digits=3,decimals=2,sign=true,Tsign=\hspace{1ex}\Large a,
        borderwidth=0pt,backgroundcol=white,approx=5
    }
    \AMCnumericChoices{0}{
        vertical=false,
        digits=2,decimals=0,sign=true,Tsign=\hspace{1ex}\Large b,
        borderwidth=0pt,backgroundcol=white,approx=5
    }
\end{questionmultx}
}

\element{serway-num}{
\begin{questionmultx}{serway-ch15-q39}
    The motion of a piston in an auto engine is simple harmonic. 
    If the piston travels back and forth over a distance of \SI{10}{\centi\meter},
        and the piston has a mass of \SI{1.5}{\kilo\gram},
        what is the maximum speed of the piston and the maximum force acting on the piston when the engine is running at 4200 rpm?
    %% ANSWER:  \SI{22}{\meter\per\second}
    %% ANSWER:  \SI{14 500}{\newton}
    \AMCnumericChoices{2.2}{
        vertical=false,
        digits=3,decimals=2,sign=true,Tsign=\hspace{1ex}\Large a,
        borderwidth=0pt,backgroundcol=white,approx=5
    }
    \AMCnumericChoices{1}{
        vertical=false,
        digits=2,decimals=0,sign=true,Tsign=\hspace{1ex}\Large b,
        borderwidth=0pt,backgroundcol=white,approx=5
    }
\end{questionmultx}
}


\endinput


