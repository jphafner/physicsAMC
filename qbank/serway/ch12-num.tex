
%%--------------------------------------------------
%% Serway: Physics for Scientists and Engineers
%%--------------------------------------------------


%% Chapter 12: Static Equilibrium and Elasticity
%%--------------------------------------------------


%% Serway Numeric Questions
%%--------------------------------------------------
\element{serway-num}{
\begin{questionmultx}{serway-ch12-q26}
    For safety in climbing, a mountaineer uses a \SI{50}{\meter} long nylon rope that is \SI{1.0}{\centi\meter} in diameter. 
    When supporting a \SI{90}{\kilo\gram} climber,
        the rope elongates \SI{1.6}{\meter}. 
    Find the Young's modulus for the rope material.
    %% ANSWER:  \SI{3.51e8}{\newton\per\meter\squared}
    \AMCnumericChoices{3.51}{
        vertical=false,
        digits=3,decimals=2,sign=true,Tsign=\hspace{1ex}\Large a,
        borderwidth=0pt,backgroundcol=white,approx=5
    }
    \AMCnumericChoices{8}{
        vertical=false,
        digits=2,decimals=0,sign=true,Tsign=\hspace{1ex}\Large b,
        borderwidth=0pt,backgroundcol=white,approx=5
    }
\end{questionmultx}
}

\element{serway-num}{
\begin{questionmultx}{serway-ch12-q27}
    The four tires of an automobile are inflated to a gauge pressure of \SI{2.0e5}{\newton\per\meter\squared} (\SI{29}{\pound\per\inch\squared}). 
    Each of the four tires has an area of \SI{0.024}{\meter\squared} that is in contact with the ground. 
    Determine the weight of the auto.
    %% ANSWER:  \SI{19 200}{\newton}
    \AMCnumericChoices{1.92}{
        vertical=false,
        digits=3,decimals=2,sign=true,Tsign=\hspace{1ex}\Large a,
        borderwidth=0pt,backgroundcol=white,approx=5
    }
    \AMCnumericChoices{4}{
        vertical=false,
        digits=2,decimals=0,sign=true,Tsign=\hspace{1ex}\Large b,
        borderwidth=0pt,backgroundcol=white,approx=5
    }
\end{questionmultx}
}

\element{serway-num}{
\begin{questionmultx}{serway-ch12-q28}
    Find the minimum diameter of a steel wire \SI{18}{\meter} long that will stretch no more than \SI{9}{\milli\meter} when a load of \SI{380}{\kilo\gram} is hung on the lower end.
    ($Y_{\text{steel}}=\SI{2.0e11}{\newton\per\meter\squared}$).
    %% ANSWER:  \SI{6.89}{\milli\meter}
    \AMCnumericChoices{6.89}{
        vertical=false,
        digits=3,decimals=2,sign=true,Tsign=\hspace{1ex}\Large a,
        borderwidth=0pt,backgroundcol=white,approx=5
    }
    \AMCnumericChoices{-3}{
        vertical=false,
        digits=2,decimals=0,sign=true,Tsign=\hspace{1ex}\Large b,
        borderwidth=0pt,backgroundcol=white,approx=5
    }
\end{questionmultx}
}

\element{serway-num}{
\begin{questionmultx}{serway-ch12-q29}
    If \SI{1.0}{\meter\cubed} of concrete weighs \SI{5e4}{\newton},
        what is the height of the tallest cylindrical concrete pillar that will not collapse under its own weight? 
    (The compression strength of concrete is \SI{1.7e7}{\newton\per\meter\squared})
    %% ANSWER:  \SI{340}{meter}
    \AMCnumericChoices{3.40}{
        vertical=false,
        digits=3,decimals=2,sign=true,Tsign=\hspace{1ex}\Large a,
        borderwidth=0pt,backgroundcol=white,approx=5
    }
    \AMCnumericChoices{2}{
        vertical=false,
        digits=2,decimals=0,sign=true,Tsign=\hspace{1ex}\Large b,
        borderwidth=0pt,backgroundcol=white,approx=5
    }
\end{questionmultx}
}



\endinput


