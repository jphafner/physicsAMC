
%%--------------------------------------------------
%% Serway: Physics for Scientists and Engineers
%%--------------------------------------------------


%% Chapter 09: Linear Momentum and Collisions
%%--------------------------------------------------


%% Serway Numeric Questions
%%--------------------------------------------------
\element{serway-num}{
\begin{questionmultx}{serway-ch09-q83}
    A child bounces a \SI{50}{\gram} superball on the sidewalk. 
    The velocity of the superball changes from \SI{21}{\meter\per\second} downward to \SI{19}{\meter\per\second} upward. 
    If the contact time with the sidewalk is \SI{1/800}{\second},
        what is the magnitude of the force exerted on the superball by the sidewalk?
    %% ANSWER:  \SI{1600}{\newton}
    \AMCnumericChoices{1.6}{
        vertical=false,
        digits=3,decimals=2,sign=true,Tsign=\hspace{1ex}\Large a,
        borderwidth=0pt,backgroundcol=white,approx=5
    }
    \AMCnumericChoices{2}{
        vertical=false,
        digits=2,decimals=0,sign=true,Tsign=\hspace{1ex}\Large b,
        borderwidth=0pt,backgroundcol=white,approx=5
    }
\end{questionmultx}
}

\element{serway-num}{
\begin{questionmultx}{serway-ch09-q84}
    High-speed stroboscopic photographs show that the head of a golf club of mass \SI{200}{\gram} is traveling at \SI{55}{\meter\per\second} just before it strikes a \SI{46}{\gram} golf ball at rest on a tee. 
    After the collision, the clubhead travels (in the same direction) at \SI{40}{\meter\per\second}.
    Find the speed of the golf ball just after impact.
    %% ANSWER:  \SI{65.2}{\meter\per\second}
    \AMCnumericChoices{1.6}{
        vertical=false,
        digits=3,decimals=2,sign=true,Tsign=\hspace{1ex}\Large a,
        borderwidth=0pt,backgroundcol=white,approx=5
    }
    \AMCnumericChoices{2}{
        vertical=false,
        digits=2,decimals=0,sign=true,Tsign=\hspace{1ex}\Large b,
        borderwidth=0pt,backgroundcol=white,approx=5
    }
\end{questionmultx}
}

\element{serway-num}{
\begin{questionmultx}{serway-ch09-q85}
    A pitcher claims he can throw a baseball with as much momentum as a \SI{3.00}{\gram} bullet moving with a speed of \SI{1500}{\meter\per\second}. 
    A baseball has a mass of \SI{0.145}{\kilo\gram}. 
    What must be its speed if the pitcher's claim is valid?
    %% ANSWER:  \SI{31.0}{\meter\per\second}
    \AMCnumericChoices{3.1}{
        vertical=false,
        digits=3,decimals=2,sign=true,Tsign=\hspace{1ex}\Large a,
        borderwidth=0pt,backgroundcol=white,approx=5
    }
    \AMCnumericChoices{1}{
        vertical=false,
        digits=2,decimals=0,sign=true,Tsign=\hspace{1ex}\Large b,
        borderwidth=0pt,backgroundcol=white,approx=5
    }
\end{questionmultx}
}

\element{serway-num}{
\begin{questionmultx}{serway-ch09-q86}
    A U-238 nucleus (mass = 238 units) decays,
        transforming into an alpha particle (mass = 4 units)
        and a residual thorium nucleus (mass = 234 units). 
    If the uranium nucleus was at rest,
        and the alpha particle has a speed of \SI{1.5e7}{\meter\per\second},
        determine the recoil speed of the thorium nucleus.
    %% ANSWER:  \SI{2.56e5}{\meter\per\second}
    \AMCnumericChoices{2.56}{
        vertical=false,
        digits=3,decimals=2,sign=true,Tsign=\hspace{1ex}\Large a,
        borderwidth=0pt,backgroundcol=white,approx=5
    }
    \AMCnumericChoices{5}{
        vertical=false,
        digits=2,decimals=0,sign=true,Tsign=\hspace{1ex}\Large b,
        borderwidth=0pt,backgroundcol=white,approx=5
    }
\end{questionmultx}
}


\endinput


