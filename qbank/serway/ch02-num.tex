
%%--------------------------------------------------
%% Serway: Physics for Scientists and Engineers
%%--------------------------------------------------


%% Chapter 02: Motion in One Dimension
%%--------------------------------------------------


%% Serway Numberic Questions
%%--------------------------------------------------
\element{serway-num}{
\begin{questionmultx}{serway-ch02-q59}
    A \SI{50}{\gram} superball traveling at \SI{25}{\meter\per\second} is bounced off a brick wall and rebounds at 22 m/s. 
    A high-speed camera records this event. 
    If the ball is in contact with the wall for \SI{3.5}{\milli\second},
        what is the average acceleration of the ball during this time interval?
    %% ANSWER: \SI{13 430}{\meter\per\second\squared}
    \AMCnumericChoices{1.34}{
        vertical=true,
        digits=3,decimals=2,sign=true,Tsign=\hspace{1ex}\Large a,
        borderwidth=0pt,backgroundcol=white,approx=5
    }
    \AMCnumericChoices{4}{
        vertical=true,
        digits=1,decimals=0,sign=true,Tsign=\hspace{1ex}\Large b,
        borderwidth=0pt,backgroundcol=white,approx=5
    }
\end{questionmultx}
}

\element{serway-num}{
\begin{questionmultx}{serway-ch02-q60}
    A boat moves at \SI{10}{\meter\per\second} relative to the water. 
    If the boat is in a river where the current is \SI{2.0}{\meter\per\second},
        how long does it take the boat to make a complete round trip of \SI{1.0}{\kilo\meter} upstream followed by a \SI{1.0}{\kilo\meter} trip downstream?
    %% ANSWER: \SI{208.3}{\second}
    %% ANSWER: \SI{13 430}{\meter\per\second\squared}
    \AMCnumericChoices{2.08}{
        vertical=true,
        digits=3,decimals=2,sign=true,Tsign=\hspace{1ex}\Large a,
        borderwidth=0pt,backgroundcol=white,approx=5
    }
    \AMCnumericChoices{2}{
        vertical=true,
        digits=1,decimals=0,sign=true,Tsign=\hspace{1ex}\Large b,
        borderwidth=0pt,backgroundcol=white,approx=5
    }
\end{questionmultx}
}

\element{serway-num}{
\begin{questionmultx}{serway-ch02-q61}
    A bicyclist starts down a hill with an initial speed of \SI{2.0}{\meter\per\second}. 
    She moves down the hill with a constant acceleration,
    arriving at the bottom of the hill with a speed of \SI{8.0}{\meter\per\second}.
    If the hill is \SI{12}{\meter} long,
        how long did it take the bicyclist to travel down the hill?
    %% ANSWER: \SI{2.4}{\second}
    \AMCnumericChoices{2.40}{
        vertical=true,
        digits=3,decimals=2,sign=true,Tsign=\hspace{1ex}\Large a,
        borderwidth=0pt,backgroundcol=white,approx=5
    }
    \AMCnumericChoices{0}{
        vertical=true,
        digits=1,decimals=0,sign=true,Tsign=\hspace{1ex}\Large b,
        borderwidth=0pt,backgroundcol=white,approx=5
    }
\end{questionmultx}
}

\element{serway-num}{
\begin{questionmultx}{serway-ch02-q62}
    A helicopter descends from a height of \SI{600}{\meter} with uniform negative acceleration,
        reaching the ground at rest in \SI{5.00}{\minute}. 
    Determine the acceleration of the helicopter and its initial downward velocity.
    %% NOTE: split into A and B
    %% ANSWER: \SI{-0.0133}{\meter\per\second\squared} \SI{4.0}{\meter\per\second}
    \AMCnumericChoices{-1.33}{
        vertical=true,
        digits=3,decimals=2,sign=true,Tsign=\hspace{1ex}\Large a,
        borderwidth=0pt,backgroundcol=white,approx=5
    }
    \AMCnumericChoices{-2}{
        vertical=true,
        digits=1,decimals=0,sign=true,Tsign=\hspace{1ex}\Large b,
        borderwidth=0pt,backgroundcol=white,approx=5
    }
\end{questionmultx}
}

\element{serway-num}{
\begin{questionmultx}{serway-ch02-q63}
    A helicopter descends from a height of \SI{600}{\meter} with uniform negative acceleration,
        reaching the ground at rest in \SI{5.00}{\minute}. 
    Determine the acceleration of the helicopter and its initial downward velocity.
    %% ANSWER: \SI{-0.0133}{\meter\per\second\squared} \SI{4.0}{\meter\per\second}
    \AMCnumericChoices{4.0}{
        vertical=true,
        digits=3,decimals=2,sign=true,Tsign=\hspace{1ex}\Large a,
        borderwidth=0pt,backgroundcol=white,approx=5
    }
    \AMCnumericChoices{0}{
        vertical=true,
        digits=1,decimals=0,sign=true,Tsign=\hspace{1ex}\Large b,
        borderwidth=0pt,backgroundcol=white,approx=5
    }
\end{questionmultx}
}

\element{serway-num}{
\begin{questionmultx}{serway-ch02-q64}
    A speedy tortoise can run with a velocity of \SI{10}{\centi\meter\per\second} and a hare can run 20 times as fast. 
    In a race, they both start at the same time,
        but the hare stops to rest for \SI{2.0}{\minute}.
    The tortoise wins by a shell (\SI{20}{\centi\meter}).
    What was the length of the race?
    %% ANSWER: \SI{12.62}{\meter}
    \AMCnumericChoices{1.26}{
        vertical=true,
        digits=3,decimals=2,sign=true,Tsign=\hspace{1ex}\Large a,
        borderwidth=0pt,backgroundcol=white,approx=5
    }
    \AMCnumericChoices{1}{
        vertical=true,
        digits=1,decimals=0,sign=true,Tsign=\hspace{1ex}\Large b,
        borderwidth=0pt,backgroundcol=white,approx=5
    }
\end{questionmultx}
}

\element{serway-num}{
\begin{questionmultx}{serway-ch02-q65}
    A peregrine falcon dives at a pigeon. 
    The falcon starts with zero downward velocity and falls with the acceleration of gravity. 
    If the pigeon is \SI{76.0}{\meter} below the initial height of the falcon,
        how long does it take the falcon to intercept the pigeon?
    %% ANSWER: \SI{3.94}{\second}
    \AMCnumericChoices{3.94}{
        vertical=true,
        digits=3,decimals=2,sign=true,Tsign=\hspace{1ex}\Large a,
        borderwidth=0pt,backgroundcol=white,approx=5
    }
    \AMCnumericChoices{0}{
        vertical=true,
        digits=1,decimals=0,sign=true,Tsign=\hspace{1ex}\Large b,
        borderwidth=0pt,backgroundcol=white,approx=5
    }
\end{questionmultx}
}

\element{serway-num}{
\begin{questionmultx}{serway-ch02-q66}
    Starting from rest, a car travels \SI{1 350}{\meter} in \SI{1.00}{\minute}.
    It accelerated at \SI{1.0}{\meter\per\second\squared} until it reached its cruising speed. 
    Then it drove the remaining distance at constant velocity. 
    What was its cruising speed?
    %% ANSWER: \SI{30}{\meter\per\second}
    \AMCnumericChoices{3.0}{
        vertical=true,
        digits=3,decimals=2,sign=true,Tsign=\hspace{1ex}\Large a,
        borderwidth=0pt,backgroundcol=white,approx=5
    }
    \AMCnumericChoices{1}{
        vertical=true,
        digits=1,decimals=0,sign=true,Tsign=\hspace{1ex}\Large b,
        borderwidth=0pt,backgroundcol=white,approx=5
    }
\end{questionmultx}
}

\element{serway-num}{
\begin{questionmultx}{serway-ch02-q67}
    A car originally traveling at \SI{30}{\meter\per\second} manages to brake for \SI{5.0}{\second} while traveling \SI{125}{\meter} downhill. 
    At that point the brakes fail. 
    After an additional \SI{5.0}{\second} it travels an additional \SI{150}{\meter} down the hill. 
    What was the acceleration of the car after the brakes failed?
    %% ANSWER: \SI{4.0}{\meter\per\second\squared}
    \AMCnumericChoices{4.0}{
        vertical=true,
        digits=3,decimals=2,sign=true,Tsign=\hspace{1ex}\Large a,
        borderwidth=0pt,backgroundcol=white,approx=5
    }
    \AMCnumericChoices{0}{
        vertical=true,
        digits=1,decimals=0,sign=true,Tsign=\hspace{1ex}\Large b,
        borderwidth=0pt,backgroundcol=white,approx=5
    }
\end{questionmultx}
}


\endinput


