
%%--------------------------------------------------
%% Serway: Physics for Scientists and Engineers
%%--------------------------------------------------


%% Chapter 20: Heat and the First Law of Thermodynamics
%%--------------------------------------------------


%% Table of Contents
%%--------------------------------------------------

%% 20.1 Heat and Internal Energy
%% 20.2 Specific Heat and Calorimetry
%% 20.3 Latent Heat
%% 20.4 Work and Heat in Thermodynamic Processes
%% 20.5 The First Law of Thermodynamics
%% 20.6 Some Applications of the First Law of Thermodynamics
%% 20.7 Energy Transfer Mechanisms


%% Serway Multiple Choice Questions
%%--------------------------------------------------
\element{serway-mc}{
\begin{question}{serway-ch20-q01}
    Determine the heat capacity of a lake containing one million gallons (approximately 4 million kilograms) of water at \SI{15}{\degreeCelsius}.
    \begin{multicols}{2}
    \begin{choices}
        \wrongchoice{\SI{4e6}{\calorie\per\degreeCelsius}}
      \correctchoice{\SI{4e9}{\calorie\per\degreeCelsius}}
        \wrongchoice{\SI{4e3}{\calorie\per\degreeCelsius}}
        \wrongchoice{\SI{1e3}{\calorie\per\degreeCelsius}}
        \wrongchoice{\SI{4e2}{\calorie\per\degreeCelsius}}
    \end{choices}
    \end{multicols}
\end{question}
}

\element{serway-mc}{
\begin{question}{serway-ch20-q02}
    How many calories of heat are required to raise the temperature of \SI{4}{\kilo\gram} of water from \SI{50}{\degree\Fahrenheit} to the boiling point?
    \begin{multicols}{2}
    \begin{choices}
        \wrongchoice{\SI{6.5e5}{\calorie}}
      \correctchoice{\SI{3.6e5}{\calorie}}
        \wrongchoice{\SI{15e5}{\calorie}}
        \wrongchoice{\SI{360}{\calorie}}
        \wrongchoice{\SI{4e4}{\calorie}}
    \end{choices}
    \end{multicols}
\end{question}
}

\element{serway-mc}{
\begin{question}{serway-ch20-q03}
    A \SI{5}{\gallon} container of water (approximately \SI{20}{\kilo\gram}) having a temperature of \SI{212}{\degree\Fahrenheit} is added to a \SI{50}{\gallon} tub (approximately \SI{200}{\kilo\gram}) of water having a temperature of \SI{50}{\degree\Fahrenheit}. 
    What is the final equilibrium temperature of the mixture?
    \begin{multicols}{3}
    \begin{choices}
        \wrongchoice{\SI{54}{\degreeCelsius}}
        \wrongchoice{\SI{36}{\degreeCelsius}}
      \correctchoice{\SI{18}{\degreeCelsius}}
        \wrongchoice{\SI{66}{\degreeCelsius}}
        \wrongchoice{\SI{14}{\degreeCelsius}}
    \end{choices}
    \end{multicols}
\end{question}
}

\element{serway-mc}{
\begin{question}{serway-ch20-q04}
    A \SI{5}{\kilo\gram} piece of lead (specific heat \SI{0.03}{\calorie\per\gram\per\degreeCelsius}) having a temperature of \SI{80}{\degreeCelsius} is added to \SI{500}{\gram} of water having a temperature of \SI{20}{\degreeCelsius}. 
    What is the final equilibrium temperature of the system?
    \begin{multicols}{3}
    \begin{choices}
        \wrongchoice{\SI{79}{\degreeCelsius}}
        \wrongchoice{\SI{26}{\degreeCelsius}}
        \wrongchoice{\SI{54}{\degreeCelsius}}
      \correctchoice{\SI{34}{\degreeCelsius}}
        \wrongchoice{\SI{20}{\degreeCelsius}}
    \end{choices}
    \end{multicols}
\end{question}
}

\element{serway-mc}{
\begin{question}{serway-ch20-q05}
    A \SI{300}{\gram} glass thermometer initially at \SI{25}{\degreeCelsius} is put into \SI{200}{\centi\meter\cubed} of hot water at \SI{95}{\degreeCelsius}.
    Find the final temperature of the thermometer,
        assuming no heat flows to the surroundings. 
    (The specific heat of glass is \SI{0.2}{\calorie\per\gram\per\degreeCelsius}.)
    \begin{multicols}{3}
    \begin{choices}
        \wrongchoice{\SI{52}{\degreeCelsius}}
        \wrongchoice{\SI{68}{\degreeCelsius}}
        \wrongchoice{\SI{89}{\degreeCelsius}}
      \correctchoice{\SI{79}{\degreeCelsius}}
        \wrongchoice{\SI{36}{\degreeCelsius}}
    \end{choices}
    \end{multicols}
\end{question}
}

\element{serway-mc}{
\begin{question}{serway-ch20-q06}
    How much heat is needed to convert \SI{1.00}{\kilo\gram} of ice at \SI{0}{\degreeCelsius} into steam at \SI{100}{\degreeCelsius}?
    \begin{multicols}{2}
    \begin{choices}
        \wrongchoice{\SI{23.9}{\kilo\calorie}}
        \wrongchoice{\SI{79.6}{\kilo\calorie}}
        \wrongchoice{\SI{564}{\kilo\calorie}}
        \wrongchoice{\SI{643}{\kilo\calorie}}
      \correctchoice{\SI{720}{\kilo\calorie}}
    \end{choices}
    \end{multicols}
\end{question}
}

\element{serway-mc}{
\begin{question}{serway-ch20-q07}
    If \SI{25}{\kilo\gram} of ice at \SI{0}{\degreeCelsius} is combined with \SI{4}{\kilo\gram} of steam at \SI{100}{\degreeCelsius},
        what will be the final equilibrium temperature of the system?
    \begin{multicols}{3}
    \begin{choices}
        \wrongchoice{\SI{40}{\degreeCelsius}}
      \correctchoice{\SI{20}{\degreeCelsius}}
        \wrongchoice{\SI{60}{\degreeCelsius}}
        \wrongchoice{\SI{100}{\degreeCelsius}}
        \wrongchoice{\SI{8}{\degreeCelsius}}
    \end{choices}
    \end{multicols}
\end{question}
}

\element{serway-mc}{
\begin{question}{serway-ch20-q08}
    How much heat must be removed to make ice at \SI{-10}{\degreeCelsius} from \SI{2}{\kilo\gram} of water at \SI{20}{\degreeCelsius}?
    (The specific heat of ice is \SI{0.5}{\calorie\per\gram\per\degreeCelsius}.)
    \begin{multicols}{3}
    \begin{choices}
        \wrongchoice{\SI{190}{\kilo\calorie}}
        \wrongchoice{\SI{200}{\kilo\calorie}}
        \wrongchoice{\SI{240}{\kilo\calorie}}
      \correctchoice{\SI{210}{\kilo\calorie}}
        \wrongchoice{\SI{50}{\kilo\calorie}}
    \end{choices}
    \end{multicols}
\end{question}
}

\element{serway-mc}{
\begin{question}{serway-ch20-q09}
    The $R$ value of fiberglass batting, \num{3.5} inches thick,
        is \num{11} \si{\foot\squared\degree\Fahrenheit\hour}/BTU.
    What is the thermal conductivity?
    \begin{choices}
      \correctchoice{\num{7.4e-6} BTU/(\si{\foot\squared\degree\Fahrenheit\hour})}
        \wrongchoice{\num{2.7e-2} BTU/(\si{\foot\squared\degree\Fahrenheit\hour})}
        \wrongchoice{\num{8.9e-5} BTU/(\si{\foot\squared\degree\Fahrenheit\hour})}
        \wrongchoice{\num{1.4e-4} BTU/(\si{\foot\squared\degree\Fahrenheit\hour})}
        \wrongchoice{\num{3.6e-3} BTU/(\si{\foot\squared\degree\Fahrenheit\hour})}
    \end{choices}
\end{question}
}

\element{serway-mc}{
\begin{question}{serway-ch20-q10}
    A slab of concrete and an insulating board are in thermal contact with each other.
    The temperatures of their outer surfaces are \SI{68}{\degree\Fahrenheit} and \SI{50}{\degree\Fahrenheit}. 
    Determine the rate of heat transfer if the $R$ values are \num{1.93} and \num{8.7} \si{\foot\squared\degree\Fahrenheit\hour}/BTU, respectively.
    \begin{multicols}{2}
    \begin{choices}
        \wrongchoice{\num{9.7} BTU/(\si{\foot\squared\hour})}
        \wrongchoice{\num{2.5} BTU/(\si{\foot\squared\hour})}
        \wrongchoice{\num{5.3} BTU/(\si{\foot\squared\hour})}
      \correctchoice{\num{1.7} BTU/(\si{\foot\squared\hour})}
        \wrongchoice{\num{28} BTU/(\si{\foot\squared\hour})}
    \end{choices}
    \end{multicols}
\end{question}
}

\element{serway-mc}{
\begin{question}{serway-ch20-q11}
    A wall is constructed of a \num{2} inch layer of fiberglass board ($R=8$) and six inches of fiberglass batting ($R=19$). 
    If the temperature on the outside surface of the fiberglass board is \SI{50}{\degree\Fahrenheit} and the temperature on the inside surface of the fiberglass batting is \SI{68}{\degree\Fahrenheit},
        what is the temperature at the interface? 
    (The units of $R$ are \si{\foot\squared\degree\Fahrenheit\hour}/BTU.)
    \begin{multicols}{3}
    \begin{choices}
        \wrongchoice{\SI{62}{\degree\Fahrenheit}}
        \wrongchoice{\SI{58}{\degree\Fahrenheit}}
      \correctchoice{\SI{55}{\degree\Fahrenheit}}
        \wrongchoice{\SI{65}{\degree\Fahrenheit}}
        \wrongchoice{\SI{52}{\degree\Fahrenheit}}
    \end{choices}
    \end{multicols}
\end{question}
}

\element{serway-mc}{
\begin{question}{serway-ch20-q12}
    A cup of coffee is enclosed on all sides in an insulated cup \SI{1/2}{\centi\meter} thick in the shape of a cube \SI{10}{\centi\meter} on a side. 
    The temperature of the coffee is \SI{95}{\degreeCelsius},
        and the temperature of the surroundings is \SI{21}{\degreeCelsius}. 
    Find the rate of heat loss due to conduction if the thermal conductivity of the cup is \SI{2e-4}{\calorie\per\second\per\centi\meter\per\degreeCelsius}.
    \begin{multicols}{3}
    \begin{choices}
        \wrongchoice{\SI{62}{\joule\per\second}}
      \correctchoice{\SI{74}{\joule\per\second}}
        \wrongchoice{\SI{230}{\joule\per\second}}
        \wrongchoice{\SI{160}{\joule\per\second}}
        \wrongchoice{\SI{12}{\joule\per\second}}
    \end{choices}
    \end{multicols}
\end{question}
}

\element{serway-mc}{
\begin{question}{serway-ch20-q13}
    A child has a temperature of \SI{101}{\degree\Fahrenheit}. 
    If her total cross-sectional area is \SI{2}{\meter\squared},
        find the energy lost each second due to radiation,
        assuming the emissivity is \num{1}.
    (Assume the room temperature is \SI{70}{\degree\Fahrenheit}.)
    \begin{multicols}{3}
    \begin{choices}
      \correctchoice{\SI{217}{\watt}}
        \wrongchoice{\SI{180}{\watt}}
        \wrongchoice{\SI{90}{\watt}}
        \wrongchoice{\SI{68}{\watt}}
        \wrongchoice{\SI{850}{\watt}}
    \end{choices}
    \end{multicols}
\end{question}
}

\element{serway-mc}{
\begin{question}{serway-ch20-q14}
    A super-insulated house is at a temperature of \SI{20}{\degreeCelsius}.
    The temperature outside is \SI{0}{\degreeCelsius}.
    The surface area of the house is \SI{200}{\meter\squared},
        and the emissivity is \num{1}.
    Approximately how much energy is radiated (in \si{\watt}) per second?
    \begin{multicols}{2}
    \begin{choices}
      \correctchoice{\SI{20 000}{\watt}}
        \wrongchoice{\SI{2000}{\watt}}
        \wrongchoice{\SI{200}{\watt}}
        \wrongchoice{\SI{2}{\watt}}
        \wrongchoice{\SI{0.2}{\watt}}
    \end{choices}
    \end{multicols}
\end{question}
}

\element{serway-mc}{
\begin{question}{serway-ch20-q15}
    A \SI{100}{\kilo\gram} student eats a \SI{200}{\kilo\calorie} doughnut. 
    To ``burn it off'', he decides to climb the steps of a tall building.
    How high would he have to climb to expend an equivalent amount of work?
    %% NOTE: changed Calorie to kilocalorie
    %(1 food Calorie = 10 3 calories.)
    \begin{multicols}{3}
    \begin{choices}
        \wrongchoice{\SI{273}{\meter}}
        \wrongchoice{\SI{623}{\meter}}
        \wrongchoice{\SI{418}{\meter}}
      \correctchoice{\SI{854}{\meter}}
        \wrongchoice{\SI{8400}{\meter}}
    \end{choices}
    \end{multicols}
\end{question}
}

\element{serway-mc}{
\begin{question}{serway-ch20-q16}
    A \SI{5}{\gram} coin is dropped from a \SI{300}{\meter} building. 
    If it reaches a terminal velocity of \SI{45}{\meter\per\second},
        and the rest of the energy is converted to heating the coin,
        what is the change in temperature of the coin? 
    (The specific heat of copper is \SI{387}{\joule\per\kilo\gram\per\degreeCelsius}.)
    \begin{multicols}{3}
    \begin{choices}
        \wrongchoice{\SI{9}{\degreeCelsius}}
        \wrongchoice{\SI{2}{\degreeCelsius}}
      \correctchoice{\SI{5}{\degreeCelsius}}
        \wrongchoice{\SI{21}{\degreeCelsius}}
        \wrongchoice{\SI{0.5}{\degreeCelsius}}
    \end{choices}
    \end{multicols}
\end{question}
}

\element{serway-mc}{
\begin{question}{serway-ch20-q17}
    The work done in the expansion from an initial to a final state:
    \begin{choices}
      \correctchoice{is the area under the curve of a PV diagram.}
        \wrongchoice{depends only on the end point.}
        \wrongchoice{is independent of the path.}
        \wrongchoice{is the slope of a PV curve.}
        \wrongchoice{equals $P\left(V_F-V_i\right)$.}
    \end{choices}
\end{question}
}

\element{serway-mc}{
\begin{question}{serway-ch20-q18}
    Gas in a container expands at a constant pressure of \SI{3}{\atm}. 
    Find the work done by the gas if the initial volume is \SI{5}{\liter} and the final volume is \SI{10}{\liter}.
    \begin{multicols}{2}
    \begin{choices}
        \wrongchoice{\SI{0}{\joule}}
        \wrongchoice{\SI{150}{\joule}}
        \wrongchoice{\SI{15}{\joule}}
      \correctchoice{\SI{1500}{\joule}}
        \wrongchoice{\SI{1.5}{\joule}}
    \end{choices}
    \end{multicols}
\end{question}
}

\element{serway-mc}{
\begin{question}{serway-ch20-q19}
    Gas in a container increases its pressure from \SI{1}{\atm} to \SI{3}{\atm} while keeping its volume constant.
    Find the work done by the gas if the volume is \SI{5}{\liter}.
    \begin{multicols}{2}
    \begin{choices}
      \correctchoice{\SI{0}{\joule}}
        \wrongchoice{\SI{3}{\joule}}
        \wrongchoice{\SI{5}{\joule}}
        \wrongchoice{\SI{15}{\joule}}
        \wrongchoice{\SI{15e2}{\joule}}
    \end{choices}
    \end{multicols}
\end{question}
}

\element{serway-mc}{
\begin{question}{serway-ch20-q20}
    In an adiabatic free expansion:
    \begin{choices}
      \correctchoice{no heat is transferred between a system and its surroundings.}
        \wrongchoice{the pressure remains constant.}
        \wrongchoice{the temperature remains constant.}
        \wrongchoice{the volume remains constant.}
        \wrongchoice{the process is reversible.}
    \end{choices}
\end{question}
}

\element{serway-mc}{
\begin{question}{serway-ch20-q21}
    In an isothermal process:
    \begin{choices}
        \wrongchoice{the temperature remains constant.}
      \correctchoice{no heat is transferred between a system and its surroundings.}
        \wrongchoice{the pressure remains constant.}
        \wrongchoice{the volume remains constant.}
        \wrongchoice{the internal energy is constant.}
    \end{choices}
\end{question}
}

\element{serway-mc}{
\begin{question}{serway-ch20-q22}
    In an isobaric process:
    \begin{choices}
        \wrongchoice{the volume remains constant.}
        \wrongchoice{the temperature remains constant.}
      \correctchoice{the pressure remains constant.}
        \wrongchoice{no heat is transferred between a system and its surroundings.}
        \wrongchoice{the internal energy is constant.}
    \end{choices}
\end{question}
}

\element{serway-mc}{
\begin{question}{serway-ch20-q23}
    In an isovolumetric process:
    \begin{choices}
        \wrongchoice{the volume remains constant.}
        \wrongchoice{the temperature remains constant.}
        \wrongchoice{no heat is transferred between a system and its surroundings.}
      \correctchoice{the pressure remains constant.}
        \wrongchoice{the internal energy is not constant.}
    \end{choices}
\end{question}
}

\element{serway-mc}{
\begin{question}{serway-ch20-q24}
    Determine the work done by \SI{5}{\mole} of an ideal gas that is kept at \SI{100}{\degreeCelsius} in an expansion from \SI{1}{\liter} to \SI{5}{\liter}.
    \begin{multicols}{2}
    \begin{choices}
      \correctchoice{\SI{2.5e4}{\joule}}
        \wrongchoice{\SI{1.1e4}{\joule}}
        \wrongchoice{\SI{6.7e3}{\joule}}
        \wrongchoice{\SI{2.9e3}{\joule}}
        \wrongchoice{\SI{8.4e3}{\joule}}
    \end{choices}
    \end{multicols}
\end{question}
}

\element{serway-mc}{
\begin{question}{serway-ch20-q25}
    One gram of water is heated from \SI{0}{\degreeCelsius} to \SI{100}{\degreeCelsius} at a constant pressure of \SI{1}{\atm}.
    Determine the approximate change in internal energy of the water.
    \begin{multicols}{2}
    \begin{choices}
        \wrongchoice{\SI{160}{\calorie}}
        \wrongchoice{\SI{130}{\calorie}}
      \correctchoice{\SI{100}{\calorie}}
        \wrongchoice{\SI{180}{\calorie}}
        \wrongchoice{\SI{50}{\calorie}}
    \end{choices}
    \end{multicols}
\end{question}
}

\element{serway-mc}{
\begin{question}{serway-ch20-q26}
    Five moles of an ideal gas expands isothermally at \SI{100}{\degreeCelsius} to five times its initial volume. 
    Find the heat flow into the system.
    \begin{multicols}{2}
    \begin{choices}
      \correctchoice{\SI{2.5e4}{\joule}}
        \wrongchoice{\SI{1.1e4}{\joule}}
        \wrongchoice{\SI{6.7e3}{\joule}}
        \wrongchoice{\SI{2.9e3}{\joule}}
        \wrongchoice{\SI{7.0e2}{\joule}}
    \end{choices}
    \end{multicols}
\end{question}
}

\element{serway-mc}{
\begin{question}{serway-ch20-q27}
    An \SI{8 000}{\kilo\gram} aluminum flagpole \SI{100}{\meter} long is heated by the sun from a temperature of \SI{10}{\degreeCelsius} to \SI{20}{\degreeCelsius}.
    Find the work done by the aluminum if the linear expansion coefficient is \SI{24e-6}{\per\degreeCelsius}. 
    (The density of aluminum is \SI{2.7e3}{\kilo\gram\per\meter\cubed} and $\SI{1}{\atm}=\SI{1.0e5}{\newton\per\meter\squared}$.)
    \begin{multicols}{3}
    \begin{choices}
        \wrongchoice{\SI{287}{\joule}}
        \wrongchoice{\SI{425}{\joule}}
      \correctchoice{\SI{213}{\joule}}
        \wrongchoice{\SI{710}{\joule}}
        \wrongchoice{\SI{626}{\joule}}
    \end{choices}
    \end{multicols}
\end{question}
}

\element{serway-mc}{
\begin{question}{serway-ch20-q28}
    An \SI{8 000}{\kilo\gram} aluminum flagpole \SI{100}{\meter} long is heated by the sun from a temperature of \SI{10}{\degreeCelsius} to \SI{20}{\degreeCelsius}.
    Find the heat transferred to the aluminum if the specific heat of aluminum is \SI{0.215}{\calorie\per\gram\per\degreeCelsius}.
    \begin{multicols}{2}
    \begin{choices}
        \wrongchoice{\SI{7.2e5}{\joule}}
      \correctchoice{\SI{7.2e7}{\joule}}
        \wrongchoice{\SI{7.2e3}{\joule}}
        \wrongchoice{\SI{7.2e1}{\joule}}
        \wrongchoice{\SI{7.2e2}{\joule}}
    \end{choices}
    \end{multicols}
\end{question}
}

\element{serway-mc}{
\begin{question}{serway-ch20-q29}
    An \SI{8 000}{\kilo\gram} aluminum flagpole \SI{100}{\meter} high is heated by the sun from a temperature of \SI{10}{\degreeCelsius} to \SI{20}{\degreeCelsius}.
    Find the increase in internal energy (in J) of the aluminum. 
    (The coefficient of linear expansion is \SI{24e-6}{\per\degreeCelsius},
        the density is \SI{2.7e3}{\kilo\gram\per\meter\cubed},
        and the specific heat of aluminum is \SI{0.215}{\calorie\per\gram\per\degreeCelsius}.)
    \begin{multicols}{2}
    \begin{choices}
        \wrongchoice{\SI{7.2e5}{\joule}}
      \correctchoice{\SI{7.2e7}{\joule}}
        \wrongchoice{\SI{7.2e3}{\joule}}
        \wrongchoice{\SI{7.2e1}{\joule}}
        \wrongchoice{\SI{7.2e2}{\joule}}
    \end{choices}
    \end{multicols}
\end{question}
}

\element{serway-mc}{
\begin{question}{serway-ch20-q30}
    Two kilograms of water at \SI{100}{\degreeCelsius} is converted to steam at \SI{1}{\atm}.
    Find the work done. 
    (The density of steam at \SI{100}{\degreeCelsius} is \SI{0.598}{\kilo\gram\per\meter\cubed}.)
    \begin{multicols}{2}
    \begin{choices}
        \wrongchoice{\SI{3.4e5}{\joule}}
      \correctchoice{\SI{1.2e5}{\joule}}
        \wrongchoice{\SI{4.6e4}{\joule}}
        \wrongchoice{\SI{2.1e4}{\joule}}
        \wrongchoice{\SI{3.4e4}{\joule}}
    \end{choices}
    \end{multicols}
\end{question}
}

\element{serway-mc}{
\begin{question}{serway-ch20-q31}
    Two kilograms of water at \SI{100}{\degreeCelsius} is converted to steam at \SI{1}{\atm}. 
    Find the change in internal energy. 
    ($L_v =\SI{2.26e6}{\joule\per\kilo\gram}$.)
    \begin{multicols}{2}
    \begin{choices}
        \wrongchoice{\SI{2.1e4}{\joule}}
        \wrongchoice{\SI{4.5e6}{\joule}}
        \wrongchoice{\SI{3.4e5}{\joule}}
      \correctchoice{\SI{4.2e6}{\joule}}
        \wrongchoice{\SI{2.1e6}{\joule}}
    \end{choices}
    \end{multicols}
\end{question}
}

\element{serway-mc}{
\begin{question}{serway-ch20-q32}
    If an object feels cold to the touch,
        the only statement that you can make that must be correct is that:
    \begin{choices}
        \wrongchoice{the object has a smaller coefficient of thermal conductivity than your hand.}
        \wrongchoice{the volume of the object will increase while it is in contact with your hand.}
        \wrongchoice{the object contains less thermal energy than your hand.}
      \correctchoice{the object is at a lower temperature than your hand.}
        \wrongchoice{the object cannot be a liquid.}
    \end{choices}
\end{question}
}

\element{serway-mc}{
\begin{question}{serway-ch20-q33}
    Which of the following statements is correct?
    \begin{choices}
        \wrongchoice{You only need to know the amount of thermal energy a body contains to calculate its temperature.}
        \wrongchoice{The temperature of a body is directly proportional to the amount of work the body has performed.}
        \wrongchoice{The quantity of thermal energy exchanged by two bodies in contact is inversely proportional to the difference in their temperatures.}
      \correctchoice{The quantity of thermal energy exchanged by two bodies in contact is directly proportional to the difference in their temperatures.}
        \wrongchoice{Different amounts of thermal energy are transferred between two bodies in contact if different temperature scales are used to measure the temperature difference between the bodies.}
    \end{choices}
\end{question}
}

\element{serway-mc}{
\begin{question}{serway-ch20-q34}
    In which process will the internal energy of the system \emph{not} change?
    \begin{choices}
        \wrongchoice{An adiabatic expansion of an ideal gas.}
      \correctchoice{An isothermal compression of an ideal gas.}
        \wrongchoice{An isobaric expansion of an ideal gas.}
        \wrongchoice{The freezing of a quantity of liquid at its melting point.}
        \wrongchoice{The evaporation of a quantity of a liquid at its boiling point.}
    \end{choices}
\end{question}
}

\element{serway-mc}{
\begin{question}{serway-ch20-q35}
    For an astronaut working outside a spaceship,
        the greatest loss of heat would occur by means of:
    \begin{choices}
        \wrongchoice{conduction.}
        \wrongchoice{convection.}
      \correctchoice{radiation.}
        \wrongchoice{conduction and convection.}
        \wrongchoice{conduction and radiation.}
    \end{choices}
\end{question}
}

\element{serway-mc}{
\begin{question}{serway-ch20-q36}
    Which statement below regarding the First Law of Thermodynamics is most correct?
    \begin{choices}
        \wrongchoice{A system can do work externally only if its internal energy decreases.}
        \wrongchoice{The internal energy of a system that interacts with its environment must change.}
        \wrongchoice{No matter what other interactions take place, the internal energy must change if a system undergoes a heat transfer.}
        \wrongchoice{The only changes that can occur in the internal energy of a system are those produced by non-mechanical forces.}
      \correctchoice{The internal energy of a system cannot change if the heat transferred to the system is equal to the work done by the system.}
    \end{choices}
\end{question}
}

\element{serway-mc}{
\begin{question}{serway-ch20-q37}
    How much heat is required to convert \SI{1.00}{\kilo\gram} of ice at \SI{0}{\degreeCelsius} into steam at \SI{100}{\degreeCelsius}?
    ($L_{\text{ice}}=\SI{333}{\joule\per\gram}$; $L_{\text{steam}}=\SI{2.26e3}{\joule\per\gram}$.)
    \begin{multicols}{2}
    \begin{choices}
        \wrongchoice{\SI{3.35e5}{\joule}}
        \wrongchoice{\SI{4.19e5}{\joule}}
        \wrongchoice{\SI{2.36e6}{\joule}}
        \wrongchoice{\SI{2.69e6}{\joule}}
      \correctchoice{\SI{3.01e6}{\joule}}
    \end{choices}
    \end{multicols}
\end{question}
}

\element{serway-mc}{
\begin{question}{serway-ch20-q38}
    Water at room temperature, \SI{20}{\degreeCelsius},
        is pumped into a reactor core where it is converted to steam at \SI{200}{\degreeCelsius}. 
    How much heat is transferred to each kilogram of water in this process? 
    ($c_{\text{steam}}=\SI{2010}{\joule\per\kilo\gram\per\degreeCelsius}$; $L_{\text{steam}}= \SI{2.26e3}{\joule\per\gram}$; $\SI{1}{\calorie}=\SI{4.186}{\joule}$)
    \begin{multicols}{2}
    \begin{choices}
        \wrongchoice{\SI{3.35e5}{\joule}}
        \wrongchoice{\SI{7.53e5}{\joule}}
        \wrongchoice{\SI{2.67e6}{\joule}}
      \correctchoice{\SI{2.80e6}{\joule}}
        \wrongchoice{\SI{3.01e6}{\joule}}
    \end{choices}
    \end{multicols}
\end{question}
}

\element{serway-mc}{
\begin{question}{serway-ch20-q39}
    A gas expands from $A$ to $B$ as shown in the graph. 
    \begin{center}
    \begin{tikzpicture}
        %% NOTE:
    \end{tikzpicture}
    \end{center}
    Calculate the work done by the gas. 
    ($\SI{1}{\atm}=\SI{1.01e5}{\newton\per\meter\squared}$.)
    \begin{multicols}{2}
    \begin{choices}
        \wrongchoice{\SI{12}{\joule}}
        \wrongchoice{\SI{24}{\joule}}
        \wrongchoice{\SI{1.21e6}{\joule}}
      \correctchoice{\SI{2.42e6}{\joule}}
        \wrongchoice{\SI{3.64e6}{\joule}}
    \end{choices}
    \end{multicols}
\end{question}
}

\element{serway-mc}{
\begin{question}{serway-ch20-q40}
    A gas expands as shown in the graph. 
    \begin{center}
    \begin{tikzpicture}
        %% NOTE:
    \end{tikzpicture}
    \end{center}
    If the heat taken in during this process is \SI{1.02e6}{\joule} and $\SI{1}{\atm}=\SI{1.01e5}{\newton\per\meter\squared}$,
        the change in internal energy of the gas is:
    \begin{multicols}{2}
    \begin{choices}
        \wrongchoice{\SI{-2.42e6}{\joule}}
      \correctchoice{\SI{-1.40e6}{\joule}}
        \wrongchoice{\SI{-1.02e6}{\joule}}
        \wrongchoice{\SI{1.02e6}{\joule}}
        \wrongchoice{\SI{1.40e6}{\joule}}
    \end{choices}
    \end{multicols}
\end{question}
}

\element{serway-mc}{
\begin{question}{serway-ch20-q41}
    In a thermodynamic process,
        the internal energy of a system in a container with adiabatic walls decreases by \SI{800}{\joule}. 
    Which statement is correct?
    \begin{choices}
        \wrongchoice{The system lost \SI{800}{\joule} by heat transfer to its surroundings.}
        \wrongchoice{The system gained \SI{800}{\joule} by heat transfer from its surroundings.}
      \correctchoice{The system performed \SI{800}{\joule} of work on its surroundings.}
        \wrongchoice{The surroundings performed \SI{800}{\joule} of work on the system.}
        \wrongchoice{The \SI{800}{\joule} of work done by the system was equal to the \SI{800}{\joule} of heat transferred to the system from its surroundings.}
    \end{choices}
\end{question}
}

\element{serway-mc}{
\begin{question}{serway-ch20-q42}
    If a person in Alaska were locked out of his house on a day when the temperature outside was \SI{-40}{\degreeCelsius},
        he would be most likely to lose the most thermal energy by:
    \begin{choices}
        \wrongchoice{conduction.}
        \wrongchoice{convection.}
        \wrongchoice{radiation.}
        \wrongchoice{all of the above.}
      \correctchoice{convection and radiation.}
    \end{choices}
\end{question}
}

\element{serway-mc}{
\begin{question}{serway-ch20-q43}
    The Earth intercepts \SI{1.27e17}{\watt} of radiant energy from the Sun. 
    Suppose the Earth, of volume \SI{1.08e21}{\meter\cubed},
        was composed of water. 
    How long would it take for the Earth at \SI{0}{\degreeCelsius} to reach \SI{100}{\degreeCelsius},
        if none of the energy was radiated or reflected back out into space?
    \begin{multicols}{2}
    \begin{choices}
        \wrongchoice{\SI{26.9}{\year}}
        \wrongchoice{\SI{113}{\year}}
        \wrongchoice{\SI{2.69e4}{\year}}
      \correctchoice{\SI{1.13e5}{\year}}
        \wrongchoice{\SI{2.69e7}{\year}}
    \end{choices}
    \end{multicols}
\end{question}
}

\element{serway-mc}{
\begin{question}{serway-ch20-q44}
    A team of people who traveled to the North Pole by dogsled lived on butter because they needed to consume \num{6 000} dietician's calories each day. 
    Because the ice there is lumpy and irregular,
        they had to help the dogs by pushing and lifting the load. 
    Assume they had a \num{16} hour working day and that each person could lift a \SI{500}{\newton} load.
    How many times would a person have to lift this weight \SI{1.00}{\meter} upwards in a constant gravitational field where $g=\SI{9.80}{\meter\per\second\squared}$ equivalent to \num{6 000} Calories?
    \begin{multicols}{2}
    \begin{choices}
        \wrongchoice{\num{50.2}}
        \wrongchoice{\num{492}}
        \wrongchoice{\num{5130}}
      \correctchoice{\num{50 200}}
        \wrongchoice{\num{492 000}}
    \end{choices}
    \end{multicols}
\end{question}
}

\element{serway-mc}{
\begin{questionmult}{serway-ch20-q45}
    Duff states that equal masses of all substances have equal changes in internal energy when they have equal changes in temperature. 
    Javan states that the change in internal energy is equal to a constant times the change in temperature for every $\Delta T$,
        no matter how large or how small $\Delta T$ is,
        but that the constant is different for different substances. 
    Which one, if either, is correct?
    \begin{choices}
      \correctchoice{Neither, because the specific heat depends on the substance and may vary with temperature.}
      \correctchoice{Neither, because a change of state may involve release or absorption of latent heat.}
      \correctchoice{Neither, because a substance may do work during a temperature change.}
        %\correctchoice{All of the statements above are correct.}
        %\wrongchoice{Only statements (a) and (b) are correct.}
    \end{choices}
\end{questionmult}
}

\element{serway-mc}{
\begin{questionmult}{serway-ch20-q46}
    Which of the following statements is(are) correct when an ideal gas goes from an initial to a final state in a single process?
    \begin{choices}
      \correctchoice{No work is done on or by the gas when the volume remains constant.}
        \wrongchoice{No energy is transferred into or out of the gas as heat transfer when the temperature remains constant.}
        \wrongchoice{The internal energy of the gas does not change when the pressure remains constant.}
        %\wrongchoice{All the statements above are correct.}
        %\wrongchoice{Only statements (a) and (b) above are correct.}
    \end{choices}
\end{questionmult}
}

\element{serway-mc}{
\begin{question}{serway-ch20-q47}
    Beryl states that insulation with the smallest possible thermal conductivity is best to keep a house warm in winter,
        but worst for keeping a house cool in summer.
    Sapphire insists the reverse is true: low thermal conductivity is good in the summer,
        but bad in the winter. 
    Which one, if either is correct?
    \begin{choices}
        \wrongchoice{Beryl, because low thermal conductivity results in low heat transfer.}
        \wrongchoice{Beryl, because low thermal conductivity results in high heat transfer.}
        \wrongchoice{Sapphire, because low thermal conductivity results in low heat transfer.}
        \wrongchoice{Sapphire, because low thermal conductivity results in high heat transfer.}
      \correctchoice{Neither, because low heat transfer is desirable both in summer and in winter.}
    \end{choices}
\end{question}
}

\element{serway-mc}{
\begin{question}{serway-ch20-q48}
    The $R$-value of an insulating material is the thickness fo the material divided by its thermal conductivity. 
    When an insulating material consists of three layers with $R$-values $R_1$, $R_2$ and $R_3$,
        the overall $R$-value of the insulation is given by:
    \begin{choices}
      \correctchoice{$R = R_1 + R_2 + R_3$}
        \wrongchoice{$R = \dfrac{R_1 R_2 + R_1 R_3 + R_2 R_3}{R_1 + R_2 + R_3}$}
        \wrongchoice{$R = \dfrac{R_1 R_2 R_3}{R_1 R_2 + R_1 R_3 + R_2 R_3}$}
        \wrongchoice{$\dfrac{1}{R} = \dfrac{1}{R_1 + R_2 + R_3}$}
        \wrongchoice{$\dfrac{1}{R} = \dfrac{1}{R_1} + \dfrac{1}{R_2} + \dfrac{1}{R_3}$}
    \end{choices}
\end{question}
}

\element{serway-mc}{
\begin{question}{serway-ch20-q49}
    A \SI{100}{\kilo\gram} marble slab falls off a skyscraper and falls \SI{200}{\meter} to the ground without hitting anyone. 
    Its fall stops within milliseconds,
        so that there is no loss of thermal energy to its surroundings if its temperature is measured immediately after it stops. 
    By how much has its temperature changed as a result of the fall,
        if we ignore energy gained or lost as a result of its interaction with the atmosphere?
    $\left(c_{\text{marble}} = \SI{860}{\joule\per\kilo\gram\per\degreeCelsius}\right)$
    \begin{multicols}{3}
    \begin{choices}
        \wrongchoice{\SI{0.57}{\degreeCelsius}}
        \wrongchoice{\SI{1.14}{\degreeCelsius}}
      \correctchoice{\SI{2.28}{\degreeCelsius}}
        \wrongchoice{\SI{4.56}{\degreeCelsius}}
        \wrongchoice{\SI{22.8}{\degreeCelsius}}
    \end{choices}
    \end{multicols}
\end{question}
}

\element{serway-mc}{
\begin{questionmult}{serway-ch20-q50}
    A  block of material of mass m and specific heat $c$ falls from height $h$ and reaches speed $v$ just before striking the ground. 
    Its temperature is measured immediately after it strikes the ground.
    If we ignore any change in temperature owing to interaction with the air,
        the change in temperature of the block of material is:
    \begin{choices}
      \correctchoice{$\Delta T = \dfrac{v^2}{2c}$}
      \correctchoice{$\Delta T = \dfrac{gh}{c}$}
        \wrongchoice{$\Delta T = \dfrac{vgh}{c}$}
        %\wrongchoice{All of the answers above are correct.}
        %\correctchoice{Only (a) and (b) above are correct.}
    \end{choices}
\end{questionmult}
}

\element{serway-mc}{
\begin{question}{serway-ch20-q51}
    In an isothermal process:
    \begin{choices}
        \wrongchoice{$P$ is constant.}
        \wrongchoice{$V$ is constant.}
        \wrongchoice{$\dfrac{P}{T}$ is constant.}
      \correctchoice{$PV$ is constant.}
        \wrongchoice{$\dfrac{V}{n}$ is constant.}
    \end{choices}
\end{question}
}

\element{serway-mc}{
\begin{question}{serway-ch20-q52}
    The specific heat of an ideal gas at constant pressure is greater than the specific heat of an ideal gas at constant volume because:
    \begin{choices}
      \correctchoice{work is done by a gas at constant pressure.}
        \wrongchoice{work is done by a gas at constant volume.}
        \wrongchoice{no work is done by a gas at constant pressure.}
        \wrongchoice{the temperature remains constant for a gas at constant pressure.}
        \wrongchoice{the temperature remains constant for a gas at constant.}
    \end{choices}
\end{question}
}

\element{serway-mc}{
\begin{questionmult}{serway-ch20-q53}
    We are able to define a mechanical equivalent for heat because:
    \begin{choices}
      \correctchoice{some thermal energy can be converted into mechanical energy.}
      \correctchoice{mechanical energy can be converted into thermal energy.}
      \correctchoice{work can be converted into thermal energy.}
      \correctchoice{some thermal energy can be converted into work.}
      %\correctchoice{all of the above can occur.}
    \end{choices}
\end{questionmult}
}


\endinput


