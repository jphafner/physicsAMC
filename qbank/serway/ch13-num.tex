
%%--------------------------------------------------
%% Serway: Physics for Scientists and Engineers
%%--------------------------------------------------


%% Chapter 13: Universal Gravitation
%%--------------------------------------------------


%% Serway Numeric Questions
%%--------------------------------------------------
\element{serway-num}{
\begin{questionmultx}{serway-ch13-q41}
    Isaac Newton was able to estimate a value for $G$,
        the universal gravitational constant,
        from the following data:
    the radius of the Earth is about \SI{6400}{\kilo\meter},
        the average density of rocks is about \SI{5.5}{\gram\per\centi\meter\cubed},
        and $g=\SI{9.8}{\meter\per\second\squared}$ near the surface of the Earth. 
    What value did Newton obtain for $G$?
    %% ANSWER:  \SI{6.65e-11}{\newton\meter\squared\per\kilo\gram\squared}
    \AMCnumericChoices{6.65}{
        vertical=false,
        digits=3,decimals=2,sign=true,Tsign=\hspace{1ex}\Large a,
        borderwidth=0pt,backgroundcol=white,approx=5
    }
    \AMCnumericChoices{-11}{
        vertical=false,
        digits=2,decimals=0,sign=true,Tsign=\hspace{1ex}\Large b,
        borderwidth=0pt,backgroundcol=white,approx=5
    }
\end{questionmultx}
}

\element{serway-num}{
\begin{questionmultx}{serway-ch13-q42}
    At the moment of a total eclipse,
        the moon lies along a line from the Earth to the sun. 
    If your normal weight is \SI{600}{\newton},
        how much is your weight decreased by the combined pull of the sun and moon?
    M SUN = 2.0 × 10 30 kg, r S-E = 1.5 × 10 8 km
    M MOON = 7.4 × 10 22 kg, r M-E = 3.8 × 10 5 km
    %% ANSWER:  \SI{0.37}{\newton}
    \AMCnumericChoices{3.7}{
        vertical=false,
        digits=3,decimals=2,sign=true,Tsign=\hspace{1ex}\Large a,
        borderwidth=0pt,backgroundcol=white,approx=5
    }
    \AMCnumericChoices{-1}{
        vertical=false,
        digits=2,decimals=0,sign=true,Tsign=\hspace{1ex}\Large b,
        borderwidth=0pt,backgroundcol=white,approx=5
    }
\end{questionmultx}
}

\element{serway-num}{
\begin{questionmultx}{serway-ch13-q43}
    When a falling meteor is at a distance above the Earth's surface of 3 times the Earth's radius,
        what is its acceleration due to the Earth’s gravity?
    %% ANSWER:  \SI{0.613}{\meter\per\second\squared}
    \AMCnumericChoices{6.1}{
        vertical=false,
        digits=3,decimals=2,sign=true,Tsign=\hspace{1ex}\Large a,
        borderwidth=0pt,backgroundcol=white,approx=5
    }
    \AMCnumericChoices{-1}{
        vertical=false,
        digits=2,decimals=0,sign=true,Tsign=\hspace{1ex}\Large b,
        borderwidth=0pt,backgroundcol=white,approx=5
    }
\end{questionmultx}
}

\element{serway-num}{
\begin{questionmultx}{serway-ch13-q44}
    The planet Venus requires 225 days to orbit the sun,
        which has a mass $M=\SI{1.99e30}{\kilo\gram}$,
        in an almost circular trajectory. 
    Calculate the radius of the orbit and the orbital speed of Venus as it circles the sun.
    %% ANSWER:  \SI{1.08e11}{\meter}
    %% ANSWER:  \SI{34.9}{\kilo\meter\per\second}
    \AMCnumericChoices{1.08}{
        vertical=false,
        digits=3,decimals=2,sign=true,Tsign=\hspace{1ex}\Large a,
        borderwidth=0pt,backgroundcol=white,approx=5
    }
    \AMCnumericChoices{11}{
        vertical=false,
        digits=2,decimals=0,sign=true,Tsign=\hspace{1ex}\Large b,
        borderwidth=0pt,backgroundcol=white,approx=5
    }
\end{questionmultx}
}

\element{serway-num}{
\begin{questionmultx}{serway-ch13-q45}
    Imagine a hole is drilled down to the center of the Earth. 
    A small mass $m$ is dropped into the hole. 
    Ignoring the Earth's rotation,
        and all sources of friction,
        find the speed of the mass just as it reaches the Earth's center. 
    ($M_E = \SI{6.0e24}{\kilo\gram}$; $R_E = \SI{6.4e6}{\meter}$.)
    %% ANSWER:  \SI{8.0e3}{\meter\per\second}
    \AMCnumericChoices{1.08}{
        vertical=false,
        digits=3,decimals=2,sign=true,Tsign=\hspace{1ex}\Large a,
        borderwidth=0pt,backgroundcol=white,approx=5
    }
    \AMCnumericChoices{11}{
        vertical=false,
        digits=2,decimals=0,sign=true,Tsign=\hspace{1ex}\Large b,
        borderwidth=0pt,backgroundcol=white,approx=5
    }
\end{questionmultx}
}

\element{serway-num}{
\begin{questionmultx}{serway-ch13-q46}
    Calculate the Earth's angular momentum in the approximation that treats the Earth’s orbit around the sun as a circle. 
    ($M_{Sun} = \SI{1.99e30}{\kilo\gram}$; $T = \SI{3.156e7}{\second}$; $M_E = \SI{5.98e24}{\kilo\gram}$.)
    %% ANSWER:  \SI{2.68e40}{\kilo\gram\meter\squared\per\second}
    \AMCnumericChoices{2.68}{
        vertical=false,
        digits=3,decimals=2,sign=true,Tsign=\hspace{1ex}\Large a,
        borderwidth=0pt,backgroundcol=white,approx=5
    }
    \AMCnumericChoices{40}{
        vertical=false,
        digits=2,decimals=0,sign=true,Tsign=\hspace{1ex}\Large b,
        borderwidth=0pt,backgroundcol=white,approx=5
    }
\end{questionmultx}
}


\endinput


