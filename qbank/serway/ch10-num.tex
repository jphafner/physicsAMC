
%%--------------------------------------------------
%% Serway: Physics for Scientists and Engineers
%%--------------------------------------------------


%% Chapter 10: Rotation of a Rigid Object
%%             About a Fixed Axis
%%--------------------------------------------------


%% Serway Numeric Questions
%%--------------------------------------------------
\element{serway-num}{
\begin{questionmultx}{serway-ch10-q76}
    The net work done in accelerating a propeller from rest to an
        angular velocity of \SI{200}{\radian\per\second} is \SI{3000}{\joule}.
    What is the moment of inertia of the propeller?
    %% ANSWER:  \SI{0.15}{\kilo\gram\meter\squared}
    \AMCnumericChoices{1.5}{
        vertical=false,
        digits=3,decimals=2,sign=true,Tsign=\hspace{1ex}\Large a,
        borderwidth=0pt,backgroundcol=white,approx=5
    }
    \AMCnumericChoices{-1}{
        vertical=false,
        digits=2,decimals=0,sign=true,Tsign=\hspace{1ex}\Large b,
        borderwidth=0pt,backgroundcol=white,approx=5
    }
\end{questionmultx}
}

\element{serway-num}{
\begin{questionmultx}{serway-ch10-q77}
    A horizontal force of magnitude \SI{6.5}{\newton}is exerted tangentially on a Frisbee of mass \SI{32}{\gram} and radius \SI{14.3}{\centi\meter}.
    Assuming the Frisbee, a uniform disk,
        is originally at rest and the force is exerted for \SI{0.08}{\second},
        determine the angular velocity of rotation about the central axis when the Frisbee is released.
    %% ANSWER:  \SI{227}{\radian\per\second}
    \AMCnumericChoices{2.27}{
        vertical=false,
        digits=3,decimals=2,sign=true,Tsign=\hspace{1ex}\Large a,
        borderwidth=0pt,backgroundcol=white,approx=5
    }
    \AMCnumericChoices{2}{
        vertical=false,
        digits=2,decimals=0,sign=true,Tsign=\hspace{1ex}\Large b,
        borderwidth=0pt,backgroundcol=white,approx=5
    }
\end{questionmultx}
}

\element{serway-num}{
\begin{questionmultx}{serway-ch10-q78}
    A celestial object called a pulsar emits its light in short bursts that are synchronized with its rotation. 
    A pulsar in the Crab Nebula is rotating at a rate of \SI{30}{\revolution\per\second}. 
    What is the maximum radius of the pulsar,
        if no part of its surface can move faster than the speed of light (\SI{3e8}{\meter\per\second})?
    %% ANSWER:  \SI{1590}{\kilo\meter}
    \AMCnumericChoices{1.59}{
        vertical=false,
        digits=3,decimals=2,sign=true,Tsign=\hspace{1ex}\Large a,
        borderwidth=0pt,backgroundcol=white,approx=5
    }
    \AMCnumericChoices{3}{
        vertical=false,
        digits=2,decimals=0,sign=true,Tsign=\hspace{1ex}\Large b,
        borderwidth=0pt,backgroundcol=white,approx=5
    }
\end{questionmultx}
}

\element{serway-num}{
\begin{questionmultx}{serway-ch10-q79}
    A uniform solid sphere rolls without slipping along a horizontal surface. 
    What fraction of its total kinetic energy is in the form of rotational kinetic energy about the CM?
    %% ANSWER:  \num{2/7}
    \AMCnumericChoices{2.85714}{
        vertical=false,
        digits=3,decimals=2,sign=true,Tsign=\hspace{1ex}\Large a,
        borderwidth=0pt,backgroundcol=white,approx=5
    }
    \AMCnumericChoices{-1}{
        vertical=false,
        digits=2,decimals=0,sign=true,Tsign=\hspace{1ex}\Large b,
        borderwidth=0pt,backgroundcol=white,approx=5
    }
\end{questionmultx}
}


\endinput


