
%%--------------------------------------------------
%% Serway: Physics for Scientists and Engineers
%%--------------------------------------------------


%% Chapter 16: Wave Motion
%%--------------------------------------------------


%% Table of Contents
%%--------------------------------------------------

%% 16.1 Propagation of a Disturbance
%% 16.2 The Traveling Wave Model
%% 16.3 The Speed of Waves on Strings
%% 16.4 Reflection and Transmission
%% 16.5 Rate of Energy Transfer by Sinusoidal Waves on Strings
%% 16.6 The Linear Wave Equation


%% Serway Multiple Choice Questions
%%--------------------------------------------------
\element{serway-mc}{
\begin{question}{serway-ch16-q01}
    The wavelength of light visible to the human eye is on the order of \SI{5e-7}{\meter}.
    If the speed of light in air is \SI{3e8}{\meter\per\second},
        find the frequency of the lightwave.
    \begin{multicols}{2}
    \begin{choices}
        \wrongchoice{\SI{3e7}{\hertz}}
        \wrongchoice{\SI{4e9}{\hertz}}
        \wrongchoice{\SI{5e11}{\hertz}}
      \correctchoice{\SI{6e14}{\hertz}}
        \wrongchoice{\SI{4e15}{\hertz}}
    \end{choices}
    \end{multicols}
\end{question}
}

\element{serway-mc}{
\begin{question}{serway-ch16-q02}
    The speed of a \SI{10}{\kilo\hertz} sound wave in seawater is approximately \SI{1500}{\meter\per\second}.
    What is its wavelength in sea water?
    \begin{multicols}{3}
    \begin{choices}
        \wrongchoice{\SI{5.0}{\centi\meter}}
        \wrongchoice{\SI{10}{\centi\meter}}
      \correctchoice{\SI{15}{\centi\meter}}
        \wrongchoice{\SI{20}{\centi\meter}}
        \wrongchoice{\SI{29}{\centi\meter}}
    \end{choices}
    \end{multicols}
\end{question}
}

\element{serway-mc}{
\begin{question}{serway-ch16-q03}
    Bats can detect small objects such as insects that are of a size on the order of a wavelength.
    If bats emit a chirp at a frequency of \SI{60}{\kilo\hertz} and the speed of soundwaves in air is 330 m/s, what is the smallest size insect they can detect?
    \begin{multicols}{3}
    \begin{choices}
        \wrongchoice{\SI{1.5}{\milli\meter}}
        \wrongchoice{\SI{3.5}{\milli\meter}}
      \correctchoice{\SI{5.5}{\milli\meter}}
        \wrongchoice{\SI{7.5}{\milli\meter}}
        \wrongchoice{\SI{9.8}{\milli\meter}}
    \end{choices}
    \end{multicols}
\end{question}
}

\element{serway-mc}{
\begin{question}{serway-ch16-q04}
    Ocean waves with a wavelength of \SI{120}{\meter} are coming in at a rate of \SI{8}{\per\minute}.
    What is their speed?
    \begin{multicols}{3}
    \begin{choices}
        \wrongchoice{\SI{8.0}{\meter\per\second}}
      \correctchoice{\SI{16}{\meter\per\second}}
        \wrongchoice{\SI{24}{\meter\per\second}}
        \wrongchoice{\SI{30}{\meter\per\second}}
        \wrongchoice{\SI{4.0}{\meter\per\second}}
    \end{choices}
    \end{multicols}
\end{question}
}

\element{serway-mc}{
\begin{question}{serway-ch16-q05}
    An earthquake emits both S-waves and P-waves which travel at different speeds through the Earth.
    A P-wave travels at \SI{9000}{\meter\per\second} and an S-wave travels at \SI{5000}{\meter\per\second}.
    If P-waves are received at a seismic station \SI{1.00}{\minute} before an S-wave arrives,
        how far away is the earthquake center?
    \begin{multicols}{2}
    \begin{choices}
        \wrongchoice{\SI{88.9}{\kilo\meter}}
        \wrongchoice{\SI{1200}{\kilo\meter}}
      \correctchoice{\SI{675}{\kilo\meter}}
        \wrongchoice{\SI{240}{\kilo\meter}}
        \wrongchoice{\SI{480}{\kilo\meter}}
    \end{choices}
    \end{multicols}
\end{question}
}

\element{serway-mc}{
\begin{question}{serway-ch16-q06}
    A piano string of density \SI{0.0050}{\kilo\gram\per\meter} is under a tension of \SI{1350}{\newton}.
    Find the velocity with which a wave travels on the string.
    \begin{multicols}{2}
    \begin{choices}
        \wrongchoice{\SI{260}{\meter\per\second}}
      \correctchoice{\SI{520}{\meter\per\second}}
        \wrongchoice{\SI{1040}{\meter\per\second}}
        \wrongchoice{\SI{2080}{\meter\per\second}}
        \wrongchoice{\SI{4160}{\meter\per\second}}
    \end{choices}
    \end{multicols}
\end{question}
}

\element{serway-mc}{
\begin{question}{serway-ch16-q07}
    A \SI{100}{\meter} long transmission cable is suspended between two towers.
    If the mass density is \SI{2.01}{\kilo\gram\per\meter} and the tension in the cable is \SI{3.00e4}{\newton},
        what is the speed of transverse waves on the cable?
    \begin{multicols}{2}
    \begin{choices}
        \wrongchoice{\SI{60}{\meter\per\second}}
      \correctchoice{\SI{122}{\meter\per\second}}
        \wrongchoice{\SI{244}{\meter\per\second}}
        \wrongchoice{\SI{310}{\meter\per\second}}
        \wrongchoice{\SI{1500}{\meter\per\second}}
    \end{choices}
    \end{multicols}
\end{question}
}

\element{serway-mc}{
\begin{question}{serway-ch16-q08}
    Transverse waves are traveling on a \SI{1.00}{\meter} long piano string at \SI{500}{\meter\per\second}.
    If the points of zero vibration occur at one-half wavelength,
        (where the string is fastened at both ends),
        find the frequency of vibration.
    \begin{multicols}{2}
    \begin{choices}
      \correctchoice{\SI{250}{\hertz}}
        \wrongchoice{\SI{500}{\hertz}}
        \wrongchoice{\SI{1000}{\hertz}}
        \wrongchoice{\SI{2000}{\hertz}}
        \wrongchoice{\SI{2500}{\hertz}}
    \end{choices}
    \end{multicols}
\end{question}
}

\element{serway-mc}{
\begin{question}{serway-ch16-q09}
    The lowest $A$ on a piano has a frequency of \SI{27.5}{\hertz}.
    If the tension in the \SI{2.00}{\meter} string is \SI{308}{\newton},
        and one-half wavelength occupies the string,
        what is the mass of the wire?
    \begin{multicols}{2}
    \begin{choices}
        \wrongchoice{\SI{0.025}{\kilo\gram}}
        \wrongchoice{\SI{0.049}{\kilo\gram}}
      \correctchoice{\SI{0.051}{\kilo\gram}}
        \wrongchoice{\SI{0.081}{\kilo\gram}}
        \wrongchoice{\SI{0.037}{\kilo\gram}}
    \end{choices}
    \end{multicols}
\end{question}
}

\element{serway-mc}{
\begin{question}{serway-ch16-q10}
    If $y = 0.02\sin\left(30x-400t\right)$ (SI units),
        the frequency of the wave is:
    \begin{multicols}{3}
    \begin{choices}
        \wrongchoice{$30\,\si{\hertz}$}
        \wrongchoice{$\dfrac{15}{\pi}\,\si{\hertz}$}
      \correctchoice{$\dfrac{200}{\pi}\,\si{\hertz}$}
        \wrongchoice{$400\,\si{\hertz}$}
        \wrongchoice{$800\pi\,\si{\hertz}$}
    \end{choices}
    \end{multicols}
\end{question}
}

\element{serway-mc}{
\begin{question}{serway-ch16-q11}
    If $y = 0.02\sin\left(30x-400t\right)$ (SI units),
        the wavelength of the wave is:
    \begin{multicols}{3}
    \begin{choices}
      \correctchoice{$\dfrac{\pi}{15}\,\si{\meter}$}
        \wrongchoice{$\dfrac{15}{\pi}\,\si{\meter}$}
        \wrongchoice{$60\pi\,\si{\meter}$}
        \wrongchoice{$4.2\,\si{\meter}$}
        \wrongchoice{$30\,\si{\meter}$}
    \end{choices}
    \end{multicols}
\end{question}
}

\element{serway-mc}{
\begin{question}{serway-ch16-q12}
    If $y = 0.02\sin\left(30x-400t\right)$ (SI units),
        the velocity of the wave is:
    \begin{multicols}{3}
    \begin{choices}
        \wrongchoice{$\dfrac{3}{40}\,\si{\meter\per\second}$}
      \correctchoice{$\dfrac{40}{3}\,\si{\meter\per\second}$}
        \wrongchoice{$\dfrac{60\pi}{400}\,\si{\meter\per\second}$}
        \wrongchoice{$\dfrac{400}{60\pi}\,\si{\meter\per\second}$}
        \wrongchoice{$400\,\si{\meter\per\second}$}
    \end{choices}
    \end{multicols}
\end{question}
}

\element{serway-mc}{
\begin{question}{serway-ch16-q13}
    If $y = 0.02\sin\left(30x-400t\right)$ (SI units),
        the angular frequency of the wave is:
    \begin{multicols}{3}
    \begin{choices}
        \wrongchoice{$30\,\si{\radian\per\second}$}
        \wrongchoice{$\dfrac{30}{2\pi}\,\si{\radian\per\second}$}
        \wrongchoice{$\dfrac{400}{2\pi}\,\si{\radian\per\second}$}
      \correctchoice{$400\,\si{\radian\per\second}$}
        \wrongchoice{$\dfrac{40}{3}\,\si{\radian\per\second}$}
    \end{choices}
    \end{multicols}
\end{question}
}

\element{serway-mc}{
\begin{question}{serway-ch16-q14}
    If $y = 0.02\sin\left(30x-400t\right)$ (SI units),
        the wave number is:
    \begin{multicols}{3}
    \begin{choices}
      \correctchoice{$30\,\si{\per\meter}$}
        \wrongchoice{$\dfrac{30}{2\pi}\,\si{\per\meter}$}
        \wrongchoice{$\dfrac{400}{2\pi}\,\si{\per\meter}$}
        \wrongchoice{$400\,\si{\per\meter}$}
        \wrongchoice{$60\pi\,\si{\per\meter}$}
    \end{choices}
    \end{multicols}
\end{question}
}

\element{serway-mc}{
\begin{question}{serway-ch16-q15}
    If $y = 0.02\sin\left(30x-400t\right)$ (SI units) and if the mass density of the string on which the wave propagates is \SI{0.005}{\kilo\gram\per\meter},
        then the transmitted power is:
    \begin{multicols}{3}
    \begin{choices}
        \wrongchoice{\SI{1.03}{\watt}}
      \correctchoice{\SI{2.13}{\watt}}
        \wrongchoice{\SI{4.84}{\watt}}
        \wrongchoice{\SI{5.54}{\watt}}
        \wrongchoice{\SI{106}{\watt}}
    \end{choices}
    \end{multicols}
\end{question}
}

\element{serway-mc}{
\begin{question}{serway-ch16-q16}
    Write the equation of a wave,
        traveling along the +x axis with an amplitude of \SI{0.02}{\meter},
        a frequency of \SI{440}{\hertz},
        and a speed of \SI{330}{\meter\per\second}.
    \begin{choices}
      \correctchoice{$y = 0.02\sin\left[880\pi\left(\dfrac{x}{300}-t\right)\right]$}
        \wrongchoice{$y = 0.02\cos\left[880\pi\dfrac{x}{330}-440t\right]$}
        \wrongchoice{$y = 0.02\sin\left[880\pi\left(\dfrac{x}{330}+t\right)\right]$}
        \wrongchoice{$y = 0.02\sin\left[2\pi\left(\dfrac{x}{330}+440t\right)\right]$}
        \wrongchoice{$y = 0.02\cos\left[2\pi\left(\dfrac{x}{330}+440t\right)\right]$}
    \end{choices}
\end{question}
}

\element{serway-mc}{
\begin{question}{serway-ch16-q17}
    For the wave described by
    \begin{equation*}
        y = 0.15\sin\left[\dfrac{\pi}{16}\left(2x-64t\right)\right]
    \end{equation*}
    (SI units),
        determine the first positive $x$-coordinate where $y$ is a maximum when $t=0$.
    \begin{multicols}{3}
    \begin{choices}
        \wrongchoice{\SI{16}{\meter}}
        \wrongchoice{\SI{8}{\meter}}
      \correctchoice{\SI{4}{\meter}}
        \wrongchoice{\SI{2}{\meter}}
        \wrongchoice{\SI{13}{\meter}}
    \end{choices}
    \end{multicols}
\end{question}
}

\element{serway-mc}{
\begin{question}{serway-ch16-q18}
    For the wave described by
    \begin{equation*}
        y = 0.15\sin\left[\dfrac{\pi}{16}\left(2x-64t\right)\right]
    \end{equation*}
    (SI units), determine $x$ coordinate of the second maximum when $t=0$.
    \begin{multicols}{3}
    \begin{choices}
      \correctchoice{\SI{20}{\meter}}
        \wrongchoice{\SI{18}{\meter}}
        \wrongchoice{\SI{24}{\meter}}
        \wrongchoice{\SI{28}{\meter}}
        \wrongchoice{\SI{16}{\meter}}
    \end{choices}
    \end{multicols}
\end{question}
}

\element{serway-mc}{
\begin{question}{serway-ch16-q19}
    For the wave described by $y = 0.02\sin\left(kx\right)$ at $t=\SI{0}{\second}$,
        the first maximum at a positive $x$ coordinate occurs where $x=\SI{4}{\meter}$.
    Where on the positive $x$ axis does the second maximum occur?
    \begin{multicols}{3}
    \begin{choices}
      \correctchoice{\SI{20}{\meter}}
        \wrongchoice{\SI{18}{\meter}}
        \wrongchoice{\SI{24}{\meter}}
        \wrongchoice{\SI{28}{\meter}}
        \wrongchoice{\SI{16}{\meter}}
    \end{choices}
    \end{multicols}
\end{question}
}

\element{serway-mc}{
\begin{question}{serway-ch16-q20}
    For the transverse wave described by 
    \begin{equation*}
        y=0.15\sin\left[\dfrac{\pi}{16}\left(2x-64t\right)\right]
    \end{equation*}
    (SI units),
        determine the maximum transverse speed of the particles of the medium.
    \begin{multicols}{3}
    \begin{choices}
        \wrongchoice{$0.192\,\si{\meter\per\second}$}
      \correctchoice{$0.6\pi\,\si{\meter\per\second}$}
        \wrongchoice{$9.6\,\si{\meter\per\second}$}
        \wrongchoice{$4\,\si{\meter\per\second}$}
        \wrongchoice{$2\,\si{\meter\per\second}$}
    \end{choices}
    \end{multicols}
\end{question}
}

\element{serway-mc}{
\begin{question}{serway-ch16-q21}
    Which of the following is a solution to the wave equation,
    \begin{equation*}
        \dfrac{\partial^2 y}{\partial x^2} = \dfrac{1}{v^2} \dfrac{\partial^2 y}{\partial t^2}\,?
    \end{equation*}
    \begin{multicols}{2}
    \begin{choices}
        \wrongchoice{$\dfrac{\mathrm{e}^{-x}}{x}\sin x$}
      \correctchoice{$\left(\cos kx\right)\left(\sin t\right)$}
        \wrongchoice{$\mathrm{e}^{-x}\sin\omega t$}
        \wrongchoice{$\mathrm{e}^{-x}\sin\left(kx-\omega t\right)$}
        \wrongchoice{$\mathrm{e}^{-x}\cos t$}
    \end{choices}
    \end{multicols}
\end{question}
}

\element{serway-mc}{
\begin{question}{serway-ch16-q22}
    Find the period of a wave of \SI{100}{\meter} wavelength in deep water where $v = \sqrt{\dfrac{g\lambda}{2\pi}}$.
    \begin{multicols}{3}
    \begin{choices}
        \wrongchoice{\SI{5.0}{\second}}
      \correctchoice{\SI{8.0}{\second}}
        \wrongchoice{\SI{12.5}{\second}}
        \wrongchoice{\SI{15}{\second}}
        \wrongchoice{\SI{0.125}{\second}}
    \end{choices}
    \end{multicols}
\end{question}
}

\element{serway-mc}{
\begin{question}{serway-ch16-q23}
    A piano wire of length \SI{1.5}{\meter} vibrates so that one-half wavelength is contained on the string.
    If the frequency of vibration is \SI{65}{\hertz},
        the amplitude of vibration is \SI{3.0}{\milli\meter},
        and the density is \SI{15}{\gram\per\meter},
        how much energy is transmitted per second down the wire?
    \begin{multicols}{3}
    \begin{choices}
        \wrongchoice{\SI{21}{\watt}}
        \wrongchoice{\SI{11}{\watt}}
        \wrongchoice{\SI{5.4}{\watt}}
      \correctchoice{\SI{2.2}{\watt}}
        \wrongchoice{\SI{1.1}{\watt}}
    \end{choices}
    \end{multicols}
\end{question}
}

\element{serway-mc}{
\begin{question}{serway-ch16-q24}
    A student attaches a length of nylon fishing line to a fence post.
    She stretches it out and shakes the end of the rope in her hand back and forth to produce waves on the line.
    The most efficient way for her to increase the wavelength is to:
    \begin{choices}
        \wrongchoice{increase the tension on the hose and shake the end more times per second.}
        \wrongchoice{decrease the tension on the hose and shake the end more times per second.}
      \correctchoice{increase the tension on the hose and shake the end fewer times per second.}
        \wrongchoice{decrease the tension on the hose and shake the end fewer times per second.}
        \wrongchoice{keep the tension and frequency the same but increase the length of the hose.}
    \end{choices}
\end{question}
}

\element{serway-mc}{
\begin{question}{serway-ch16-q25}
    The figure below shows a sine wave at one point of a string as a function of time.
    \begin{center}
    \begin{tikzpicture}
        %% NOTE: pgfpltos
    \end{tikzpicture}
    \end{center}
    Which of the graphs below shows a wave where the amplitude and the frequency are doubled?
    \begin{multicols}{2}
    \begin{choices}
        %% NOTE: ANS is D
        \wrongchoice{
            \begin{tikzpicture}
                %% NOTE: tikz graphs
            \end{tikzpicture}
        }
    \end{choices}
    \end{multicols}
\end{question}
}

\element{serway-mc}{
\begin{question}{serway-ch16-q26}
    The figure below shows a sine wave at one point of a string as a function of time.
    \begin{center}
    \begin{tikzpicture}
        %% NOTE: pgfpltos
    \end{tikzpicture}
    \end{center}
    Which of the graphs below shows a wave where the amplitude and frequency are each reduced in half?
    \begin{multicols}{2}
    \begin{choices}
        %% NOTE: ANS is E
        \wrongchoice{
            \begin{tikzpicture}
                %% NOTE: tikz graphs
            \end{tikzpicture}
        }
    \end{choices}
    \end{multicols}
\end{question}
}

\element{serway-mc}{
\begin{question}{serway-ch16-q27}
    The figure below shows a sine wave on a string at one instant of time.
    \begin{center}
    \begin{tikzpicture}
        %% NOTE: pgfpltos
    \end{tikzpicture}
    \end{center}
    Which of the graphs below shows a wave where the frequency and wave velocity are both doubled?
    \begin{multicols}{2}
    \begin{choices}
        %% NOTE: ANS is A
        \wrongchoice{
            \begin{tikzpicture}
                %% NOTE: tikz graphs
            \end{tikzpicture}
        }
    \end{choices}
    \end{multicols}
\end{question}
}

\element{serway-mc}{
\begin{question}{serway-ch16-q28}
    The figure below shows a sine wave on a string at one instant of time.
    \begin{center}
    \begin{tikzpicture}
        %% NOTE: pgfpltos
    \end{tikzpicture}
    \end{center}
    Which of the graphs below shows a wave where the wavelength is twice as large?
    \begin{multicols}{2}
    \begin{choices}
        %% NOTE: ANS is D
        \wrongchoice{
            \begin{tikzpicture}
                %% NOTE: tikz graphs
            \end{tikzpicture}
        }
    \end{choices}
    \end{multicols}
\end{question}
}

\element{serway-mc}{
\begin{questionmult}{serway-ch16-q29}
    Superposition of waves can occur:
    \begin{choices}
      \correctchoice{in transverse waves.}
      \correctchoice{in longitudinal waves.}
      \correctchoice{in sinusoidal waves.}
        %\correctchoice{in all ofthe above.}
        %\wrongchoice{only in (a) and (c) above.}
    \end{choices}
\end{questionmult}
}

\element{serway-mc}{
\begin{question}{serway-ch16-q30}
    Two pulses are traveling towards each other at \SI{10}{\centi\meter\per\second} on a long string at $t=\SI{0}{\second}$,
        as shown below.
    \begin{center}
    \begin{tikzpicture}
        %% NOTE: tikz
    \end{tikzpicture}
    \end{center}
    Which diagram below correctly shows the shape of the string at \SI{0.5}{\second}?
    \begin{multicols}{2}
    \begin{choices}
        %% NOTE: ANS is B
        \wrongchoice{
            \begin{tikzpicture}
                %% NOTE: tikz graphs
            \end{tikzpicture}
        }
    \end{choices}
    \end{multicols}
\end{question}
}

\element{serway-mc}{
\begin{questionmult}{serway-ch16-q31}
    Suppose that you were selected for a ``Survivor''-type TV show.
    To help keep your group connected, you suggest that long vines can be tied together and used to transmit signals in cases of emergency.
    To get the signals to travel faster, you should
    \begin{choices}
      \correctchoice{select lighter vines.}
      \correctchoice{increase the tension on the vines.}
        \wrongchoice{hang weights from the vines at evenly spaced intervals.}
        %\wrongchoice{do all of the above.}
        %\correctchoice{do (a) and (b) above only.}
    \end{choices}
\end{questionmult}
}

\element{serway-mc}{
\begin{question}{serway-ch16-q32}
    Two ropes are spliced together as shown.
    \begin{center}
    \begin{tikzpicture}
        %% NOTE: tikz
    \end{tikzpicture}
    \end{center}
    A short time after the incident pulse shown in the diagram reaches the splice,
        the ropes appearance will be that in:
    \begin{choices}
        %% NOTE: ANS is A
        \wrongchoice{
            \begin{tikzpicture}
                %% NOTE: tikz graphs
            \end{tikzpicture}
        }
    \end{choices}
\end{question}
}

\element{serway-mc}{
\begin{question}{serway-ch16-q33}
    Two ropes are spliced together as shown.
    \begin{center}
    \begin{tikzpicture}
        %% NOTE: tikz
    \end{tikzpicture}
    \end{center}
    A short time after the incident pulse shown in the diagram reaches the splice,
        the ropes appearance will be that in:
    \begin{choices}
        %% NOTE: ANS is C
        \wrongchoice{
            \begin{tikzpicture}
                %% NOTE: tikz graphs
            \end{tikzpicture}
        }
    \end{choices}
\end{question}
}

\element{serway-mc}{
\begin{question}{serway-ch16-q34}
    The fundamental frequency of a above middle C on the piano is \SI{440}{\hertz}.
    This is the tenor high A,
        but a convenient note in the mid-range of women's voices.
    When we calculate the wavelength,
        we find that it is:
    \begin{choices}
        \wrongchoice{much shorter than the length of either a man's or woman's lips.}
        \wrongchoice{shorter than the length of a man's lips, but about the length of a woman's lips.}
        \wrongchoice{longer than a woman's lips, but about the length of a man's lips.}
      \correctchoice{much longer than the length of either a man's or a woman's lips.}
        \wrongchoice{about the same length as either a man's or woman's lips.}
    \end{choices}
\end{question}
}

\element{serway-mc}{
\begin{question}{serway-ch16-q35}
    Ariel claims that a pulse is described by the equation
    \begin{equation*}
        y(x,t) = \dfrac{2}{x^2 - 6.0xt + 9t^2 + 9}
    \end{equation*}
    where $x$ and $y$ are measured in \si{\centi\meter} and $t$ in \si{\second}.
    Miranda says that it is not possible to represent a pulse with this function because a wave must be a function of $x+vt$ or $x-vt$.
    Which one, if either, is correct, and why?
    \begin{choices}
      \correctchoice{Ariel, because $x^2-6.0xt+9t^2 = \left(x-3.0t\right)^2$.}
        \wrongchoice{Ariel, because a pulse is not an infinite wave.}
        \wrongchoice{Miranda, because $\left(x-3.0t\right)^2$ is the same as $\left(3.0t-x\right)^2$.}
        \wrongchoice{Miranda, because a pulse is not an infinite wave.}
        \wrongchoice{Miranda, because $x^2-6.0xt+9t^2=x^2\left(1-\dfrac{6.00t}{x}+\dfrac{9t^2}{x^2}\right)$ is infinite when $x=0$.}
    \end{choices}
\end{question}
}

\element{serway-mc}{
\begin{question}{serway-ch16-q36}
    How does the wave function $y^{\prime}=\sin\left(k^2x-\omega^2t\right)$ differ from the wave function $y=\sin\left(kx-\omega t\right)$?
    (In $y^{\prime}$, $k^2$ has units \si{\per\centi\meter} and $\omega^2$ units \si{\per\second}.
    In $y$, $k$ has units \si{\per\centi\meter} and $\omega$ units \si{\per\second}.)
    \begin{choices}[o]
        \wrongchoice{$\lambda^{\prime} = \dfrac{\lambda}{2\pi}$}
        \wrongchoice{$f^{\prime} = 2\pi f$}
        \wrongchoice{$v^{\prime} = v^2$}
      \correctchoice{All of the above are correct.}
        \wrongchoice{The only difference is that $v^{\prime}=2v$.}
    \end{choices}
\end{question}
}

\element{serway-mc}{
\begin{question}{serway-ch16-q37}
    The figure below represents a string which has a heavy section and a light section.
    \begin{center}
    \begin{tikzpicture}
        %% NOTE:
    \end{tikzpicture}
    \end{center}
    The mass per unit length of the heavy section is 16 times as large as the mass per unit length of the light section.
    When the string is under tension,
        the speed of a pulse traveling in the heavy section is \rule[-0.1pt]{4em}{0.1pt} times the speed of that same pulse traveling in the light section.
    \begin{multicols}{3}
    \begin{choices}
         \wrongchoice{\num{1/16}}
       \correctchoice{\num{1/4}}
         \wrongchoice{\num{1/2}}
         \wrongchoice{\num{2}}
         \wrongchoice{\num{4}}
    \end{choices}
    \end{multicols}
\end{question}
}

\element{serway-mc}{
\begin{questionmult}{serway-ch16-q38}
    To transmit four times as much energy per unit time along a string,
        you must:
    \begin{choices}
      \correctchoice{double the frequency.}
      \correctchoice{double the amplitude.}
        \wrongchoice{increase the tension by a factor of 16.}
        %\wrongchoice{do any one of the above.}
        %\correctchoice{do only (a) or (b) above.}
    \end{choices}
\end{questionmult}
}

\element{serway-mc}{
\begin{question}{serway-ch16-q39}
    The wave equation is written down in an exam as
    \begin{equation*}
        \dfrac{\mathrm{d}^2 y}{\mathrm{d}x^2} = v^2 \dfrac{\mathrm{d}^2y}{\mathrm{d}t^2} .
    \end{equation*}
    From dimensional considerations we see that:
    \begin{choices}
        \wrongchoice{$v^2$ should be replaced by $\dfrac{T}{\mu}$.}
        \wrongchoice{$v^2$ should be replaced by $\sqrt{T}{\mu}$.}
        \wrongchoice{$v^2$ should be replaced by $v$.}
        \wrongchoice{$v^2$ should be replaced by $\dfrac{1}{v}$.}
      \correctchoice{$v^2$ should be replaced by $\dfrac{1}{v^2}$.}
    \end{choices}
\end{question}
}

\element{serway-mc}{
\begin{question}{serway-ch16-q40}
    Four wave functions are given below.
    Rank them in order of the magnitude of the wave speeds,
        from least to greatest.
    \begin{itemize}
        \item[I.]   $y(x,t) = 5\sin\left(4x-20t+4\right)$
        \item[II.]  $y(x,t) = 5\sin\left(3x-12t+5\right)$
        \item[III.] $y(x,t) = 5\cos\left(4x+24t+6\right)$
        \item[IV.]  $y(x,t) = 14\cos\left(2x-8t+3\right)$
    \end{itemize}
    \begin{choices}
        \wrongchoice{IV, II, I, III}
      \correctchoice{IV=II, I, III}
        \wrongchoice{III, I, II, IV}
        \wrongchoice{IV, I, II, III}
        \wrongchoice{III, IV, II, I}
    \end{choices}
\end{question}
}

\element{serway-mc}{
\begin{question}{serway-ch16-q41}
    Four wave functions are given below.
    Rank them in order of the magnitude of the frequencies of the waves,
        from least to greatest.
    \begin{itemize}
        \item[I.]   $y(x,t) = 5\sin\left(4x-20t+4\right)$
        \item[II.]  $y(x,t) = 5\sin\left(3x-12t+5\right)$
        \item[III.] $y(x,t) = 5\cos\left(4x+24t+6\right)$
        \item[IV.]  $y(x,t) = 14\cos\left(2x-8t+3\right)$
    \end{itemize}
    \begin{choices}
      \correctchoice{IV, II, I, III}
        \wrongchoice{IV=II, I, III}
        \wrongchoice{III, I, II, IV}
        \wrongchoice{IV, I, II, III}
        \wrongchoice{III, IV, II, I}
    \end{choices}
\end{question}
}

\element{serway-mc}{
\begin{question}{serway-ch16-q42}
    Four wave functions are given below.
    Rank them in order of the magnitude of the wavelengths,
        from least to greatest.
    \begin{itemize}
        \item[I.]   $y(x,t) = 5\sin\left(4x-20t+4\right)$
        \item[II.]  $y(x,t) = 5\sin\left(3x-12t+5\right)$
        \item[III.] $y(x,t) = 5\cos\left(4x+24t+6\right)$
        \item[IV.]  $y(x,t) = 14\cos\left(2x-8t+3\right)$
    \end{itemize}
    \begin{choices}
        \wrongchoice{IV, II, I, III}
        \wrongchoice{IV, I, II, III}
        \wrongchoice{I, II, III, IV}
        \wrongchoice{IV, II, III=I}
      \correctchoice{I=III, II, IV}
    \end{choices}
\end{question}
}

\element{serway-mc}{
\begin{question}{serway-ch16-q43}
    You are holding on to one end of a long string that is fastened to a rigid steel light pole.
    After producing a wave pulse that was \SI{5}{\milli\meter} high and \SI{4}{\centi\meter} wide,
        you want to produce a pulse that is \SI{4}{\centi\meter} wide but \SI{7}{\milli\meter} high.
    You must move your hand up and down once,
    \begin{choices}
        \wrongchoice{the same distance up as before, but take a shorter time.}
        \wrongchoice{the same distance up as before, but take a longer time.}
        \wrongchoice{a smaller distance up, but take a shorter time.}
        \wrongchoice{a greater distance up, but take a longer time.}
      \correctchoice{a greater distance up, but take the same time.}
    \end{choices}
\end{question}
}

\element{serway-mc}{
\begin{question}{serway-ch16-q44}
    You are holding on to one end of a long string that is fastened to a rigid steel light pole. 
    After producing a wave pulse that was \SI{5}{\milli\meter} high and \SI{4}{\centi\meter} wide,
        you want to produce a pulse that is \SI{6}{\centi\meter} wide but still \SI{5}{\milli\meter} high. 
    You must move your hand up and down once,
    \begin{choices}
        \wrongchoice{the same distance up as before, but take a shorter time.}
      \correctchoice{the same distance up as before, but take a longer time.}
        \wrongchoice{a smaller distance up, but take a shorter time.}
        \wrongchoice{a greater distance up, but take a longer time.}
        \wrongchoice{a greater distance up, but take the same time.}
    \end{choices}
\end{question}
}

\element{serway-mc}{
\begin{question}{serway-ch16-q45}
    You are holding on to one end of a long string that is fastened to a rigid steel light pole. 
    After producing a wave pulse that was \SI{5}{\milli\meter} high and \SI{4}{\centi\meter} wide,
        you want to produce a pulse that is \SI{6}{\centi\meter} wide and \SI{7}{\milli\meter} high. 
    You must move your hand up and down once,
    \begin{choices}
        \wrongchoice{the same distance up as before, but take a shorter time.}
        \wrongchoice{the same distance up as before, but take a longer time.}
        \wrongchoice{a smaller distance up, but take a shorter time.}
      \correctchoice{a greater distance up, but take a longer time.}
        \wrongchoice{a greater distance up, but take the same time.}
    \end{choices}
\end{question}
}


\endinput


