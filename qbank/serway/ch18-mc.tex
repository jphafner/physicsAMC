
%%--------------------------------------------------
%% Serway: Physics for Scientists and Engineers
%%--------------------------------------------------


%% Chapter 18: Superposition and Standing Waves
%%--------------------------------------------------


%% Table of Contents
%%--------------------------------------------------

%% 18.1 Superposition and Interference
%% 18.2 Standing Waves
%% 18.3 Standing Waves in a String Fixed at Both Ends
%% 18.4 Resonance
%% 18.5 Standing Waves in Air Columns
%% 18.6 Standing Waves in Rods and Membranes
%% 18.7 Beats: Interference in Time
%% 18.8 Nonsinusoidal Wave Patterns


%% Serway Multiple Choice Questions
%%--------------------------------------------------
\element{serway-mc}{
\begin{question}{serway-ch18-q01}
    Two harmonic waves are described by
    \begin{align*}
        y_1 &= \left(\SI{3}{\meter}\right) \sin\left(\dfrac{4}{\si{\meter}}x-\dfrac{700}{\si{\second}}t\right) \\
        y_2 &= \left(\SI{3}{\meter}\right) \sin\left(\dfrac{4}{\si{\meter}}x-\dfrac{700}{\si{\second}}t-2\right)\,.
    \end{align*}
    What is the amplitude of the resultant wave?
    \begin{multicols}{3}
    \begin{choices}
        \wrongchoice{\SI{8.0}{\meter}}
        \wrongchoice{\SI{4.3}{\meter}}
        \wrongchoice{\SI{6.0}{\meter}}
      \correctchoice{\SI{3.2}{\meter}}
        \wrongchoice{\SI{3.0}{\meter}}
    \end{choices}
    \end{multicols}
\end{question}
}

\element{serway-mc}{
\begin{question}{serway-ch18-q02}
    Two harmonic waves are described by
    \begin{align*}
        y_1 &= \left(\SI{4}{\meter}\right) \sin\left(\dfrac{8}{\si{\meter}}x-\dfrac{300}{\si{\second}}t\right) \\
        y_2 &= \left(\SI{4}{\meter}\right) \sin\left(\dfrac{8}{\si{\meter}}x-\dfrac{300}{\si{\second}}t-2\right)\,.
    \end{align*}
    What is the frequency of the resultant wave?
    \begin{multicols}{3}
    \begin{choices}
        \wrongchoice{\SI{300}{\hertz}}
      \correctchoice{\SI{48}{\hertz}}
        \wrongchoice{\SI{8}{\hertz}}
        \wrongchoice{\SI{0.8}{\hertz}}
        \wrongchoice{\SI{150}{\hertz}}
    \end{choices}
    \end{multicols}
\end{question}
}

\element{serway-mc}{
\begin{question}{serway-ch18-q03}
    Two harmonic waves are described by
    \begin{align*}
        y_1 &= \left(\SI{5}{\meter}\right) \sin\left(\dfrac{6}{\si{\meter}}x-\dfrac{900}{\si{\second}}t\right) \\
        y_2 &= \left(\SI{5}{\meter}\right) \sin\left(\dfrac{6}{\si{\meter}}x-\dfrac{900}{\si{\second}}t-2\right)\,.
    \end{align*}
    What is the wavelength of the resultant wave?
    \begin{multicols}{3}
    \begin{choices}
        \wrongchoice{\SI{3}{\meter}}
        \wrongchoice{\SI{2}{\meter}}
      \correctchoice{\SI{1}{\meter}}
        \wrongchoice{\SI{4}{\meter}}
        \wrongchoice{\SI{6}{\meter}}
    \end{choices}
    \end{multicols}
\end{question}
}

\element{serway-mc}{
\begin{question}{serway-ch18-q04}
    Two harmonic waves are described by
    \begin{align*}
        y_1 &= \left(\SI{7}{\meter}\right) \sin\left(\dfrac{5}{\si{\meter}}x-\dfrac{100}{\si{\second}}t\right) \\
        y_2 &= \left(\SI{7}{\meter}\right) \sin\left(\dfrac{5}{\si{\meter}}x-\dfrac{100}{\si{\second}}t-2\right)\,.
    \end{align*}
    What is the phase of the resultant wave when $x=t=0$?
    \begin{multicols}{3}
    \begin{choices}
        %% radian has no dimension
        \wrongchoice{\SI{3}{\radian}}
        \wrongchoice{\SI{0}{\radian}}
        \wrongchoice{\SI{2}{\radian}}
      \correctchoice{\SI{1}{\radian}}
        \wrongchoice{\SI{4}{\radian}}
    \end{choices}
    \end{multicols}
\end{question}
}

\element{serway-mc}{
\begin{question}{serway-ch18-q05}
    The path difference between two waves is \SI{5}{\meter}. 
    If the wavelength of the waves emitted by the two sources is \SI{4}{\meter},
        what is the phase difference?
    \begin{multicols}{3}
    \begin{choices}
      \correctchoice{\ang{90}}
        \wrongchoice{\ang{400}}
        \wrongchoice{\ang{1.57}}
        \wrongchoice{\ang{7.85}}
        \wrongchoice{\ang{15}}
    \end{choices}
    \end{multicols}
\end{question}
}

\element{serway-mc}{
\begin{question}{serway-ch18-q06}
    Two harmonic waves are described by
    \begin{align*}
        y_1 &= \left(\SI{3}{\centi\meter}\right) \sin\left(\dfrac{8}{\si{\meter}}x-\dfrac{2}{\si{\second}}t\right) \\
        y_2 &= \left(\SI{3}{\centi\meter}\right) \sin\left(\dfrac{8}{\si{\meter}}x-\dfrac{2}{\si{\second}}t\right)\,.
    \end{align*}
    What is the magnitude of the speed of the two traveling waves?
    \begin{multicols}{3}
    \begin{choices}
        \wrongchoice{\SI{16}{\meter\per\second}}
        \wrongchoice{\SI{4}{\meter\per\second}}
        \wrongchoice{\SI{8}{\meter\per\second}}
      \correctchoice{\SI{0.25}{\meter\per\second}}
        \wrongchoice{\SI{2}{\meter\per\second}}
    \end{choices}
    \end{multicols}
\end{question}
}

\element{serway-mc}{
\begin{question}{serway-ch18-q07}
    Two harmonic waves are described by
    \begin{align*}
        y_1 &= \left(\SI{6}{\centi\meter}\right) \sin\left(\pi\left(\dfrac{8}{\si{\meter}}x-\dfrac{2}{\si{\second}}t\right)\right) \\
        y_2 &= \left(\SI{6}{\centi\meter}\right) \sin\left(\pi\left(\dfrac{8}{\si{\meter}}x-\dfrac{2}{\si{\second}}t\right)\right)\,.
    \end{align*}
    From the choices given,
        determine the smallest positive value of $x$ corresponding to a node of the resultant standing wave.
    \begin{multicols}{3}
    \begin{choices}
        \wrongchoice{\SI{3}{\centi\meter}}
        \wrongchoice{\SI{0.25}{\centi\meter}}
      \correctchoice{zero}
        \wrongchoice{\SI{6}{\centi\meter}}
        \wrongchoice{\SI{1.5}{\centi\meter}}
    \end{choices}
    \end{multicols}
\end{question}
}

\element{serway-mc}{
\begin{question}{serway-ch18-q08}
    Two harmonic waves are described by
    \begin{align*}
        y_1 &= \left(\SI{6}{\centi\meter}\right) \sin\left(\pi\left(\dfrac{2.00}{\si{\meter}}x-\dfrac{3.00}{\si{\second}}t\right)\right) \\
        y_2 &= \left(\SI{6}{\centi\meter}\right) \sin\left(\pi\left(\dfrac{2.00}{\si{\meter}}x-\dfrac{3.00}{\si{\second}}t\right)\right)\,.
    \end{align*}
    What is the magnitude of the displacement of this wave at $x=\SI{3}{\centi\meter}$ and $t=\SI{5}{\second}$?
    \begin{multicols}{3}
    \begin{choices}
        \wrongchoice{\SI{12.0}{\centi\meter}}
        \wrongchoice{\SI{3.00}{\centi\meter}}
        \wrongchoice{\SI{6.00}{\centi\meter}}
      \correctchoice{\SI{2.25}{\centi\meter}}
        \wrongchoice{zero}
    \end{choices}
    \end{multicols}
\end{question}
}

\element{serway-mc}{
\begin{question}{serway-ch18-q09}
    Two harmonic waves traveling in opposite directions interfere to produce a standing wave described by
    \begin{equation*}
        y = 3\sin\left(2x\right) \cos\left(5t\right)\,,
    \end{equation*}
        where $x$ is in \si{\meter} and $t$ is in \si{\second}.
    What is the wavelength of the interfering waves?
    \begin{multicols}{3}
    \begin{choices}
      \correctchoice{\SI{3.14}{\meter}}
        \wrongchoice{\SI{1.00}{\meter}}
        \wrongchoice{\SI{6.28}{\meter}}
        \wrongchoice{\SI{12.0}{\meter}}
        \wrongchoice{\SI{2.00}{\meter}}
    \end{choices}
    \end{multicols}
\end{question}
}

\element{serway-mc}{
\begin{question}{serway-ch18-q10}
    Two harmonic waves traveling in opposite directions interfere to produce a standing wave described by
    \begin{equation*}
        y = 4\sin\left(5x\right) \cos\left(6t\right)\,,
    \end{equation*}
        where $x$ is in \si{\meter} and $t$ is in \si{\second}.
    What is the approximate frequency of the interfering waves?
    \begin{multicols}{3}
    \begin{choices}
        \wrongchoice{\SI{3}{\hertz}}
      \correctchoice{\SI{1}{\hertz}}
        \wrongchoice{\SI{6}{\hertz}}
        \wrongchoice{\SI{12}{\hertz}}
        \wrongchoice{\SI{5}{\hertz}}
    \end{choices}
    \end{multicols}
\end{question}
}

\element{serway-mc}{
\begin{question}{serway-ch18-q11}
    Two harmonic waves traveling in opposite directions interfere to produce a standing wave described by
    \begin{equation*}
        y = 2\sin\left(4x\right) \cos\left(3t\right)\,,
    \end{equation*}
        where $x$ is in \si{\meter} and $t$ is in \si{\second}.
    What is the speed of the interfering waves?
    \begin{multicols}{3}
    \begin{choices}
       \correctchoice{\SI{0.75}{\meter\per\second}}
         \wrongchoice{\SI{0.25}{\meter\per\second}}
         \wrongchoice{\SI{1.3}{\meter\per\second}}
         \wrongchoice{\SI{12}{\meter\per\second}}
         \wrongchoice{\SI{3.0}{\meter\per\second}}
    \end{choices}
    \end{multicols}
\end{question}
}

\element{serway-mc}{
\begin{question}{serway-ch18-q12}
    Two harmonic waves traveling in opposite directions interfere to produce a standing wave described by
    \begin{equation*}
        y = 2\sin\left(\pi x\right) \cos\left(3\pi t\right)\,,
    \end{equation*}
        where $x$ is in \si{\meter} and $t$ is in \si{\second}.
    What is the distance between the first two antinodes?
    \begin{multicols}{3}
    \begin{choices}
        \wrongchoice{\SI{8}{\meter}}
        \wrongchoice{\SI{2}{\meter}}
        \wrongchoice{\SI{4}{\meter}}
      \correctchoice{\SI{1}{\meter}}
        \wrongchoice{\SI{0.5}{\meter}}
    \end{choices}
    \end{multicols}
\end{question}
}

\element{serway-mc}{
\begin{question}{serway-ch18-q13}
    A string is stretched and fixed at both ends,
        \SI{200}{\centi\meter} apart. 
    If the density of the string is \SI{0.015}{\gram\per\centi\meter},
        and its tension is \SI{600}{\newton},
        what is the wavelength of the first harmonic?
    \begin{multicols}{3}
    \begin{choices}
        \wrongchoice{\SI{600}{\centi\meter}}
      \correctchoice{\SI{400}{\centi\meter}}
        \wrongchoice{\SI{800}{\centi\meter}}
        \wrongchoice{\SI{1000}{\centi\meter}}
        \wrongchoice{\SI{200}{\centi\meter}}
    \end{choices}
    \end{multicols}
\end{question}
}

\element{serway-mc}{
\begin{question}{serway-ch18-q14}
    A string is stretched and fixed at both ends,
        \SI{200}{\centi\meter} apart. 
    If the density of the string is \SI{0.015}{\gram\per\centi\meter},
        and its tension is \SI{600}{\newton},
        what is the fundamental frequency?
    \begin{multicols}{3}
    \begin{choices}
        \wrongchoice{\SI{316}{\hertz}}
        \wrongchoice{\SI{632}{\hertz}}
      \correctchoice{\SI{158}{\hertz}}
        \wrongchoice{\SI{215}{\hertz}}
        \wrongchoice{\SI{79}{\hertz}}
    \end{choices}
    \end{multicols}
\end{question}
}

\element{serway-mc}{
\begin{question}{serway-ch18-q15}
    A stretched string is observed to vibrate in three equal segments when driven by a \SI{480}{\hertz} oscillator. 
    What is the fundamental frequency of vibration for this string?
    \begin{multicols}{3}
    \begin{choices}
        \wrongchoice{\SI{480}{\hertz}}
        \wrongchoice{\SI{320}{\hertz}}
      \correctchoice{\SI{160}{\hertz}}
        \wrongchoice{\SI{640}{\hertz}}
        \wrongchoice{\SI{240}{\hertz}}
    \end{choices}
    \end{multicols}
\end{question}
}

\element{serway-mc}{
\begin{question}{serway-ch18-q16}
    A clarinet behaves like a tube closed at one end. 
    If its length is \SI{1.0}{\meter},
        and the velocity of sound is \SI{344}{\meter\per\second},
        what is its fundamental frequency?
    \begin{multicols}{3}
    \begin{choices}
        \wrongchoice{\SI{264}{\hertz}}
        \wrongchoice{\SI{140}{\hertz}}
      \correctchoice{\SI{86}{\hertz}}
        \wrongchoice{\SI{440}{\hertz}}
        \wrongchoice{\SI{172}{\hertz}}
    \end{choices}
    \end{multicols}
\end{question}
}

\element{serway-mc}{
\begin{question}{serway-ch18-q17}
    An organ pipe open at both ends has a radius of \SI{4.0}{\centi\meter} and a length of \SI{6.0}{\meter}.
    What is the frequency of the third harmonic? 
    (Assume the velocity of sound is \SI{344}{\meter\per\second}.)
    \begin{multicols}{3}
    \begin{choices}
        \wrongchoice{\SI{76}{\hertz}}
      \correctchoice{\SI{86}{\hertz}}
        \wrongchoice{\SI{54}{\hertz}}
        \wrongchoice{\SI{28}{\hertz}}
        \wrongchoice{\SI{129}{\hertz}}
    \end{choices}
    \end{multicols}
\end{question}
}

\element{serway-mc}{
\begin{question}{serway-ch18-q18}
    A vertical tube one meter long is open at the top. 
    It is filled with \SI{75}{\centi\meter} of water.
    If the velocity of sound is \SI{344}{\meter\per\second},
        what will the fundamental resonant frequency be?
    \begin{multicols}{3}
    \begin{choices}
        \wrongchoice{\SI{3.4}{\hertz}}
        \wrongchoice{\SI{172}{\hertz}}
      \correctchoice{\SI{344}{\hertz}}
        \wrongchoice{\SI{1.7}{\hertz}}
        \wrongchoice{\SI{688}{\hertz}}
    \end{choices}
    \end{multicols}
\end{question}
}

\element{serway-mc}{
\begin{question}{serway-ch18-q19}
    A length of organ pipe is closed at one end. 
    If the speed of sound is \SI{344}{\meter\per\second},
        what length of pipe is needed to obtain a fundamental frequency of \SI{50}{\hertz}?
    \begin{multicols}{3}
    \begin{choices}
        \wrongchoice{\SI{28}{\centi\meter}}
        \wrongchoice{\SI{86}{\centi\meter}}
        \wrongchoice{\SI{344}{\centi\meter}}
      \correctchoice{\SI{172}{\centi\meter}}
        \wrongchoice{\SI{688}{\centi\meter}}
    \end{choices}
    \end{multicols}
\end{question}
}

\element{serway-mc}{
\begin{question}{serway-ch18-q20}
    Two tuning forks with frequencies \SI{264}{\hertz} and \SI{262}{\hertz} produce ``beats''.
    What is the beat frequency?
    \begin{multicols}{2}
    \begin{choices}[o]
        \wrongchoice{\num{4}}
      \correctchoice{\num{2}}
        \wrongchoice{\num{1}}
        \wrongchoice{\num{3}}
        \wrongchoice{zero (no beats are produced)}
    \end{choices}
    \end{multicols}
\end{question}
}

\element{serway-mc}{
\begin{question}{serway-ch18-q21}
    Two instruments produce a beat frequency of \SI{5}{\hertz}. 
    If one has a frequency of \SI{264}{\hertz},
        what could be the frequency of the other instrument?
    \begin{multicols}{3}
    \begin{choices}
      \correctchoice{\SI{269}{\hertz}}
        \wrongchoice{\SI{254}{\hertz}}
        \wrongchoice{\SI{264}{\hertz}}
        \wrongchoice{\SI{5}{\hertz}}
        \wrongchoice{\SI{274}{\hertz}}
    \end{choices}
    \end{multicols}
\end{question}
}

\element{serway-mc}{
\begin{question}{serway-ch18-q22}
    Two waves are described by
    \begin{align*}
        y_1 &= \left(\SI{6}{\meter}\right) \cos\left(\dfrac{180}{\si{\second}}t\right) \\
        y_2 &= \left(\SI{6}{\meter}\right) \cos\left(\dfrac{186}{\si{\second}}t\right)\,.
    \end{align*}
    With what angular frequency does the maximum amplitude
        of the resultant wave vary with time?
    \begin{multicols}{2}
    \begin{choices}
        \wrongchoice{\SI{366}{\radian\per\second}}
        \wrongchoice{\SI{6}{\radian\per\second}}
      \correctchoice{\SI{3}{\radian\per\second}}
        \wrongchoice{\SI{92}{\radian\per\second}}
        \wrongchoice{\SI{180}{\radian\per\second}}
    \end{choices}
    \end{multicols}
\end{question}
}

\element{serway-mc}{
\begin{question}{serway-ch18-q23}
    Two waves are described by
    \begin{align*}
        y_1 &= \left(\SI{6}{\meter}\right) \cos\left(\dfrac{180}{\si{\second}}t\right) \\
        y_2 &= \left(\SI{6}{\meter}\right) \cos\left(\dfrac{186}{\si{\second}}t\right)\,.
    \end{align*}
    What effective frequency does the resultant vibration have at a point?
    \begin{multicols}{3}
    \begin{choices}
        \wrongchoice{\SI{92}{\hertz}}
      \correctchoice{\SI{183}{\hertz}}
        \wrongchoice{\SI{6}{\hertz}}
        \wrongchoice{\SI{3}{\hertz}}
        \wrongchoice{\SI{366}{\hertz}}
    \end{choices}
    \end{multicols}
\end{question}
}

\element{serway-mc}{
\begin{question}{serway-ch18-q24}
    An organ pipe open at both ends is \SI{1.5}{\meter} long. 
    A second organ pipe that is closed at one end and open at the other is \SI{0.75}{\meter} long. 
    The speed of sound in the room is \SI{330}{\meter\per\second}.
    Which of the following sets of frequencies consists of frequencies which can be produced by both pipes?
    \begin{choices}
        \wrongchoice{\SI{110}{\hertz}, \SI{220}{\hertz}, \SI{330}{\hertz}}
        \wrongchoice{\SI{220}{\hertz}, \SI{440}{\hertz}, \SI{660}{\hertz}}
      \correctchoice{\SI{110}{\hertz}, \SI{330}{\hertz}, \SI{550}{\hertz}}
        \wrongchoice{\SI{330}{\hertz}, \SI{440}{\hertz}, \SI{550}{\hertz}}
        \wrongchoice{\SI{220}{\hertz}, \SI{660}{\hertz}, \SI{1100}{\hertz}}
    \end{choices}
\end{question}
}

\element{serway-mc}{
\begin{question}{serway-ch18-q25}
    Two strings are respectively \SI{1.00}{\meter} and \SI{2.00}{\meter} long. 
    Which of the following wavelengths,
        could represent harmonics present on both strings?
    \begin{choices}
        \wrongchoice{\SI{0.800}{\meter}, \SI{0.670}{\meter}, \SI{0.500}{\meter}}
        \wrongchoice{\SI{1.33}{\meter}, \SI{1.00}{\meter}, \SI{0.500}{\meter}}
      \correctchoice{\SI{2.00}{\meter}, \SI{1.00}{\meter}, \SI{0.500}{\meter}}
        \wrongchoice{\SI{2.00}{\meter}, \SI{1.33}{\meter}, \SI{1.00}{\meter}}
        \wrongchoice{\SI{4.00}{\meter}, \SI{2.00}{\meter}, \SI{1.00}{\meter}}
    \end{choices}
\end{question}
}

\element{serway-mc}{
\begin{question}{serway-ch18-q26}
    Two point sources emit sound waves of \SI{1.0}{\meter} wavelength. 
    The sources, \SI{2.0}{\meter} apart, as shown below,
    \begin{center}
    \begin{tikzpicture}
        %% NOTE:
    \end{tikzpicture}
    \end{center}
        emit waves which are in phase with each other at the instant of emission. 
    Where, along the line between the sources,
        are the waves out of phase with each other by $\pi\,\si{\radian}$?
    \begin{choices}
        %% NOTE: format spacing
        \wrongchoice{$x = \SI{0}{\meter}, \SI{1.0}{\meter}, \SI{2.0}{\meter}$}
        \wrongchoice{$x = \SI{0.50}{\meter}, \SI{1.5}{\meter}$}
        \wrongchoice{$x = \SI{0.50}{\meter}, \SI{1.0}{\meter}, \SI{1.5}{\meter}$}
      \correctchoice{$x = \SI{0.75}{\meter}, \SI{1.25}{\meter}$}
        \wrongchoice{$x = \SI{0.25}{\meter}, \SI{0.75}{\meter}, \SI{1.25}{\meter}, \SI{1.75}{\meter}$}
    \end{choices}
\end{question}
}

\element{serway-mc}{
\begin{question}{serway-ch18-q27}
    Two identical strings have the same length and same mass per unit length. 
    String $B$ is stretched with four times as great a tension as that applied to string $A$.
    Which statement is correct for all $n$ harmonics on the two strings,
        $n = 1, 2, 3\ldots$?
    \begin{multicols}{2}
    \begin{choices}
        \wrongchoice{$f_{n,B} = \dfrac{1}{4} f_{n,A}$}
        \wrongchoice{$f_{n,B} = \dfrac{1}{2} f_{n,A}$}
        \wrongchoice{$f_{n,B} = \sqrt{2} f_{n,A}$}
      \correctchoice{$f_{n,B} = 2 f_{n,A}$}
        \wrongchoice{$f_{n,B} = 4 f_{n,A}$}
    \end{choices}
    \end{multicols}
\end{question}
}

\element{serway-mc}{
\begin{question}{serway-ch18-q28}
    The superposition of two waves
    \begin{align*}
        y_1 &= \left(\SI{0.006}{\centi\meter}\right) \cos\left[2\pi\left(\dfrac{156}{\si{\second}}t\right)\right] \\
        y_2 &= \left(\SI{0.004}{\centi\meter}\right) \cos\left[2\pi\left(\dfrac{150}{\si{\second}}t\right)\right]\,,
    \end{align*}
    at the location $x=\text{zero}$ in space results in
    \begin{choices}
        \wrongchoice{beats at a beat frequency of \SI{3}{\hertz}.}
        \wrongchoice{a pure tone at a frequency of \SI{153}{\hertz}.}
        \wrongchoice{a pure tone at a frequency of \SI{156}{\hertz}.}
        \wrongchoice{beats at a beat frequency of \SI{6}{\hertz} in a \SI{153}{\hertz} tone.}
      \correctchoice{a tone at a frequency of \SI{156}{\hertz}, as well as beats at a beat frequency of \SI{6}{\hertz} in a \SI{153}{\hertz} tone.}
    \end{choices}
\end{question}
}

\element{serway-mc}{
\begin{question}{serway-ch18-q29}
    The superposition of two waves,
    \begin{align*}
        y_1 &= \left(\SI{2e-8}{\meter}\right) \sin\left[\pi\left(\dfrac{x}{\SI{2}{\meter}}-\dfrac{170}{\si{\second}}t\right)\right] \\
        y_2 &= \left(\SI{2e-8}{\meter}\right) \sin\left[\pi\left(\dfrac{x}{\SI{2}{\meter}}-\dfrac{170}{\si{\second}}t-\dfrac{1}{2}\right)\right]\,,
    \end{align*}
    results in a wave with a phase angle of:
    \begin{multicols}{3}
    \begin{choices}
        \wrongchoice{$0\,\si{\radian}$}
        \wrongchoice{$\dfrac{1}{2}\,\si{\radian}$}
      \correctchoice{$\dfrac{\pi}{4}\,\si{\radian}$}
        \wrongchoice{$\dfrac{\pi}{2}\,\si{\radian}$}
        \wrongchoice{$\pi\,\si{\radian}$}
    \end{choices}
    \end{multicols}
\end{question}
}

\element{serway-mc}{
\begin{question}{serway-ch18-q30}
    The superposition of two waves,
    \begin{align*}
        y_1 &= \left(\SI{2e-8}{\meter}\right) \sin\left[\pi\left(\dfrac{x}{\SI{2}{\meter}}-\dfrac{170}{\si{\second}}t\right)\right] \\
        y_2 &= \left(\SI{2e-8}{\meter}\right) \sin\left[\pi\left(\dfrac{x}{\SI{2}{\meter}}-\dfrac{170}{\si{\second}}t-\dfrac{1}{2}\right)\right]\,,
    \end{align*}
    results in a wave with a wavelength of:
    \begin{multicols}{3}
    \begin{choices}
        \wrongchoice{$\dfrac{\pi}{2}\,\si{\meter}$}
        \wrongchoice{$2\,\si{\meter}$}
        \wrongchoice{$\pi\,\si{\meter}$}
      \correctchoice{$4\,\si{\meter}$}
        \wrongchoice{$4\pi\,\si{\meter}$}
    \end{choices}
    \end{multicols}
\end{question}
}

\element{serway-mc}{
\begin{question}{serway-ch18-q31}
    The superposition of two waves,
    \begin{align*}
        y_1 &= \left(\SI{2e-8}{\meter}\right) \sin\left[\pi\left(\dfrac{x}{\SI{2}{\meter}}-\dfrac{170}{\si{\second}}t\right)\right] \\
        y_2 &= \left(\SI{2e-8}{\meter}\right) \sin\left[\pi\left(\dfrac{x}{\SI{2}{\meter}}-\dfrac{170}{\si{\second}}t-\dfrac{1}{2}\right)\right]\,,
    \end{align*}
    results in a wave with a frequency of:
    \begin{multicols}{3}
    \begin{choices}
      \correctchoice{$85\,\si{\hertz}$}
        \wrongchoice{$170\,\si{\hertz}$}
        \wrongchoice{$85\pi\,\si{\hertz}$}
        \wrongchoice{$340\,\si{\hertz}$}
        \wrongchoice{$170\pi\,\si{\hertz}$}
    \end{choices}
    \end{multicols}
\end{question}
}

\element{serway-mc}{
\begin{question}{serway-ch18-q32}
    In a standing wave, not necessarily at the fundamental frequency,
        on a string of length $L$, the distance between nodes is:
    \begin{multicols}{3}
    \begin{choices}
        \wrongchoice{$\dfrac{\lambda}{4}$}
      \correctchoice{$\dfrac{\lambda}{2}$}
        \wrongchoice{$\lambda$}
        \wrongchoice{$\dfrac{L}{4}$}
        \wrongchoice{$\dfrac{L}{2}$}
    \end{choices}
    \end{multicols}
\end{question}
}

\element{serway-mc}{
\begin{question}{serway-ch18-q33}
    Which of the following wavelengths could \emph{not} be present as a standing wave in a \SI{2}{\meter} long organ pipe open at both ends?
    \begin{multicols}{3}
    \begin{choices}
        \wrongchoice{\SI{4}{\meter}}
        \wrongchoice{\SI{2}{\meter}}
        \wrongchoice{\SI{1}{\meter}}
      \correctchoice{\SI{0.89}{\meter}}
        \wrongchoice{\SI{0.5}{\meter}}
    \end{choices}
    \end{multicols}
\end{question}
}

\element{serway-mc}{
\begin{question}{serway-ch18-q34}
    Which of the following wavelengths could \emph{not} be present as a harmonic on a \SI{2}{\meter} long string?
    \begin{multicols}{3}
    \begin{choices}
        \wrongchoice{\SI{4}{\meter}}
        \wrongchoice{\SI{2}{\meter}}
        \wrongchoice{\SI{1}{\meter}}
      \correctchoice{\SI{0.89}{\meter}}
        \wrongchoice{\SI{0.5}{\meter}}
    \end{choices}
    \end{multicols}
\end{question}
}

\element{serway-mc}{
\begin{question}{serway-ch18-q35}
    Which of the following frequencies could \emph{not} be present as a standing wave in a \SI{2}{\meter} long organ pipe open at both ends.
    The fundamental frequency is \SI{85}{\hertz}.
    \begin{multicols}{3}
    \begin{choices}
        \wrongchoice{\SI{85}{\hertz}}
        \wrongchoice{\SI{170}{\hertz}}
        \wrongchoice{\SI{255}{\hertz}}
        \wrongchoice{\SI{340}{\hertz}}
      \correctchoice{\SI{382}{\hertz}}
    \end{choices}
    \end{multicols}
\end{question}
}

\element{serway-mc}{
\begin{question}{serway-ch18-q36}
    An observer stands \SI{3}{\meter} from speaker $A$ and \SI{4}{\meter} from speaker $B$.
    Both speakers, oscillating in phase, produce \SI{170}{\hertz} waves. 
    \begin{center}
    \begin{tikzpicture}
        %% NOTE:
    \end{tikzpicture}
    \end{center}
    The speed of sound in air is \SI{340}{\meter\per\second}.
    What is the phase difference (in radians) between the waves from $A$ and $B$ at the observer’s location, point P?
    \begin{multicols}{3}
    \begin{choices}
        \wrongchoice{$0\,\si{\radian}$}
        \wrongchoice{$\dfrac{\pi}{2}\,\si{\radian}$}
      \correctchoice{$\pi\,\si{\radian}$}
        \wrongchoice{$2\pi\,\si{\radian}$}
        \wrongchoice{$4\pi\,\si{\radian}$}
    \end{choices}
    \end{multicols}
\end{question}
}

\element{serway-mc}{
\begin{questionmult}{serway-ch18-q37}
    As shown below, a garden room has three walls,
        a floor and a roof, but is open to the garden on one side.
    \begin{center}
    \begin{tikzpicture}
        %% NOTE:
    \end{tikzpicture}
    \end{center}
    The wall widths are $L$ and $w$.
    The roof height is $h$.
    When traveling sound waves enter the room,
        standing sound waves can be present in the room if the wavelength of the standing waves is
    \begin{choices}
      \correctchoice{$\dfrac{L}{n}$, where $n$ is a positive integer.}
      \correctchoice{$\dfrac{w}{n}$, where $n$ is an odd integer.}
        \wrongchoice{$\dfrac{h}{n}$, where $n$ is an even integer.}
        %\wrongchoice{in all cases listed above.}
        %\correctchoice{given by (a) or (b) above, but not by (c).}
    \end{choices}
\end{questionmult}
}

\element{serway-mc}{
\begin{question}{serway-ch18-q38}
    Transverse waves $y_1 = A_1\sin\left(k_1 x - \omega_1 t\right)$
        and $y_2 = A_2\sin\left(k_2 x + \omega_2 t\right)$,
        with $A_2 > A_1$ start at opposite ends of a long rope when $t=0$. 
    The magnitude of the maximum displacement, $y$,
        of the rope at any point is:
    \begin{multicols}{2}
    \begin{choices}
        \wrongchoice{$A_1 - A_2$}
        \wrongchoice{$A_2 - A_1$}
      \correctchoice{$A_1 + A_2$}
        \wrongchoice{$\left( A_1 - A_2 \right) \dfrac{k_1}{k_2}$}
        \wrongchoice{$\left( A_2 - A_1 \right) \dfrac{k_2}{k_1}$}
    \end{choices}
    \end{multicols}
\end{question}
}

\element{serway-mc}{
\begin{questionmult}{serway-ch18-q39}
    Two speakers in an automobile emit sound waves that are in phase at the speakers. 
    One speaker is \SI{40}{\centi\meter} ahead of and \SI{30}{\centi\meter} to the left of the driver's left ear. 
    The other speaker is \SI{50}{\centi\meter} ahead of and \SI{120}{\centi\meter} to the right of the driver's right ear. 
    Which of the following wavelengths is(are) in phase at the left ear for the speaker on the left and the right ear for the speaker on the right?
    \begin{multicols}{3}
    \begin{choices}
      \correctchoice{\SI{10}{\centi\meter}}
        \wrongchoice{\SI{20}{\centi\meter}}
      \correctchoice{\SI{650}{\centi\meter}}
        %\wrongchoice{All of the wavelengths listed above.}
        %\correctchoice{Only the wavelengths listed in (a) and (c).}
    \end{choices}
    \end{multicols}
\end{questionmult}
}

\element{serway-mc}{
\begin{questionmult}{serway-ch18-q40}
    A very long string is tied to a rigid wall at one end while the other end is attached to a simple harmonic oscillator. 
    Which of the following can be changed by changing the frequency of the oscillator?
    \begin{choices}
        \wrongchoice{The speed of the waves traveling along the string.}
        \wrongchoice{The tension in the string.}
      \correctchoice{The wavelength of the waves on the string.}
        %\wrongchoice{All of the above.}
        %\wrongchoice{None of the above.}
    \end{choices}
\end{questionmult}
}

\element{serway-mc}{
\begin{question}{serway-ch18-q41}
    %% Open
    When two organ pipes open at both ends sound a perfect fifth,
        such as two notes with fundamental frequencies at \SI{440}{\hertz} and \SI{660}{\hertz},
        both pipes produce overtones. 
    Which choice below correctly describes overtones present in both pipes?
    \begin{choices}
        \wrongchoice{\SI{440}{\hertz}, \SI{880}{\hertz} and \SI{1320}{\hertz}.}
        \wrongchoice{\SI{660}{\hertz}, \SI{1320}{\hertz} and \SI{1980}{\hertz}.}
        \wrongchoice{\SI{880}{\hertz}, \SI{1320}{\hertz} and \SI{1760}{\hertz}.}
      \correctchoice{\SI{1320}{\hertz}, \SI{2640}{\hertz} and \SI{3960}{\hertz}.}
        \wrongchoice{They have no overtones in common.}
    \end{choices}
\end{question}
}

\element{serway-mc}{
\begin{question}{serway-ch18-q42}
    %% Closed
    When two organ pipes closed at both ends sound a perfect fifth,
        such as two notes with fundamental frequencies at \SI{440}{\hertz} and \SI{660}{\hertz},
        both pipes produce overtones. 
    Which choice below correctly describes overtones present in both pipes?
    \begin{choices}
        \wrongchoice{\SI{440}{\hertz}, \SI{880}{\hertz} and \SI{1320}{\hertz}.}
        \wrongchoice{\SI{660}{\hertz}, \SI{1320}{\hertz} and \SI{1980}{\hertz}.}
        \wrongchoice{\SI{880}{\hertz}, \SI{1320}{\hertz} and \SI{1760}{\hertz}.}
        \wrongchoice{\SI{1320}{\hertz}, \SI{2640}{\hertz} and \SI{3960}{\hertz}.}
      \correctchoice{They have no overtones in common.}
    \end{choices}
\end{question}
}

\element{serway-mc}{
\begin{question}{serway-ch18-q43}
    Two organ pipes, a pipe of fundamental frequency \SI{440}{\hertz},
        closed at one end, and a pipe of fundamental frequency \SI{660}{\hertz},
        open at both ends, produce overtones. 
    Which choice below correctly describes overtones present in both pipes?
    \begin{choices}
        \wrongchoice{After the first overtone of each pipe, every second overtone of the first pipe matches every second overtone of the second pipe.}
        \wrongchoice{After the first overtone of each pipe, every second overtone of the first pipe matches every third overtone of the second pipe.}
        \wrongchoice{After the first overtone of each pipe, every third overtone of the first pipe matches every second overtone of the second pipe.}
        \wrongchoice{After the first overtone of each pipe, every second overtone of the first pipe matches every fourth overtone of the second pipe.}
      \correctchoice{After the first overtone of each pipe, every third overtone of the first pipe matches every fourth overtone of the second pipe.}
    \end{choices}
\end{question}
}

\newcommand{\serwayChEighteenQFortyFour}{
\begin{tikzpicture}
    %% NOTE:
\end{tikzpicture}
}

\element{serway-mc}{
\begin{question}{serway-ch18-q44}
    The figure below shows wave crests after a stone is thrown into a pond.
    \begin{center}
        \serwayChEighteenQFortyFour
    \end{center}
    The phase difference in radians between points $A$ and $B$ is:
    \begin{multicols}{3}
    \begin{choices}
      \correctchoice{zero}
        \wrongchoice{$\dfrac{\pi}{4}$}
        \wrongchoice{$\dfrac{\pi}{2}$}
        \wrongchoice{$\pi$}
        \wrongchoice{$\dfrac{3\pi}{2}$}
    \end{choices}
    \end{multicols}
\end{question}
}

\element{serway-mc}{
\begin{question}{serway-ch18-q45}
    The figure below shows wave crests after a stone is thrown into a pond.
    \begin{center}
        \serwayChEighteenQFortyFour
    \end{center}
    The phase difference in radians between points $A$ and $C$ is:
    \begin{multicols}{3}
    \begin{choices}
        \wrongchoice{zero}
        \wrongchoice{$\dfrac{\pi}{2}$}
        \wrongchoice{$\pi$}
        \wrongchoice{$\dfrac{3\pi}{2}$}
      \correctchoice{$2\pi$}
    \end{choices}
    \end{multicols}
\end{question}
}

\element{serway-mc}{
\begin{question}{serway-ch18-q46}
    The figure below shows wave crests after a stone is thrown into a pond.
    \begin{center}
        \serwayChEighteenQFortyFour
    \end{center}
    The phase difference in radians between points $A$ and $D$ is:
    \begin{multicols}{3}
    \begin{choices}
        \wrongchoice{$\pi$}
        \wrongchoice{$2\pi$}
      \correctchoice{$3\pi$}
        \wrongchoice{$4\pi$}
        \wrongchoice{$5\pi$}
    \end{choices}
    \end{multicols}
\end{question}
}


\endinput


