
%%--------------------------------------------------
%% Serway: Physics for Scientists and Engineers
%%--------------------------------------------------


%% Chapter 21: The Kinetic Theory of Gases
%%--------------------------------------------------


%% Serway Numeric Questions
%%--------------------------------------------------
\element{serway-num}{
\begin{questionmultx}{serway-ch21-q31}
    A \SI{50}{\gram} sample of dry ice (solid \ce{CO2}) is placed in a \SI{4}{\liter} container. 
    The system is sealed and allowed to reach room temperature (\SI{20}{\degreeCelsius}). 
    By approximately how much does the pressure inside the container increase when the dry ice turns to gas? 
    (Ignore the initial volume of the sample.)
    %% ANSWER:  increases by 7 atm
    \AMCnumericChoices{1.5}{
        vertical=false,
        digits=3,decimals=2,sign=true,Tsign=\hspace{1ex}\Large a,
        borderwidth=0pt,backgroundcol=white,approx=5
    }
    \AMCnumericChoices{2}{
        vertical=false,
        digits=2,decimals=0,sign=true,Tsign=\hspace{1ex}\Large b,
        borderwidth=0pt,backgroundcol=white,approx=5
    }
\end{questionmultx}
}

\element{serway-num}{
\begin{questionmultx}{serway-ch21-q32}
    One mole of helium gas expands adiabatically from \SI{2}{\atm} pressure to \SI{1}{\atm} pressure. 
    If the original temperature of the gas is \SI{20}{\degreeCelsius},
        what is the final temperature of the gas? ($\gamma=\num{1.67}$)
    %% ANSWER:  \SI{222}{\kelvin}
    \AMCnumericChoices{2.2}{
        vertical=false,
        digits=3,decimals=2,sign=true,Tsign=\hspace{1ex}\Large a,
        borderwidth=0pt,backgroundcol=white,approx=5
    }
    \AMCnumericChoices{2}{
        vertical=false,
        digits=2,decimals=0,sign=true,Tsign=\hspace{1ex}\Large b,
        borderwidth=0pt,backgroundcol=white,approx=5
    }
\end{questionmultx}
}

\element{serway-num}{
\begin{questionmultx}{serway-ch21-q33}
    Air expands adiabatically (no heat in, no heat out) from $T=\SI{300}{\kelvin}$ and $P=\SI{100}{\atm}$ to a final pressure of 1 atm. 
    Treat the gas as ideal with $\gamma=\num{1.4}$,
        and determine the final temperature. 
    Compare your result to the boiling points of nitrogen (\SI{77.4}{\kelvin}) and oxygen (\SI{90.2}{\kelvin}). 
    Could this method result in the liquification of air?
    %% ANSWER:  \SI{80.5}{\kelvin}, some oxygen would liquify
    \AMCnumericChoices{8.05}{
        vertical=false,
        digits=3,decimals=2,sign=true,Tsign=\hspace{1ex}\Large a,
        borderwidth=0pt,backgroundcol=white,approx=5
    }
    \AMCnumericChoices{1}{
        vertical=false,
        digits=2,decimals=0,sign=true,Tsign=\hspace{1ex}\Large b,
        borderwidth=0pt,backgroundcol=white,approx=5
    }
\end{questionmultx}
}

\element{serway-num}{
\begin{questionmultx}{serway-ch21-q34}
    According to kinetic theory, a typical gas molecule in thermal equilibrium at room temperature has a kinetic energy $K=\SI{6.00e-21}{\joule}$, regardless of mass.
    Estimate the speed at room temperature of a hydrogen molecule \ce{H2} ($m=\SI{3.34e-27}{\kilo\gram}$) and a xenon atom ($m=\SI{2.00e-25}{\kilo\gram}$). 
    [$k_B=\SI{1.38e-23}{\joule\per\kelvin}$] 
    %% ANSWER:  \SI{1895}{\meter\per\second}
    %% ANSWER:  \SI{245}{\meter\per\second}
    \AMCnumericChoices{1.895}{
        vertical=false,
        digits=3,decimals=2,sign=true,Tsign=\hspace{1ex}\Large a,
        borderwidth=0pt,backgroundcol=white,approx=5
    }
    \AMCnumericChoices{3}{
        vertical=false,
        digits=2,decimals=0,sign=true,Tsign=\hspace{1ex}\Large b,
        borderwidth=0pt,backgroundcol=white,approx=5
    }
\end{questionmultx}
}

\element{serway-num}{
\begin{questionmultx}{serway-ch21-q35}
    During the volcanic eruption of Mt. Pelee in 1902, an incredibly hot ``burning cloud” rolled down the mountain and incinerated the town of Saint-Pierre. 
    From the damage done, the temperature in the cloud was estimated at \SI{700}{\degreeCelsius}. 
    If the air temperature was \SI{20}{\degreeCelsius} and the molecular weight of air is \SI{29}{\gram},
        estimate the molecular weight of the gas in the ``burning cloud'' that made it heavier than the surrounding air. 
    [As a follow-on, estimate the most probable composition of the cloud. 
    Some typical volcanic gases are \ce{H2S}, \ce{SO2} , \ce{H2SO4}, \ce{CO2}, \ce{NO}.]
    %% ANSWER:  \num{96}, \ce{H2SO4} sulfuric acid
    \AMCnumericChoices{1.895}{
        vertical=false,
        digits=3,decimals=2,sign=true,Tsign=\hspace{1ex}\Large a,
        borderwidth=0pt,backgroundcol=white,approx=5
    }
    \AMCnumericChoices{3}{
        vertical=false,
        digits=2,decimals=0,sign=true,Tsign=\hspace{1ex}\Large b,
        borderwidth=0pt,backgroundcol=white,approx=5
    }
\end{questionmultx}
}


\endinput


